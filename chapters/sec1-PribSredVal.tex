\section{Приближение функций из пространств Лебега и Соболева с переменным показателем средними Валле Пуссена}


\begin{center}

\textbf{Аннотация:}

\textit{
Рассматривается  пространство $L^{p(x)}_{2\pi}$, состоящее из $2\pi$-периодических действительных измеримых функций $f(x)$, для которых
существует конечный интеграл $\int_{-\pi}^{\pi}|f(x)|^{p(x)}dx$, где $1\le p(x)$ -- $2\pi$-периодическая измеримая функция.  Если $ p(x)\le \overline p<\infty$, то пространство $L^{p(x)}_{2\pi}$ можно превратить в банахово пространство с нормой
$\|f\|_{p(\cdot)}=\inf\{\alpha>0:\int_{-\pi}^{\pi}|{f(x)/\alpha}|^{p(x)}dx\le1\}.$
В пространстве $L^{p(x)}_{2\pi}$ выделяется подпространство типа Соболева $W^{r,p(x)}_{2\pi}$. В статье исследованы аппроксимативные свойства
средних Валле Пуссена тригонометрических сумм Фурье для функций из пространств  $W^{r,p(x)}_{2\pi}$. В частности,
доказано, что если переменный показатель $p(x)$ удовлетворяет условию Дини-Липшица $|p(x)-p(y)|\ln{2\pi\over|x-y|}\le c$ и $f\in W^{r,p(x)}_{2\pi}$,
то для средних Валле -- Пуссена $V_m^n(f)=V_m^n(f,x)$ с $n\le am$  имеет место неравенство
 $$
 \|f-V_m^n(f)\|_{p(\cdot)}\le \frac{c_r(p,a)}{n^r}\Omega(f^{(r)},\frac1n)_{p(\cdot)},
 $$
 где $\Omega(g,\delta)_{p(\cdot)}$ -- модуль непрерывности функции $g\in L^{p(x)}_{2\pi}$, определенный с помощью функций В.А.Стеклова.
Если $1<p(x)$, $r\ge1$, $f\in W^{r,p(x)}_{2\pi}$ и выполнено условие Дини-Липшица, то доказано, что
$$
     |f(x)-V_m^n(f,x)|\le \frac{c_r(p)}{m+1}\sum_{k=n}^{n+m}{E_k(f^{(r)})_{p(\cdot)}\over (k+1)^{r-{\frac{1}{p(x)}}}},
$$
 где $E_k(g)_{p(\cdot)}$ -- наилучшее приближение функции $g\in L^{p(x)}_{2\pi}$ тригонометрическими полиномами порядка $k$.
 }
\end{center}


\begin{center}
\textbf{Ключевые слова:} пространства Лебега и Соболева с переменным показателем, приближение функций средними Валле-Пуссена.
\end{center}

\markright{Приближение функций суммами средними Валле-Пуссена}








\section{ Введение}\label{s1}

Пусть $f=f(x)$ -- $2\pi$-периодическая функция, интегрируемая на периоде,
\begin{equation}\label{1.1}
    a_k=a_k(f)=\frac1\pi\int_{-\pi}^\pi f(t)\cos ktdt,\quad b_k=b_k(f)=\frac1\pi\int_{-\pi}^\pi f(t)\sin ktdt
\end{equation}
-- коэффициенты Фурье,
\begin{equation}\label{1.2}
    f(x) \sim \frac{a_0}{2}+ \sum_{k=1}^\infty a_k\cos kx+b_k\sin kx
\end{equation}
-- ряд Фурье функции $f(x)$. Далее, пусть

\begin{equation} \label{1.3}
 S_n(f)=   S_n(f,x)=\frac{a_0}{2}+ \sum_{k=1}^n a_k\cos kx+b_k\sin kx
\end{equation}
-- сумма Фурье,
\begin{equation}\label{1.4}
 V_m^n(f)=V_m^n(f,x)=\frac1{m+1}[S_n(f,x)+\cdots+S_{n+m}(f,x)]
\end{equation}
-- средние Валле Пуссена. В настоящей статье рассмотрена задача
об аппроксимативных свойствах операторов $V_m^n=V_m^n(f)$ в пространствах Лебега и Соболева с переменным показателем.
В частности, показано, что $V_m^n(f)$ могут быть использованы для доказательства второй теоремы Джексона в пространстве
Лебега $L_{2\pi}^{p(x)}$ с $2\pi$-периодическим переменным показателем  $p(x)\ge1$, удовлетворяющим условию
\begin{equation}\label{1.5}
  |p(x)-p(y)|\ln\frac{2\pi}{|x-y|}\le c \quad (x,y\in [0,2\pi]).
  \end{equation}

Теория приближения функций в пространствах Лебега $L_{2\pi}^{p(x)}$  и Соболева $W^{r,p(x)}_{2\pi}$ с $2\pi$-периодическим переменным показателем $p(x)\ge1$ в настоящее время достаточно хорошо разработана \cite{Shar5}  -- \cite{Shar6}. При этом следует отметить, что методы исследования задач этой теории и полученные результаты существенно зависят от  величины $p_-=p_-([-\pi,\pi])=\min_{x\in[-\pi,\pi]}p(x)$. Дело в том, что если $p_->1$ и выполнено условие \eqref{1.5}, то, как было показано впервые в работе автора \cite{Shar2}, тригонометрическая система $1, \{\cos kx, \sin kx\}_{k=1}^\infty$ является базисом Шаудера в пространстве  $L_{2\pi}^{p(x)}$ и, как следствие, суммы Фурье $S_n(f)$ (см. \eqref{1.3} )  доставляют функции $f\in L_{2\pi}^{p(x)}$ наилучший порядок приближения среди тригонометрических полиномов $T_n(x)$ порядка $n$. Этот результат и тот факт  \cite{Shar2}, что оператор сопряженной функции
$$
\tilde f=\tilde f(x)=\int_{-\pi}^{\pi}{f(t)dt\over \operatorname{tg}\frac12(x-t)}
$$
 ограниченно действует в  $L_{2\pi}^{p(x)}$, послужили основой   для доказательства прямых и обратных теорем теории приближений в $L_{2\pi}^{p(x)}$ и $W_{2\pi}^{r,p(x)}$ при $p_-([-\pi,\pi])>1$  \cite{GuvIsr}--\cite{Chaich}. Если же  $p_-([-\pi,\pi])=1$, то ситуация принимает существенно иной характер: тригонометрическая система теряет в этом случае свойство базисности в $L_{2\pi}^{p(x)}$ и, как следствие, потребовалось \cite{Shar6} разработать принципиально новые подходы для получения прямых и обратных теорем теории приближений в пространстве $L_{2\pi}^{p(x)}$. Частично, а именно в части, касающейся первой теоремы Джексона, такие подходы были разработаны в работах \cite{Shar3}, \cite{Shar6}. Что же касается задачи о справедливости второй теоремы Джексона, которая касается оценки вида
 \begin{equation}\label{eq1.6}
   E_n(f)_{p(\cdot)}\le \frac{c_r(p)}{n^r}\Omega(f^{(r)},\frac1n)_{p(\cdot)},
 \end{equation}
 где $f\in W_{2\pi}^{r,p(x)}$, $E_n(f)_{p(\cdot)}$ -- наилучшее приближение функции $f$ тригонометрическими полиномами степени $n$ по норме пространства $L_{2\pi}^{p(x)}$, $\Omega(g,\frac1n)_{p(\cdot)}$ -- выбранный модуль непрерывности функции $g\in L_{2\pi}^{p(x)}$, то в случае $p_-([-\pi,\pi])=1$ она оставалась не решенной. Эту проблему удалось решить недавно в работах \cite{Shar7}, \cite{Shar8}, в которых в связи с этой задачей были  исследованы аппроксимативные свойства  средних Валле Пуссена $V_m^n(f,x)$ c $m=n-1$ и  $m=n$, определенные равенством   \eqref{1.4}. В настоящей работе рассмотрены аппроксимативные свойства операторов $V_m^n(f)=V_m^n(f,x)$ в пространстве $ L_{2\pi}^{p(x)}$
 в наиболее общем случае, когда параметры $n$   и $m$ независимо друг от друга могут принимать любые неотрицательные целые значения.
Основное внимание будет уделено задачам трех типов. Первая задача касается условий на переменный показатель $p(x)$, которые гарантируют  равномерную ограниченность в $L_{2\pi}^{p(x)}$ семейства операторов  $V_m^n(f)$ $(m,n=0,1,\ldots)$. Доказано, что если переменный показатель $p(x)\ge1$ удовлетворяет условию \eqref{1.5}, то семейство операторов  $V_m^n(f)$ с $n\le cm$ равномерно ограничено в $L_{2\pi}^{p(x)}$. Если же $p_-([-\pi,\pi])>1$ и выполнено условие \eqref{1.5}, то  в $L_{2\pi}^{p(x)}$ равномерно ограничено (\eqref{4.10} из \S4 ) семейство всех операторов  $V_m^n(f)$ с $n,m=0,1,\ldots$. Следующая проблема, рассмотренная в \S7, связана с оценкой отклонения в метрике  $L_{2\pi}^{p(x)}$
средних Валле Пуссена $V_m^n(f)$  от функции $f\in W_{2\pi}^{r,p(x)}$. Доказано (теорема 2), что если переменный показатель $p(x)$ удовлетворяет условию Дини-Липшица \eqref{1.5} и $f\in W^{r,p(x)}_{2\pi}$,
то для средних Валле  Пуссена $V_m^n(f)=V_m^n(f,x)$ с $ n\le am$ ($0<a$ -- фиксировано) имеет место неравенство
 $\|f-V_m^n(f)\|_{p(\cdot)}\le \frac{c_r(p)}{(n+1)^r}\Omega(f^{(r)},\frac1{n+1})_{p(\cdot)}$,  где здесь и всюду в дальнейшем $\Omega(g,\delta)_{p(\cdot)}$ -- модуль непрерывности функции $g\in L^{p(x)}_{2\pi}$, определенный с помощью функций В.А.Стеклова, $c(p), c(p,a), c_r(p,a),\ldots $ -- положительные числа, зависящие лишь от указанных параметров.
 Если  $r\ge3$, $f\in W^{r,p(x)}_{2\pi}$, $a>0$, $ n\le am$, то доказано (теорема 3), что имеют место следующие оценки
\begin{equation*}
   \|f-V_m^n(f\|_{p(\cdot)}\le {c_r(a,p)\over(m+1) (n+1)^{r-1}}E_n(f^{(r)})_{p(\cdot)},
\end{equation*}
\begin{equation*}
   \|f-V_m^n(f)\|_{p(\cdot)}\le {c_r(a,p)\over(m+1) (n+1)^{r-1}}\Omega\left(f^{(r)},\frac1{n+1}\right)_{p(\cdot)}.
\end{equation*}
Для $r=2$ также имеют место (теоремы 4 ) аналогичные оценки при $m^\delta\le n\le am$, где $0<\delta<1$.
В случае $r=1$  характер соответствующих оценок принимают несколько иной вид (теорема 5).

 В \S8 рассмотрена  задача о приближении функций из пространства Соболева с переменным показателем средними Валле Пуссена $V_m^n(f,x)$  в заданной точке $x$.  В частности, если  $p=p(x)>1$, $r\ge1$, $f\in W^{r,p(x)}_{2\pi}$, то доказано (теорема 7), что имеет место следующая оценка
 \begin{equation}\label{1.7}
    |f(x)-V_m^n(f,x)|\le \frac{c_r(p)}{m+1}\sum_{k=n}^{n+m}{E_k(f^{(r)})_{p(\cdot)}\over (k+1)^{r-{\frac{1}{p(x)}}}}.
 \end{equation}
Эта оценка показывает, что применение пространств Соболева  $W^{r,p(x)}_{2\pi}$ с переменным показателем $p=p(x)$ позволяет учитывать существенно переменное поведение производных функции  $f(x)$  при оценке погрешности $|f(x)-V_m^n(f,x)|$ приближения  $f(x)$ средними Валле Пуссена $V_m^n(f,x)$. Говоря более точно, имеется ввиду следующее: если $p=p(x)>1$, $r\ge1$, $f\in W^{r,p(x)}_{2\pi}$, то оставшись в шкале пространств Соболева $W^{r,p}_{2\pi}$ с постоянным показателем $p$, мы можем утверждать лишь, что  $f\in W^{r,p_0}_{2\pi}$, где $p_0=\min\limits_x p(x)$. Поэтому вместо оценки \eqref{1.7} мы будем иметь
\begin{equation}\label{1.8}
    |f(x)-V_m^n(f,x)|\le \frac{c_r(p_0)}{m+1}\sum_{k=n}^{n+m}{E_k(f^{(r)})_{p_0}\over (k+1)^{r-{\frac{1}{p_0}}}}.
 \end{equation}
Но для точки $x$, в которой $p(x)>p_0$ оценка \eqref{1.7} на порядок хуже, чем оценка \eqref{1.8}.









\section{ Некоторые сведения о пространствах Лебега и Соболева с переменным показателем}\label{s2}
Пусть $p(x)$ -- измеримая функция, определенная на измеримом множестве $E$ и такая, что $p(x)\ge1$ почти всюду на $E$. Через $L^{p(x)}(E)$ обозначим пространство измеримых функций $f(x)$, заданных на $E$, для каждой из  которых найдется такое число $\alpha=\alpha(f)>0$, что $\int_E|f(x)/\alpha|^{p(x)}dx<\infty$.  Целенаправленное исследование топологии пространств $L^{p(x)}(E)$  было дано впервые в работе автора
\cite{Shar4}. В частности, в \cite{Shar4} было показано, что в пространстве $L^{p(x)}(E)$ можно ввести норму
\begin{equation}\label{2.1}
    \|f\|_{p(\cdot)}(E)=\inf \left\{\alpha>0:   \int_E\left|{f(x)\over\alpha}\right|^{p(x)}dx\le1\right\}.
\end{equation}
Если $p(x)>1$ для почти всех $x\in E$, то мы можем определить сопряженную функцию $q(x)=p(x)/(p(x)-1)$. В \cite{Shar4} было показано, что пространство $(L^{p(x)}(E))'$, сопряженное к   $L^{p(x)}(E)$, совпадает с $L^{q(x)}(E)$,  в том смысле, что произвольный элемент $F\in(L^{p(x)}(E))' $ допускает представление
\begin{equation}\label{(2.2)}
    F(f)=\int_E f(x)g(x)dx,
\end{equation}
где $g(x)\in L^{q(x)}(E)$. Исходя из представления \eqref{(2.2)},  в $L^{p(x)}(E)$ можно ввести (см.\cite{Shar3} ) еще одну эквивалентную норму

\begin{equation}\label{2.3}
    \|f \|^*_{p(\cdot)}(E)=\sup_{g\in L^{q(x)}(E),\atop \|g\|_{q(\cdot)}\le1}\int_E f(x)g(x)dx.
\end{equation}

 Пусть $E_1$, $E_2$ -- измеримые множества,
$p(x)$ -- измеримая функция, заданная на $E_1$, $1\le
p^-(E_1)<\infty$, $f(t,x)$ -- измеримая функция, заданная на
прямоугольнике $E_1\times E_2$. Тогда имеет место \cite{Shar6} неравенство

\begin{equation}\label{2.4}
    \left\|\int\limits_{E_2}|f(*,x)|dx\right\|_{p(\cdot)}(E_1)\le
r_p \int\limits_{E_2}\|f(*,x)\|_{p(\cdot)}(E_1)dx\quad
(r_p\le2).
\end{equation}
Отметим также следующее неравенство
\begin{equation}\label{2.5}
    \int_E |f(x)g(x)|dx\le \|f \|^*_{p(\cdot)}(E)\|g\|_{q(\cdot)}(E),
\end{equation}
которое вытекает из \eqref{2.3}.




\section{Пространства $L_{2\pi}^{p(x)}$ и $W_{2\pi}^{r,p(x)}$.}\label{s3}

Если переменный показатель $p(x)$ является $2\pi$-периодической функцией, то через  $L_{2\pi}^{p(x)}$ обозначим пространство
$2\pi$-периодических функций $f(x)$, сужение на $[-\pi,\pi]$ которых принадлежит пространству $L^{p(x)}([-\pi,\pi])$. Норма в $L_{2\pi}^{p(x)}$ мы определим равенством
\begin{equation}\label{3.1}
    \|f\|_{p(\cdot)}=\inf \left\{\alpha>0:   \int_{-\pi}^{\pi}\left|{f(x)\over\alpha}\right|^{p(x)}dx\le1\right\}.
\end{equation}
 В пространстве $L_{2\pi}^{p(x)}$ мы выделим подпространство $W_{2\pi}^{r,p(x)}$, состоящее из \\ $r-1$-раз непрерывно дифференцируемых функций $f(x)$, для которых $f^{(r-1)}(x)$ абсолютно непрерывна на $[-\pi,\pi]$, а $f^{(r)}(x)\in L_{2\pi}^{p(x)}$.




\section{Классы переменных показателей ${\cal P}^\alpha_{2\pi}$ и равномерная ограниченность в $L_{2\pi}^{p(x)}$ некоторых семейств операторов свертки }\label{s4}

Пусть $\alpha>0$. Через ${\cal P}^\alpha_{2\pi}$ обозначим множество всех $2\pi$-периодических переменных показателей $p(x)$, удовлетворяющих условию
\begin{equation}\label{4.1}
    |p(x)-p(y)|\left(\ln\frac{2\pi}{|x-y|}\right)^\alpha\le c \quad (x,y\in [0,2\pi]),
\end{equation}
где здесь и далее через $c(p),c(p,q), \ldots$ обозначаются положительные числа, зависящие лишь от указанных параметров, вообще говоря различные в разных местах. При $\alpha=1$ мы положим  ${\cal P}_{2\pi}= {\cal P}^1_{2\pi}$. Через  $\hat{\cal P}^\alpha_{2\pi}$
обозначим подкласс переменных показателей из  $p(x)\in {\cal P}^\alpha_{2\pi}$, для которых $p_-=p_-([-\pi,\pi])>1$ и, соответственно, положим  $\hat{\cal P}_{2\pi}=\hat{\cal P}^1_{2\pi}$. Аналогичные классы переменных показателей
$p(x)$, определенных на $[0,1]$, впервые были рассмотрены в работе автора \cite{Shar1}, в которой были найдены необходимые и достаточные условия, которые гарантируют базисность системы Хаара в пространстве $L^{p(x)}([0,1])$. Что касается классов  ${\cal P}^\alpha_{2\pi}$,
то они впервые появились в работе автора \cite{Shar5}, в которой были получены необходимые и достаточные условия, которые обеспечивают
равномерную ограниченность в $L_{2\pi}^{p(x)}$ некоторых семейств операторов свертки.

Пусть для каждого $\lambda\geq1$ задана измеримая
 $2\pi$-периодическая и существенно ограниченная  функция (ядро)
 $K_{\lambda}=K_{\lambda}(x)$. Тогда мы можем определить линейный
 оператор
  \begin{equation}\label{4.2}
    k_{\lambda}f=(k_{\lambda}f)(x)=\int\limits_{-\pi}^{\pi}f(t)K_{\lambda}(t-x)dt,
 \end{equation}
действующий в пространстве $L_{2\pi}^{p(x)}$. В работе автора \cite{Shar5} был рассмотрен
 вопрос о равномерной ограниченности в $L^{p(x)}_{2\pi}$
семейства операторов $\{k_{\lambda}\}_{n\geq1}$.   Мы сформулируем один результат из \cite{Shar5} (см. также \cite{Shar3}), который нам понадобится в дальнейшем.
Предположим, что $K_{\lambda}=K_{\lambda}(x)$ ---
измеримая $2\pi$-периодическая функция. Будем говорить, что
семейство ядер $\{K_{\lambda}(x)\}_{1\leq\lambda<\infty}$
удовлетворяет, соответственно, условиям $A)$, $B)$, $C)$, если имеют
место следующие оценки:\\
$A)\qquad\int\limits_{-\pi}^{\pi}|K_{\lambda}(x)|dx\le c_{1},$\\
$B)\qquad\sup\limits_{x}|K_{\lambda}(x)|\leq c_{2}\lambda^{v},$\\
$C)\qquad|K_{\lambda}(x)|\leq c_{3}\text{ при
}\lambda^{-\gamma}\leq|x|<\pi$,\\
где $v,\gamma, c_{j}>0$ и не зависят от $\lambda$.
\begin{theorem}\label{t A}
 Пусть $K_{\lambda}(x)$
$(1\leq\lambda<\infty)$ удовлетворяет условиям $A) - C)$. Тогда,
если $ p(x)\in {\cal P}_{2\pi} $ ,  то семейство операторов $\{k_{\lambda}\}_{\lambda\geq1}$ равномерно
ограничено в $L_{2\pi}^{p(x)}$, другими словами, найдется положительное число $c=c(p)$, для которого имеет место неравенство
$$
 \|k_{\lambda}(f)\|_{p(\cdot)}\le c(p)\|f\|_{p(\cdot)},\quad 1\leq\lambda<\infty, f\in L_{2\pi}^{p(x)}.
 $$
\end{theorem}
Для $f(x)\in L_{2\pi}^{p(x)}$  рассмотрим средние Фейера
\begin{equation}\label{4.3}
    F_m(f,x)=V_m^0(f,x)=\frac1{m+1}[S_0(f,x)+\cdots+S_{m}(f,x)].
\end{equation}
Известно, что \cite{Zigmund}, что
\begin{equation}\label{4.4}
    F_m(f)=F_m(f,x)=\int\limits_{-\pi}^{\pi}f(t)K_{m}(t-x)dt,
\end{equation}
где
\begin{equation}\label{4.5}
    K_{m}(u)=\frac1{2(m+1)}\left({\sin  \frac{(m+1)u}{2}\over\sin \frac{u}{2}}\right)^2,
\end{equation}
причем
\begin{equation}\label{4.6}
    \int\limits_{-\pi}^{\pi}K_{m}(t)dt=1.
\end{equation}
Из \eqref{4.5} и \eqref{4.6} следует, что последовательность ядер Фейера $K_{m}(u)$ ($m=1,2,\ldots$) удовлетворяет условиям $A)$ -- $C)$ c $v=1$, $\gamma=\frac12$, поэтому в силу теоремы 1 имеет место
\begin{corollary}\label{1}
 Если $ p(x)\in {\cal P}_{2\pi} $ ,  то семейство операторов $\{F_{m}\}_{m\geq1}$ равномерно
ограничено в $L_{2\pi}^{p(x)}$, другими словами, найдется положительное число $c=c(p)$, для которого имеет место неравенство
$$
 \|F_{m}(f)\|_{p(\cdot)}\le c(p)\|f\|_{p(\cdot)}\quad  (f\in L_{2\pi}^{p(x)},\quad m=1,2,\ldots).
 $$
\end{corollary}

Из определения \eqref{1.4} для средних Валле Пуссена $V_m^n(f,x)$ находим
\begin{equation}\label{4.7}
    V_m^n(f,x)=\frac{m+n+1}{m+1}F_{m+n}(f,x)-\frac{n}{m+1}F_{n-1}(f,x).
\end{equation}
Из \eqref{4.7} следствия 1 вытекает
\begin{corollary}\label{2}
 Если $ p(x)\in {\cal P}_{2\pi} $ , $a>0$,  то семейство операторов $\{V_{m}^n(f)\}_{n\le am}$ равномерно
ограничено в $L_{2\pi}^{p(x)}$, другими словами, найдется положительное число $c=c(p,a)$, для которого имеет место неравенство
$$
 \|V_m^n(f)\|_{p(\cdot)}\le c(p,a)\|f\|_{p(\cdot)}\quad  (f\in L_{2\pi}^{p(x)},\quad n\le am).
 $$
\end{corollary}

Если $ p(x)\in \hat{\cal P}_{2\pi} $, то  для частичных сумм
 $S_n(f)=S_n(f,x)$ ряда Фурье функции $f(x)\in L^{p(x)}_{2\pi}$  имеет место \cite{Shar2} оценка
 \begin{equation}\label{4.8}
    \|S_n(f)\|_{p(\cdot)}\le c(p)\|f\|_{p(\cdot)}\quad(n=1,2,\ldots).
 \end{equation}

С другой стороны, в силу \eqref{1.4} имеем
\begin{equation}\label{4.9}
  \|V_m^n(f)\|_{p(\cdot)}\le \frac{1}{m+1}\sum_{k=n}^{m+n}\|S_k(f)\|_{p(\cdot)}.
 \end{equation}
Из \eqref{4.8} и \eqref{4.9} выводим следующую оценку
\begin{equation}\label{4.10}
  \|V_m^n(f)\|_{p(\cdot)}\le c(p)\|f\|_{p(\cdot)}\quad(m,n=0,1,2,\ldots),
 \end{equation}
где $ p(x)\in \hat{\cal P}_{2\pi} $, $f(x)\in L^{p(x)}_{2\pi}$.










\section{Модуль непрерывности и первая теорема Джексона в  $L_{2\pi}^{p(x)}$}\label{s5}

Если $f(x)\in L_{2\pi}^{p(x)} $, то мы можем определить функцию Стеклова с помощью равенства
\begin{equation}\label{4.1}
    s_h(f)=s_h(f)(x)=\frac1h\int_0^hf(x+t)dt=\frac1h\int_x^{x+h}f(t)dt.
\end{equation}
Оператор, сопоставляющий функции  $f=f(x)$ функцию Стеклова $s_h(f)=s_h(f)(x)$ назовем оператором Стеклова. В работах автора  \cite{Shar5},
\cite{Shar2} (см. также \cite{Shar3}) было показано, что если $p(x)\in{\cal P}^\alpha_{2\pi}$, то семейство операторов $\{s_h(f)=s_h(f)(x)\}_{|h|>0}$ равномерно ограничено в $ L_{2\pi}^{p(x)} $, т.е.
\begin{equation}\label{5.2}
    \|s_h(f)\|_{p(\cdot)}\le c(p)\|f\|_{p(\cdot)}.
\end{equation}
Отсюда легко показать (см. \cite{Shar5}, \cite{Shar3}), что
\begin{equation}\label{5.3}
    \lim_{h\to0}\|f-s_h(f)\|_{p(\cdot)}=0.
\end{equation}
Рассмотрим в $L^{p(x)}_{2\pi}$ при $0<|h|\le1$, $\gamma>0$, $|\tau|\le\pi |h|^\gamma$ семейство
сдвигов функций  В.А.Стеклова
\begin{equation}\label{5.4}
   {\cal S}_{h,\tau}(f)={\cal
S}_{h,\tau}(f)(x)=s_h(f)(x+\tau-\frac{h}{2})=
\frac{1}{h}\int\limits_{x+\tau-\frac{h}{2}}^{x+\tau+\frac{h}{2}}f(t)dt.
\end{equation}

В работе \cite{Shar2}  доказано, что если $p(x)\in {\cal P}_{2\pi}$, то
имеет место оценка
\begin{equation}\label{5.5}
 \|{\cal S}_{h,\tau}(f)\|_{p(\cdot)}\le
c(p,\gamma)(2\pi+1)^{p^-}\|f\|_{p(\cdot)} \quad 0<|h|\le1, |\tau|\le\pi
|h|^\gamma,
\end{equation}
где $p^-=\max\limits_{x}p(x)$.

Введем следующую величину \cite{Shar2}, \cite{Shar3}, \cite{Shar6}
\begin{equation}\label{5.6}
    \Omega(f,0)_{p(\cdot)}=0,\quad \Omega(f,\delta)_{p(\cdot)}=\sup_{0<h<\delta}\|f-s_h(f)\|_{p(\cdot)}.
\end{equation}
Из \eqref{5.3} и \eqref{5.6} непосредственно следует, что
\begin{equation}\label{5.7}
    \lim_{\delta\to0}\Omega(f,\delta)_{p(\cdot)}=0.
\end{equation}
Будем называть величину $\Omega(f,\delta)_{p(\cdot)}$ {\it модулем непрерывности} функции $f\in L_{2\pi}^{p(x)} $.
Далее, через $E_n(f)_{p(\cdot)}$  обозначим наилучшее приближение функции $f\in L_{2\pi}^{p(x)} $ тригонометрическими полиномами $T_n(x)$ порядка $n$, т.е.

\begin{equation}\label{5.8}
    E_n(f)_{p(\cdot)}=\inf_{T_n}\|f-T_n\|_{p(\cdot)}.
\end{equation}
 В работе автора \cite{Shar6} доказано, что если $p(x)\in{\cal P}_{2\pi}$ и $f\in L_{2\pi}^{p(x)} $, то имеет место следующая оценка (первая теорема типа Джексона)
\begin{equation}\label{5.9}
    E_n(f)_{p(\cdot)}\le c(p)\Omega(f,\frac{1}{n+1})_{p(\cdot)}.
\end{equation}


\section{Некоторые вспомогательные результаты}\label{s6}

Пусть $r\ge1$, $f(x)\in W_{2\pi}^{r,p(x)} $. Тогда имеет место (\cite{Zigmund}, стр.75) равенство
\begin{equation}\label{6.1}
    f(x)=\frac{a_0}{2}+ \frac{1}{\pi}\int_{-\pi}^\pi f^{(r)}(t)B_r(t-x)dt,
\end{equation}
где
\begin{equation}\label{6.2}
    B_r(u)=\sum_{k=1}^\infty {\cos(ku+\frac{\pi r}{2})\over k^r}
\end{equation}
функция Бернулли. Если запишем \eqref{6.1} для функции $f(x)-S_n(f,x)$, где $S_n(f,x)$ -- частичная сумма ряда Фурье функции $f(x)$ (см. \eqref{1.3}), то получим ($S_n^{(r)}(f,x)=S_n(f^{(r)},x)$)
$$
R_n(f,x)=f(x)-S_n(f,x)=\frac{1}{\pi}\int_{-\pi}^\pi( f^{(r)}(t)-S_n(f^{(r)},x))B_r(t-x)dt=
$$
\begin{equation}\label{6.3}
    \frac{1}{\pi}\int_{-\pi}^\pi f^{(r)}(t)R_{r,n}(t-x)dt,
\end{equation}
где
\begin{equation}\label{6.4}
     R_{r,n}(u)=\sum_{k=n+1}^\infty {\cos(ku+\frac{\pi r}{2})\over k^r}.
\end{equation}
Чтобы получить представление, аналогичное \eqref{6.3} для разности $f(x)-V_m^n(f,x)$ рассмотрим функцию
\begin{equation}\label{6.5}
    {\cal V}_{r,m}^n(u)=\sum_{k=n}^{n+m}R_{r,k}(u).
\end{equation}
Из (1.4) и (6.3) имеем
$$
f(x)-V_m^n(f,x)=\frac{1}{m+1}\sum_{k=n}^{n+m}(f(x)-S_k(f,x))
$$
$$
=\frac{1}{m+1}\sum_{k=n}^{n+m}\frac{1}{\pi}\int_{-\pi}^\pi f^{(r)}(t)R_{r,k}(t-x)dt
$$
\begin{equation}\label{6.6}
    =\frac{1}{(m+1)\pi}\int_{-\pi}^\pi f^{(r)}(t){\cal V}_{r,m}^n(t-x)dt.
\end{equation}
Заметим, что $R_{r,k}(t-x)$ ортогонален к произвольному тригонометрическому полиному $T_k(t)$
порядка $k$, поэтому из (6.6) имеем
\begin{equation}\label{6.7}
  f(x)-V_m^n(f,x)= \frac{1}{m+1}\sum_{k=n}^{n+m}\frac{1}{\pi}\int_{-\pi}^\pi (f^{(r)}(t)-T_k(f^{(r)},t))R_{r,k}(t-x)dt,
 \end{equation}
где $T_k(g)=T_k(g,t)$ полином наилучшего приближения к функции  $g\in L_{2\pi}^{p(x)}$, т.е.
\begin{equation}\label{6.8}
    E_k(g)_{p(\cdot)}=\|g-T_k(g)\|_{p(\cdot)}=\inf_{T_k}\|g-T_k\|_{p(\cdot)}.
\end{equation}
В частном случае, если $p(x)>1$ почти всюду на $[-\pi,\pi]$, то из \eqref{2.3} и \eqref{6.7} следует, что
\begin{equation}\label{6.9}
    |f(x)-V_m^n(f,x)|\le \frac{1}{\pi(m+1)}\sum_{k=n}^{n+m}E_k(f^{(r)})_{p(\cdot)}\|R_{r,k}(*-x)\|_{q(\cdot)}^*
\end{equation}
где $q(x)={p(x)\over p(x)-1}$ -- сопряженная функция.

Рассмотрим некоторые свойства семейства функций
\begin{equation}\label{6.10}
    \kappa_{r,m}^n(u)=\frac{(n+1)^r}{m+1}{\cal V}_{r,m}^n(u) \quad (n,m=0,1,\ldots).
\end{equation}

\begin{lemma}\label{l6.1} Имеют следующие равенства
 $$
 \kappa_{2s,m}^n(u)=
 $$
 $$
 {(-1)^s(n+1)^{2s}\over m+1}\sum\limits_{l=0}^{m-1}
\sum\limits_{k=0}^{\infty}\frac{\sin\frac{l+1}{2}u\sin\frac{k+1}{2}u\cos(2n+k+l+2)
\frac u2}{2\sin^2\frac u2}\Delta^2g_s(n+1+k+l)
$$
 \begin{equation}\label{6.11}
    +{(-1)^{s-1}(n+1)^{2s}\over m+1}\sum\limits_{k=0}^{\infty}
\frac{\sin\frac {m+1}{2}u\sin\frac{k+1}{2}u\cos(2n+m+k+2)\frac u2}{2\sin^2\frac u2}
\Delta q_s(n+m+k+1),
 \end{equation}
$$
 \kappa_{2s-1,m}^n(u)=
 $$
 $$
 {(-1)^s(n+1)^{2s-1}\over m+1}\sum\limits_{l=0}^{m-1}
\sum\limits_{k=0}^{\infty}\frac{\sin\frac{l+1}{2}u\sin\frac{k+1}{2}u\sin(2n+k+l+2)
\frac u2}{2\sin^2\frac u2}\Delta^2g_s(n+1+k+l)
 $$
 \begin{equation}\label{6.12}
 +{(-1)^{s-1}(n+1)^{2s-1}\over m+1}\sum\limits_{k=0}^{\infty}
\frac{\sin\frac {m+1}{2}u\sin\frac{k+1}{2}u\sin(2n+m+k+2)\frac u2}{2\sin^2\frac u2}
\Delta q_s(n+m+k+1),
  \end{equation}
 где $g_s(t)=t^{-2s}$, $q_s(t)=t^{-2s+1}$, $\Delta\varphi(t)=\varphi(t+1)-\varphi(t)$,
 $\Delta^2\varphi(t)=\varphi(t+2)-2\varphi(t+1)+\varphi(t)$.
\end{lemma}
 \begin{proof}
Из \eqref{6.4}, \eqref{6.5} и \eqref{6.10} имеем
$$
\kappa_{r,m}^{n}(u)={(n+1)^r\over m+1}\sum\limits_{l=0}^m\sum\limits_{k=0}^\infty
\frac{\cos\left[(n+k+l+1)u+\frac{\pi r}{2}\right]}{(n+k+l+1)^r},
$$
поэтому, с помощью преобразования Абеля мы можем записать
\begin{equation}\label{6.13}
    \kappa_{r,m}^{n}(u)={(n+1)^r\over m+1}\sum\limits_{l=0}^m
\sum\limits_{k=0}^\infty\left[\frac{1}{(n+1+k+l)^r}-\frac{1}{(n+2+k+l)^r}
\right]v^n_{k,l}(u),
\end{equation}
где
\begin{equation}\label{6.14}
    v^n_{k,l}(u)=\sum\limits_{j=0}^{k}\cos[(n+1+l+j)u+\frac{\pi r}{2}].
\end{equation}
Мы рассмотрим два случая, когда $r$ является четным или нечетным. Если
$r=2s$, то $\cos(\mu u+\frac{\pi r}{2})=(-1)^s\cos\mu u$, следовательно, \eqref{6.14} принимает вид
$$
v^n_{k,l}(u)=(-1)^s\sum\limits_{j=0}^k\cos(n+1+l+j)u=
$$
\begin{equation}\label{6.15}
    =(-1)^s\cdot\frac{\sin(2(n+1+l+k)+1)\frac u2-\sin(2(n+l)+1)\frac u2}{2\sin\frac u2}
\end{equation}


Из \eqref{6.13} и \eqref{6.15} имеем
$$
\kappa_{2s,m}^n(u)={(-1)^{s-1}(n+1)^{2s}\over m+1}\times
$$
\begin{equation}\label{6.16}
    \sum\limits_{k=0}^\infty\sum\limits_{l=0}^m \Delta
g_s(n+1+k+l)\frac{\sin(2(n+1+l+k)+1)\frac u2- \sin(2(n+l)+1)\frac u2}{2\sin\frac u2}.
\end{equation}
Применяя к внутренней сумме из \eqref{6.16} преобразование Абеля, мы получим
$$
\kappa_{2s,m}^n(u)={(-1)^{s}(n+1)^{2s}\over m+1}\sum\limits_{k=0}^\infty
\sum\limits_{l=0}^{m-1}\Delta^2 g_s(n+1+k+l)\frac{W^n_{k,l}(u)-W^n_{-1,l}(u)}{2\sin\frac u2}+
$$
$$+{(-1)^{s-1}(n+1)^{2s}\over m+1}\sum\limits_{k=0}^\infty\Delta g_s(m+n+k+1)
\frac{W^n_{k,m}(u)-W^n_{-1,m}(u)}{2\sin\frac u2},
$$
где
$$W^n_{k,l}(u)=\sum\limits_{\mu=0}^l\sin(2(n+1+\mu+k)+1)\frac u2=
$$
$$
\frac{\sin^2(n+l+k+2)\frac u2-\sin^2(n+k+1)\frac u2}{\sin\frac u2}=
$$
$$
\frac{1}{\sin\frac u2}(\sin(n+l+k+2)\frac u2-\sin(n+k+1)\frac u2)\times
$$
$$
(\sin(n+l+k+2)\frac u2+\sin(n+k+1)\frac u2)=
$$
$$\frac{4}{\sin\frac u2}\sin\frac{l+1}{4}u\cdot\cos(n+k+1+\frac{l+1}{2})\frac u2\times
$$
\begin{equation}\label{6.17}
    \sin(n+k+1+\frac{l+1}{2})\frac u2\cos\frac{l+1}{4}u
\end{equation}
В силу  \eqref{6.17}
$$
W_{k,l}^n(u)-W_{-1,l}^n(u)=\frac{4}{\sin\frac u2}\sin\frac{l+1}{4}u\cos\frac{l+1}{4}u\times
$$
$$
\left(\sin(n+k+1+\frac{l+1}{2})\frac u2\cos(n+k+1+\frac{l+1}{2})\frac u2-\right.
$$
$$
\left.-\sin(n+\frac{l+1}{2})\frac u2\cos(n+\frac{l+1}{2})\frac u2\right)=
$$
$$
\frac{\sin(l+1)\frac{u}{2}}{\sin\frac u2}
\left(\sin(2(n+k+1)+l+1)\frac u2-\sin(2n+l+1)\frac u2\right)=
$$
\begin{equation}\label{6.18}
    \frac{1}{\sin\frac u2}\sin(l+1)\frac{u}{2}\sin(k+1)\frac{u}{2}
\cos(2n+k+l+2)\frac u2,
\end{equation}
поэтому равенство \eqref{6.11}  вытекает из \eqref{6.16} и \eqref{6.18}.

Перейдем к случаю $r=2s-1$. Имеем $\cos(\mu u+\frac{\pi (2s-1)}{2})=(-1)^s\sin\mu u$, следовательно, \eqref{6.14} принимает вид
$$
v^n_{k,l}(u)=(-1)^s\sum\limits_{j=0}^k\sin(n+1+l+j)u=(-1)^s\times
$$
$$
    \frac{\sin(n+1+l+k)\frac u2\sin(n+2+l+k)\frac u2-\sin(n+l)\frac u2\sin(n+l+1)\frac u2} {\sin\frac u2}=
$$
\begin{equation}\label{6.19}
    \frac{\cos(2n+2l+1)\frac u2-\cos(2n+2l+2k+3)\frac u2} {2\sin\frac u2}.
\end{equation}


Из \eqref{6.13} и \eqref{6.19} имеем
$$
\kappa_{2s-1,m}^n(u)={(-1)^{s-1}(n+1)^{2s-1}\over m+1}\times
$$
\begin{equation}\label{6.20}
    \sum\limits_{k=0}^\infty\sum\limits_{l=0}^m \Delta
g_s(n+1+k+l) \frac{\cos(2n+2l+1)\frac u2-\cos(2n+2l+2k+3)\frac u2} {2\sin\frac u2}.
\end{equation}
Применяя к внутренней сумме из \eqref{6.20} преобразование Абеля, мы получим
$$
\kappa_{2s-1,m}^n(u)={(-1)^{s}(n+1)^{2s-1}\over m+1}\sum\limits_{k=0}^\infty
\sum\limits_{l=0}^{m-1}\Delta^2 g_s(n+1+k+l)\frac{Y^n_{-1,l}(u)-Y^n_{k,l}(u)}{2\sin\frac u2}+
$$
$$+{(-1)^{s-1}(n+1)^{2s-1}\over m+1}\sum\limits_{k=0}^\infty\Delta g_s(m+n+k+1)
\frac{Y^n_{-1,m}(u)-Y^n_{k,m}(u)}{2\sin\frac u2},
$$
где
$$Y^n_{k,l}(u)=\sum\limits_{\mu=0}^l\cos(2(n+1+\mu+k)+1)\frac u2=
$$
$$
\frac{\sin(n+k+l+2)u-\sin(n+k+1)u}{2\sin\frac u2}=
$$
$$
\frac{\sin(l+1)\frac{u}{2}\cos(2n+2k+2l+3)\frac{u}{2}}{2\sin\frac u2}.
$$
Отсюда имеем
$$
Y_{-1,l}^n(u)-Y_{k,l}^n(u)=
$$
$$
\frac{\sin(l+1)\frac{u}{2}}{2\sin\frac u2}(\cos(2n+2l+1)\frac{u}{2}-\cos(2n+2k+2l+3)\frac{u}{2})=
$$
\begin{equation}\label{6.21}
  \frac{1}{\sin\frac u2}\sin(l+1)\frac{u}{2}\sin(k+1)\frac{u}{2}
\sin(2n+l+k+2)\frac{u}{2}.
\end{equation}
Равенства (6.12) вытекает из \eqref{6.20} и \eqref{6.21}. Лемма 6.1 доказана.


 \end{proof}

\begin{lemma}\label{l6.2}
  Пусть $0\le k\le l$. Тогда
$$A_{k,l}=\int\limits_{0}^{\pi}\frac{|\sin\frac{k+1}{2}u\sin\frac{l+1}{2}u|}
{\sin^2\frac u2}du\le2(k+1)(2+\ln\frac{l+1}{k+1})+\pi.$$
\end{lemma}
\begin{proof} Имеем
$$
A_{k,l}=2\int\limits_{0}^{\frac\pi2}\frac{|\sin(k+1)u\sin(l+1)|}{\sin^2u}du=
2\int\limits_{0}^{\frac\pi2}\frac{|\sin(k+1)u\sin(l+1)|}{u^2}du+
$$
\begin{equation}\label{6.22}
2\int\limits_{0}^{\frac\pi2}|\sin(k+1)u\sin(l+1)u|\varphi(u)du,
   \end{equation}
где
$$
\varphi(u)=\frac{1}{\sin^2u}-\frac{1}{u^2}=\frac{u^2-\sin^2u}{u^2\sin^2u}.
$$

Пусть $0<u<\frac\pi2$, тогда
$$
\varphi(u)=\frac{(u+\sin u)(u-\sin u)}{u^2\sin^2u}=
\frac{(2u-\frac{u^3}{3!}+\frac{u^5}{5!}-\ldots)(\frac{u^3}{3!}-\frac{u^5}{5!}+\ldots)}{u^2\sin^2u}=
$$
$$
\frac{(2-\frac{u^2}{3!}+\frac{u^4}{5!}-\ldots)
(\frac{1}{3!}-\frac{u^2}{5!}+\ldots)}{\left(\frac{\sin u}
{u}\right)^2}=
$$
$$
\frac{(2-\frac{u^2}{3!}+\frac{u^4}{5!}-\ldots)(\frac{1}{3!}-
\frac{u^2}{5!}+\ldots)}{(1-\frac{u^2}{3!}+\frac{u^4}{5!}-\ldots)^2}
<\frac{\frac{2}{3!}}{\left(1-\frac{u^2}{3!}\right)^2}<1
$$
и, следовательно,
\begin{equation}\label{6.23}
 2\int\limits_{0}^{\frac\pi2}|\sin(k+1)u\sin(l+1)u|\varphi(u)du<\pi.
\end{equation}

С другой стороны
$$\int\limits_{0}^{\frac\pi2}\frac{|\sin(k+1)u\sin(l+1)u|}{u^2}du=
(k+1)\int\limits_{0}^{\frac\pi2(k+1)}\frac{|\sin u \sin\frac{l+1}{k+1}u|}{u^2}du\le
$$
$$(k+1)\int\limits_{0}^{1}\frac{|\sin\frac{l+1}{k+1}u|}{u}du+
(k+1)\int\limits_{1}^{\frac\pi2(k+1)}\frac{du}{u^2}=
$$
$$
(k+1)\int\limits_{0}^{\frac{l+1}{k+1}}\frac{|\sin u|}{u}du+
(k+1)=
$$
\begin{equation}\label{6.24}
    (k+1)\left[\int\limits_{0}^{1}\frac{\sin u}{u}du+
\int\limits_{0}^{\frac{l+1}{k+1}}\frac{du}{u}+1\right]<
(k+1)\left(2+\ln\frac{l+1}{k+1}\right).
\end{equation}
Утверждение леммы следует  \eqref{6.22} и неравенств \eqref{6.23} и \eqref{6.24}.
\end{proof}

\begin{lemma}\label{l6.3}
Пусть $r\ge1$. Тогда для произвольных $n$ и $m$ имеет место неравенство
$$
\int\limits_{-\pi}^{\pi}|\kappa_{r,m}^n(u)|du\le c(r)\ln\left(2+\frac{n}{m+1}\right) .
$$
Если, кроме того $n\le m$, то справедливы  следующие неравенства
$$
\int\limits_{-\pi}^{\pi}|\kappa_{r,m}^n(u)|du\le c(r)\frac{n+1}{m+1}\quad (r\ge2),
$$
$$
\int\limits_{-\pi}^{\pi}|\kappa_{1,m}^n(u)|du\le c\frac{n+1}{m+1}\left(1+\ln\frac{m+1}{n+1}\right).
$$





 \end{lemma}
\begin{proof}
 В силу леммы 6.1 мы можем записать
 \begin{equation}\label{6.25}
    \int\limits_{-\pi}^{\pi}|\kappa_{r,m}^n(u)|du\le L_1+L_2,
 \end{equation}
 где
 \begin{equation}\label{6.26}
    L_1={(n+1)^{r}\over m+1}
\sum\limits_{l=0}^{m-1}\sum\limits_{k=0}^{\infty}
\Delta^2F_r(n+k+l+1)\frac12\int\limits_{-\pi}^{\pi}
\frac{|\sin\frac{k+1}{2}u\sin\frac{l+1}{2}u|}{\sin^2\frac
u2}du,
 \end{equation}
 \begin{equation}\label{6.27}
L_2={(n+1)^{r}\over m+1}\sum\limits_{k=0}^{\infty}\Delta F_r(n+m+k+1)\frac12
\int\limits_{-\pi}^{\pi}\frac{|\sin\frac{m+1}{2}u\sin\frac{k+1}{2}u|}{\sin^2\frac u2}du,
 \end{equation}
где $F_r(t)=t^{-r}$ и в силу леммы 6.2

\begin{equation}\label{6.28}
    \frac12\int\limits_{-\pi}^{\pi}\frac{|\sin\frac{k+1}{2}u\sin\frac{l+1}{2}u|}{\sin^2\frac u2}du=
\begin{cases}
2(k+1)(2+\ln\frac{l+1}{k+1})+\pi,\quad k\le l,   \\
2(l+1)(2+\ln\frac{k+1}{l+1})+\pi, \quad l<k.
\end{cases}
\end{equation}
Далее, поскольку $\Delta^2F_r(t)=F_r''(\bar{t})$ $(t\le\bar{t}\le t+2)$, то
\begin{equation}\label{6.29}
   \Delta^2F_r(n+1+k+l)=F_r''(\bar{t})=
\frac{r(r+1)}{\bar{t}^{r+2}}\le\frac{r(r+1)}{(n+1+k+l)^{r+2}},
\end{equation}
where $n+1+k+l<\bar{t}<n+k+l+3$ and, similarly,
\begin{equation}\label{6.30}
    \Delta F_r(n+m+k+1)\le\frac{r}{(n+m+k+1)^{r+1}}.
\end{equation}
Оценим $L_1$. Из (6.26), (6.28) и (6.29) имеем
$$
L_1\le {r(r+1)(n+1)^r\over m+1}\sum_{l=0}^{m-1}\sum\limits_{k=0}^l\frac{2(k+1)(2+\ln\frac{l+1}{k+1})+\pi}{(n+1+k+l)^{r+2}}+ $$
\begin{equation}\label{6.31}
{r(r+1)(n+1)^r\over m+1}\sum_{l=0}^{m-1}\sum\limits_{k=l+1}^\infty\frac{2(l+1)(2+\ln\frac{k+1}{l+1})+\pi}{(n+1+k+l)^{r+2}}=L_{11}+L_{12}.
  \end{equation}
  Далее имеем
  $$
  L_{11}\le {r(r+1)(n+1)^r\over m+1}\sum_{l=0}^{m-1}\frac{1}{(n+l)^{r+2}}\sum\limits_{k=0}^l(2(k+1)(2+\ln\frac{l+1}{k+1})+\pi)\le
  $$
$$
  {c(r)\over m+1}\sum_{l=0}^{m-1}\frac{1}{(n+l)^{2}}\sum\limits_{k=0}^l(4(k+1)+2(k+1)\ln\frac{l+1}{k+1}+\pi)\le
  $$
$$
  {c(r)\over m+1}\sum_{l=0}^{m-1}\frac{1}{(n+l)^{2}}\sum\limits_{k=0}^l(4(k+1)+2(l-k)+\pi)\le
  $$
\begin{equation}\label{6.32}
    {c(r)\over m+1}\sum_{l=0}^{m-1}\frac{l^2}{(n+l)^{2}}\le c(r).
\end{equation}
Оценим $L_{12}$. С этой целью заметим, что
$$
\sum\limits_{k=l+1}^\infty\frac{2(l+1)(2+\ln\frac{k+1}{l+1})+\pi}{(n+1+k+l)^{r+2}}\le
$$
$$
\sum\limits_{k=l+1}^\infty\frac{4(l+1)+\pi}{(n+1+k+l)^{r+2}}+
\sum\limits_{k=l+1}^\infty\frac{2(l+1)\ln(1+\frac{k-l}{l+1})}{(n+1+k+l)^{r+2}}\le
$$
$$
{c(r)(l+1)\over(n+l)^{r+1}}+2(l+1)\sum\limits_{k=l+1}^{n+2l}\frac{\ln(1+\frac{k-l}{l+1})}{(n+1+k+l)^{r+2}}+
$$
$$
2(l+1)\sum\limits_{k=n+2l+1}^\infty\frac{\ln(1+\frac{k-l}{l+1})}{(n+1+k+l)^{r+2}}\le
$$
$$
{c(r)(l+1)\over(n+l)^{r+1}}+{c(r)(l+1)\ln(1+\frac{n+l+1}{l+1})\over(n+l)^{r+1}}+
$$
$$
\frac{2(l+1)}{(l+1)^{r+2}}\sum\limits_{k=n+2l+1}^\infty\frac{\ln(1+\frac{k-l}{l+1})}{({n+2l+1\over l+1}+{k-l\over l+1})^{r+2}}\le
$$
$$
{c(r)(n+l+1)\over(n+l)^{r+1}}+\frac{2}{(l+1)^{r+1}}\sum\limits_{j=n+l+1}^\infty\frac{\ln(1+\frac{j}{l+1})}{(1+{j\over l+1})^{r+2}}\le
$$
$$
{c(r)(n+l+1)\over(n+l)^{r+1}}+\frac{2}{(l+1)^{r}}\int\limits_\frac{n+l}{l+1}^\infty {\ln(1+x)dx\over(1+x)^{r+2}}\le
$$
$$
{c(r)(n+l+1)\over(n+l)^{r+1}}+\frac{2\ln(1+{n+l\over l+1})}{(r+1)(l+1)^{r}(1+{n+l\over l+1})^{r+1}}+
$$
$$
\frac{2}{(r+1)(l+1)^{r}}\int\limits_\frac{n+l}{l+1}^\infty {dx\over(1+x)^{r+2}}\le
{c(r)\over(n+l)^{r}}+
$$
\begin{equation}\label{6.33}
 \frac{2(n+l)}{(n+2l+1)^{r+1}}+\frac{2}{(r+1)^2(l+1)^{r}(1+{n+l\over l+1})^{r+1}}\le {c(r)\over(n+2l+1)^{r}}.
 \end{equation}
Из \eqref{6.31} и \eqref{6.33}  выводим
\begin{equation}\label{6.34}
    L_{12}\le c(r).
\end{equation}
Оценки \eqref{6.32} и \eqref{6.34}, взятые вместе, дают
\begin{equation}\label{6.35}
    L_1\le c(r).
\end{equation}
Оценим $L_2$. Из \eqref{6.27}, \eqref{6.28} и \eqref{6.30} имеем
$$
L_2\le r{(n+1)^{r}\over m+1}\sum\limits_{k=0}^m\frac{2(k+1)(2+\ln\frac{m+1}{k+1})+\pi}{(n+1+k+m)^{r+1}}+
 $$
\begin{equation}\label{6.36}
    r{(n+1)^{r}\over m+1}\sum_{k=m+1}^\infty\frac{2(m+1)(2+\ln\frac{k+1}{m+1})+\pi}{(n+1+k+m)^{r+1}}=L_{21}+L_{22}.
 \end{equation}
Для $L_{21}$ находим
$$
L_{21}\le c(r){(n+1)^{r}\over m+1}\left[\sum_{k=0}^m\frac{k+1+\pi}{(n+1+k+m)^{r+1}}+\sum_{k=0}^m\frac{(k+1)\ln(1+\frac{m-k}{k+1})}{(n+1+k+m)^{r+1}}\right]\le
$$
\begin{equation}\label{6.37}
    {c(r)(n+1)^{r}m^2\over (m+1)(n+m+1)^{r+1}}+{c(r)(n+1)^{r}\over m+1}\sum_{k=0}^m{m-k\over (n+m+1)^{r+1}}\le c(r)\left({n+1\over n+m+1}\right)^r.
\end{equation}
Переходя к  $L_{22}$, имеем (см. \eqref{6.36})
$$
L_{22}\le c(r)n^r\sum_{k=m+1}^\infty\frac{1}{(n+1+k+m)^{r+1}}+c(r)n^r\sum_{k=m+1}^\infty\frac{\ln\frac{k+1}{m+1}}{(n+1+k+m)^{r+1}}
$$
$$
\le {c(r)n^r\over(n+m+1)^{r}}+{c(r)n^r\over(m+1)^{r+1}}\sum_{k=m+1}^\infty{\ln(1+\frac{k-m}{m+1})\over(\frac{n+2m+1}{m+1}+\frac{k-m}{m+1})^{r+1}}
$$
$$
\le {c(r)n^r\over(n+m+1)^{r}}+{c(r)n^r\over(m+1)^{r+1}}\sum_{j=1}^\infty{\ln(1+\frac{j}{m+1})\over(\frac{n+2m+1}{m+1}+\frac{j}{m+1})^{r+1}}
$$
$$
\le {c(r)n^r\over(n+m+1)^{r}}+{c(r)n^r\over(m+1)^{r+1}}\int_{0}^\infty{\ln(1+x)dx\over(\frac{n+2m+1}{m+1}+x)^{r+1}}
$$
$$
={c(r)n^r\over(n+m+1)^{r}}+{c(r)n^r\over(m+1)^{r}}\int_{0}^\infty{dx\over(1+x)(\frac{n+2m+1}{m+1}+x)^{r+1}}=
$$
$$
{c(r)n^r\over(n+m+1)^{r}}+{c(r)n^r\over(m+1)^{r}}\left[\left({m+1\over n+2m+1}\right)^r\int\limits_{0}^\frac{n+2m+1}{m+1}{dx\over 1+x}+\int\limits_\frac{n+2m+1}{m+1}^\infty{dx\over(1+x)^{r+1}}\right]
$$
$$
\le c(r)\left({n\over n+2m+1}\right)^r\ln\frac{n+2m+1}{m+1}+{c(r)n^r\over(m+1)^{r}}{1\over(1+\frac{n+2m+1}{m+1})^r}\le
$$
$$
c(r)\left({n+1\over n+2m+1}\right)^r\ln\frac{n+2m+1}{m+1}\le
$$
\begin{equation}\label{6.38}
    c(r)\left({n+1\over n+2m+1}\right)^r\ln\left(2+\frac{n}{m+1}\right).
\end{equation}
Из \eqref{6.36} -- \eqref{6.38} получаем
\begin{equation}\label{6.39}
    L_2\le  c(r)\ln\left(2+\frac{n}{m+1}\right)
\end{equation}
Первое утверждение леммы 6.3 вытекает из \eqref{6.25}, \eqref{6.35} и \eqref{6.39}. Переходя к доказательству второго и третьего утверждений  лемм 6.3, мы вновь оценим $L_{11}$  и $L_{12}$ (см. \eqref{6.31}) для $n\le m$.  Имеем (см.\eqref{6.32})

$$
  L_{11}\le
  {c(r)\over m+1}\sum_{l=0}^{m-1}\frac{1}{(n+l+1)^{2}}\sum\limits_{k=0}^l(4(k+1)+2(l-k)+\pi)\le
  $$
$$
    c(r){(n+1)^r\over m+1}\sum_{l=0}^{m-1}\frac{l^2}{(n+l+1)^{r+2}}\le
$$
$$
    c(r){(n+1)^r\over m+1}\sum_{l=0}^n\frac{l^2}{(n+l+1)^{r+2}}+
   c(r){(n+1)^r\over m+1}\sum_{l=n}^m\frac{l^2}{(n+l+1)^{r+2}}\le
$$
$$
    c(r){(n+1)^r\over m+1}\frac{1}{(n+1)^{r+2}}\sum_{l=0}^nl^2+
   c(r){(n+1)^r\over m+1}\sum_{l=n}^m\frac{1}{(l+1)^{r}}\le
$$
$$
c(r){n+1\over m+1}\begin{cases}1,&\text{$r\ge2$},\\1+\ln{m+1\over n+1},&\text{$r=1$.}\end{cases}
$$
Далее, из \eqref{6.31} и \eqref{6.33} при $n\le m$ имеем
$$
L_{12}\le c(r){(n+1)^r\over m+1}\sum_{l=0}^{m-1}\frac{1}{(n+2l+1)^r}\le
$$
$$
    c(r){(n+1)^r\over m+1}\sum_{l=0}^n\frac{1}{(n+2l+1)^r}+
   c(r){(n+1)^r\over m+1}\sum_{l=n}^m\frac{1}{(l+1)^{r}}\le
$$
$$
c(r){n+1\over m+1}\begin{cases}1,&\text{$r\ge2$},\\1+\ln{m+1\over n+1},&\text{$r=1$.}\end{cases}
$$
Из этих оценок для $L_{11}$ и $L_{12}$, учитывая также оценки \eqref{6.37} и \eqref{6.38}, мы убеждаемся в справедливости второго и третьего
утверждений леммы 6.3. Таким образом эта лемма полностью доказана.
\end{proof}

\begin{lemma}\label{l6.4}
 Пусть  $n^{-\frac12}\le u\le2\pi-n^{-\frac12}$, $r\ge1$. Тогда найдется такое,  число $c(r)>0$, что
 $$|\kappa_{r,m}^{n}(u)|\le c(r)\frac{n+1}{m+1}.$$
  \end{lemma}

\begin{proof}

 Если $n^{-\frac12}\le u\le2\pi-n^{-\frac12}$, то
  $\frac{1}{\sin^2\frac u2}\le\frac{\pi^2}{4}n$ и, поэтому в силу леммы 6.1
  и неравенств \eqref{6.29},\eqref{6.30} имеем
$$
|\kappa_{r,m}^n(u)|\le
$$
$$
\frac{c(r)n^{r+1}}{m+1}\left(\sum\limits_{l=0}^{m-1}
\sum\limits_{k=0}^{\infty}\frac{1}{(n+1+k+l)^{r+2}}+
\sum\limits_{k=0}^{\infty}\frac{1}{(n+m+k)^{r+1}}\right)\le c(r)\frac{n+1}{m+1}.
$$
 Лемма 6.4 доказана.
\end{proof}

\begin{lemma}\label{l6.5} Имеют место следующие оценки:
 $$|\kappa_{1,m}^{n}(u)|\le c(n+1), $$
 $$
 |\kappa_{2,m}^{n}(u)|\le c\frac{(n+1)^2}{n+m+1}\ln\left(1+\frac{m+1}{n+1}\right),
 $$
 $$
 |\kappa_{r,m}^{n}(u)|\le c(r)\frac{(n+1)^2}{n+m+1}\quad (r\ge3).
 $$
  \end{lemma}

\begin{proof}
Если $r\ge 2$, то справедливость утверждения леммы 6.5 непосредственно вытекает из определения \eqref{6.10} и равенств \eqref{6.4} и \eqref{6.5}. Поэтому достаточно рассмотреть случай $r=1$. В этом случае мы имеем $(0\le u\le2\pi)$
\begin{equation}\label{6.40}
    R_{1,n}(u)=\sum\limits_{k=n+1}^{\infty}\frac{\sin ku}{k}=
\frac{\pi-u}{2}-\sum\limits_{k=1}^{n}\frac{\sin ku}{k}.
\end{equation}
Хорошо известно  \cite{Zigmund} ( стр.105), что
\begin{equation}\label{6.41}
 \left|\sum\limits_{k=1}^{n}\frac{\sin ku}{k}\right|\le c\quad(n=1,2,\ldots),
 \end{equation}
Утверждение леммы следует из  \eqref{6.4}, \eqref{6.5}, \eqref{6.10}, \eqref{6.40} и \eqref{6.41}.
\end{proof}
Рассмотрим в пространстве $L^{p(x)}_{2\pi}$ семейство  линейных операторов свертки с помощью следующего равенства
\begin{equation}\label{6.42}
    \mathcal{K}_{n}(f)=\mathcal{K}_{n,m}(f)(x)=
\int\limits_{-\pi}^{\pi}f(t)\kappa_{r,m}^{n}(t-x)dt\quad(n=1,2,\ldots),
\end{equation}
в котором ядра $\kappa_{r,m}^{n}(u)$ определены равенством \eqref{6.10}. Имеет место следующая

\begin{lemma}\label{l6.6}
 Пусть $p=p(x)\in{\cal P}_{2\pi}$, $f\in L^{p(x)}_{2\pi}$, $a>0$, $n\le am$. Тогда
$$\|\mathcal{K}_{n,m}(f)\|_{p(\cdot)}\le c_r(a,p)\|f\|_{p(\cdot)}.$$
  \end{lemma}

\begin{proof}
Чтобы убедиться в справедливости утверждения леммы 6.6 достаточно заметить, что  для ядер $\kappa_{r,m}^{n}(u)$, определенных равенством \eqref{6.10}, условия  A), B) и С) из теоремы 1 (см. \S 4 ) имеют место равномерно относительно  $n$ и $m$ таких, что $n\le am$.
\end{proof}






\section{Приближение функций средними Валле Пуссена $V_m^n(f)$ по норме пространства $ L^{p(x)}_{2\pi}$  }\label{s7}

Из равенств \eqref{6.7} и  \eqref{6.10} мы можем записать
\begin{equation}\label{7.1}
  f(x)-V_m^n(f,x)= \frac{1}{(n+1)^r}\int_{-\pi}^\pi (f^{(r)}(t)-T_n(f^{(r)},t))\kappa_{r,m}^n(t-x)dt,
\end{equation}
где $T_n(f^{(r)})=T_n(f^{(r)},t)$ -- полином наилучшего приближения производной  $f^{(r)}(t)$ по норме пространства
 $L^{p(x)}_{2\pi}$. Обратимся теперь к лемме 6.6, в которой вместо $f(x)$ фигурирует разность $f^{(r)}(t)-T_n(f^{(r)},t)$. Тогда из \eqref{7.1} имеем
 $$
 \|f(x)-V_m^n(f,x)\|_{p(\cdot)}\le
  $$
 \begin{equation}\label{7.2}
  {c_r(a,p)\over (n+1)^r}\|f^{(r)}-T_n(f^{(r)})\|_{p(\cdot)}={c_r(a,p)\over (n+1)^r}E_n(f^{(r)})_{p(\cdot)} \quad (n\le am) .
 \end{equation}

Мы можем теперь сформулировать следующий результат.
\begin{theorem}\label{t2}
Пусть  $p=p(x)\in{\cal  P}_{2\pi}$, $r\ge1$, $f\in W^{r,p(x)}_{2\pi}$, $a>0$, $n\le am$. Тогда имеют место следующие оценки
\begin{equation}\label{7.3}
   \|f-V_m^n(f)\|_{p(\cdot)}\le {c_r(a,p)\over (n+1)^r}E_n(f^{(r)})_{p(\cdot)},
\end{equation}
\begin{equation}\label{7.4}
   \|f-V_m^n(f)\|_{p(\cdot)}\le {c_r(a,p)\over (n+1)^r}\Omega\left(f^{(r)},\frac1{n+1}\right)_{p(\cdot)}.
\end{equation}
\end{theorem}
\begin{proof}
Утверждение теоремы 2 непосредственно вытекает из оценок \eqref{7.2} и \eqref{5.7}.
\end{proof}
Положим
\begin{equation}\label{7.5}
    \tilde\kappa_{r,m}^{n}(u)= \frac{m+1}{n+1}\kappa_{r,m}^{n}(u)
\end{equation}
и рассмотрим семейство операторов
\begin{equation}\label{7.6}
    \tilde{\cal K}_n^r(f)=\tilde{\cal K}_{n,m}^r(f)(x)=
\int\limits_{-\pi}^{\pi}f(t)\tilde\kappa_{r,m}^{n}(t-x)dt\quad(n=1,2,\ldots).
\end{equation}
Если $r\ge3$, то из леммы 6.5 вытекают следующие оценки
 $$
 |\tilde\kappa_{r,m}^{n}(u)|\le c(r)(n+1),
 $$
$$
 |\tilde\kappa_{r,m}^{n}(u)|\le c(r) \quad (n^{-\frac12}\le u\le2\pi-n^{-\frac12}),
 $$
$$
\int_{-\pi}^\pi |\tilde\kappa_{r,m}^{n}(u)|du \le c(r).
$$
Отсюда следует, что семейство ядер $\tilde\kappa_{r,m}^{n}(u)$ при $r\ge3$ и $n\le m$ удовлетворяет условиям A), B) и B), при соблюдении
которых справедливо утверждение теоремы 1 (см. \S 4). Поэтому мы можем сформулировать следующее утверждение.
\begin{lemma}\label{l7.1}
 Пусть $p=p(x)\in{\cal P}_{2\pi}$, $f\in L^{p(x)}_{2\pi}$, $r\ge3$, $n\le m$. Тогда
$$\|\tilde{\cal K}_{n,m}^r(f)\|_{p(\cdot)}\le c_r(p)\|f\|_{p(\cdot)}.$$
  \end{lemma}
С учетом (7.5) перепишем   равенство \eqref{7.1} следующим образом
\begin{equation}\label{7.7}
  f(x)-V_m^n(f,x)= \frac{1}{(m+1)(n+1)^{r-1}}\int_{-\pi}^\pi (f^{(r)}(t)-T_n(f^{(r)},t))\tilde\kappa_{r,m}^n(t-x)dt.
\end{equation}

\begin{theorem}\label{t3}
Пусть  $p=p(x)\in{\cal  P}_{2\pi}$, $r\ge3$, $f\in W^{r,p(x)}_{2\pi}$, $a>0$, $ n\le am$. Тогда имеют место следующие оценки
\begin{equation}\label{7.8}
   \|f-V_m^n(f\|_{p(\cdot)}\le {c_r(p,a)\over(m+1) (n+1)^{r-1}}E_n(f^{(r)})_{p(\cdot)},
\end{equation}
\begin{equation}\label{7.9}
   \|f-V_m^n(f)\|_{p(\cdot)}\le {c_r(p,a)\over(m+1) (n+1)^{r-1}}\Omega\left(f^{(r)},\frac1{n+1}\right)_{p(\cdot)}.
\end{equation}
\end{theorem}
\begin{proof}
Если $n\le m$, то мы можем применить к правой части равенства \eqref{7.7} лемму 7.1, в которой вместо $f$ фигурирует $f^{(r)}(t)-T_n(f^{(r)},t)$ и это приводит к оценке \eqref{7.8}, а из нее вытекает оценка \eqref{7.9}. Если же $m\le n\le am$, то утверждение
теоремы 3 непосредственно вытекает из теоремы 2.
\end{proof}
Оценки \eqref{7.8} и  \eqref{7.9} доказаны в теореме 3 при условии $r\ge3$.
Рассмотрим теперь вопрос о справедливости этих оценок для $r=2$. С этой целью введем семейство операторов
\begin{equation}\label{7.10}
    \hat{\cal K}_m(f)=\hat{\cal K}_{n,m}(f)(x)=
\int\limits_{-\pi}^{\pi}f(t)\tilde\kappa_{2,m}^{n}(t-x)dt\quad(m=1,2,\ldots)
\end{equation}
в предположении $m^\delta\le n\le am$, где $0<\delta\le 1$, $a\ge1$. В этом случае из лемм 6.3 -- 6.5 вытекают следующие оценки (равномерно относительно $m^\delta\le n\le am$, $m=1,2,\ldots$)
 $$
 |\tilde\kappa_{2,m}^{n}(u)|\le c(\delta,a)(m+1),
 $$
$$
 |\tilde\kappa_{2,m}^{n}(u)|\le c(\delta,a) \quad (m^{-\frac\delta2}\le u\le2\pi-m^{-\frac\delta2}),
 $$
$$
\int_{-\pi}^\pi |\tilde\kappa_{2,m}^{n}(u)|du \le c(\delta,a).
$$
Это означает, что семейство ядер $ \tilde\kappa_{r,m}^{n}(u)$ ($m^\delta\le n\le am$, $m=1,2,\ldots$) удовлетворяет условиям A), B) и B), при соблюдении которых справедливо утверждение теоремы 1. Поэтому мы можем сформулировать следующее утверждение.
\begin{lemma}\label{l7.2}
 Пусть $p=p(x)\in{\cal P}_{2\pi}$, $f\in L^{p(x)}_{2\pi}$, $m^\delta\le n\le am$, $m=1,2,\ldots$ где $0<\delta\le 1$, $a\ge1$. Тогда
$$\|\hat{\cal K}_{m}(f)\|_{p(\cdot)}\le c(p,a,\delta)\|f\|_{p(\cdot)}.$$
  \end{lemma}


\begin{theorem}\label{t4}
Пусть  $p=p(x)\in{\cal  P}_{2\pi}$,  $f\in W^{2,p(x)}_{2\pi}$, $m^\delta\le n\le am$, $m=1,2,\ldots$, где $0<\delta\le 1$, $a\ge1$. Тогда имеют место следующие оценки
\begin{equation}\label{7.11}
   \|f-V_m^n(f)\|_{p(\cdot)}\le {c(p,a,\delta)\over(m+1) (n+1)}E_n(f'')_{p(\cdot)},
\end{equation}
\begin{equation}\label{7.12}
   \|f-V_m^n(f)\|_{p(\cdot)}\le {c(p,a,\delta)\over(m+1) (n+1)}\Omega\left(f'',\frac1{n+1}\right)_{p(\cdot)}.
\end{equation}
\end{theorem}
\begin{proof}
Если $m^\delta\le n\le am$, то мы можем применить к правой части равенства \eqref{7.7} лемму 7.2, в которой вместо $f$ фигурирует $f''(t)-T_n(f'',t)$ и это приводит к оценке \eqref{7.11}, а из нее вытекает оценка \eqref{7.12}.
\end{proof}

Перейдем к исследованию аппроксимативных свойств средних Валле Пуссена $V_m^n(f,x)$ для функций $f\in W^{1,p(x)}_{2\pi}$. С этой целью
заметим, что $V_m^n(f,x)$ допускает следующее интегральное представление
\begin{equation}\label{7.13}
    V_m^n(f,x)=\frac1\pi\int_{-\pi}^\pi v_m^n(t)f(x-t)dt,
\end{equation}
где
\begin{equation}\label{7.14}
    v_m^n(t)=\frac1{m+1}{\cos nt-\cos(n+m+1)t\over4\sin^2\frac{t}{2}}=\frac1{m+1}{\sin m\frac{t}{2}\sin(2n+m)\frac{t}{2}\over2\sin^2\frac{t}{2}}.
\end{equation}

Если $f\in W^{1,p(x)}_{2\pi}$ из \eqref{7.13} имеем
$$
f(x)-V_m^n(f,x)=
$$
\begin{equation}\label{7.15}
  \frac1\pi\int_{-\pi}^\pi v_m^n(t)(f(x)-f(x-t))dt=  \frac1\pi\int_{-\pi}^\pi v_m^n(t)\int_{x-t}^xf'(t)dt.
\end{equation}
С другой стороны, если  $T_n(x)$ -- произвольный тригонометрический полином порядка $n$, то
$V_m^n(T_n,x)=T_n(x)$ и, следовательно,
 \begin{equation}\label{7.16}
0 \equiv T_n(x)-V_m^n(T_n,x)=  \frac1\pi\int_{-\pi}^\pi v_m^n(t)\int_{x-t}^xT_n'(t)dt.
\end{equation}
Из \eqref{7.15} и \eqref{7.16} имеем
$$
f(x)-V_m^n(f,x)= \frac1\pi\int_{-\pi}^\pi v_m^n(t)\int_{x-t}^x(f'(t)-T_n'(t))dt=
$$
\begin{equation}\label{7.17}
 \frac1\pi\int_{-\pi}^\pi v_m^n(t)tS_{t,-\frac{t}{2}}(f'-T_n')(x)dt,
\end{equation}
где $S_{h,\tau}(g)(t)$ -- сдвиг функции Стеклова, определенный с помощью равенства (5.4).
Обратимся теперь к неравенству \eqref{2.4} и применим его к правой части равенства \eqref{7.17}, что в результате дает
\begin{equation}\label{7.18}
    \|f-V_m^n(f)\|_{p(\cdot)}\le \frac1\pi\int_{-\pi}^\pi |tv_m^n(t)|\|S_{t,-\frac{t}{2}}(f'-T_n')\|_{p(\cdot)} dt.
\end{equation}

Из \eqref{7.18} и \eqref{5.5} выводим следующее неравенство
\begin{equation}\label{7.19}
    \|f-V_m^n(f)\|_{p(\cdot)}\le c(p)\|f'-T_n'\|_{p(\cdot)}\frac1\pi\int_{-\pi}^\pi |tv_m^n(t)|dt,
\end{equation}
где $T_n=T_n(x)$ -- произвольный тригонометрический полином порядка $n$. С другой стороны, с учетом равенства (7.14) нетрудно увидеть, что
\begin{equation}\label{7.20}
    \frac1\pi\int_{-\pi}^\pi |tv_m^n(t)|dt\le c{\ln(m+1)\over m+1}.
\end{equation}
Из \eqref{7.19} и \eqref{7.20} получаем
\begin{equation}\label{7.21}
    \|f-V_m^n(f)\|_{p(\cdot)}\le c(p) {\hat E}_n(f'){\ln(m+1)\over m+1},
\end{equation}
где
\begin{equation}\label{7.22}
    {\hat E}_n(f')=\inf_{T_n}\|f'-T_n'\|_{p(\cdot)}
\end{equation}
- наилучшее приближение функции $f'(x)$ тригонометрическими полиномами вида $T_n'(x)$.

\begin{theorem}\label{t5}
Пусть  $p=p(x)\in{\cal  P}_{2\pi}$,  $f\in W^{1,p(x)}_{2\pi}$. Тогда имеют место следующие оценки
\begin{equation}\label{7.23}
   \|f-V_m^n(f)\|_{p(\cdot)}\le c(p){\ln(m+1)\over m+1 }{\hat E}_n(f')_{p(\cdot)},
\end{equation}
\begin{equation}\label{7.24}
   \|f-V_m^n(f)\|_{p(\cdot)} \le c(p){\ln(m+1)\over m+1 }\Omega\left(f',\frac2{n}\right)_{p(\cdot)}.
\end{equation}
\end{theorem}
\begin{proof}
Оценка \eqref{7.23} непосредственно вытекает из \eqref{7.21}, поэтому  остается доказать оценку \eqref{7.24}.
Для этого нам понадобятся некоторые свойства хорошо известных ядер  и операторов Джексона, которые определяются следующим образом. Ядро Джексона имеет вид
\begin{equation}\label{7.25}
J_k(x)=\frac{3}{2k(2k^2+1)}\left(\frac{\sin\frac{kx}{2}}{\sin\frac{x}{2}}\right)^4,
k=1,2,\ldots
\end{equation}
а оператор Джексона выражается через ядро $J_k(x)$ в виде свертки:
\begin{equation}\label{7.26}
{\cal D}_k (f)={\cal D}_k (f)(x)=\frac1\pi \int\limits_{-\pi}^\pi
f(x-t)J_k(t)dt\quad(k=1,2,\ldots)
\end{equation}
и представляет собой тригонометрический полином порядка $2k-2$.
Заметим, что из \eqref{7.25} и \eqref{7.26} имеем
$$
\int\limits_{-\pi}^\pi{\cal D}_k (f')(x)dx=
$$
\begin{equation}\label{7.27}
      \frac1\pi \int\limits_{-\pi}^\pi\int\limits_{-\pi}^\pi
f'(x-t)J_k(t)dtdx=\frac1\pi\int\limits_{-\pi}^\pi J_k(t)dt\int\limits_{-\pi}^\pi f'(x-t)dx=0.
\end{equation}
Это означает, что  полином  ${\cal D}_k (f')(x)$ может быть представлен в виде $T_{2k-2}'(x)$, где  $T_{2k-2}(x)$ -- тригонометрический полином степени $2k-2$. Поэтому мы можем записать (см. \eqref{7.22})
\begin{equation}\label{7.28}
    {\hat E}_{2k-2}(f')_{p(\cdot)}\le \|f'-{\cal D}_k (f')\|_{p(\cdot)}.
\end{equation}

С другой стороны, в \cite{Shar6} доказано, что
\begin{equation}\label{7.29}
    \|f'-{\cal D}_k (f')\|_{p(\cdot)}\le c(p)\Omega\left(f',\frac1{k}\right)_{p(\cdot)}.
\end{equation}
Из  из \eqref{7.28} и \eqref{7.29} выводим
 \begin{equation}\label{7.30}
    {\hat E}_{2k-2}(f')_{p(\cdot)}\le c(p)\Omega\left(f',\frac1{k}\right)_{p(\cdot)}.
\end{equation}
Для натурального $n$ выберем такое $k$, для которого $2k-2\le n < 2k$. Тогда ${\hat E}_{n}(f')_{p(\cdot)}\le{\hat E}_{2k-2}(f')_{p(\cdot)}$
и, следовательно, в силу \eqref{7.30}
\begin{equation}\label{7.31}
    {\hat E}_{n}(f')_{p(\cdot)}\le c(p)\Omega\left(f',\frac1{k}\right)_{p(\cdot)}\le c(p)\Omega\left(f',\frac2{n}\right)_{p(\cdot)}.
\end{equation}
Оценка \eqref{7.24} вытекает из \eqref{7.23} и \eqref{7.31}. Теорема 5 доказана.
\end{proof}

Перейдем к рассмотрению вопроса об аппроксимативных свойствах средних Валле Пуссена $V_m^n(f,x)$ в пространстве  $L^{p(x)}_{2\pi}$
в том случае, когда  $p(x)>1$. Пусть  $p=p(x)\in\hat{\cal  P}_{2\pi}$ (см. \S 4), тогда, как уже отмечалось выше, для частичных сумм
 $S_n(f)=S_n(f,x)$ ряда Фурье функции $f(x)\in L^{p(x)}_{2\pi}$  имеет место  оценка \eqref{4.8} и, как следствие, мы выводим
\begin{equation}\label{7.32}
    \|f-S_n(f)\|_{p(\cdot)}\le c(p)E_n(f)_{p(\cdot)}\quad(n=0,1,2,\ldots).
 \end{equation}
С другой стороны, в силу \eqref{1.4} имеем
\begin{equation}\label{7.33}
  \|f-V_m^n(f)\|_{p(\cdot)}\le \frac{1}{m+1}\sum_{k=n}^m\|f-S_k(f)\|_{p(\cdot)}.
 \end{equation}
Из \eqref{7.32} и \eqref{7.33} выводим следующий результат.
\begin{theorem}\label{t6}
Пусть  $p=p(x)\in\hat{\cal  P}_{2\pi}$,  $f\in L^{p(x)}_{2\pi}$. Тогда имеют место следующие оценки
$$
   \|f-V_m^n(f)\|_{p(\cdot)}\le \frac{c(p)}{m+1}\sum_{k=0}^mE_{n+k}(f)_{p(\cdot)},
$$
$$
   \|f-V_m^n(f)\|_{p(\cdot)}\le \frac{c(p)}{m+1}\sum_{k=0}^m\Omega\left(f,\frac2{n+k+1}\right)_{p(\cdot)}.
$$
\end{theorem}
\section{Приближение функций $f(x)\in W^{r,p(x)}_{2\pi}$ средними Валле Пуссена $V_m^n(f,x)$ в заданной точке $x$  }\label{s8}

Рассмотрим задачу об приближении  функции $f(x)\in  W^{r,p(x)}_{2\pi}$ средними Валле Пуссена $V_m^n(f,x)$ в заданной точке $x$. С этой целью обратимся к неравенству \eqref{6.9} и оценим величину $\|R_{r,l}(*-x)\|_{q(\cdot)}^*$ для $q(x)\in \hat{\cal  P}_{2\pi}$. Заметим, что если   $p=p(x)\in\hat{\cal  P}_{2\pi}$, то сопряженная функция $q(x)={p(x)\over p(x)-1}$ также принадлежит классу $\hat{\cal  P}_{2\pi}$.
  Поскольку нормы $\|R_{r,l}(*-x)\|_{q(\cdot)}^*$ и $\|R_{r,l}(*-x)\|_{q(\cdot)}$ эквивалентны, то нам достаточно оценить $\|R_{r,l}(*-x)\|_{q(\cdot)}$.
 Рассмотрим сначала случай четного $r\ge2$. Из \eqref{6.4}, путем преобразования Абеля мы можем записать
$$
 R_{r,l}(u)  =(-1)^{\frac{r}{2}}\sum\limits_{k=l+1}^{\infty}\frac{\cos ku}{ k^r}=
$$
\begin{equation} \label{8.1}
    =-\frac{(-1)^\frac{r}{2}}{(l+1)^r}D_l(u)+\sum\limits_{k=l+1}^\infty[\frac{1}{ k^r}-\frac{1}{(k+1)^r}]D_k(u),
\end{equation}
где
$$
D_{l}(u)=\frac12+\sum\limits_{s=1}^{l}\cos su=\frac{\sin(l+\frac12)u}{2\sin\frac{u}2}.
$$

\begin{lemma}\label{8.1}
 Пусть  $p(x)\in \hat {\cal P}_{2\pi}$,
$\frac{1}{p(x)}+\frac{1}{q(x)}=1$. Тогда имеет место оценка
$$
    \|D_{l}(x-*)\|_{q(\cdot)}\le c(p)l^{\frac{1}{p(x)}}\quad(l=1,2,\ldots).
$$
\end{lemma}
\begin{proof}
 Рассмотрим интеграл
\begin{equation}\label{8.2}
    \int\limits_{-\pi}^{\pi}\left|\frac{D_{l}(x-t)}{n^{\frac{1}{p(x)}}}\right|^{q(t)}dt=
    \int\limits_{-\pi}^{\pi}\left|\frac{D_{l}(t)}{l^{\frac{1}{p(x)}}}\right|^{q(x-t)}dt=S^{L}+S^{R},
\end{equation}
где
$$
    S^{L}=\int\limits_{-\pi}^{0}\left|\frac{D_{l}(t)}{l^{\frac{1}{p(x)}}}\right|^{q(x-t)}dt,
    \quad S^{R}=\int\limits_{0}^{\pi}\left|\frac{D_{l}(t)}{l^{\frac{1}{p(x)}}}\right|^{q(x-t)}dt.
$$
Имеем
$$
    S^{R}<
    \int\limits_{0}^{\pi}\left(1+\left|\frac{D_{l}(t)}{l^{\frac{1}{p(x)}}}\right|\right)^{q(x-t)}dt\leq
$$
\begin{equation}\label{8.3}
    \int\limits_{0}^{\pi}\left(1+\left|\frac{D_{l}(t)}{l^{\frac{1}{p(x)}}}\right|\right)^{q(x)}
    \left(1+\left|\frac{D_{l}(t)}{l^{\frac{1}{p(x)}}}\right|\right)^{|q(x-t)-q(x)|}dt.
\end{equation}
Поскольку $q(t)$ входит в класс $\hat{\cal P}_{2\pi}$
одновременно с $p(t)$, то $(0<t<\pi)$
$$
    |q(x-t)-q(x)|\leq \frac{c(p)}{\ln\frac\pi t},
$$
поэтому при $0<t<\pi$
\begin{equation}\label{8.4}
    \left(1+\left|\frac{D_{l}(t)}{l^{\frac{1}{p(x)}}}\right|\right)^{|q(x-t)-q(x)|}<
    \left(1+\left|\frac{D_{l}(t)}{l^{\frac{1}{p(x)}}}\right|\right)^{\frac{c(p)}{\ln\frac1t}}
    \leq c(p).
\end{equation}
Из \eqref{8.3} и \eqref{8.4} находим
\begin{equation}\label{8.5}
    S^{R}\leq c(p)\int\limits_{0}^{\pi}\left(1+\left|\frac{D_{l}(t)}{l^{\frac{1}{p(x)}}}
    \right|\right)^{q(x)}dt.
\end{equation}
С другой стороны,
$$
    \int\limits_{0}^{\pi}\left|\frac{D_{l}(t)}{l^{\frac{1}{p(x)}}}\right|^{q(x)}dt=
    l^{-\frac{q(x)}{p(x)}}\int\limits_{0}^{\pi}|D_{l}(t)|^{q(x)}dt
    \leq
$$
\begin{equation}\label{8.6}
    \leq c(p)l^{-\frac{q(x)}{p(x)}}\int\limits_{0}^{\pi}\left|\frac{\sin(l+\frac12) t}{ t}
    \right|^{q(x)}dt.
\end{equation}
Обозначим через $t_{k}$ нули функции $\sin(l+\frac12) t$ из
$(0,\pi)$, т. е. $t^{l}_{k}=\frac{\pi k}{l+\frac12},\quad k=1,2,\ldots,l$.
Тогда
\begin{equation}\label{8.7}
    \frac{\sin(l+\frac12) t}{ t}\leq\frac{l+\frac12}{k}\quad(t_{k}^{l}
    \leq t\leq t^{l}_{k+1},k=1,2,\ldots,l-1),
\end{equation}
\begin{equation}\label{8.8}
    \frac{\sin(l+\frac12) t}{ t}\leq l+\frac12\quad(t\in(0,t^{l}_{1})).
\end{equation}

Из \eqref{8.5} -- \eqref{8.8} имеем
$(q(x)-\frac{q(x)}{p(x)}=q(x)(1-\frac{1}{p(x)})=1)$
$$
    S^{R}\leq c(p)l^{-\frac{q(x)}{p(x)}}
    \sum\limits_{k=1}^{l-1}\int\limits_{t_{k}}^{t_{k+1}}\left(\frac{l+\frac12}{k}\right)^{q(x)}dt
    \leq c(p)l^{q(x)-\frac{q(x)}{p(x)}}\sum\limits_{k=1}^{l-1}k^{-q(x)}\int\limits_{t_{k}}^{t_{k+1}}dt\leq
$$
\begin{equation}\label{8.9}
    \leq c(p)\sum\limits_{k=1}^{l-1}\frac{l}{k^{q(x)}}\int\limits_{t_{k}}^{t_{k+1}}dt\leq l(p)
    \sum\limits_{k=1}^{l-1}k^{-q(x)}\leq c(p).
\end{equation}
Аналогично
\begin{equation}\label{8.10}
    S^{L}\leq c(p).
\end{equation}
Сопоставляя \eqref{8.9}  и \eqref{8.10} с \eqref{8.2}, убеждаемся в
справедливости утверждения леммы 8.1 для четного $r$. Доказательство утверждения леммы 8.1, относящееся к случаю  нечетного $r$, почти ничем не отличается.
\end{proof}
Из леммы 8.1 и эквивалентности норм $\|D_{l}(x-*)\|_{q(\cdot)}$ и $\|D_{l}(x-*)\|^*_{q(\cdot)}$ непосредственно вытекает
\begin{lemma}\label{8.2}
 Пусть  $p(x)\in \hat {\cal P}_{2\pi}$,
$\frac{1}{p(x)}+\frac{1}{q(x)}=1$. Тогда имеет место оценка
$$
    \|D_{l}(x-*)\|^*_{q(\cdot)}\le c(p)l^{\frac{1}{p(x)}}\quad(l=1,2,\ldots).
$$
\end{lemma}
Из леммы 8.2 и равенства \eqref{8.1}, в свою очередь, вытекает
\begin{lemma}\label{8.3}
 Пусть  $p(x)\in \hat {\cal P}_{2\pi}$,
$\frac{1}{p(x)}+\frac{1}{q(x)}=1$, $r\ge1$. Тогда имеет место оценка
$$
    \|R_{r,l}(x-*)\|^*_{q(\cdot)}\le c_r(p)l^{-r+{\frac{1}{p(x)}}}\quad(l=1,2,\ldots).
$$
\end{lemma}
Из леммы 8.3 и неравенства \eqref{6.9} выводим следующее утверждение.

\begin{theorem}\label{t7}
Пусть  $p=p(x)\in \hat{\cal  P}_{2\pi}$, $r\ge1$, $f\in W^{r,p(x)}_{2\pi}$. Тогда имеет место следующая оценка
$$
     |f(x)-V_m^n(f,x)|\le \frac{c_r(p)}{m+1}\sum_{k=n}^{n+m}{E_k(f^{(r)})_{p(\cdot)}\over (k+1)^{r-{\frac{1}{p(x)}}}}.
$$
 \end{theorem}

Из теоремы 7 непосредственно вытекает


\begin{corollary}\label{3}
Пусть  $p=p(x)\in \hat{\cal  P}_{2\pi}$, $r\ge1$, $f\in W^{r,p(x)}_{2\pi}$. Тогда имеют место следующие оценки:
$$
|f(x)-S_n(f,x)|= |f(x)-V_0^n(f,x)|\le c_r(p){E_n(f^{(r)})_{p(\cdot)}\over (n+1)^{r-{\frac{1}{p(x)}}}},
$$
$$
 |f(x)-F_m(f,x)| =   |f(x)-V_m^0(f,x)|\le \frac{c_r(p)}{m+1}\sum_{k=0}^{m}{E_k(f^{(r)})_{p(\cdot)}\over (k+1)^{r-{\frac{1}{p(x)}}}} .
$$
 \end{corollary}
 Из следствия 3 и оценки \eqref{5.9} вытекает

\begin{corollary}\label{4}
Пусть  $p=p(x)\in \hat{\cal  P}_{2\pi}$, $r\ge1$, $f\in W^{r,p(x)}_{2\pi}$. Тогда имеют место следующие оценки:
$$
|f(x)-S_n(f,x)|= |f(x)-V_0^n(f,x)|\le c_r(p){\Omega\left(f^{(r)},\frac1{n+1}\right)_{p(\cdot)}\over (n+1)^{r-{\frac{1}{p(x)}}}},
$$
$$
 |f(x)-F_m(f,x)| =   |f(x)-V_m^0(f,x)|\le \frac{c_r(p)}{m+1}\sum_{k=0}^{m}{\Omega\left(f^{(r)},\frac1{k+1}\right)_{p(\cdot)}\over (k+1)^{r-{\frac{1}{p(x)}}}} .
$$
 \end{corollary}










