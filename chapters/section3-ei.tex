\chapter{Существование и единственность положительного решения задачи Дирихле в шаре для нелинейного эллиптического уравнения с $p-$ лапласианом}

\section{Предварительные сведения}
Рассмотрим в шаре
$ S=\left\{x\in R^n:\vert x \vert\ < 1 \right\} $
 с границей $\Gamma $ задачу Дирихле

\begin{equation}\label{EI1}
\Delta_{p} u+a(|x|){\vert u \vert}^q=0,\ x \in S,
\end{equation}

\begin{equation}\label{EI2}
u_\Gamma=0,      
\end{equation}
где $ \Delta_{p} u=div(\vert \nabla u\vert^{p-2}\nabla u),
 1<p \le2, q > 1$ -- константы, $a(t)-$ непрерывная неотрицательная
 при $t\geq 0 $ функция.

Под положительным решением
задачи \eqref{EI1},\eqref{EI2} понимается ее решение из класса $ C^2(\overline S) $
 положительное в $S$ и удовлетворяющее граничному условию \eqref{EI2}.

В настоящее время имеется много работ российских и зарубежных
математиков, посвященных положительным решениям задач вида \eqref{EI1},\eqref{EI2}
(см., например \cite{LitEI1, LitEI2, LitEI3, LitEI4, LitEI5, LitEI6, LitEI7, LitEI8, LitEI9}). Существование и единственность
положительного радиально-симметричного (п.р-с.)
 решения задачи \eqref{EI1},\eqref{EI2} при $p=n=2$ доказано в работе
 \cite{LitEI7} в случае $a(|x|)=|x|^m$ и $q>1,$ где $m \geq 0.$
 В этом же случае для $ n\geq 3, p=2 $  в работе  \cite{LitEI8} cуществование и
единственность п.р-с. решения задачи \eqref{EI1},\eqref{EI2} доказано
 при условии  $ q \leq \frac {m+n}{n-2} $. А в работе \cite{LitEI9} для  
 $n\geq 2, 1<p\leq 2$ -- при условии  $ 1<q\leq \frac{(p-1)(m+n)}{n-p}$.
 
\section{Достаточные условия существования}
Предположим, что $1<p\leq 2 $ и функция $a(r)$ непрерывна при $r
\in[0,1]$ и удовлетворяет условию
\begin{equation}\label{EI3}
 c_0r^m \leq a(r)\leq c_1,
\end{equation}
 где $c_0, c_1- $ положительные постоянные. Доказана

\begin{theorem}\label{EItheo1}
Если $ 1<p \leq 2 $ при $n\geq 3$ и $ 1<p<2 $
при $n=2,$ функция $a(r)$ удовлетворяет условию (3) и число $q$
удовлетворяет условию
\begin{equation}\label{EI4}
1< q<\frac{(p-1)(m+n)}{n-p},
\end{equation}
то задача Дирихле \eqref{EI1},\eqref{EI2} имеет единственное положительное
радиально-симметричное решение.
\end{theorem}
Доказательство этой теоремы опирается на следующие две леммы.

\begin{lemma}\label{EIlemma1}
Если $ 1<p\leq 2 $ при $ n\geq 3 $ и $ 1<p<2  $
при $ n=2 $, выполняются условия (3) и (4),  то при любом $ A>0$
существует единственное положительное число $ r_0=r_0(A) $ такое,
что существует единственное решение $ v( r) $ задачи Коши
\begin{equation}\label{EI5}
(p-1)v^{\prime\prime}+\frac{n-1}{r}v^{\prime}= -A^{q+1-p}{a(r)\vert
v \vert}^q{\vert  v^{\prime} \vert}^{p-2},  
\end{equation}
\begin{equation}\label{EI6}
v(0)=1, v'(0)=0.
\end{equation}
 такое,
что $ v( r_0)=0, v( r)>0 $ при $
 r \in [0, r_0) $ и $ v \in C^2[0, r_0].$
\end{lemma}
С помощью замены
$$
 t=r^{\frac{p-n}{p-1}}
$$
задача \eqref{EI5}, \eqref{EI6} приводится к виду
\begin{equation}\label{EI7}
w^{\prime\prime}(t)=-\frac{(p-1)^{p-1}}{(n-p)^p}A^{q+1-p}
t^{-\frac{p(n-1)}{n-p}} a(t^{\frac{p-1}{p-n}})|w|^q
|w^{\prime}|^{2-p},
\end{equation}

\begin{equation}\label{EI8}
\lim_{t \to \infty}w(t)=1,
 \lim_{t \to \infty}w^{\prime}(t)=0,
\end{equation}
где $ w(t)=v(r^{\frac{p-n}{p-1}}).$

При доказательстве леммы \ref{EIlemma1} показано, что каждому значению $A>0$
соответствует  единственное значение $t_0>0 $ такое, что $w(t_0)=0,$
где $w(t)-$
решение задачи Коши \eqref{EI7}, \eqref{EI8}, т.е. определена функция $t_0=t_0(A).$

\begin{lemma}\label{EIlemma2}
 Предположим, что выполняется  неравенство \eqref{EI4} и $p$ удовлетворяет условию
  $$
     1 < p \leq 2.
  $$
  Тогда функция $t_0=t_0(A)$ непрерывна и возрастает при $A >0$ и
  $$
 \lim_{A\rightarrow \infty}t_0(A)=\infty, \quad \lim_{A \rightarrow
 0}t_0(A)=0.
  $$
\end{lemma}

\chapter{Двухточечная краевая задача нелинейного дифференциального уравнения  с дробными производными, имеющего экспоненциальный рост по решению}

\section{Предварительные сведения}
В последние десятилетия вышло много публикаций, посвященных
дифференциальным уравнениям с дробными производными. Часть данных
публикаций посвящена краевым задачам для таких уравнений (см.,
например, \cite{LitEI10, LitEI11, LitEI12, LitEI13, LitEI14, LitEI15, LitEI16, LitEI17, LitEI18, LitEI19}). В \cite{LitEI15} доказано существование и
единственность положительного решения краевой задачи
\begin{equation}\label{EI9}
D_{0+}^{\alpha} u(t)+f(t,u(t) )=0,0<t<1,
\end{equation}
\begin{equation}\label{EI10}
 u(0)=u(1)=0
\end{equation}
в случае $f(t,u)$ имеет степенной рост по $u$, а также предложен
численный метод его построения. В \cite{LitEI9} $5/4 \leq \alpha \leq 2-$
порядок дробного дифференцирования. Производная понимается в смысле
Римана-Лиувилля.

Следуя \cite{LitEI10}, введем обозначения
$$
M(\alpha)=\left ( \int_0^1 G(\alpha,s,s)ds \right )^{-1},
N(\alpha)=\left
(\int_{1/4}^{3/4}\gamma(\alpha,s)G(\alpha,s,s)ds)\right )^{-1},
$$
$$
\gamma(\alpha,s)=\begin{cases} \frac{\left [\frac{3}{4}(1-s)\right
]^{\alpha-1}-\left (\frac{3}{4}-s \right )^{\alpha-1}}
{[s(1-s)]^{\alpha-1}}, s \in (0,r]  \\
\frac{1}{(4s)^{\alpha-1}},s \in [r,1),
\end{cases}
$$
где $1/4<r<3/4 - $ единственное решение уравнения
$$
\left [\frac{3}{4}(1-s)\right ]^{\alpha-1}-\left (\frac{3}{4}-s
\right )^{\alpha-1}=\frac{(1-s)^{\alpha-1}}{4^{\alpha-1}}.
$$
Здесь $G(\alpha,t,s)-$ функция Грина, которая имеет вид \cite{LitEI10}
$$
G(\alpha,t,s)=\begin{cases}
\frac{[t(1-s)]^{\alpha-1}-(t-s)^{\alpha-1}}{\Gamma(\alpha)}, 0\leq s
\leq t \leq 1, \\
\frac{[t(1-s)]^{\alpha-1}} {\Gamma(\alpha)}, 0\leq t \leq s \leq 1.
\end{cases}
$$

Справедлива
\begin{theoremA}
Пусть $f(t,u)$ непрерывна на
$[0,1]\times[0,\infty)$. Предполагаем, что существуют две
положительные константы $r_2>r_1>0$ такие, что
$$
f(t,u)\leq M(\alpha)r_2 \text{для всех} (t,u) \in [0,1]\times
[0,r_2],
$$
$$
f(t,u)\geq N(\alpha)r_2 \text{для всех} (t,u) \in [0,1]\times
[0,r_1],
$$
Тогда задача \eqref{EI9}, \eqref{EI10} имеет не менее одного положительного решения
и такого, что $r_1\leq \|u \| \leq r_2$.
\end{theoremA}

\section{Существование положительного решения}

Предположим, что функция $f(t,u)$ непрерывна при $ t \in [0,1]$   и
$u \in R$ и удовлетворяет условию
\begin{equation}\label{EI11}
 b_0\leq f(t,u)\leq A\exp(\lambda u)+b(t), b_0\leq b(t)\leq B,
\end{equation}
где $B,b_0, A,\lambda $- положительные константы.

Доказана

\begin{theorem}\label{EItheo2}
Если функция $f(t,u)$ удовлетворяет условию \eqref{EI11}, в котором $A,B$ и $\lambda $
удовлетворяют неравенствам
\begin{equation}\label{EI12}
 1.46>A\lambda\exp(1),
\end{equation}
\begin{equation}\label{EI13}
 \exp\left(\frac{\lambda B}{1.46}\right )\leq
 \frac{1.46}{A\lambda\exp(1)}
\end{equation}
 то задача \eqref{EI9}, \eqref{EI10}  имеет по крайней мере одно положительное
 решение.
\end{theorem}
 
\section{Единственность и численный метод построения положительного решения краевой задачи }
Применяя принцип неподвижной точки, доказана теорема о
единственности положительного решения и разработан численный метод
построения этого решения.

Справедлива

\begin{theorem}\label{EItheo3}
Предположим, что выполняются условия \eqref{EI11}--\eqref{EI13}, $\lambda \leq 0.73 $ и существует  положительное число
$P_0$ такое, что
$$
\frac{1}{2}\leq P_0 \leq \frac{M(\alpha)}{2A\lambda },
$$
функция $ f(t,u)$  дифференцируема по $ u \in R $ и удовлетворяет
условию
$$
0\leq f'_u(t,u)\leq P_0 A\lambda \exp (\lambda u)
$$
Тогда граничная задача \eqref{EI9}, \eqref{EI10} имеет единственное положительное
решение и к нему сходится итерационный процесс
$$
u_{k+1}(t)=\int_0^1G(\alpha,t,s)f(s, u_k(s))ds, r=0,1,\\cdots
$$
с оценкой погрешности
$$
\|u-u_k\|\leq 0.5^{k-1}\|u_1-u_0\|.
$$
\end{theorem}