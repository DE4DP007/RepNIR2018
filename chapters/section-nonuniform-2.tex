
\subsection{Аппроксимативные свойства сумм Фурье по полиномам $P_{n,N}(t)$}

Далее нами была исследована задача об аппроксимативных свойствах  алгебраических полиномов $S_{n,N}(t)$, реализующих весовой метод наименьших квадратов на некоторых неравномерных сетках  $\Omega_N = \{ t_j \}_{j=0}^{N-1} \in [-1,1]$, в которых узлы $t_j$ подчинены условиям $\eta_{j}\leq t_{j} \leq \eta_{j+1}$ $(0\leq j \leq N-1)$, а точки $\eta_{j}$, в свою очередь, выбраны так, что $-1=\eta_{0}<\eta_{1}<\eta_{2}<\cdots<\eta_{N-1}<\eta_{N}=1$. Рассматривается случай, когда для  непрерывной на $[-1,1]$  функции $f(t)$ заданы измерения $y_j = f(t_j) + \xi_{j} \quad  (0 \leq j \leq N-1)$, в которых независимые случайные погрешности $\xi_{j}$ удовлетворяют условиям $E[\xi_j] = 0$, $E[\xi_i \xi_j] = \sigma_j^2 \delta_{ij}$, $(0 \leq j \leq N-1)$, где $E[X]$ -- математическое ожидание случайной величины $X$.




Пусть заданы измерения
$y_j = f(t_j) + \xi_{j} \quad  (0 \leq j \leq N-1)$
функции $f = f(t)$, непрерывной на $[-1, 1]$, где $\xi_j$ -- ошибки наблюдений, представляющие собой независимые случайные величины, удовлетворяющие следующим условиям:
\begin{equation}
\label{SMS2.link1}
 E[\xi_j] = 0, \quad E[\xi_i \xi_j] = \sigma_j^2 \delta_{ij}, \quad (0 \leq j \leq N-1),
\end{equation}
где $E[X]$ -- математическое ожидание случайной величины $X$.
Требуется приближенно восстановить $f(t)$ в точке $t \in [-1, 1]$ по дискретной информации $\{ y_j \}_{j=0}^{N-1}$.
Для решения этой задачи, введем алгебраический полином $S_{n,N}(t)$, минимизирующий сумму
\begin{equation*}
 J(a_0, \ldots, a_n) = \sum\nolimits_{j=0}^{N-1} \left( y_j - q_n(t_j)\right)^2 \rho_j
\end{equation*}
среди всех полиномов $q_n(t)=a_0+a_1 t + \ldots + a_n t^n$ степени $n \leq N-1$, где $\rho_j$ -- положительные весовые множители. Возникает вопрос о том, насколько точно $S_{n,N}(t)$ приближает исходную функцию $f(t)$ при $t \in [-1,1]$, т.е. требуется исследовать величину $\left( f(t) - S_{n,N}(t)\right)^2$. Поскольку эта величина зависит от случайных ошибок $\xi_0, \ldots, \xi_{N-1}$, то более точная постановка задачи состоит в том, чтобы оценить среднее значение указанной величины, равное
\begin{equation}
\label{SMS2.neww1}
 J_{n, N} (f, t) = E \left[ \left( f(t) - S_{n,N}(t)\right)^2 \right].
\end{equation}
Эта задача в случае, когда $t_j=-1+\frac{2j}{N-1}$, $\rho_j=1$ была исследована в \cite{nushii5}.
Нами же был рассмотрен более общий случай, когда узлы $t_j$ могут образовывать неравномерную сетку $\Omega_N = \{ t_j \}_{j=0}^{N-1} \in [-1,1]$, а веса $\rho_j$ подчинены определенным естественным условиям, описанным выше.

%Более точно, пусть на отрезке $[-1, 1]$ задана система точек $ \eta_j (0 \leq j \leq N)$, таких что
%\begin{equation}
%\label{SMS2.dots1}
%-1=\eta_{0}<\eta_{1}<\eta_{2}<\dots<\eta_{N-1}<\eta_{N}=1.
%\end{equation}
%
%Положим
%$\vartriangle\eta_{j} = \eta_{j+1}-\eta_{j}\text{ } (0 \leq j \leq N-1)$, $\lambda_N = \max_{0\leq j \leq N-1} \vartriangle\eta_j$.
%Пусть, кроме того, на каждом частичном отрезке $[\eta_{j}, \eta_{j+1}]$ выбрана точка $t_{j} (\eta_{j}\leq t_{j} \leq \eta_{j+1})$. Тогда мы можем составить сетку $\Omega_N = \{ t_j \}_{j=0}^{N-1}$, в которой будем считать, что узлы $t_j$ попарно различны $(t_i \neq t_j$, при $i \neq j)$.
%Величины $\sigma_j$, фигурирующие в \eqref{SMS2.link1}, определим с помощью равенства
%$\sigma_j^2 = \sigma^2 \frac{\lambda_N}{\vartriangle\eta_{j}}$, где $\sigma$ -- фиксированное число. Кроме того, будем считать $\rho_j = \left(\sigma / \sigma_j\right)^2 = \frac{\vartriangle\eta_{j}}{\lambda_N}$.


%Скалярное произведение в случае единичного веса упрощается и принимает следующий вид
%\begin{equation}
%\label{SMS2.scal}
%<f,g> = \sum_{j=0}^{N-1} f(t_j)g(t_j) \vartriangle\eta_{j} = \lambda_N \sum_{j=0}^{N-1} f(t_j)g(t_j) \rho_{j}.
%\end{equation}

Для полиномов, ортогональных с единичным весом, введем обозначение $$\hat{P}_{n,N}(t) = \hat{P}^{0,0}_{n,N}(t) \quad (0 \leq n \leq N-1)$$  и будем называть их \textit{дискретными ортонормированными полиномами Лежандра}.
Хорошо известно, что полиномы $S_{n,N}(t)$ минимизирующие величину \eqref{SMS2.neww1}, могут быть представлены следующим образом
\begin{equation*}
S_{n,N}(t) = \sum\limits_{k=0}^{n} \hat{y}_k \hat{P}_{k,N}(t), \quad \text{где} \quad \hat{y}_k = \sum\limits_{j=0}^{N-1} y_j \hat{P}_{k,N}(t_j)\vartriangle\eta_{j}.
\end{equation*}
Обозначим через $\Lambda_{n,N}(f,t)$ частичную сумму Фурье порядка $n$ функции $f=f(t)$ по системе $\left\{ \hat{P}_{k,N} \right\}_{k=0}^{N-1}$, т.е.
\begin{equation*}
\Lambda_{n,N}(f,t) = \sum\limits_{k=0}^{n} \hat{f}_k \hat{P}_{k,N}(t), \quad \text{где} \quad \hat{f}_k = \sum\limits_{j=0}^{N-1} f(t_j) \hat{P}_{k,N}(t_j)\vartriangle\eta_{j}.
\end{equation*}
Отсюда и из \eqref{SMS2.link1} следует, что $\Lambda_{n,N}(f,t) = E \left[ S_{n,N}(t) \right]$. Кроме того, можно показать, что %(см. [1])

\begin{equation*}
J_{n, N} (f, t) = \left( f(t) - \Lambda_{n,N}(f,t)\right)^2 + \sigma^2\lambda_N\sum\limits_{k=0}^{n} \left(\hat{P}_{k,N}(t)\right)^2
\end{equation*}
\begin{equation}
\label{SMS2.problem}
= R^2_{n,N}(f,t) + D_{n,N}(t).
\end{equation}
Таким образом, задача об оценке величины $J_{n,N}(f,t)$ свелась к вопросу об оценке двух величин $\left| R_{n,N}(f,t)\right|$ и $D_{n,N}(t)$.
Для осуществления этих оценок, нам и понадобилось изучить асимптотические свойства дискретных полиномов Лежандра $\hat{P}_{k,N}(t)$.

\noindent Введем обозначение
\begin{equation*}
B_0 = \left( \frac{3-4\lambda_N \chi n^2}{1-16 \lambda_N^2 \chi^2 n^4} \right)^{\frac{1}{2}}.
\end{equation*}
Тогда асимптотическая формула из теоремы \ref{sms2:th1} для дискретных полиномов Лежандра принимает следующий вид

\begin{theorem}
\label{sms2:thA}
Пусть $4\lambda_N \chi n^2 < 1$. Тогда имеет место равенство:
\begin{equation*}
%\label{theor1eq}
\hat{P}_{n,N}(t) = \hat{P}_{n}(t) + \upsilon_{n,N}(t),
\end{equation*}
где для остаточного члена $\upsilon_{n,N}$ имеет место оценка $($здесь $t=\cos\theta)$
\begin{equation*}
\left|\upsilon_{n,N}(\cos{\theta})\right| \leq
c(a) B_0
\left\{
\begin{aligned}
n^{2}\sqrt{\lambda_N},\quad 0\leq\theta\leq a n^{-1}\\
\theta^{-\frac{1}{2}}n^{\frac{3}{2}}\sqrt{\lambda_N},\quad a n^{-1}\leq\theta\leq \frac{\pi}{2}\\
\end{aligned}
\right. .
\end{equation*}
\textit{(Для краткости все рассуждения проводятся на отрезке $\theta \in \left[ 0,\frac{\pi}{2}\right]$, для отрезка $\theta \in \left[ \frac{\pi}{2}, \pi\right]$ они получаются абсолютно аналогично.)}
\end{theorem}


\noindent Соответственно, весовая оценка из теоремы \ref{sms2:th2} может быть переписана как 

\begin{theorem}
\label{sms2:thB}
Положим $4\lambda_N \chi n^2 < 1$, тогда существует постоянная $c(a)$, такая что:
\begin{equation}
\label{sms2.weightestim}
\left|\hat{P}_{n,N}(\cos{\theta})\right| \leq
c(a) \left(1 + B_0\sqrt{n^{3}\lambda_N}\right)
\left\{
\begin{aligned}
n^{\frac12},& \quad 0\leq \theta \leq an^{-1},\\
\theta^{-\frac12},& \quad an^{-1}\leq \theta \leq \frac{\pi}{2}.\\
\end{aligned}
\right.
\end{equation}
\end{theorem}

\noindent Следует отметить также, что исследования величины $|R_{n,N}(f,t)|$ для полиномов Лежандра в случае $t_j = \eta_{j}$ были проведены в работе \cite{nunurik3}, однако полученный там результат справедлив лишь при $n = O(\lambda_N^{1/5})$, тогда как нам удалось получить оценки при $n = O(\lambda_N^{1/3})$.


%В случае, когда сетка $\Omega_N$ состоит из равноотстоящих узлов $t_j = -1 + \frac{2j}{N-1}$, асимптотические свойства и весовые оценки полиномов, ортогональных на $\Omega_N$, впервые были исследованы в работах Шарапудинова И.И. (см. [2]). В дальнейшем в работах Шарапудинова И.И. [3-5] и Нурмагомедова А.А. [6,7] изучались асимптотические свойства полиномов, ортогональных на неравномерных сетках числовой оси. В частности, в работе [7] рассмотрены асимптотические свойства дискретных полиномов Якоби $\hat{P}_{n,N}^{\alpha,\beta}(t)$ с целыми $\alpha,\beta$, ортогональных на неравномерных дискретных сетках $\Omega_N$ c
%$t_{j} = \frac{\eta_{j}+\eta_{j+1}}{2} \quad (0\leq j \leq N-1)$.
%В работе автора [8] исследован более общий случай, когда $\eta_{j}\leq t_{j} \leq \eta_{j+1} \quad (0\leq j \leq N-1)$. 
%При $n = O(\lambda_N^{-\frac{1}{3}})$ и $n,N \rightarrow \infty$ в [8] получена асимптотическая формула
%\begin{equation*}
%\hat{P}_{n,N}^{\alpha,\beta}(t) = \hat{P}_{n}^{\alpha,\beta}(t) + \upsilon_{n,N}^{\alpha,\beta}(t),
%\end{equation*}
%где $\hat{P}_{n}^{\alpha,\beta}(t)$ -- нормированный полином Якоби, $\hat{P}_{n,N}^{\alpha,\beta}(t)$ -- нормированный дискретный аналог полинома Якоби, $\upsilon_{n,N}^{\alpha,\beta}(t)$ -- остаточный член, для которого установлена следующая оценка:
%\begin{equation*}
%\left|\upsilon_{n,N}^{\alpha,\beta}(\cos{\theta})\right| \leq
%c(\alpha,\beta,c) \left( \frac{3-\lambda_N \chi (2n+\alpha+\beta)^2}{1-\lambda_N^2 \chi^2 (2n+\alpha+\beta)^4} \right)^{\frac{1}{2}}
%\left\{
%\begin{aligned}
%\theta^{-\alpha-\frac{1}{2}}n^{\frac{3}{2}}\sqrt{\lambda_N},\quad cn^{-1}\leq\theta\leq \frac{\pi}{2}.\\
%n^{\alpha+2}\sqrt{\lambda_N},\quad 0\leq\theta\leq cn^{-1},\\
%\end{aligned}
%\right.
%\end{equation*}
%Здесь и далее $c, c(\alpha), c(\alpha,\beta), c(\alpha,\beta,\ldots,\gamma)$ -- положительные константы, зависящие лишь от указанных параметров. Кроме того, в [8] получена весовая оценка
%\begin{equation*}
%\left|\hat{P}_{n,N}^{\alpha,\beta}(\cos{\theta})\right| \leq
%c(\alpha,\beta,a) \left(1 + B\sqrt{n^{3}\lambda_N}\right)
%\left\{
%\begin{aligned}
%n^{\alpha+\frac12},& \quad 0\leq \theta \leq an^{-1},\\
%\theta^{-\alpha-\frac12},& \quad an^{-1}\leq \theta \leq \frac{\pi}{2}.\\
%\end{aligned}
%\right.
%\end{equation*}

%В настоящей статье указанные асимптотические свойства и весовые оценки полиномов $\hat{P}_{n,N}(t)$ непосредственно используются при исследовании поведения указанных выше величин $|R_{n,N}(f,t)|$ и $D_{n,N}(t)$.
%Следует отметить также, что исследования величины $|R_{n,N}(f,t)|$ для полиномов Лежандра в случае $t_j = \eta_{j}$ были проведены в работе [9], однако полученный там результат справедлив лишь при $n = O(\lambda_N^{1/5})$, тогда как нам удалось получить оценки при $n = O(\lambda_N^{1/3})$.







\subsection{Аппроксимативные свойства сумм Фурье по полиномам $\hat{P}_{n,N}^{\alpha,\beta}(t)$}


Как было показано выше, исходная задача об оценке отклонения частичных сумм по дискретным полиномам Лежандра $\hat P_{n,N}(t)$ от искомой функции $f(t)$ сводится к оцениванию двух величин из равенства \eqref{SMS2.problem}.

Используя весовую оценку \eqref{sms2.weightestim}, полученную в теореме \ref{sms2:thB}, рассмотрим величину $D_{n,N}(t)$:
\begin{equation*}
D_{n,N}(t) = \sigma^2\lambda_N\sum\limits_{k=0}^{n} \left(\hat{P}_{k,N}(t)\right)^2 \leq \sigma^2\lambda_N \sum\limits_{k=0}^{n} \left(c(a) \left(1 + B\sqrt{k^{3}\lambda_N}\right) k^{\frac12}\right)^2 \leq C(a) (n^2 \lambda_N) \sigma^2.
\end{equation*}
Можно показать, что эта оценка неулучшаема по порядку.



Перейдем к оценке величины $\left| R_{n,N}(f,t)\right|$. Хорошо известно, что погрешность приближения частичной суммой $\Lambda_{n,N}(f,t)$ может быть выражена следующим образом
\begin{equation*}
\left| R_{n,N}(f,t)\right| = \left| f(t) - \Lambda_{n,N}(f,t) \right| \leq E^*_{n,\Omega_N}(f,t) \left( 1 + |L_{n,N}(t)|\right),
\end{equation*}
где $E^*_{n,N}(f,t)$ -- погрешность наилучшего приближения функции $f$ полиномом, $L_{n,N}(t)$ -- функция Лебега для системы полиномов $\left\{\hat P_{n,N}\right\}_{n=0}^{N-1}$:
\begin{equation*}
L_{n,N}(t) = \sum\limits_{j=0}^{N-1} \left|\sum\limits_{k=0}^{n}\hat P_{k,N}(t_j)\hat P_{k,N}(t)\right|\vartriangle\eta_{j} =
\sum\limits_{j=0}^{N-1} \left|K_{n,N}(t_j,t)\right|\vartriangle\eta_{j}.
\end{equation*}
Таким образом, требуется исследовать функцию Лебега $L_{n,N}(t)$. Нам понадобится следующая лемма, устанавливающая связь между полиномами порядков $n$ и $n+1$.


\begin{lemma}
\label{sms2:lemm5}
Имеют место следующие равенства:
\begin{equation}
\label{SMS2.lemmaf1eq}
(1-t)\hat P_{n,N}^{1,0}(t) =
\sqrt{\frac{\hat{k}_n}{\hat{k}_{n+1}}}
\left(
\hat P_{n,N}(t)\sqrt{\frac{\hat P_{n+1,N}(1)}{\hat P_{n,N}(1)} }
  - \hat P_{n+1,N}(t)\sqrt{ \frac{\hat P_{n,N}(1)}{\hat P_{n+1,N}(1)} }
\right)
\end{equation}


\begin{equation}
\label{SMS2.lemmaf2eq}
(1+t)\hat P_{n,N}^{0,1}(t) =
\sqrt{\frac{\hat{k}_n}{\hat{k}_{n+1}}}
\left(
\hat P_{n,N}(t)\sqrt{-\frac{\hat P_{n+1,N}(-1)}{\hat P_{n,N}(-1)} }
  - \hat P_{n+1,N}(t)\sqrt{ -\frac{\hat P_{n,N}(-1)}{\hat P_{n+1,N}(-1)} }
\right)
\end{equation}
\end{lemma}

\noindent С помощью леммы \ref{sms2:lemm5}, нам удается доказать также

\begin{lemma}
\label{sms2:lemm6}
Для выражения ядра формулы Кристоффеля-Дарбу справедливо следующее равенство
\begin{equation}
\label{SMS2.kristof}
K_{n,N}(x,y) =
\sqrt{\frac{\hat{k}_{n} \hat P_{n+1,N}(1)}{\hat{k}_{n+1} \hat P_{n,N}(1)}}
\left[ \frac{1-x}{y-x}\hat P_{n,N}^{1,0}(x)\hat P_{n,N}(y) - \frac{1-y}{y-x}\hat P_{n,N}^{1, 0}(y)\hat P_{n,N}(x)
\right].
\end{equation}
\end{lemma}


\noindent Перейдем к формулировке основного результата.

\begin{theorem}
\label{sms2:th1}
Существует такое достаточно малое фиксированное число $\gamma$, что при $n \leq \gamma \lambda_N^{-1/3}$ имеют место оценки
\begin{equation*}
\max_{-1\leq t\leq 1} \left| L_{n,N}(t) \right| \leq c(\gamma) n^{\frac12},
\end{equation*}
\begin{equation*}
\left| L_{n,N}(t) \right| \leq c(\gamma) \ln{n}, \quad -1+\varepsilon \leq t \leq 1 - \varepsilon.
\end{equation*}
\end{theorem}

\noindent Следствием теоремы \ref{sms2:th1} является следующее утверждение.


\begin{theorem}
\label{sms2:th2}
Пусть $\gamma > \frac12$, $f(t) \in Lip_{\gamma}$. Тогда если параметры $n$ и $\lambda_N$ удовлетворяют условию теоремы \ref{sms2:th1}, имеет место следующая оценка
\begin{equation*}
\max_{-1\leq t\leq 1} \left| f(t) - S_{n,N}(f, t)\right| \leq c(\gamma) n^{\frac12 - \gamma},
\end{equation*}
\begin{equation*}
\left| f(t) - S_{n,N}(f, t)\right| \leq c(\gamma) \frac{\ln{n}}{n^{\gamma}}, \quad t \in [-1,1].
\end{equation*}
\end{theorem}
