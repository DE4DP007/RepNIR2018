
В отчетном году были продолжены исследования, связанные с задачей о сходимости рядов Фурье -- Якоби в пространствах Лебега  $L^{p(x)}_\mu([-1,1])$ с переменным показателем $p(x)$, начатые в работе \cite{ShIIBJWShar41}, в которой рассмотрена задача о сходимости в весовом пространстве Лебега $L^{p(x)}_\mu([-1,1])$ с весом $\mu(x)=(1-x^2)^\alpha$ сумм Фурье по ультрасферическим полиномам Якоби $p_n^{\alpha,\alpha}(x)=(h_n^{\alpha,\alpha})^{-\frac12}P_n^{\alpha,\alpha}(x)$ $(n=0,1,\ldots)$, ортонормированных с весом $\mu(x)=(1-x^2)^\alpha$. В настоящей статье мы обобщаем результаты, полученные в  \cite{ShIIBJWShar41} на случай, когда вес имеет вид $\mu(x)=(1-x)^\alpha(1+x)^\beta$, где $\alpha,\beta>-1/2$.


Была рассмотрена задача о базисности системы полиномов Якоби $P_n^{\alpha,\beta}(x)$ в весовом пространстве Лебега $L^{p(x)}_\mu([-1,1])$ с переменным показателем $p(x)$ и весом $\mu=\mu(x)=(1-x)^\alpha(1+x)^\beta$. В случае $\alpha,\beta>-1/2$ показано, что если переменный показатель $p=p(x)$ подчинен  на $[-1,1]$ некоторым естественным условиям, то ортонормированная система  полиномов Якоби $p_n^{\alpha,\beta}(x)=(h_n^{\alpha,\beta})^{-\frac12}P_n^{\alpha,\beta}(x)$ (n=0,1,\ldots) является базисом в $L^{p(x)}_\mu([-1,1])$, если
$4\frac{\alpha+1}{2\alpha+3}<p(1)<4\frac{\alpha+1}{2\alpha+1}$, $4\frac{\beta+1}{2\beta+3}<p(-1)<4\frac{\beta+1}{2\beta+1}$.
При этом отметим, что если это  условие нарушено, то система полиномов Якоби, как это показано в работах \cite{ShIIBJWNewmanRudin} и \cite{ShIIBJWMuckenhoupt1},  не является базисом пространства $L^{p}_\mu([-1,1])$ даже для постоянного показателя $p$. Полученный нами  результат представляет собой обобщение на пространства Лебега $L^{p(x)}_\mu([-1,1])$ с переменным показателем $p(x)$ хорошо известных свойств сумм Фурье -- Якоби, установленных в работах   \cite{ShIIBJWPollard1,ShIIBJWPollard2,ShIIBJWPollard3,ShIIBJWNewmanRudin,ShIIBJWMuckenhoupt1} для весовых пространств Лебега $L^{p}_\mu([-1,1])$ с постоянным показателем $p$, для которого $4\frac{\alpha+1}{2\alpha+3}<p<4\frac{\alpha+1}{2\alpha+1}$.
Отметим также работу \cite{ShIIBJWSharap1}, в которой аналогичная проблема рассмотрена в  пространстве Лебега $L^{p(x)}_\mu([-1,1])$ с переменным показателем $p(x)$ в случае, когда $\alpha=\beta=0$ и, следовательно, $\mu(x)\equiv1$ (случай полиномов Лежандра).


%%%%%%%%%%%%%%%%%%%%%%%%%%%%%
%%%%%%%%%%%%%%%%%%%%%%%%%%%%%
%%%%%%%%%%%%%%%%%%%%%%%%%%%%%


В связи с задачей о приближении функций тригонометрическими полиномами в весовых пространствах Лебега $L^{p(x)}_{2\pi,w}=L^{p(x)}_w ([-\pi,\pi])$  с переменным показателем $p(x)$ был исследован вопрос о равномерной ограниченности семейства сдвигов функции Стеклова
$s_{\lambda,\tau}(f)(x) = \lambda \int_{x-\frac{1}{2\lambda}+\tau}^{x+\frac{1}{2\lambda}+\tau} f(t)dt$, ($1 \leq \lambda < \infty$, $| \tau | \leq \frac{\pi}{\lambda^{\gamma}}$, $0 < \gamma \le 1$).
Было показано, что если весовая функция $w$ удовлетворяет условию, аналогичному известному условию Макенхоупта,
а переменный показатель --- условию Дини -- Липшица, то семейство сдвигов функции
Стеклова равномерно ограничено в весовом пространстве Лебега $L^{p(x)}_{2\pi,w}$.
Данный результат может быть использован при конструировании модулей гладкости функций из пространства $L^{p(x)}_{2\pi,w}$, которые, в свою очередь, могут быть использованы при доказательстве прямых и обратных теорем теории приближений в этих пространствах.









%Исследованы вопросы, связанные с прямыми и обратными теоремами теории приближений функций в весовых пространствах Лебега  с переменным показателем. В частности, показано, что семейство сдвигов функции Стеклова вида
%$$s_{\lambda,\tau}(f)(x) = \lambda \int_{x-\frac{1}{2\lambda}+\tau}^{x+\frac{1}{2\lambda}+\tau} f(t)dt.$$
%равномерно ограничено в весовых пространствах Лебега с переменным показателем $L_{2\pi, w}^{p(x)}$, где $w = w(x)$ --- весовая функция, удовлетворяющая аналогу известного условия Макенхоупта.
%Этот результат может быть использован при конструировании модуля гладкости в весовом пространстве Лебега с переменным показателем и доказательстве прямых теорем в этих пространствах.
%будет использоваться при получении оценки величины наилучшего приближения функции через наилучшее приближение ее производной.













%%%%%%%%%%%%%%%%%%%%%%%%%%%%%
%%%%%%%%%%%%%%%%%%%%%%%%%%%%%
%%%%%%%%%%%%%%%%%%%%%%%%%%%%%


Другое направление исследований, проведенных в ОМИ в отчетном году, связано с приближением непериодических кусочно-гладких функций методом перекрывающих приближений посредством тригонометрических сумм Фурье, средних типа Валле Пуссена и повторных средних типа Валле Пуссена.
Ясно, что приблизить с наперед заданной точностью  непериодическую функцию тригонометрическим полиномом  равномерно на всем периоде $[-\pi,\pi]$ нельзя. Однако, если мы возьмем половину этого отрезка, а именно  $[0,\pi]$, то любую достаточно гладкую на $[0,\pi]$ функцию $f(x)$ можно разложить в ряд Фурье по косинусам,
частичные суммы $S_n(f,x)$ которого можно использовать для ее равномерного  приближения на $[0,\pi]$ с наперед заданной точностью, выбрав порядок $n$ достаточно большим. Но дело заключается в том, что выбранное для этой цели $n$ может оказаться чрезмерно большим. Чтобы убедиться в этом достаточно рассмотреть функцию $f(x)=x$,  для которой разложение в ряд Фурье имеет вид
\begin{equation}\label{1.1}
x=\frac{\pi}{2}-\frac{4}{\pi}\sum_{k=0}^\infty\frac{\cos(2k+1)x}{(2k+1)^2}, \quad x\in [0,\pi],
\end{equation}
 из которой следует, что для отклонения  $f(x)$ от ее частичной суммы $S_{2n+1}(f,x)$ имеет место оценка снизу:
\begin{equation}\label{1.2}
|f(0)-S_{2n+1}(f,0)|=\frac{4}{\pi}\sum_{k=n+1}^\infty\frac{1}{(2k+1)^2}>
\frac{1}{\pi}\sum_{k=n+1}^\infty\frac{1}{(k+1)^2}>\frac{1}{\pi(n+2)}.
\end{equation}
Оценка \eqref{1.2} показывает, что даже для аналитической функции, какой является $f(x)=x$,
частичные суммы косинус-разложений не дают удовлетворительной скорости приближения на всем отрезке $[0,\pi]$. С другой стороны, скорость приближения рассматриваемой функции $f(x)$ посредством $S_{n}(f,x)$ на внутреннем отрезке  $[\varepsilon,\pi-\varepsilon]\subset[0,\pi]$ имеет порядок $O(n^{-2})$. Это свойство сумм Фурье, которое сохраняется  также и для непрерывных кусочно-гладких функций $f(x)$, заданных на $[0,\pi]$, заложено в основу разработки алгоритмов, осуществляющих, так называемые, перекрывающие преобразования Фурье (lapped transform), широко применяемые в вопросах, связанных с обработкой и сжатием временных рядов, в том числе в задаче обработки, сжатия и хранения речевых и звуковых сигналов \cite{LapVPMalvar}. С другой стороны, нетрудно заметить, что частичная сумма $S_{2n+1}(f,x)$ ряда \eqref{1.1} обладает свойством насыщения в том смысле, что не может приближать функцию $f(x)=x$ на всем отрезке  $[\varepsilon,\pi-\varepsilon]\subset[0,\pi]$ быстрее, чем $O(n^{-2})$. Отсюда возникает вопрос о том, чтобы конструировать новые операторы, обладающие существенно лучшими, чем у сумм Фурье  локальными аппроксимативными свойствами в задаче приближения гладких непериодических функций и кусочно-гладких периодических функций. Нами предпринята попытка решить эту проблему на основе повторных средних  Валле Пуссена.




На основе тригонометрических сумм Фурье $S_n(f,x)$ и классических средних Валле Пуссена
$$
_1V_{n,m}(f,x)= \frac1n\sum\nolimits_{l=m}^{m+n-1}S_l(f,x)
$$
 в отчетном году были введены повторные средние Валле Пуссена следующим образом
 $$
_2V_{n,m}(f,x)= \frac1n\sum\nolimits_{k=m}^{m+n-1}{}_1V_{n,k}(f,x),
$$
$$
{}_{l+1}V_{n,m}(f,x)= \frac1n\sum\nolimits_{k=m}^{m+n-1} {}_{l}V_{n,k}(f,x)\quad(l\ge1).
$$
На основе средних $_2V_{n,m}(f,x)$ и перекрывающих преобразований сконструированы операторы, осуществляющие   приближения непрерывных (вообще говоря, непериодических) функций и исследованы их аппроксимативные свойства.


%%%%%%%%%%%%%%%%%%%%%%%%
%%%%%%%%%%%%%%%%%%%%%%%%
%%%%%%%%%%%%%%%%%%%%%%%%


Кроме того, были изучены аппроксимативные свойства частичных сумм ряда Фурье по модифицированным полиномам Мейкснера $M_{n,N}^\alpha(x)=M_n^\alpha(Nx)$ $(n=0, 1, \dots)$, которые при $\alpha>-1$ образуют ортогональную систему на сетке $\Omega_{\delta}=\{0, \delta, 2\delta, \ldots\}$, где $\delta=\frac{1}{N}$, $N>0$ с весом $w(x)=e^{-x}\frac{\Gamma(Nx+\alpha+1)}{\Gamma(Nx+1)}$.
Получена верхняя оценка для функции Лебега указанных частичных сумм.






%%%%%%%%%%%%%%%%%%%%%%%%
%%%%%%%%%%%%%%%%%%%%%%%%
%%%%%%%%%%%%%%%%%%%%%%%%


Далее, была рассмотрена задача о приближении кусочно-линейных периодических функций полиномами $L_{n,N}(f,x)$.
Для заданного натурального числа $N \geq 2$ на отрезке $[0, 2\pi]$ выбрано $N$ равноотстоящих узлов $t_k = 2\pi k / N$ $(0 \leq k \leq N - 1)$. Для каждого натурального числа $n$, удовлетворяющего неравенству $1\leq n\leq\lfloor N/2\rfloor$ обозначим через $L_{n,N}(f)=L_{n,N}(f,x)$
тригонометрический полином порядка $n$ наименьшего квадратического отклонения от функции $f$ в точках $t_k$, который доставляет минимум сумме $\sum_{k=0}^{N-1}|f(t_k)-T_n(t_k)|^2$
среди всех тригонометрических полиномов $T_n$ порядка $n$.


На конкретных примерах показано, что полиномы $L_{n,N}(f,x)$ приближают кусочно-линейную непрерывную периодическую функцию со скоростью $O(1/n)$ равномерно относительно $x \in \mathbb{R}$ и $1 \leq n \leq N/2$, а также приближают такую функцию $f(x)$ со скоростью $O(1/n^2)$ вне сколь угодно малых окрестностей, содержащих точки <<излома>> рассматриваемой ломаной $f(x)$.
Кроме того, на примерах показано, что полиномы $L_{n,N}(f,x)$ приближают кусочно-линейную разрывную функцию со скоростью $O(1/n)$ вне сколь угодно малых окрестностей, содержащих точки разрыва $f(x)$.


Особое внимание уделено приближению полиномами $L_{n,N}(f,x)$ $2\pi$-периодических функций $f_1$ и $f_2$, которые на отрезке $[-\pi, \pi]$ совпадают с функциями $|x|$ и $\mbox{sign } x$ соответственно.
Для первой из этих функций показано, что вместо оценки $\left|f_{1}(x)-L_{n,N}(f_{1},x)\right| \leq c\ln n/n$,
вытекающей из известного неравенства Лебега, для полиномов $L_{n,N}(f,x)$ установлена точная по порядку оценка
$\left|f_{1}(x)-L_{n,N}(f_{1},x)\right| \leq c/n$ ($x \in \mathbb{R}$), которая имеет место равномерно относительно $1 \leq n \leq \lfloor N/2\rfloor$.
Кроме того, получена локальная оценка  $\left|f_{1}(x)-L_{n,N}(f_{1},x)\right| \leq c(\varepsilon)/n^2$ ($\left|x - \pi k\right| \geq \varepsilon$), которая также имеет место равномерно относительно $1 \leq n \leq \lfloor N/2\rfloor$.
Что касается второй из указанных функций $f_2(x)$, то для нее равномерно относительно $1 \leq n \leq \lfloor N/2\rfloor$ получена оценка
$\left|f_{2}(x)-L_{n,N}(f_{2},x)\right| \leq c(\varepsilon)/n$ ($\left|x - \pi k\right| \geq \varepsilon$).
Доказательства полученных оценок базируются на сравнении аппроксимативных свойств дискретных и непрерывных тригонометрических сумм Фурье.


%%%%%%%%%%%%%%%% Рамазанов

Также были продолжены исследования вопросов теории приближений полиномами, рациональными функциями и сплайнами.
Построены гладкие сплайны по трехточечным рациональным интерполянтам, по-новому выбирая полюсы по сетке узлов и свободному параметру, которые сами и их производные первого и второго порядков обладают свойством безусловной сходимости на каждом из трех классов $C_{[a,b]}^{(i)}$ $(i=0,1,2)$ в зависимости от значений параметра.



Даны оценки скорости сходимости сплайнов по трехточечным рациональным
интерполянтам для непрерывных функций в случае равномерных сеток через модуль
 непрерывности  третьего порядка и для непрерывно дифференцируемых функций с выбором узлов сетки через вариацию и через модуль изменения производных первого и второго порядков.







%%%%%%%%%%%%%%%%%%%%%%%%
%%%%%%%%%    ПРОШЛЫЙ ГОД
%%%%%%%%%%%%%%%%%%%%%%%%

%
%Через $L^p_w(a,b)$ обозначим пространство  функций $f(x)$, измеримых  на  $[a,b]$, для которых $\int_a^b|f(x)|^p w(x)dx<\infty$.
%Если $w(x)\equiv1$, то будем писать $L^p_w(a,b)=L^p(a,b)$ и $L(a,b)=L^1(a,b)$.
%
%Через $W^r_{L^p_w(a,b)}$ обозначим пространство Соболева, состоящее из функций $f(x)$, непрерывно дифференцируемых на $[a,b]$ $r-1$ раз, причем $f^{(r-1)}(x)$ абсолютно непрерывна на $[a,b]$  и $f^{(r)}(x)\in L^p_\rho(a,b)$.
%
%
%Пусть $1 \le r$ -- целое, $f(x)\in {W}^{r}_{L(-1,1)}$, $l_{2r-1}(f)=l_{2r-1}(f)(x)$ -- интерполяционный полином Эрмита степени $2r-1$, для которого имеют место равенства
%\begin{equation}
%l_{2r-1}^{(\nu)}(f)(\pm 1)=f^{(\nu)}(\pm 1),\quad \nu=0,1,\ldots,r-1.\label{atpshiieq20}
%\end{equation}
%Сопоставим функции
%\begin{equation}
%F_r(x)={f(x)-l_{2r-1}(f)(x)\over(1-x^2)^r}\label{atpshiieq21}
%\end{equation}
%ee ряд Фурье -- Якоби
%\begin{equation}
%\sum_{k=0}^\infty \hat{F}_{r,k}^\alpha\hat P_k^\alpha(x),\label{atpshiieq22}
%\end{equation}
%где $\alpha>r-1$,
%\begin{equation}
%\hat{F}_{r,k}^\alpha=\int_{-1}^1 F_r(t)\hat P_k^\alpha(t)(1-t^2)^\alpha dt=
%\int_{-1}^1 \bigl(f(x)-l_{2r-1}(f)(x)\bigr)\hat P_k^\alpha(t)(1-t^2)^{\alpha-r} dt.
%\label{atpshiieq23}
%\end{equation}
%Если ряд Фурье -- Якоби сходится, то, учитывая \eqref{atpshiieq21}, мы можем записать
%\begin{equation}
%f(x)=l_{2r-1}(f)(x)+(1-x^2)^r\sum_{k=0}^\infty F_{r,k}^\alpha\hat P_k^\alpha(x).\label{atpshiieq24}
%\end{equation}
%Через $\sigma_n^{\alpha,r}(g)=\sigma_n^{\alpha,r}(g,x)$ обозначим частичную сумму ряда \eqref{atpshiieq24} следующего вида
%\begin{equation}
%\sigma_n^{\alpha,r}(g,x)=l_{2r-1}(g)(x)+(1-x^2)^r\sum_{k=0}^{n-2r} q_k^\alpha\hat P_k^\alpha(x).\label{atpshiieq25}
%\end{equation}
%Нетрудно увидеть, что $\sigma_n^{\alpha,r}(g)$ представляет собой проектор на подпространство алгебраических полиномов $p_n(x)$ степени $n$, т.е. $\sigma_n^{\alpha,r}(p_n)=p_n$. Кроме того, из \eqref{atpshiieq20} и \eqref{atpshiieq25} вытекает, что $g^{(\nu)}(\pm1)=(\sigma_n^{\alpha,r}(g,x))^{(\nu)}_{x=\pm1}=(\sigma_m^{\alpha,r}(g,x))^{(\nu)}_{x=\pm1}$ $(\nu=0,1,\ldots, r-1)$ для любых $n$ и $m$. Это означает, что ряд \eqref{atpshiieq24} является специальным рядом, частичные суммы которого гладко <<склеены>>  в точках $x=\pm1$, причем степень гладкости равна $r$.
%
%Можно показать, что специальные ряды вида \eqref{atpshiieq24} при $\alpha=r$, где $1 \le r$ -- целое, могут быть истолкованы как ряды Фурье по классическим полиномам Якоби
%$P^{-r,-r}_k(x) = \frac{(-1)^r}{n!\kappa(x)} \left(  \kappa(x)(1-x^2)^n \right)^{(n)}$ с $\kappa(x)=\kappa(x;-r-r)$, где $\kappa(x;\alpha,\beta)=(1-x)^\alpha(1+x)^\beta$, ортогональным в смысле некоторого соболевского скалярного произведения. Теория полиномов, ортогональных относительно скалярных произведений типа Соболева, получила в последнее время интенсивное развитие в работах многочисленных авторов (см., например, работы \cite{atpshii24, atpshii25, atpshii26, atpshii27, atpshii28, atpshii29}
%и др.). С другой стороны, можно показать, что ряды вида \eqref{atpshiieq24} при $\alpha=r$, где $1 \le r$ -- целое, представляют собой не что иное как смешанные ряды по полиномам Лежандра.
%Это позволяет использовать при исследовании аппроксимативных свойств ряда \eqref{atpshiieq24} и ряда Фурье по полиномам Якоби, ортогональным по Соболеву, методы и подходы, разработанные при решении аналогичной задачи для смешанных рядов по полиномам Лежандра.
%
%
%
%Пусть $f(t)$ достаточно гладкая функция, заданная на $[-1,1]$.
%Рассмотрим задачу о приближении $f(t)$ комбинациями вида
%$p_n(t)+\tau_m(t)$, где $p_n(t)$ -- алгебраический полином степени
%$n$,
%$$
%\tau_m(t)=a_0+\sum_{k=1}^ma_k\cos k\pi t+b_k\sin k\pi t
%$$
%-- тригонометрический полином порядка $m$. Подобные задачи часто
%встречаются в различных областях приложений, в которых для заданного
%временного ряда наблюдений $f(t)$ требуется найти так называемый
%тренд $p_n(t)$ и периодическую составляющую $\tau_m(t)$.
%
%Обозначим через $X$ некоторое нормированное с нормой $\|f\|_X$
%пространство функций $f=f(t)$, заданных  на $[-1,1]$.  Если $X$
%содержит все алгебраические и тригонометрические полиномы, то мы
%можем определить величину
%$$E_{n,m}(f)_X=\inf_{p_n,\tau_m}\|f-p_n-\tau_m\|_X,$$
%представляющую собой наилучшее приближение функции $f\in X$ алгебро-тригонометричес\-кими полиномами $p_n(x)+\tau_m(x)$ порядка $n+m$.
%Если $\mathcal{ Y}$ -- некоторый подкласс из $X$, то положим
%$$\mathcal{ E}_{n,m}(\mathcal{ Y})_X=\sup_{f\in \mathcal{ Y}}E_{n,m}(f)_X.$$
%
%Ставится задача об исследовании поведения величины $\mathcal{
%E}_{n,m}(\mathcal{ Y})_X$ для различных  нормированных пространств $X$ и
%классов $\mathcal{ Y}\subset X$. Основное внимание ниже будет уделено
%случаю, когда $X=C[-1,1]$ - пространство функций $f=f(x)$, заданных
%и непрерывных на $[-1,1]$, для которых норма определяется обычным
%образом:
%$$
%\|f\|=\|f\|_\infty =\max_{-1\le x\le1}|f(x)|.
%$$
% В пространстве $C[-1,1]$ мы будем выделять классы Соболева
% $W^r_{p(\cdot)}(M)$. Напомним, что $W^r_{p(\cdot)}(M)$
%состоит из $r-1$-раз непрерывно дифференцируемых функций $f(x)$, для
%которых $f^{(r-1)}(x)$ абсолютно непрерывна  а $f^{(r)}(x)\in
%L^{p(x)}(-1,1)$ и
%\begin{equation}\label{atpshiieq4.1.1}
%    \|f^{(r)}\|_{p(\cdot)}([a,b])\leq M.
%\end{equation}
%Ясно, что $\widetilde{W}^{r}_{p(\cdot)}(M,-1,1)$ является подклассом
%класса $W_{p(\cdot)}^{r}(M,-1,1)$, состоящим из функций $f(x)\in
%W_{p(\cdot)}^{r}(M,-1,1)$, допускающих $2$-периодическое продолжение
%с сохранением гладкости, т. е. $f^{(\nu)}(-1)=f^{(\nu)}(1)$
%$(0\leq\nu\leq r-1)$. Положим $W_{p(\cdot)}^r=\bigcup_{M>0}
%W_{p(\cdot)}^r(M)$. При $p=\infty$ будем считать, что
%$W^r_\infty(M)$ состоит из $r$-раз непрерывно дифференцируемых
%функций $f(x)$, заданных на $[-1,1]$ и, соответственно, $\tilde
%W^r_\infty(M)\subset W^r_\infty(M)$.
%
%Основная идея, которая позволяет успешно решить поставленную задачу
%для классов $W^r_{p(\cdot)}(M)$ заключается в следующем. Для
%заданной функции $f\in W^r_{p(\cdot)}(M)$ подбирается алгебраический
%полином $p_n(x)=p_n(f,x)$, удовлетворяющий условиям
%$$p_n^{(\nu)}(\pm1)=f^{(\nu)}(\pm1)\quad (\nu=0,1,\ldots, r-1),$$
%который доставляет для функции $f(x)$ и ее производных
%$f^{(\nu)}(x)$ одновременное равномерное на $[-1,1]$ приближение
%порядка $n^{\nu-r}$. Затем разность $g_n(x)=f(x)-p_n(x)$
%продолжается 2-периодически на всю числовую ось  сохранением
%гладкости. После этого функцию $g_n(x)$ приближаем
%тригонометрическими полиномами. Этим методом в случае $p=\infty$
%получены неулучшаемые порядковые оценки для величины
%$\mathcal{ E}_{n,m}(W^r_{p(\cdot)}(M))_{C[-1,1]}$ при $n+m\to \infty$, а
%именно
%$$\mathcal{ E}_{n,m}(W^r_\infty(M))_{C[-1,1]}\asymp(n+m)^{-r}.$$
% В случае когда $p(x)\in \hat{\mathcal{
%P}}\cap\tilde {\mathcal{ P}}$ и $4/3<p(\pm1)<4$ показано,
%что для $f(x)\in W^r_{p(\cdot)}((M,-1,1)) $ и каждой пары $(n,m)$
%найдется алгебро-тригономет\-ри\-ческий полином $p_n(x)+\tau_m(x)$, для
%которого равномерно относительно $x\in[-1,1]$ справедлива оценка
%$$
%|f^{(\nu)}(x)-p_n^{(\nu)}(x)-\tau_m^{(\nu)}(x)|=c(r,M,p)(n+m)^{-r+\nu+\frac1{p(x)}},\quad
%0\le\nu\le r-1.
%$$
%
% При решении этой задачи основную роль играют так называемые смешанные ряды по полиномам Лежандра,
% введенные и исследованные в работах \cite{atpshii24, atpshii25, atpshii26, atpshii27, atpshii28, atpshii29}. Значительное
% место в отчетном году отведено исследованию свойств этих рядов.
%
%В главе 3 книги \cite{shiimonog} мы рассмотрели некоторые традиционные задачи
%теории приближений периодических функций тригонометрическими
%полиномами. Среди  них особенного внимания заслуживает  задача,
%рассмотренная в заключительной части главы 3 книги \cite{shiimonog},
%связанная с приближением  в заданной точке $x\in\mathbb{R}$ функций
%из классов $\widetilde{W}^{r}_{p(\cdot)}(M,-1,1)$ суммами Фурье по
%тригонометрической системе. Мы обращаем внимание на эту задачу
%потому, что классы Соболева $\widetilde{W}^{r}_{p(\cdot)}(M,-1,1)$ с
%переменным показателем $p=p(x)$ по своей природе состоят из функций
%$f=f(x)$, обладающих существенно переменным поведением, зависящим от
%расположения точки $x$ на периоде $[-1,1]$.   В настоящей работе нам
%потребуется решить аналогичную задачу для функций из классов
%$W^{r}_{p(\cdot)}(M,-1,1)$. Если $f(x)\in W^{r}_{p(\cdot)}(M,-1,1)$,
%но в класс $\widetilde{W}_{p(\cdot)}^{r}(M,-1,1)$ не входит, то
%суммы Фурье $S^{T}_{m}(f,x)$ по тригонометрической системе,
%определенные равенством (3.6.2) из  \cite{shiimonog}, не смогут аппроксимировать $f(x)$ с
%заданной точностью в каждой точке $x\in[-1,1]$ по той простой
%причине, что $S^{T}_{m}(f,x)$ является $2$-периодической функцией, а
%$f(x)\in
%W^r_{p(\cdot)}(M,-1,1)\backslash\widetilde{W}^{r}_{p(\cdot)}(M,-1,1)$
%таковой не является. В связи с этим возникает задача о нахождении
%метода приближения непериодических гладких функций $f(x)$, который
%дает одновременное приближение функций $f(x)\in
%W^{r}_{p(\cdot)}(M,-1,1)$ и ее производных $f^{(\nu)}(x)$
%$(0\leq\nu\leq r)$, удовлетворяющие оценкам, аналогичным (3.6.40) и
%(3.6.41) из \cite{shiimonog}. Отметим сразу, что, как это было показано в работах
%\cite{atpshii24, atpshii25, atpshii26, atpshii27, atpshii28, atpshii29}, суммы Фурье по классическим ортогональным полиномам
%Чебышева, Лежандра и Якоби для  этой цели не подходят. В настоящей
%работе поставленная задача решается с помощью  смешанных рядов по
%полиномам Лежандра. Мы получим для частичных сумм
%$\mathcal{Y}_{n+2r}(f,x)$ смешанного ряда функции $f\in
%W_{p(\cdot)}^{r}(M,-1,1)$ по полиномам Лежандра аналог теоремы 3.6.2 из \cite{shiimonog}
%при $\frac43< p(\pm1)<4$. Более того, мы покажем, что
%$\mathcal{Y}_{n+2r}^{(\nu)}(f,x)$ вблизи точек $\pm1$ дает
%приближение функции $f^{(\nu)}(x)$, существенно лучше, чем вдали от
%этих точек (см. теорему \ref{atpshiith-4.5.3}). Это обстоятельство играет важную
%роль при решении задачи приближения гладких функций $f(x)$
%алгебро-тригонометрическими полиномами.
%
%
%
%Хорошо известно, что полиномы Бернштейна, определяемые формулой
%\begin{equation*}
%  B_n(f,x)=\sum\limits_{k=0}^np_{nk}(x)f\left(\frac kn\right),\, f\in C([0,1]),\, x\in[0,1],
%\end{equation*}
%где $p_{nk}(x)=C_n^kx^k(1-x)^{n-k}$, равномерно сходятся в пространстве $C([0,1])$, однако они не подходят для аппроксимации разрывных функций. В работе \cite{shtn1} Л.В. Канторовичем были введены операторы, представляющие собой аналог полиномов Бернштейна для суммируемых функций. Для $f\in L^1([0,1])$ определим, следуя \cite{shtn1}, оператор Бернштейна -- Канторовича следующим образом
%\begin{equation}\label{BernK}
%  K_n(f)=K_n(f,x)=\sum\limits_{k=0}^np_{nk}(x)(n+1)\int\limits_{\Delta_{nk}}f(t)dt,
%\end{equation}
%где $\Delta_{nk}=[\frac{k}{n+1},\frac{k+1}{n+1}]$, $n\in \mathbb{N}$. В работе \cite{shtn2} доказано, что для произвольного постоянного показателя $p\ge1$ и $f\in L^p([0,1])$ имеет место соотношение $\|f-K_n(f)\|_p\to0$ ($n\to\infty$).
%В настоящей работе приведен аналогичный результат для функций из пространств Лебега с переменным показателем, полученный в отчетном году.
%Для точной формулировки этого результата нам понадобятся некоторые обозначения.
%
%Пусть $E$ -- измеримое подмножество числовой оси, $p=p(x)$ -- измеримая и существенно ограниченная на $E$ функция. Через $L^{p(x)}(E)$ обозначим множество измеримых на $E$ функций $f$, для которых
%$
%  \int_{E}\left|f(x)\right|^{p(x)}dx<\infty.
%$
%Из \cite{vpmshiiShar4} известно, что если переменный показатель $1\le p(x)$ существенно ограничен на $E$, то $L^{p(x)}(E)$ можно превратить в банахово пространство с нормой
%\begin{equation}\label{LpxNorm}
%  \|f\|_{p(\cdot)}=\|f\|_{p(\cdot)}(E)=\inf\left\{\alpha>0:\quad\int\limits_E\left|\frac{f(x)}\alpha\right|^{p(x)}dx\le1\right\}.
%\end{equation}
%При этом отметим  \cite{shiimonog}, что если $1\le p(x)\le q(x)\le\bar{q}(E)<\infty$, то имеет место неравенство
%\begin{equation}\label{normexpineq}
%  \|f\|_{p(\cdot)}(E)\le r_{p,q}\|f\|_{q(\cdot)}(E),
%\end{equation}
%в котором \mbox{$r_{p,q}=\max\left\{1/\underline{\beta}(E)+\mu(E)/\underline{\beta}(E),1\right\}$}, \mbox{$\beta(x)=q(x)/p(x)$},
%где $\bar{g}(M)=\operatorname*{ess\,sup}_{x\in M}g(x)$, $\underline{g}(M)=\operatorname*{ess\,inf}_{x\in M}g(x)$.
%



