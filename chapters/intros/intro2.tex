Теория полиномов, ортогональных относительно скалярных произведений типа Соболева, получила в последние три десятилетия интенсивное развитие и нашла ряд важных приложений (см. \cite{sobleg-IserKoch, sobleg-MarcelAlfaroRezola, charlier-Shar3, sobleg-KwonLittl1, sobleg-KwonLittl2, sobleg-MarcelXu} и цитированную там литературу). Характерной особенностью скалярных произведений типа Соболева является, в частности, то, что они, как правило, содержат слагаемые, которые <<контролируют>> поведение соответствующих ортогональных полиномов в одной или нескольких точках числовой оси. Например, часто рассматривают скалярное произведение вида
\begin{equation}\label{Shar_eq1}
\langle f,g \rangle = \sum_{\nu=0}^{r-1} f^{(\nu)}(a)g^{(\nu)}(a) + \int\limits_{a}^{b} f^{(r)}(t)g^{(r)}(t)\rho(t)dt,
\end{equation}
в котором $f$ и $g$ --- функции, заданные на $[a,b]$ и непрерывно дифференцируемые там $r-1$ раз, для которых $f^{(r-1)}(x)$ и $g^{(r-1)}(x)$ абсолютно непрерывны и $f^{(r)}(x),\ g^{(r)}(x)\in L^2_\rho(a,b)$, где $L^2_\rho(a,b)$ --- пространство Лебега с весом $\rho(x)$.
Следует отметить, что полиномы, ортогональные по Соболеву, по своим свойствам могут весьма существенно отличаться от обычных ортогональных на интервале полиномов. Например, в некоторых случаях оказывается так, что полиномы, ортогональные по Соболеву на интервале $(a,b)$, могут иметь нули, совпадающие с одним или с обоими концами этого интервала. Это обстоятельство имеет важное значение для некоторых приложений, в которых требуется, чтобы значения частичных сумм ряда Фурье функции $f(x)$ по рассматриваемой системе ортогональных полиномов совпали в концах интервала $(a,b)$ со значениями $f(a)$ и $f(b)$. Заметим, что обычные ортогональные с положительным на $(a,b)$ весом полиномы этим важным свойством не обладают. Скалярное произведение \eqref{Shar_eq1} имеет одну особую точку, а именно, точку $a$, в окрестности которой <<контролируется>>, поведение соответствующих полиномов, ортогональных по Соболеву. Это достигается за счет наличия в скалярном произведении \eqref{Shar_eq1} слагаемого вида $\sum\limits_{\nu=0}^{r-1}f^{(\nu)}(a)g^{(\nu)}(a)$.

Надо отметить, что полиномы, ортогональные по Соболеву, совпадают в частных случаях с так называемыми смешанными рядами, 
исследования по которым были начаты в \cite{sobleg-Shar11} и продолжены в работах
\cite{sobleg-Shar12, sobleg-Shar13, sobleg-Shar15, sobleg-Shar16, sobleg-Shar17, sobleg-Shar18, sobleg-sharap3}.  В этих работах смешанные ряды не трактовались как ряды Фурье по полиномам, ортогональным по Соболеву. Вместо этого при исследовании аппроксимативных свойств смешанных рядов в \cite{sobleg-Shar11, sobleg-Shar12, sobleg-Shar13, sobleg-Shar15, sobleg-Shar16, sobleg-Shar17, sobleg-Shar18, sobleg-sharap3} использовался тот факт, что  частичная сумма смешанного ряда представляет собой проектор на соответствующее подпространство алгебраических полиномов, не обращая внимание на то, что она является суммой Фурье по тем самым полиномам, которые фигурируют в смешанном ряде и образуют ортонормированную систему относительно скалярного произведения типа Соболева. В данной работе, напротив,  мы будем рассматривать смешанные ряды именно как ряды Фурье по полиномам, ортонормированным относительно скалярных произведений типа Соболева с дискретными массами, что позволяет применять к исследованию некоторых важных свойств этих рядов  терминологию и аппарат теории гильбертовых пространств. С этой целью для  натурального $r$ и $\alpha,\beta>-1$ рассмотрены полиномы $p^{\alpha,\beta}_{r,k}(x)$ $(k=0,1,\ldots)$, ассоциированные с  классическими полиномами Якоби, ортонормированные относительно скалярного произведения типа Соболева
\begin{equation}\label{sobleg-1.1}
<f,g>=\sum_{k=0}^{r-1}f^{(k)}(-1)g^{(k)}(-1)+\int_{-1}^1 f^{(r)}(t)g^{(r)}(t)\kappa(t)dt,
\end{equation}
где $\kappa(t)=(1-t)^\alpha(1+t)^\beta$. Для этих полиномов установлены явный вид (теорема \ref{soblegtheo1})  и связь с полиномами Якоби $P_{k+r}^{\alpha-r,\beta-r}(x)$ (теорема \ref{soblegtheo2}).  Особое внимание уделено случаю $\alpha=\beta=0$. Соответствующие полиномы $p_{r,k}(x)=p^{0,0}_{r,k}(x)$, ортогональные по Соболеву, определяются равенством \eqref{sobleg-3.27} при $\alpha=\beta=0$ (порождены полиномами Лежандра). Они связаны с полиномами Якоби особенно просто (см.  \eqref{sobleg-3.42} и \eqref{sobleg-3.44}). Именно этот факт позволяет   (см. п. \ref{sobleg5}) ввести в рассмотрение некоторые специальные ряды по ультрасферическим полиномам Якоби $p_k^{\alpha,\alpha}(x)$, частичные суммы $\sigma_{r,n}^\alpha(f)$ которых обладают важным для приложений (см. начало п. \ref{sobleg5}) свойством <<прилипания>> в точках $\pm1$. Эти ряды возникают как естественное обобщение смешанного ряда по полиномам Лежандра $p_k(x)=p_k^{0,0}(x)$, представляющего собой ряд Фурье  по полиномам $p_{r,k}(x)$  $(k=0,1,\ldots)$. При этом отметим, что в случае $\alpha=r$ введенный в п. \ref{sobleg5} специальный ряд совпадает с рядом Фурье по полиномам $p_{r,k}(x)$  $(k=0,1,\ldots)$ (или, что то же, со смешанным рядом по полиномам Лежандра).
Упомянутое выше свойство прилипания частичных сумм $\sigma_{r,n}^\alpha(f)=\sigma_{r,n}^\alpha(f,x)$  с исходной функцией $f(x)$ в точках $\pm1$  (и, следовательно, $\sigma_{r,n}^\alpha(f,x)$ и $\sigma_{r,m}^\alpha(f,x)$ друг с другом) состоит в том, что если функция $f(x)$ $r-1$-раз дифференцируема в точках $\pm1$ , то  $f^{(\nu)}(\pm1)=(\sigma_{r,N}^\alpha(f,\pm1))^{(\nu)}$, $\nu=0,1,\ldots, r-1$. Это свойство позволяет использовать $\sigma_{r,n}^\alpha(f)$ как аппарат приближения в задаче о взвешенной с весом $(1-x^2)^{-r+\frac\alpha2+\frac14}$ аппроксимации алгебраическими полиномами дифференцируемых и аналитических функций, заданных на $[-1,1]$. Результаты работы, сформулированные в теоремах \ref{soblegtheo3} и \ref{soblegtheo4}, касаются оценок  взвешенного приближения на $[-1,1]$ с весом $(1-x^2)^{-r+\frac\alpha2+\frac14}$  функции $f(x)$ частичными суммами $\sigma_{r,n}^\alpha(f,x)$. Оценка
$ |f(x)-\sigma_{r,N}^\alpha(f,x)|/(1-x^2)^{r-\frac{\alpha}{2}-\frac14}\le c(r,\alpha)\sum_{k=0}^N\frac{E_{N+k}^r(f)}{k+1}$, установленная в теореме \ref{soblegtheo3}, в которой величина $E_{n}^r(f)$ представляет собой взвешенное с весом $(1-x^2)^{-r/2}$ наилучшее приближение функции $f$ на $[-1,1]$ алгебраическими полиномами $p_n(x)$ степени $n$, которые подчиняются условиям $p_n^{(\nu)}(\pm1)=f^{(\nu)}(\pm1)$ $(\nu=0,\ldots, r-1)$, показывает, что  частичные суммы $\sigma_{r,n}^\alpha(f,x)$ обладают весьма привлекательными с точки зрения различных приложений аппроксимативными свойствами. Кроме того, оценка $|f(x)-\sigma_{r,N}^\alpha(f,x)|/(1-x^2)^{r-\frac{\alpha}{2}-\frac14}\le\frac{c(r,\alpha, B)}{(1-q)^2}q^{N-2r+1}$, полученная в теореме \ref{soblegtheo4} для аналитической в эллипсе $\mathcal{ E}_q$  с фокусами в точках $\pm1$, сумма полуосей которого равна $R=1/q>1$, функции $f$ показывает, что   по своим аппроксимативным свойствам на классах аналитических функций $A_q(B)$ (см. п. \ref{sobleg5}) $\sigma_{r,n}^\alpha(f,x)$ не уступают суммам Фурье $S_n(f,x)$ по полиномам Чебышева первого рода. При этом, как уже отмечалось, $\sigma_{r,n}^\alpha(f,x)$ обладает свойством прилипания:  $f^{(\nu)}(\pm1)=(\sigma_{r,N}^\alpha(f,\pm1))^{(\nu)}$, $\nu=0,1,\ldots, r-1$, тогда как суммы Фурье --- Чебышева $S_n(f,x)$ этим важным свойством, вообще говоря, не обладают.

Результаты, сформулированные в виде теорем \ref{soblegtheo1} и \ref{soblegtheo2} (п.\ref{sobleg4}), направлены на исследование свойств самих полиномов $p^{\alpha,\beta}_{r,n+r}(x)$.   В теореме \ref{soblegtheo1} получен явный вид полинома $p^{\alpha,\beta}_{r,n+r}(x)$, представляющий собой разложение $p^{\alpha,\beta}_{r,n+r}(x)$ по степеням $(x+1)^{k+r}$ с $0\le k\le n$. Этот результат нацелен в первую очередь на изучение асимптотического поведения полиномов  $p^{\alpha,\beta}_{r,n+r}(x)$  в окрестности точки $x=-1$ при $n\to\infty$. Кроме того, он может быть использован для нахождения значений полиномов $p^{\alpha,\beta}_{r,n+r}(x)$ с небольшими степенями $n+r$ в заданной точке $x$. Это касается также наиболее детально изученного нами ранее случая $\alpha=\beta=0$.  Дело заключается в том, что результат, полученный в теореме \ref{soblegtheo2}, в которой установлена связь полиномов $p^{\alpha,\beta}_{r,n+r}(x)$ с классическими полиномами Якоби $P_{n+r}^{\alpha-r,\beta-r}(x)$, не справедлив для нескольких начальных степеней $n+r$ и, стало быть, не может быть использован для описания свойств полиномов $p^{\alpha,\beta}_{r,n+r}(x)$ с такими степенями. Например, для отмеченного уже частного случая $\alpha=\beta=0$ равенства \eqref{sobleg-3.42} и \eqref{sobleg-3.44} теряют смысл для полиномов $p_{r,k}(x)=p^{0,0}_{r,k}(x)$, у которых $r\le k\le 2r-1$. Теорема \ref{soblegtheo1} позволяет, в частности, восполнить этот пробел (см. \eqref{sobleg-3.46}). Что касается самой теоремы \ref{soblegtheo2}, то она может быть использована при исследовании асимптотических свойств полиномов $p^{\alpha,\beta}_{r,n+r}(z)$ при $|1+z|\ge\varepsilon>0$ и $n\to\infty$. 

Частичные суммы рядов Фурье --- Соболева (смешанных рядов) по классическим ортогональным полиномами, в отличие от сумм Фурье по этим же полиномам, успешно могут быть использованы в задачах, в которых требуется одновременно приближать дифференцируемую функцию и ее несколько производных. Такие проблемы непременно возникают, например, при решении краевых задач для обыкновенных дифференциальных уравнений.


В параграфе \ref{sect-laplas} рассмотрено применение рядов Фурье по полиномам, порожденным классическими полиномами Лагерра $L_n^\alpha(x)$, к задаче обращения преобразования Лапласа. В этом параграфе в связи задачей об обращении преобразования Лапласа вводятся \textit{обобщенные специальные ряды} по полиномам Лагерра, обладающие на конечных отрезках вида $[0,A]$ лучшими, чем у классических рядов Фурье --- Лагерра, аппроксимативными свойствами для дифференцируемых функций   (см. п. \ref{laplas6}).

В параграфе \ref{sect-ode} рассмотрена задача о приближенном решении задачи Коши для ОДУ вида
\begin{equation}\label{intro-equ102-3.1}
y'(x)=f(x,y), \quad y(0)=y_0,
\end{equation}
суммами Фурье по системе $\{\varphi_{r,n}(x)\}_{n=0}^\infty$, ортогональной по Соболеву и порожденной ортонормированной системой функций $\{\varphi_{n}(x)\}_{n=0}^\infty$ посредством равенств \eqref{sobleg-3.5} и \eqref{sobleg-3.6}.
Наряду с различными сеточными методами для решения этой задачи часто применяют так называемые спектральные методы \cite{sobleg-Tref1, sobleg-Tref2, sobleg-SolDmEg, sobleg-Pash, sobleg-MMG2016}. Напомним, что суть спектрального метода решения задачи Коши  для ОДУ \eqref{sobleg-3.1} заключается в том, что в первую очередь искомое решение $y(x)$ представляется в виде ряда Фурье 
\begin{equation}\label{intro-fourier-series}
y(x)=\sum_{k=0}^\infty \hat y_k\psi_k(x)
\end{equation}
по подходящей ортонормированной системе $\{\psi_k(x)\}_{k=0}^\infty$ (чаще всего в качестве $\{\psi_k(x)\}_{k=0}^\infty$ используют    тригонометрическую систему, ортогональные полиномы, вэйвлеты, корневые функции того или иного дифференциального оператора  и некоторые другие). На втором этапе осуществляется подстановка вместо $y(x)$ упомянутого ряда Фурье в уравнение \eqref{intro-equ102-3.1}. Это приводит к системе уравнений относительно неизвестных коэффициентов $\hat y_k$ ($k=0,1,\ldots$). На третьем этапе требуется решить эту систему с учетом начальных условий  $y^{(k)}(-1)=y_k$, $k=0,1,\ldots, r-1$ исходной задачи Коши.
Одна из основных трудностей, которая возникает на этом этапе, состоит в том, чтобы
выбрать такой ортонормированный базис $\{\psi_k(x)\}_{k=0}^\infty$, для которого искомое решение $y(x)$ уравнения \eqref{intro-equ102-3.1}, представленное в виде ряда  \eqref{intro-fourier-series}, удовлетворяло начальным условиям $y^{(k)}(-1)=y_k$, $k=0,1,\ldots,r-1$. Более того, поскольку в результате решения системы уранений относительно неизвестных коэффициентов $\hat y_k$  будет найдено только конечное их число с $k=0,1,\ldots, n$, то весьма важно, чтобы частичная сумма ряда \eqref{intro-fourier-series} вида
$ y_n(x)=\sum_{k=0}^n\hat y_k\psi_k(x)$, будучи приближенным решением рассматриваемой задачи Коши, также удовлетворяла  начальным условиям $y_n^{(k)}(-1)=y_k$, $k=0,1,\ldots,r-1$. Как показано в параграфе~\ref{sect-ode}, базис $\{\psi_k(x)=p_{r,k}^{\alpha,\beta}(x)\}_{k=0}^\infty$, состоящий из полиномов
$p_{r,k}^{\alpha,\beta}x)$, ортонормированных по Соболеву относительно скалярного произведения \eqref{sobleg-1.1} и порожденных ортонормированными полиномами Якоби $p_k^{\alpha,\beta}(x)$  посредством равенства \eqref{sobleg-3.27}, требуемыми свойствами обладает. С использованием рядов Фурье по системе $p_{r,k}^{\alpha,\beta}x)$ разработан итерационный метод решения задачи Коши для ОДУ.
Таким образом, полиномы, ортогональные по Соболеву относительно скалярного произведения \eqref{sobleg-1.1}, тесно связаны с задачей Коши для уравнения \eqref{intro-equ102-3.1}. Можно также показать,что полиномы $p_{r,k}^{0,0}(x)$ могут служить удобным и эффективным методом приближенного решения двухточечной краевой задачи для уравнений типа \eqref{intro-equ102-3.1}.




%%%%%%%%%%%%%%%%%%
%%%%%%%%%%%%%%%%%%
%%%%%%%%%%%%%%%%%%




В отчетном году было продолжено рассмотрение систем дискретных функций $\mathcal{\psi}_{r,n}(x)$
$(r\in\mathbb{N}, n=0,1,\ldots)$, ортогональных относительно скалярных произведений типа Соболева следующего вида
\begin{equation}\label{1.1}
\langle f,g\rangle=\sum_{k=0}^{r-1}\Delta^kf(0)\Delta^kg(0)+
\sum_{j=0}^\infty\Delta^rf(j)\Delta^rg(j)\rho(j),
\end{equation}
где функции $f$ и $g$ заданы на  множестве $\Omega=\{0,1,\ldots,\}$, $\rho=\rho(j)$ $(j\in \Omega)$ -- дискретная весовая функция.   В случае, когда $r=0$ мы будем считать, что $\sum_{k=0}^{r-1}\Delta^kf(0)\Delta^kg(0)=0$. При $r\ge1$ особой точкой в скалярном произведении \eqref{1.1} является  $x=0$, в которой <<контролируется>> поведение соответствующих ортогональных по Соболеву функций дискретной переменной, благодаря присутствию в  \eqref{1.1} выражения  $\sum_{k=0}^{r-1}\Delta^kf(0)\Delta^kg(0)$. Нам удалось показать, что системы дискретных функций, ортонормированные по Соболеву, могут быть использованы для приближенного решения разностных уравнений (в том числе и нелинейных) спектральными методами путем представления искомого решения рассматриваемого уравнения (систем уравнений) в виде рядов по функциям, ортогональным по Соболеву. Основная идея рассматриваемого подхода заключается в конструировании некоторого итерационного процесса для приближенного нахождения неизвестных коэффициентов указанного разложения искомого решения разностного уравнения. При доказательстве сходимости сконструированного итерационного процесса решающую роль играют свойства функций, ортонормированных по Соболеву и порожденных заданной ортонормированной системой функций $\mathcal{\psi}_{n}(x)$ $( n=0,1,\ldots)$.

Как частный случай системы $\mathcal{\psi}_{r,n}(x)$ рассмотрена система полиномов $s_{r,n}(x)$, порожденная системой классических ортонормированных полиномов Шарлье $s_{n}(x)$. Основное внимание уделено получению различных свойств полиномов $s_{r,n}(x)$.
%Следует отметить, что \eqref{1.1} является дискретным аналогом скалярного произведения следующего вида
%\begin{equation}\label{Gadz_eq1}
%\langle f,g\rangle=\sum_{\nu=0}^{r-1}f^{(\nu)}(0)g^{(\nu)}(0)+\int_{0}^{\infty} f^{(r)}(x)g^{(r)}(x)\rho(x) dx,
%\end{equation}
%в котором $\rho(x)$ -- весовая функция, определенная равенством
%$
%\rho(x)=x^{\alpha} e^{-x}.
%$
%$f$ и $g$ -- функции, заданные на $[0,\infty)$ и непрерывно дифференцируемые там $r-1$--раз,
%для которых $f^{(r-1)}(x)$ и $g^{(r-1)}(x)$ абсолютно непрерывны и $f^{(r)}(x)$, $g^{(r)}(x)\in L^2_\rho(0, \infty)$, где $L^2_\rho(0, \infty)$--пространство Лебега с весом $\rho(x)$.
%Рассмотрена система полиномов $l_{r,n}^\alpha(x)$ $(r-\text{натуральное число}, n=0, 1, \ldots)$, введенная в работах \cite{Gadz1} и \cite{Gadz2}, ортонормированная при $\alpha>-1$ относительно скалярного произведения \eqref{Gadz_eq1}.
%Полиномы $l_{r,n}^{\alpha}(x)$, порожденные классическими ортонормированными полиномами Лагерра $l_n^{\alpha}(x)$ $(n=0,1,\ldots)$, определяются \cite{Gadz1}, \cite{Gadz2} с помощью равенств \eqref{ramis-Gadz_eq8} и \eqref{ramis-Gadz_eq9}.
%Для полиномов $l_{r,n}^\alpha(x)$ получены рекуррентные соотношения, которые могут быть использованы для изучения различных свойств этих полиномов и вычисления их значений при любых $x$ и $n$.
%
%
%
%
%
%%%%%%%%%%%%%%%%%%%
%%%%%%%%%%%%%%%%%%%
%%%%%%%%%%%%%%%%%%%
%
%
%Теория полиномов, ортогональных относительно скалярных произведений типа Соболева, получила в последние три десятилетия интенсивное развитие и нашла ряд важных приложений (см. \cite{Shar1,Shar2,Shar3,Shar4,Shar5,Shar6} и цитированную там литературу). Характерной особенностью скалярных произведений типа Соболева является, в частности, то, что они, как правило, содержат слагаемые, которые <<контролируют>> поведение соответствующих ортогональных полиномов в одной или нескольких точках числовой оси. Например, часто рассматривают скалярное произведение вида
%\begin{equation}\label{Shar_eq1}
%\langle f,g \rangle = \sum_{\nu=0}^{r-1} f^{(\nu)}(a)g^{(\nu)}(a) + \int\limits_{a}^{b} f^{(r)}(t)g^{(r)}(t)\rho(t)dt,
%\end{equation}
%в котором $f$ и $g$ --- функции, заданные на $[a,b]$ и непрерывно дифференцируемые там $r-1$ раз, для которых $f^{(r-1)}(x)$ и $g^{(r-1)}(x)$ абсолютно непрерывны и $f^{(r)}(x),\ g^{(r)}(x)\in L^2_\rho(a,b)$, где $L^2_\rho(a,b)$ --- пространство Лебега с весом $\rho(x)$.
%Следует отметить, что полиномы, ортогональные по Соболеву, по своим свойствам могут весьма существенно отличаться от обычных ортогональных на интервале полиномов. Например, в некоторых случаях оказывается так, что полиномы, ортогональные по Соболеву на интервале $(a,b)$, могут иметь нули, совпадающие с одним или с обоими концами этого интервала. Это обстоятельство имеет важное значение для некоторых приложений, в которых требуется, чтобы значения частичных сумм ряда Фурье функции $f(x)$ по рассматриваемой системе ортогональных полиномов совпали в концах интервала $(a,b)$ со значениями $f(a)$ и $f(b)$. Заметим, что обычные ортогональные с положительным на $(a,b)$ весом полиномы этим важным свойством не обладают. Скалярное произведение \eqref{Shar_eq1} имеет одну особую точку, а именно, точку $a$, в окрестности которой <<контролируется>>, поведение соответствующих полиномов, ортогональных по Соболеву. Это достигается за счет наличия в скалярном произведении \eqref{Shar_eq1} слагаемого вида $\sum\limits_{\nu=0}^{r-1}f^{(\nu)}(a)g^{(\nu)}(a)$.
%
%В настоящей работе, следуя \cite{Shar7}, мы рассмотрим дискретный аналог скалярного произведения \eqref{Shar_eq1} следующего вида
%\begin{equation}\label{Shar_eq2}
%\langle f,g \rangle = \sum_{k=0}^{r-1} \Delta^k f(0) \Delta^k g(0) + \sum_{j=0}^\infty \Delta^r f(j) \Delta^r g(j) \rho(j),
%\end{equation}
%где функции $f$ и $g$ заданы на множестве $\Omega=\{0,1,\ldots,\}$, $\rho=\rho(j)$ --- дискретная весовая функция, заданная на множестве $\Omega$. В случае, когда $r=0$ мы будем считать, что $\sum\limits_{k=0}^{r-1}\Delta^kf(0)\Delta^kg(0)=0$. При $r\ge1$ особой точкой в скалярном произведении \eqref{Shar_eq2} является $x=0$, в которой контролируется поведение соответствующих ортогональных по Соболеву полиномов дискретной переменной, благодаря присутствию в \eqref{Shar_eq2} выражения $\sum\limits_{k=0}^{r-1}\Delta^k f(0)\Delta^k g(0)$. Основное внимание будет уделено изучению свойств полиномов, ортогональных по Соболеву, порожденных классическими ортогональными полиномами Шарлье дискретной переменной.


%%%%%%%%%%%%%%%%%%
%%%%%%%%%%%%%%%%%%
%%%%%%%%%%%%%%%%%%
