%\section{Двухточечная краевая задача.}
\section{Условия существования положительного решения двухточечной краевой задачи для ОДУ 4 порядка.}

%\textbf{Улучшены достаточные условия существования положительного решения двухточечной краевой задачи для одного класса нелинейных обыкновенных дифференциальных уравнений четвертого порядка.}

\textit{ Доказано существование и единственность положительного решения для одного
класса нелинейных дифференциальных уравнений четвертого порядка. Предложен
также неитерационный численный метод нахождения положительного решения.
}


%\subsection{Предварительные сведения}

Исследованиям положительных решений краевых задач для нелинейных
дифференциальных уравнений посвящен ряд работ российских и зарубежных
математиков (см., например, \cite{krasnosel, pohojOvs, pohojVariaz, gidasSpruck, kuoShung, galahov, gaponenko} и цитированную в них литературу). Во многих из
них рассматриваются вопросы существования положительного решения, его поведения,
асимптотики и другие. Работ, посвященных  единственности положительного решения
и численным методам его построения, мало, особенно в случае сильно нелинейных
уравнений вида $ u=Au $ с выпуклым оператором $ A.$  Нами предпринята попытка
устранить указанный пробел.

\subsection{Формулировка и вспомогательные утверждения}

Рассматривается семейство двухточечных краевых задач:
\begin{equation}\label{eq.1.3.2.1_{i}}
{y_{i}}^{(4)}+x^m{\vert y_{i}\vert}^n=0, 0<x<1,
\end{equation}
\begin{equation}\label{eq.1.3.2.2_{i}}
y_{i}(0)={y_{i}}^{\prime}(0)={y_{i}}^{\prime\prime}(0)=0,
\end{equation}
\begin{equation}\label{eq.1.3.2.3_{i}}
{y_{i}}^{(i)}(1)=0,\ i=0,1,2,3,
\end{equation}
где $m\geq 0, n>1$- константы.

Очевидно, $y_i \equiv 0$ -- тривиальное решение задачи \eqref{eq.1.3.2.1_{i}} -- \eqref{eq.1.3.2.3_{i}}.
Под положительным решением задачи \eqref{eq.1.3.2.1_{i}}--\eqref{eq.1.3.2.3_{i}} понимается функция
$y_i \in C^4[0,1]$   положительная при $x\in (0,1)$, удовлетворяющая
уравнению \eqref{eq.1.3.2.1_{i}} и граничным условиям \eqref{eq.1.3.2.2_{i}}--\eqref{eq.1.3.2.3_{i}}.

Нами доказано существование и единственность положительного решения
задачи \eqref{eq.1.3.2.1_{i}} -- \eqref{eq.1.3.2.3_{i}}. Кроме того, предлагается численный метод построения этого
решения. Отметим, что существование положительного решения также можно доказать,
пользуясь методом  расслоения Похожаева С.И. \cite{pohojVariaz}. В качестве примеров приводится
положительное решение (в виде таблиц значений) задачи \eqref{eq.1.3.2.1_{i}} -- \eqref{eq.1.3.2.3_{i}} при
$m=0$, $n=4,$  построенное приведенным здесь методом.
Отметим, что в этом году продолжены исследования ученых Отдела по данной тематике, начатые ранее и опубликованные в работах \cite{abdurag1,abdurag2,abdurag3,abdurag4}.

Приведем здесь некоторые определения и утверждения, которые понадобятся нам в дальнейшем.


Пусть $A$ -- произвольное положительное число. Рассмотрим задачу Коши:

\begin{equation}\label{eq.1.3.2.1.1}
    y^{(4)}+x^m{\vert y\vert}^n=0,
\end{equation}
\begin{equation}\label{eq.1.3.2.1.2}
   y(0)=y^{\prime}(0)=y^{\prime\prime}(0)=0,
\end{equation}
\begin{equation}\label{eq.1.3.2.1.3}
         y^{\prime\prime\prime}(0)=A.
\end{equation}
Интегрируя уравнение \eqref{eq.1.3.2.1.1} с учетом начальных условий \eqref{eq.1.3.2.1.2}, \eqref{eq.1.3.2.1.3}, имеем
\begin{equation}\label{eq.1.3.2.1.4}
y^{\prime\prime\prime}(x)=A-\int_0^xs^m{\vert y(s)\vert}^n ds,
\end{equation}
\begin{equation}\label{eq.1.3.2.1.5}
y^{\prime\prime}(x)=Ax-\int_0^x(x-s)s^m{\vert y(s)\vert}^n ds,
\end{equation}
\begin{equation}\label{eq.1.3.2.1.6}
y^{\prime}(x)=
A\frac{x^2}{2}-\int_0^x\frac{(x-s)^2}{2}s^m{\vert y(s)\vert}^n ds,
\end{equation}
\begin{equation}\label{eq.1.3.2.1.7}
y(x)=
A\frac{x^3}{6}-\int_0^x\frac{(x-s)^3}{6}s^m{\vert y(s)\vert}^n ds.
\end{equation}

\begin{lemma}\label{AbEI:lemm1} При любом $A>0$  существует единственная
точка $x_3>0$ такая, что существует единственное решение $ y \in C^4[0,x_3] $ задачи Коши
\eqref{eq.1.3.2.1.1} -- \eqref{eq.1.3.2.1.3} такое, что $y^{\prime\prime\prime}(x_3)=0,
y^{\prime\prime\prime}(x)>0 $ при $ x \in (0,x_3) $  и
$ y^{\prime\prime\prime}(x)<0 $ при $ x>x_3 $.
\end{lemma}

%Доказательство Леммы 1
%
%\textbf{  Доказательство.} Так как $ y^{\prime\prime\prime}(0)=A>0 $ , то
%$ y^{\prime\prime\prime}(x)>0 $ в некоторой окрестности $ (0,\delta) $   нуля.
%Из уравнения \eqref{eq.1.3.2.1.1} следует, что $  y^{(4)}(x) \leq 0 $.  Следовательно,
%$ y^{\prime\prime\prime}(x)- $ невозрастающая функция.
%
%Предположим противное, т.е. не существует точки $x,$ в которой
%$ y^{\prime\prime\prime}(x)=0.$ Тогда $ y^{\prime\prime\prime}(x)>0 $  при всех
%$ x>0. $    В силу \eqref{eq.1.3.2.1.4} из \eqref{eq.1.3.2.1.7} м \eqref{eq.1.3.2.1.6} следует
%$$
%y(x)\geq \frac{x^3}{6}(A-\int_0^x s^m{\vert y(s) \vert}^n ds)=
%\frac{x^2}{2}y^{\prime\prime\prime}(x)>0,
%$$
%$$
%y^{\prime}\geq \frac{x^2}{2}(A-\int_0^x s^m{\vert y(s) \vert}^n ds)=
%\frac{x^3}{6}y^{\prime\prime\prime}(x)>0.
%$$
%
%Это значит $ y(x)-$  положительная всюду на $ (0,\infty),$ возрастающая функция.
%Из \eqref{eq.1.3.2.1.7} при $ x>0 $    имеем
%  \begin{equation}\label{eq.1.3.2.1.8}
%Ax^3>\int_0^x(x-s)^3s^m{\vert y(s) \vert}^n ds.
%\end{equation}
%Пусть $ x_0\geq \delta.$    Поскольку $ y(x)$   возрастающая функция,
% то в силу \eqref{eq.1.3.2.1.8} при $ x>x_0 $   имеем
%
%\begin{equation}\label{eq.1.3.2.1.9}
% Ax^3>\int_{x_0}^x(x-s)^3s^m{\vert y(s) \vert}^n ds \geq x_0^my^n(x_0)
% \frac{(x-x_0)^4}{4}.
%\end{equation}
%Отсюда при $ x=Mx_0(M>1) $  имеем
%$$
%AM^3>{x_0}^{m+1} \frac{(M-1)^4}{4}y^n(x_0)
%$$
%или
%\begin{equation}\label{eq.1.3.2.1.10}
%1>\frac{{x_0}^{m+1}(M-1)^4}{4AM^3}y^n(x_0).
%\end{equation}
%Так как $ y^n(x_0)>0 $  и $ A>0 $,  то \eqref{eq.1.3.2.1.10}  при $ M\to {+\infty} $
%приводит к противоречию. Следовательно, существует точка $ x_3 $  такая, что
%$ y^{\prime\prime\prime}(x_3)=0.$ Пусть $ \delta- $ произвольное число из $ (0,x_3).$
%Из \eqref{eq.1.3.2.1.4} имеем
%$$
% y^{\prime\prime\prime}(x_3-\delta)=
%A-\int_0^{x_3}s^m{\vert y(s) \vert}^n ds+
%\int_{x_3-\delta}^{x_3}s^m{\vert y(s) \vert}^n ds=
%\int_{x_3-\delta}^{x_3}s^m{\vert y(s) \vert}^n ds>0.
%$$
%Пусть теперь $\delta$ -- произвольное положительное число. Снова из \eqref{eq.1.3.2.1.4} имеем
%
%$$
%y^{\prime\prime\prime}(x_3+\delta)=
%A-\int_0^{x_3}s^m{\vert y(s) \vert}^n ds-
%\int_{x_3}^{x_3+\delta}s^m{\vert y(s) \vert}^n ds=
%-\int_{x_3-\delta}^{x_3}s^m{\vert y(s) \vert}^n ds<0.
%$$
%Из уравнения \eqref{eq.1.3.2.1.1} и равенств \eqref{eq.1.3.2.1.4}--\eqref{eq.1.3.2.1.7} следует ограниченность
%$ \left\|y\right\|_{C^4[0,x_3]} $. Следовательно, существует единственное решение
%задачи Коши \eqref{eq.1.3.2.1.1}-\eqref{eq.1.3.2.1.3} на $ [0,x_3].$  Лемма доказана.$ \square $

\begin{lemma}\label{AbEI:lemm2} При любом $ A>0 $ существует единственная точка
$ x_2>0 $ такая, что существует единственное решение $ y \in C^4[0,x_2] $ задачи
Коши \eqref{eq.1.3.2.1.1} -- \eqref{eq.1.3.2.1.3} такое, что $ y^{\prime\prime}(x_2)=0,\quad y^{\prime\prime}(x)>0$
при $ x \in (0,x_2) $ и $ y^{\prime\prime}(x)<0 $  при $ x>x_2. $
\end{lemma}

%Доказательство Леммы 1
%
%{\textbf{Доказательство.}   Так как по лемме \ref{AbEI:lemm1}
%$ y^{\prime\prime\prime}(x)>0 $    при $ x \in (0,x_3),$
%то $ y^{\prime\prime}(x) $ возрастает  при $ x \in (0,x_3).$
%Поскольку $ y^{\prime\prime}(0)=0, $  то $ y^{\prime\prime}(x)>0, $
%на  $ (0,x_3). $
%
%Предположим противное, т.е.  не существует точки $x,$
%в которой $ y^{\prime\prime}(x)=0. $ Тогда $ y^{\prime\prime}(x)>0 $
%при всех $ x>0 $ . Из \eqref{eq.1.3.2.1.7} и \eqref{eq.1.3.2.1.6} в силу \eqref{eq.1.3.2.1.5} при $ x>0 $  имеем
%$$
%y^{\prime}\geq \frac{x}{2}(Ax-\int_0^x (x-s)s^m{\vert y(s) \vert}^n ds)=
%\frac{x}{2}y^{\prime\prime}(x)>0,
%$$
%$$
%  y(x)\geq \frac{x^2}{6}(Ax-\int_0^x (x-s)s^m{\vert y(s) \vert}^n ds)=
%\frac{x^2}{6}y^{\prime\prime}(x)>0.
%$$
%
%Следовательно, $ y(x)>0 $ и возрастает при $ x>0 $.
%Пусть $ x_0\geq x_3, $ где  $ x_3 $ определяется леммой \ref{AbEI:lemm1}.
%Тогда \eqref{eq.1.3.2.1.7} в силу \eqref{eq.1.3.2.1.8} при $ x>x_0 $ приводит к неравенству \eqref{eq.1.3.2.1.9},
%далее --- к неравенству \eqref{eq.1.3.2.1.10}, из которого получаем противоречие. Следовательно,
%существует точка $ x_2 $ такая, что $ y^{\prime\prime}(x_2)=0. $
%Очевидно, $ x_2\geq x_3. $  Так как $ (y^{\prime\prime)})^{\prime\prime}(x)=y^{(4)}(x)<0 $
%при $ x>0, $  то $ y^{\prime\prime}(x)- $ выпуклая вниз функция. Поэтому
%точка $ x_2, $  в которой $ y^{\prime\prime}(x_2)=0, $    единственна.
%Следовательно,$ y^{\prime\prime}(x)>0 $   при $ x \in (0,x_2) $  и
%$ y^{\prime\prime}(x)<0 $  при $ x>x_2. $ Ограниченность $ \left\|y\right\|_{C^4[0,x_2]} $
%следует из \eqref{eq.1.3.2.1.1} и \eqref{eq.1.3.2.1.4}-\eqref{eq.1.3.2.1.7}. Следовательно,существует единственное решение задачи
%Коши \eqref{eq.1.3.2.1.1}-\eqref{eq.1.3.2.1.3} на $ [0,x_2] $. Лемма доказана.$ \square $


{\begin{lemma}\label{AbEI:lemm3} При любом $ A>0 $  существует единственная
точка $ x_1 $   такая, что существует единственное решение $ y \in C^4[0,x_1] $ задачи
Коши \eqref{eq.1.3.2.1.1} -- \eqref{eq.1.3.2.1.3} такое, что $ y^{\prime}(x_1)=0,\quad y^{\prime}(x)>0  $    при
$ x \in (0,x_1) $ и $ y^{\prime}(x)<0 $ при $ x>x_1, $ где $ y(x)-$ решение
задачи \eqref{eq.1.3.2.1.1} --\eqref{eq.1.3.2.1.3}.
\end{lemma}

%Доказательство Леммы 3
%
%{\textbf{  Доказательство.}    Так как по лемме   \ref{AbEI:lemm1} $ y^{\prime\prime}(x)>0 $
%при $ x \in (0,x_2), $    то $ y^{\prime}(x) $ возрастает при
%$ x \in (0,x_2). $ Поскольку $ y^{\prime}(0)=0, $   то  $ y^{\prime}(x)>0 $
%на $  (0,x_2). $
%
%Предположим противное, т.е. не существует точки $ x, $
% в которой  $ y^{\prime}(x)=0. $    Тогда $ y^{\prime}(x)>0 $   при всех
%$ x>0. $ Следовательно, $ y(x) $ возрастает при $ x>0 $ .  Поскольку $ y(0)=0, $
%то $ y(x)>0 $   при $ x>0. $  Пусть $ x_0 \geq x_2, $   где $ x_2 $
%определяется леммой \ref{AbEI:lemm1}. Тогда  так же, как в лемме \ref{AbEI:lemm1},  приходим к неравенству
%\eqref{eq.1.3.2.1.9}. Полагая в нем $ x=Mx_0(M>1), $ получим неравенство \eqref{eq.1.3.2.1.10}, из которого
%получаем противоречие.  Следовательно, существует точка $ x_1 $   такая,
%что $ y^{\prime}(x_1)=0. $  Очевидно, $ x_1\geq x_2 \geq x_3.$  Так как
%$ (y^{\prime}(x))^{\prime\prime}=y^{\prime\prime\prime}(x)<0 $ при $ x>x_3 $  по лемме \ref{AbEI:lemm1},
%то $ y^{\prime}(x)- $  выпуклая вниз при $ x>x_3 $   функция. Тем более она
%выпукла вниз при $ x>x_1\geq x_2 \geq x_3. $ Поэтому  точка $ x_1 $ , в которой
%$ y^{\prime}(x_1)=0, единственна, $ \quad $ y^{\prime}(x)>0 $  при
%$ x \in (0,x_1) $ и  $ y^{\prime}(x)<0 $    при $ x>x_1 $  .
%Ограниченность $ \left\|y\right\|_{C^4[0,x_1]} $
%следует из \eqref{eq.1.3.2.1.1} и \eqref{eq.1.3.2.1.4}--\eqref{eq.1.3.2.1.7}. Следовательно,существует единственное решение задачи
%Коши \eqref{eq.1.3.2.1.1}--\eqref{eq.1.3.2.1.3} на $ [0,x_1] $.  Лемма доказана.$ \square $

\begin{lemma}\label{AbEI:lemm4} При любом $ A>0 $   существует единственная точка
$ x_0>0 $ такая, что существует единственное решение $ y \in C^4[0,x_0] $
задачи Коши \eqref{eq.1.3.2.1.1}--\eqref{eq.1.3.2.1.3} такое, что  $ y(x_0)=0, \quad y(x)>0 $  при $ x \in (0,x_0). $
\end{lemma}

%Доказательство Леммы 4
%
%{\textbf{Доказательство.}  Так как по лемме \ref{AbEI:lemm1} $ y^{\prime}(x)>0 $
%при $ x \in (0,x_1) $ и $ y(0)=0,$ то $ y(x)>0 $ и возрастает при $ x \in (0,x_1).$
%
%Предположим противное, т.е. $ y(x)>0 $  при  всех $ x>0.$
%Так как  по лемме \ref{AbEI:lemm1}.2 $ y^{\prime\prime}(x)<0 $ при $ x>x_2 $ ,
%то $ y^{\prime\prime}(x)<0 $  и при  $ x>x_1 \geq x_2 $ , т.е. $ y(x)-$  выпуклая вниз
%функция при $ x>x_1. $  Следовательно, существует единственная точка $ x_0 $
%такая, что $ y(x_0)=0,\quad y(x)>0 $   при $ x \in (0,x_0) $   и $ y(x)<0 $
%при $ x>x_0 $.  Ограниченность $ {\left \| y \right \|}_{C^4[0,x_0]} $
%следует из \eqref{eq.1.3.2.1.1} и \eqref{eq.1.3.2.1.4}--\eqref{eq.1.3.2.1.7}. Следовательно,существует единственное решение задачи
%Коши \eqref{eq.1.3.2.1.1}-\eqref{eq.1.3.2.1.3} на $ [0,x_0] $.  Лемма доказана.$ \square $

\subsection{Cуществование и единственность положительного решения}

Следуя Ц.На \cite{CeNa}, введем линейную группу преобразований
\begin{equation}\label{eq.1.3.2.2.1}
\begin{cases}
x=A_i^\alpha \overline {x},\\
y_i=A_i^\beta \overline{y_i}, \quad i=0,1,2,3,
\end{cases}
\end{equation}
где $ \alpha,\beta- $ константы, подлежащие определению,$ A_i- $  положительный
параметр преобразо\-вания.В новых координатах $ (\overline {x},\overline{y_i}) $
уравнение $ (1_i) $ примет вид
\begin{equation}\label{eq.1.3.2.2.2}
A_i^{\beta-4 \alpha} {\overline{y_i}}^{(4)}+
A_i^{\alpha m+\beta n}{\overline {x}}^m {\overline{y_i}}^n=0.
\end{equation}
Выберем константы $ \alpha $ и $ \beta $  так, чтобы это уравнение не зависело
от параметра $ A_i: $
\begin{equation}\label{eq.1.3.2.2.3}
\beta-4 \alpha=\alpha m+\beta n.
\end{equation}
Тогда из \eqref{eq.1.3.2.2.2} имеем
\begin{equation}\label{eq.1.3.2.2.4}
{\overline{y_i}}^{(4)}+{\overline {x}}^m {\overline{y_i}}^n=0, \quad i=0,1,2,3.
\end{equation}
т.е уравнение \eqref{eq.1.3.2.1_{i}} оказалось инвариантным относительно преобразования \eqref{eq.1.3.2.2.1}.

Обозначим через $ A_i $  недостающее начальное условие в задаче \eqref{eq.1.3.2.1_{i}} -- \eqref{eq.1.3.2.3_{i}}:
\begin{equation}\label{eq.1.3.2.2.5}
y_i^{\prime\prime\prime}(0)=A_i.
\end{equation}
Это условие в координатах $ (\overline {x},\overline{y_i}) $   запишется в виде
\begin{equation}\label{eq.1.3.2.2.6}
A_i^{\beta-3 \alpha}{\overline{y_i}}^{\prime\prime\prime}(0)=A_i.
\end{equation}
и  оно не будет зависеть от параметра $ A_i, $  если
\begin{equation}\label{eq.1.3.2.2.6}
\beta-3 \alpha=1.
\end{equation}
Тогда из \eqref{eq.1.3.2.2.6} получим
\begin{equation}\label{eq.1.3.2.2.8}
{\overline y}_i^{\prime\prime\prime}(0)=1.
\end{equation}
Решая систему \eqref{eq.1.3.2.2.3}, находим
\begin{equation}\label{eq.1.3.2.2.9}
\alpha=-\frac{n-1}{m+3n+1},
\end{equation}
\begin{equation}\label{eq.1.3.2.2.10}
 \beta=\frac{m+4}{m+3n+1}.
\end{equation}
В силу \eqref{eq.1.3.2.2.4}, \eqref{eq.1.3.2.2.8} и того, что условия \eqref{eq.1.3.2.2_{i}} в новых
координатах $ (\overline {x},\overline{y_i}) $ будут иметь вид
$ \overline{y_i}(0)={\overline{y_i}}^{\prime}(0)=
{\overline{y_i}}^{\prime\prime}(0)=0,$
приходим к следующей  задаче Коши для $ \overline{y_i}(\overline {x}): $
\begin{equation}\label{eq.1.3.2.2.11}
 {\overline{y_i}}^{(4)}+{\overline {x}}^m {\overline{y_i}}^n=0,
\end{equation}
\begin{equation}\label{eq.1.3.2.2.12}
\overline{y_i}(0)={\overline{y_i}}^{\prime}(0)=
{\overline{y_i}}^{\prime\prime}(0)=0,
\end{equation}
\begin{equation}\label{eq.1.3.2.2.13}
{\overline y}_i^{\prime\prime\prime}(0)=1.
\end{equation}


Из лемм \ref{AbEI:lemm1}--\ref{AbEI:lemm4} с $ A=1 $ следует, что существует единственная
 точка $ \overline {x_{i0}},i=0,1,2,3, $  такая, что решение
$ \overline{y_i}(\overline {x}) $ задачи Коши \eqref{eq.1.3.2.2.11}-\eqref{eq.1.3.2.2.13} на
$ [0,\overline {x_{i0}}] $ определяется единственным образом, удовлетворяет условию
$ {\overline{y_i}}^{(i)}(\overline {x_{i0}})=0, i=0,1,2,3, $ и
$ {\overline{y_i}}^{(i)}(\overline {x})>0 $ при $ \overline {x} \in (0,x_{i0}). $
Выберем параметр $ A_i $  в \eqref{eq.1.3.2.2.1} так, чтобы $ x=1 $ при
$  \overline {x}=\overline {x_{i0}},i=0,1,2,3, $ т.е. из равенства
$ 1=A_i^{\alpha}\overline {x_{i0}}. $
Отсюда положительный параметр $ A_i $ определяется однозначно:
\begin{equation}\label{eq.1.3.2.2.14}
A_i=(\overline {x_{i0}})^{-\frac{1}{\alpha}}, i=0,1,2,3,
\end{equation}
где $ \alpha $ определяется равенством \eqref{eq.1.3.2.2.9}. Поэтому задача \eqref{eq.1.3.2.1_{i}} -- \eqref{eq.1.3.2.3_{i}}
имеет единственное положительное решение $ y_i \in C^4[0,1] $. Доказана

\begin{theorem}\label{AbEI:th1} Задача \eqref{eq.1.3.2.1_{i}} -- \eqref{eq.1.3.2.3_{i}} имеет единственное
положительное решение $y_i \in C^4[0,1]$, $i=0,1,2,3. $
\end{theorem}

\begin{remark}\label{AbEI:remrk1}  Отрезок $[0,a]$ с  произвольным
положительным $ a $ заменой $ t=\frac{x}{a} $  сводится к отрезку $[0,1]$.
Поэтому сформулированная здесь теорема имеет место  для любого отрезка
$ [0,a] $  с заменой условия $ (3_i) $ на $ {y_i}^{(i)}(a)=0, i=0,1,2,3.$
\end{remark}

\subsection{Численный метод построения положительного решения}

Приведенные выше рассуждения позволяют сформулировать алгоритм построения единственного положительного решения задачи \eqref{eq.1.3.2.1_{i}} -- \eqref{eq.1.3.2.3_{i}} состоящий из следующих шагов:

1. Вычисляем $ \alpha $  и $ \beta $   по формулам \eqref{eq.1.3.2.2.9}, \eqref{eq.1.3.2.2.10};

2. Решаем каким-либо численным методом, например, методом Рунге-Кутта
четвертого порядка задачу Коши \eqref{eq.1.3.2.2.11}--\eqref{eq.1.3.2.2.13}, начиная с $ \overline {x}=0 $
до тех пор, пока по одной из лемм \ref{AbEI:lemm1}.1-1.4 не выполнится равенство
$ {\overline{y_i}}^{(i)}(\overline {x_{i0}})=0 $   с
$ \overline {x_{i0}}>0,\quad i=0,1,2,3; $

3. Вычисляем $ A_i $ по формуле \eqref{eq.1.3.2.2.14};

4. Находим решение по формулам \eqref{eq.1.3.2.2.1}.

\begin{remark}\label{AbEI:remrk2}  Для уменьшения вычислительной погрешности,
связанной с вычислением степени   $ A_i $ пункт 4 можно заменить пунктом

$4'$. Решаем задачу Коши \eqref{eq.1.3.2.1_{i}} -- \eqref{eq.1.3.2.2_{i}} тем же численным методом,
что и в пункте 2, начиная с $x=0$  до $x=1$.
\end{remark}

В качестве примера приведем таблицу значений положительного решения задачи
\eqref{eq.1.3.2.1_{i}} -- \eqref{eq.1.3.2.2_{i}} (случай $i=0$) при $m=0$, $n=4$, полученного указанным здесь методом.



\textbf{Положительное решение задачи \eqref{eq.1.3.2.1_{i}} -- \eqref{eq.1.3.2.3_{i}} при $ m=0, n=4 $}


\begin{tabular}[c]{|l|l|l|l|l|l|l|l|l|l|l|l|}
\hline
x& 0,00 &0,1  & 0,2 & 0,3 & 0,4 & 0,5 & 0,6 & 0,7 & 0,8 & 0,9 & 1,0 \\
\hline
y& 0,00 & 0,04 & 0,31 & 1,05 & 2,49 & 4,86 &  8,39  & 13,18
& 18,54 & 19,65 & 0,00 \\
\hline
\end{tabular}
\bigskip



\subsection{ Заключение }

Доказано, что каждая задача из семейства двухточечных краевых задач
$$
{y_{i}}^{(4)}+x^m{\vert y_{i}\vert}^n=0, 0<x<1,
 $$
$$
y_{i}(0)={y_{i}}^{\prime}(0)={y_{i}}^{\prime\prime}(0)=0,
$$
$$
{y_{i}}^{(i)}(1)=0,\ i=0,1,2,3,
$$
где $m\geq 0, n>1$ -- константы, имеет единственное положительное решение и предложен
неитерационный численный метод его построения. Для данного класса уравнений результат
о единственности является новым. Также нов, предложенный здесь неитерационный
численный метод построения положительного решения.
