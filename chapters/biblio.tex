\begin{thebibliography}{1} %% здесь библиографический список

% chapter 2
\input chapters/chapter2/biblio.tex

%\bibitem{filosofyNewestdict}
%{Грицанов} А.А.~и др.
%\newblock {\em Новейший философский словарь}.
%\newblock Мн.: Книжный Дом., 2003.

%1.
\bibitem{shii1}{Шарапудинов И.И.}
Предельные ультрасферические ряды и их аппроксимативные свойства// Матем. заметки. 2013. Том. 94. Вып. 2.

%2.
\bibitem{shii2}{Шарапудинов И.И.}
Специальные  ряды по ультрасферическим полиномам и их аппроксимативные свойства// Известия РАН. Серия математическая. (Принята к печати).

%3.
\bibitem{dedus3}{Дедус Ф.Ф., Махортых С.А., Устинин М.Н., Дедус А.Ф.}
Обобщенный спектрально-аналитический метод обработки информационных массивов. Задачи анализа изображений и распознавания образов. -М. "Машиностроение". 1999.

%4.
\bibitem{pash4}{Пашковский С.}
Вычислительные применения многочленов и рядов Чебышева:  М.: Наука. 1983. - 384 с.

%5.
\bibitem{arush5}{Арушанян О.Б., Волченскова Н.И., Залеткин С.Ф.}
О вычислении коэффициентов рядов Чебышева для решений обыкновенных дифференциальных уравнений// Сибирские электронные  математические известия. 2011. Том 8. С.273 -- 283.

%6.
\bibitem{tref6}{Trefethen L.N.}
Spectral methods in Matlab. SIAM, Philadelphia, 2000.

%7.
\bibitem{tref7}{Trefethen L.N.}
Finite difference and spectral methods for ordinary and partial differential equation. Cornell University. 1996.

%8.
\bibitem{muku8}{Mukundan R., Ramakrishnan K.R.}
Moment functions in image analysis. Theory and Applications.World Scientific. Publishing Co.Pte.Ltd. 1998.

%9.
\bibitem{malvarSign}{Malvar H.S.}
Signal processing with lapped transforms. Artech House. Boston{ $\cdot$} London. 1992.

\bibitem{Sego}{Сеге Г.}
Ортогональные многочлены.  Физматлит. М. 1962.

\bibitem{badCow}{Бадков В.М.}
Оценки функции Лебега и остатка ряда Фурье-Якоби //СМЖ, 1968. Т. IX, N 6. Стр. 1263 -- 1283.

\bibitem{natanson}{Натансон И.П.}
Конструктивная теория функций. Гостехиздат.М.-Л. 1949.

%====chapter 1======
\bibitem{shii-lpx}
Шарапудинов~И.И.
{О топологии пространства $L^{p(t)}([0,1])$~/\!/ Матем. заметки.---1979.---Т.~26, \No~4.---С.~613--632.}

\bibitem{shii-legendre} И.И. Шарапудинов. О базисности системы полиномов
Лежандра в пространстве $L^{p(x)}(-1,1)$  переменным показателем
 $p(x)$// Матем. сб. 200:1 (2009), 137--160.

\bibitem{shii-quad}
Шарапудинов И.И. Приближение функций в метрике
пространства $ L^{p(x)}([a,b])$ и квадратурные формулы
//Constructive function theory'81. Procedings of the International
Conference on Constructive Function Theory. Varna, June 1-5, 1981.
С. 189--193.


\bibitem{shii-monog-2012}
Шарапудинов~И.И.
{Некоторые вопросы теории приближений в пространствах Лебега с переменным показателем.---Владикавказ: ЮМИ ВНЦ РАН и РСО-А, 2012.---С.~270.}

\bibitem{diening-book-2011}
L.~Diening, P.~Harjulehto, P.~Hasto, Ruzicka, M.
{Lebesgue and Sobolev Spaces with Variable Exponents.---Berlin: Springer, 2011.---С.~509.}

\bibitem{shii-haar-basis}
Шарапудинов~И.И.
{О базисности системы Хаара в пространстве $L^{p(t)}([0,1])$ и принципе локализации в среднем~/\!/ Матем. сб.---1986.---Т.~130(172), \No~2(6).---С.~275--283.}

\bibitem{shii-approx-lpx-2007}
Шарапудинов~И.И.
{Некоторые вопросы теории приближений в пространствах $L^{p(x)}$~/\!/ Anal. Math.---2007.---Т.~33, \No~2.---С.~135--153.}

\bibitem{shii-smesh-ultra} Шарапудинов И.И. Смешанные ряды по ультрасферическим  полиномам и их аппроксимативные свойства // Матем. сб. 194:3 (2003), 115--148.

\bibitem{shii-varsmooth-legendre}
Шарапудинов  И.И. Приближение функций с переменной гладкостью суммами Фурье–Лежандра // Матем. сб., 191:5 (2000), 143–160.

\bibitem{shii-smesh-legendre}
Шарапудинов И.И. Аппроксимативные свойства смешанных рядов по полиномам Лежандра на классах $W^r$ // Матем. сб., 197:3 (2006), 135–154

\bibitem{izuki-lpx}
Izuki~M.
{Wavelets and modular inequalities in variable $L^p$ spaces~/\!/ Georgian Math. J.---2008.---Т.~15, \No~2.---С.~281--293.}

\bibitem{izuki-lpx-pre}
Izuki~M.
{Wavelets and modular inequalities in variable $L^p$ spaces. 2007. Preprint.}

\bibitem{mmg-haar2d}
Магомед-Касумов~М.Г.
{Сходимость прямоугольных сумм Фурье-Хаара в пространствах Лебега с переменным показателем $L^{p(x,y)}$~/\!/ Изв. Сарат. ун-та. Нов. сер. Сер. Математика. Механика. Информатика.---2013.---Т.~13, \No~1(2).---С.~76--81.}

\bibitem{shii-conv}
Шарапудинов~И.И.
{О равномерной ограниченности в $L^p$ $(p=p(x))$ некоторых семейств операторов свертки~/\!/ Матем. заметки.---1996.---Т.~59,\No~2.---С.~291--302.}


\bibitem{diening-muckenhoupt}
Diening~L., Hasto~P.
{Muckenhoupt weights in variable exponent spaces. 2008. Preprint.}

\bibitem{cruz-maxop}
Cruz-Uribe~D., Diening~L., Hasto~P.
{The maximal operator on weighted variable Lebesgue spaces~/\!/ Fract. Calc. Appl. Anal.---2011.---Т.~14, \No~3.---С.~361--374.}


\bibitem{kashin}
Кашин~Б.С., Саакян~А.А.{Ортогональные ряды.---М.: Изд-во АФЦ, 1999.---С.~560.}


\bibitem{vulih-fan}
Вулих~Б.З.
{Введение в функциональный  анализ.---М.: Наука, 1967.---С.~416.}

\bibitem{sobol-quadform}
Соболь~И.М.
{Многомерные квадратные формулы и функции Хаара.---М.: Наука, 1969.---С.~288.}

\bibitem{mmg-haar-basis}
Магомед-Касумов М.Г. Базисность системы Хаара в весовых пространствах Лебега с переменным показателем // Владикавказский математический журнал. 2014. Том 16, вып. 3, с. 38-46.

\bibitem{cruz-book-2013}
Cruz-Uribe D., Fiorenza A.
{Variable Lebesgue Spaces: Foundations and Harmonic Analysis. – Springer, 2013. P.~312. DOI 10.1007/978-3-0348-0548-3}

\bibitem{shii-haarspeed}
Шарапудинов И.И.
{Приближение функций из пространств Лебега и Соболева с переменным показателем суммами Фурье-Хаара // Математический сборник. 2014. Т.~205, \No~2. С.~145-160. DOI: 10.4213/sm8274}

\bibitem{shii-sinkx2D}
Шарапудинов И.И.
{Некоторые специальные двумерные ряды по системе $\{\sin x\sin nx\}_{n=1}^\infty$ и их аппроксимативные свойства // Изв. Сарат. ун-та. Нов. сер. Сер. Математика. Механика. Информатика. 2014. Т. 14, вып. 4, часть 1. С. 407-412}
%{Двумерные  специальные ряды по системе $\{\sin x\sin nx\}_{n=1}^\infty$  и их аппроксимативные свойства // Математический сборник. 2014. Т.~205, \No~2. С.~145-160. DOI: 10.4213/sm8274}

\bibitem{guven-trigapp}
Guven,Israfilov.
{Trigonometric approximation in generalized lebesgue spaces Lp(x)~//~J. Math. Inequ. 2010. Т. 4. С. 285-299.}

\bibitem{shakh-conv}
Шах-Эмиров Т.Н.
{О равномерной ограниченности семейства операторов Стеклова в весовых пространствах Лебега с переменным показателем // Вестник ДНЦ РАН. 2014 (принята к печати).}

\bibitem{shakh-conv-2013}
Шах-Эмиров Т.Н.
{О равномерной ограниченности в $L^{p(x)}_{2\pi}$ некоторых семейств интегральных операторов свертки // Вестник дагестанского научного центра. 2013. 51. стр.13-17.}


\bibitem{mmg-haarspeed}
Магомед-Касумов М.Г.
Приближение функций суммами Хаара в весовых пространствах Лебега и Соболева с переменным показателем // Изв. Сарат. ун-та. Нов. сер. Сер. Математика. Механика. Информатика. 2014. Т. 14, вып. 3. с. 295-304.

\bibitem{mmg-classic-vallee-pussen}
Магомед-Касумов М.Г.
Аппроксимативные свойства классических средних Валле-Пуссена для кусочно гладких функций // Вестник ДНЦ РАН. 2014. Вып. 54, с. 5-11.

\bibitem{samko-denseness-2000}
Samko S.G.
{Denseness of $C_0^\infty(R^n)$ in generalized Sobolev Spaces $W^{m,p(x)}(R^n)$ // Intern. Soc. for Analysis, Applic. and Comput. 2000. v. 5. "Direct and Inverse Problems of Math. Physics".
Ed. by R. Gilbert, J. Kajiwara and Yongzhi S. Xu. Kluwer Acad. Publ. pp. 333-342.}

%Абдурагимова литра
\bibitem {krasnosel}
Красносельский М.А. Положительные решения операторных уравнений. ~М., 1962. 394~с.

\bibitem {pohojOvs}
Похожаев С.И. Об одной задаче Овсянникова. ~// ПМТФ. 1989.\No\,2. C. ~5--10.

\bibitem {pohojVariaz}
Похожаев С.И. Об одном конструктивном методе вариационного исчисления.~// ДАН СССР. 1988. T.~298. \No\, 6. C.~1330-1333.

\bibitem {pohojIntSolve}
Похожаев С.И. О целых радиальных решениях некоторых квазилинейных эллиптических  уравнений // Математический сборник. 1992. Т. 83. №11. С. 3-18.

\bibitem {gidasSpruck}
Gidas B., Spruck  J. ~// Global and local behavior of positive solutions
of nonlineare elliptic equations. 1982. V.~ 4. P.~525--598.

\bibitem {kuoShung}
Kuo-Shung Cheng and Jenn-Tsann Lin.~// On the elliptic equations $\Delta u=K(x)u^{\alpha}$ and $\Delta u=K(x)\exp^{2u}$.
 ~// Transactions of American mathematical society. 1987. V.~304.
\No\, 2. P.~633--668.

\bibitem {galahov}
Галахов Е.И. Положительные решения квазилинейного эллиптического уравнения.
 ~// Математические заметки. 2005. Т.~78. Вып.~ 2. C. ~202--211.

\bibitem {gaponenko}
Гапоненко Ю.Л. О положительных решениях нелинейных краевых задач.
~// Вестник Московского университета, серия 15. Вычислительная математика и
кибернетика. 1983. \No\, 4. C.~8--12.

\bibitem {abdurag1}
Абдурагимов Э.И. О единственности положительного решения одной нелинейной
двухточечной краевой задачи. ~// Изв.вузов. Математика. 2002. \No\,6. C. ~3--6.

\bibitem {abdurag2}
Абдурагимов Э.И. Положительное решение двухточечной краевой задачи для
одного нелинейного оду четвертого порядка. ~//Дагестанский математический
сборник. 2005. Т. ~1. C. ~7--12.

\bibitem {abdurag3}
Абдурагимов Э.И. О положительном решении двухточечной краевой задачи  для
одного нелинейного оду четвертого порядка. ~// Материалы Международной
конференции. Махачкала. 2005. C.~12--13.

\bibitem {abdurag4}
Абдурагимов Э.И. Положительное  решение двухточечной краевой задачи
для одного нелинейного оду четвертого порядка.~// Изв.вузов.
Математика. 2006. \No\, 8. C. ~3--6.

\bibitem {CeNa}
На Ц. Вычислительные методы решения прикладных граничных задач.
М.: Мир, 1982. 296~с.

\bibitem {DancerNorman}
Dancer E. Norman, Shi Junping. Uniqueness and nonexistence of positive solutions to semipositive problems. // London Math. Soc. 2006. V. 38. №6. P. 1033-1044.

\bibitem {KavanoNichiro}
Kavano Nichiro, Satsuma Junkichi, Youtsutani Shoji. On the positive solution of an emden-type elliptic equation.  // Proc. Jap. Acad. 1985. Ser A. V. 61. №6. P. 186-189.

\bibitem {JiangJu}
Jiang Ju. On radially symmetric solutions to singular nonlinear Dirichlet problems. // Nonlinear Anal. Theory, Methods and Applications. 1995. V. 24. P. 159-163.

\bibitem {bahvalov}
Бахвалов Н.С. Численные методы. Т. I; М.: «Наука», 1973.

\bibitem {abdurag5}
Абдурагимов Э.И. О единственности положительного радиально-симметричного решения задачи Дирихле в шаре для одного нелинейного дифференциального уравнения второго порядка // Изв. Вузов. Математика. 2008. №12. С. 3-6.

\bibitem {abdurag6}
Абдурагимов Э.И. Положительное решение двухточечной краевой задачи для одного нелинейного ОДУ второго порядка со степенным ростом. // Вестник ДГУ, выпуск 6.- 2011.-с. 104-110.

%===

\bibitem {bailu}
Zhanbing Bai, Haishen Lu. Positive solutions for boundary value problem of nonlinear fractional differential equation. // J.Math.Anal.Appl, 2005,  v.311,  p. 495-505.

\bibitem {wangwang}
Changyou Wang, Ruifang Wang, Shu Wang and Chunde Yang. Positive solution of singular Boundary Value Problem for a nonlinear Fraction Differential Equation. // Hindawi Publishing Corporation Boundary Value Problems. 2011,  p.1-12.

\bibitem {zhangS}
S. Zhang. Existence of solution for a boundary value problem of fractional order. // Acta Mathematica Scientis, 2006, V.26, №2, p.220-228.


\bibitem {qiubai}
T. Qiu and Z. Bai. Existence of positive solutions for singular fractional differential equations. // Electronic Journal of Differential Equations, 2008, V.149, p.1-9.

\bibitem {caballero}
J. Caballero Mena, J. Harjani and K. Sadarangani. Existence and uniqueness of positive and nondecreasing solution for a class of singular fractional boundary value problems. // Boundary Value Problems, 2009, v.2009, 10 pages.

\bibitem {changnieto}
Y.-K. Chang and J.J. Nieto. Some new existence results for fractional differential inclusions with boundary conditions.// Mathematical and Computer Modelling, 2009, V.58, №9, p.1838-1843.


\bibitem {shangSQ}
S.Q. Shang. Positive solutions to singular boundary value problem for nonlinear fractional differential equation. // Computers and Mathematics with Applications, 2010, v.59, №3, p.1300-1309.

\bibitem {beyb}
Бейбалаев В.Д. О численном решении задачи Дирихле для уравнения Пуассона с производными дробного порядка.// Вест. Сам. гос. техн. ун-та. Сер. Физ.-мат. Науки, 2012, т. 27, №2, с.183-187.

\bibitem {beybShab}
Бейбалаев  В.Д., Шабанова М.Р. Численный метод решения начально --граничной задачи для двумерного уравнения теплопроводности с производными дробного порядка. // Вест. Сам. гос. техн. ун-та. Сер. Физ.-мат. Науки, 2010, т. 21, №5, с.  244-254.

\bibitem {aleroev}
Алероев Т.С.  Краевые задачи для дифференциальных уравнений с дробными производными. // Дисс. доктора физ.-мат. наук. МГУ. 2000.

\bibitem {beybDavud}
Бейбалаев В.Д., Давудова Ф.Ф., Ламетов А.Г. Численное решение краевой задачи для нелинейного уравнения теплопроводности с производными дробного порядка.// Вестник ДГУ. 2013, вып. 6, с.86-92.


%Сиражудинов
%1.
\bibitem {JikovKozlov}{Жиков В. В., Козлов С. М., Олейник О. А.}
Усреднение дифференциальных операторов. --М.: Наука, 1993; англ. пер.: \textit{V. V.  Jikov, S. M. Kozlov, O. A. Oleў$\imath$nik }Homogenization of differential operators and integral functionals, CityplaceSpringer--Verlag, StateBerlin, 1994.

%2.
\bibitem {JikovSirazh1}
{Жиков В. В., Сиражудинов М. М.}
О G--компактности одного класса недивергентных эллиптических операторов второго порядка// Изв. АН СССР. Сер. матем. -- 1981. --Т. 45, № 4. -- С. 718--733; англ. пер.: V. V. Zhikov, M. M. Sirazhudinov. On G--compactness of a class of nondivergence elliptic operators of second order // Math.USSR--Izv., 19:1, 1982, p. 27--40.

%3.
\bibitem {JikovSirazh2}
{Жиков В. В., Сиражудинов М. М.} Усреднение недивергентных эллиптических и параболических операторов второго порядка и      стабилизация решения задачи Коши// Матем. сб. -- 1981. -- Т. 116, 158:2, 10. -- С. 166--186; англ. пер.: \textit{V. V. Zhikov, M. M. Sirazhudinov}  The averaging of nondivergence second order elliptic and parabolic operators and the stabilization of solutions of the Cauchy problem, Math.USSR--Sb., 44:2, 1983, p. 149--166.

%4.
\bibitem {Sirazh1}
{Сиражудинов М.М.}
G--сходимость и усреднение некоторых недивергентных эллиптических операторов высокого порядка//
Дифференц. Уравнения. -- 1983. -- Т. 19, №11. -- С. 1949--1956; англ. пер.: \textit{M. M. Sirazhudinov} The G--convergence and averaging of some high--order  nondivergence elliptic operators, Differential Equations, 19:11, 1983, 1429--1435.

%5.
\bibitem {Sirazh2}
{Сиражудинов М.М. }
О G--компактности одного класса эллиптических систем первого порядка// Дифференц. Уравнения. --  1990. -- Т. 26, № 2. -- С. 298--305; англ. пер.: \textit{M. M. Sirazhudinov} G--compactness of a class of first--order elliptic systems'', Differential Equations, 26:2, 1990, 229--235.

%6.
\bibitem {Sirazh3}
{Сиражудинов М.М.}
О G--сходимости и усреднении обобщенных операторов Бельтрами//Мат.сб. --2008. --Т. 199, №5. -- С. 124--155.

%7.
\bibitem {Jamaludinova}
{Джамалудинова  С. П.}
Задача Пуанкаре для одного эллиптического уравнения второго порядка// Весткник ДГУ -- 2013. -- Вып. 1. -- С. 65--67.

%Хачлаев???
%WARNING
\bibitem {sirazhNew1}
{Сиражудинов М.М., Джамалудинова С.П.}
О G–компактности одного класса эллиптических операторов второго порядка с комплекснозначными коэффициентами // Вестник ДГУ -- 2014. -- Вып. 1 -- С.77-80.

\bibitem {sirazhNew2}
{Сиражудинов М.М., Джамалудинова С.П.}
Усреднение одного эллиптического уравнения второго порядка с комплекснозначными периодическими коэффициентами // Вестник ДГУ -- 2014. -- Вып. 6 %ERROR

%======Medjidov=======
\bibitem{m1-1-denisjuk} Denisjuk A. Inversion of the X-ray transform for 3D symmetric tensor fields with sources on a curve. Inverse problems. 2006. 22. Pp. 399-411.

\bibitem{m1-2-gelfand}  Гельфанд И.М., Гиндикин С.Г., Граев М.И. Избранные задачи интегральной геометрии. М.: Добросвет, КГУ, 2010.

\bibitem{m1-3-palamodov}  Palamodov V. P. Reconstruction of a differential form from Doppler transform. placecountry-regionSIAM J. Math. Anal. 2009. №41. Pp. 1713-1720.

\bibitem{m1-4-palamodov}  Palamodov V. P. Reconstruction in Doppler tomography. Tel Aviv University, 2009.

\bibitem{m1-5-pal}  Palamodov V. P. Reconstructive Integral Geometry. Tel Aviv University, 2003.

\bibitem{m1-6-sharaf}  Sharafutdinov V.A. Slice-by-slice reconstruction algorithm for vector tomography with incomplete data. Inverse problems. 2007. №23. Pp. 2603-2627.

\bibitem{m1-7-sharaf}  Sharafutdinov V.A. Integral Geometry of Tensor Fields. 1994. (Utrech: VSP).

\bibitem{m1-8-medz}  Меджидов З.Г. Обращение лучевого преобразования симметричного тензорного поля с источниками на кривой. Вестник ДГУ. Вып 6. 2013. С. 107-113.

\bibitem{m1-9-medz}  Меджидов З.Г. Восстановление векторного поля по данным его лучевого преобразования на трехмерных многообразиях прямых. Вестник ДГУ. Вып 6. 2012. С. 159-166.

\bibitem{m1-10-helg}  Хелгасон С. Преобразование Радона. М.: Мир, 1983.
%%%%%%%%%%%%%%%%%%%%%%%%%
\bibitem{m2-1-gel}  Гельфанд И.М., Гиндикин С.Г., Граев М.И. Избранные задачи интегральной геометрии. М.: Добросвет, КГУ, 2010.

\bibitem{m2-2-den}  Denisjuk A. Inversion of the X-ray transform for 3D symmetric tensor fields with sources on a curve. Inverse problems. 2006. 22. Pp. 399-411.

\bibitem{m2-3-sha}  Sharafutdinov V.A. Integral Geometry of Tensor Fields. 1994. (Utrech: VSP).

\bibitem{m2-4-ver}  Vertgeim L.B. Integral geometry problems for symmetric tensor fields with incomplete data. 2000. J. Inverse III-Posed Probl. 8(3). Pp. 353-362.

\bibitem{m2-5-pal}  Palamodov V. P. Reconstruction of a differential form from Doppler transform. SIAM J. Math. Anal. 2009. №41. Pp. 1713-1720.

\bibitem{m2-6-shu}  Shuster T. An efficient mollifier method for three-dimensional vector tomography: convergence and implementation. Inverse Problems. 2001. №17. Pp. 739-766.

\bibitem{m2-7-med}  Меджидов З.Г. Восстановление векторного поля по данным его лучевого преобразования на трехмерных многообразиях прямых. Вестник ДГУ. Вып 6. 2012. С. 159-166.

\bibitem{m2-8}  Наттерер Ф. Математические аспекты компьютерной томографии. М.: Мир,1990.
%%%%%%%%%%%%%%%%%%%%%%%%%%%%
\bibitem{sharaf-int} Sharafutdinov V.A. Integral Geometry of Tensor Fields. 1994. (Utrech: VSP).

\bibitem{sharaf-slice}  Sharafutdinov V.A. Slice-by-slice reconstruction algorithm for vector tomography with incomplete data. Inverse problems. 2007. №23. Pp. 2603-2627.

\bibitem{medzid-obr}  Меджидов З.Г. Обращение лучевого преобразования симметричного тензорного поля с источниками на кривой. Вестник ДГУ. Вып 6. 2013. С. 107-113.

\bibitem{natter-matem}  Наттерер Ф. Математические аспекты компьютерной томографии. М.: Мир,1990.

%=====/Medjidov========


%Магомедов
\bibitem{lit02}
{Гэри M., Джонсон Д.} Вычислительные машины и труднорешаемые задачи. Пер. с англ. М.: Мир, 1982.

\bibitem{lit04}
{Визинг В.Г.} Об оценке хроматического класса p-графа // Дискретный анализ. Сб. науч. тр. Вып. 3. Новосибирск: Ин-т математики СО АН СССР. 1964. С. 25--30.

\bibitem{lit05}
{Ловас Л., Пламмер М.} Прикладные задачи теории графов. Теория паросочетаний в математике, физике, химии / Пер. с англ. -- М.: Мир, 1998. -- 653 с.

\bibitem{lit06}
{ Holyer I.} The $NP$-completeness of edge-coloring // SIAM J. Comput. 1981. V. 10. \No 4. P. 718--720.

\bibitem{lit07}
{ Асратян А.С., Камалян Р.Р.} Интервальные раскраски ребер мультиграфа / Прикладная математика. Вып. 5. Ереван: Изд-во Ереван. ун-та. 1987. С. 25--34.

\bibitem{lit08}
{ Магомедов А.М.} Раскраска графа с непрерывным спектром. М., 1985. Деп. в ВИНИТИ, \No 478--85.

\bibitem{lit01New08}
{ Even S., Itai A., Shamir A.} On the complexity of timetable and integral multi-commodity flow problems // SIAM J. Comput. 1976. V.\,5. \No 4. P. 691--703.

\bibitem{lit11}
{ Севастьянов С.В.} Об интервальной раскрашиваемости ребер двудольного графа // Методы дискретного анализа. 1990. Т. 50. C. 61--72.

\bibitem{lit12}
{ Hansen H.M.} Scheduling with minimum waiting periods (In Danish) // Master Thesis, Odense University. Odense, Denmark. 1992.

\bibitem{lit13}
{ Магомедов А.М.} К вопросу об условиях уплотнения матрицы из 6 столбцов. М., 1991. Деп. в ВИНИТИ.

\bibitem{lit14}
{ Hanson D., Loten C.O.M., Toft B.} On interval colourings of bi-regular bipartite graphs // Ars Combinat. 1998. V. 50. P. 23--32.


\bibitem{lit16}
{ Giaro K.} Compact task scheduling on dedicated processors with no waiting period (in Polish) / PhD thesis, Technical University of Gdansk, IETI Faculty. Gdansk. 1999.

\bibitem{lit20new13}
{ Магомедов А.М.} Условия и алгоритм уплотнения матрицы из 4 столбцов. М., 1992. Деп. в ВИНИТИ, \No 175--В92.

\bibitem{lit17}
{ Giaro K.} The complexity of consecutive $\Delta$-coloring of bipartite graphs: 4 is easy, 5 is hard // Ars Combin. 1997. V. 47. P. 287--298.

\bibitem{lit22}
{ Asratian A.S, Casselgren C.J.} Some results on interval edge colorings of ($\alpha,\beta$)-biregular bipartite graphs // Department Math. 2007. Linkoping University S--581 83. Linkoping, Sweden.

\bibitem{lit18}
{ Giaro K., Kubale M., Malafiejcki M.} On the deficiency of bipartite graphs // Discrete Appl. Math. 94. Gdansk. 1999. P. 193--203.

\bibitem{lit23}
{ Танаев В.С., Сотсков Ю.Н., Струсевич В.А.} Теория расписаний. Многостадийные системы. М.: Наука, 1989.


%Magomedov 2
\bibitem{harary}
Харари Ф. \textit{Теория графов}: Пер. с англ. -- М.: Мир, 1973. -- 300 с.

\bibitem{grunb1}
{Grünbaum B.} Conjecture 6. In Recent Progress in Combinatorics (W.T. Tutte Ed.), New York: Academic Press (1969) 343.

\bibitem{archdea1}
{Archdeacon D.} Problems in Topological Graph Theory: Three-Edge-Coloring Orientable Triangulations / Website: http://www.emba.uvm.edu/\~{}darchdea/problems/grunbaum.htm.

\bibitem{archdea2}
{Archdeacon D.} Problems in Topological Graph Theory / website:http://www.emba.uvm.edu/\~{}darchdea/problems/problems.html.

\bibitem{bondymurty}
{Bondy J.A., Murty U.S.R.} Graph Theory with Applications, New York: Elsevier Science (1976).

\bibitem{saatykai}
{Saaty T.L., Kainen P.C.} The Four-Color Problem: Assaults and Conquest, New York: Dover, 1986.

\bibitem{kochol}
{Kochol M.} Polyhedral embeddings of snarks in orientable surfaces, Proc. Amer. Math. Soc. 137 (2009), 1613--1619.

\bibitem{konigD}
{König D.} Gráfok és alkalmazásuk a determinánsok Žs a halmazok elmé- letére. // Matematikai és Természettudományi Értesítő. 34 (1916), 104--119.

\bibitem{petersen}
{Petersen J.} Die Theorie der regulдren Graphen, Acta Math 15 (1891), 193--220.

\bibitem{ringel}
{Рингель Г.} Теорема о раскраске карт: Пер. с англ. -- М.: Мир, 1977. -- 256 с.

\bibitem{granel1}
{Grannell M.J., Griggs T.S., Širáň J.} Face 2-colourable triangular embeddings of complete graphs, J. Combin. Theory Ser. B 74 (1998), № 1, 8--19.

\bibitem{yungjwt}
{Youngs J.W.T.} The mystery of the Heawood conjecture. In \textit{Graph Theory and its Applications} (B. Harris, Ed.), New York: Academic Press (1970), 17--50.

\bibitem{granel2}
{Grannell M.J., Griggs T.S.} Designs and Topology. In \textit{Surveys in Combinatorics} 2007, London Math. Soc. Lecture Note Series 346, Cambridge Univ. Press, Cambridge (2007), 121--174.

\bibitem{lawrnegam}
{Lawrencenko S., Negami S., White A.T.} Three nonisomorphic triangulations of an orientable surface with the same complete graph, Discrete Math 135 (1994), № 1--3, 367--369.

\bibitem{granel3}
{Bonnington C.P., Grannell M.J., Griggs T.S., Širáň J.} Exponential families of non-isomorphic triangulations of complete graphs, J. Combin. Theory Ser. B 78 (2000), № 2, 169--184.

\bibitem{korzhikvp}
{Grannell M.J., Korzhik V.P.} Nonorientable biembeddings of Steiner triple systems, Discrete Math. 285 (2004), 121--126.

\bibitem{ringyung}
{Ringel G., Youngs J.W.T.} Das Geschlecht des symmetrischen vollstдndigen dreifдrbbaren Graphen, Comment. Math. Helv. 45 (1970), 152--158.

\bibitem{granel4}
{Grannell M.J., Griggs T.S., Knor M.} Biembeddings of Latin squares and Hamiltonian decompositions, Glasgow Math. J. 46 (2004), № 3, 443--457.

\bibitem{granel5}
{Grannell M.J., Griggs T.S., Knor M., Širáň J.} Triangulations of orientable surfaces by complete tripartite graphs, Discrete Math. 306 (2006), 600--606.

\bibitem{knorm}
{Grannell M.J., Knor M.} Dihedral biembeddings and triangulations by complete and complete tripartite graphs, Graphs and Combin. 29 (2013), № 4, 921--932.


%============Kadiev===============
\bibitem{kad-1}{Kadiev R., Ponosov A.}{The W-transform in stability analysis for stochastic linear functional difference equations //~J.~Math. Anal. Appl. 2012. V.~389. Issue~2. P.~1239--1250.}
\bibitem{kad-2}{Азбелев Н.В., Березанский Л.М., Симонов П.М., Чистяков А.В.}{Устойчивость линейных систем с последствием.~I //~Дифференц. уравнения. 1987. Т.~28. \No~5. С.~745--754.}
\bibitem{kad-3}{Азбелев Н.В., Максимов В.П., Рахматуллина Л.Ф.}{Введение в теорию функционально-дифферен\-циальных уравнений. М.,~1991.}
\bibitem{kad-4}{Азбелев Н.В., Симонов П.М.}{Устойчивость решений уравнений с обыкновенными производными. Пермь,~2001.}
\bibitem{kad-5}{Березанский Л.М.}{Развитие W-метода Н.В.~Азбелева в задачах устойчивости решений линейных функционально-дифференциальных уравнений //~Дифференц. уравнения. 1986. Т.~22. \No~5.\linebreak С.~739--750.}
\bibitem{kad-6}{Berezansky L., Braverman E.}{On exponential dichotomy, Bohl Perron type theorems and stability of difference equations //~J.~Math. Anal. Appl. 2005. V.~304. P.~511--530.}
\bibitem{kad-7}{Braverman E., Karabach I.M.}{Bohl--Perron-type stability theorems for linear difference equations with infinite delay //~J.~of Differ. Equat. Appl. 2012. V.~5. \No~5. P.~909--939.}
\bibitem{kad-8}{Кадиев Р.И.}{Достаточные условия устойчивости стохастических систем с последствием //~Дифференц. уравнения. 1994. Т.~30. \No~2. С.~555--564.}
\bibitem{kad-9}{Кадиев Р.И.}{Устойчивость решений стохастических функционально-дифференциальных уравнений: Автореф. дис.~\ldots~д-ра физ.-мат. наук. Екатеринбург,~2000.}
\bibitem{kad-10}{Кадиев Р.И., Поносов А.В.}{Устойчивость стохастических функционально-дифференциальных уравнений при постоянно действующих возмущениях //~Дифференц. уравнения. 1992. Т.~28. \No~2.\linebreak С.~198--207.}
\bibitem{kad-11}{Kadiev R.I., Ponosov A.V.}{Relations between stability and admissibility for stochastic linear functional differential equations //~J.~Func. Differ. Equat. 2004. P.~1--28.}
\bibitem{kad-12}{Ватанабэ С., Икэда Н.}{Стохастические дифференциальные уравнения и диффузионные процессы. М.,~1981.}
\bibitem{kad-13}{Канторович Л.В., Акилов Г.П.}{Функциональный анализ. М.,~1984.}

\bibitem{kad-vestdgu2014}
Кадиев Р.И., Шахбанова З.И. Устойчивость решений скалярных линейных разностных уравнений Ито с последействием //Вестник Дагестанского госуниверситета 2014 , № 1, с. 97-103.
%============/Kadiev===============


%======= tadg added this
\bibitem{idprm35}
{Чебышев П.Л.} О непрерывных дробях (1855). Полн.собр.соч.
     Т.2. М.: Изд.АН СССР. 1947. С. 103 -- 126.

\bibitem{idprm36}
{Чебышев П.Л.} Об одном новом ряде.   Полн.собр.соч.
Т.2.  М.: Изд.АН
         СССР. 1947.  С. 236 -- 238.

\bibitem{idprm37}
{Чебышев П.Л.} Об интерполировании по способу наименьших квадратов (1859). Полн.собр.соч.  Т.2.  М.: Изд.АН СССР. 1947.  С. 314--334.
\bibitem{idprm38}{Чебышев П.Л.} Об интерполировании (1864). Полн.собр.соч.  Т.2.  М.: Изд.АН СССР. 1947.  С. 357--374.

\bibitem{idprm39}
{Чебышев П.Л.} Об интерполировании величин  равноотстоящих (1875).   Полн.собр.соч.  Т.3.  М.: Изд.АН  СССР. 1948.  С.
66--87.

\bibitem{idprm76}{Шарапудинов И.И.} Многочлены, ортогональные на
дискретных      сетках. -- Махачкала. Изд -- во Даг.гос.пед.
ун--та, 1997 г.


\bibitem{idprm98}{Gasper G.} Positiviti and special function//Theory and
appl.Spec.Funct. Edited by Richard A.Askey. 1975.P. 375-433.

\bibitem{idprm99}{Шарапудинов Т.И.} Аппроксимативные свойства смешанных рядов по полиномам Чебышева, ортогональным на равномерной сетке.
Вестник Дагестанского научного центра РАН.  Вып. 29. стр. 12 -- 23. Махачкала, 2007 г

\bibitem{idprm9}{Сеге Г.} Ортогональные многочлены. -- Москва: Государственное издательство физико-математической литературы. 1962. 500 с.

\bibitem{idprmmnk}{Шарапудинов И.И.} О сходимости метода наименьших квадратов. Матем. заметки, 53:3 (1993),  131–143.

\bibitem{idprmasympPropsAndWeightEst}{Шарапудинов И.И.}. Асимптотические свойства и весовые оценки для ортогональных многочленов Чебышёва–Хана. Матем. сб., 182:3 (1991),  408–420.

\bibitem{idprmasympProps}{Шарапудинов И.И.} Асимптотические свойства ортогональных многочленов Хана дискретной переменной. Матем. сб., 180:9 (1989),  1259–1277.

\bibitem{idprmasympCheb}{Шарапудинов И.И.} Об асимптотике многочленов Чебышева, ортогональных на конечной системе точек. Вестник МГУ. Серия 1. 1992. Вып. 1. С. 29–35.





\bibitem{idprmapproxYn}{Шарапудинов И.И.} 	Аппроксимативные свойства операторов Yn+2r(f) и их дискретных аналогов. Матем. заметки, 72:5 (2002),  765–795.

\bibitem{idprmapproxByCheb}{Шарапудинов И.И.} Приближение дискретных функций и многочлены Чебышева, ортогональные на равномерной сетке. Матем. заметки, 67:3 (2000),  460–470.

\bibitem{idprmgreenBook}{Шарапудинов И.И.} Смешанные ряды по ортогональным полиномам. Теория и приложения. --- Махачкала: ДНЦ РАН, 2004. 276 с.

\bibitem{kadbook1}{Азбелев Н. В.}Как это было // Проблемы нелинейного анализа в инженерных системах. 2003. Т. 9, вып. 1(17). С. 22-39.

\bibitem{kadbook2}{Азбелев Н.В., Симонов П.М.} Устойчивость решений уравнений с обыкновенными производными. Пермь: Перм. ун-т, 2001. 230 с.

\bibitem{kadbook3}{Азбелев Н.В., Березанский Л.М., Симонов П.М., Чистяков А.В.} Устойчивость линейных систем с последействием. I // Дифференц. уравнения. 1987. Т. 23, No 5. С. 745-754.

\bibitem{kadbook4}{Азбелев Н.В., Березанский Л.М., Симонов П.М., Чистяков А.В.} Устойчивость линейных систем с последействием. II // Дифференц. уравнения. 1991. Т. 27, No 4. С. 555-562.

\bibitem{kadbook5}{Азбелев Н.В., Березанский Л.М., Симонов П.М., Чистяков А.В.} Устойчивость линейных систем с последействием. III // Дифференц. уравнения. 1991. Т. 27, No 190. С. 1659-1668.

\bibitem{kadbook6}{Азбелев Н.В., Березанский Л.М., Симонов П.М., Чистяков А.В.} Устойчивость линейных систем с последействием. IV // Дифференц. уравнения. 1993. Т. 29, No 2. С. 196-204.

\bibitem{kadbook7}{Азбелев Н.В., Симонов П.М.} Устойчивость уравнений с запаздывающим аргументом // Изв. вузов. Математика. 1997. No 6 (421). С. 3-16.

\bibitem{kadbook8}{Кадиев Р. И., Поносов А. В.} Устойчивость стохастических функционально-дифференциальных уравнений относительно постоянно действующих возмущений // Дифференц. уравнения. 1992. Т. 28. No 2. С. 198-207.

\bibitem{kadbook9}{Кадиев Р. И.} Достаточные условия устойчивости стохастических систем с последействием // Дифференц. уравнения. 1994. Т. 30. No 2. С. 555-564.

\bibitem{kadbook10}{Кадиев Р. И.} Устойчивость решений стохастических функционально-дифференциальных уравнений // Дис. ... д-р физ.-мат. Наук. Екатеринбург. 2000. 231 с.

\bibitem{kadbook11}{Липцер Р.Ш., Ширяев А.Н.} Теория мартингалов. M.: Наука, 1986. 512 с.

\bibitem{kadbook12}{Kadiev R.I., Ponosov A.V.} Stability of stochastic functional differential equations and the W-transform // E. J. Diff. Eqs. 2004, no. 92: 1-36.

\bibitem{kadbook13}{Kadiev R.I., Ponosov A.V.} Relations between stability and admissibility for stochastic linear functional differential equations // J. Func. Diff. Eqs. 2005, no. 12: 117-141.

\bibitem{kadbook14}{Kadiev R.I., Ponosov A.V.} Stability of solutions of linear impulsive systems of Itô differential equations with aftereffect Springer. 2007, 43, no. 7: 898-904.

\bibitem{kadbook15}{Kadiev R.I., Ponosov A.V.} Exponential stability of linear stochastic differential equations with bounded delay and the W-transform // E. J. Qualitative Theory of Diff. Eq. 2008, no. 23: 1-16.

\bibitem{kadbook16}{Kadiev R.I., Ponosov A.V.} Stability of impulsive stochastic differential equations with linear delays // J. of Abstract Diff. Eqs. and Applications, 2012, 2, no. 2: 7–25.



%SMS

\bibitem{nushii2}
{Шарапудинов~И.И.} Асимптотика полиномов, ортогональных на сетках из единичной окружности и числовой прямой // "Современные проблемы математики, механики, информатики": материалы междунар. науч. конф. Тула: Изд-во ТулГУ, 2009 г. С.~100-106.

\bibitem{nushii3}
{Шарапудинов~И.И.} Некоторые свойства полиномов, ортогональных на неравномерных сетках из единичной окружности и отрезка // Современные проблемы теории функций и их приложения. Материалы 15-й Саратовской зимней школы, посвященной 125-летию со дня рождения В.В. Голубева и 100-летию СГУ. Саратов: Изд-во «Научная книга», 2010 г. С.~187.

\bibitem{nushii4}
{Шарапудинов~И.И.} Асимптотические свойства полиномов, ортогональных на конечных сетках единичной окружности. // Вестник Дагестанского научного центра, 2011. №42. С.~5-14.

\bibitem{nunurik1}
{Нурмагомедов~А.А.} Об асимптотике многочленов, ортогональных на произвольных сетках. // Известия Саратовского университета. Новая серия. Серия Математика. Механика. Информатика. 2008. Т.~8,  №~1. С.~25-31.

\bibitem{nunurik2}
{Нурмагомедов~А.А.} Асимптотические свойства многочленов $\hat{p}_{n}^{\alpha,\beta}(x)$, ортогональных на произвольных сетках в случае целых $\alpha$ и $\beta$. // Известия Саратовского университета. Новая серия. Серия Математика. Механика. Информатика. 2010. Т.~10, №~2. С.~10-19.

\bibitem{mcbaik}
{Baik J., Kriecherbauer T., McLaughlin K. T.-R., Miller P.D.} Discrete
orthogonal polynomials. Asymptotics and applications. Princeton:
Princeton University Press, 2007. 184 p.

\bibitem{ouwong}
{Ou C., Wong R.} The Riemann-Hilbert approach to global asymptotics
of discrete orthogonal polynomials with infinite nodes. // Analysis and
Applications, 2010. Vol. 8. P. 247-286.

\bibitem{lopezsinus}
{Ferreira C., L\'{o}pez J.L., Sinus\'{\i}a E.P.} Asymptotic relations between the
Hahn-type polynomials and Meixner-Pollaczek, Jacobi, Meixner and
Krawtchouk polynomials. // Journal of Computational and Applied
Mathematics, 2008. Vol. 217. P. 88-109.


\bibitem{baritri}
{Бари~Н.К.} Обобщение неравенств С. Н. Бернштейна и А. А. Маркова. // Изв. АН СССР. Сер. матем., 1954. Т.~18:2. С.~159–176.

\bibitem{konyagin}
{Конягин~C.В.} О неравенстве В.А. Маркова для многочленов в метрике $L$. // Труды Математического института им. В.А. Стеклова, 1980. №~145. С.~117-125.

\bibitem{nushii5}
{Шарапудинов~И.И.} О сходимости метода наименьших квадратов. // Математические заметки, 1993. Т.~53, вып. 3, С.~131–143.

\bibitem{nunurik3}
{Нурмагомедов~А.А.} Многочлены, ортогональные на неравномерных сетках. // Известия Саратовского университета. Новая серия. Серия Математика. Механика. Информатика. 2011. Т.~11,  №~3, ч.2. С.~29-42.



%===========Алишаев========
\bibitem{alish-df-1} \textbf{ }\textit{Воронец Д., Козич Д. }Влажный воздух. Термодинамические свойства и применение. -- М.:  Энергоатомиздат, 1984. 136 с.

\bibitem{alish-df-2}  \textit{Шпайдель К.} Диффузия и конденсация водяного пара в ограждающих конструкциях. -- М.: Стройиздат, 1985. 48 с., ил.

\bibitem{alish-df-3}  \textit{Матвеев Л.Т.} Курс общей метеорологии. Физика атмосферы. -- Л., Гидрометеоиздат, 1984, 751 с.

\bibitem{alish-df-4}  \textit{Алишаев М.Г.} О конденсации и осаждении влаги в приземном слое атмосферы // Метеорология и гидрология, №8, 2013, С. 17-28.

\bibitem{alish-df-5}  \textit{Чекалюк Э.Б. }Термодинамика нефтяного пласта. М.: Недра, 1965. 238 с.

\bibitem{alish-df-6}  \textit{Куликовский А.Г.} О фронтах испарения и конденсации в пористых средах // Изв. РАН. МЖГ. 2002. №5. С. 85-92.

\bibitem{alish-df-7}  \textit{Цыпкин Г.Г.}\textbf{ }Течения с фазовыми переходами в пористых средах. - М.: Наука. ФИЗМАТЛИТ. 2009. -220 с.\textbf{~}

\bibitem{alish-df-8} \textbf{ }\textit{Алишаев М.Г}\textbf{.} Неизотермическая фильтрация почвенного воздуха с фазовыми переходами вода-пар //\textbf{ }Возобновляемая энергетика: проблемы и перспективы// Выпуск 3. Махачкала, АЛЕФ, 2014. С. 100-110.

\bibitem{alish-df-9}  \textit{Бояринцев Е.Л., Гопченко Е.Д., Сербов Н.Г.,Завалий Н.В.} Экспериментальные исследования испарения и конденсации в горных регионах зоны многолетнемёрзлых пород // Вiсник Одеського державного екологiчного унiверситету, 2010, вип. 10. С. 162-168.

\bibitem{alish-df-10}  \textit{Волчек А.А., Кирвель П.И.} Прогнозирование колебаний испарения с поверхности водоемов Беларуси// Вестник БГУ. Сер. 2. 2008. №2. С. 86-93.

\bibitem{alish-df-11}  \textit{Панжин  Д.А., Галишев М.А., Сивенков А.Б.}\textbf{ }Исследование критических явлений при распространении нефтяных загрязнений по почвенному слою // Интернет-журнал «Технологии техносферной безопасности». Выпуск 4\eqref{alish_temp_38_}, август, 2011.

\bibitem{alish-df-12}  \textit{Полубаринова-Кочина П.Я.} Теория движения грунтовых вод. М., Наука, 1977, 664 с.

\bibitem{alish-df-13}  \textit{Полянин А.Д., Зайцев В.Ф., Журов А.И.} Методы решения нелинейных уравнений математической физики и механики. ФИЗМАТЛИТ. 2005. 256 с.

\bibitem{alish-df-14}  \textit{Шлычков В.А.} Численное моделирование тепловлагообмена в системе атмосфера-почва в засушливый период // Вычислительные технологии. Том 9, №1, 2004. С. 105-112.

%==============/Алишаев==============

\bibitem{taimazov}
{Таймазов Д.Г.} Патент № 2316027 <<Способ определения изменений напряженно-деформированного состояния земной коры>>. БИ, 2008, №~3.


\end{thebibliography}
