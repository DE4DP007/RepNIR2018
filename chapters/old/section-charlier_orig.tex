

\chapter{Полиномы, ортогональные по Соболеву, порожденные полиномами Шарлье}

%
%
%\section{Системы дискретных функций, ортонормированных по Соболеву, порожденные ортонормированными функциями}
%
%Как уже отмечалось выше, системы дискретных функций, ортонормированных по Соболеву относительно скалярного произведения \eqref{Shar_eq2}, порожденных заданной системой $\{\psi_k(x)\}_{k=0}^\infty$, ортонормированной на дискретном множестве $\Omega=\{0,1,\ldots\}$ с весом $\rho(x)$, были рассмотрены в \cite{Shar7}. В дальнейшем нам понадобятся некоторые результаты из \cite{Shar7}, поэтому мы вкратце напомним их в этом параграфе. С этой целью, следуя \cite{Shar7}, введем некоторые обозначения и понятия.
%
%
%Если целое $k\ge0$, то положим $a^{[k]}=a(a-1)\cdots (a-k+1)$, $a^{[0]}=1$ и рассмотрим следующие функции
%\begin{equation}\label{Shar_eq3}
%\psi_{r,k}(x)= \frac{x^{[k]}}{k!},\ k=0,1,\ldots,r-1,
%\end{equation}
%\begin{equation}\label{Shar_eq4}
%\psi_{r,k}(x) = \begin{cases} \frac{1}{(r-1)!} \sum\limits_{t=0}^{x-r} (x-1-t)^{[r-1]} \psi_{k-r}(t), &\text{$r\le x$,} \\
%0, &\text{$x=0,1,\ldots, r-1$}, \end{cases}
%\end{equation}
%которые определены на сетке $\Omega=\{0,1,\ldots\}$. Рассмотрим некоторые важные разностные свойства системы функций $\psi_{r,k}(x)$, определенных равенствами \eqref{Shar_eq3} и \eqref{Shar_eq4}. Введем оператор конечной разности $\Delta f$: $\Delta f(x)=f(x+1)-f(x)$ и положим $\Delta^{\nu+1} f(x)=\Delta\Delta^\nu f(x)$. Имеет место следующий факт \cite{Shar7}:
%
%\begin{equation}\label{Shar_eq5}
%\Delta^\nu \psi_{r,k}(x) = \begin{cases} \psi_{r-\nu,k-\nu}(x), &\text{если $0\le\nu\le r-1$, $r\le k$,} \\
%\psi_{k-r}(x), &\text{если $\nu=r\le k$,} \\
%\psi_{r-\nu,k-\nu}(x), &\text{если $\nu\le k< r$,} \\
%0, &\text{если $k< \nu\le r$}. \end{cases}
%\end{equation}
%
%
%Пусть $\rho: \Omega \to \mathbb{R}$ --- положительная функция, для которой $\sum_{x=0}^\infty \rho(x) < \infty$. Обозначим через $l_\rho$ пространство дискретных функций $f,g,\ldots$, в котором скалярное произведение определяется обычным образом с помощью равенства $(f,g) = \sum\limits_{x\in\Omega}	 f(x)g(x)\rho(x)$.
%Через $W^r_{l_\rho}$ обозначим подпространство в $l_\rho$, состоящее из функций $f,g,\ldots$, для которых определено скалярное произведение \eqref{Shar_eq2}.
%Рассмотрим задачу об ортонормированности и полноте в $W^r_{l_\rho}$ системы $\{\psi_{r,k}(x)\}_{k=0}^\infty$, состоящей из функций, определенных равенствами \eqref{Shar_eq3} и \eqref{Shar_eq4}. В работе \cite{Shar7} эта задача была решена для случая, когда $\l_\rho = W^r_{l_\rho}$. В настоящей работе мы обобщаем этот результат на тот случай, когда подпространство $W^r_{l_\rho} \subset l_\rho$ не обязательно совпадает со всем пространством $l_\rho$. А именно, справедлива следующая
%\begin{theorem}\label{Shar_thm1}
%	Предположим, что функции $\psi_k(x)$ $(k=0,1,\ldots)$ образуют полную в $l_\rho$ ортонормированную систему c весом $\rho(x)$. Тогда система $\{\psi_{r,k}(x)\}_{k=0}^\infty$, порожденная системой $\{\psi_{k}(x)\}_{k=0}^\infty$
%	посредством равенств \eqref{Shar_eq3} и \eqref{Shar_eq4}, полна в $W^r_{l_\rho}$ и ортонормирована относительно скалярного произведения \eqref{Shar_eq2}.
%\end{theorem}
%
%Систему функций $\{\psi_{r,k}(t)\}_{k=0}^\infty$ мы будем называть системой, ортонормированной по Соболеву относительно скалярного произведения \eqref{Shar_eq2}
%
%Из теоремы \ref{Shar_thm1} следует, что система дискретных функций $\{\psi_{r,k}(t)\}_{k=0}^{\infty}$ является ортонормированным базисом (ОНБ) в пространстве $W^r_{l_\rho}$, поэтому для произвольной функции $f(x)\in W^r_{l_\rho}$ мы можем записать равенство
%\begin{equation}\label{Shar_eq6}
%f(x) = \sum_{k=0}^\infty \langle f,\psi_{r,k} \rangle \psi_{r,k}(x),
%\end{equation}
%которое представляет собой ряд Фурье функции $f(x)\in W^r_{l_\rho}$ по системе
%$\{\psi_{r,k}(t)\}_{k=0}^\infty$, ортонормированной по Соболеву. Заметим, что из полноты системы функций $\{\psi_{r,k}(t)\}_{k=0}^{\infty}$ в пространстве $W^r_{l_\rho}$ (теорема \ref{Shar_thm1}) следует, что ряд \eqref{Shar_eq6} сходится по норме пространства $W^r_{l_\rho}$. Нетрудно также показать, что ряд \eqref{Shar_eq6} сходится в каждой точке $x\in\{0,1,\ldots\}$.
%
%Поскольку коэффициенты Фурье $\langle f,\psi_{r,k} \rangle$ имеют вид
%$$ f_{r,k}=\langle f,\psi_{r,k} \rangle =\sum_{\nu=0}^{r-1}\Delta^\nu f(0)\Delta^\nu\psi_{r,k}(0)=\Delta^kf(0),\, k=0,\ldots, r-1, $$
%$$ f_{r,k}= \langle f,\psi_{r,k} \rangle =\sum_{j=0}^\infty\Delta^rf(j)\Delta^r\psi_{r,k}(j)\rho(j)= $$
%$$ =\sum_{j=0}^{\infty}\Delta^rf(j)\psi_{k-r}(j)\rho(j),\, k=r,r+1,\ldots, $$
%то равенство \eqref{Shar_eq6} можно переписать в следующем смешанном виде
%\begin{equation}\label{Shar_eq7}
%f(x)= \sum_{k=0}^{r-1}\Delta^kf(0)\frac{x^{[k]}}{k!} +\sum_{k=r}^\infty f_{r,k} \psi_{r,k}(x),\, x\in \Omega.
%\end{equation}
%Поэтому ряд Фурье по системе $\{\psi_{r,k}(t)\}_{k=0}^\infty$ мы будем, следуя \cite{Shar8}, называть смешанным рядом по исходной ортонормированной $\{\psi_{k}(t)\}_{k=0}^\infty$. Отметим некоторые важные свойства смешанных рядов \eqref{Shar_eq7} и их частичных сумм вида
%\begin{equation}\label{Shar_eq8}
%\mathcal{ Y}_{r,n}(f,x)= \sum_{k=0}^{r-1}\Delta^kf(0)\frac{x^{[k]}}{k!} +\sum_{k=r}^{n}f_{r,k} \psi_{r,k}(x).
%\end{equation}
%Из \eqref{Shar_eq7} и \eqref{Shar_eq8} с учетом равенств \eqref{Shar_eq5} мы можем записать ($0\le\nu\le r-1$, $x\in \Omega$ )
%\begin{equation}\label{Shar_eq9}
%\Delta^\nu f(x)= \sum_{k=0}^{r-\nu-1}\Delta^kf(0)\frac{x^{[k]}}{k!} +\sum_{k=r-\nu}^\infty f_{r,k+\nu} \psi_{r-\nu,k}(x),
%\end{equation}
%\begin{equation}\label{Shar_eq10}
%\Delta^\nu\mathcal{ Y}_{r,n}(f,x)= \sum_{k=0}^{r-\nu-1}\Delta^kf(0)\frac{x^{[k]}}{k!} +\sum_{k=r-\nu}^{n-\nu} f_{r,k+\nu} \psi_{r-\nu,k}(x),
%\end{equation}
%\begin{equation}\label{Shar_eq11}
%\Delta^\nu\mathcal{ Y}_{r,n}(f,x) = \mathcal{ Y}_{r-\nu,n-\nu}(\Delta^\nu f,x).
%\end{equation}

\section{Некоторые сведения о полиномах Шарлье}

При конструировании полиномов, ортогональных по Соболеву и порожденных классическими полиномами Шарлье нам понадобится ряд свойств этих полиномов, которые мы приведем в настоящем параграфе.
Для произвольного $\alpha$ положим
\begin{equation}\label{Shar_eq12}
\rho(x)=\rho(x;\alpha)=\frac{\alpha^x e^{-\alpha}}{\Gamma(x+1)},
\end{equation}
\begin{equation}\label{Shar_eq13}
S_n^{\alpha}(x)=\frac{1}{\alpha^n \rho(x)} \Delta^n \{\rho(x)x^{[n]}\},
\end{equation}
%где $\Delta^nf(x)$ --- конечная разность $n$-го порядка функции
%$f(x)$ в точке $x$, т.е. $\Delta^0f(x)=f(x)$,
%$\Delta^1f(x)=\Delta f(x)=f(x+1)-f(x)$, $\Delta^nf(x)=\Delta
%\Delta^{n-1}f(x)$ $(n\ge1)$, $a^{[0]}=1$,
%$a^{[k]}=a(a-1)\cdots(a-k+1)$ при $k\ge1$. 
Для каждого $0\le n$ равенство \eqref{Shar_eq13} определяет \cite{Shar9, Shar10} алгебраический полином степени $n$.
Полные доказательства приведенных ниже свойств полиномов Шарлье $S_n^{\alpha}(x)$
можно найти, например, в \cite{Shar9}.

Если $\alpha>0$, то полиномы $S_n^{\alpha}(x)$ ($n=0,1,\ldots$) образуют полную \cite[стр. 243]{Shar9}, \cite[стр. 375]{Shar11} в $l_\rho$ ортогональную с весом $\rho(x)$ (см. \eqref{Shar_eq12}) систему на множестве $\Omega=\{0,1,\ldots\}$:
\begin{equation}\label{Shar_eq14}
\sum_{x\in\Omega}S_k^\alpha(x)S_n^\alpha(x)\rho(x)=\delta_{nk} h_n(\alpha),
\end{equation}
где
\begin{equation}\label{Shar_eq15}
h_n(\alpha) = \sum_{x=0}^{\infty} \rho(x) \{S_n^\alpha(x)\}^2 = \alpha^{-n} n! .
\end{equation}

Из \eqref{Shar_eq14} и \eqref{Shar_eq15} следует, что полиномы
\begin{equation}\label{Shar_eq16}
s_n^\alpha(x)=(h_n(\alpha))^{-\frac12} S_n^\alpha(x) \quad (n=0,1,\ldots)
\end{equation}
образуют ортонормированную систему на множестве $\Omega$ с весом $\rho(x)=\rho(x,\alpha)$, т.е.
\begin{equation*}
\sum_{x\in\Omega} s_k^\alpha(x) s_n^\alpha(x)\rho(x)=\delta_{nk}.
\end{equation*}

Полиномы Шарлье допускают следующее явное представление
\begin{equation} \label{Shar_eq17}
S_n^\alpha(x) = \sum_{l=0}^n \frac{(-n)_l(-x)_l}{l!} (-\alpha)^{-l} = \sum_{l=0}^n \frac{n^{[l]}x^{[l]}}{l!} (-\alpha)^{-l},
\end{equation}
где $(a)_l=a(a+1)\ldots(a+l-1)$ --- символ Похгаммера. Из \eqref{Shar_eq17} непосредственно следует, что
\begin{equation}\label{Shar_eq18}
\Delta S_n^{\alpha}(x) = -\frac{n}{\alpha} S_{n-1}^{\alpha}(x).
\end{equation}

\section{Ортогональные по Соболеву полиномы, порожденные полиномами Шарлье}

При $\alpha>0$ рассмотрим на $\Omega$ полиномы $s_n^\alpha(x)\ (n=0,1,\ldots)$. Эта система порождает на $\Omega$ систему полиномов $s_{r,k}^{\alpha}(x)$ $(k=0, 1,\ldots)$, определенных равенствами
\begin{equation}\label{Shar_eq19}
s_{r,k+r}^{\alpha}(x)=\frac{1}{(r-1)!} \sum_{t=0}^{x-r}(x-1-t)^{[r-1]}s_{k}^{\alpha}(t), \quad k=0,1,\ldots
\end{equation}
\begin{equation}\label{Shar_eq20}
s_{r,k}^{\alpha}(x)=\frac{x^{[k]}}{k!},\, k=0,1,\ldots,r-1.
\end{equation}

Равенство \eqref{Shar_eq19} определяет для целых $x\geq r$ полином степени $k+r$, который мы можем продолжить на всю комплексную плоскость по принципу аналитического продолжения. Покажем, что продолженный полином, который согласно \eqref{Shar_eq19} удовлетворяет первому из равенств определения \eqref{2.2}, удовлетворяет также и второму из равенств \eqref{2.2}. Другими словами, покажем, что $s_{r,k+r}^{\alpha}(x)$ обращается в нуль в точках $x=0,1,\ldots,r-1$. С этой целью мы рассмотрим следующий дискретный аналог формулы Тейлора ($x\in\{r,r+1,\ldots\}$)
\begin{equation}\label{Shar_eq21}
F(x)=Q_{r-1}(F,x) + \frac{1}{(r-1)!}\sum_{t=0}^{x-r} (x-1-t)^{[r-1]}\Delta^rF(t),
\end{equation}
где
\begin{equation}\label{Shar_eq22}
Q_{r-1}(F,x)= F(0)+\frac{\Delta F(0)}{1!}x+\frac{\Delta^2 F(0)}{2!}
x^{[2]}+\ldots+\frac{\Delta^{r-1} F(0)}{(r-1)!}x^{[r-1]}.
\end{equation}
Так как для функции $F(t)=t^{[l+r]}$, где целое $l\ge0$, имеем $\Delta^r F(t)=(l+r)^{[r]}t^{[l]}$ и $Q_{r-1}(F,t)\equiv0$ , то из \eqref{Shar_eq21} следует, что
$$ \frac{1}{(r-1)!}\sum_{t=0}^{x-r} (x-1-t)^{[r-1]}t^{[l]}= $$
\begin{equation}\label{Shar_eq23}
= \frac{1}{(l+r)^{[r]}(r-1)!}\sum_{t=0}^{x-r} (x-1-t)^{[r-1]}\Delta^rF(t)=\frac{x^{[l+r]}}{(l+r)^{[r]}}.
\end{equation}
С другой стороны, для любого целого $l\geq 0$ функция $x^{[l+r]}$ обращается в нуль в узлах $x\in\{0,1,\ldots,r-1\}$. Поэтому полином $s_{r,k+r}^{\alpha}(x)$ также обращается в нуль при $x=0,1,\ldots,r-1$, так как в силу \eqref{Shar_eq19}, \eqref{Shar_eq16} и \eqref{Shar_eq17} его можно представить в виде линейной комбинации функций вида $x^{[l+r]}$. Таким образом, для полинома $s_{r,k}^{\alpha}(x)$, заданного при $k\geq r$ равенством \eqref{Shar_eq19}, имеет место равенство \eqref{2.2}, в котором вместо $\psi_{r,k}$ фигурирует $s_{r,k}^\alpha$. Поэтому из теоремы \ref{RItheo} и равенств \eqref{Shar_eq19}, \eqref{Shar_eq20} вытекает следующее соотношение ортогональности
$$ \langle s^\alpha_{r,n},s^\alpha_{r,m} \rangle =\sum_{k=0}^{r-1}\Delta^ks^\alpha_{r,n}(0)\Delta^ks^\alpha_{r,m}(0) + \sum_{j=0}^\infty\Delta^rs^\alpha_{r,n}(j)\Delta^rs^\alpha_{r,m}(j)\rho(j) = \delta_{nm}. $$
Тем самым, мы можем сформулировать следующий результат.
\begin{theorem}\label{Shar_thm2}
	Если $\alpha>0$, то система полиномов $s_{r,k}^{\alpha}(x)$ $(k=0, 1,\ldots)$, порожденная полиномами Шарлье $s_n^{\alpha}(x)$ $(n=0,1,\ldots)$ посредством равенств \eqref{Shar_eq19} и \eqref{Shar_eq20}, полна в $W^r_{l_\rho}$ и ортонормирована относительно скалярного произведения \eqref{Shar_eq2}.
\end{theorem}


\section{Дальнейшие свойства полиномов $s_{r,k}^{\alpha}(x)$ }


Перейдем к исследованию дальнейших свойств полиномов $s_{r,k}^{\alpha}(x)$. В первую очередь мы установим явный вид этих полиномов, представляющий собой разложение $s_{r,k}^{\alpha}(x)$ по обобщенным степеням $x^{[l]}$ $(l=r,r+1,\ldots,k)$.
\begin{theorem}\label{Shar_thm3}
	Для $\alpha>0$ имеют место равенства
	$$ 	s_{r,k+r}^{\alpha}(x) = \frac{1}{(h_n(\alpha))^{1/2}} \sum_{l=0}^{k} \frac{k^{[l]}x^{[l+r]}}{l!(l+r)^{[r]}} (-\alpha)^{-l}, \quad k=0,1,\ldots	$$
\end{theorem}

Теперь установим связь полиномов $s_{r,k}^{\alpha}(x)$ с порождающими их полиномами Шарлье $S_{k}^{\alpha}(x)$, которая не содержит знаков суммирования с переменным верхним пределом типа \eqref{Shar_eq19}. Имеет место следующая
\begin{theorem}\label{Shar_thm4}
	При $k\geq 0$ имеют место равенства
	\begin{equation}\label{Shar_eq24}
	s_{r,k+r}^{\alpha}(x)=\frac{(-\alpha)^r}{(k+r)^{[r]}} \left( \frac{\alpha^k}{k!} \right)^{\frac 12} \left[S_{k+r}^{\alpha}(x)-\sum_{\nu=0}^{r-1}\frac{(k+r)^{[\nu]}x^{[\nu]}}{(-\alpha)^\nu \nu!}\right] ,
	\end{equation}
	\begin{equation}\label{Shar_eq25}
	s_{r,k+r}^{\alpha}(x)=(-1)^r \left(\frac{\alpha^r}{(k+r)^{[r]}}\right)^{\frac 12}
	\left[s_{k+r}^{\alpha}(x) - \left( \frac{\alpha^{k+r}}{(k+r)!} \right)^{\frac12}\sum_{\nu=0}^{r-1}\frac{(k+r)^{[\nu]}x^{[\nu]}}{(-\alpha)^\nu \nu!}\right] .
	\end{equation}
\end{theorem}
%
%\section{Разностные свойства частичных сумм Фурье по системе $\{s_{r,k}^{\alpha}(x)\}_{k=0}^\infty $}
%
%Основные разностные свойства сумм Фурье по полиномам $s_{r,k}^{\alpha}(x)$, которые согласно \eqref{Shar_eq8} имеют вид
%$$ \mathcal{ Y}_{r,n}^{\alpha}(f,x)= \sum_{k=0}^{r-1}\Delta^kf(0)\frac{x^{[k]}}{k!} +\sum_{k=r}^{n}f_{r,k}s_{r,k}^{\alpha}(x),$$
%где
%$$ f_{r,k}= \langle f,s_{r,k}^{\alpha} \rangle =\sum_{j=0}^{\infty}\Delta^rf(j)s_{k-r}^{\alpha}(j)\rho(j),\, k=r,r+1,\ldots ,$$
%выражены равенствами \eqref{Shar_eq9}, \eqref{Shar_eq10} и \eqref{Shar_eq11}. Для системы
%$\{s_{r,k}^{\alpha}(x)\}_{k=0}^\infty$ они принимают вид $(0\le\nu\le r-1)$
%$$ \Delta^\nu f(x)= \sum_{k=0}^{r-\nu-1}\Delta^kf(0)\frac{x^{[k]}}{k!} +\sum_{k=r-\nu}^\infty f_{r,k+\nu} s_{r-\nu,k}^{\alpha}(x), $$
%$$ \Delta^\nu\mathcal{ Y}_{r,n}^{\alpha}(f,x)= \sum_{k=0}^{r-\nu-1}\Delta^kf(0)\frac{x^{[k]}}{k!} +\sum_{k=r-\nu}^{n-\nu} f_{r,k+\nu}s_{r-\nu,k}^{\alpha}(x), $$
%$$ \Delta^\nu\mathcal{ Y}_{r,n}^{\alpha}(f,x) = \mathcal{ Y}_{r-\nu,n-\nu}^{\alpha}(\Delta^\nu f,x). $$
%Из \eqref{Shar_eq9} и \eqref{Shar_eq10} мы также можем записать для $n\ge r>\nu\ge0$
%\begin{equation}\label{Shar_eq32}
%\Delta^\nu f(x)-\Delta^\nu\mathcal{ Y}_{r,n}^{\alpha}(f,x)= \sum_{k=n-\nu+1}^\infty f_{r,k+\nu} s_{r-\nu,k}^{\alpha}(x).
%\end{equation}
%Равенство \eqref{Shar_eq32} дает выражение для погрешности, возникающей в результате замены конечной разности $\Delta^\nu f(x)$ ее приближенным значением $\Delta^\nu\mathcal{ Y}_{r,n}^{\alpha}(f,x)$.




















