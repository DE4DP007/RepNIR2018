\section{Воссоздание вне родительской среды объектов трехмерного моделирования и выполнения преобразований над ними}

Решена задача воссоздания вне родительской среды объектов, созданных в среде трехмерного моделирования, и выполнения преобразований над ними.

\subsection{Воссоздание экспортированных объектов}

Разработана программа, осуществляющая воссоздание экспортированных трехмерных объектов.
Если исходные данные к программе являются компьютерными рельефами или иными сложными трехмерными объектами, то одним из наиболее распространенных способов их создания является разработка в среде <<Autodesk 3ds Max>> с последующим экспортом в файлы открытого формата <<.3ds>>. Сложность заключается как в воспроизведении этих трехмерных данных средствами среды программирования, так и в выполнении в интерактивном режиме тех или иных действий, не предусмотренных ранее при создании этих объектов в родительской среде.
Настоящая программа позволяет воссоздать экспортированные из среды <<Autodesk 3ds Max>> трехмерные объекты в проекте <<Delphi>> с применением <<OpenGL>> и управлять ими с помощью удобного интерфейса, не требующего от пользователя профессиональной подготовки.
Разработанный в программе интерфейс позволяет выбрать количество объектов и выполнить в интерактивном режиме действия: масштабирование, поворот вокруг выбранных осей и параллельный перенос, причем для каждого из объектов независимо.
Положительным свойством программы является и то, что перечисленные аффинные преобразования распространяются и на <<фоновый>> трехмерный рельеф, причем выполняются с высокой скоростью.
Дополнительные элементы, повышающие реалистичность вывода: звуковое сопровождение и динамическое изменение координат источника света.



\subsection{Интерактивное преобразование трехмерной модели смешанного графа}

Средствами открытой графической библиотеки <<OpenGL>> создается трехмерная модель графа, где допускаются как ориентированные, так и неориентированные ребра. Начальная ориентация ребер считывается из файла, как и исходные координаты вершин.
Координаты вершин допускают изменение в интерактивном режиме, программа обеспечивает выполнение <<вручную>> перемещения выбранных вершин без артефактов и с плавным сопровождением инцидентных ребер и числовых меток. При этом стрелки-указатели ориентации дуг перестраиваются корректно как по направлению, так и по размерам.
Для визуализации вершин используются сферы; изменение их стиля, радиуса, материала не представляет трудностей. Предусмотрено освещение сцены.
Цифровые метки вершин имеют трехмерный характер с масштабированием и изменяемой интенсивностью выдавливания. Цвета вершин, ребер и фона (по отдельности) могут быть изменены пользователем в процессе работы.
В методических целях программа полезна лекторам по дискретной математике; программа прошла аппробацию на лекциях по теории графов и на занятиях со слушателями ФПК. Программа может быть успешно использована студентами при выполнении лабораторных работ по теории графов, поскольку визуальная перестройка графа в ряде случаев способствует выявлению его существенных ха-рактеристик.




\subsection{Компьютерное доказательство отсутствия гармонической раскраски у специального двудольного $(6,3)$-бирегулярного графа}

Под гармонической раскраской двудольного $(6,3)$--бирегулярного графа $G=(X,Y,E)$ будем понимать отображение множества ребер $E$ в множество из двух цветов: $-1$ и $+1$, такое, что для каждой вершины $Y$ все $3$ инцидентных ребра --- одного цвета, а каждой вершине $X$ инцидентны по $3$ ребра каждого цвета.
Из результатов Asratian A.S., Casselgren C.J. (2007-2011 гг) известно, что задача эквивалентна $NP$-полной задаче о существовании реберной интервальной раскраски двудольного $(6,3)$--бирегулярного графа в 6 цветов.
Результат программы относится к тематике <<компьютерного>> решения математических проблем --- элиминацией перебора доказано отсутствие гармонической раскраски у двудольного $(6,3)$--бирегулярного графа, заданного следующими списками смежности:

 $$x1(y1,y2,y3,y4,y5,y6), \quad x2(y1,y2,y3,y4,y7,y8),$$

 $$x3(y5,y6,y7,y8,y9,y10), \quad x4(y1,y2,y3,y4,y11,y12),$$

 $$x5(y5,y6,y9,y14,y15,y16), \quad x6(y7,y8,y10,y13,y15,y16),$$

 $$x7(y9,y11,y12,y13,y17, 18), \quad x8(y10,y11, 12,y14,y17,y18),$$

 $$x9(y13, y14, y15, y16, y17, y18).$$

Актуальным является вопрос поиска минимального по числу вершин $N$ двудольного $(6,3)$--бирегулярного графа, не допускающего гармонической раскраски. Из результатов программы следует, что $N$ не превышает $27$. Дополнительными элементами программы являются вывод рисунка графа и его сохранение, а также визуализация промежуточных результатов вычислений.

