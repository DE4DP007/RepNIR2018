\chapter{Интерполяционные рациональные сплайны}

Вопросы сходимости полиномиальных сплайнов и их производных для классических
функциональных пространств исследованы в достаточно полной форме (см., например, \cite{ark-1}--\cite{ramazanov-6} и цитированную в них литературу).

Изучены также некоторые аппроксимативные свойства рациональных сплайнов специальных видов при
дополнительных ограничениях типа монотонности, выпуклости и др. (см., например, \cite{ark-7}--\cite{ark-10} и цитированные в них источники).

Нами построены двухточечные, трехточечные и четырехточечные рациональные интерполянты и на их основе
построены интерполяционные рациональные сплайны. Как показано (\cite{ark-11}--\cite{ark-13}), аппроксимативные свойства сплайнов по
двухточечным интерполянтам аналогичны свойствам полиномиальных сплайнов первой степени, а аппроксимативные свойства
по четырехточечным интерполянтам аналогичны свойствам кубических сплайнов. Что же касается сплайнов по трехточечным
рациональным интерполянтам, они обладают свойством, которое не наблюдается в случае гладких полиномиальных сплайнов,
а именно, последовательности таких рациональных сплайнов и их производных для любой последовательности сеток
узлов с диаметром, стремящимся к нулю, равномерно сходятся соответственно к самой функции в случае произвольной непрерывной
на данном отрезке функции и к производной функции в случае произвольной непрерывно дифференцируемой функции.

Даны оценки скорости сходимости рациональных сплайнов всех трех рассматриваемых видов к данной функции,
 а производных сплайнов -- к соответствующим производным
функции через модули непрерывности функции и соответствующей производной функции.

В случае рациональных сплайнов по трехточечным интерполянтам получена также оценка скорости их сходимости
к данной непрерывной функции через модуль непрерывности индуцированной функции.



\section{Аппроксимативные свойства сплайнов по рациональным интерполянтам}
%\begin{abstract}
%
%Построены двухточечные, трехточечные и четырехточечные рациональные интерполянты и на их базе --
%интерполяционные сплайны.
%Даны оценки скорости сходимости таких сплайнов и их производных.
%
%\end{abstract}
Ниже исследуется вопрос существования для произвольной непрерывной на данном отрезке
функции гладких сплайнов, последовательности которых для любой последовательности сеток
попарно различных узлов с диаметром, стремящимся к нулю, равномерно на всем отрезке сходится
к этой функции.

Как показано, этим свойством обладают рациональные сплайны по трехточечным интерполянтам
и не обладают построенные рациональные сплайны по четырехточечным интерполянтам.

Всюду ниже положим $h_k=x_k-x_{k-1}$ $(k=1,2,\dots,N)$, $\|\Delta\|=\max\{h_k:k=1,2,\dots,N\}$.

\subsection{Сплайны по двухточечным интерполянтам}

Пусть функция $f(x)$ непрерывна на отрезке $[a,b]$, на котором задана некоторая сетка узлов
$\Delta: a=x_0<x_1<\dots<x_N=b$ $(N\geqslant 1)$.

Для каждой пары узлов $x_{k-1}<x_k$ $(k=1,2,\dots,N)$ на отрезке $[x_{k-1}, x_k]$
рассмотрим рациональную функцию вида
\begin{equation}\label{1.1}
q_k(x)=q_k(x,H)=a_k+\frac{A_k}{x-u_k}
\end{equation}
с $u_k=x_k+H$ при $H>b-a$ такую, что $q_k(x_j)=f(x_j)$ при $j=k-1,k$.


Построим теперь по сетке узлов $\Delta$  непрерывную на отрезке $[a,b]$ кусочно--ра\-циональную
функцию $Q_N(x)$ такую, что $Q_N(x)=q_k(x)$ при $x\in [x_{k-1}, x_k]$ $(k=1,2,\dots, N)$.

Тогда для непрерывной на $[a,b]$ функции $f(x)$ имеет место неравенство
\begin{equation}\label{1.4}
|Q_N(x)-f(x)|\leqslant \omega (\|\Delta\|, f)\quad (x\in [a,b]).
\end{equation}


Если же  $f\in C^{(1)}[a,b]$, $H$ ---
любое наперед заданное число, большее $b-a$, $Q_N(x)$ --- кусочно--рациональная по сетке
$\Delta$ функция с $Q_N(x)=q_k(x, H)$ при $x\in [x_{k-1}, x_k]$ $(k=1,2,\dots,N)$, то
при всех натуральных $N$ выполняются неравенства
$$
\sup_{a\leqslant x\leqslant b} |Q_N^\prime (x\pm 0)-f^\prime (x)|\leqslant
 \omega (\|\Delta\|, f^\prime)+\frac{\|f^\prime\|}{H}\|\Delta\|,
$$
$$
\sup_{a\leqslant x\leqslant b} |Q_N (x)-f (x)|\leqslant
\frac{\|\Delta\|} 2 \omega (\|\Delta\|, f^\prime)+\frac{\|f^\prime\|}{2H}\|\Delta\|^2;
$$
здесь $\|f^\prime\|=\sup\limits_{a\leqslant x\leqslant b}|f^\prime(x)|$.

\subsection{Сплайны по трехточечным интерполянтам}

Пусть функция $f(x)$ непрерывна на отрезке $[a,b]$, на котором задана произвольная сетка узлов
$\Delta: a=x_0<x_1<\dots<x_N=b$ $(N\geqslant 2)$.

Тогда для любой тройки узлов $x_{i-1}<x_i<x_{i+1}$
$(i=1,2,\dots,{N-1})$ и любого фиксированного числа $g_i\not \in [x_{i-1}, x_{i+1}]$
существует (единственная) непрерывная на отрезке $[x_{i-1}, x_{i+1}]$
рациональная функция вида
\begin{equation}\label{2.1}
R_i (x)=\alpha_i+\beta_i(x-x_i)+\frac{\gamma_i}{x-g_i}
\end{equation}
такая, что
$$
R_i(x_j)=f(x_j)\quad (j=i-1,i,i+1).
$$

Ниже будем считать, что для каждой тройки узлов $x_{i-1}< x_i< x_{i+1}$
${(i=1,2,\dots,N-1)}$ выбираем
\begin{equation}\label{2.3}
g_i=\begin{cases}
2x_{i+1}-x_i \quad\text{ при }\quad x_{i+1}-x_i\leqslant x_i-x_{i-1},\\
2x_{i-1}-x_i \quad \text{ при }\quad x_{i+1}-x_i> x_i-x_{i-1}.
\end{cases}
\end{equation}


Будем считать $R_0(x)\equiv R_1(x)$, $R_N (x)\equiv R_{N-1}(x)$
(или в случае непрерывной $(b-a)$--периодической функции $f(x)$ сетку узлов $\Delta$
будем считать продолженной $(b-a)$--периодически).

При $x\in [x_{i-1}, x_i]$ $(i=1,2,\dots,N)$ положим
\begin{equation}\label{2.6}
R_{N, k}(x)=\frac{R_i(x) (x-x_{i-1})^k+R_{i-1} (x) (x_i-x)^k}
{(x-x_{i-1})^k+(x_i-x)^k}.
\end{equation}


При $k=1$ равенство \eqref{2.6} приобретает форму
$$
R_{N,1} (x)=R_i (x) \frac{x-x_{i-1}}{x_i-x_{i-1}}+R_{i-1} (x)\frac{x_i-x}{x_i-x_{i-1}}.
$$

Следующее утверждение лежит в основе безусловной сходимости рациональных сплайнов.


\begin{theorem}\label{teor2.1}
 Пусть данное натуральное число $N\geqslant 2$ и на отрезке
$[a,b]$ задана произвольная сетка узлов $\Delta: a=x_0<x_1<\dots<x_N=b$.

Тогда для любой непрерывной на отрезке $[a,b]$ функции $f(x)$ и интерполяционного
рационального сплайна $R_{N,k} (x)=R_{N,k}(x;f)$
при любом натуральном $k$ и всех $x\in [a,b]$ выполняется неравенство
$$
|f(x)-R_{N,k}(x)|\leqslant 19\,\omega (\|\Delta\|,f).
$$
 \end{theorem}


Следующее утверждение дает оценку скорости сходимости рациональных сплайнов по
трехточечным интерполянтам для непрерывно дифференцируемых функций.

\begin{theorem}\label{teor2.2} Пусть натуральное $N\geqslant 2$  и задана произвольная
сетка узлов $\Delta: a=x_0<x_1<\dots<x_N=b$.

Тогда для любой непрерывно дифференцируемой на отрезке $[a,b]$ функции $f(x)$
и интерполяционного рационального сплайна $R_{N,k}(x)=R_{N,k} (x; f)$ при
любом натуральном $k$ и всех $x\in [a,b]$ выполняются неравенства
$$
|f(x)-R_{N, k}(x)|\leqslant 16\, \|\Delta\| \,\omega (\|\Delta\|, f^\prime),
$$
$$
|f^\prime (x)-R^\prime_{N,k}(x)|\leqslant (20k+14) \omega (\|\Delta\|, f^\prime).
$$
\end{theorem}


Приведем также оценку скорости сходимости рациональных сплайнов по трехточечным интерполянтам для
 дважды непрерывно дифференцируемых функций.

\begin{theorem}\label{teor2.3}  Для произвольной сетки узлов $ \Delta: a=x_0<x_1<\dots<x_N=b$
 $(N \geqslant 2)$, любой дважды непрерывно дифференцируемой на отрезке $[a, b]$ функции $f(x)$
и интерполяционного рационального сплайна
$R_{N, k}(x)=R_{N, k} (x; f)$ при любом натуральном $k$ и всех
$x\in [a,b]$ выполняется неравенство
$$
|f(x)-R_{N,k}(x)|\leqslant 6\, \|\Delta\|^2
\max_{a\leqslant x \leqslant b}|f^{\prime\prime}(x)|.
$$
\end{theorem}

\subsection{Сплайны по четырехточечным интерполянтам}

Пусть функция $f(x)$ непрерывна на отрезке $[a,b]$, на котором задана
некоторая сетка узлов $\Delta: a=x_0<x_1<\dots<x_N=b$ $(N\geqslant 3)$.

Тогда для любой четверки узлов $x_{k-2}<x_{k-1}<x_k<x_{k+1}$ $(k=2,3,\dots, N-1)$
существует непрерывная на отрезке $[x_{k-2}, x_{k+1}]$ рациональная функция вида
\begin{equation}\label{3.1}
r_k(x)=a_k+b_k(x-x_k)+c_k(x-x_{k-1})(x-x_k)+\frac{A_k}{x-u_k}
\end{equation}
такая, что
$$
r_k(x_j)=f(x_j)\quad (j=k-2,k-1,k,k+1).
$$

Всюду ниже будем пользоваться также обозначениями:
$$
 \alpha_k=\min\{h_{k-1},h_k, h_{k+1}\},\quad
\beta_k=\max \{h_{k-1},h_k, h_{k+1}\}\quad (k=1,2,\dots,N-1);
$$
$$
g_k=\begin{cases} \max\{h_{k-1}, h_k\}, \text{ если } h_{k-1}<h_{k+1},\\
\max\{h_k, h_{k+1}\}, \text{ если } h_{k+1}\leqslant h_{k-1}.
\end{cases}
$$

Пользуясь  четырехточечными рациональными интерполянтами $r_k(x)$, построим
рациональные сплайны.


Полученные результаты сформулируем для случая непрерывной $(b-a)$--периодической функции
(результаты для непрерывных на отрезке $[a,b]$ функций вполне аналогичны периодическому случаю).
  Сетку узлов
 $\Delta: a=x_0<x_1<\dots<x_N=b$ продолжим также $(b-a)$--периодически и в соответствии
с этим распространим определение приведенных выше рациональных интерполянтов $r_k(x)$  на
узлы продолженной сетки.


Для каждого $k$ $(k=1,2,\dots,N)$ составим рациональную функцию
\begin{equation}\label{3.14}
\begin{array}{c}
Q_k(x)=r_k(x)+(r_{k-1}(x)-r_k(x))\dfrac{(x_k-x)^2}{(x_k-x_{k-2})(x_k-x_{k-1})}+\\
+(r_{k+1}(x)-r_k(x))\dfrac{(x-x_{k-1})^2}{(x_{k+1}-x_{k-1})(x_k-x_{k-1})},
\end{array}
\end{equation}
непрерывную на $[x_{k-1}, x_k]$, причем $Q_k(x_j)=f(x_j)$ при $j=k-1,k$.

Рассмотрим непрерывную на отрезке $[a,b]$ кусочно--рациональную функцию
$\rho_N(x)=\rho_N(x;f)$ $(N=3,4,\dots)$ такую, что при $x\in[x_{k-1}, x_k]$
$(k=1,2,\dots,N)$ выполняется равенство $\rho_N(x)=Q_k(x)$.


Тогда $\rho_N(x)$ $(N\geqslant 3)$ представляет собой дважды непрерывно дифференцируемую
на отрезке $[a,b]$ функцию с $\rho_N(x_k)=f(x_k)$ $(k=0,1,\dots,N)$
такую, что на частичном отрезке $[x_{k-1}, x_k]$ $(k=1,2,\dots, N)$ совпадает с рациональной функцией $Q_k(x)$.

Исследуя аппроксимативные свойства рациональных сплайнов $\rho_N(x)=\rho_N(x; f)$, в приводимых ниже
теоремах \ref{teor3.1}--\ref{teor3.3}
будем предполагать, что на отрезке $[a,b]$ задана произвольная сетка узлов
 $\Delta: a=x_0<x_1<\dots<x_N=b$ $(N\geqslant 3)$, продолженная $(b-a)$--периодически.

Следующее утверждение в принятых выше обозначениях дает оценку скорости сходимости сплайнов по
четырехточечным рациональным интерполянтам $r_k(x)$ в случае непрерывной функции.

\begin{theorem}\label{teor3.1}
Для непрерывной $(b-a)$--периодической функции $f(x)$ и рационального сплайна $\rho_N(x)=\rho_N(x;f)$
$(N\geqslant 3)$ при $x\in[a,b]$ выполняется неравенство
$$
|f(x)-\rho_N(x)|\leqslant 38\,\sup\left\{\frac{\beta_k}{\alpha_k}\omega(\beta_k, f):
 k=2,3,\dots, N-1\right\};
$$
в частности,
$$
|f(x)-\rho_N(x)|\leqslant 38\,\frac{\|\Delta\|}{\alpha}\,\omega(\|\Delta\|, f),
$$
где $\alpha=\min\{h_k: k=1, 2,\dots,N\}$.
\end{theorem}

\begin{theorem}\label{teor3.2}
Для непрерывно дифференцируемой $(b-a)$--периодической функции $f(x)$ и
 рационального сплайна $\rho_N(x)=\rho_N(x;f)$ $(N\geqslant 3)$ при $x\in[a,b]$ выполняются
неравенства
\begin{equation*}
|f(x)-\rho_N(x)|\leqslant \frac{57}2\, \|\Delta\| \omega(\|\Delta\|, f^\prime),
\end{equation*}
\begin{equation*}
|f^\prime(x)-\rho^\prime_N(x)|\leqslant 285 \,\omega(\|\Delta\|, f^\prime).
\end{equation*}
\end{theorem}

\begin{theorem}\label{teor3.3}
Для дважды непрерывно дифференцируемой $(b-a)$--периодической функции $f(x)$ и
 рационального сплайна $\rho_N(x)=\rho_N(x;f)$ $(N\geqslant 3)$ при $x\in[a,b]$ выполняются
неравенства
\begin{equation*}
|f(x)-\rho_N(x)|\leqslant \frac{39}2\, \|\Delta\|^2 \omega(\|\Delta\|, f^{\prime\prime}),
\end{equation*}

\begin{equation*}
|f^\prime(x)-\rho^\prime_N(x)|\leqslant 195\, \|\Delta\| \omega(\|\Delta\|, f^{\prime\prime}),
\end{equation*}

\begin{equation*}
|f^{\prime\prime}(x)-\rho^{\prime\prime}_N(x)|\leqslant 429 \,\omega(\|\Delta\|, f^{\prime\prime}).
\end{equation*}
\end{theorem}

Как видно из приведенных результатов, аппроксимативные свойства рациональных
 сплайнов по двухточечным интерполянтам аналогичны соответствующим свойствам полиномиальных сплайнов
первой степени.

Рациональные сплайны по трехточечным интерполянтам (в противоположность параболическим и кубическим сплайнам
минимального дефекта) обладают свойством безусловной сходимости, т.е. будучи гладкими сплайнами,
они для любой последовательности сеток узлов с диаметром, стремящимся к нулю, равномерно на всем отрезке
сходятся к самой непрерывной на этом отрезке функции.

Аппроксимативные свойства рациональных сплайнов по четырехточечным интерполянтам аналогичны свойствам
кубических сплайнов мимнимального дефекта.

\section{Оценка скорости сходимости рациональных сплайнов через индуцированные функции}
%\begin{abstract}
%
%Показано, что для произвольной непрерывной на данном отрезке функции  скорость равномерной
%сходимости к ней    рациональных сплайнов на базе трехточечных интерполянтов можно оценить
%через модуль непрерывности индуцированной функции.
%
%\end{abstract}
Исследована задача о возможности оценки скорости сходимости интерполяционных рациональных  сплайнов
в случае произвольной непрерывной на данном отрезке функции и произвольной сетки попарно различных узлов
через индуцированные функции. Это позволяет получить более точные оценки скорости сходимости сплайнов
для индивидуальных функций.

Для краткости ниже рассмотрен случай функций, непрерывных на отрезке $[-1,1]$. Полученный результат
легко переносится на случай функций, непрерывных на произвольном отрезке $[a,b]$.

\subsection{Оценка скорости сходимости сплайнов}

Пусть функция $f(x)$ непрерывна на отрезке $[-1,1]$, на котором задана произвольная сетка узлов
$\Delta: -1=x_0<x_1<\dots<x_N=1$ $(N\geqslant 2)$.

Для каждой тройки узлов $x_{i-1}<x_i<x_{i+1}$
$(i=1,2,\dots,{N-1})$  всюду ниже положим также
\begin{equation}\label{2.3}
g_i=\begin{cases}
2x_{i+1}-x_i \quad\text{ при }\quad x_{i+1}-x_i\leqslant x_i-x_{i-1},\\
2x_{i-1}-x_i \quad \text{ при }\quad x_{i+1}-x_i> x_i-x_{i-1}.
\end{cases}
\end{equation}


Тогда существует непрерывная на отрезке $[x_{i-1}, x_{i+1}]$
рациональная функция вида
\begin{equation}\label{2.1}
R_i (x)=\alpha_i+\beta_i(x-x_i)+\frac{\gamma_i}{x-g_i}
\end{equation}
такая, что
$$
R_i(x_j)=f(x_j)\quad (j=i-1,i,i+1).
$$

При этом для непрерывной на отрезке $[-1,1]$ функции $f(x)$ и рациональной функции $R_i(x)$
выполняется неравенство
$$
|f(x)-R_i(x)|\leqslant 19 \,\,\omega (\delta, f),
$$
где $\delta=\max\{h_i, h_{i+1}\}$.


Имеет место

\begin{theorem}\label{teor1}  Пусть на отрезке $[-1,1]$ задана произвольная сетка узлов
$ -1=x_0<x_1<\dots<x_N=1$  $(N \geqslant 2)$,
точки $t_j\in[0,\pi]$ и $\cos t_j=x_j$ $(j=0,1,\dots, N)$. Тогда
любой непрерывной на отрезке $[-1, 1]$ функции $f(x)$
и рационального интерполянта $R_i(x)$ $(i=1,2, \dots, N-1)$  при $x\in [x_{i-1},x_{i+1}]$
 выполняется неравенство
$$
|f(x)-R_i(x)|\leqslant 19\, \omega(h, F),
$$
где $h=\max\{t_{i-1}-t_i, t_i-t_{i+1}\}$, $F(t)=f(\cos t)$.
\end{theorem}

По трехточечным рациональным интерполянтам $R_i(x)$ $(i=1,2,\dots, N_1)$; $N\geqslant 2$
построим на отрезке $[-1,1]$ следующий рациональный сплайн:
$$
R_N(x; f)=R_i (x)\frac{x-x_{i-1}}{x_i-x_{i-1}}+R_{i-1}\frac{x_i-x}{x_i-x_{i-1}}
$$
(при $x\in [x_{i-1}, x_i]$); считаем $R_0(x)=R_1(x)$ и $R_N (x)=R_{N-1} (x)$.


Следующее утверждение существенно уточняет полученную выше оценку скорости сходимости
рациональных сплайнов по трехточечным интерполянтам в случае непрерывной на отрезке функции.

\begin{theorem}\label{teor2}   Для любой непрерывной на отрезке $[-1,1]$
функции $f(x)$, произвольной сетки узлов $ -1=x_0<x_1<\dots<x_N=1$  $(N \geqslant 2)$,
и интерполяционного рационального сплайна $R_N(x;f)$ при $x\in [-1,1]$ выполняется неравенство
$$
|f(x)-R_N(x;f)|\leqslant 19\, \omega(\tau, F),
$$
где $\tau=\max\{t_{j-1}-t_j: j=1,2,\dots, N\}$, $t_j\in [0,\pi]$ и $\cos t_j=x_j$ $(j=0,1,\dots, N)$,
$F(t)=f(\text{cos}\, t)$.

\end{theorem}

Теорема \ref{teor2} с помощью линейной замены переменной легко распространяется на функции $f(x)$,
непрерывные на произвольном конечном отрезке $[a,b]$.



%\section*{Заключение}







