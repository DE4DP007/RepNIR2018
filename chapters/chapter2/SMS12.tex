\chapter{Специальные вейвлеты на основе полиномов Чебышева второго рода}

%\textit{В работе рассмотрена ортогональная система вейвлетов и скалярных функций, основанных на полиномах Чебышева второго рода и их нулях. На их базе построена полная ортонормированная  система функций. Показан недостаток в аппроксимативных свойствах частичных сумм соответствующего вейвлет-ряда, связанный со свойствами самих полиномов Чебышева и заключающийся в существенном ухудшении скорости их сходимости к исходной функции на концах отрезка ортогональности. В качестве альтернативы предлагается модифицировать вейвлет-ряд Чебышева второго рода по аналогии со специальными рядами по ортогональным полиномам со свойством <<прилипания>> на концах отрезка ортогональности. На примере лакунарных частичных сумм доказано, что такой новый специальный вейвлет-ряд лишен указанного недостатка, а, следовательно, обладает более привлекательными аппроксимативными свойствами.}

%\section{Введение}
\section{Предварительные сведения}
На сегодняшний день вейвлеты зарекомендовали себя как достаточно эффективный инструмент
в задачах приближения функций, обработки и сжатия цифровых сигналов (временных рядов, звука, изображений и т.д.) (см., напр.,
\cite{sms1}\cite{sms2}\cite{sms3}).
В последние годы многими
авторами активно проводятся исследования теории полиномиальных вейвлетов.
Так Chui C.K. и Mhaskar H.N. в работе \cite{sms4}
впервые рассмотрели вейвлеты на основе тригонометрических полиномов.
Позднее, Kilgore T. и Prestin J. в \cite{sms5}
заменили тригонометрические полиномы алгебраическими и доказали ортогональность полученной системы функций в смысле чебышевского веса первого рода.
Далее, Fischer B. и Prestin J. в \cite{sms6},
разработали обобщенную теорию конструирования полиномиальных вейвлетов.
В дальнейшем техника разложения функций в ряды по полиномиальным вейвлетам получила развитие в ряде работ
(\cite{sms7}\cite{sms8}\cite{sms9} и др.).

Mohd~F. и Mohd~I. в \cite{sms10}
представили новый, отличный от описанных ранее, способ построения вейвлетов и масштабирующих функций на основе полиномов Чебышева первого рода и их нулей. Используя аналогичную технику, в \cite{sms11}
автором построена ортогональная система вейвлетов на основе полиномов Чебышева второго рода и исследованы аппроксимативные свойства лакунарных частичных сумм $\mathcal{V}_n(f,x)$ соответствующего вейвлет-ряда в случае $n = 2^m$.
В настоящей работе показан недостаток в свойствах сходимости частичных сумм $\mathcal{V}_n(f,x)$ к исходной функции $f(x)$, связанный со свойствами самих полиномов Чебышева второго рода. Предлагается модифицировать вейвлет-ряд по аналогии со специальными рядами по ортогональным полиномам со свойством <<прилипания>>, введенным в недавних работах Шарапудинова И.И.
(\cite{sob-leg-sharap3},\cite{sms12}). Доказано, что такой специальный вейвлет-ряд обладает значительно более привлекательными аппроксимативными свойствами на концах отрезка $[-1, 1]$.

%\section{Предварительные сведения}

Пусть $w(x) = \sqrt{1-x^2}$. Обозначим тогда через $L_{2, w}([-1; 1])$ евклидово пространство интегрируемых функций $f(x)$, таких что
$\int\limits_{-1}^{1} f^2(x)w(x)dx < \infty$.
Определим скалярное произведение в нем с помощью равенства
\begin{equation}
\label{scal}
<f, g> = \int\limits_{-1}^{1} f(x) g(x) w(x) dx.
\end{equation}
Хорошо известно, что полиномы Чебышева второго рода
 \begin{equation*}
\label{u2direct}
U_n(x) = \frac{\sin((n+1)\arccos{x})}{\sqrt{1-x^2}}, \quad n = 0,1,2, \ldots .
\end{equation*}
образуют ортогональный базис в $L_{2, w}([-1; 1])$, а именно
\begin{equation}
\label{sms1ortho}
<U_n, U_m> = \frac{\pi}{2}\delta_{nm} =
\left\{
\begin{aligned}
\frac{\pi}{2}, \quad n=m,\\
0, \quad  n \neq m,
\end{aligned}
\right.
\end{equation}
где $\delta_{nm}$ --- символ Кронекера.
Нули $n$-го полинома $U_n(x)$, очевидно, могут быть определены равенством
\begin{equation*}
\label{sms1zeros}
\xi_{k}^{(n)} = \cos{\theta_{k}^{(n)}} =  \cos{\frac{\pi (k+1)}{n+1}}, \quad (k = 0,...,n-1).
\end{equation*}



\section{Вейвлеты Чебышева второго рода}


\begin{definition}
%\noindent\textbf{Определение 1. }\textit{
Масштабирующей функцией Чебышева второго рода назовем полином вида
\begin{equation*}
\label{scaling}
\phi_{n,k}(x) = \sum\limits_{j=0}^{n}U_{j}(x)U_{j}(\xi_{k}^{(n+1)}),
\end{equation*}
где
$n=1,2,\ldots$ и $k=0,1,\ldots,n$.
%}
\end{definition}

\begin{definition}
%\noindent\textbf{Определение 2. }\textit{
Назовем вейвлет--функцией Чебышева второго рода полином
\begin{equation*}
\label{sms1wavelet}
\psi_{n,k}(x) = \sum\limits_{j=n+1}^{2n}U_{j}(x)U_{j}(\xi_{k}^{(n)}),
\end{equation*}
для любых
$n=1,2,\ldots$ и $k=0,1,\ldots,n-1$.
%}
\end{definition}

В работе \cite{sms11}
нами доказано, что системы масштабирующих функций  $\left\{ \phi_{n,k}(x)\right\}_{k=0}^{n}$ \linebreak и
вейвлет--функций $\left\{ \psi_{n,k}(x)\right\}_{k=0}^{n-1}$ являются ортогональными в $L_{2, w}([-1; 1])$, а именно
\begin{equation*}
<\phi_{n,k}, \phi_{n,l}> =
\frac{\pi(n+2)}{4\sin^2{\frac{\pi(k+1)}{n+2}}} \delta_{kl}, \qquad <\psi_{n,k}, \psi_{n,l}> =
\frac{\pi(n+1)}{4\sin^2{\frac{\pi(k+1)}{n+1}}} \delta_{kl}.
\end{equation*}

Положим тогда
\begin{equation*}
\hat{\phi}_{m,k}(x) = \frac{\phi_{2^m,k}(x)}{\sqrt{<\phi_{2^m,k}, \phi_{2^m,k}>}} = \phi_{2^m,k}(x)   \frac{2\left|\sin{\frac{\pi(k+1)}{2^m+2}}\right|}{\sqrt{\pi(2^m+2)}},
\end{equation*}
\begin{equation*}
\hat{\psi}_{m,k}(x) = \frac{\psi_{2^m,k}(x)}{\sqrt{<\psi_{2^m,k}, \psi_{2^m,k}>}} = \psi_{2^m,k}(x)   \frac{2\left| \sin{\frac{\pi(k+1)}{2^m+1}}\right|}{\sqrt{\pi(2^m+1)}}
\end{equation*}
и введем обозначения

$\Phi_{0} = \left\{\hat{\phi}_{0,0}(x), \hat{\phi}_{0,1}(x) \right\}$,
$\Psi_{1} =  \left\{\hat{\psi}_{0,0}(x) \right\}$,
$\Psi_{2} =  \left\{\hat{\psi}_{1,0}(x), \hat{\psi}_{1,1}(x) \right\}$, \ldots ,

$\Psi_{m} =  \left\{\hat{\psi}_{m-1,0}(x), \hat{\psi}_{m-1,1}(x), \ldots, \hat{\psi}_{m-1, 2^{m-1}-1}(x)  \right\}$,

$\Psi_{m+1} =  \left\{\hat{\psi}_{m,0}(x), \hat{\psi}_{m,1}(x), \ldots, \hat{\psi}_{m, 2^{m}-1}(x)  \right\}$, \ldots,

$\mathcal{P}_m = \left\{\Phi_{0}, \Psi_1, \Psi_2, \ldots, \Psi_m \right\} =
\left\{\hat{\phi}_{0,0}(x), \hat{\phi}_{0,1}(x), \hat{\psi}_{0,0}(x), \hat{\psi}_{1,0}(x), \hat{\psi}_{1,1}(x), \ldots, \right.$

$\left. \hat{\psi}_{m-1,0}(x), \hat{\psi}_{m-1,1}(x), \ldots, \hat{\psi}_{m-1, 2^{m-1}-1}(x) \right\}$.

Далее, пусть $H_{2^{m}, w}([-1; 1])$ --- подпространство в $L_{2, w}([-1; 1])$, состоящее из алгебраических полиномов степени не выше $2^{m}$. В \cite{sms11}
доказано, что система функций $\mathcal{P}_m$ образует ортонормированный базис в $H_{2^{m}, w}([-1; 1])$, т.е. любой полином $P_{n}(x) \in H_{2^{m}, w}([-1; 1])$ степени $n \leq 2^{m}$,
представим в виде линейной комбинации
\begin{equation}
\label{sms1polydistr}
P_{n}(x) = a_{0}\hat{\phi}_{0,0}(x) + a_{1}\hat{\phi}_{0,1}(x) + \sum\limits_{j=0}^{m-1} \sum\limits_{k=0}^{2^j-1} b_{j,k}\hat{\psi}_{j,k}(x).
\end{equation}

\noindent Кроме того, если рассмотреть теперь бесконечную систему функций

$\mathcal{P} = \left\{\Phi_{0}, \Psi_1, \Psi_2, \ldots, \Psi_m, \ldots \right\} =
\left\{\hat{\phi}_{0,0}(x), \hat{\phi}_{0,1}(x), \hat{\psi}_{0,0}(x), \hat{\psi}_{1,0}(x), \hat{\psi}_{1,1}(x), \ldots,\right.$

$\left. \hat{\psi}_{m-1,0}(x), \hat{\psi}_{m-1,1}(x), \ldots, \hat{\psi}_{m-1, 2^{m-1}-1}(x), \ldots \right\},$

\noindent то она образует ортонормированный базис в $L_{2, w}([-1; 1])$.


Из  последнего утверждения вытекает, что произвольная функция $f(x) \in L_{2, w}([-1; 1])$, может быть представлена в виде сходящегося в $L_{2, w}([-1; 1])$ ряда
\begin{equation}
\label{sms1fdistr}
f(x) = \hat{a}_{0}\hat{\phi}_{0,0}(x) + \hat{a}_{1}\hat{\phi}_{0,1}(x) + \sum\limits_{j=0}^{\infty} \sum\limits_{k=0}^{2^j-1} \hat{b}_{j,k}\hat{\psi}_{j,k}(x),
\end{equation}
\begin{equation*}
\label{sms1fcoeffA}
\text{где}\quad \hat{a}_{0} = \int\limits_{-1}^{1} f(t)\hat{\phi}_{0,0}(t)w(t)dt, \quad \hat{a}_{1} = \int\limits_{-1}^{1} f(t)\hat{\phi}_{0,1}(t)w(t)dt,
\end{equation*}
\begin{equation*}
\label{sms1fcoeffB}
\hat{b}_{j,k} = \int\limits_{-1}^{1} f(t)\hat{\psi}_{j,k}(t)w(t)dt, \quad (j=0,1, \ldots, m; k = 0, 1, \ldots, 2^j-1).
\end{equation*}
Через $\mathcal{V}_{2^m}(f,x)$ обозначим частичную сумму ряда \eqref{sms1fdistr} следующего вида
\begin{equation}
\label{sms1wavepartsum}
\mathcal{V}_{2^m}(f,x) = \hat{a}_{0}\hat{\phi}_{0,0}(x) + \hat{a}_{1}\hat{\phi}_{0,1}(x) + \sum\limits_{j=0}^{m-1} \sum\limits_{k=0}^{2^j-1} \hat{b}_{j,k}\hat{\psi}_{j,k}(x).
\end{equation}

\begin{remark} \label{sms_remark_1}
В силу равенства \eqref{sms1polydistr}, $\mathcal{V}_{2^m}(f, x)$ представляет собой линейный оператор, проектирующий пространство $L_{2, w}([-1; 1])$ на $H_{2^m, w}([-1; 1])$.
\end{remark}


\section{Аппроксимативные свойства частичных сумм $\mathcal{V}_{2^m}(f,x)$}

В работе \cite{sms11}
нами было показано, что частичные суммы \eqref{sms1wavepartsum} ряда \eqref{sms1fdistr} обладают теми же аппроксимативными свойствами, что и частичные суммы $S_{2^{m}}(f,x) = \sum_{k=0}^{n} \hat{f}_k U_k(x)$ ряда Фурье по полиномам Чебышева второго рода:
\begin{equation}
\label{sms1approxprops}
f(x) - \mathcal{V}_{2^m}(f,x) = f(x) - S_{2^{m}}(f,x).
\end{equation}

Опираясь на этот факт и используя оценки из \cite{sms14}\cite{sms15}
для каждой внутренней точки отрезка $x \in (-1, 1)$ получаем
\begin{equation}
\label{sms1v2mforC}
\left| f(x) - \mathcal{V}_{2^m}(f, x) \right| \leq
E_{2^{m}}(f) \left(\frac{4 \ln{2}}{\pi^2}m + O(1)\right),  \quad f \in C[-1,1],
\end{equation}
где $E_{2^m}(f)$ -- наилучшее приближение функции $f(x)$ алгебраическими полиномами \\ $p_n\in H_{2^{m}, w}([-1; 1])$. Кроме того, при $m \rightarrow \infty$ имеет  место равенство %%%pro lipschitza i (-1,1)
\begin{equation*}
\sup_{f \in \operatorname{Lip}{\alpha}, \atop 0 < \alpha < 1} \left| f(x) - \mathcal{V}_{2^m}(f,x) \right| =
\end{equation*}
\begin{equation}
\label{sms1lipshitz}
2^{-\alpha m} \left[
\frac{2^{\alpha+1}\ln{2}}{\pi} \left( 1-x^2 \right)^{\frac{\alpha}{2}}m
\int_{0}^{\frac{\pi}{2}} t^{\alpha} \sin{t} dt +
O \left( \frac{\sin{2^{m} \arccos {x}}}{\sqrt{1-x^2}} + 1\right)
\right].
\end{equation}

\begin{remark} \label{sms_remark2}
В случае $\alpha = 1$ последнее равенство выполняется равномерно на всем отрезке $[-1, 1]$.
\end{remark}

В то же время можно привести простой пример аналитической функции $f(x)$, для которой имеет место неравенство
\begin{equation}
\label{sms15.2.16}
\frac{|f(\pm1)-\mathcal{V}_{2^m}(f, \pm1)|}{ E_{2^m}(f)}\ge
c_1 2^m.
\end{equation}
Для этого обратимся к обобщению полиномов Чебышева второго рода --- ультрасферическим полиномам $P_n^{\alpha, \alpha}(x)$ ($n =0,1,2,\ldots$; $\alpha > -1$), образующим ортогональную систему в пространстве $L_{2, \rho}([-1,1])$, где $\rho(x)=(1-x^2)^{\alpha}$.
Рассмотрим известное разложение (см. \cite{Haar-Tcheb-Sege})
производящей функции для них
\begin{equation}
\label{sms15.2.17}
\sum_{n=0}^\infty\frac{\Gamma(\alpha)}{\Gamma(2\alpha+1)}\frac{
     \Gamma(n+2\alpha+1)}{\Gamma(n+\alpha)}P_n^{\alpha,\alpha}(x)
     z^n=(1-2xz+z^2)^{-\alpha-\frac12}.
\end{equation}
При $0<z<1$ функция
\begin{equation}
\label{sms15.2.18}
f(x)=f_z(x)=(1-2xz+z^2)^{-\alpha-\frac12}=(2z)^{-\alpha-\frac12}
     \left({\frac{1+z^2}{2z}}-x\right)^{-\alpha-\frac12}
\end{equation}
аналитична  на всей плоскости  разрезанной вдоль положительной полуоси с началом в точке $x=\frac{1+z^2}{2z}=a>1$. Известно (см. \cite{sob-jac-discrete-Timan}),
что
\begin{equation}
 E_n[(a-x)^{-\alpha-\frac12}]\asymp \frac{n^{\alpha+\frac12}}{
\Gamma\left(\alpha+\frac12\right)(\sqrt{a^2-1})^{\alpha+\frac32}(a+\sqrt{a^2-1})^n}
     \asymp n^{\alpha+\frac12}z^n.
\end{equation}
Отсюда находим
\begin{equation}
\label{sms15.2.19}
E_n(f)\asymp n^{\alpha+\frac12}z^n.
\end{equation}
С другой стороны, как показано в работе \cite{sob-leg-sharap2},
имеет место
\begin{equation}
\label{sms15.2.20}
|f(\pm1)- S_n^{\alpha,\alpha}(f,\pm1)| \asymp n^{2\alpha+1}z^n.
\end{equation}
Сопоставляя \eqref{sms15.2.19} и \eqref{sms15.2.20}, получаем
\begin{equation*}
\label{sms15.2.21}
{|f(1)-S_n^{\alpha,\alpha}(f, 1)|\over E_n(f)}\ge
c(\alpha)n^{\alpha+\frac12}, \quad (\alpha>-1/2).
\end{equation*}
В интересующем нас частном случае, при $\alpha=\frac12$ и $n = 2^m$, это неравенство принимает вид
\begin{equation}
\label{sms15.2.16.1}
{|f(\pm1)-S_{2^m}(f, \pm1)|\over E_{2^m}(f)} \ge c_1 2^m.
\end{equation}
Из \eqref{sms1approxprops} и \eqref{sms15.2.16.1} приходим к справедливости \eqref{sms15.2.16}.


Приведенный пример показывает, что частичные суммы $\mathcal{V}_{2^m}(f, x)$ на концах отрезка $[-1,1]$ плохо приближают не только непрерывные функции $f \in C[-1,1]$, но также аналитические функции (за исключением алгебраических полиномов).
Может случится, что $\mathcal{V}_{2^m}(f, x)$ приближает $f(x)$ по порядку в $2^m$ раз хуже, чем полином наилучшего приближения $P^{*}_{2^m}(f,x)$.
Этот отрицательный факт является следствием того, что функция Лебега для сумм Фурье-Чебышева второго рода
\begin{equation*}
L_{n}(x) = \int\limits_{-1}^{1}  \left|\sum\limits_{k=0}^{n}\hat U_{k}(x)\hat U_{k}(t)\right| w(t)dt
\end{equation*}
в точках $x = \pm1$ имеет порядок роста, равный $n$ $(n\to\infty)$ (см., напр., \cite{sms14}).

\section{Специальный вейвлет-ряд со свойством <<прилипания>>}

В нашей работе \cite{sms112} для того чтобы устранить указанный негативный эффект, предлагается модифицировать вейвлет-ряд \eqref{sms1fdistr} по схеме, схожей с построением введенных в недавних работах \cite{sms12,sob-leg-sharap3}
предельных и специальных рядов по ультрасферическим полиномам, обладающих свойством <<прилипания>> на концах отрезка ортогональности. Следуя \cite{sob-leg-sharap3},
введем в рассмотрение функцию

\begin{equation}
\label{sms1Ffunk}
F(x) = \frac{f(x)-c(f,x)}{1-x^2} = \frac{g(f,x)}{1-x^2},
\end{equation}
\begin{equation*}
\label{sms1af}
\text{где}\quad c(f,x) = \frac{f(-1) + f(1)}{2} - \frac{f(-1) - f(1)}{2} x.
\end{equation*}
Если $f(x) \in L_{2, w}([-1; 1])$ то, очевидно, также и $F(x) \in L_{2, w}([-1; 1])$. Тогда она может быть представлена в виде ряда
\begin{equation}
\label{sms1eFdistr}
F(x) = \tilde{a}_{0}\hat{\phi}_{0,0}(x) + \tilde{a}_{1}\hat{\phi}_{0,1}(x) + \sum\limits_{j=0}^{\infty} \sum\limits_{k=0}^{2^j-1} \tilde{b}_{j,k}\hat{\psi}_{j,k}(x),
\end{equation}
\begin{equation*}
\label{sms1efcoeffA0}
\text{где}\quad \tilde{a}_{0} = \int\limits_{-1}^{1} F(t)\hat{\phi}_{0,0}(t) w(t)dt,
\end{equation*}
\begin{equation*}
\label{sms1efcoeffA1}
\tilde{a}_{1} = \int\limits_{-1}^{1} F(t)\hat{\phi}_{0,1}(t) w(t) dt,
\end{equation*}
\begin{equation*}
\label{sms1efcoeffB}
\tilde{b}_{j,k} = \int\limits_{-1}^{1} F(t)\hat{\psi}_{j,k}(t) w(t) dt, \quad (j=0,1, \ldots, m; k = 0, 1, \ldots, 2^j-1).
\end{equation*}
Выразим теперь из \eqref{sms1Ffunk} и \eqref{sms1eFdistr} исходную функцию
\begin{equation*}
\label{sms1fFunk}
f(x)=
c(f,x)+(1-x^2) \left[ \tilde{a}_{0}\hat{\phi}_{0,0}(x) + \tilde{a}_{1}\hat{\phi}_{0,1}(x) + \sum\limits_{j=0}^{\infty} \sum\limits_{k=0}^{2^j-1} \tilde{b}_{j,k}\hat{\psi}_{j,k}(x) \right].
\end{equation*}
Будем называть такой модифицированный ряд \textit{специальным вейвлет-рядом Чебышева второго рода}. Обозначим его частичную сумму
\begin{equation}
\label{sms1fFunkSum}
\tilde{\mathcal{V}}_{2^m}(f,x)  = c(f,x)+(1-x^2) \left[ \tilde{a}_{0}\hat{\phi}_{0,0}(x) + \tilde{a}_{1}\hat{\phi}_{0,1}(x) + \sum\limits_{j=0}^{m-1} \sum\limits_{k=0}^{2^j-1} \tilde{b}_{j,k}\hat{\psi}_{j,k}(x) \right].
\end{equation}

\noindent Рассмотрим некоторые свойства $\tilde{\mathcal{V}}_{2^m}(f,x)$. Из явного вида \eqref{sms1fFunkSum} легко следует

\begin{theorem} \label{sms_th1}
  Частичная сумма $\tilde{\mathcal{V}}_{2^m}(f,x)$ на концах отрезка $[-1,1]$ совпадает с исходной функцией $f(x)$, т.е. $\tilde{\mathcal{V}}_{2^m}(f,\pm1) = f(\pm1)$.
\end{theorem}

\noindent Как отмечено в \cite{sob-leg-sharap3},
это свойство имеет важное значение в задачах, связанных с обработкой временных рядов и изображений.

\begin{theorem} \label{sms_th2}
  Частичная сумма $\tilde{\mathcal{V}}_{2^m}(f,x)$ представляет собой линейный оператор, проектирующий пространство $L_{2, w}([-1; 1])$ на $H_{2^m+2, w}([-1; 1])$, т.е. для любого полинома $P_n(x)$ степени не выше $n\le 2^{m}+2$ справедливо равенство $\tilde{\mathcal{V}}_{2^m}(P_n,x) \equiv P_n(x)$.
\end{theorem}


\begin{theorem} \label{sms_th3}
  Для любой функции $f\in C[-1,1]$ и любого $x \in (-1, 1)$ имеет место оценка
\begin{equation*}
\label{sms1thrm3eq}
|f(x)-\tilde{\mathcal{V}}_m(f,x)|\le c E_{2^{m}+2}(f)(1+\ln(1+(2^{m}+2)\sqrt{1-x^2})),
\end{equation*}
где $c >0$ -- константа.
\end{theorem}


Рассмотрим теперь верхнюю грань отклонения частичных сумм $\tilde{\mathcal{V}}_{2^m}(f,x)$ от функций из класса Липшица $\operatorname{Lip} \alpha$ $(0< \alpha <1)$.

\begin{theorem} \label{sms_th4}
  Для каждой внутренней точки отрезка $x \in (-1, 1)$ при $m \rightarrow \infty$ справедливо асимптотическое равенство
\begin{equation*}
\sup_{f \in \operatorname{Lip}{\alpha}, \atop 0 < \alpha < 1} \left| f(x) - \tilde{\mathcal{V}}_{2^m}(f,x) \right| =
\end{equation*}
\begin{equation*}
\label{sms1thrm4eq}
2^{-\alpha m} \left[
\frac{2^{\alpha+1}\ln{2}}{\pi} \left( 1-x^2 \right)^{\frac{\alpha}{2}+1}m
\int_{0}^{\frac{\pi}{2}} t^{\alpha} \sin{t} dt +
O \left( \frac{\sin{2^{m} \arccos {x}}}{\sqrt{1-x^2}} + 1\right)
\right].
\end{equation*}
\end{theorem}

\begin{remark} \label{remark2}
Из замечания \ref{remark2} следует, что в случае $\alpha = 1$ утверждение теоремы \ref{sms_th4} справедливо на всем отрезке $[-1, 1]$.
\end{remark}



Теоремы \ref{sms_th1}, \ref{sms_th2}, \ref{sms_th3}, \ref{sms_th4} показывают, что частичные суммы $\tilde{\mathcal{V}}_{2^m}(f,x)$ как аппарат приближения обладают весьма привлекательными аппроксимативными свойствами.
