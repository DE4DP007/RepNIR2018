\documentclass[12pt]{article}
\usepackage[utf8]{inputenc}  %% - кодировка символов (UTF-8)
%\usepackage{graphicx,bm,makeidx}
%\usepackage{wrapfig}  %% - обтекаемые рисунки
%\usepackage[small,labelsep=period,belowskip=1ex]{caption} %% - стиль подписей к рисункам
\usepackage[a4paper,hmargin=2cm,vmargin=2.5cm,includefoot,nohead,nomarginpar]{geometry} %% - параметры страницы
\usepackage[russian]{babel}  %% - русификация
\usepackage[shortlabels]{enumitem} %% - оформление нумерованных списков
\usepackage{amsmath, amsfonts, amssymb, dsfont}
\usepackage{hyperref}
\usepackage{amsthm}
%\usepackage{mathtools}
%\mathtoolsset{showonlyrefs}
%\setenumerate{itemsep=-0.5em,leftmargin=\parindent,labelsep=0.5ex,label=\arabic*.}

\newtheorem{theoremrus}{Теорема}
\newtheorem{lemmarus}{Лемма}
\newtheorem{statementrus}{Утверждение}
\newtheorem{remarkrus}{Замечание}
\newtheorem{corollaryrus}{Следствие}

\DeclareMathOperator*{\sign}{sign}
\newcommand{\suml}{\sum\limits}
\newcommand{\intl}{\int\limits}

\bibliographystyle{unsrt}

\begin{document}
\section{Введение 1}
Были проведены исследования аппроксимативных свойств дискретных сумм Фурье $L_{n,N}(f,x)$ для функций
$f_1(x) = |x|$ и $f_2(x) = \sign x$ на отрезке $[-\pi, -\pi]$. Другими словами, оценивались следующие величины
$|L_{n,N}(f_1,x) - |x||$
и
$|L_{n,N}(f_2,x) - \sign x|$.
Аппроксимативные свойства оценивались как на всём отрезке $[-\pi,\pi]$ (для функции $|x|$), так и на множестве $[-\pi+\varepsilon, -\varepsilon]\cup[\varepsilon,\pi-\varepsilon]$,
$0 < \varepsilon < \frac{\pi}{2}$, где исключены, вместе с некоторой окрестностью, точки, в которых функции $\sign x$ и $|x|$ не имеют производных.

\section{Введение 2}
Были проведены исследования аппроксимативных свойств сумм Фурье для непрерывных $2\pi$-периодических ломаных. Отдельно рассматривались ломаные, вписанные в заданную функцию  из пространства Соболева $W_{2\pi}^{2,1}$. Были получены оценки аппроксимативных свойств как на всей числовой оси, так и для числовой оси, исключая, вместе с некоторой окрестностью, точки, в которых определены вершины ломаных.

\section{Реферат 1}
Была исследована скорость приближения функций $f_1(x) = |x|$ и $f_2(x) = \sign x$ на отрезке $[-\pi, -\pi]$ дискретными суммами Фурье $L_{n,N}(f,x)$, то есть были оценены величины
$|L_{n,N}(f_1,x) - |x||$
и
$|L_{n,N}(f_2,x) - \sign x|$.
Скорость приближения оценена как на всём отрезке $[-\pi,\pi]$ (для функции $|x|$), так и на множестве $[-\pi+\varepsilon, -\varepsilon]\cup[\varepsilon,\pi-\varepsilon]$,
$0 < \varepsilon < \frac{\pi}{2}$, где исключены, вместе с некоторой окрестностью, точки, в которых функции $\sign x$ и $|x|$ не имеют производных.

\section{Реферат 2}
Была исследована скорость приближения суммами Фурье непрерывных $2\pi$-периодических ломаных. Отдельно рассматривались ломаные, вписанные в заданную функцию  из пространства Соболева $W_{2\pi}^{2,1}$. Были получены оценки аппроксимативных свойств как на всей числовой оси, так и для всей оси, исключая, вместе с некоторой окрестностью, точки, в которых определены вершины ломаных.

\section{Заключение 1}
Были исследованы аппроксимативные свойства дискретных сумм Фурье $L_{n,N}(f,x)$ для функций
$f_1(x) = |x|$ и $f_2(x) = \sign x$ на отрезке $[-\pi, -\pi]$. Были оценены величины
$|L_{n,N}(f_1,x) - |x||$
и
$|L_{n,N}(f_2,x) - \sign x|$ как на всём отрезке $[-\pi,\pi]$ (для функции $|x|$), так и на множестве $[-\pi+\varepsilon, -\varepsilon]\cup[\varepsilon,\pi-\varepsilon]$,
$0 < \varepsilon < \frac{\pi}{2}$, где исключены, вместе с некоторой окрестностью, точки, в которых функции $\sign x$ и $|x|$ не имеют производных. Было показано, что аппроксимативные свойства для функции $|x|$ в точках $0$, $-\pi$ и $\pi$ на порядок хуже, чем между этими точками.

\section{Заключение 2}
Были исследованы аппроксимативные свойства сумм Фурье для непрерывных $2\pi$-периодических ломаных. Отдельно рассматривались ломаные, вписанные в заданную функцию из пространства Соболева $W_{2\pi}^{2,1}$. Были получены оценки аппроксимативных свойств как на всей числовой оси, так и для всей оси, исключая, вместе с некоторой окрестностью, точки, в которых определены вершины ломаных. Было показано, что аппроксимативные свойства на порядок лучше между вершинами ломаных, чем в самих вершинах.
Также была получена оценка аппроксимативных свойств сумм Фурье для ломаных, вписанных в некоторую функцию из пространства Соболева $W_{2\pi}^{2,1}$ и показано, что полученная оценка не улучшаема по порядку.
\end{document}
