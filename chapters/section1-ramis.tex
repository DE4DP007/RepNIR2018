\chapter{Приближение функций, заданных на сетке $\{0, \delta, 2\delta, \ldots\}$ суммами Фурье-Мейкснера}


Пусть $\Omega_{\delta}=\{0, \delta, 2\delta, \ldots\}$, где $\delta=\frac{1}{N}$, $N>0$.
Через $M_{n,N}^\alpha(x)$ $(n=0, 1, \dots)$ обозначим, следуя \cite{Ram1}, модифицированные полиномы Мейкснера, образующие при $\alpha>-1$ ортогональную систему на дискретном множестве $\Omega_{\delta}$ с весом
$w(x)=e^{-x}\frac{\Gamma(Nx+\alpha+1)}{\Gamma(Nx+1)},
$
то есть
$$
\sum_{x\in\Omega_{\delta}} M_{n,N}^\alpha(x)M_{k,N}^\alpha(x)w(x)=(1-e^{-\delta})^{-\alpha-1}h_{n,N}^\alpha\delta_{nk}, \quad 0<q<1,
\ \alpha>-1,
$$
где $h_{n,N}^\alpha={n+\alpha\choose n}e^{-n\delta}\Gamma(\alpha+1),$
а соответствующие ортонормированные с весом $\rho_N(x)=(1-e^{-\delta})^{\alpha+1}w(x)$ полиномы обозначим через $m_{n,N}^\alpha(x)=(h_{n,N}^\alpha)^{-1/2}M_{n,N}^\alpha(x)$ $(n=0, 1, \dots)$.
Для функции $f$, заданной на множестве $\Omega_{\delta}$, рассмотрена задача об исследовании аппроксимативных свойств частичных сумм $S_{n,N}^{\alpha}(f,x)$ ее ряда Фурье по полиномам $m_{n,N}^\alpha(x)$. Основное внимание уделено получению верхней оценки для функции Лебега $\lambda_{n,N}^{\alpha}(x)$ указанных частичных сумм при $x\in[0,\frac{\theta_n}{2}]$, где $\theta_n=4n+2\alpha+2$.



\section{Некоторые сведения о полиномах Мейкснера}

Пусть $\alpha$ произвольное действительное число и $q\ne0$. Тогда для классических полиномов Мейкснера $M_{n}^{\alpha}(x)=M_{n}^{\alpha}(x,q)$ имеют место \cite{Ram1, Ram2, Ram3}:

явное представление
\begin{equation*}
M_n^\alpha(x)=M_n^\alpha(x,q)={n+\alpha\choose n}\sum_{k=0}^n{n^{[k]}x^{[k]}\over(\alpha+1)_kk!}\left(1-{1\over q}\right)^k,
\end{equation*}
где $x^{[k]}=x(x-1)\ldots (x-k+1)$, $(a)_k=a(a+1)\ldots(a+k-1)$;

соотношение ортогональности
\begin{equation*}
\sum_{x=0}^\infty M_n^\alpha(x)M_k^\alpha(x)\rho(x)=(1-q)^{-\alpha-1}h_n^\alpha(q)\delta_{nk}, \quad 0<q<1,\ \alpha>-1,
\end{equation*}
где  $\rho(x)=\rho(x;\alpha,q)=q^{-x}\frac{\Gamma(x+\alpha+1)}{\Gamma(x+1)}$,
$h_n^\alpha(q)={n+\alpha\choose n}q^{-n}\Gamma(\alpha+1)$,
$\delta_{nk}$ -- символ Кронекера;

формула Кристоффеля-Дарбу
$$
K_{n}^{\alpha,q}(x,y)=\sum_{k=0}^nq^k{k!\over
\Gamma(k+\alpha+1)}M_k^\alpha(x)M_k^\alpha(y)
$$
\begin{equation}\label{ForKris}
={(n+1)!q^{n+1}\over
\Gamma(n+\alpha+1)(q-1)}{M_{n+1}^\alpha(x)M_n^\alpha(y)-
M_n^\alpha(x)M_{n+1}^\alpha(y)\over x-y}.
\end{equation}


Пусть $N>0$,   $\delta=1/N$,   $q=e^{-\delta}$.
Через $M_{n,N}^\alpha(x)=M_n^\alpha(Nx,e^{-\delta})$ $(n=0, 1, \dots)$ обозначим, следуя \cite{Ram1}, модифицированные полиномы Мейкснера, которые при $\alpha>-1$ образуют ортогональную на $\Omega_{\delta}=\{0, \delta, 2\delta, \ldots\}$ систему с весом
$\rho_N(x)=(1-e^{-\delta})^{\alpha+1}\rho(Nx;\alpha,e^{-\delta})$, т.е.,
\begin{equation}\label{orth}
\sum_{x\in\Omega_{\delta}} M_{n,N}^\alpha(x)M_{k,N}^\alpha(x)\rho_N(x)=h_{n,N}^\alpha\delta_{nk}, \quad 0<q<1,
\ \alpha>-1,
\end{equation}
где $h_{n,N}^\alpha={n+\alpha\choose n}e^{n\delta}\Gamma(\alpha+1).$
Из \eqref{orth} следует, что соответствующая ортонормированная система полиномов имеет вид $m_{n,N}^\alpha(x)=m_n^\alpha(Nx,e^{-\delta})=\linebreak(h_{n,N}^{\alpha})^{-1/2}M_{n,N}^\alpha(x)$.

Далее, пусть
\begin{equation}\label{gadz-eq2}
 K_{n,N}^{\alpha}(t,x)=\sum_{k=0}^n m_{k,N}^{\alpha}(t)m_{k,N}^{\alpha}(x).
\end{equation}
Тогда в силу формулы Кристоффеля-Дарбу \eqref{ForKris} мы можем
записать
$$
K_{n,N}^\alpha (x,y)={\delta\sqrt{(n+1)(n+\alpha+1)}\over
(e^{\delta/2}-e^{-\delta/2})(y-x)}\times
$$
$$
\left[m_{n+1,N}^\alpha (x) m_{n,N}^\alpha (y)-
m_{n,N}^\alpha (x) m_{n+1,N}^\alpha (y) \right]
$$
Полагая здесь $\alpha_{n}=\sqrt{(n+1)(n+\alpha+1)}$, имеем
$$
{1\over \alpha_{n}} K_{n,N}^\alpha (x,y) =
{\delta\over (e^{\delta/2}-e^{-\delta/2})(y-x)}\times
$$

\begin{equation}\label{Kernel}
\left[m_{n+1,N}^\alpha(x)
m_{n,N}^\alpha(y)-m_{n,N}^\alpha(x)m_{n+1,N}^\alpha(y) \right],
\end{equation}

$$
{1\over \alpha_{n-1}} K_{n,N}^{\alpha}(x,y)=
{\delta\over (e^{\delta/2}-e^{-\delta/2})(y-x)}\times
$$

\begin{equation}\label{Kernel1}
\left[m_{n,N}^{\alpha}(x)
m_{n-1,N}^{\alpha}(y)-m_{n-1,N}^{\alpha}(x)
m_{n,N}^{\alpha}(y) \right]+
{1\over {\alpha}_{n-1}}m_{n,N}^{\alpha}(x)m_{n,N}^{\alpha}(y)
\end{equation}
Складывая правые и левые части равенств \eqref{Kernel} и \eqref{Kernel1} имеем
$$
\left({1\over\alpha_n}+ {1\over\alpha_{n-1}}\right)
K_{n,N}^\alpha(x,y)={1\over\alpha_{n-1}}m_{n,N}^\alpha(x)
m_{n,N}^\alpha(y)+
$$

$$
{\delta\over(e^{\delta/2}-e^{-\delta/2})(y-x)}
\left[m_{n,N}^\alpha(y)\left(m_{n+1,N}^\alpha(x)-m_{n-1,N}^\alpha(x)
\right)-\right.
$$

$$
\left.m_{n,N}^\alpha(x)\left(m_{n+1,N}^\alpha(y)-
m_{n-1,N}^\alpha(y)\right)\right],
$$
стало быть
$$
K_{n,N}^\alpha(x,y)={\alpha_n\over(\alpha_n+\alpha_{n-1})}
m_{n,N}^{\alpha}(x)m_{n,N}^{\alpha}(y)+
$$

$$
{\alpha_n\alpha_{n-1}
\over(\alpha_n+\alpha_{n-1})}{\delta\over(e^{\delta/2}-e^{-\delta/2})} {1\over(y-x)}\times
$$

$$
\left[m_{n,N}^\alpha(y)\left(m_{n+1,N}^\alpha(x)-
m_{n-1,N}^\alpha(x)
\right)\right.
$$

\begin{equation*}\label{Kernel2}
\left.-m_{n,N}^\alpha(x)
\left(m_{n+1,N}^\alpha(y)-m_{n-1,N}^\alpha(y)
\right)\right]
\end{equation*}

Для $0<\delta\le1$, $N={1\over\delta}$, $\lambda>0$,\ $1\le n\le\lambda N$,  $\alpha>-1$, $0\le x<\infty$, $\theta_n=4n+2\alpha+2$ справедливы \cite{Ram1}, \cite{Ram5} следующие оценки:
$$
e^{-{x\over2}}\left|m_{n,N}^\alpha(x)\right|\le c(\alpha,\lambda)\theta_n^{-{\alpha\over2}}A_n^\alpha(x),\
$$
\begin{equation*}
A_n^\alpha(x)=\begin{cases}
\theta_n^{\alpha},&  0\le x\le \frac{1}{\theta_n},\\
\theta_n^{\alpha/2-1/4}x^{-\alpha/2-1/4},&     \frac{1}{\theta_n}<x\le {\theta_n\over 2},\\
\left[\theta_n(\theta_n^{1\over3}+|x-\theta_n)\right]^{-\frac{1}{4}},& {\theta_n\over2}<x\leq{3\theta_n\over2},\\
e^{-{x/4}}, & {3\theta_n\over2}<x<\infty,
\end{cases}
\end{equation*}
$$
e^{-{x\over2}}\left|m_{n+1,N}^{\alpha}(x)-m_{n-1,N}^{\alpha}(x)\right|\leq
$$
$$
c(\alpha,\lambda)\begin{cases}
\theta_n^{{\alpha\over2}-1},&  0\le x\le \frac{1}{\theta_n},\\
\theta_n^{-{3\over4}}x^{-{\alpha\over2}+{1\over4}},&     \frac{1}{\theta_n}<x\le {\theta_n\over 2},\\
x^{-{\alpha\over2}}\theta_n^{-{3\over4}}\left[(\theta_n^{1\over3}+|x-\theta_n)\right]^{1\over4},& {\theta_n\over2}<x\leq{3\theta_n\over2},\\
e^{-{x\over4}}, & {3\theta_n\over2}<x<\infty,
\end{cases}
$$
где здесь и далее $c$, $c(\alpha)$, $c(\alpha, \ldots,\lambda)$ -- положительные числа, зависящие только от указанных параметров, причем различные в разных местах.

\section{Неравенство Лебега для частичных сумм Фурье-Мейкснера}

Обозначим через $C(\Omega_\delta)$ пространство дискретных функций $f(x)$ вида $f:\Omega_\delta\rightarrow\mathbb{R}$ и таких, что
\begin{equation*}\label{cond}
\lim_{x\to\infty}|f(x)|e^{-x/2}=0.
\end{equation*}
Норму в этом пространстве определим следующим образом:
\begin{equation*}\label{norma}
\|f\|_{C(\Omega_\delta)}=\sup_{x\in\Omega_\delta}e^{-x/2}|f(x)|.
\end{equation*}
Справедлива следующая лемма


\begin{lemma}
Пусть $\alpha>-1$, $p>1$, $l_{\rho_N}^p$ -- пространство функций, заданных на $\Omega_\delta$ и таких, что
$$
\|f\|_{l_{\rho_N}^p}=\left(\sum_{x\in\Omega_\delta}|f(x)|^p\rho_N(x)\right)^{1/p}<\infty.
$$
Тогда $C(\Omega_\delta)\subset l_{\rho_N}^p$ при $1<p<2$.
\end{lemma}



Из леммы следует, что для произвольной функции $f\in C(\Omega_\delta)$ мы можем определить коэффициенты Фурье-Мейкснера
$$
f_k^{\alpha}=\sum_{t\in\Omega_\delta}f(t)m_{k,N}^\alpha(t)\rho_N(t)
$$
и ряд Фурье-Мейкснера
\begin{equation}\label{fser}
f(x)\sim\sum_{k=0}^{\infty}f_k^\alpha m_{k,N}^\alpha(x).
\end{equation}
Через $S^\alpha_{n,N}(f,x)$ обозначим частичную сумму ряда \eqref{fser}:
$$
S^\alpha_{n,N}(f,x)=\sum_{k=0}^{n}f_k^\alpha m_{k,N}^\alpha(x),
$$
которую в силу \eqref{gadz-eq2} можем представить в виде
\begin{equation}\label{Fsum}
S^\alpha_{n,N}(f,x)=\sum_{t\in\Omega_{\delta}}f(t)\mathcal{ K}_{n,N}^\alpha(t,x)e^{-t}{\Gamma(Nt+\alpha+1)\over\Gamma(Nt+1)}(1-e^{-\delta})^{\alpha+1}.
\end{equation}
Будем рассматривать $S^\alpha_{n,N}(f,x)$ как аппарат приближения функций из $C(\Omega_\delta)$.
Для $f\in C(\Omega_\delta)$ через $E_n(f)$ обозначим наилучшее приближение $f$ в метрике пространства
$ C(\Omega_\delta)$ алгебраическими полиномами степени $n$, то есть,
$$
E_n(f)=\inf_{p_n\in H^n}\|f-p_n\|_{C(\Omega_\delta)},
$$
где $H^n$ -- подпространство алгебраических полиномов $p_n(x)$ степени не выше $n$.
Пусть, далее $p_n(f)=p_n(f,x)$ полином наилучшего приближения к функции $f$ в $C(\Omega_\delta)$, для которого
$
E_n(f)=\|f-p_n(f)\|_{C(\Omega_\delta)}.
$
Тогда, пользуясь тем, что для $p_n\in H^n$ будет $S_{n,N}^\alpha(p_{n})=p_n$ мы можем записать
$$
\left|f(x)-S_{n,N}^\alpha(f,x)\right|=\left|f(x)-p_{n}(f,x)+p_{n}(f,x)-S_{n,N}^\alpha(f,x)\right|\leq
$$
$$
\left|f(x)-p_{n}(f,x)\right|+\left|S_{n,N}^\alpha(p_{n}-f,x)\right|.
$$
Отсюда и из \eqref{Fsum}
$$
e^{-{x\over 2}}\left|f(x)-S_{n,N}^\alpha(f,x)\right|\leq e^{-{x\over 2}}\left|f(x)-p_{n}(f,x)\right|+
e^{-{x\over 2}}\left|S_{n,N}^\alpha(p_{n}-f,x)\right|\leq
$$
\begin{equation}\label{ineq}
E_n(f)(1+\lambda_{n,N}^\alpha(x)),
\end{equation}
где
\begin{equation}\label{funcLeb}
\lambda_{n,N}^{\alpha}(x)=\sum_{t\in\Omega_{\delta}}e^{-{t+x\over 2}}{\Gamma(Nt+\alpha+1)\over\Gamma(Nt+1)}(1-e^{-\delta})^{\alpha+1}\left|\mathcal{ K}_{n,N}^\alpha(t,x)\right|.
\end{equation}
В связи с неравенством \eqref{ineq} возникает задача об оценке на $[0,\infty)$ функции Лебега $\lambda_{n,N}^{\alpha}(x)$, определяемой равенством \eqref{funcLeb}. Здесь мы ограничимся рассмотрением этой задачи для $x\in G_1=[0,\frac{3}{\theta_n}]$ и $x\in G_2=[\frac{3}{\theta_n},{\theta_n\over 2}]$,
Имеет место следующая теорема.

\begin{theorem}
Пусть $\alpha>-1,$ $\theta_n=4n+2\alpha+2,$ $\lambda> 0,$ $1\le n\leq\lambda N.$ Тогда имеют место следующие оценки:\\
1) если $x\in G_1=[0, \frac{3}{\theta_n}],$ то
\begin{equation*}\label{gadz-eq19}
\lambda_{n,N}^{\alpha}(x)\leq c(\alpha, \lambda)
\begin{cases}
1, & -1<\alpha<-{1\over2},\\
\ln (n+1),&   \alpha=-\frac{1}{2},\\
n^{\alpha+\frac{1}{2}},&  \alpha> -\frac{1}{2}.
\end{cases};
\end{equation*}
2) если $x\in G_2=[\frac{3}{\theta_n}, {\theta_n\over 2}],$ то
\begin{equation*}\label{gadz-eq20}
\lambda_{n,N}^{\alpha}(x)\leq c(\alpha, \lambda)\begin{cases}
\ln(nx+1),&     -1<\alpha\leq-\frac{1}{2},\\
\ln(n+1)+{\left(n\over x\right)}^{{\alpha\over2}+{1\over4}},& \alpha>-\frac{1}{2}.
\end{cases}
\end{equation*}
\end{theorem}
