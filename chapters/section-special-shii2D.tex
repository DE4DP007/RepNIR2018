\section{Двумерные  специальные ряды по системе $\{\sin x\sin nx\}_{n=1}^\infty$  и их аппроксимативные свойства}\label{sect-2.3}
%\textbf{Изучены аппроксимативные свойства двумерных специальных рядов по системе $\{\sin x\sin nx\}_{n=1}^\infty$}

\textit{В настоящей статье вводятся двумерные специальные ряды по системе \linebreak $\{\sin x\sin nx\}_{n=1}^\infty$.
Показано, что эти ряды выгодно отличаются от двумерных косинус-рядов Фурье тем, что их частичные суммы вблизи границы квадрата $[0,\pi]^2$ обладают значительно лучшими аппроксимативными свойствами, чем суммы Фурье. Приводится оценка скорости сходимости частичных сумм специального ряда к функциям $f(x,y)$ из пространства четных $2\pi$-периодических по каждой переменной непрерывных функций.}


%\subsection{Введение}
%Представление функций в виде рядов по тем или иным ортонормированным системам с целью последующего их приближения
%частичными суммами выбранного ортогонального ряда является, пожалуй, одним из самых часто применяемых подходов в теории приближений и ее приложениях. Наряду с задачами математической физики, для решения которых указанный подход является традиционным, появились и продолжают появляться все новые важные задачи, для решения которых также все чаще применяются методы, основанные на представлении функций(сигналов) в виде рядов по подходящим ортонормированным системам (см., например, \cite{shii1, shii2, dedus3, pash4, arush5, tref6, tref7, muku8, malvarSign}). При этом часто возникает такая ситуация, когда функция (сигнал, временной ряд, изображение и.т.д) $f=f(t)$ задана на достаточно длинном промежутке $[0,T]$ и нам требуется разбить этот промежуток на части $[a_j,a_{j+1}]$ $(j=0,1,\ldots,m)$, рассмотреть отдельные фрагменты функции определенные на этих частичных отрезках, представить их в виде рядов по выбранной ортонормированной системе и аппроксимировать каждый такой фрагмент частичными суммами соответствующего ряда. Такая ситуация является типичной для задач, связанных с решением нелинейных дифференциальных уравнений численно-аналитическими методами \cite{pash4, tref6}, обработкой временных рядов и изображений и других \cite{arush5,tref6,tref7}, в которых
%возникает необходимость разбить заданный ряд данных на части,
%аппроксимировать каждую часть и заменить приближенно исходный
%временный ряд (изображение) функцией, полученной в результате
%<<пристыковки>> функций, аппроксимирующих отдельные части. Но тогда в
%местах <<стыка>> возникают нежелательные разрывы (артефакты) (см.\cite{muku8}), которые искажают общий вид временного ряда (изображения). %Такая
%картина непременно возникает при использовании для приближения
%<<кусков>> исходной функции сумм Фурье по классическим ортонормированным системам.
В работах \cite{shii1, shii2} введены некоторые специальные ряды по ультрасферическим полиномам Якоби, частичные суммы $\sigma_n^\alpha(f,x)$ которых на на концах отрезка $[-1,1]$ совпадает с исходной функцией $f(x)$, т.е. $\sigma_n^\alpha(f,\pm1)=f(\pm1)$.
В качестве одного из частных случаев таких рядов возникает ряд вида

\begin{equation}\label{ssap2deq.1.2.4.1.1}
 \Phi(\theta)=a_\Phi(\theta)+\sin\theta \sum_{k=1}^\infty\varphi_k\sin k \theta,
\end{equation}
где
$$
a_\Phi(\theta)={\Phi(0)+\Phi(\pi)\over2}+
{\Phi(0)-\Phi(\pi)\over2}\cos\theta,
$$


$$
\varphi(\theta)=\Phi(\theta)-a_\Phi(\theta),\quad \varphi_k={\frac2\pi}
\int\limits_{0}^\pi \varphi(\tau){\sin k\tau\over\sin\tau}d\tau.
$$
В работе \cite{shii2} исследованы, в частности, аппроксимативные свойства ряда \eqref{ssap2deq.1.2.4.1.1} в пространстве $C^e_{2\pi}$, состоящем из четных непрерывных $2\pi$ -- периодических функций .






В настоящей статье мы введем двумерные ряды вида \eqref{ssap2deq.1.2.4.1.1} и рассмотрим некоторые свойства частичных сумм этих рядов.
Пусть $f(x,y)$ -- четная $2\pi$ -- периодическая по каждой из переменных $x$ и $y$ и интегрируемая на квадрате $[0,\pi]^2$ функция, которая в точках $(i\pi,j\pi)$, $i,j\in \mathbb{Z}$ принимает конечные значения. Положим
\begin{equation}\label{ssap2deq.1.2.4.1.2}
S(f)=S(f)(u,v)=f(u,v)-{f(0,v)+f(\pi,v)\over2}-{f(0,v)-f(\pi,v)\over2}\cos u,
\end{equation}
\begin{equation}\label{ssap2deq.1.2.4.1.3}
H(f)=H(f)(u,v)=f(u,v)-{f(u,0)+f(u,\pi)\over2}-{f(u,0)-f(u,\pi)\over2}\cos v,
\end{equation}
\begin{equation}\label{ssap2deq.1.2.4.1.4}
O(f)=O(f)(u,v)=S(u,v)-{S(u,0)+S(u,\pi)\over2}-{S(u,0)-S(u,\pi)\over2}\cos v,
\end{equation}
$$
\Theta(f)=\Theta(f)(u,v)={f(0,0)-f(0,\pi)-f(\pi,0)+f(\pi,\pi)\over4}\cos u\cos v
$$
$$
+{f(0,0)+f(0,\pi)-f(\pi,0)-f(\pi,\pi)\over4}\cos u
$$
$$
+{f(0,0)-f(0,\pi)+f(\pi,0)-f(\pi,\pi)\over4}\cos v
$$
\begin{equation}\label{ssap2deq.1.2.4.1.5}
+{f(0,0)+f(0,\pi)+f(\pi,0)+f(\pi,\pi)\over4}
\end{equation}
и заметим, что
$$
f(x,y)=\Theta(f)(x,y)+
$$
\begin{equation}\label{ssap2deq.1.2.4.1.6}
O(f)(x,y)+A(f)(x)+B(f)(x)\cos y+c(f)(y)+D(f)(y)\cos x,
\end{equation}
где
$$
A(f)(u)={S(f)(u,0)+S(f)(u,\pi)\over2},\quad B(f)(u)={S(f)(u,0)-S(f)(u,\pi)\over2},
$$
$$
C(f)(u)={H(f)(0,v)+H(f)(\pi,v)\over2},\quad D(f)(u)={H(f)(0,v)-H(f)(\pi,v)\over2}.
$$
Определим следующие коэффициенты
\begin{equation}\label{ssap2deq.1.2.4.1.7}
o_{k,l}(f)=\frac4{\pi^2}\int\limits_0^\pi\int\limits_0^\pi
 O(f)(u,v){\sin ku\sin lv\over\sin u\sin v}dudv,
 \end{equation}
\begin{equation}\label{ssap2deq.1.2.4.1.8}
a_k(f)=\frac2{\pi}\int\limits_0^\pi A(f)(u){\sin ku\over\sin u}du,\quad
b_k(f)=\frac2{\pi}\int\limits_0^\pi B(f)(u){\sin ku\over\sin u}du,
\end{equation}
\begin{equation}\label{ssap2deq.1.2.4.1.9}
c_l(f)=\frac2{\pi}\int\limits_0^\pi C(f)(u){\sin lv\over\sin v}dv,\quad
d_l(f)=\frac2{\pi}\int\limits_0^\pi D(f)(u){\sin lv\over\sin v}dv
\end{equation}
и рассмотрим ряды
\begin{equation}\label{ssap2deq.1.2.4.1.10}
O(f)(x,y)\sim \sin x\sin y\sum_{k=1}^\infty\sum_{l=0}^\infty o_{k,l}(f)\sin kx\sin ly,
\end{equation}
\begin{equation}\label{ssap2deq.1.2.4.1.11}
A(f)(x)\sim \sin x\sum_{k=1}^\infty a_{k}(f)\sin kx,
\end{equation}
\begin{equation}\label{ssap2deq.1.2.4.1.12}
B(f)(x)\sim \sin x\sum_{k=1}^\infty b_{k}(f)\sin kx,
\end{equation}
\begin{equation}\label{ssap2deq.1.2.4.1.13}
C(f)(y)\sim \sin y\sum_{l=1}^\infty c_{l}(f)\sin ly,
\end{equation}
\begin{equation}\label{ssap2deq.1.2.4.1.14}
D(f)(y)\sim \sin y\sum_{l=1}^\infty d_{l}(f)\sin ly.
\end{equation}
Тогда  мы можем сопоставить  функции $f(x,y)$ следующий специальный ряд
$$
f(x,y)\sim \Theta(f)(x,y)+
$$
$$
 \sin x\sin y\sum_{k=1}^\infty\sum_{l=1}^\infty o_{k,l}(f)\sin kx\sin ly+
$$
$$
\sin x\sum_{k=1}^\infty (a_{k}(f)+b_{k}(f)\cos y)\sin kx
$$
\begin{equation}\label{ssap2deq.1.2.4.1.15}
 +\sin y\sum_{l=1}^\infty (c_{l}(f)+ d_{l}(f)\cos x)\sin ly.
\end{equation}
Ряд \eqref{ssap2deq.1.2.4.1.15} будем называть двумерным специальным рядом по системе $\{\sin x\sin nx\}_{n=1}^\infty$. В настоящей работе исследованы аппроксимативные свойства таких рядов.  Показано, что специальные ряды вида \eqref{ssap2deq.1.2.4.1.14} выгодно отличаются от двумерных косинус-рядов Фурье тем, что их частичные суммы вида
$$
S_{n,m}^{\nu,\mu}(f)=S_{n,m}^{\nu,\mu}(f)(x,y)= \Theta(f)(x,y)+
$$
$$
 \sin x\sin y\sum_{k=1}^{n-1}\sum_{l=1}^{m-1} o_{k,l}(f)\sin kx\sin ly+
$$
$$
\sin x\sum_{k=1}^{\nu-1} (a_{k}(f)+b_{k}(f)\cos y)\sin kx+
$$
\begin{equation}\label{ssap2deq.1.2.4.1.16}
 +\sin y\sum_{l=1}^{\mu-1} (c_{l}(f)+ d_{l}(f)\cos x)\sin ly
\end{equation}
вблизи границы квадрата $[0,\pi]^2$ обладают значительно лучшими  аппроксимативными свойствами, чем суммы Фурье вида
\begin{equation}\label{ssap2deq.1.2.4.1.17}
S_{n,m}(f)(x,y)=\sum_{k=0}^n\sum_{l=0}^ma_{k,l}(f)\cos kx\cos ly.
\end{equation}
 Чтобы убедиться в этом достаточно, например,  сопоставить  оценку для функции Лебега $L_{n,m}^{\nu,\mu}(x,y)$ частичных сумм  $S_{n,m}^{\nu,\mu}(f)(x,y)$ вида  \eqref{ssap2deq.1.2.4.1.16} , полученную в настоящей работе (см.\S 2), с хорошо известной оценкой функции Лебега сумм Фурье $S_{n,m}(f)(x,y)$ вида \eqref{ssap2deq.1.2.4.1.17}.


\subsection{Аппроксимативные свойства специальных рядов по системе \\ $\{\sin x\sin nx\}_{n=1}^\infty$}
Прежде всего покажем, что частичная сумма  $S_{n,m}^{\nu,\mu}(f)$ специального ряда \eqref{ssap2deq.1.2.4.1.15} вида \eqref{ssap2deq.1.2.4.1.16}  является проектором на подпространство четных  тригонометрических полиномов вида
$$
T_{n,m}^{\nu,\mu}(x,y)=
 \sum_{k=0}^{n}\sum_{l=0}^{m} p_{k,l}\cos kx\cos ly+
$$
\begin{equation}\label{ssap2deq.1.2.4.2.1}
\sum_{k=n+1}^{\nu} (p_{k,0}+p_{k,1}\cos y)\cos kx
 +\sum_{l=m+1}^{\mu} (p_{0,l}+ p_{1,l}\cos x)\cos ly,
\end{equation}
  т.е. если $f(x,y)=T_{n,m}^{\nu,\mu}(x,y)$, то
  \begin{equation}\label{ssap2deq.1.2.4.2.2}
  S_{n,m}^{\nu,\mu}(f)\equiv T_{n,m}^{\nu,\mu}(x,y).
  \end{equation}
Поскольку  $S_{n,m}^{\nu,\mu}(f)$ -- линейный оператор, то тождество \eqref{ssap2deq.1.2.4.2.2} будет доказано, если мы покажем, что оно верно для $f(x,y)=\cos kx\cos ly$ с $(k,l)\in G_{n,m}^{\nu,\mu}$, где
$$
G_{n,m}^{\nu,\mu}=\{(k,l):0\le k\le n,0\le l\le m\}\cup
$$
$$
\{(k,l):0\le k\le 1,\mu-m+1\le l\le \mu\}\cup
$$
\begin{equation}\label{ssap2deq.1.2.4.2.3}
\{(k,l):0\le \nu-n+1\le l\le \nu,0\le l\le 1\}.
\end{equation}
Пусть, например, $k,l$ -- четные, $0\le k\le n$, $0\le l\le m$. Положим $k=2i$, $l=2j$. Тогда из \eqref{ssap2deq.1.2.4.1.2}  -- \eqref{ssap2deq.1.2.4.1.5} для  $f(x,y)=\cos kx\cos ly$ имеем
\begin{equation}\label{ssap2deq.1.2.4.2.4}
S(f)(x,y)=\cos ly(\cos kx-1),\quad H(f)(x,y)=\cos kx(\cos ly-1),
\end{equation}
\begin{equation}\label{ssap2deq.1.2.4.2.5}
A(f)(x)={S(f)(x,0)+S(f)(x,\pi)\over2}=\cos kx-1,\quad B(f)(x)=0,
\end{equation}
\begin{equation}\label{ssap2deq.1.2.4.2.6}
B(f)(y)={H(f)(0,y)+H(f)(\pi,y)\over2}=\cos ly-1,\quad D(f)(y)=0,
\end{equation}
\begin{equation}\label{ssap2deq.1.2.4.2.7}
O(f)(x,y)=(\cos kx-1)(\cos ly-1), \quad \Theta(x,y)=1.
\end{equation}
Из  \eqref{ssap2deq.1.2.4.2.4} -- \eqref{ssap2deq.1.2.4.2.7}, находим
\begin{equation}\label{ssap2deq.1.2.4.2.8}
f(x,y)=\cos kx\cos ly=1+O(f)(x,y)+A(f)(x)+C(f)(y),
\end{equation}
причем
\begin{equation}\label{ssap2deq.1.2.4.2.9}
{A(f)(x)\over \sin x}=\sum_{j=1}^{k-1}\alpha_j\sin jx,\quad
{C(f)(y)\over \sin y}=\sum_{i=1}^{l-1}\beta_i\sin iy,
\end{equation}
\begin{equation}\label{ssap2deq.1.2.4.2.10}
{O(f)(x,y)\over\sin x \sin y}=\sum_{j=1}^{k-1}\sum_{i=1}^{l-1}\alpha_j\beta_i\sin jx\sin iy,
\end{equation}
где
\begin{equation}\label{ssap2deq.1.2.4.2.11}
\alpha_j=\frac2\pi\int_0^\pi A(f)(t){\sin jt\over\sin t}dt =a_j(f),
\end{equation}
\begin{equation}\label{ssap2deq.1.2.4.2.12}
\beta_i=\frac2\pi\int_0^\pi C(f)(t){\sin it\over\sin t}dt =c_i(f),
\end{equation}
\begin{equation}\label{ssap2deq.1.2.4.2.13}
\alpha_j\beta_i=\frac4{\pi^2}\int_0^\pi \int_0^\pi O(f)(u,v){\sin ju\sin iv\over\sin u\sin v}dudv=o_{ij}.
\end{equation}
Кроме того
\begin{equation}\label{ssap2deq.1.2.4.2.14}
b_j(f)=0(1\le j\le k-1),\quad d_i=0 (1\le i\le l-1).
\end{equation}
Из \eqref{ssap2deq.1.2.4.2.8} -- \eqref{ssap2deq.1.2.4.2.14}  получаем
$$
f(x,y)=\cos kx\cos ly=
$$
$$
\Theta(x,y)+O_{k,l}(f)(x,y)+A_k(f)(x)+C_l(f)(y)=S_{k,l}^{k,l}(f)(x,y).
$$
Тем самым мы доказали, что тождество \eqref{ssap2deq.1.2.4.2.2} справедливо для $T_{k,l}^{\nu,\mu}(x,y)=\cos kx\cos ly$ в том случае, когда $\nu=k$, $\mu=l$,
$k,l$ -- четные. Его справедливость в остальных случаях $(k,l)\in G_{n,m}^{\nu,\mu}$ проверяется аналогично.

Заметим, что если функция $f(x,y)$ обращается в нуль в точках множества \linebreak $P=\{(0,0), (0,\pi),(\pi,0),(\pi,\pi)\}$, то $\Theta(x,y)\equiv0$ и, как следствие, равенство \eqref{ssap2deq.1.2.4.1.16} принимает следующий вид

$$
S_{n,m}^{\nu,\mu}(f)(x,y)=  \sin x\sin y\sum_{k=1}^{n-1}\sum_{l=1}^{m-1} o_{k,l}(f)\sin kx\sin ly+
$$
$$
\sin x\sum_{k=1}^{\nu-1} (a_{k}(f)+b_{k}(f)\cos y)\sin kx+
$$
\begin{equation}\label{ssap2deq.1.2.4.2.15}
 +\sin y\sum_{l=1}^{\mu-1} (c_{l}(f)+ d_{l}(f)\cos x)\sin ly.
\end{equation}
Полагая
\begin{equation}\label{ssap2deq.1.2.4.2.16}
R_{n}(t,s)=\sum_{k=1}^{n-1}\sin kt\sin ks,
\end{equation}
мы получаем из   \eqref{ssap2deq.1.2.4.2.15}  следующее интегральное представление
$$
S_{n,m}^{\nu,\mu}(f)(x,y)=  \frac{4\sin x\sin y}{\pi^2}\int\limits_0^\pi\int\limits_0^\pi
 O(f)(u,v){R_{n}(x,u)R_{m}(y,v)\over\sin u\sin v}dudv+
$$
$$
\frac{2\sin x}\pi\int_0^\pi[A(f)(u)+B(f)(u)\cos y]{R_{\nu}(x,u)\over\sin u}du+
$$
\begin{equation}\label{ssap2deq.1.2.4.2.17}
 \frac{2\sin y}\pi\int_0^\pi[C(f)(v)+D(f)(v)\cos x]{R_{\mu}(y,v)\over\sin v}dv.
\end{equation}
Перейдем к вопросу об аппроксимативных свойствах частичных сумм $S_{n,m}^{\nu,\mu}(f)(x,y)$. Прежде всего заметим, что если $T_{n,m}^{\nu,\mu}(x,y)$ -- тригонометрический полином вида \eqref{ssap2deq.1.2.4.2.1}, то в силу \eqref{ssap2deq.1.2.4.2.2} имеет место равенство
$$
f(x,y)-S_{n,m}^{\nu,\mu}(f)(x,y)=f(x,y)-T_{n,m}^{\nu,\mu}(x,y)+T_{n,m}^{\nu,\mu}(x,y)-S_{n,m}^{\nu,\mu}(f)(x,y)=
$$
\begin{equation}\label{ssap2deq.1.2.4.2.18}
f(x,y)-T_{n,m}^{\nu,\mu}(x,y)+S_{n,m}^{\nu,\mu}(T_{n,m}^{\nu,\mu}-f)(x,y).
\end{equation}
Если, кроме того, полином  $T_{n,m}^{\nu,\mu}(x,y)$  вида \eqref{ssap2deq.1.2.4.2.1} совпадает с $f(x,y)$ при $(x,y)\in P$, то из \eqref{ssap2deq.1.2.4.2.17} следует, что
$$
S_{n,m}^{\nu,\mu}(T_{n,m}^{\nu,\mu}-f)(x,y)=
$$
$$
 \frac{4\sin x\sin y}{\pi^2}\int\limits_0^\pi\int\limits_0^\pi
 O(T_{n,m}^{\nu,\mu}-f)(u,v){R_{n}(x,u)R_{m}(y,v)\over\sin u\sin v}dudv+
$$
$$
\frac{2\sin x}\pi\int_0^\pi[A(T_{n,m}^{\nu,\mu}-f)(u)+B(T_{n,m}^{\nu,\mu}-f)(u)\cos y]{R_{\nu}(x,u)\over\sin u}du+
$$
\begin{equation}\label{ssap2deq.1.2.4.2.19}
 \frac{2\sin y}\pi\int_0^\pi[C(T_{n,m}^{\nu,\mu}-f)(v)+D(T_{n,m}^{\nu,\mu}-f)(v)\cos x]{R_{\mu}(y,v)\over\sin v}dv.
\end{equation}

Обозначим через $C^{2e}_{2\pi}$
пространство четных $2\pi$-периодических по каждой из переменных $x$ и  $y$ и непрерывных функций $f=f(x,y)$, для которых определена норма
$\|f\|=\max_{(x,y)\in \mathbb{R}^2} $, $\mathcal{T}_{n,m}^{\nu,\mu}(f)$ -- пространство тригонометрических полиномов вида \eqref{ssap2deq.1.2.4.2.1}, совпадающих с $f(x,y)$ при $(x,y)\in P$. Тогда  в силу (1.21) -- (1.23) имеем
\begin{equation}\label{ssap2deq.1.2.4.2.20}
|S(T_{n,m}^{\nu,\mu}-f)(u,v)|\le 3\|T_{n,m}^{\nu,\mu}-f\|, \quad |H(T_{n,m}^{\nu,\mu}-f)(u,v)|\le 3\|T_{n,m}^{\nu,\mu}-f\|
\end{equation}
и, как следствие,
\begin{equation}\label{ssap2deq.1.2.4.2.21}
|O(T_{n,m}^{\nu,\mu}-f)(u,v)|\le 9\|T_{n,m}^{\nu,\mu}-f\|.
\end{equation}
Аналогично
\begin{equation}\label{ssap2deq.1.2.4.2.22}
|A(T_{n,m}^{\nu,\mu}-f)|\le 3\|T_{n,m}^{\nu,\mu}-f\|,\quad |B(T_{n,m}^{\nu,\mu}-f)|\le 3\|T_{n,m}^{\nu,\mu}-f\|,
\end{equation}
\begin{equation}\label{ssap2deq.1.2.4.2.23}
|C(T_{n,m}^{\nu,\mu}-f)|\le 3\|T_{n,m}^{\nu,\mu}-f\|,\quad |D(T_{n,m}^{\nu,\mu}-f)|\le 3\|T_{n,m}^{\nu,\mu}-f\|.
\end{equation}
 Пусть
$$
 \tilde E_{n,m}^{\nu,\mu}(f)=\inf_{T_{n,m}^{\nu,\mu}\in \mathcal{T}_{n,m}^{\nu,\mu}(f)}\|f-T_{n,m}^{\nu,\mu}\|
$$
-- наилучшее приближение функции $f\in C^{2e}_{2\pi}$  тригонометрическими полиномами $T_{n,m}^{\nu,\mu}=T_{n,m}^{\nu,\mu}(x,y)\in\mathcal{T}_{n,m}^{\nu,\mu}(f)$. Из \eqref{ssap2deq.1.2.4.2.18} и \eqref{ssap2deq.1.2.4.2.19} с учетом \eqref{ssap2deq.1.2.4.2.20} -- \eqref{ssap2deq.1.2.4.2.23} получаем
\begin{equation}\label{ssap2deq.1.2.4.2.24}
|f(x,y)-S_{n,m}^{\nu,\mu}(f)(x,y)|\le \tilde E_{n,m}^{\nu,\mu}(f)[1+9\Lambda_n(x)\Lambda_m(y)+6\Lambda_\nu(x)+6\Lambda_\mu(y)],
\end{equation}
где
\begin{equation}\label{ssap2deq.1.2.4.2.25}
\Lambda_s(t)=|\sin t|\frac2\pi\int_0^\pi{|R_s(t,u)|\over\sin u}du=|\sin t|\frac2\pi\int_0^\pi\left|\sum_{k=1}^{s-1}{\sin kt\sin ku\over \sin u}\right|du.
\end{equation}
Далее, пусть $ E_{n,m}^{\nu,\mu}(f)$ -- наилучшее приближение функции $f\in C^{2e}_{2\pi}$  тригонометрическими полиномами $T_{n,m}^{\nu,\mu}$ вида \eqref{ssap2deq.1.2.4.2.1}, не обязательно совпадающими с $f(x,y)$ в точках множества $P$. Тогда нетрудно заметить, что
\begin{equation}\label{ssap2deq.1.2.4.2.26}
 E_{n,m}^{\nu,\mu}(f)\le \tilde E_{n,m}^{\nu,\mu}(f)\le 2E_{n,m}^{\nu,\mu}.
\end{equation}
Сопоставляя \eqref{ssap2deq.1.2.4.2.26} с \eqref{ssap2deq.1.2.4.2.24}, мы можем записать
\begin{equation}\label{ssap2deq.1.2.4.2.27}
|f(x,y)-S_{n,m}^{\nu,\mu}(f)(x,y)|\le 2 E_{n,m}^{\nu,\mu}(f)[1+9\Lambda_n(x)\Lambda_m(y)+6\Lambda_\nu(x)+6\Lambda_\mu(y)].
\end{equation}
В связи с оценками \eqref{ssap2deq.1.2.4.2.24} и \eqref{ssap2deq.1.2.4.2.27} возникает задача об исследовании поведения  величины $\Lambda_n(t)$ при $n\to\infty$ и $-1\le t\le 1$.
Эта задача была решена в работе автора \cite{shii2}, в которой получена следующая оценка
\begin{equation}\label{ssap2deq.1.2.4.2.28}
\Lambda_n(t)\le c(1+\ln(1+n|\sin t|)).
\end{equation}
С учетом \eqref{ssap2deq.1.2.4.2.28} из \eqref{ssap2deq.1.2.4.2.27} мы выводим следующий результат

%\textbf{ Теорема 2.1. } \textit{

\begin{theorem}\label{ssap2d:t1}
Имеет место оценка
$$
|f(x,y)-S_{n,m}^{\nu,\mu}(f)(x,y)|\le
$$
$$
 cE_{n,m}^{\nu,\mu}(f)[1+\ln(1+n|\sin x|)\ln(1+m|\sin y|)+\ln(1+\nu|\sin x|)+\ln(1+\mu|\sin y|)].
$$
\end{theorem}
