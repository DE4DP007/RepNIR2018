\chapter{Ортогональные на сетке по Соболеву функции, порожденные дискретными ортогональными функциями}

\section{Системы дискретных функций, ортонормированных по Соболеву, порожденные  ортонормированными  функциями}

Нам понадобятся некоторые понятия и результаты из работы \cite{ramis-shGadj}, в которой рассмотрены системы дискретных функций, ортонормированных по Соболеву относительно скалярного произведения \eqref{1.1}, порожденных заданной системой $\{\psi_k(x)\}_{k=0}^\infty$, ортонормированной на дискретном множестве $\Omega=\{0,1,\ldots\}$ с весом $\rho(x)$.  Если целое $k\ge0$, то положим $a^{[k]}=a(a-1)\cdots (a-k+1)$, $a^{[0]}=1$ и рассмотрим следующие функции
\begin{equation}\label{ramis-2.1}
\psi_{r,k}(x)={x^{[k]}\over k!},\, k=0,1,\ldots,r-1,
\end{equation}
\begin{equation}\label{ramis-2.2}
\psi_{r,k}(x)=\begin{cases}\frac{1}{(r-1)!}\sum\limits_{t=0}^{x-r}(x-1-t)^{[r-1]}\psi_{k-r}(t),
&\text{$r\le x$,}\\ 0,&\text{$x=0,1,\ldots, r-1$,}
\end{cases}
\end{equation}
которые определены на сетке $\Omega=\{0,1,\ldots\}$. Рассмотрим некоторые важные разностные  свойства системы функций $\psi_{r,k}(x)$, определенных равенствами \eqref{ramis-2.1} и \eqref{ramis-2.2}.

\begin{lemma}\label{ramis-RII}
Имеют место равенства
\begin{equation}\label{ramis-2.3}
\Delta^\nu \psi_{r,k}(x)=\begin{cases}\psi_{r-\nu,k-\nu}(x),&\text{если $0\le\nu\le r-1$, $r\le k$,}\\
\psi_{k-r}(x),&\text{если  $\nu=r\le k$,}\\
\psi_{r-\nu,k-\nu}(x),&\text{если $\nu\le k< r$,}\\
0,&\text{если $k< \nu\le r$}.
  \end{cases}
\end{equation}
\end{lemma}
Систему функций $\psi_{r,k}(t)\, (k=0,1,\ldots)$ мы будем, следуя \cite{ramis-shGadj} \textit {называть системой, ортонормированной по Соболеву относительно скалярного произведения \eqref{1.1}.}

Пусть $r$ -- натуральное число. Обозначим через $l_\rho$ пространство дискретных функций $f, g, \ldots$, в котором скалярное произведение определяется обычным образом с помощью равенства $\langle f,g\rangle = \sum\limits_{x\in\Omega}f(x)g(x)\rho(x)$.
Через $W^r_{l_\rho}$ обозначим подпространство в $l_\rho$, состоящее из функций $f, g, \ldots$, для которых определено скалярное произведение \eqref{1.1}.
%Обозначим через $W^r_{l_\rho}$ пространство дискретных функций $f=f(x)$, заданных на сетке $\Omega=\{0,1,\ldots\}$, для которых $f(x+\nu)\in l_\rho$, $\nu=0, 1, \ldots, r$. Тогда в пространстве $W^r_{l_\rho}$ определим скалярное произведение $\langle f,g\rangle$ с помощью равенства \eqref{1.1}.
Рассмотрим  задачу об ортогональности, нормированности и полноте в $W^r_{l_\rho}$ системы $\{\psi_{r,k}(x)\}_{k=0}^\infty$, состоящей из функций, определенных равенствами   \eqref{ramis-2.1} и \eqref{ramis-2.2}.

\begin{theorem}\label{ramis-RItheo}
Предположим, что функции $\psi_k(x)$ $(k=0,1,\ldots)$ образуют полную в $l_\rho$ ортонормированную систему  c весом   $\rho(x)$. Тогда система $\{\psi_{r,k}(x)\}_{k=0}^\infty$, порожденная системой $\{\psi_{k}(x)\}_{k=0}^\infty$
 посредством равенств \eqref{ramis-2.1} и \eqref{ramis-2.2}, полна  в $W^r_{l_\rho}$ и ортонормирована относительно скалярного произведения \eqref{1.1}.
\end{theorem}

Из теоремы \ref{ramis-RItheo} следует, что система дискретных функций $\psi_{r,k}(t)\, (k=0,1,\ldots)$
является ортонормированным базисом (ОНБ) в пространстве $W^r_{l_\rho}$, поэтому для произвольной функции $f(x)\in W^r_{l_\rho}$ мы можем записать равенство
  \begin{equation}\label{ramis-2.8}
 f(x)= \sum_{k=0}^\infty\langle f,\psi_{r,k}\rangle \psi_{r,k}(x),
  \end{equation}
которое представляет собой  ряд Фурье функции $f(x)\in W^r_{l_\rho}$ по системе
$\{\psi_{r,k}(t)\}_{k=0}^\infty$, ортонормированной по Соболеву. Поскольку коэффициенты Фурье $\langle f,\psi_{r,k}\rangle$ имеют  вид
\begin{equation}\label{ramis-2.9}
f_{r,k}=\langle f,\psi_{r,k}\rangle =\sum_{\nu=0}^{r-1}\Delta^\nu f(0)\Delta^\nu\psi_{r,k}(0)=\Delta^kf(0),\, k=0,\ldots, r-1,
 \end{equation}
 $$
f_{r,k}= \langle f,\psi_{r,k}\rangle =\sum_{j=0}^\infty\Delta^rf(j)\Delta^r\psi_{r,k}(j)\rho(j)=
 $$
\begin{equation}\label{ramis-2.10}
\sum_{j=0}^{\infty}\Delta^rf(j)\psi_{k-r}(j)\rho(j),\, k=r,\ldots,
 \end{equation}
то равенство \eqref{ramis-2.8} можно переписать в следующем  смешанном виде
  \begin{equation}\label{ramis-2.11}
 f(x)= \sum_{k=0}^{r-1}\Delta^kf(0){x^{[k]}\over k!} +\sum_{k=r}^\infty f_{r,k} \psi_{r,k}(x).
  \end{equation}
Поэтому ряд фурье по системе $\{\psi_{r,k}(t)\}_{k=0}^\infty$ мы будем следуя
\cite{ramis-Shar9, sobleg-Shar17, equ130-Shar13, laplas-Shar14, sobleg-Shar15, ramis-SharT1, equ102-Shar19, sobleg-Shar18, ramis-SharII} называть смешанным рядом по исходной ортонормированной системе $\{\psi_{k}(t)\}_{k=0}^\infty$.

\begin{theorem}\label{ramis-RItheo2}
Предположим, что    функции $\psi_k(x)$ $(k=0,1,\ldots)$ образуют полную в $l_\rho$ ортонормированную систему  c весом   $\rho(x)$. Тогда если $f(x)\in W^r_{l_\rho}$, то смешанный ряд \eqref{ramis-2.11} сходится к $f(x)$ в каждой точке $x\in \{0,1,\ldots\}$.
\end{theorem}
Отметим некоторые важные свойства смешанных рядов  \eqref{ramis-2.11} и их частичных сумм вида
\begin{equation}\label{ramis-2.13}
 \mathcal{ Y}_{r,n}(f,x)= \sum_{k=0}^{r-1}\Delta^kf(0){x^{[k]}\over k!} +\sum_{k=r}^{n}f_{r,k} \psi_{r,k}(x).
  \end{equation}
Из \eqref{ramis-2.11} и \eqref{ramis-2.13} с учетом равенств \eqref{ramis-2.3} мы можем записать ($0\le\nu\le r-1$, $x\in \Omega$ )
 \begin{equation}\label{ramis-2.130}
 \Delta^\nu f(x)= \sum_{k=0}^{r-\nu-1}\Delta^{k+\nu}f(0){x^{[k]}\over k!} +\sum_{k=r-\nu}^\infty f_{r,k+\nu} \psi_{r-\nu,k}(x),
  \end{equation}
  \begin{equation}\label{ramis-2.14}
 \Delta^\nu\mathcal{ Y}_{r,n}(f,x)= \sum_{k=0}^{r-\nu-1}\Delta^{k+\nu}f(0){x^{[k]}\over k!} +\sum_{k=r-\nu}^{n-\nu} f_{r,k+\nu} \psi_{r-\nu,k}(x),
  \end{equation}
 \begin{equation}\label{ramis-2.15}
 \Delta^\nu\mathcal{ Y}_{r,n}(f,x) = \mathcal{ Y}_{r-\nu,n-\nu}(\Delta^\nu f,x).
  \end{equation}


%\chapter{Полиномы, ортогональные по Соболеву, порожденные полиномами Шарлье}

%
%
%\section{Системы дискретных функций, ортонормированных по Соболеву, порожденные ортонормированными функциями}
%
%Как уже отмечалось выше, системы дискретных функций, ортонормированных по Соболеву относительно скалярного произведения \eqref{Shar_eq2}, порожденных заданной системой $\{\psi_k(x)\}_{k=0}^\infty$, ортонормированной на дискретном множестве $\Omega=\{0,1,\ldots\}$ с весом $\rho(x)$, были рассмотрены в \cite{Shar7}. В дальнейшем нам понадобятся некоторые результаты из \cite{Shar7}, поэтому мы вкратце напомним их в этом параграфе. С этой целью, следуя \cite{Shar7}, введем некоторые обозначения и понятия.
%
%
%Если целое $k\ge0$, то положим $a^{[k]}=a(a-1)\cdots (a-k+1)$, $a^{[0]}=1$ и рассмотрим следующие функции
%\begin{equation}\label{Shar_eq3}
%\psi_{r,k}(x)= \frac{x^{[k]}}{k!},\ k=0,1,\ldots,r-1,
%\end{equation}
%\begin{equation}\label{Shar_eq4}
%\psi_{r,k}(x) = \begin{cases} \frac{1}{(r-1)!} \sum\limits_{t=0}^{x-r} (x-1-t)^{[r-1]} \psi_{k-r}(t), &\text{$r\le x$,} \\
%0, &\text{$x=0,1,\ldots, r-1$}, \end{cases}
%\end{equation}
%которые определены на сетке $\Omega=\{0,1,\ldots\}$. Рассмотрим некоторые важные разностные свойства системы функций $\psi_{r,k}(x)$, определенных равенствами \eqref{Shar_eq3} и \eqref{Shar_eq4}. Введем оператор конечной разности $\Delta f$: $\Delta f(x)=f(x+1)-f(x)$ и положим $\Delta^{\nu+1} f(x)=\Delta\Delta^\nu f(x)$. Имеет место следующий факт \cite{Shar7}:
%
%\begin{equation}\label{Shar_eq5}
%\Delta^\nu \psi_{r,k}(x) = \begin{cases} \psi_{r-\nu,k-\nu}(x), &\text{если $0\le\nu\le r-1$, $r\le k$,} \\
%\psi_{k-r}(x), &\text{если $\nu=r\le k$,} \\
%\psi_{r-\nu,k-\nu}(x), &\text{если $\nu\le k< r$,} \\
%0, &\text{если $k< \nu\le r$}. \end{cases}
%\end{equation}
%
%
%Пусть $\rho: \Omega \to \mathbb{R}$ --- положительная функция, для которой $\sum_{x=0}^\infty \rho(x) < \infty$. Обозначим через $l_\rho$ пространство дискретных функций $f,g,\ldots$, в котором скалярное произведение определяется обычным образом с помощью равенства $(f,g) = \sum\limits_{x\in\Omega}	 f(x)g(x)\rho(x)$.
%Через $W^r_{l_\rho}$ обозначим подпространство в $l_\rho$, состоящее из функций $f,g,\ldots$, для которых определено скалярное произведение \eqref{Shar_eq2}.
%Рассмотрим задачу об ортонормированности и полноте в $W^r_{l_\rho}$ системы $\{\psi_{r,k}(x)\}_{k=0}^\infty$, состоящей из функций, определенных равенствами \eqref{Shar_eq3} и \eqref{Shar_eq4}. В работе \cite{Shar7} эта задача была решена для случая, когда $\l_\rho = W^r_{l_\rho}$. В настоящей работе мы обобщаем этот результат на тот случай, когда подпространство $W^r_{l_\rho} \subset l_\rho$ не обязательно совпадает со всем пространством $l_\rho$. А именно, справедлива следующая
%\begin{theorem}\label{Shar_thm1}
%	Предположим, что функции $\psi_k(x)$ $(k=0,1,\ldots)$ образуют полную в $l_\rho$ ортонормированную систему c весом $\rho(x)$. Тогда система $\{\psi_{r,k}(x)\}_{k=0}^\infty$, порожденная системой $\{\psi_{k}(x)\}_{k=0}^\infty$
%	посредством равенств \eqref{Shar_eq3} и \eqref{Shar_eq4}, полна в $W^r_{l_\rho}$ и ортонормирована относительно скалярного произведения \eqref{Shar_eq2}.
%\end{theorem}
%
%Систему функций $\{\psi_{r,k}(t)\}_{k=0}^\infty$ мы будем называть системой, ортонормированной по Соболеву относительно скалярного произведения \eqref{Shar_eq2}
%
%Из теоремы \ref{Shar_thm1} следует, что система дискретных функций $\{\psi_{r,k}(t)\}_{k=0}^{\infty}$ является ортонормированным базисом (ОНБ) в пространстве $W^r_{l_\rho}$, поэтому для произвольной функции $f(x)\in W^r_{l_\rho}$ мы можем записать равенство
%\begin{equation}\label{Shar_eq6}
%f(x) = \sum_{k=0}^\infty \langle f,\psi_{r,k} \rangle \psi_{r,k}(x),
%\end{equation}
%которое представляет собой ряд Фурье функции $f(x)\in W^r_{l_\rho}$ по системе
%$\{\psi_{r,k}(t)\}_{k=0}^\infty$, ортонормированной по Соболеву. Заметим, что из полноты системы функций $\{\psi_{r,k}(t)\}_{k=0}^{\infty}$ в пространстве $W^r_{l_\rho}$ (теорема \ref{Shar_thm1}) следует, что ряд \eqref{Shar_eq6} сходится по норме пространства $W^r_{l_\rho}$. Нетрудно также показать, что ряд \eqref{Shar_eq6} сходится в каждой точке $x\in\{0,1,\ldots\}$.
%
%Поскольку коэффициенты Фурье $\langle f,\psi_{r,k} \rangle$ имеют вид
%$$ f_{r,k}=\langle f,\psi_{r,k} \rangle =\sum_{\nu=0}^{r-1}\Delta^\nu f(0)\Delta^\nu\psi_{r,k}(0)=\Delta^kf(0),\, k=0,\ldots, r-1, $$
%$$ f_{r,k}= \langle f,\psi_{r,k} \rangle =\sum_{j=0}^\infty\Delta^rf(j)\Delta^r\psi_{r,k}(j)\rho(j)= $$
%$$ =\sum_{j=0}^{\infty}\Delta^rf(j)\psi_{k-r}(j)\rho(j),\, k=r,r+1,\ldots, $$
%то равенство \eqref{Shar_eq6} можно переписать в следующем смешанном виде
%\begin{equation}\label{Shar_eq7}
%f(x)= \sum_{k=0}^{r-1}\Delta^kf(0)\frac{x^{[k]}}{k!} +\sum_{k=r}^\infty f_{r,k} \psi_{r,k}(x),\, x\in \Omega.
%\end{equation}
%Поэтому ряд Фурье по системе $\{\psi_{r,k}(t)\}_{k=0}^\infty$ мы будем, следуя \cite{Shar8}, называть смешанным рядом по исходной ортонормированной $\{\psi_{k}(t)\}_{k=0}^\infty$. Отметим некоторые важные свойства смешанных рядов \eqref{Shar_eq7} и их частичных сумм вида
%\begin{equation}\label{Shar_eq8}
%\mathcal{ Y}_{r,n}(f,x)= \sum_{k=0}^{r-1}\Delta^kf(0)\frac{x^{[k]}}{k!} +\sum_{k=r}^{n}f_{r,k} \psi_{r,k}(x).
%\end{equation}
%Из \eqref{Shar_eq7} и \eqref{Shar_eq8} с учетом равенств \eqref{Shar_eq5} мы можем записать ($0\le\nu\le r-1$, $x\in \Omega$ )
%\begin{equation}\label{Shar_eq9}
%\Delta^\nu f(x)= \sum_{k=0}^{r-\nu-1}\Delta^kf(0)\frac{x^{[k]}}{k!} +\sum_{k=r-\nu}^\infty f_{r,k+\nu} \psi_{r-\nu,k}(x),
%\end{equation}
%\begin{equation}\label{Shar_eq10}
%\Delta^\nu\mathcal{ Y}_{r,n}(f,x)= \sum_{k=0}^{r-\nu-1}\Delta^kf(0)\frac{x^{[k]}}{k!} +\sum_{k=r-\nu}^{n-\nu} f_{r,k+\nu} \psi_{r-\nu,k}(x),
%\end{equation}
%\begin{equation}\label{Shar_eq11}
%\Delta^\nu\mathcal{ Y}_{r,n}(f,x) = \mathcal{ Y}_{r-\nu,n-\nu}(\Delta^\nu f,x).
%\end{equation}



\section{Полиномы, порожденные полиномами Шарлье}

При $\alpha>0$ рассмотрим на $\Omega$ полиномы $s_n^\alpha(x)\ (n=0,1,\ldots)$. Эта система порождает на $\Omega$ систему полиномов $s_{r,k}^{\alpha}(x)$ $(k=0, 1,\ldots)$, определенных равенствами
\begin{equation}\label{charlier-Shar_eq19}
s_{r,k+r}^{\alpha}(x)=\frac{1}{(r-1)!} \sum_{t=0}^{x-r}(x-1-t)^{[r-1]}s_{k}^{\alpha}(t), \quad k=0,1,\ldots
\end{equation}
\begin{equation}\label{charlier-Shar_eq20}
s_{r,k}^{\alpha}(x)=\frac{x^{[k]}}{k!},\, k=0,1,\ldots,r-1.
\end{equation}

Равенство \eqref{charlier-Shar_eq19} определяет для целых $x\geq r$ полином степени $k+r$, который мы можем продолжить на всю комплексную плоскость по принципу аналитического продолжения. Покажем, что продолженный полином, который согласно \eqref{charlier-Shar_eq19} удовлетворяет первому из равенств определения \eqref{ramis-2.2}, удовлетворяет также и второму из равенств \eqref{ramis-2.2}. Другими словами, покажем, что $s_{r,k+r}^{\alpha}(x)$ обращается в нуль в точках $x=0,1,\ldots,r-1$. С этой целью мы рассмотрим следующий дискретный аналог формулы Тейлора ($x\in\{r,r+1,\ldots\}$)
\begin{equation}\label{charlier-Shar_eq21}
F(x)=Q_{r-1}(F,x) + \frac{1}{(r-1)!}\sum_{t=0}^{x-r} (x-1-t)^{[r-1]}\Delta^rF(t),
\end{equation}
где
\begin{equation}\label{charlier-Shar_eq22}
Q_{r-1}(F,x)= F(0)+\frac{\Delta F(0)}{1!}x+\frac{\Delta^2 F(0)}{2!}
x^{[2]}+\ldots+\frac{\Delta^{r-1} F(0)}{(r-1)!}x^{[r-1]}.
\end{equation}
Так как для функции $F(t)=t^{[l+r]}$, где целое $l\ge0$, имеем $\Delta^r F(t)=(l+r)^{[r]}t^{[l]}$ и $Q_{r-1}(F,t)\equiv0$ , то из \eqref{charlier-Shar_eq21} следует, что
$$ \frac{1}{(r-1)!}\sum_{t=0}^{x-r} (x-1-t)^{[r-1]}t^{[l]}= $$
\begin{equation}\label{charlier-Shar_eq23}
= \frac{1}{(l+r)^{[r]}(r-1)!}\sum_{t=0}^{x-r} (x-1-t)^{[r-1]}\Delta^rF(t)=\frac{x^{[l+r]}}{(l+r)^{[r]}}.
\end{equation}
С другой стороны, для любого целого $l\geq 0$ функция $x^{[l+r]}$ обращается в нуль в узлах $x\in\{0,1,\ldots,r-1\}$. Поэтому полином $s_{r,k+r}^{\alpha}(x)$ также обращается в нуль при $x=0,1,\ldots,r-1$, так как в силу \eqref{charlier-Shar_eq19}, \eqref{charlier-Shar_eq16} и \eqref{charlier-Shar_eq17} его можно представить в виде линейной комбинации функций вида $x^{[l+r]}$. Таким образом, для полинома $s_{r,k}^{\alpha}(x)$, заданного при $k\geq r$ равенством \eqref{charlier-Shar_eq19}, имеет место равенство \eqref{ramis-2.2}, в котором вместо $\psi_{r,k}$ фигурирует $s_{r,k}^\alpha$. Поэтому из теоремы \ref{ramis-RItheo} и равенств \eqref{charlier-Shar_eq19}, \eqref{charlier-Shar_eq20} вытекает следующее соотношение ортогональности
$$ \langle s^\alpha_{r,n},s^\alpha_{r,m} \rangle =\sum_{k=0}^{r-1}\Delta^ks^\alpha_{r,n}(0)\Delta^ks^\alpha_{r,m}(0) + \sum_{j=0}^\infty\Delta^rs^\alpha_{r,n}(j)\Delta^rs^\alpha_{r,m}(j)\rho(j) = \delta_{nm}. $$
Тем самым, мы можем сформулировать следующий результат.
\begin{theorem}\label{charlier-Shar_thm2}
	Если $\alpha>0$, то система полиномов $s_{r,k}^{\alpha}(x)$ $(k=0, 1,\ldots)$, порожденная полиномами Шарлье $s_n^{\alpha}(x)$ $(n=0,1,\ldots)$ посредством равенств \eqref{charlier-Shar_eq19} и \eqref{charlier-Shar_eq20}, полна в $W^r_{l_\rho}$ и ортонормирована относительно скалярного произведения \eqref{Shar_eq2}.
\end{theorem}


\subsection{Дальнейшие свойства полиномов $s_{r,k}^{\alpha}(x)$ }


Перейдем к исследованию дальнейших свойств полиномов $s_{r,k}^{\alpha}(x)$. В первую очередь мы установим явный вид этих полиномов, представляющий собой разложение $s_{r,k}^{\alpha}(x)$ по обобщенным степеням $x^{[l]}$ $(l=r,r+1,\ldots,k)$.
\begin{theorem}\label{charlier-Shar_thm3}
	Для $\alpha>0$ имеют место равенства
	$$ 	s_{r,k+r}^{\alpha}(x) = \frac{1}{(h_n(\alpha))^{1/2}} \sum_{l=0}^{k} \frac{k^{[l]}x^{[l+r]}}{l!(l+r)^{[r]}} (-\alpha)^{-l}, \quad k=0,1,\ldots	$$
\end{theorem}

Теперь установим связь полиномов $s_{r,k}^{\alpha}(x)$ с порождающими их полиномами Шарлье $S_{k}^{\alpha}(x)$, которая не содержит знаков суммирования с переменным верхним пределом типа \eqref{charlier-Shar_eq19}. Имеет место следующая
\begin{theorem}\label{charlier-Shar_thm4}
	При $k\geq 0$ имеют место равенства
	\begin{equation}\label{charlier-Shar_eq24}
	s_{r,k+r}^{\alpha}(x)=\frac{(-\alpha)^r}{(k+r)^{[r]}} \left( \frac{\alpha^k}{k!} \right)^{\frac 12} \left[S_{k+r}^{\alpha}(x)-\sum_{\nu=0}^{r-1}\frac{(k+r)^{[\nu]}x^{[\nu]}}{(-\alpha)^\nu \nu!}\right] ,
	\end{equation}
	\begin{equation}\label{charlier-Shar_eq25}
	s_{r,k+r}^{\alpha}(x)=(-1)^r \left(\frac{\alpha^r}{(k+r)^{[r]}}\right)^{\frac 12}
	\left[s_{k+r}^{\alpha}(x) - \left( \frac{\alpha^{k+r}}{(k+r)!} \right)^{\frac12}\sum_{\nu=0}^{r-1}\frac{(k+r)^{[\nu]}x^{[\nu]}}{(-\alpha)^\nu \nu!}\right] .
	\end{equation}
\end{theorem}
%
%\section{Разностные свойства частичных сумм Фурье по системе $\{s_{r,k}^{\alpha}(x)\}_{k=0}^\infty $}
%
%Основные разностные свойства сумм Фурье по полиномам $s_{r,k}^{\alpha}(x)$, которые согласно \eqref{Shar_eq8} имеют вид
%$$ \mathcal{ Y}_{r,n}^{\alpha}(f,x)= \sum_{k=0}^{r-1}\Delta^kf(0)\frac{x^{[k]}}{k!} +\sum_{k=r}^{n}f_{r,k}s_{r,k}^{\alpha}(x),$$
%где
%$$ f_{r,k}= \langle f,s_{r,k}^{\alpha} \rangle =\sum_{j=0}^{\infty}\Delta^rf(j)s_{k-r}^{\alpha}(j)\rho(j),\, k=r,r+1,\ldots ,$$
%выражены равенствами \eqref{Shar_eq9}, \eqref{Shar_eq10} и \eqref{Shar_eq11}. Для системы
%$\{s_{r,k}^{\alpha}(x)\}_{k=0}^\infty$ они принимают вид $(0\le\nu\le r-1)$
%$$ \Delta^\nu f(x)= \sum_{k=0}^{r-\nu-1}\Delta^kf(0)\frac{x^{[k]}}{k!} +\sum_{k=r-\nu}^\infty f_{r,k+\nu} s_{r-\nu,k}^{\alpha}(x), $$
%$$ \Delta^\nu\mathcal{ Y}_{r,n}^{\alpha}(f,x)= \sum_{k=0}^{r-\nu-1}\Delta^kf(0)\frac{x^{[k]}}{k!} +\sum_{k=r-\nu}^{n-\nu} f_{r,k+\nu}s_{r-\nu,k}^{\alpha}(x), $$
%$$ \Delta^\nu\mathcal{ Y}_{r,n}^{\alpha}(f,x) = \mathcal{ Y}_{r-\nu,n-\nu}^{\alpha}(\Delta^\nu f,x). $$
%Из \eqref{Shar_eq9} и \eqref{Shar_eq10} мы также можем записать для $n\ge r>\nu\ge0$
%\begin{equation}\label{Shar_eq32}
%\Delta^\nu f(x)-\Delta^\nu\mathcal{ Y}_{r,n}^{\alpha}(f,x)= \sum_{k=n-\nu+1}^\infty f_{r,k+\nu} s_{r-\nu,k}^{\alpha}(x).
%\end{equation}
%Равенство \eqref{Shar_eq32} дает выражение для погрешности, возникающей в результате замены конечной разности $\Delta^\nu f(x)$ ее приближенным значением $\Delta^\nu\mathcal{ Y}_{r,n}^{\alpha}(f,x)$.





\chapter{О представлении решения задачи Коши для разностного уравнения рядом Фурье по функциям $\psi_{r,n}(x)$}
В настоящем разделе мы рассмотрим задачу о приближении решения разностного уравнения  суммами  Фурье по системе $\{\psi_{r,n}(x)\}_{n=0}^\infty$, ортогональной по Соболеву и порожденной ортонормированной системой функций $\{\psi_{n}(x)\}_{n=0}^\infty$ посредством равенств \eqref{ramis-2.1} и \eqref{ramis-2.2}.
Полученные ниже (теорема \ref{ramis-RItheo3}) результаты можно перенести на системы разностных  уравнений вида
$\Delta y(x)=hf(x,y), \quad y(0)=y_0$, где $f=(f_1, \ldots, f_m)$, $y=(y_1, \ldots, y_m)$. Но для простоты выкладок мы ограничимся рассмотрением задачи  вида $(\Delta y(x)=y(x+1)-y(x))$
\begin{equation}\label{ramis-3.1}
\Delta y(x)=h f(x,y), \quad y(0)=y_0, \quad h>0,
\end{equation}
в которой функцию   $f(x,y)$  будем считать заданной   на   декартовом произведении $\Omega\times\mathbb{R}$ и ограниченной на нем, т.е. $M(f)=\sup_{(x,y)\in \Omega\times\mathbb{R}}|f(x,y)|<\infty$.   Требуется аппроксимировать с заданной точностью  функцию $y=y(x)$, определенную на $\Omega$, которая является решением задачи \eqref{ramis-3.1}.

Будем считать, что  система $\{\psi_{n}(x)\}_{n=0}^\infty$ удовлетворяет условиям теоремы \ref{ramis-RItheo}, а порожденная система $\{\psi_{1,n}(x)\}_{n=0}^\infty$ условию
\begin{equation}\label{ramis-3.2}
\delta_\psi(x)=\sum_{k=1}^{\infty}(\psi_{1,k}(x))^2<\infty\quad (x\in \Omega),
\end{equation}
положительная весовая функция $\rho(x)$ и некоторая заданная неотрицательная функция $\gamma(x)$ удовлетворяют  условиям
\begin{equation}\label{ramis-3.3}
\sum\limits_{x=0}^\infty\rho(x)<\infty,\quad \kappa_{\psi}=\left(\sum_{t=0}^\infty\sum_{k=1}^{\infty}
(\psi_{1,k}(t))^2\rho(t)\gamma(t)\right)^{\frac12}<\infty.
\end{equation}
  Поскольку, по предположению, функция $f(x,y)$ ограничена на множестве $\Omega\times\mathbb{R}$, то из равенства \eqref{ramis-3.1} и первого неравенства \eqref{ramis-3.3} следует, что $$\sum_{x=0}^\infty (\Delta y(x))^2\rho(x)<\infty$$  и, следовательно,   $y(x)\in W^1_{l_\rho}$. Поэтому, в силу теоремы \ref{ramis-RItheo2}, мы можем представить $y(x)$ в виде сходящего ряда
\begin{equation}\label{ramis-3.4}
y(x)=y(0)+\sum\limits_{k=0}^\infty y_{1,k+1}\psi_{1,k+1}(x),\quad x\in \Omega.
\end{equation}
где
  \begin{equation}\label{ramis-3.5}
y_{1,k+1}=\sum_{t=0}^{\infty} \Delta y(t)\psi_{k}(t)\rho(t)\quad(k\ge0)
\end{equation}
-- коэффициенты Фурье функции $g(t)=\Delta y(t)$ по ортонормированной системе $\{\psi_k(x)\}$. С учетом \eqref{ramis-3.5} мы, в свою очередь,  можем записать
\begin{equation}\label{ramis-3.6}
\Delta y(x)=\sum\nolimits_{k=0}^\infty y_{1,k+1}\psi_{k}(x),\quad x\in \Omega.
\end{equation}
Кроме того, имея ввиду равенство \eqref{ramis-3.1}, этому равенству можно придать еще такой вид
\begin{equation}\label{ramis-3.7}
q(x)=f(x,y(x))=\frac{1}{h}\Delta y(x)=\sum\nolimits_{k=0}^\infty c_k(q)\psi_{k}(x),\quad x\in \Omega,
\end{equation}
где
\begin{equation}\label{ramis-3.8}
c_k(q)=\frac{1}{h}y_{1,k+1}=\sum_{t=0}^{\infty} f(t,y(t))\psi_{k}(t)\rho(t)\quad(k\ge0).
\end{equation}
С учетом обозначения \eqref{ramis-3.8} мы можем переписать равенство \eqref{ramis-3.4} так
\begin{equation}\label{ramis-3.9}
y(t)=y(0)+h\sum\nolimits_{j=0}^\infty c_j(q)\psi_{1,j+1}(t),\quad t\in \Omega.
\end{equation}
В равенстве  \eqref{ramis-3.8} заменим $y(t)$ правой частью равенства \eqref{ramis-3.9} и запишем
\begin{equation}\label{ramis-3.10}
c_k(q)=\sum_{t=0}^{\infty} f\left[t,y(0)+h\sum\nolimits_{j=0}^\infty c_j(q)\psi_{1,j+1}(t)\right]\psi_{k}(t)\rho(t)\quad k=0,1,\ldots.
\end{equation}
Наша цель состоит в том, чтобы сконструировать итерационный процесс для нахождения приближенных значений коэффициентов $c_k(q)$ $(k=0,1,\ldots)$.
Введем в рассмотрение гильбертово пространство $l_2$, состоящее из последовательностей $C=(c_0,c_1,\ldots)$, для которых определена норма
$\|C\|=\left(\sum_{j=0}^\infty c_j^2\right)^\frac12$.  В пространстве $l_2$ рассмотрим оператор $A$, сопоставляющий точке $C\in l_2$ точку $C'\in l_2$ по правилу
\begin{equation}\label{ramis-3.11}
c'_k=\sum_{t=0}^{\infty} f\left[t,y(0)+h\sum\nolimits_{j=0}^\infty c_j\psi_{1,j+1}(t)\right]\psi_{k}(t)\rho(t)\quad k=0,1,\ldots.
\end{equation}
Из \eqref{ramis-3.10} следует, что точка $C(q)=(c_0(q),c_1(q),\ldots)$ является неподвижной точкой оператора $A:l_2\to l_2$. Для того чтобы найти эту точку  методом простых итераций, достаточно показать, что оператор $A:l_2\to l_2$ является сжимающим в метрике пространства $l_2$. С этой целью рассмотрим две точки $P,Q\in l_2$, где $P=(p_0, p_1, \ldots)$, $Q=(q_0, q_1, \ldots)$ и положим $P'=A(P)$, $Q'=A(Q)$. Имеем
\begin{equation}\label{ramis-3.12}
p'_k-q'_k=\sum_{t=0}^{\infty}F_{P,Q}(t)\psi_k(t)\rho(t)dt,\quad k=0,1,\ldots
\end{equation}
где
\begin{equation}\label{ramis-3.13}
 F_{P,Q}(t)=f\left[t,y(0)+h\sum\nolimits_{j=0}^\infty p_j\psi_{1,j+1}(t)\right]
  -f\left[t,y(0)+h\sum\nolimits_{j=0}^\infty q_j\psi_{1,j+1}(t)\right].
\end{equation}
Из \eqref{ramis-3.12}, пользуясь неравенством Бесселя, находим
 \begin{equation}\label{ramis-3.14}
\sum\nolimits_{k=0}^\infty (p'_k-q'_k)^2\le\sum_{t=0}^{\infty}(F_{P,Q}(t))^2\rho(t).
\end{equation}
 Предположим, что по переменной $y$ функция $f(x,y)$ удовлетворяет условию Липшица
 \begin{equation}\label{ramis-3.15}
|f(x,y')-f(x,y'')|\le \sqrt{\gamma(x)}\lambda|y'-y''|, \quad  x\in\Omega,
\end{equation}
где $\gamma(x)$ -- функция, для которой имеет место неравенство \eqref{ramis-3.3}.
Из \eqref{ramis-3.13} и \eqref{ramis-3.15}  имеем
$$
(F_{P,Q}(t))^2\le (\lambda h)^2\gamma(t) \left(\sum\nolimits_{j=0}^\infty( p_j-q_j)\psi_{1,j+1}(t)\right)^2,
$$
откуда,  воспользовавшись неравенством Коши-Буняковского и условием \eqref{ramis-3.2}, выводим
 \begin{equation}\label{ramis-3.16}
(F_{P,Q}(t))^2\le(\lambda h)^2 \gamma(t)  \sum\nolimits_{j=0}^\infty( p_j-q_j)^2\sum\nolimits_{j=0}^\infty(\psi_{1,j+1}(t))^2.
\end{equation}
Сопоставляя \eqref{ramis-3.16} с \eqref{ramis-3.14}, находим
\begin{equation}\label{ramis-3.17}
\sum\nolimits_{k=1}^\infty (p'_k-q'_k)^2\le(\lambda h)^2 \sum\nolimits_{k=0}^\infty( p_k-q_k)^2\sum\nolimits_{t=0}^{\infty} \sum\nolimits_{j=0}^\infty(\psi_{1,j+1}(t))^2\rho(t)\gamma(t).
\end{equation}
Из  \eqref{ramis-3.17}  и \eqref{ramis-3.3} имеем
\begin{equation}\label{ramis-3.18}
\left(\sum\nolimits_{k=0}^\infty (p'_k-q'_k)^2\right)^\frac12\le \kappa_\psi\lambda h \left(\sum\nolimits_{k=0}^\infty (p_k-q_k)^2\right)^\frac12. \end{equation}
Неравенство \eqref{ramis-3.18} показывает, что если $\kappa_\psi\lambda h<1$, то оператор  $A:l_2\to l_2$ является сжимающим и, как следствие, итерационный процесс $C^{\nu+1}=A(C^{\nu})$  сходится к точке $C(q)$ при $\nu\to\infty$. Однако с точки зрения приложений важно рассмотреть конечномерный аналог оператора $A$. Мы рассмотрим оператор $A_N:\mathbb{R}^N\to \mathbb{R}^N$, cопоставляющий точке
$C_N=(c_0,\ldots,c_{N-1})\in \mathbb{R}^N $ точку  $C'_N=(c_0',\ldots,c_{N-1}')\in \mathbb{R}^N $ по правилу
\begin{equation}\label{ramis-3.19}
c'_k=\sum_{t=0}^{\infty} f\left[t,y(0)+h\sum\nolimits_{j=0}^{N-1} c_j\psi_{1,j+1}(t)\right]\psi_{k}(t)\rho(t)\quad k=0,1,\ldots,N-1
\end{equation}
 Рассмотрим две точки $P_N,Q_N\in \mathbb{R}^N$, где $P_N=(p_0,p_1,\ldots,p_{N-1})$,\\   $Q_N=(q_0,q_1,\ldots,p_{N-1})$ и положим $P'_N=A_N(P_N)$, $Q'_N=A_N(Q_N)$. Дословно повторяя рассуждения, которые привели нас к неравенству \eqref{ramis-3.18}, мы получим
\begin{equation}\label{ramis-3.20}
\left(\sum\nolimits_{k=0}^{N-1} (p'_k-q'_k)^2\right)^\frac12\le \kappa_\psi\lambda h \left(\sum\nolimits_{k=0}^{N-1} (p_k-q_k)^2\right)^\frac12.
\end{equation}
Неравенство \eqref{ramis-3.20} показывает, что если $\kappa_\psi\lambda h<1$, то оператор  $A_N:\mathbb{R}^N\to \mathbb{R}^N$ является сжимающим и, как следствие, итерационный процесс $C_N^{\nu+1}=A_N(C_N^{\nu})$  при $\nu\to\infty$ сходится к его неподвижной точке, которую мы обозначим через  $\bar C_N(q)=(\bar c_0(q),\ldots,\bar c_{N-1}(q))$ . С другой стороны, рассмотрим точку $C_N(q)=(c_0(q),\ldots,c_{N-1}(q))$, составленную из искомых коэффициентов Фурье функции $q$ по системе $\{\psi_k\}$. Нам остается оценить погрешность, проистекающую в результате замены точки $C_N(q)$ точкой $\bar C_N(q)$. Другими словами, требуется оценить величину
$\|C_N(q)-\bar C_N(q)\|_N= \left(\sum_{j=0}^{N-1}(c_j(q)-\bar c_j(q))^2\right)^\frac12$. С этой целью рассмотрим точку $C'_N(q)=A_N(C_N(q))=(c_0'(q),\ldots,c_{N-1}'(q))$ и запишем
\begin{equation}\label{ramis-3.21}
\|C_N(q)-\bar C_N(q)\|_N\le \|C_N(q)- C_N'(q)\|_N+\|C_N'(q)-\bar C_N(q)\|_N.
\end{equation}
Далее, пользуясь неравенством \eqref{ramis-3.20}, имеем
$$
\|C_N'(q)-\bar C_N(q)\|_N=\|A_N(C_N(q))-A_N(\bar C_N(q))\|_N\le
$$
\begin{equation}\label{ramis-3.22}
\kappa_\psi\lambda h\|C_N(q)-\bar C_N(q)\|_N.
\end{equation}
Из \eqref{ramis-3.21} и \eqref{ramis-3.22} выводим
\begin{equation}\label{ramis-3.23}
\|C_N(q)-\bar C_N(q)\|_N\le \frac1{1-\kappa_\psi\lambda h}\|C_N(q)- C_N'(q)\|_N.
\end{equation}
Чтобы оценить норму в правой части неравенства \eqref{ramis-3.22}, заметим, что в силу неравенства Бесселя
\begin{equation}\label{ramis-3.24}
\|C_N(q)- C_N'(q)\|_N^2\le \sum_{t=0}^{\infty} (F_{C(q),C_N(q)}(t))^2\rho(t),
\end{equation}
где
\begin{multline}\label{ramis-3.25}
 F_{C(q),C_N(q)}(t)=f\left[t,y(0)+ h\sum\nolimits_{j=0}^\infty c_j(q)\psi_{1,j+1}(t)\right] \\
  -f\left[t,y(0)+ h\sum\nolimits_{j=0}^{N-1}c_j(q)\psi_{1,j+1}(t)\right].
\end{multline}
Из \eqref{ramis-3.24} и \eqref{ramis-3.15} следует, что
$$
(F_{C(q),C_N(q)}(t))^2\le \lambda^2 \gamma(t)\left(\sum\nolimits_{j=N}^\infty hc_j(q)\psi_{1,j+1}(t)\right)^2,
$$
отсюда с учетом \eqref{ramis-3.8} имеем
\begin{equation}\label{ramis-3.26}
(F_{C(q),C_N(q)}(t))^2\le \lambda^2 \gamma(t)  \left(\sum\nolimits_{j=N}^\infty  y_{1,j+1}\psi_{1,j}(t)\right)^2.
\end{equation}
Сопоставляя \eqref{ramis-3.26} с \eqref{ramis-3.24}, получаем
\begin{equation}\label{ramis-3.27}
\|C_N(q)- C_N'(q)\|_N^2\le \lambda^2\sum_{t=0}^\infty\left(\sum\nolimits_{j=N}^\infty y_{1,j+1} \psi_{1,j+1}(t)\right)^2\rho(t)\gamma(t).
\end{equation}

Подводя итоги, из \eqref{ramis-3.23} и \eqref{ramis-3.27}  мы можем вывести следующий результат.

\begin{theorem}\label{ramis-RItheo3}
Пусть функция   $f(x,y)$  определена и ограничена на декартовом произведении $\Omega\times\mathbb{R}$ и удовлетворяет условию Липшица \eqref{ramis-3.15}, а $h$ и $\lambda$ удовлетворяет неравенству $h\lambda\kappa_\psi<1$, где величина $\kappa_\psi$ определена равенством \eqref{ramis-3.3}. Далее, пусть $l_2$ гильбертово пространство, состоящее из последовательностей $C=(c_0,\ldots)$, для которых введена норма $\|C\|=\left(\sum_{j=0}^\infty c_j^2\right)^\frac12$,   оператор $A: l_2\to l_2$ сопоставляет точке $C\in l_2$ точку $C'\in l_2$ по правилу \eqref{ramis-3.11}. Кроме того, пусть $A_N:\mathbb{R}^N\to \mathbb{R}^N$ -- конечномерный аналог оператора $A$, cопоставляющий точке $C_N=(c_0,\ldots,c_{N})\in \mathbb{R}^N $ точку  $C'_N=(c_0',\ldots,c_{N}')\in \mathbb{R}^N $ по правилу \eqref{ramis-3.19}.
Тогда операторы $A: l_2\to l_2$ и $A_N:\mathbb{R}^N\to \mathbb{R}^N$ являются сжимающими и, следовательно, существуют  их неподвижные точки $C(q)=(c_0(q),c_1(q),\ldots)=A(C(q))\in l_2$ и $\bar C_N(q)=(\bar c_0(q),\bar c_0(q),\ldots,\bar c_{N}(q))=A_N(\bar C_N(q))\in \mathbb{R}^N$, для которых имеет место неравенство
\begin{equation}\label{ramis-3.28}
\|C_N(q)-\bar C_N(q)\|_N\le \frac{\lambda \sigma_N^\psi(y)}{1-h\kappa_\psi\lambda},
\end{equation}
где
\begin{equation}\label{ramis-3.30}
\sigma_N^\psi(y)=\left(\sum_{t=0}^\infty\left(\sum\nolimits_{j=N+1}^\infty  y_{1,j}\psi_{1,j}(t)\right)^2\rho(t)\gamma(t) \right)^\frac12,
\end{equation}
 a $C_N(q)=(c_0(q),\ldots,c_{N-1}(q))$ -- конечная последовательность, составленная из первых $N$ компонент точки  $C(q)$, при этом в силу  \eqref{ramis-3.8} справедливо также равенство  $hC_N(q)=(y_{1,1}, y_{1,2}, \ldots, y_{1,N})$.
\end{theorem}






%%%%%%%%%%%%%%%%%%%%%%%
%%%%%%%%%%%%%%%%%%%%%%%
%%%%%%%%%%%%%%%%%%%%%%%





%\chapter{Рекуррентные соотношения для полиномов, ортонормированных по Соболеву, порождённых полиномами Лагерра}
%
%%\section{Некоторые сведения о полиномах Лагерра}
%%При получении рекуррентных формул для полиномов, ортонормированных по Соболеву, порождённых полиномами Лагерра, нам понадобится некоторые свойства самих полиномов Лагерра $L_n^\alpha(x)$, которые приведём в этом пункте.
%%
%%Пусть $\alpha$ -- произвольное действительное число. Тогда для полиномов Лагерра имеют место \cite{ramis-Gadz3}:
%%
%%формула Родрига
%%\begin{equation*}
%%L_n^{\alpha}(x) = \frac{1}{n!}x^{-\alpha}e^{x} \left\{ x^{n+\alpha} e^{-x} \right\}^{(n)};
%%\end{equation*}
%%
%%явный вид
%%\begin{equation}\label{ramis-Gadz_eq2}
%%L_n^\alpha(x) =\sum\limits_{k=0}^{n}\binom{n+\alpha}{n-k}\frac{(-x)^k}{k!};
%%\end{equation}
%%
%%соотношение ортогональности
%%\begin{equation}\label{ramis-Gadz_eq3}
%%\int_0^{\infty} x^{\alpha} e^{-x} L^{\alpha}_{n}(x) L^{\alpha}_{m}(x) dx = \delta_{nm} h^{\alpha}_n \quad (\alpha > -1),
%%\end{equation}
%%где $\delta_{nm}$ -- символ Кронекера,
%%\begin{equation*}
%%h^{\alpha}_n = \left( n+\alpha \atop n \right) \Gamma(\alpha +1);
%%\end{equation*}
%%
%%равенства
%%\begin{equation}\label{ramis-Gadz_eqRav}
%%\frac{d}{dx} L_n^{\alpha}(x) = -L_{n-1}^{\alpha+1}(x);
%%\end{equation}
%%
%%\begin{equation}\label{ramis-Gadz_eqRav1}
%%L_{n}^{\alpha}(x) =L_{n}^{\alpha+1}(x) -L_{n-1}^{\alpha+1}(x);
%%\end{equation}
%%
%%рекуррентная формула
%%\begin{equation}\label{ramis-Gadz_eq4}
%%\left.\begin{gathered}
%%L_{0}^{\alpha}(x)=1, \quad L_1^{\alpha}(x)=-x+\alpha+1,\\
%%nL_n^{\alpha}(x)=(-x+2n+\alpha-1)L_{n-1}^{\alpha}(x)-(n+\alpha-1)L_{n-2}^{\alpha}(x), \quad n=2, 3, \ldots
%%\end{gathered}\right\}.
%%\end{equation}
%%
%%Из \eqref{ramis-Gadz_eq3} следует, что соответствующая ортонормированная система полиномов Лагерра имеет вид:
%%\begin{equation}\label{ramis-Gadz_eq5}
%%l_n^{\alpha}(x)=(h^{\alpha}_n)^{-\frac{1}{2}}L_n^{\alpha}(x), \quad n=0, 1, \ldots,
%%\end{equation}
%%т.е.
%%\begin{equation*}
%%\int_0^{\infty} x^{\alpha} e^{-x} l^{\alpha}_{n}(x) l^{\alpha}_{m}(x) dx = \delta_{nm} \quad (\alpha > -1).
%%\end{equation*}
%%
%%Равенства \eqref{ramis-Gadz_eq2} и \eqref{ramis-Gadz_eq5} дают следующий явный вид для $l^{\alpha}_{n}(x)$:
%%\begin{equation}\label{ramis-Gadz_eq6}
%%l_n^\alpha(x) =\frac{1}{(h^{\alpha}_n)^{\frac{1}{2}}}
%%\sum\limits_{k=0}^{n}\binom{n+\alpha}{n-k}\frac{(-x)^k}{k!};
%%\end{equation}
%%Из \eqref{ramis-Gadz_eq4} и \eqref{ramis-Gadz_eq5} легко можно получить рекуррентную формулу для ортонормированных полиномов Лагерра $l_n^\alpha(x)$:
%%\begin{equation}\label{ramis-Gadz_eq7}
%%\left.\begin{gathered}
%%l_{0}^{\alpha}(x)=\frac{1}{(\Gamma(\alpha+1))^{1\over2}}, \quad l_1^{\alpha}(x)=\frac{-x+\alpha+1}{(\Gamma(\alpha+2))^{1\over2}},\\
%%l_n^{\alpha}(x)=(a_n^\alpha-b_n^\alpha x)l_{n-1}^{\alpha}(x)-c_n^\alpha l_{n-2}^{\alpha}(x), \quad n=2, 3, \ldots
%%\end{gathered}\right\},
%%\end{equation}
%%где
%%\begin{equation*}
%%a_n^\alpha=\frac{2n+\alpha-1}{[n(n+\alpha)]^\frac{1}{2}},\quad
%%b_n^\alpha=\frac{1}{[n(n+\alpha)]^\frac{1}{2}},\quad
%%c_n^\alpha=\Big[\frac{(n-1)(n+\alpha-1)}{n(n+\alpha)}\Big]^\frac{1}{2}.
%%\end{equation*}
%%
%%Далее, через $\mu_n^\alpha(x)$ обозначим функции Лагерра, которые задаются следующим образом:
%%\begin{equation}\label{ramis-funcLag}
%%\mu_n^\alpha(x)=(\rho(x))^\frac12l_n^\alpha(x)\ (n=0, 1, \ldots).
%%\end{equation}
%%Они ортонормированы на множестве $[0, \infty)$ с единичным весом, т.е.
%%\begin{equation*}
%%\langle \mu_n^\alpha, \mu_k^\alpha \rangle=\int\limits_0^\infty \mu_n^\alpha(x)\mu_k^\alpha(x)=\delta_{nk}, \quad (\alpha>-1).
%%\end{equation*}
%%
%%Так как функции $\mu_n^\alpha(x)$ отличаются от полиномов $l_n^\alpha(x)$ множителем, не зависящим от номера функции, следовательно, аналогичная рекуррентная формула справедлива и для функций $\mu_n^\alpha(x)$:
%%\begin{equation*}
%%\left.\begin{gathered}
%%\mu_{0}^{\alpha}(x)=\frac{(\rho(x))^{1\over2}}{(\Gamma(\alpha+1))^{1\over2}}, \quad \mu_1^{\alpha}(x)=\frac{(\rho(x))^{1\over2}(-x+\alpha+1)}{(\Gamma(\alpha+2))^{1\over2}},\\
%%\mu_n^{\alpha}(x)=(a_n^\alpha-b_n^\alpha x)\mu_{n-1}^{\alpha}(x)-c_n^\alpha \mu_{n-2}^{\alpha}(x), \quad n=2, 3, \ldots
%%\end{gathered}\right\}.
%%\end{equation*}
%
%%\section{Некоторые сведения о полиномах, ортонормированных по Соболеву, порожденных полиномами Лагерра}
%%Пусть $\alpha>-1$, $\rho=\rho(x)=e^{-x}x^\alpha$, $L^2_\rho$ -- пространство измеримых функций $f$, определенных на полуоси $[0, \infty)$ и таких, что
%%$$
%%\|f\|_{L^2_\rho}=\left(\int\limits_0^{\infty}f^2(x)\rho(x)dx\right)^\frac{1}{2}<\infty.
%%$$
%%Через $W^r_{L^2_\rho}$ обозначим подкласс функций $f$, непрерывно дифференцируемых $r-1$ раз, для которых $f^{(r-1)}$ абсолютно непрерывна на произвольном сегменте $[a, b]\subset[0, \infty)$, а $f^{(r)}\in L^2_\rho$.    В пространстве $W^r_{L^2_\rho}$ мы введем скалярное произведение \eqref{Gadz_eq1}.
%%
%%Ортонормированная система полиномов Лагерра $l_{n}^\alpha(x)$ ($n=0, 1, \ldots$) порождает \cite{ramis-Gadz1}, \cite{ramis-Gadz2} на $[0, \infty)$ систему полиномов $l_{r,n}^\alpha(x)$ ($r$ -- натуральное число, $n=0, 1, \ldots$), определенных равенствами
%%\begin{equation}\label{ramis-Gadz_eq8}
%%l_{r,r+n}^{\alpha}(x) =\frac{1}{(r-1)!}\int\limits_{0}^x(x-t)^{r-1}l_{n}^{\alpha}(t)dt, \quad n=0,1,\ldots.
%%\end{equation}
%%
%%\begin{equation}\label{ramis-Gadz_eq9}
%%l_{r,n}^{\alpha}(x) =\frac{x^n}{n!}, \quad n=0,1,\ldots, r-1.
%%\end{equation}
%%В работе \cite{ramis-Gadz2} доказана следующая теорема
%%
%%\begin{theoremA}\label{ramis-RItheoA}
%%Если $\alpha>-1$, то система полиномов $l_{r,n}^{\alpha}(x)$ $(n=0, 1,\ldots)$, порожденная полиномами Лагерра $l_n^{\alpha}(x)$ $(n=0,1,\ldots)$ посредством равенств \eqref{ramis-Gadz_eq8} и \eqref{ramis-Gadz_eq9}, полна  в $W^r_{L^2_\rho}$ и ортонормирована относительно скалярного произведения \eqref{Gadz_eq1}.
%%\end{theoremA}
%%
%%Кроме того в работах \cite{ramis-Gadz1}, \cite{ramis-Gadz2} получены следующие представления для полиномов $l_{r,r+n}^{\alpha}(x)$:
%%
%%\begin{equation}\label{ramis-Gadz_eq10}
%%l_{r,n+r}^{\alpha}(x)=\frac{1}{(h_n^\alpha)^{1\over2}}
%%\sum\limits_{\nu=0}^{n}(-1)^\nu \binom{n+\alpha}{n-\nu}
%%\frac{x^{\nu+r}}{\nu!(\nu+r)^{[r]}}\quad (n=0,1,\ldots),
%%\end{equation}
%%
%%\begin{equation*}
%%l_{r,r+n}^{\alpha}(x)=\frac{(-1)^r}{(h_n^\alpha)^{1\over2}}\left[L_{n+r}^{\alpha-r}(x)-
%%\sum_{\nu=0}^{r-1}{B_{n,\nu}^\alpha x^\nu\over\nu!}\right],
%%\end{equation*}
%%где
%%$$
%%B_{n,\nu}^\alpha=\{L_{n+r}^{\alpha-r}(x)\}^{(\nu)}_{x=0}=
%%{(-1)^\nu\Gamma(n+\alpha+1)\over\Gamma(\nu-r+ \alpha+1)(n+r-\nu)!}.
%%$$
%%Сравнивая равенства \eqref{ramis-Gadz_eq6} и \eqref{ramis-Gadz_eq10} замечаем, что порожденные полиномы $l_{r,r+n}^{\alpha}(x)$ при $r=0$ совпадают с классическими полиномами Лагерра $l_{n}^{\alpha}(x)$.
%%
%%В свою очередь, система функций Лагерра $\mu_n^\alpha(x)$, определенная равенством \eqref{ramis-funcLag}, порождает на $[0, \infty)$ систему функций $\mu_{r,n}^\alpha(x)$ ($r$ -- натуральное число, $n=0, 1, \ldots$) посредством равенств
%%\begin{equation}\label{ramis-Gadz_eq12}
%%\mu_{r,r+n}^{\alpha}(x) =\frac{1}{(r-1)!}\int\limits_{0}^x(x-t)^{r-1}\mu_{n}^{\alpha}(t)dt, \quad n=0,1,\ldots.
%%\end{equation}
%%
%%\begin{equation}\label{ramis-Gadz_eq13}
%%\mu_{r,n}^{\alpha}(x) =\frac{x^n}{n!}, \quad n=0,1,\ldots, r-1.
%%\end{equation}
%%
%%В качестве следствия теоремы 1 из работы \cite{ramis-Gadz1} можно сформулировать следующее
%%\textbf{Следствие.}
%%\textsl{
%%Если $\alpha>-1$, то система функций $\mu_{r,n}^{\alpha}(x)$ $(n=0, 1,\ldots)$, порожденная функциями Лагерра $\mu_n^{\alpha}(x)$ $(n=0,1,\ldots)$ посредством равенств \eqref{ramis-Gadz_eq12} и \eqref{ramis-Gadz_eq13}, полна  в $W^r_{L^2_1}$ и ортонормирована относительно скалярного произведения
%%\begin{equation*}
%%\langle \mu_{r,n}^\alpha,\mu_{r,k}^\alpha\rangle=
%%\sum_{\nu=0}^{r-1}(\mu_{r,n}^\alpha(x))^{(\nu)}|_{x=0}(\mu_{r,k}^\alpha(x))^{(\nu)}|_{x=0}+
%%\int_{0}^{\infty} (\mu_{r,n}^\alpha(x))^{(r)}(\mu_{r,k}^\alpha(x))^{(r)} dx.
%%\end{equation*}
%%}
%%
%
