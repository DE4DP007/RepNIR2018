%\chapter{Программа MixedHaarDeqSolver для решения задачи Коши на отрезке [0,1]}
\chapter{Прикладные пакеты программ}


\section{Программа  для решения задачи Коши на отрезке [0,1]}

Программа <<MixedHaarDeqSolver>> позволяет найти решение задачи Коши для линейных обыкновенных дифференциальных уравнений численным методом, основанным на разложении самой функции и ее производных в смешанный ряд по системе Хаара. Основное преимущество данного метода заключается в том, что в разложении самой функции и всех ее производных участвуют одни и те же коэффициенты. При этом ввиду особенностей конструкции смешанного ряда самым естественным образом оказываются учтенными начальные условия задачи Коши.

\textit{Численный метод }включает в себя следующие шаги:
\begin{enumerate}[1)]
  \item неизвестная функция и ее производные разлагаются в смешанный ряд по системе Хаара. В результате дифференциальное уравнение переходит в алгебраическое относительно неизвестных коэффициентов разложения;
  \item рассматривая полученное уравнение в фиксированных узлах, составляется система уравнений относительно коэффициентов;
  \item для решения полученной системы применяется метод наименьших квадратов.
\end{enumerate}


Программа может быть использована для решения задач, возникающих в математической физике, математическом моделировании, в задачах идентификации линейных систем автоматического регулирования и управления.







\section{Программа для приближения дискретных функций средними Валле Пуссена частичных сумм конечных предельных рядов}

Пусть $\tau_{n}^{\alpha,\beta}(x) = \tau_{n}^{\alpha,\beta}(x,N), (n=0 \ldots N-1)$ --- дискретные полиномы Чебышева, образующие ортонормированную систему на сетке $\Omega_N = \{0,1,\ldots,N-1 \}$ в смысле скалярного произведения
\begin{equation} \sum_{x=0}^{N-1}\mu(x)\tau_n^{\alpha,\beta}(x,N)
\tau_m^{\alpha,\beta}(x,N)=\delta_{nm},
\end{equation}
где
\begin{equation}\label{smsweig}
 \mu(x)=\mu(x;\alpha,\beta,N)={\Gamma(N)2^{\alpha+\beta+1} \over
\Gamma(N+\alpha+\beta+1)}{\Gamma(x+\beta+1)
\Gamma(N-x+\alpha)\over \Gamma(x+1)\Gamma(N-x)}.
\end{equation}
Тогда для заданной на сетке $\Omega_N$ дискретной функции $f(x)$ мы можем определить дискретный ряд Фурье -- Чебышева
\begin{equation}
\label{sms2eq1}
f(x)=\sum\limits_{k=0}^{N-1}\hat{f}^{\alpha,\beta}_k\tau_{k}^{\alpha,\beta}(x),\quad (x \in \Omega_N),
\end{equation}
где коэффициенты Фурье -- Чебышева задаются формулой
\begin{equation}
\label{sms2eq2}
\hat{f}_k = \sum\limits_{x=0}^{N-1}f(x)\tau_{k}^{\alpha,\beta}(x)\mu(x),\quad (0\le n\le N-1).
\end{equation}
Частичная сумма $n$-го порядка может быть записана в виде
\begin{equation}
\label{sms2eq5}
S_{n,N}^{\alpha,\beta}(f,x) = \sum\limits_{k=0}^{n}\hat{f}^{\alpha,\beta}_k \tau_{k}^{\alpha,\beta}(x),\quad (0\le x\le N-1).
\end{equation}
Далее для краткости в случаях, когда $\alpha = \beta$, мы будем записывать в качестве индекса лишь один параметр, например: $\tau_{n}^{\alpha}(x) = \tau_{n}^{\alpha,\alpha}(x)$, $S_{n,N}^{\alpha}(f,x) = S_{n,N}^{\alpha,\alpha}(f,x)$ и т.д.

Хорошо известно, что система дискретных полиномов Чебышева теряет свойство ортогональности при нарушении условия $\alpha, \beta > -1$.
Если взять $\alpha = \beta = -1$, то вес $\eqref{smsweig}$ обращается в бесконечность в точках $x=0$ и $x=N-1$, и суммы $\sum\limits_{x=0}^{N-1} f(x)\tau_{k}^{-1}(x)\mu^{-1}(x)$ теряют смысл.

В работе \cite{smsshti1} показано, что можно произвести почленный предельный переход в дискретном ряде \eqref{sms2eq1} при $\alpha,\beta \rightarrow -1$, т.е. $f(x) = \sum\limits_{k=0}^{N-1}\hat{f}^{-1}_k\tau_{k}^{-1}(x),\quad (x \in \Omega_N)$.
Получаемый в результате дискретный ряд называется конечным предельным рядом:
\begin{equation}
\label{limitR}
 f(x) = \sum\limits_{k=0}^{N-1}\hat{f}^{-1}_k  \tau_{k}^{-1}(x) =
a_N(f,x) + \frac{8x(N-x-1)}{N(N-1)} \sum\limits_{k=0}^{N-3} \hat{g}_k \tau_{k}^{1}(x-1,N-2),
\end{equation}
где
\begin{equation}
\label{af}
a_N(f,x) = \frac{f(N-1)+f(0)}{2} + \frac{f(N-1)-f(0)}{2}\left( \frac{2x}{N-1}-1\right),
\end{equation}
%\begin{equation}
%\label{bnN}
%b_{n,N} = \frac{8x(N-x-1)}{N(N-1)},
%\end{equation}
а коэффициенты
\begin{equation}
\label{gk}
\hat{g}_k = \hat{g}_k(N) = \frac{1}{N-2} \sum\limits_{j=1}^{N-2} g(j) \tau_{k}^{1}(j-1,N-2) ,
\end{equation}
получены для функции $g(x) = f(x) - a_N(f,x)$.
Частичные суммы конечного предельного ряда могут быть записаны
\begin{equation}
\label{limitS}
 S_{n,N}^{-1}(f,x) = \lim_{\alpha \rightarrow -1} S_{n,N}^{\alpha}(f,x) =
 a_N(f,x) + \frac{8x(N-x-1)}{N(N-1)} \sum\limits_{k=0}^{n-2} \hat{g}_k \tau_{k}^{1}(x-1,N-2).
\end{equation}

В работе \cite{smsshti1} показано, что они обладают следующими тремя свойствами:
\begin{enumerate}[1)]
  \item $S_{n,N}^{-1}(f,x)$ совпадает с $f(x)$ в точках $x=0$ и $x=N-1$:%, т.е. $S_{n,N}^{-1}(f,0) = f(0)$, $S_{n,N}^{-1}(f,N-1)=f(N-1)$;
    \begin{equation}\label{limprop1}
        S_{n,N}^{-1}(f,0) = f(0), \quad S_{n,N}^{-1}(f,N-1)=f(N-1);
    \end{equation}
  \item $S_{n,N}^{-1}(f,x)$ представляет собой проектор на пространство $H_n$ всех алгебраических полиномов $P_n(x)$ степени, не выше $n$:%, т.е.   $S_{n,N}^{-1}(P_n,x) = P_n(x)$;
    \begin{equation}\label{limprop2}
        S_{n,N}^{-1}(P_n,x) = P_n(x);
    \end{equation}
  \item Если $0\le x\le N-3$, $n\le a\sqrt{N}$, то имеет место следующая оценка
    \begin{equation}\label{applimitR}
        |f(x)-S_{n,N}^{-1}(f,x)|\le c(a)E_{n,N}^{*}(f)\left[1+\ln\left(2+\frac{n}{N-1}\sqrt{x(N-3-x)}\right)\right],
    \end{equation}
    где $c(a)>0$ --- константа,
    $$E_{n,N}^{*}(f) = \inf_{p_n \in \hat{H}^n} \sup_{0\le x\le N-3} |f(x)-p_n(x)|$$
    -- наилучшее приближение функции $f(x)$ алгебраическими полиномами, которые в точках $0$ и $N-1$ совпадают со значениями самой функции $f(x)$.
\end{enumerate}


Аналогично классическому случаю мы вводим в рассмотрение усреднение частичных сумм предельного ряда \eqref{limitS} вида
\begin{equation}\label{limitVP}
  \mathcal{V}^{-1}_{m,n}(f,x) = \frac{S_{m,N}^{-1}(f,x) + S_{m+1,N}^{-1}(f,x) + \ldots + S_{m+n,N}^{-1}(f,x)}{n+1},
\end{equation}
которое назовем \textit{средними Валле Пуссена для частичных сумм предельного ряда}.

Эти новые операторы могут быть записаны с помощью следующего выражения
$$
\mathcal{V}_{m,n,N}^{-1}(f,x)=
a_f(x) + { 8x(N-1-x)\over N(N-1)}
\left[
\sum_{k=0}^{m-2}\hat g_k\tau_{k,N-2}^{1,1}(x-1) + \right.
$$
\begin{equation*}
\left.
\sum_{k=m-1}^{m+n-2} \frac{m+n-k+1}{n+1} \hat g_k \tau_{k,N-2}^{1,1}(x-1)
\right],
\end{equation*}
откуда
$$
\mathcal{V}_{m,n,N}^{-1}(f,x)=
S_{m+n,N}^{-1}(f,x) -
{ 8x(N-1-x)\over N(N-1)}
\sum_{k=m-1}^{m+n-2} \frac{k-m}{n+1} \hat g_k \tau_{k,N-2}^{1,1}(x-1).
$$

Нетрудно показать, что операторы $\mathcal{V}_{m,n,N}^{-1}(f,x)$ обладают свойством совпадения с исходной функцией в концевых узлах сетки
$$
\mathcal{V}_{m,n,N}^{-1}(f,0)= f(0), \quad \mathcal{V}_{m,n,N}^{-1}(f,N-1)= f(N-1).
$$
а также свойством проективности над пространством $H_{m}$ алгебраических полиномов степени не выше $m$: $\mathcal{V}_{m,n,N}^{-1}(P_m,x)\equiv P_m(x)$.

%Остается открытой задача исследования аппроксимативных свойств операторов \linebreak $\mathcal{V}_{m,n,N}^{-1}(f,x)$, которая в свою очередь сводится к задаче исследования нормы оператора в пространстве $C[0, N-1]$:
%$\| \mathcal{V}^{-1}_{m,n} \| = \sup_{\| f \| \leq 1} \| \mathcal{V}^{-1}_{m,n}(f,x) \|$.
%
%В отчетном году на пути к решению этой задачи было доказано несколько вспомогательных утверждений и разработан специализированный пакет прикладных программ, с помощью которого проведена серия численных экспериментов по исследованию поведения указанного оператора на дискретных функциях разных типов.

Разработан специализированный пакет прикладных программ для обработки и сжатия временных рядов на основе оператора \eqref{limitVP}. В качестве исходных данных программа принимает дискретную функцию в одной из двух форм: либо в виде аналитического выражения с указанием частоты дискретизации $N$, либо в виде массива действительных чисел, представляющих собой значения функции в узлах равномерной сетки $\Omega_N$.
Пользователю предоставляется возможность выбора задания параметров и уровня сжатия.
Программа возвращает результат преобразования исходных данных, а также погрешность приближения.
Имеется возможность визуализации при помощи построения  графиков исходного ряда данных и результатов аппроксимации.

Кроме того, пользователь может выбрать один из альтернативных операторов для осуществления обработки исходных данных и сравнения результатов с преобразованием на основе оператора \eqref{limitVP}.
С помощью разработанного программного пакета проведена серия численных экспериментов по исследованию поведения указанного оператора на дискретных функциях разных типов.


%Полученные в указанных пунктах фундаментальные результаты нашли практическую реализацию в разработанном пакете программ для обработки и сжатия временных рядов, звука и изображений на основе предельных и специальных дискретных рядов по полиномам, ортогональным на равномерных и неравномерных сетках. В качестве исходных данных программа принимает массив одномерных или двумерных данных. Пользователю предоставляется возможность выбора оператора для осуществления обработки, задания параметров, а также уровня сжатия. Программа возвращает результат преобразования исходных данных, а также погрешность приближения. Кроме того, есть возможность визуализации при помощи построения графиков исходных ряда данных и аппроксимации, отрисовки исходного и восстановленного изображений, воспроизведения исходного и восстановленного звукового сигнала.








\section{Обработка массивов данных с помощью специальных вейвлет-рядов}


Пусть $w(x) = \sqrt{1-x^2}$. Обозначим тогда через $L_{2, w}([-1; 1])$ евклидово пространство интегрируемых функций $f(x)$, таких что
$\int\limits_{-1}^{1} f^2(x)w(x)dx < \infty$.
Определим скалярное произведение в нем с помощью равенства
$ <f, g> = \int\limits_{-1}^{1} f(x) g(x) w(x) dx$.
Хорошо известно, что полиномы Чебышева второго рода
 \begin{equation*}
\label{u2direct}
U_n(x) = \frac{\sin((n+1)\arccos{x})}{\sqrt{1-x^2}}, \quad n = 0,1,2, \ldots .
\end{equation*}
образуют ортогональный базис в $L_{2, w}([-1; 1])$.
Нули $n$-го полинома $U_n(x)$, очевидно, могут быть определены равенством
\begin{equation*}
\label{sms1zeros}
\xi_{k}^{(n)} = \cos{\theta_{k}^{(n)}} =  \cos{\frac{\pi (k+1)}{n+1}}, \quad (k = 0,...,n-1).
\end{equation*}

\begin{definition}
Масштабирующей функцией Чебышева второго рода назовем полином вида
\begin{equation*}
\label{scaling}
\phi_{n,k}(x) = \sum\limits_{j=0}^{n}U_{j}(x)U_{j}(\xi_{k}^{(n+1)}),
\end{equation*}
где
$n=1,2,\ldots$ и $k=0,1,\ldots,n$.
%}
\end{definition}

\begin{definition}
%\noindent\textbf{Определение 2. }\textit{
Назовем вейвлет--функцией Чебышева второго рода полином
\begin{equation*}
\label{sms1wavelet}
\psi_{n,k}(x) = \sum\limits_{j=n+1}^{2n}U_{j}(x)U_{j}(\xi_{k}^{(n)}),
\end{equation*}
для любых
$n=1,2,\ldots$ и $k=0,1,\ldots,n-1$.
%}
\end{definition}

В нашей работе \cite{sms11}
на основе функций  $\left\{ \phi_{n,k}(x)\right\}_{k=0}^{n}$ и $\left\{ \psi_{n,k}(x)\right\}_{k=0}^{n-1}$ строится ортонормированный базис в $L_{2, w}([-1; 1])$:
$\mathcal{P} = \left\{\Phi_{0}, \Psi_1, \Psi_2, \ldots, \Psi_m, \ldots \right\} =
\left\{\hat{\phi}_{0,0}(x), \hat{\phi}_{0,1}(x), \hat{\psi}_{0,0}(x), \hat{\psi}_{1,0}(x), \right.$

\noindent$\left. \hat{\psi}_{1,1}(x), \ldots, \hat{\psi}_{m-1,0}(x), \hat{\psi}_{m-1,1}(x), \ldots, \hat{\psi}_{m-1, 2^{m-1}-1}(x), \ldots \right\}.$


Следовательно, произвольная функция $f(x) \in L_{2, w}([-1; 1])$, может быть представлена в виде сходящегося в $L_{2, w}([-1; 1])$ ряда
\begin{equation}
\label{sms1fdistr}
f(x) = \hat{a}_{0}\hat{\phi}_{0,0}(x) + \hat{a}_{1}\hat{\phi}_{0,1}(x) + \sum\limits_{j=0}^{\infty} \sum\limits_{k=0}^{2^j-1} \hat{b}_{j,k}\hat{\psi}_{j,k}(x),
\end{equation}
\begin{equation*}
\label{sms1fcoeffA}
\text{где}\quad \hat{a}_{0} = \int\limits_{-1}^{1} f(t)\hat{\phi}_{0,0}(t)w(t)dt, \quad \hat{a}_{1} = \int\limits_{-1}^{1} f(t)\hat{\phi}_{0,1}(t)w(t)dt,
\end{equation*}
\begin{equation*}
\label{sms1fcoeffB}
\hat{b}_{j,k} = \int\limits_{-1}^{1} f(t)\hat{\psi}_{j,k}(t)w(t)dt, \quad (j=0,1, \ldots, m; k = 0, 1, \ldots, 2^j-1).
\end{equation*}
Через $\mathcal{V}_{2^m}(f,x)$ обозначим частичную сумму ряда \eqref{sms1fdistr} следующего вида
\begin{equation}
\label{sms1wavepartsum}
\mathcal{V}_{2^m}(f,x) = \hat{a}_{0}\hat{\phi}_{0,0}(x) + \hat{a}_{1}\hat{\phi}_{0,1}(x) + \sum\limits_{j=0}^{m-1} \sum\limits_{k=0}^{2^j-1} \hat{b}_{j,k}\hat{\psi}_{j,k}(x).
\end{equation}
Для каждой внутренней точки отрезка $x \in (-1, 1)$ справедлива оценка погрешности приближения
\begin{equation}
\label{sms1v2mforC}
\left| f(x) - \mathcal{V}_{2^m}(f, x) \right| \leq
E_{2^{m}}(f) \left(\frac{4 \ln{2}}{\pi^2}m + O(1)\right),  \quad f \in C[-1,1].
\end{equation}

Тем не менее, как показано в работе \cite{sms112}, частичные суммы $\mathcal{V}_{2^m}(f, x)$ на концах отрезка $[-1,1]$ плохо приближают не только непрерывные функции $f \in C[-1,1]$, но также аналитические функции (за исключением алгебраических полиномов).
Может случится, что $\mathcal{V}_{2^m}(f, x)$ приближает $f(x)$ по порядку в $2^m$ раз хуже, чем полином наилучшего приближения $P^{*}_{2^m}(f,x)$.

Чтобы устранить указанный негативный эффект, предлагается модифицировать вейвлет-ряд \eqref{sms1fdistr} по схеме, схожей с построением введенных в недавних работах \cite{sms12,sms13}
предельных и специальных рядов по ультрасферическим полиномам, обладающих свойством <<прилипания>> на концах отрезка ортогональности. Следуя \cite{sms13},
введем в рассмотрение функцию
\begin{equation}
\label{sms1Ffunk}
F(x) = \frac{f(x)-c(f,x)}{1-x^2} = \frac{g(f,x)}{1-x^2},
\end{equation}
\begin{equation*}
\label{sms1af}
\text{где}\quad c(f,x) = \frac{f(-1) + f(1)}{2} - \frac{f(-1) - f(1)}{2} x.
\end{equation*}
Cпециальным вейвлет-рядом Чебышева второго рода тогда назовем:
\begin{equation*}
\label{sms1fFunk}
f(x)=
c(f,x)+(1-x^2) \left[ \tilde{a}_{0}\hat{\phi}_{0,0}(x) + \tilde{a}_{1}\hat{\phi}_{0,1}(x) + \sum\limits_{j=0}^{\infty} \sum\limits_{k=0}^{2^j-1} \tilde{b}_{j,k}\hat{\psi}_{j,k}(x) \right],
\end{equation*}
\begin{equation*}
\text{где}\quad \tilde{a}_{0} = \int\limits_{-1}^{1} F(t)\hat{\phi}_{0,0}(t) w(t)dt, \quad
\tilde{a}_{1} = \int\limits_{-1}^{1} F(t)\hat{\phi}_{0,1}(t) w(t) dt,
\end{equation*}
\begin{equation*}
\label{sms1efcoeffB}
\tilde{b}_{j,k} = \int\limits_{-1}^{1} F(t)\hat{\psi}_{j,k}(t) w(t) dt, \quad (j=0,1, \ldots, m; k = 0, 1, \ldots, 2^j-1).
\end{equation*}
Обозначим его частичную сумму
\begin{equation}
\label{sms1fFunkSum}
\tilde{\mathcal{V}}_{2^m}(f,x)  = c(f,x)+(1-x^2) \left[ \tilde{a}_{0}\hat{\phi}_{0,0}(x) + \tilde{a}_{1}\hat{\phi}_{0,1}(x) + \sum\limits_{j=0}^{m-1} \sum\limits_{k=0}^{2^j-1} \tilde{b}_{j,k}\hat{\psi}_{j,k}(x) \right].
\end{equation}
В \cite{sms112} доказаны некоторые важные свойства $\tilde{\mathcal{V}}_{2^m}(f,x)$, из которых следует, что частичные суммы $\tilde{\mathcal{V}}_{2^m}(f,x)$ как аппарат приближения обладают весьма привлекательными аппроксимативными свойствами. В частности, имеют место следующие факты

\begin{theorem} \label{sms_th1}
  Частичная сумма $\tilde{\mathcal{V}}_{2^m}(f,x)$ на концах отрезка $[-1,1]$ совпадает с исходной функцией $f(x)$, т.е. $\tilde{\mathcal{V}}_{2^m}(f,\pm1) = f(\pm1)$.
\end{theorem}

\noindent Как отмечено в \cite{sms13},
это свойство имеет важное значение в задачах, связанных с обработкой временных рядов и изображений.

\begin{theorem} \label{sms_th2}
  Частичная сумма $\tilde{\mathcal{V}}_{2^m}(f,x)$ представляет собой линейный оператор, проектирующий пространство $L_{2, w}([-1; 1])$ на $H_{2^m+2, w}([-1; 1])$, т.е. для любого полинома $P_n(x)$ степени не выше $n\le 2^{m}+2$ справедливо равенство $\tilde{\mathcal{V}}_{2^m}(P_n,x) \equiv P_n(x)$.
\end{theorem}


\begin{theorem} \label{sms_th3}
  Для любой функции $f\in C[-1,1]$ и любого $x \in (-1, 1)$ имеет место оценка
\begin{equation*}
\label{sms1thrm3eq}
|f(x)-\tilde{\mathcal{V}}_m(f,x)|\le c E_{2^{m}+2}(f)(1+\ln(1+(2^{m}+2)\sqrt{1-x^2})),
\end{equation*}
где $c >0$ -- константа.
\end{theorem}




Разработан специализированный пакет прикладных программ для обработки и сжатия функций, заданных в узлах равномерной сетки
$\hat{\Omega}_N = \left\{ -1 + \frac{2 j}{N-1} \right\}_{j=0}^{N-1}$ на основе оператора \eqref{sms1fFunkSum}.
На вход программе подается дискретно заданная функция в одной из двух форм: в виде аналитического выражения с указанием частоты дискретизации $N$, либо в виде массива действительных чисел, представляющих собой значения функции в узлах $\hat{\Omega}_N$.
Пользователю предоставляется возможность выбора различных параметров и уровня сжатия.
Программа возвращает набор вычисленных коэффициентов, значения восстановленной обратным преобразованием приближенной функции на выбранной пользователем сетке (возможно более густой, чем исходная), а также несколько типов отклонений приближенной функции от исходной на новой сетке.
Имеется возможность визуализации при помощи построения графиков исходного ряда данных и результатов аппроксимации.

Кроме того, пользователь может выбрать один из альтернативных операторов для осуществления обработки исходных данных и сравнения результатов с преобразованием на основе оператора \eqref{sms1fFunkSum}.
С помощью разработанного программного пакета проведена серия численных экспериментов по исследованию поведения указанного оператора на дискретных функциях разных типов.
