\section{К гипотезе Грюнбаума о раскраске ребер графа}


\subsection{Терминология и обозначения}

В этой работе рассматриваются только \textit{простые} графы, т.е. графы без петель и параллельных ребер, и эти графы укладываются только на \textit{замкнутые} поверхности.

В основном мы следуем стандартной терминологии и обозначениям теории графов \cite{harary}. \textit{d}-\textit{ангуляция} \textit{P }поверхности определяется как \textit{d}-угольная укладка на эту поверхность 3-связного графа \textit{G} = \textit{G}(\textit{P}), в которой каждая грань ограничена простым циклом из $d\ge 3$ инцидентных ребер. С комбинаторной точки зрения \textit{P} определяется тройкой множеств \textit{V}(\textit{P}), \textit{E}(\textit{P}) и \textit{F}(\textit{P}) вершин, ребер и граней, соответственно. \textit{Двойственный граф} $G^*(P)$ определяется как граф, множество вершин которого соответствует \textit{F}(\textit{P}), причем две вершины в $G^*(P)$ смежны тогда и только тогда, когда соответствующие им грани смежны в \textit{P}.

\textit{Смежные} грани имеют общее инцидентное ребро. Мы рассматриваем только \linebreak\textit{d}-ангуляции, двойственные графы которых являются простыми. В частности, простота $G^*(P)$ гарантируется 3-связностью \textit{G} в следующих двух важных случаях:

\begin{enumerate}[1) ]
\item  \textit{ d }= 3 или

\item  поверхность-носитель -- сфера.
\end{enumerate}

Таким образом, $G^*(P)$ всегда 3-связен и \textit{d}-\textit{регулярен} (т.е. степень каждой вершины равна \textit{d}).

\textit{Вершинной} (соответственно \textit{реберной} или \textit{граневой}) \textit{k}-раскраской \linebreak \textit{d}-ангуляции \textit{P} называется сюръекция множества \textit{V}(\textit{P}) (соответственно \textit{E}(\textit{P}) или \textit{F}(\textit{P})) на множество цветов \{1, 2, \dots , \textit{k}\} такая, что образы любых двух смежных вершин (соответственно ребер или граней) различны. Для реберной 3-раскраски 3-регулярного графа используется термин \textit{раскраска по Тейту}. Минимальное число цветов, достаточное для раскраски вершин (ребер или граней) \textit{d}-ангуляции \textit{P}, называется ее \textit{вершинным }(\textit{реберным} или \textit{граневым})\textit{ хроматическим числом} и обозначается \textit{$\chi $}(\textit{P}) (соответственно \textit{$\chi $'}(\textit{P}) или \textit{$\chi $"}(\textit{P})). Для \textit{$\chi $}(\textit{P}) и \textit{$\chi $'}(\textit{P}) применяют также термины \textit{вершинное} и \textit{реберное хроматические числа графа G}(\textit{P}) и обозначения \textit{$\chi $}(\textit{G}(\textit{P})) и \textit{$\chi $'}(\textit{G}(\textit{P})).

Очевидно, граневая \textit{k}-раскраска произвольной \textit{d}-ангуляции \textit{P} соответствует вершинной \textit{k}-раскраске двойственного графа $G^*(P)$, и поэтому \textit{$\chi $"}(\textit{P}) = \textit{$\chi $}$(G^*(P))$. Далее, поскольку граф $G^*(P)$ является $d$-регулярным, то $\chi (G^*(P)) \in  \{2, 3, \dots , d, d+1\}$. Заметим также, что из теоремы Визинга \cite{lit04} следует, что $\chi '(G^*(P)) \in  \{3, \dots , d, d+1\}$.

\subsection{Гипотеза Грюнбаума}

Назовем \textit{раскраской по Грюнбауму} всякую раскраску ребер \textit{d}-ангуляции \textit{P} в \textit{d} цветов, такую, что для каждой грани \textit{f} все \textit{d} цветов присутствуют у инцидентных \textit{f} ребер. Если \textit{T} -- триангуляция поверхности, равенство \textit{$\chi $'}( $G^*(T)$ ) = 3 означает, что граф  $G^*(T)$  допускает раскраску по Тейту, а сама триангуляция -- раскраску по Грюнбауму.

\begin{hypothesis}
\label{akm2.hypo1}
(Грюнбаум \cite{grunb1}). Любая триангуляция \textit{T} ориентируемой поверхности раскрашивается по Грюнбауму, т.е. \textit{$\chi $'}( $G^*(T)$ ) = 3.
\end{hypothesis}

Гипотеза \ref{akm2.hypo1} находится \cite{archdea1} в списке нерешенных задач Дэна Арчдикона \cite{archdea2}. Заметим, что теорема о четырех красках эквивалентна утверждению, что каждый 3-регулярный планарный граф без мостов допускает раскраску по Тейту, откуда получается (см. \cite{harary}, \cite{bondymurty}, \cite{saatykai} и \cite{archdea1}), что из справедливости гипотезы 1 (в применении к случаю сферы) следует теорема о четырех красках.

Грюнбаум верил, что гипотеза \ref{akm2.hypo1} верна, а Арчдикон полагает, что она ложна (см. \cite{archdea1}). В 2009 г. Кохол \cite{kochol} построил контрпримеры к гипотезе \ref{akm2.hypo1} на каждой ориентируемой поверхности рода \textit{g} при $g \ge 5$. Мы выдвигаем новую гипотезу для триангуляций, усиливая неравенство \textit{$\chi $"}(\textit{T}) $\le 4$ (которое в действительности справедливо для любой \textit{T}) до ограничения \textit{$\chi $"}(\textit{T}) $\le 3$, но не ограничиваясь при этом классом ориентируемых поверхностей.

\begin{hypothesis}
\label{akm2.hypo2}
Если \textit{T} -- триангуляция поверхности (ориентируемой или неориентируемой) с $\chi ' (T) \le 3$, то $\chi '( G^*(T) ) = 3$.
\end{hypothesis}



Арчдикон также поставил \cite{archdea1} задачу проверки гипотезы \ref{akm2.hypo1} для более узких классов триангуляций, в частности -- попробовать доказать ее для всех триангуляций ориентируемых поверхностей с полным $n$-графом \textit{Kn} при каждом $n$, для которого такая триангуляция существует. Доказано (теорема \ref{akm2.th2}), что для раскрашиваемости \textit{d}-ангуляции \textit{P} по Грюнбауму (т.е. для выполнения равенства \textit{$\chi $'}( $G^*(P)$ ) = \textit{d}) достаточно, чтобы \textit{P} обладала свойством граневой 2-раскрашиваемости (скажем, в черный и белый цвета), причем без требования ориентируемости поверхности. Оказалось, что некоторые известные (и достаточно широкие) классы триангуляций гранево $2$-раскрашиваемы. В частности, гипотеза \ref{akm2.hypo1} доказана при каждом $n\ge2$ для всех триангуляций соответствующей ориентируемой поверхности с полным трехдольным графом $K_{n,n,n}$.

\subsection{Ключевая теорема}

Пусть \textit{P} -- \textit{d}-ангуляция ориентируемой или неориентируемой поверхности (c простым двойственным графом). Поскольку двойственный граф  $G^*(P)$  является $d$-регулярным, утверждение следующей леммы очевидно.

\begin{lemma}
\label{akm2.lemm1}
Для достижения равенства $\chi '(G^*(P)) = d$ необходимо и достаточно, чтобы граф $G^*(P)$ был $1$-факторизуемым, т.е. был суммой $d$ $1$-факторов.
\end{lemma}


Очевидно, что граф вершинно $2$-раскрашиваем тогда и только тогда, когда этот граф двудольный. В классических статьях Кёнига за 1916 г. доказано (см. \cite{lit05}, \cite{konigD}), что каждый двудольный $d$-регулярный граф раскладывается в сумму $d$ $1$-факторов. Из этого результата непосредственно вытекает следующая теорема.

\begin{theorem}
\label{akm2.th1}
(Кёниг) Если $\chi(G^*(P)) = 2$, то граф $G^*(P)$ $1$-факторизуем.
\end{theorem}

Комбинацией леммы \ref{akm2.lemm1} и теоремы \ref{akm2.th1} получается ключевая теорема, которая утверждает, что каждая гранево $2$-раскрашиваемая $d$-ангуляция ориентируемой или неориентируемой поверхности раскрашиваема по Грюнбауму.

\begin{theorem}
\label{akm2.th2}
(ключевая теорема) Если $\chi (G^*(P)) = 2$ или, другими словами, $\chi ' (P) = 2$, то $\chi '(G^*(P)) = d$ или, другими словами, $P$ раскрашиваема по Грюнбауму.
\end{theorem}

Из теоремы \ref{akm2.th2} получается как частный случай (при $d = 3$), что гипотеза \ref{akm2.hypo1} верна для всех гранево $2$-раскрашиваемых триангуляций ориентируемых и неориентируемых поверхностей.

Заметим, что ложность гипотезы \ref{akm2.hypo1} в неориентируемом случае очевидна. Самый известный контрпример \cite{archdea1} -- минимальная триангуляция $T_{min}$ проективной плоскости с полным $6$-графом $G = K_6$. Здесь $G^*(T_{min})$ оказывается известным графом Петерсена \cite{petersen}, который не раскладывается в сумму трех $1$-факторов \cite{harary, lit05, petersen} и по лемме \eqref{akm2.lemm1} имеет реберное хроматическое число, равное $4$.

Поскольку $\chi '(T) =  \chi (G^*(T)) \in  \{2, 3, 4\}$ для любой триангуляции $T$, гипотеза \ref{akm2.hypo2} является минимально возможным расширением теоремы \ref{akm2.th2} и представляется нам правдоподобной.

\subsection{Триангуляции с полными графами}

В этом разделе устанавливается существование раскрашиваемых по Грюнбауму триангуляций ориентируемых поверхностей с полными графами $K_n$ по меньшей мере для половины классов вычетов в спектре возможных значений $n$.

Хорошо известно \cite{ringel}, что $K_n$ триангулирует ориентируемую поверхность тогда и только тогда, когда $n\ \equiv\ 0,\ 3,\ 4 \text{ или } 7\ (\text{mod }12)$. Грэннелл, Григгс и Ширан \cite{granel1} заметили, что при $n\ \equiv\ 0 \text{ или }4\ (\text{mod }12)$ такие триангуляции заведомо не являются гранево 2-раскрашиваемыми, потому что для граневой 2-раскрашиваемости необходимо, чтобы степень каждой вершины была четной, т.е. \textit{n} должно быть нечетным. Далее, они установили, что построенные Рингелем \cite{ringel} для всех $n\ \equiv\ 3 (\text{mod }12)$ ориентируемые триангуляции оказались гранево 2-раскрашиваемыми, и что среди ориентируемых триангуляций, построенных Янгсом \cite{yungjwt}, можно найти гранево 2-раскрашиваемую для каждого $n\ \equiv\ 7 (\text{mod }12)$. Подведем итог сказанному в виде следующей теоремы.

\begin{theorem}
\label{akm2.th3}
(Рингель \cite{ringel}; Янгс \cite{yungjwt}; Грэннелл, Григгс, Ширан \cite{granel1}) Гранево $2$-раскраши- \linebreak ваемая триангуляция ориентируемой поверхности с полным графом $K_n$ существует тогда и только тогда, когда $n \equiv 3$ или $7 (\mod 12)$.
\end{theorem}


Если одна триангуляция гранево $2$-раскрашиваема, а другая -- нет, эти триангуляции заведомо \textit{неизоморфны}, т.е. между множествами их вершин невозможно установить биекцию, продолжающуюся до гомеоморфизма их поверхностей-носителей.

Исторически первые примеры пар неизоморфных ориентируемых триангуляций с одним и тем же полным графом были построены \cite{yungjwt} в 1970~г. В тех примерах неизоморфность следовала из того факта, что одна из триангуляций была гранево $2$-раскрашиваема, а другая -- нет (см. обзорную работу \cite{granel2}). Через 24 года первый пример \textit{более двух} неизоморфных ориентируемых триангуляций с одним и тем же полным графом был построен в \cite{lawrnegam}, а именно были построены \textit{три} такие триангуляции, только одна из которых гранево $2$-раскрашиваема. В 2000 г. было показано \cite{granel3} (см. также \cite{granel2}), что число неизоморфных ориентируемых триангуляций с графом $K_n$ растет очень быстро (при $n\to\infty$) даже внутри только класса гранево 2-раскрашиваемых триангуляций; например, при $n\ \equiv\ 7\text{ или }19\ (\text{mod }36)$ это число не менее $2^{s(n)}$, где $s(n) = n^2/54 - o(n^2)$.

Из комбинации теорем \ref{akm2.th3} и \ref{akm2.th2} вытекает следующее следствие.


\begin{corollary}
\label{akm2.cor1}
При каждом $n \equiv 3$  или $7 (\mod 12)$  существуют раскрашиваемые по Грюнбауму триангуляции ориентируемой поверхности с графом $K_n$.
\end{corollary}

Известно \cite{ringel}, что $K_n$ триангулирует неориентируемую поверхность тогда и только тогда, когда $n\ \equiv\ 0\text{ или }1\ (\text{mod }3)$, $n\ge6\text{ и }n\neq7$. Нечетность $n$ получается только при $n \equiv 1$  или $3 (\mod 6)$, $n \ge 9$, и для всех этих значений $n$ гранево $2$-раскрашиваемые триангуляции с графом $K_n$ соответствующей неориентируемой поверхности построены в \cite{ringel} и \cite{korzhikvp}.

\begin{theorem}
\label{akm2.th4}
(Рингель \cite{ringel}; Грэннелл, Коржик \cite{korzhikvp}) Для существования гранево \linebreak $2$-раскрашиваемой триангуляции неориентируемой поверхности с графом $K_n$ необходимо и достаточно, чтобы $n\equiv 1$ или $3 (\mod 6)$, $n\ge9$.
\end{theorem}


\begin{corollary}
\label{akm2.cor2}
При каждом $n\equiv 1$  или $3  (\mod 6)$, $n\ge9$, существуют раскрашиваемые по Грюнбауму триангуляции неориентируемой поверхности с графом $K_n$.
\end{corollary}

Теоремы \ref{akm2.th3} и \ref{akm2.th4} гарантируют, что мы не пропустили никаких гранево  $2$-раскра-\linebreak шиваемых триангуляций, применяя теорему \ref{akm2.th2} для получения следствий \ref{akm2.cor1} и \ref{akm2.cor2} (соответственно).



\subsection{Триангуляции с полными трехдольными графами}

Заметим, во-первых, что существование для каждого \textit{n} ориентируемой треугольной укладки полного трехдольного графа $K_{n,n,n}$ было установлено Рингелем и Янгсом \cite{ringyung}. Во-вторых, граневая $2$-раскрашиваемость каждой такой триангуляции установлена Грэннеллом, Григгсом и Кнором \cite{granel4} (см. также \cite{granel2}). Из комбинации этих двух результатов с теоремой \ref{akm2.th2} непосредственно вытекает следующая теорема.

\begin{theorem}
\label{akm2.th5}
Гипотеза \ref{akm2.hypo1} (Грюнбаума) справедлива при каждом  $n \ge 2$  для всех триангуляций соответствующей ориентируемой поверхности с полным трехдольным графом $K_{n,n,n}$.
\end{theorem}

На первый взгляд утверждение теоремы \ref{akm2.th5} может показаться малопредметным, однако в \cite{granel5} (см. также \cite{granel2}) показано, что когда параметр $n$ -- простое число, существует не менее $(n-2)!/(6n)$ неизоморфных ориентируемых триангуляций с графом $K_{n,n,n}$. Более того, в \cite{knorm} найдены улучшенные оценки числа таких триангуляций; например, при $n \equiv 6\text{ или }30 (\text{mod }36)$ существует не менее $n^{t(n)}$ неизоморфных ориентируемых триангуляций с графом $K_{n,n,n}$, где $n^{t(n)}=n^2/144-o(n^2)$.
