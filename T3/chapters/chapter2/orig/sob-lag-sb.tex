\section{Некоторые специальные ряды по общим полиномам Лагерра и ряды Фурье по полиномам Лагерра, ортогональным по Соболеву }
%\begin{abstract}
%Рассмотрены ряды Фурье по полиномам Лагерра $L_k^{-r}(x)$ $(k=r,r+1,\ldots)$, ортогональным относительно скалярного произведения типа Соболева следующего вида
%\begin{equation*}
%<f,g>=\sum_{\nu=0}^{r-1}f^{(\nu)}(0)g^{(\nu)}(0)+\int_0^\infty f^{(r)}(t)g^{(r)}(t)e^{-t}dt.
%\end{equation*}
%Показано, что такие ряды представляют собой частный случай смешанных рядов по полиномам Лагерра $L_k^{\alpha}(x)$ $(k=0,1,\ldots)$, рассмотренных автором ранее. Введены некоторые новые специальные ряды по полиномам Лагерра, которые в частном случае также совпадают с рядом
%Фурье по полиномам Лагерра $L_k^{-r}(x)$ $(k=r,r+1,\ldots)$. Исследованы вопросы сходимости смешанных  рядов и аппроксимативные свойства специальных рядов  по полиномам Лагерра.
%\end{abstract}
\subsection{Введение}

В последние годы интенсивное развитие получила (см.\cite{sob-lag-sb-KwonLittl1}--\cite{sob-lag-sb-MarcelXu} и цитированную там литературу) теория полиномов, ортогональных относительно различных скалярных произведений соболевского типа (полиномы, ортогональные по Соболеву). Скалярные произведения соболевского типа характеризуются тем, что они включают в себя слагаемые, которые <<контролируют>> поведение соответствующих ортогональных полиномов на границе области ортогональности. Например, в некоторых случаях оказывается так, что полиномы, ортогональные по Соболеву на отрезке $[a,b]$, могут иметь нули, совпадающие с одним или с обоими концами этого отрезка. Это обстоятельство имеет важное значение для некоторых приложений, в которых требуется, чтобы значения  частичных сумм ряда Фурье функции $f(x)$ по рассматриваемой системе ортогональных полиномов совпали в концах отрезка $[a,b]$ со значениями $f(a)$ и $f(b)$. Заметим, что обычные ортогональные с положительным на  $[a,b]$ весом полиномы этим важным свойством не обладают.

С другой стороны отметим, что в ряде работ автора \cite{sob-lag-sb-Shar11} -- \cite{sob-lag-sb-Shar16}  были введены, так называемые смешанные ряды по классическим ортогональным полиномам, частичные суммы которых также обладают свойством совпадения их значений в концах области ортогональности  со  значениями исходной функции.  В работах \cite{sob-lag-sb-Shar11} -- \cite{sob-lag-sb-Shar16} были подробно исследованы аппроксимативные свойства смешанных рядов для функций из различных функциональных пространств и классов. В частности, было показано, что частичные суммы смешанных рядов по классическим ортогональным полиномами, в отличие от сумм Фурье по этим же полиномам, успешно могут быть использованы в задачах, в которых требуется одновременно приближать дифференцируемую функцию и ее несколько производных. Кроме того отметим, что в тех случаях, когда для целого $r\ge1$ классические полиномы Якоби $P_n^{\alpha-r,\beta-r}(x)$  и Лагерра $L_n^{\alpha-r}(x)$ образуют ортогональные системы в смысле Соболева, то ряды Фурье по этим системам являются частным случаем смешанных рядов по соответствующим полиномам Якоби и Лагерра.

В настоящей статье эти вопросы рассмотрены для классических полиномов Лагерра $L_n^\alpha(x)$. В частности, показано, что     ряд Фурье по ортогональным полиномам Лагерра-Соболева   являются частным случаем смешанных рядов по полиномам Лагерра  (см.\S3 ). Это, в свою очередь,  позволяет (см.\S5) применить к исследованию аппроксимативных свойств рядов Фурье по полиномам Лагерра-Соболева методы и подходы, разработанные нами ранее \cite{sob-lag-sb-Shar11} -- \cite{sob-lag-sb-Shar16} для решения аналогичной задачи для смешанных рядов по классическим ортогональным полиномам. В \S4 введены некоторый новый специальный ряд по классическим ортогональным полиномам Лагерра $L_n^\alpha(x)$ с $\alpha>-1$, который в случае  натурально $\alpha=r$ совпадает с соответствующим смешанным рядом по полиномам Лагерра $L_n^0(x)$, а также с рядом Фурье по полиномам Лагерра $L_n^{-r}(x)$, ортогональным относительно скалярного произведения типа Соболева
\begin{equation}\label{sob-lag-sb-1.1}
<f,g>=\sum_{\nu=0}^{r-1}f^{(\nu)}(0)g^{(\nu)}(0)+\int_0^\infty f^{(r)}(t)g^{(r)}(t)e^{-t}dt.
\end{equation}


Таким образом, в случае натурально $\alpha=r$ для полиномов Лагерра $L_n^\alpha(x)$ все три понятия -- смешанный ряд по полиномам Лагерра, специальный ряд по полиномам Лагерра и ряд Фурье  по полиномам Лагерра $L_n^{-r}(x)$, ортогональным относительно скалярного произведения типа Соболева \eqref{sob-lag-sb-1.1}, совпадают.

Изложение материала настоящей статьи разбито на параграфы. В \S 2 собраны некоторые необходимые сведения о полиномах Лагерра, в \S 3
приводится определение смешанных рядов по полиномам Лагерра, в \S4 введены новые специальные ряды по полиномам Лагерра $L_n^\alpha(x)$, ортогональным в классическом смысле, в \S 5 вводятся ряды Фурье по  полиномам Лагерра $L_n^{-r}(x)$, ортогональным относительно скалярного произведения типа Соболева \eqref{sob-lag-sb-1.1}, в \S 6 выводится неравенство типа Лебега для частичных сумм специального ряда  по полиномам Лагерра и сформулирована теорема, содержащая поточечную оценку соответствующей  функции Лебега, в \S7 изложено доказательство этой теоремы.



\subsection{Некоторые сведения о полиномах Лагерра}
При исследовании аппроксимативных свойств частичных сумм новых   специальных рядов,  введенных в настоящей работе,  нам понадобится ряд свойств полиномов Лагерра $L_n^\alpha(t)$, которые мы соберем в данном параграфе.

Пусть $\alpha$ -- произвольное действительное число. Тогда для полиномов Лагерра  имеют место \cite{sob-lag-sb-Sege}:

\textit{Формула Родрига}
\begin{equation}\label{sob-lag-sb-2.1}
L_n^{\alpha}(t) = \frac{1}{n!}t^{-\alpha}e^{t} \left\{ t^{n+\alpha} e^{-t} \right\}^{(n)};
\end{equation}

\textit{Явный вид}
\begin{equation}\label{sob-lag-sb-2.2}
L_n^\alpha(t) =
\sum\limits_{\nu=0}^{n}
\binom{n+\alpha}{n-\nu}
\frac{(-x)^\nu}{\nu!};
\end{equation}

\textit{Соотношение ортогональности}

\begin{equation}
\label{sob-lag-sb-2.3}
\int_0^{\infty} t^{\alpha} e^{-t} L^{\alpha}_{n}(t) L^{\alpha}_{m}(t) dt = \delta_{nm} h^{\alpha}_n \quad (\alpha > -1),
\end{equation}
где $\delta_{nm}$ --- символ Кронекера,
\begin{equation}\label{sob-lag-sb-2.4}
h^{\alpha}_n = \left( n+\alpha \atop n \right) \Gamma(\alpha +1);
\end{equation}
В частности, для $L_{n}(t) = L^{0}_{n}(t)$ имеет место равенство
\begin{equation*}
\int_0^{\infty} e^{-t} L_{n}(t) L_{m}(t) dt = \delta_{nm};
\end{equation*}

\textit{Формула Кристоффеля -- Дарбу}
\begin{equation}\label{sob-lag-sb-2.5}
\mathcal{K}_n^\alpha(t,\tau)=
\sum\limits_{k=0}^{n}\frac{L_\nu^\alpha(t)L_\nu^\alpha(\tau)}{h_\nu^\alpha}=
\frac{n+1}{h_n^\alpha}
\frac{L_n^\alpha(t)L_{n+1}^\alpha(\tau) - L_n^\alpha(\tau)L_{n+1}^\alpha(t)}{t-\tau};
\end{equation}

\textit{Свертка}
\begin{equation}
\label{sob-lag-sb-2.6}
\int_0^{t} L_{n}(t-\tau) L_{m}(\tau) d\tau = L_{n+m}(t) - L_{n+m+1}(t).
\end{equation}

Далее отметим следующие равенства
\begin{equation}\label{sob-lag-sb-2.7}
\frac{d}{dt} L_n^{\alpha}(t) = -L_{n-1}^{\alpha+1}(t),
\end{equation}

\begin{equation} \label{sob-lag-sb-2.8}
\frac{d^r}{dt^r} L_{k+r}^{\alpha-r}(t) = (-1)^{r} L_{k}^{\alpha}(t),
\end{equation}
\begin{equation}\label{sob-lag-sb-2.9}
L_{k}^{-r}(t) = \frac{(-t)^{r}}{k^{[r]}} L_{k-r}^{r}(t),
\end{equation}
где $k^{[r]} = k(k-1)\ldots(k-r+1)$,
\begin{equation}\label{sob-lag-sb-2.10}
L_n^{\alpha+1}(t)-L_{n-1}^{\alpha+1}(t)=L_n^\alpha(t),
     \end{equation}
 \begin{equation}\label{sob-lag-sb-2.11}
(n+\alpha)L_n^{\alpha-1}(t)=\alpha L_n^\alpha(t)-
xL_{n-1}^{\alpha+1}(t),
\end{equation}

а также весовую оценку \cite{sob-lag-sb-AskeyWaiger}
\begin{equation}\label{sob-lag-sb-2.12}
e^{-\frac{t}{2}}|L_n^\alpha(t)| \le c(\alpha) B_n^\alpha(t), \quad \alpha>-1,
\end{equation}
где здесь и далее $c,c(\alpha),c(\alpha,\ldots,\beta)$ -- положительные числа, зависящие лишь от указанных параметров,
\begin{equation*}
B_n^\alpha(t)=
\begin{cases}
\theta^\alpha, &0 \le t \le \frac{1}{\theta},\\
\theta^{\frac{\alpha}{2} - \frac{1}{4}}\,t^{-\frac{\alpha}{2} - \frac{1}{4}}, & \frac{1}{\theta} < t \le \frac{\theta}{2},\\
\Bigl[
\theta(\theta^{\frac{1}{3}}+|t-\theta|)
\Bigr]^{-\frac{1}{4}}, & \frac{\theta}{2} < t \le \frac{3\theta}{2},\\
e^{-\frac{t}{4}}, &\frac{3\theta}{2}< t,
\end{cases}
\end{equation*}
где $\theta=\theta_n=\theta_n(\alpha)=4n+2\alpha+2$.

Для нормированных полиномов Лагерра
\begin{equation}\label{sob-lag-sb-2.13}
\hat{L}_n^\alpha(t)=
\Bigl\{h_n^\alpha \Bigr\}^{-\frac{1}{2}} L_n^\alpha(t)
\end{equation}
имеет место оценка \cite{sob-lag-sb-AskeyWaiger}
\begin{equation}\label{sob-lag-sb-2.14}
e^{-\frac{t}{2}}
\Bigl|
\hat{L}_{n+1}^\alpha(t)-
\hat{L}_{n-1}^\alpha(t)
\Bigr|\le
\begin{cases}
\theta^{\frac{\alpha}{2}-1}, &0 \le t \le \frac{1}{\theta},\\
\theta^{-\frac{3}{4}}\,t^{-\frac{\alpha}{2} + \frac{1}{4}}, & \frac{1}{\theta} < t \le \frac{\theta}{2},\\
t^{-\frac{\alpha}{2}}\,
\theta^{-\frac{3}{4}}
\Bigl[
\theta^{\frac{1}{3}}+|t-\theta|
\Bigr]^{\frac{1}{4}}, & \frac{\theta}{2} < t \le \frac{3\theta}{2},\\
e^{-\frac{t}{4}}, &\frac{3\theta}{2}< t.
\end{cases}
\end{equation}
Поскольку $h_n^\alpha=\frac{\Gamma(n+\alpha+1)}{n!} \asymp n^\alpha$, то из \eqref{sob-lag-sb-2.12} и \eqref{sob-lag-sb-2.13} следует, что
\begin{equation}\label{sob-lag-sb-2.15}
e^{-\frac{t}{2}}
|\hat{L}_n^\alpha(t)|\le
c(\alpha)\theta_n^{-\frac{\alpha}{2}}B_n^\alpha(t), \quad t \ge 0.
\end{equation}

\subsection{Смешанные ряды по полиномам Лагерра}
Смешанные ряды по классическим ортогональным полиномам были впервые введены в работах автора \cite{sob-lag-sb-Shar11}--\cite{sob-lag-sb-Shar16}, как альтернативный
рядам Фурье по тем же полиномам аппарат для одновременного приближения функций и их производных. Не является исключением и смешанные ряды по полиномам Лагерра \cite{sob-lag-sb-Shar13}. В настоящем параграфе мы напомним определение этих рядов, следуя работе \cite{sob-lag-sb-Shar13}. Пусть $-1<\alpha<1$,      $L_n^\alpha(x)$ соответствующие классические ортогональные полинома Лагерра, $\rho=\rho(x)=x^\alpha e^{-x}$, $1\le p<\infty $,  $\mathcal{L}_{p,\rho}$ -- пространство измеримых функций,
определенных на полуоси $[0,\infty)$ и таких, что
     $
\|f\|_{\mathcal{L}_{p,\rho}}=
\left(\int_0^\infty|f(x)|^p\rho(x)dx\right)^{1/p}<\infty.
   $
 Через $W_{\mathcal{L}_{p,\rho}}^r(0,\infty)$ обозначим  подкласс функций $f=f(x)$ из $\mathcal{L}_{p,\rho}$,
непрерывно дифференцируемых $r-1$ раз, для которых $f^{(r-1)}(x)$
абсолютно непрерывна на произвольном сегменте $[a,b]\subset[0,\infty)$,
а $f^{(r)}\in \mathcal{L}_{p,\rho}$. Тогда мы можем рассмотреть
коэффициенты Фурье-Лагерра функции $f^{(r)}(x)$ по полиномам Лагерра
     $L_n^\alpha(x)$:
     \begin{equation}\label{sob-lag-sb-3.1}
f_{r,k}^\alpha=
{1\over h_k^\alpha}\int\limits_0^\infty\rho(t)f^{(r)}(t)L_k^\alpha(t)dt.
\end{equation}

Соответствующий ряд Фурье-Лагерра функции $f^{(r)}$ имеет вид
     \begin{equation}\label{sob-lag-sb-3.2}
f^{(r)}\sim\sum_{k=0}^\infty f_{r,k}^\alpha
L_k^\alpha(x).
     \end{equation}
Рассмотрим формулу Тейлора
\begin{equation}\label{sob-lag-sb-3.3}
f(x)=Q_{r-1}(f,x)+{1\over (r-1)!}\int\limits_{0}^x(x-t)^{r-1}
     f^{(r)}(t)dt,
     \end{equation}
     где $Q_{r-1}(f,x) =\sum_{\nu=0}^{r-1}{f^{(\nu)}(0)\over\nu!}x^\nu$ --
полином Тейлора и выполним формальную подстановку в \eqref{sob-lag-sb-3.3} вместо
$f^{(r)}(t)$ ряда Фурье-Лагерра \eqref{sob-lag-sb-3.2}. Если эта операция
законна, то мы придем к следующему равенству
  \begin{equation}\label{sob-lag-sb-3.4}
f(x)=Q_{r-1}(f,x)+
\sum_{k=0}^\infty f_{r,k}^\alpha L_{r,k+r}^\alpha(x),
\end{equation}
где
\begin{equation}\label{sob-lag-sb-3.5}
L_{r,k+r}^\alpha(x)={1\over (r-1)!}\int_{0}^x(x-t)^{r-1}L^\alpha_k(t)dt
\end{equation}
-- полином степни $k+r$ $( k=0,1,\ldots)$. Воспользовашись  равенствами  \eqref{sob-lag-sb-2.2} и  \eqref{sob-lag-sb-2.8}, нетрудно показать, что
 \begin{equation}\label{sob-lag-sb-3.6}
L_{r,k+r}^\alpha(x)=(-1)^r(L_{k+r}^{\alpha-r}(x)-\Lambda_{r-1}^{\alpha,k}(x)),
\end{equation}
где
$
\Lambda_{r-1}^{\alpha,k}(x)=     \sum_{\nu=0}^{r-1}
{\Gamma(k+\alpha+1)(-x)^\nu\over\Gamma(\nu-r+\alpha+1)(k+r-\nu)!\nu!}
$
-- полином степени $r-1$. Равенство \eqref{sob-lag-sb-3.4} с учетом \eqref{sob-lag-sb-3.6} перепишем еще так
 \begin{equation}\label{sob-lag-sb-3.7}
  f(x)=Q_{r-1}(f,x)+(-1)^r\sum_{k=0}^\infty f_{r,k}^\alpha
[L_{k+r}^{\alpha-r}(x)-\Lambda_{r-1}^{\alpha,k}(x)].
 \end{equation}
  Ряд  \eqref{sob-lag-sb-3.4} (или, что то же, ряд \eqref{sob-lag-sb-3.7})  будем называть \cite{sob-lag-sb-Shar13} \textit{смешанным} рядом по полиномам Лагерра
$L_k^\alpha(x)$. Смешанный ряд содержит
     коэффициенты $f_{r,k}^\alpha$ $(k=0,1,\ldots)$ $r$-той производной
     функции $f(x)$ по полиномам Лагерра $L_k^\alpha(x)$, умноженные
     на полиномы  $L_{r,k+r}^{\alpha}(x)$ вида \eqref{sob-lag-sb-3.6}. В этом
     заключается принципиальное отличие смешанного ряда \eqref{sob-lag-sb-3.4}
     по полиномам Лагерра $L_k^\alpha(x)$ от ряда Фурье    по этим же полиномам.
Если  мы положим $\alpha=0$, то $\Lambda_{r-1}^{\alpha,k}(x)=0$ и равенство \eqref{sob-lag-sb-3.7} принимает следующий вид
\begin{equation}\label{sob-lag-sb-3.8}
  f(x)=\sum_{\nu=0}^{r-1}  {f^{(\nu)}(0)\over \nu!}x^\nu +(-1)^r\sum_{k=0}^\infty f_{r,k}^0
L_{k+r}^{-r}(x).
 \end{equation}
Если, кроме того, воспользуемся равенством \eqref{sob-lag-sb-2.9}, то отсюда получим
\begin{equation}\label{sob-lag-sb-3.9}
  f(x)=\sum_{\nu=0}^{r-1}  {f^{(\nu)}(0)\over \nu!}x^\nu +x^r\sum_{k=0}^\infty f_{r,k}^0
{L_{k}^{r}(x)\over (k+r)^{[r]}}.
 \end{equation}
В дальнейшем будет показано (см. \S5 и \S 4 ), что  \eqref{sob-lag-sb-3.8} представляет собой  ряд Фурье по полиномам Лагерра $L_{n}^{-r}(x)$, ортогональным относительно скалярного произведения \eqref{sob-lag-sb-1.1}, а \eqref{sob-lag-sb-3.9} есть не что иное, как  специальный ряд по полиномам Лагерра
$L_{k}^{\alpha}(x)$ с $\alpha=r$, где $1\le r$ -- целое.

     Перейдем к рассмотрению достаточных условий на функцию $f(x)$,
     обеспечивающих сходимость смешанных рядов  \eqref{sob-lag-sb-3.4}.
 \begin{theorem}
      Пусть $-1<\alpha<1$, $r\ge1$,
     $A>0$, $f\in W^r_{\mathcal{L}_{2,\rho}}$. Тогда смешанный ряд
    \eqref{sob-lag-sb-3.4}  сходится равномерно относительно $x\in[0,A]$.
\end{theorem}










\subsection{Специальные ряды по полиномам Лагерра}

Пусть $1\le r$ -- целое, $f(t)$ -- $r-1$ раз дифференцируемая в точке $t=0$,
\begin{equation}\label{sob-lag-sb-4.1}
  P_{r-1}(f)=P_{r-1}(f)(t)=\sum\limits_{i=0}^{r-1}\frac{f^{(i)}(0)}{i!}t^i,
\end{equation}
\begin{equation}\label{sob-lag-sb-4.2}
  f_r(t)=\frac1{t^r}[f(t)-P_{r-1}(f)(t)].
\end{equation}

Предположим, что для функции $f_r(t)$, определенной равенством \eqref{sob-lag-sb-4.2}, существуют коэффициенты Фурье-Лагерра
\begin{equation}\label{sob-lag-sb-4.3}
\hat{f}_{r,k}^\alpha=\frac1{h_k^\alpha}\int\limits_0^\infty f_r(\tau)t^\alpha e^{-t}L_k^\alpha(t)dt=
  \frac1{h_k^\alpha}\int\limits_0^\infty [f(t)-P_{r-1}(f)(t)]t^{\alpha-r}e^{-t}L_k^\alpha(t)dt,
\end{equation}
где $h_n^\alpha=\Gamma(n+\alpha+1)/n!$.
Тогда мы можем рассмотреть ряд Фурье-Лагерра функции $f_r(t)$
\begin{equation}\label{sob-lag-sb-4.4}
  f_r(t)\sim\sum\limits_{k=0}^\infty\hat{f}_{r,k}^\alpha L_k^\alpha(t).
\end{equation}
Если ряд \eqref{sob-lag-sb-4.4} сходится к $f_r(t)$, то с учетом \eqref{sob-lag-sb-4.2} мы можем записать
\begin{equation}\label{sob-lag-sb-4.5}
  f(t)=P_{r-1}(f)(t)+t^r\sum\limits_{k=0}^\infty\hat{f}_{r,k}^\alpha L_k^\alpha(t).
\end{equation}
 Это и есть \textit{специальный ряд по полиномам Лагерра}.
 Нетрудно показать, что если $\alpha=r$, то ряд \eqref{sob-lag-sb-4.5} совпадает с рядом \eqref{sob-lag-sb-3.9}. В самом деле, в силу \eqref{sob-lag-sb-2.1},  \eqref{sob-lag-sb-3.1} и \eqref{sob-lag-sb-2.9} имеем
\begin{equation*}
  {f}_{r,k}^0=\int\limits_0^\infty f^{(r)}(\tau)e^{-\tau}L_k(\tau)d\tau=\frac1{k!}\int\limits_0^\infty(f(\tau)-P_{r-1}(f)(\tau))^{(r)}(e^{-\tau}\tau^k)^{(k)}d\tau=
\end{equation*}
\begin{equation*}
  \frac{(-1)^r}{k!}\int\limits_0^\infty(f(\tau)-P_{r-1}(f)(\tau))(e^{-\tau}\tau^k)^{(k+r)}d\tau=
\end{equation*}
\begin{equation*}
  \frac{(-1)^r}{k!}\int\limits_0^\infty(f(\tau)-P_{r-1}(f)(\tau))
  \tau^{-r}e^{-\tau}L_{k+r}^{-r}(\tau)(k+r)!d\tau=
\end{equation*}
\begin{equation*}
  \frac{(k+r)!}{k!}(-1)^r\int\limits_0^\infty\frac{f(\tau)-P_{r-1}(f)(\tau)}
  {\tau^r}e^{-\tau}\frac{(-\tau)^r}{(k+r)^{[r]}}L_k^r(\tau)d\tau=
\end{equation*}
\begin{equation}\label{sob-lag-sb-4.6}
  \int\limits_0^\infty f_r(\tau)\tau^re^{-\tau}L_k^r(\tau)d\tau=h_k^r\hat{f}_{r,k}^r.
\end{equation}
В силу \eqref{sob-lag-sb-4.6} ряд \eqref{sob-lag-sb-3.9} приобретает вид
\begin{equation*}
  f(t)=P_{r-1}(f)(t)+t^r\sum\limits_{k=0}^\infty\frac{h_k^r\hat{f}_{r,k}^rL_k^r(t)}{(k+1)_r}=
  P_{r-1}(f)(t)+t^r\sum\limits_{k=0}^\infty\hat{f}_{r,k}^rL_k^r(t),
\end{equation*}
так как $h_k^r=(k+1)_r$. Таким образом, в случае $\alpha=r$ ряды \eqref{sob-lag-sb-3.9} и \eqref{sob-lag-sb-4.5} совпадают.

\subsection{Ряды Фурье по полиномам Лагерра $L_k^{-r}(t)$, ортогональным относительно скалярного произведения типа Соболева}

Для $\rho=\rho(x)=e^{-t}$ положим $W^r[0,\infty)= W^r_{\mathcal{L}_{2,\rho}}$, другими словами, $W^r[0,\infty)$ представляет собой  пространство функций $f=f(t)$, заданных и непрерывно дифференцируемых  на полуоси $[0,\infty)$
$(r-1)$-раз, причем $f^{(r-1)}(t)$ абсолютно непрерывна на произвольном сегменте $[a,b]\subset [0,\infty)$ и
\begin{equation}\label{sob-lag-sb-5.1}
  \int_0^\infty |f^{(r)}(t)|^2e^{-t}dt <\infty.
  \end{equation}

Рассмотрим систему функций $\{\varphi_k(t)\}_{k=0}^\infty$,  в которой
\begin{equation}\label{sob-lag-sb-5.2}
  \varphi_k(t)=\frac{t^k}{k!}, \quad 0\le k\le r-1,
  \end{equation}
\begin{equation}\label{sob-lag-sb-5.3}
  \varphi_k(t)=L_k^{-r}(t), \quad r\le k .
  \end{equation}

\begin{theorem}
Функции $\varphi_k(t)\, (k=0,2,\ldots) $, определенные равенствами \eqref{sob-lag-sb-5.2} и \eqref{sob-lag-sb-5.3}, образуют  полную  в $W^r[0,\infty)$ ортонормированную  систему относительно скалярного произведения \eqref{sob-lag-sb-1.1}.
\end{theorem}
\textit{Замечание 1. Соотношение ортогональности \eqref{sob-lag-sb-5.4} ранее было установлено  в работе \cite{sob-lag-sb-KwonLittl1}}.

\textit{Замечание 2. Почти дословно повторяя рассуждения, проведенные при доказательстве теоремы 2, можно показать, что полиномы $L_{r,k+r}^\alpha(x)\,\, (k=0,1,\ldots)$, определенные равенством \eqref{sob-lag-sb-3.5}, образуют ортогональную систему относительно следующего скалярного произведения типа Соболева
\begin{equation*}
<f,g>=\sum_{\nu=0}^{r-1}f^{(\nu)}(0)g^{(\nu)}(0)+\int_0^\infty f^{(r)}(t)g^{(r)}(t)e^{-t}t^\alpha dt.
\end{equation*}
}

Пусть $f(x)\in W^r[0,\infty)$. Тогда  мы можем рассмотреть коэффициенты Фурье этой функции
\begin{equation}\label{sob-lag-sb-5.6}
\hat f_k= <f,\varphi_k>=f^{(k)}(0), \quad 0\le k\le r-1,
  \end{equation}
\begin{equation}\label{sob-lag-sb-5.7}
 \hat f_k= <f,\varphi_k>= \int_0^\infty e^{-x}f^{(r)}(x) \frac{d^r}{dx^r}L_{k}^{-r}(x)dx, \quad k\ge r
  \end{equation}
и ее ряд Фурье
\begin{equation}\label{sob-lag-sb-5.8}
f(t)\sim \sum_{k=0}^\infty \hat f_k\varphi_k(t)=  \sum_{k=0}^{r-1}f^{(k)}(0)\frac{t^k}{k!}+ \sum_{k=r}^\infty \hat f_k L_{k}^{-r}(t).
   \end{equation}
С другой стороны, из \eqref{sob-lag-sb-5.7} и \eqref{sob-lag-sb-3.1} с учетом  $\left(L_k^{-r}(t)\right)^{(r)}=(-1)^rL_{k-r}^{0}(t)$ имеем
\begin{equation}\label{sob-lag-sb-5.9}
 \hat f_k=  \int_0^\infty e^{-x}f^{(r)}(x) L_{k-r}^{0}(x)dx=(-1)^rf_{r,k-r}^0
  \end{equation}
 Из \eqref{sob-lag-sb-5.8} и \eqref{sob-lag-sb-5.9} получаем
\begin{equation}\label{sob-lag-sb-5.10}
f(t)\sim  \sum_{k=0}^{r-1}f^{(k)}(0)\frac{t^k}{k!}+(-1)^r\sum_{k=0}^\infty f_{r,k}^0L_{k+r}^{-r}(t) .
   \end{equation}
Сопоставляя \eqref{sob-lag-sb-5.10} с \eqref{sob-lag-sb-3.8}, заключаем, что ряд Фурье по системе $\{\varphi_k(t)\}_{k=0}^\infty$ относительно скалярного
произведения \eqref{sob-lag-sb-1.1} совпадает  со смешанным рядом по полиномам Лагерра $L_k^0(t)$.
 Это позволяет использовать при исследовании  аппроксимативных свойств ряда Фурье \eqref{sob-lag-sb-5.8} методы и подходы, разработанные ранее в работах автора \cite{sob-lag-sb-Shar11} -- \cite{sob-lag-sb-Shar16} для решения  аналогичной задачи для смешанных рядов по классическим ортогональным полиномам, в том числе и по полиномам Лагерра. Например, в качестве следствия теоремы 1 может быть  сформулирована
\begin{theorem}   Пусть  $r\ge1$,   $A>0$, $f\in W^r[0,\infty)$. Тогда  ряд Фурье \eqref{sob-lag-sb-5.8}  сходится равномерно относительно $x\in[0,A]$ и для   произвольного $x\in[0,\infty)$ имеет место равенство
\begin{equation}\label{sob-lag-sb-5.11}
f(t)= \sum_{k=0}^\infty \hat f_k\varphi_k(t)=  \sum_{k=0}^{r-1}\hat f_k\frac{t^k}{k!}+ \sum_{k=r}^\infty \hat f_k L_{k}^{-r}(t).
   \end{equation}



\end{theorem}









\subsection{Неравенство Лебега для частичных сумм специального ряда  по полиномам Лагерра}

Через $\mathcal{L}_n^\alpha(f)=\mathcal{L}_n^\alpha(f)(t)$ обозначим частичную сумму специального ряда \eqref{sob-lag-sb-4.5} вида
\begin{equation*}
  \mathcal{L}_n^\alpha(f)(t)=P_{r-1}(f)+t^r\sum\limits_{k=0}^{n-r}\hat{f}_{r,k}^\alpha L_k^\alpha(t).
\end{equation*}

 Заметим, что если $f(t)=q_n(t)$ представляет собой алгебраический полином степени $n$, то при $\alpha>-1$
\begin{equation*}
  \mathcal{L}_n^\alpha(q_n)(t)\equiv q_n(t),
\end{equation*}
другими словами, оператор $\mathcal{L}_n^\alpha(f)$ является проектором на подпространство $H^n$, состоящем из алгебраических полиномов степени $n$.
Это свойство частичных сумм  $\mathcal{L}_n^\alpha(f)(t)$ играет важную роль при решении задачи об оценке отклонения $\mathcal{L}_n^\alpha(f)(t)$ от исходной функции $f=f(t)$. Эта задача является одной из основных в настоящей работе.

Пусть $f(t)$ -- непрерывная функция, заданная на полуоси $[0,\infty)$ и такая, что в точке $t=0$ существуют производные $f^{(\nu)}(0)$ $(\nu=0,1,\dots,r-1)$. Кроме того будем считать, что для всех $k=0,1,\ldots$ существуют коэффициенты $\hat{f}_{r,k}^\alpha$, определяемые равенством \eqref{sob-lag-sb-4.3}. Тогда мы можем определить специальный ряд \eqref{sob-lag-sb-4.5} и его частичную сумму $\mathcal{L}_n^\alpha(f)(t)$. В настоящем параграфе  рассмотрены некоторые вопросы, которые касаются  сходимости $\mathcal{L}_n^\alpha(f)(t)$ к $f(x)$ при $n\to\infty$. В первую очередь рассмотрим задачу об оценке величины
\begin{equation}\label{sob-lag-sb-6.1}
  R_{n,r}^\alpha(f)(t)=|f(t)-\mathcal{L}_n^\alpha(f)(t)|t^{-\frac r2+\frac14}e^{-\frac t2}.
\end{equation}
Весовой множитель $t^{-\frac r2+\frac14}$, фигурирующий в правой части равенства \eqref{sob-lag-sb-6.1}, связан с тем обстоятельством, что разность
$|f(t)-\mathcal{L}_n^\alpha(f)(t)|$ стремится к нулю вместе с $t$ со скоростью, не меньшей, чем  $t^{\frac r2-\frac14}$.
Обозначим через $q_n(t)$ - алгебраический полином степени $n$, для которого
\begin{equation}\label{sob-lag-sb-6.2}
  f^{(\nu)}(0)=q_n^{(\nu)}(0)\text{ }(\nu=0,1,\ldots,r-1).
\end{equation}
Тогда
\begin{equation*}
  f(t)-\mathcal{L}_n^\alpha(f)(t)=f(t)-q_n(t)+q_n(t)-\mathcal{L}_n^\alpha(f)(t)=
\end{equation*}
\begin{equation}\label{sob-lag-sb-6.3}
  f(t)-q_n(t)+\mathcal{L}_n^\alpha(q_n-f)(t),
\end{equation}
поэтому в силу \eqref{sob-lag-sb-6.1} и \eqref{sob-lag-sb-6.3}
\begin{equation}\label{sob-lag-sb-6.4}
  |R_{n,r}(f)(t)|\le|f(t)-q_n(t)|t^{-\frac r2+\frac14}e^{-\frac t2}+|\mathcal{L}_n^\alpha(q_n-f)(t)|t^{-\frac r2+\frac14}e^{-\frac t2}.
\end{equation}
С другой стороны, в силу \eqref{sob-lag-sb-6.2} $P_{r-1}(q_n-f)\equiv0$, поэтому  имеем
\begin{equation*}
  \mathcal{L}_n^\alpha(q_n-f)(t)=t^r\sum\limits_{k=0}^{n-r}(\widehat{q_n-f})_{r,k}L_k^\alpha(t)=
\end{equation*}
\begin{equation*}
  t^r\sum\limits_{k=0}^{n-r}\frac1{h_k^\alpha}\int\limits_0^\infty(q_n(\tau)-f(\tau))\tau^{\alpha-r}e^{-\tau}L_k^\alpha(\tau)L_k^\alpha(t)d\tau.
\end{equation*}
Отсюда
$$
e^{-\frac t2}t^{-\frac r2+\frac14}\mathcal{L}_n^\alpha(q_n-f)(t)=
$$
\begin{equation}\label{sob-lag-sb-6.5}
  e^{-\frac t2}t^{\frac r2+\frac14}\int\limits_0^\infty(q_n(\tau)-f(\tau))e^{-\tau}\tau^{\alpha-r}\sum\limits_{k=0}^{n-r}\frac{L_k^\alpha(t)L_k^\alpha(\tau)}{h_k^\alpha}d\tau.
\end{equation}
Положим
\begin{equation}\label{sob-lag-sb-6.6}
  E_n^r(f)=\inf\limits_{q_n}\sup\limits_{t>0}|q_n(t)-f(t)|e^{-\frac t2}t^{-\frac r2+\frac14},
\end{equation}
где нижняя грань берется по всем алгебраическим полиномам $q_n(t)$ степени $n$ для которых $f^{(\nu)}(0)=q_n^{(\nu)}(0)$ $(\nu=0,\ldots,r-1)$. Тогда из \eqref{sob-lag-sb-6.5} находим
\begin{equation}\label{sob-lag-sb-6.7}
 e^{-\frac{t}{2}} t^{-\frac r2+\frac14}|\mathcal{L}_n^\alpha(q_n-f)(t)|\le E_n^r(f)\lambda_{r,n}^\alpha(t),
\end{equation}
где
\begin{equation}\label{sob-lag-sb-6.8}
  \lambda_{r,n}^\alpha(t)=t^{\frac r2+\frac14}\int\limits_0^\infty e^{-\frac{\tau+t}2}\tau^{\alpha-\frac r2-\frac14}|\mathcal{K}_{n-r}^\alpha(t,\tau)|d\tau,
\end{equation}
 $q_n$  среди рассматриваемых полиномов степени $n$ выбран так, чтобы было
$E_n^r(f)=\sup_{t>0}|q_n(t)-f(t)|e^{-\frac t2}t^{-\frac r2+\frac14}$,
а ядро $\mathcal{K}_{n-r}^\alpha(t,\tau)$ определяется равенством \eqref{sob-lag-sb-2.5}.
Из \eqref{sob-lag-sb-6.4}, \eqref{sob-lag-sb-6.6} -- \eqref{sob-lag-sb-6.8} выводим следующее неравенство типа Лебега
\begin{equation}\label{sob-lag-sb-6.9}
  |R_{n,r}^\alpha(f)(t)|\le E_n^r(f)(1+\lambda_{r,n}^\alpha(t)).
\end{equation}
В связи с неравенством \eqref{sob-lag-sb-6.9} возникает задача об оценке функции Лебега $\lambda_{r,n}^\alpha(t)$, определяемой равенством \eqref{sob-lag-sb-6.8}. С этой целью мы введем следующие обозначения: $G_1=[0,\frac3{\theta_n}]$, $G_2=[\frac3{\theta_n},\frac{\theta_n}2]$, $G_3=[\frac{\theta_n}2,\frac{3\theta_n}2]$, $G_4=[\frac{3\theta_n}2,\infty]$. Мы будем оценивать $\lambda_{r,n}^\alpha(t)$ для $t\in G_s$ $(s=1,2,3,4)$.
\begin{theorem}
Пусть $1\le r$ -- целое, $r-\frac12<\alpha< r+\frac12$, $\theta_n=4n+2\alpha+2$. Тогда имеют место следующие оценки:

1) если $t \in G_1=[0,\frac3{\theta_n}]$,  то
\begin{equation}\label{sob-lag-sb-6.10}
\lambda^\alpha_{r,n}(t) \leq c(\alpha,r)[\ln(n+1)+n^{\alpha-r}];
\end{equation}

2) если $t \in G_2=[\frac3{\theta_n},\frac{\theta_n}2]$, то
\begin{equation}
\lambda_{r,n}^\alpha(t) \leq c(\alpha,r)\left[\ln(n+1)+\left({n\over t}\right)^{\alpha-r\over2}\right];
\label{sob-lag-sb-6.11}
\end{equation}

3) если $t \in G_3=[\frac{\theta_n}2,\frac{3\theta_n}2]$, то
\begin{equation}
\lambda_{r,n}^\alpha(t) \leq c(\alpha,r)\left[\ln(n+1)+\left({t\over \theta_n^{\frac13}+|t-\theta_n| }\right)^\frac14\right];
\label{sob-lag-sb-6.12}
\end{equation}

4) если $t \in G_4=[\frac{3\theta_n}2,\infty)$, то
\begin{equation}
\lambda_{r,n}^\alpha(t) \leq c(\alpha,r)n^{-\frac{r}{2}+\frac54}t^{\frac r2+\frac14}e^{-\frac{t}{4}}.
\label{sob-lag-sb-6.13}
\end{equation}
\end{theorem}








\subsection{Доказательство теоремы 4 }
  Нам понадобятся некоторые преобразования для ядер $\mathcal{K}_n^\alpha(t,\tau)$, определенных равенством \eqref{sob-lag-sb-2.5}. Пользуясь равенством \eqref{sob-lag-sb-2.13} мы можем записать
\begin{equation}\label{sob-lag-sb-7.1}
  \mathcal{K}_n^\alpha(t,\tau)=\frac{\sqrt{(n+1)(n+\alpha+1)}}{\tau-t}\left[\hat{L}_{n+1}^\alpha(t)\hat{L}_n^\alpha(\tau)-\hat{L}_n^\alpha(t)\hat{L}_{n+1}^\alpha(\tau)\right],
\end{equation}
откуда, полагая $\alpha_n=\sqrt{(n+1)(n+\alpha+1)}$, имеем
\begin{equation}\label{sob-lag-sb-7.2}
\frac{1}{\alpha_n}\mathcal{K}_n^\alpha (t,\tau) =
\frac{1}{\tau - t}
\left[\hat{L}_{n+1}^{\alpha}(t) \hat{L}_{n}^{\alpha}(\tau) - \hat{L}_n^{\alpha}(t)\hat{L}_{n+1}^\alpha (\tau)
\right],
\end{equation}

\begin{equation}
\frac{1}{\alpha_{n-1}}\mathcal{K}_n^\alpha (t,\tau) =
\frac{1}{\tau - t}
\left[
\hat{L}_{n}^{\alpha}(t) \hat{L}_{n-1}^{\alpha}(\tau) - \hat{L}_{n-1}^{\alpha}(t)\hat{L}_{n}^\alpha (\tau)
\right]
+\frac{1}{\alpha_{n-1}}\hat{L}_n^\alpha(t)\hat{L}_n^\alpha(\tau).
\label{sob-lag-sb-7.3}
\end{equation}

Складывая правые и левые части равенств \eqref{sob-lag-sb-7.2} и \eqref{sob-lag-sb-7.3}, имеем
\begin{multline*}
\left(\frac{1}{\alpha_n} + \frac{1}{\alpha_{n-1}}\right)\mathcal{K}_n^\alpha(t,\tau) = \frac{1}{\alpha_{n-1}}\hat{L}_{n}^{\alpha}(t)\hat{L}_{n}^{\alpha}(\tau) +\\
\frac{1}{\tau - t}
\left[
\hat{L}_n^\alpha(\tau)\left(\hat{L}_{n+1}^\alpha(t) - \hat{L}_{n-1}^\alpha(t)\right)-
\hat{L}_{n}^\alpha(t)\left(\hat{L}_{n+1}^{\alpha}(\tau) - \hat{L}_{n-1}^{\alpha}(\tau)\right)
\right],
\end{multline*}

стало быть,

\begin{multline}
\mathcal{K}_n^\alpha(t,\tau) =
\frac{\alpha_n}{\alpha_n + \alpha_{n-1}}
\hat{L}_{n}^{\alpha}(t)\hat{L}_{n}^{\alpha}(\tau) +\\
\frac{\alpha_n\alpha_{n-1}}{(\alpha_n+\alpha_{n-1})(\tau - t)}
\left[
\hat{L}_n^\alpha(\tau)\left(\hat{L}_{n+1}^\alpha(t) - \hat{L}_{n-1}^\alpha(t)\right) -
\hat{L}_n^\alpha(t)\left(\hat{L}_{n+1}^\alpha(\tau) - \hat{L}_{n-1}^\alpha(\tau)\right)
\right].
\label{sob-lag-sb-7.4}
\end{multline}

Перейдем к доказательству оценки \eqref{sob-lag-sb-6.10}. Пусть $t \in G_1$,
\begin{equation}
  I_1(t) = t^{r/2+1/4}\int\limits_{0}^{4/\theta_n}
  e^{-\frac{\tau+t}{2}}\tau^{\alpha - r/2 - 1/4} \left| \mathcal{K}_{n-r}^\alpha(t, \tau) \right| d\tau,
\label{sob-lag-sb-7.5}
\end{equation}

\begin{equation}
  I_2(t) = t^{r/2+1/4}\int\limits_{4/\theta_n}^{\infty}
  e^{-\frac{\tau+t}{2}}\tau^{\alpha - r/2 - 1/4} \left| \mathcal{K}_{n-r}^\alpha(t, \tau) \right| d\tau.
\label{sob-lag-sb-7.6}
\end{equation}
Тогда из \eqref{sob-lag-sb-6.8} имеем
\begin{equation}
  \lambda_{r,n}^\alpha(t) \leq I_1(t) + I_2(t).
  \label{sob-lag-sb-7.7}
\end{equation}
Оценим $I_1(t)$. Из \eqref{sob-lag-sb-2.5}, \eqref{sob-lag-sb-2.12} -- \eqref{sob-lag-sb-2.15} имеем
\begin{equation*}
  |\mathcal{K}_{n-r}^\alpha(t,\tau)| \leq c(\alpha)\sum\limits_{k=0}^{n-r}\frac{B_k^\alpha(t) B_k^\alpha(\tau)}{\theta_k^\alpha} \leq c(\alpha)\sum\limits_{k=0}^{n-r}\theta_k^\alpha \leq c(\alpha)\theta_n^{\alpha+1}.
\end{equation*}
Поэтому из \eqref{sob-lag-sb-7.5} находим
\begin{equation}\label{sob-lag-sb-7.8}
I_1(t) \leq c(\alpha)\theta_n^{-r/2-1/4}\int\limits_{0}^{4/\theta_n} \tau^{\alpha-r/2-1/4}\theta_n^{\alpha+1}dt \leq  c(\alpha).
\end{equation}

  Оценим $I_2(t)$. С этой целью обратимся к формуле \eqref{sob-lag-sb-7.4} и запишем
\begin{equation}
  I_2(t) \le I_{21}+I_{22}+I_{23},
  \label{sob-lag-sb-7.9}
\end{equation}
где
\begin{equation*}
  I_{21} = \frac{\alpha_n e^{-\frac{t}{2}}}{\alpha_n + \alpha_{n-1}} |\hat{L}_n^\alpha(t)|t^{\frac{r}{2}+\frac{1}{4}}
  \int\limits_{4/\theta_n}^{\infty} e^{-\frac{\tau}{2}}\tau^{\alpha - \frac{r}{2} -\frac{1}{4}} |\hat{L}_n^\alpha(\tau)|d\tau,
\end{equation*}
\begin{equation*}
  I_{22} = \frac{\alpha_n\alpha_{n-1} e^{-\frac{t}{2}}}{\alpha_n+\alpha_{n-1}}|\hat{L}_{n+1}^\alpha(t) - \hat{L}_{n-1}^\alpha(t)|t^{\frac{r}{2}+\frac{1}{4}}
  \int\limits_{4/\theta_n}^{\infty} \frac{e^{-\frac{\tau}{2}}\tau^{\alpha - \frac{r}{2} -\frac{1}{4}} |\hat{L}_n^\alpha(\tau)|}{\tau - t}d\tau,
\end{equation*}
\begin{equation*}
  I_{23} = \frac{\alpha_n\alpha_{n-1} e^{-\frac{t}{2}}}{\alpha_n+\alpha_{n-1}}
  |\hat{L}_{n}^\alpha(t)|
  t^{\frac{r}{2}+\frac{1}{4}}
  \int\limits_{4/\theta_n}^{\infty} \frac{e^{-\frac{\tau}{2}}\tau^{\alpha - \frac{r}{2} -\frac{1}{4}} |\hat{L}_{n+1}^\alpha(\tau) - \hat{L}_{n-1}^\alpha(\tau)|}
  {\tau - t}d\tau.
\end{equation*}

Положим
\begin{equation}
  W =\int\limits_{4/\theta_n}^{\infty} e^{-\frac{\tau}{2}}\tau^{\alpha - \frac{r}{2} -\frac{1}{4}} |\hat{L}_n^\alpha(\tau)|d\tau= W_1 + W_2, \label{sob-lag-sb-7.10}
\end{equation}
где
\begin{equation}
  W_1 = \int\limits_{4/\theta_n}^{3\theta_n/2} e^{-\tau/2}\tau^{\alpha - r/2 - 1/4}|\hat{L}_n^\alpha(\tau)|d\tau,
  \label{sob-lag-sb-7.11}
\end{equation}
\begin{equation}
  W_2 = \int\limits_{3\theta_n/2}^{\infty} e^{-\tau/2}\tau^{\alpha - r/2 - 1/4} |\hat{L}_n^\alpha(\tau)|d\tau.
  \label{sob-lag-sb-7.12}
\end{equation}
Пользуясь неравенством Коши-Шварца имеем:
\begin{multline}
  W_1 \leq
  \left(
  \int\limits_{4/\theta_n}^{3\theta_n/2} \tau^{\alpha-r-1/2}d\tau
  \right)^{1/2}
  \left(
  \int\limits_{4/\theta_n}^{3\theta_n/2} \tau^\alpha e^{-\tau}(\hat{L}_n^{\alpha}(\tau))^2
  \right)^{1/2}
  < \\
  \left[
  \frac{1}{\alpha-r+\frac12}\left(\left(\frac{3\theta_n}{2}\right)^{\alpha-r+1/2} - \left(\frac{4}{\theta_n}\right)^{\alpha-r+1/2}\right)
  \right]^{1/2}
  \leq
  c(\alpha,r)\theta_n^{\frac{\alpha-r+1/2}{2}}.
  \label{sob-lag-sb-7.13}
\end{multline}
Далее, в силу \eqref{sob-lag-sb-7.12} с учетом \eqref{sob-lag-sb-2.15} имеем
\begin{multline}
  W_2 \leq c(\alpha,r)\theta_n^{-\frac\alpha2}
  \int\limits_{3\theta_n/2}^{\infty}\tau^{\alpha-r/2-1/4}e^{-\tau/4}d\tau = \\
  c(\alpha, r)\theta_n^{-\frac\alpha2}\int\limits_{\frac32(4n+2\alpha+r)}^{\infty} \tau^{\alpha-r/2-1/4}e^{-\tau/4}d\tau
  \leq   c(\alpha,r)e^{-n}.
  \label{sob-lag-sb-7.14}
\end{multline}

Из \eqref{sob-lag-sb-7.10}-\eqref{sob-lag-sb-7.14} находим
\begin{equation*}
  W \leq c(\alpha,r)\theta_n^{\frac{\alpha - r + 1/2}{2}}.
\end{equation*}
Отсюда и из \eqref{sob-lag-sb-2.15}   следует, что если $r - \frac12 < \alpha \leq r + \frac12$, то
\begin{equation}
  I_{21} \leq c(\alpha,r)\theta_n^{\alpha/2}\theta_n^{-r/2-1/4}\theta_n^{\frac{\alpha-r+1/2}{2}}=
  c(\alpha,r)\theta_n^{\alpha-r} \leq c(\alpha,r)n^{\alpha - r}.
  \label{sob-lag-sb-7.15}
\end{equation}

Оценим $I_{22}$. В силу \eqref{sob-lag-sb-2.14} и \eqref{sob-lag-sb-2.15}
\begin{multline}
  I_{22} \le
  c(\alpha)n\theta_n^{-r/2-1/4}\theta_n^{\alpha/2-1}\theta_n^{-\alpha/2}
  \int\limits_{4/\theta_n}^{\infty}\frac{B_n^\alpha(\tau)\tau^{\alpha-r/2-1/4}}{\tau - t}d\tau \le\\
  \frac{c(\alpha)}{\theta_n^{r/2+1/4}}\int\limits_{4/\theta_n}^{\infty}\frac{B_n^\alpha(\tau)\tau^{\alpha - r/2 - 1/4}}{\tau - t}d\tau =
  I{'}_{22}+I^{''}_{22}+I{'''}_{22},
  \label{sob-lag-sb-7.16}
\end{multline}
где
$
  I^{'}_{22} = \frac{c(\alpha)}{\theta_n^{r/2+1/4}}\int_{4/\theta_n}^{\theta_n/2}
$,
$
  I^{''}_{22} = \frac{c(\alpha)}{\theta_n^{r/2+1/4}}\int_{\theta_n/2}^{3\theta_n/2}
$,
$
I_{22}^{'''}=\frac {c(\alpha)}{\theta_n^{ r/2 + 1/4 }}\int_{3\theta_n/2}^\infty
$.
Пользуясь определением функции $B_n^\alpha(t)$ (см. \eqref{sob-lag-sb-2.12}) имеем
$$
I_{22}^{'}\le\frac {c(\alpha)}{\theta_n^{\frac r 2+ \frac 1 4}}\int\limits_{4/\theta_n}^{\theta_n/2}\frac{\theta_n^{\frac \alpha 2-\frac 1 4}\tau^{-\frac \alpha 2- \frac 1 4}\tau^{\alpha-\frac r 2 -\frac 1 4}}{\tau-t}d\tau=
\frac {c(\alpha)\theta_n^{\frac \alpha 2 -\frac 1 4}} {\theta _n^{\frac r 2+\frac 1 4}}\int\limits_{4/\theta_n}^{\theta_n/2}\frac{\tau^{\frac{\alpha-r}{2} - \frac 1 2}}{\tau-t}d\tau\le
$$
\begin{equation}\label {7.17}
c(\alpha)\theta_n^{\frac {\alpha-r} 2-\frac 1 2}\int\limits_{4/\theta}^{\theta_n/2}\tau^{\frac {\alpha-r} 2 - \frac 3 2}d\tau\le
c(\alpha,r)\theta_n^{\frac {\alpha-r} 2-\frac 1 2 }\theta_n^{-\frac {\alpha-r}2+\frac 1 2 }\le c(\alpha,r),
\end{equation}
$$
I_{22}^{''}\le \frac {c(\alpha)}{\theta_n^{\frac r2+\frac 1 4}}\int\limits_{\theta_n/2}^{3\theta_n/2}\frac {[\theta_n(\theta_n^{\frac 1 3}+|\tau-\theta_n|)]^{-\frac 1 4 }\tau^{\alpha-\frac r 2 - \frac 1 4}}{\tau-t}d\tau\le
$$
$$
\frac {c(\alpha)}{\theta_n^{\frac r 2 +\frac 1 2}}\int\limits_{\theta_n/2}^{3\theta_n/2}(\theta_n^{\frac 1 3}+|\tau-\theta_n|)^{-\frac14}\tau^{\alpha-\frac r 2 -\frac 5 4}d\tau\le
$$
\begin{equation}\label{sob-lag-sb-7.18}
\frac {c(\alpha)}{\theta_n^{\frac r 2 +\frac 1 2}}\int\limits_{\theta_n/2}^{3\theta_n/2}(\theta_n^{\frac 1 3}+|\tau-\theta_n|)^{-\frac14}\theta_n^{\alpha-\frac r 2 -\frac 5 4}d\tau\le
c(\alpha)\theta_n^{\alpha-r- \frac 3 4}\theta_n^{\frac 3 4}\le c(\alpha)\theta_n^{\alpha-r}.
\end{equation}
Далее, для $I_{22}^{'''}$ из \eqref{sob-lag-sb-7.19} имеем
$$
I_{22}^{'''}\le \frac {c(\alpha)}{\theta_n^{\frac r 2 +\frac 1 4}}\int_{3\theta_n/2}^\infty \frac {e^{-\frac \tau 4}\tau^{\alpha-\frac r 2 - \frac 1 4}} {\tau -t} d\tau\le
 \frac {c(\alpha)}{\theta_n^{\frac r 2 +\frac 1 4}} \int_{3\theta_n/2}^\infty e^{-\frac \tau 4}\tau^{\alpha-\frac r 2 - \frac 5 4} d\tau \le
$$
\begin{equation}\label{sob-lag-sb-7.19}
c(\alpha,r)n^{-\frac r 2- \frac 1 4}e^{-n}.
\end{equation}
Собирая оценки \eqref{sob-lag-sb-7.17} -- \eqref{sob-lag-sb-7.19} и сопоставляя их с \eqref{sob-lag-sb-7.16}, выводим
\begin{equation}\label{sob-lag-sb-7.20}
I_{22}\le c(\alpha, r)(1+\theta _n^{\alpha - r}).
\end{equation}
Перейдем к оценке $I_{23}$ при $t\in G_{1}$. Используя оценки \eqref{sob-lag-sb-2.14} и \eqref{sob-lag-sb-2.15}, мы можем записать
\begin{equation}\label{sob-lag-sb-7.21}
I_{23}\le c(\alpha)\theta_n^{-\frac \alpha 2}\theta_n^\alpha\theta_n^{-\frac r 2+\frac 3 4}[H_1+H_2+H_3]=c(\alpha)\theta_n^{\frac {\alpha-r} 2 +\frac 3 4 }[H_1+H_2+H_3],
\end{equation}
где
$$
H_1=\int\limits_{4/\theta_n}^{\theta_n/2}\frac{\tau^{\alpha-\frac r 2 -\frac 14} \theta_n^{- \frac 3 4}\tau^{-\frac \alpha 2+ \frac 1 4}} {\tau-t}=
$$
\begin{equation}\label{sob-lag-sb-7.22}
    \theta_n^{-\frac 3 4}\int\limits_{4/\theta _n}^{\theta_n/2}\tau^{\frac {\alpha-r} 2 -1}d\tau\le \theta_n^{-\frac 3 4}
    \left\{
    \begin{aligned}
        2\ln \theta_n -3\ln 2,\qquad \alpha =r,\\
        \frac 2{\alpha -r}\left[\left(\frac {\theta_n^\frac {\alpha-r} 2} 2\right)- \left(\frac 4 {\theta_n}\right)^\frac {\alpha -r}2\right], \qquad \alpha \not=r,
    \end{aligned}
        \right.
\end{equation}
$$
H_2=\theta_n^{-\frac 3 4}\int\limits_{\theta_n/2}^{3\theta_n/2}\frac{\tau^{-\frac \alpha 2}(\theta_n^{\frac 1 3}+|\tau-\theta_n|)^{\frac 1 4}\tau^{\alpha - \frac r 2 - \frac 1 4}}{\tau-t}d\tau \le
$$
\begin{equation}\label {7.23}
\theta_n^{-\frac 3 4}\theta_n^{\frac {\alpha-r}2-\frac 5 4}\int\limits_{\theta_n/2}^{3\theta_n/2}(\theta_n^\frac 1 3+|\tau-\theta_n|)^{\frac 1 4}d
\tau\le c\theta_n^{\frac {\alpha - r} 2-2}\theta_n^{\frac 5 4}\le c(\alpha, r)\theta_n^{\frac {\alpha - r} 2 - \frac 3 4},
\end{equation}
\begin{equation}\label{sob-lag-sb-7.24}
H_3=\int\limits_{3\theta_n/2}^\infty\frac{\tau^{\alpha-\frac r 2- \frac 1 4}e^{-\frac \tau 4}}{\tau - t}d\tau<\int\limits_{3\theta_{n/2}}^\infty \tau^{\alpha-\frac r 2 - \frac 5 4}
e^{-\frac \tau 4}d\tau\le c(\alpha, r)e^{-n}.
\end{equation}
Из \eqref{sob-lag-sb-7.22} -- \eqref{sob-lag-sb-7.24} имеем
\begin{equation}\label {7.25}
I_{23}\le c(\alpha, r)\begin{cases}\ln n,&\text{$\alpha= r$}\\n^{\alpha - r},&\text{$\alpha\ne r$}
\end{cases}.
\end{equation}
Из  оценок \eqref{sob-lag-sb-7.9}, \eqref{sob-lag-sb-7.15}, \eqref{sob-lag-sb-7.16},  \eqref{sob-lag-sb-7.20}  и   \eqref{sob-lag-sb-7.25} выводим
\begin{equation} \label{sob-lag-sb-7.26}
I_2(t)\le c(\alpha, r)
\left\{
\begin{aligned}
\ln n, \quad \mbox{если } \alpha = r,\\
1+n^{\alpha - r},\quad \mbox{если }\alpha \not= r.
\end{aligned}
\right.
\end{equation}
А из \eqref{sob-lag-sb-7.7}, \eqref{sob-lag-sb-7.8} и \eqref{sob-lag-sb-7.26} имеем
$$
\lambda_{r,n}^{\alpha}(t)\le c(\alpha, r)
\left\{
\begin{aligned}
\ln n, \quad \mbox{если } \alpha = r,\\
1+n^{\alpha - r},\quad \mbox{если }\alpha \not= r.
\end{aligned}
\right.
$$
Тем самым оценка \eqref{sob-lag-sb-6.10} доказана.

Перейдем к доказательству оценки \eqref{sob-lag-sb-6.11}. Пусть $t\in G_2=\left[\frac{3}{\theta_n},\frac{\theta_n}{2}\right]$. Тогда мы можем записать
$(0,\infty)=Q_1\cup Q_2\cup Q_3$, где
$$
Q_1=[0,t-\sqrt{t/\theta_n}], Q_2=[t-\sqrt{t/\theta_n},t+\sqrt{t/\theta_n}], Q_3=[t+\sqrt{t/\theta_n},\infty).
$$
Используя эти обозначения, из \eqref{sob-lag-sb-6.8} имеем

\begin{equation}\label{sob-lag-sb-7.27}
\lambda_{r,n}^\alpha(t)=J_1+J_2+J_3,
\end{equation}
где
\begin{equation}\label{sob-lag-sb-7.28}
  J_k=t^{\frac r2+\frac14}\int\limits_{Q_k} e^{-\frac{\tau+t}2}\tau^{\alpha-\frac r2-\frac14}|\mathcal{K}_{n-r}^\alpha(t,\tau)|d\tau\quad(1\le k\le 3).
\end{equation}
Оценим $J_2$. Для этого сначала заметим, что в силу неравенства Коши-Шварца
\begin{equation}\label {7.29}
|\mathcal{K}_{n-r}^\alpha(t,\tau)|\le (\mathcal{K}_{n-r}^\alpha(t,t))^{1/2}(\mathcal{K}_{n-r}^\alpha(\tau,\tau))^{1/2}.
\end{equation}
Далее, если $3/\theta_n\le t\le 3\theta_n/2$, то $t-\sqrt{t/\theta_n}\ge1/\theta_n$, кроме того для $\tau\in[t-\sqrt{t/\theta_n},t+\sqrt{t/\theta_n}]$ имеем $c_1t\le \tau\le c_2t$. Поэтому из \eqref{sob-lag-sb-7.28} и \eqref{sob-lag-sb-7.29} имеем
$$
J_2=\int\limits_{Q_2}\left(\frac{t}{\tau}\right)^{r/2+1/4}\tau^\alpha e^{-\frac{\tau+t}2}|\mathcal{K}_{n-r}^\alpha(t,\tau)|d\tau\le
$$
\begin{equation}\label{sob-lag-sb-7.30}
c(e^{-t}\mathcal{K}_{n-r}^\alpha(t,t))^{1/2} \int\limits_{Q_2}\tau^\alpha  (e^{-\tau}\mathcal{K}_{n-r}^\alpha(\tau,\tau))^{1/2}d\tau.
\end{equation}
Неравенство \eqref{sob-lag-sb-7.30} приводит, в свою очередь, к задаче об оценке для $e^{-\tau}\mathcal{K}_{n}^\alpha(\tau,\tau)$ при
$3/\theta_n\le \tau\le 3\theta_n/2$. Следующее утверждение дает ответ на этот вопрос.

\begin{lemma}\label{sob-lag-sb-7.1}
Пусть $\alpha>-1$, $\theta_k=4k+2\alpha+2$, $\tau\ge3/\theta_n$. Тогда имеет место оценка
\begin{equation}\label{sob-lag-sb-7.31}
e^{-\tau}\mathcal{K}_{n}^\alpha(\tau,\tau)\le c(\alpha)\tau^{-\alpha-1/2}n^{1/2}.
\end{equation}
\end{lemma}
\begin{lemma}\label{sob-lag-sb-7.2}
Пусть $u=\sqrt{t/\theta_n}$, $\alpha>-1$, $\theta_n=4n+2\alpha+2$, $3/\theta_n\le t \le 3\theta_n/2$. Тогда имеет место оценка
\begin{equation}\label{sob-lag-sb-7.41}
I=(e^{-t}\mathcal{K}_{n}^\alpha(t,t))^{1/2}\int_{Q_2} \tau^{\alpha}(e^{-\tau}\mathcal{K}_{n}^\alpha(\tau,\tau))^{1/2}d\tau\le c(\alpha).
\end{equation}
\end{lemma}
 Из равенства \eqref{sob-lag-sb-7.28} и леммы 7.2 выводим
\begin{equation}\label{sob-lag-sb-7.42}
J_2\le c(\alpha).
\end{equation}

Оценим $J_3$ при $t\in G_2$. Воспользовавшись равенствами \eqref{sob-lag-sb-7.28} и \eqref{sob-lag-sb-7.4} мы можем записать
\begin{equation}\label{sob-lag-sb-7.43}
J_3\le c(\alpha)(J_{31}+J_{32}+J_{33}),
\end{equation}
где
\begin{equation}\label{sob-lag-sb-7.44}
J_{31}=t^{\frac{r}{2}+\frac14}e^{-t/2}|\hat L_{n-r}^\alpha(t)|\int_{Q_3}\tau^{\alpha-\frac{r}{2}-\frac14}e^{-\tau/2}|\hat L_{n-r}^\alpha(\tau)|d\tau,
\end{equation}
\begin{equation}\label{sob-lag-sb-7.45}
J_{32}=nt^{\frac{r}{2}+\frac14}e^{-\frac{t}{2}}|\hat L_{n-r+1}^\alpha(t)-\hat L_{n-r-1}^\alpha(t)|\int_{Q_3}{\tau^{\alpha-\frac{r}{2}-\frac14}e^{-\frac{\tau}{2}}\over \tau-t}|\hat L_{n-r}^\alpha(\tau)|d\tau, \end{equation}
\begin{equation}\label{sob-lag-sb-7.46}
J_{33}=nt^{\frac{r}{2}+\frac14}e^{-\frac{t}{2}}|\hat L_{n-r}^\alpha(t)|\int_{Q_3}{\tau^{\alpha-\frac{r}{2}-\frac14}e^{-\frac{\tau}{2}}\over \tau-t}|\hat L_{n-r+1}^\alpha(\tau)-\hat L_{n-r-1}^\alpha(\tau)|d\tau.
\end{equation}
Чтобы оценить величину $J_{31}$ представим ее в следующем виде
\begin{equation}\label{sob-lag-sb-7.47}
J_{31}=J_{31}^1+J_{31}^2+J_{31}^3,
\end{equation}
в котором $(k=1,2,3)$
\begin{equation}\label{sob-lag-sb-7.48}
J_{31}^k=t^{\frac{r}{2}+\frac14}e^{-t/2}|\hat L_{n-r}^\alpha(t)|\int_{Q_{31}^k}\tau^{\alpha-\frac{r}{2}-\frac14}e^{-\tau/2}|\hat L_{n-r}^\alpha(\tau)|d\tau,
\end{equation}
где $Q_{31}^1=(t+\sqrt{t/\theta_n},\theta_n/2)$, $Q_{31}^2=(\theta_n/2,3\theta_n/2)$, $Q_{31}^3=(3\theta_n/2,\infty)$. Обратимся к  неравенству \eqref{sob-lag-sb-2.15}, тогда из  \eqref{sob-lag-sb-7.48} находим
$$
J_{31}^1\le c(\alpha,r)n^{-\frac12}t^\frac{r-\alpha}{2}\int_{Q_{31}^1}\tau^{\frac{\alpha-r}{2}-\frac12}d\tau=
$$
\begin{equation}\label{sob-lag-sb-7.49}
{c(\alpha,r)\over\frac{\alpha-r}{2}+\frac12}n^{-\frac12}t^\frac{r-\alpha}{2}\left[\left(\frac{\theta_n}{2}\right)^{\frac{\alpha-r}{2}+\frac12}-
\left(t+\sqrt{\frac{t}{\theta_n}}\right)^{\frac{\alpha-r}{2}+\frac12}\right]\le
c(\alpha,r)\left(\frac{n}{t}\right)^{\frac{\alpha-r}{2}},
\end{equation}
$$
J_{31}^2\le c(\alpha,r)n^{-\frac12}t^\frac{r-\alpha}{2}\int_{Q_{31}^2}{\tau^{\frac{\alpha-r}{2}-\frac14}d\tau\over(\theta_n^{1/3}+|\tau-\theta_n|)^{1/4}}
$$
$$
\le c(\alpha,r)n^{-\frac34+\frac{\alpha-r}{2}}t^\frac{r-\alpha}{2}\int_{Q_{31}^2}{d\tau\over(\theta_n^{1/3}+|\tau-\theta_n|)^{1/4}}
$$
$$
\le c(\alpha,r)n^{-\frac34}\left(\frac{n}{t}\right)^{\frac{\alpha-r}{2}}
\int_{Q_{32}^2}{d\tau\over(\theta_n^{1/3}+|\tau-\theta_n|)^{1/4}}
$$
\begin{equation}\label{sob-lag-sb-7.50}
\le c(\alpha,r)\left(\frac{n}{t}\right)^{\frac{\alpha-r}{2}}
2n^{-\frac34}\int_{\theta_n}^{3\theta_n/2}{d\tau\over(\tau+ \theta_n^{1/3}-\theta_n)^{1/4}}
\le c(\alpha,r)\left(\frac{n}{t}\right)^{\frac{\alpha-r}{2}},
\end{equation}
$$
J_{31}^3\le c(\alpha,r)n^{-\frac14}t^\frac{r-\alpha}{2}\int_{Q_{31}^3}\tau^{\alpha-\frac{r}{2}-\frac14}\theta_n^{-\frac{\alpha}{2}}e^{-\tau/4}d\tau
$$
$$
\le c(\alpha,r)n^{-\frac14-\frac{\alpha}{2}}t^\frac{r-\alpha}{2}\int_{Q_{31}^3}\tau^{\alpha-\frac{r}{2}-\frac14}e^{-\tau/4}d\tau
$$
\begin{equation}\label{sob-lag-sb-7.51}
\le c(\alpha,r)\left(\frac{n}{t}\right)^{\frac{\alpha-r}{2}}n^{\frac{r-\alpha}{2}-\frac14}\le c(\alpha,r)\left(\frac{n}{t}\right)^{\frac{\alpha-r}{2}}.
\end{equation}
Из \eqref{sob-lag-sb-7.47},  \eqref{sob-lag-sb-7.49} -- \eqref{sob-lag-sb-7.51} выводим
\begin{equation}\label{sob-lag-sb-7.52}
J_{31}\le c(\alpha,r)\left(\frac{n}{t}\right)^{\frac{\alpha-r}{2}}\quad (t\in G_2, r-\frac12<\alpha<r+\frac12).
\end{equation}
Переходя к оценке величины $J_{32}$, представим ее в виде
\begin{equation}\label{sob-lag-sb-7.53}
J_{32}=J_{32}^1+J_{32}^2+J_{32}^3,
\end{equation}
в котором $(k=1,2,3)$
\begin{equation}\label{sob-lag-sb-7.54}
J_{32}^k=nt^{\frac{r}{2}+\frac14}e^{-\frac{t}{2}}|\hat L_{n-r+1}^\alpha(t)-\hat L_{n-r-1}^\alpha(t)|\int_{Q_{32}^k}{\tau^{\alpha-\frac{r}{2}-\frac14}e^{-\frac{\tau}{2}}\over \tau-t}|\hat L_{n-r}^\alpha(\tau)|d\tau,
\end{equation}
где $Q_{32}^1=(t+\sqrt{t/\theta_n},\theta_n/2+\sqrt{t/\theta_n})$, $Q_{32}^2=(\theta_n/2+\sqrt{t/\theta_n},3\theta_n/2)$, $Q_{32}^3=(3\theta_n/2,\infty)$. Обратимся  к неравенству \eqref{sob-lag-sb-2.13}, тогда из  \eqref{sob-lag-sb-7.48} находим
$$
J_{32}^1\le c(\alpha,r)t^{\frac{r-\alpha}{2}+\frac12}\int_{Q_{32}^1}{\tau^{\frac{\alpha-r}{2}-\frac12}\over \tau-t}d\tau\le
$$
$$
c(\alpha,r)\int\limits_{t+\sqrt{t/\theta_n}}^{2t}{d\tau\over \tau-t}+c(\alpha,r)t^{\frac{r-\alpha}{2}+\frac12}\int\limits_{2t}^{\frac{\theta_n}{2}+\sqrt{\frac{t}{\theta_n}}}\tau^{\frac{\alpha-r}{2}-\frac32}d\tau\le
$$
$$
c(\alpha,r)\ln\frac{t}{\sqrt{t/\theta_n}}+{c(\alpha,r)t^{\frac{r-\alpha}{2}+\frac12}\over\frac{\alpha-r}{2}-\frac12}
\left[\left(\frac{\theta_n}{2}+\sqrt{\frac{t}{\theta_n}}\right)^{\frac{\alpha-r}{2}-\frac12}-(2t)^{\frac{\alpha-r}{2}-\frac12}\right]
$$
\begin{equation}\label{sob-lag-sb-7.55}
\le c(\alpha,r)\ln\sqrt{\frac{\theta_n}{t}}+c(\alpha,r)\left[1-\left({\frac{\theta_n}{2}+\sqrt{\frac{t}{\theta_n}}\over t}\right)^{\frac{\alpha-r}{2}-\frac12}\right]
 \le c(\alpha,r)\ln\sqrt{\frac{\theta_n}{t}},
\end{equation}
$$
J_{32}^2\le c(\alpha,r)nt^{\frac{r}{2}+\frac14}\theta_n^{-\frac34}t^{-\frac{\alpha}{2}+\frac14}
\int_{Q_{32}^2}\theta_n^{-\frac14}(\theta_n^{\frac13}+|\tau-\theta_n|)^{-\frac14}{\tau^{\alpha-\frac{r}{2}
-\frac14}d\tau\over\theta_n^{\frac{\alpha}2} (\tau-t)}
$$
$$
\le c(\alpha,r)t^{\frac{r-\alpha}{2}+\frac12}n^{\frac{\alpha-r}{2}-\frac14}
\int_{Q_{32}^2}(\theta_n^{\frac13}+|\tau-\theta_n|)^{-\frac14}{d\tau\over \tau-t}
$$
$$
\le c(\alpha,r)t^{\frac{r-\alpha}{2}+\frac12}n^{\frac{\alpha-r}{2}-\frac14}\left[
\int\limits_{\frac{\theta_n}{2}+\sqrt{\frac{t}{\theta_n}}}^{\theta_n-\theta_n^{1/3}}+
\int\limits_{\theta_n-\theta_n^{1/3}}^{3\theta_n/2}\right](\theta_n^{\frac13}+|\tau-\theta_n|)^{-\frac14}{d\tau\over \tau-t}
$$
$$
\le c(\alpha,r)\left(\frac{n}{t}\right)^{\frac{\alpha-r}{2}}\sqrt{\frac{t}{\theta_n}}
\left(\ln{\frac{\theta_n}{t}-1\over\frac{\theta_n}{2t}+\sqrt{\frac{1}{t\theta_n}}-1}+1\right)
$$
\begin{equation}\label{sob-lag-sb-7.56}
 \le c(\alpha,r)\left(\frac{n}{t}\right)^{\frac{\alpha-r}{2}}\sqrt{\frac{t}{\theta_n}}
\ln{\frac{\theta_n}{t}-1\over\frac{\theta_n}{2t}+\sqrt{\frac{1}{t\theta_n}}-1},
 \end{equation}
$$
J_{32}^3\le c(\alpha,r)nt^{\frac{r}{2}+\frac14}\theta_n^{-\frac34}t^{-\frac{\alpha}{2}+\frac14}
\int_{Q_{32}^3}e^{-\tau/4}\theta_n^{-\frac{\alpha}{2}}\tau^{\alpha-\frac{r}{2}-\frac54}d\tau
$$
\begin{equation}\label{sob-lag-sb-7.57}
\le c(\alpha,r)t^{\frac{r-\alpha}{2}+\frac12}n^{-\frac{\alpha}{2}+\frac14}
\int_{Q_{32}^3}e^{-\tau/4}\tau^{\alpha-\frac{r}{2}-\frac54}d\tau
 \le c(\alpha,r)e^{-3\theta_n/8}.
 \end{equation}
Из \eqref{sob-lag-sb-7.53} -- \eqref{sob-lag-sb-7.57} выводим оценку
\begin{equation}\label{sob-lag-sb-7.58}
J_{32} \le c(\alpha,r)\left[\ln\frac{\theta_n}{t}+\left(\frac{n}{t}\right)^{\frac{\alpha-r}{2}}\sqrt{\frac{t}{\theta_n}}
\ln{\frac{\theta_n}{t}-1\over\frac{\theta_n}{2t}-1+\sqrt{\frac{1}{t\theta_n}}}\right].
 \end{equation}
Оценим $J_{33}$ по той же схеме, что и $J_{32}$. Имеем представление
\begin{equation}\label{sob-lag-sb-7.59}
J_{33}=J_{33}^1+J_{33}^2+J_{33}^3,
\end{equation}
в котором $(k=1,2,3)$
\begin{equation}\label{sob-lag-sb-7.60}
J_{33}^k=nt^{\frac{r}{2}+\frac14}e^{-\frac{t}{2}}|\hat L_{n-r}^\alpha(t)|\int_{Q_{{33}}^k}{\tau^{\alpha-\frac{r}{2}-\frac14}e^{-\frac{\tau}{2}}\over \tau-t}|\hat L_{n-r+1}^\alpha(\tau)-\hat L_{n-r-1}^\alpha(\tau)|d\tau,
\end{equation}
где $Q_{33}^1=(t+\sqrt{t/\theta_n},\theta_n/2+\sqrt{t/\theta_n})$, $Q_{33}^2=(\theta_n/2+\sqrt{t/\theta_n},3\theta_n/2)$, $Q_{33}^3=(3\theta_n/2,\infty)$. Обратимся  к неравенству \eqref{sob-lag-sb-2.13}, тогда из  \eqref{sob-lag-sb-7.60} находим
$$
J_{33}^1\le c(\alpha,r)t^{\frac{r-\alpha}{2}}\int_{Q_{33}^1}{\tau^{\frac{\alpha-r}{2}}\over \tau-t}d\tau\le
$$
$$
c(\alpha,r)\int\limits_{t+\sqrt{t/\theta_n}}^{2t}{d\tau\over \tau-t}+c(\alpha,r)t^{\frac{r-\alpha}{2}}\int\limits_{2t}^{\frac{\theta_n}{2}+\sqrt{\frac{t}{\theta_n}}}\tau^{\frac{\alpha-r}{2}-1}d\tau\le
$$
\begin{equation}\label{sob-lag-sb-7.61}
c(\alpha,r)\left(\ln\sqrt{t\theta_n}+\left(\frac{\theta_n}{t}\right)^\frac{\alpha-r}{2}\right),
\end{equation}

$$
J_{33}^2\le c(\alpha,r)nt^{\frac{r}{2}+\frac14}\theta_n^{-\frac14}t^{-\frac{\alpha}{2}-\frac14}
\int_{Q_{33}^2}\theta_n^{-\frac34}(\theta_n^{\frac13}+|\tau-\theta_n|)^{\frac14}{\tau^{\frac{\alpha-r}{2}
-\frac14}d\tau\over  \tau-t}
$$
$$
\le c(\alpha,r)t^{\frac{r-\alpha}{2}}\int_{Q_{32}^2}\left(\theta_n^{\frac13}+|\tau-\theta_n|\over \tau\right)^{\frac14}{\tau^{\frac{\alpha-r}{2}}d\tau\over \tau-t}
$$
$$
\le c(\alpha,r)t^{\frac{r-\alpha}{2}}\theta_n^{\frac{\alpha-r}{2}}
\int\limits_{\frac{\theta_n}{2}+\sqrt{\frac{t}{\theta_n}}}^{\theta_n}{(\theta_n^{\frac13}+\theta_n-\tau)^{1/4}\over  \tau^{1/4}}{d\tau\over \tau-t}
$$
$$
+ c(\alpha,r)t^{\frac{r-\alpha}{2}}\theta_n^{\frac{\alpha-r}{2}}
\int\limits_{\theta_n}^{\frac{3\theta_n}{2}}{(\theta_n^{\frac13}-\theta_n+\tau)^{1/4}\over \theta_n^{1+1/4}}d\tau
$$
$$
\le c(\alpha,r)(t/n)^{\frac{r-\alpha}{2}}\left(1+\ln\frac{\theta_n-t}{\theta_n/2+\sqrt{t/\theta_n}-t}\right)
$$
\begin{equation}\label{sob-lag-sb-7.62}
=c(\alpha,r)\left(\frac{n}{t}\right)^\frac{\alpha-r}{2}
\left(1+\ln\frac{\theta_n/t-1}{\theta_n/(2t)+\sqrt{1/(t\theta_n)}-1}\right),
\end{equation}
$$
J_{33}^3\le c(\alpha,r)nt^{\frac{r}{2}+\frac14}\theta_n^{-\frac14}t^{-\frac{\alpha}{2}-\frac14}
\int_{Q_{33}^3}{\tau^{\alpha-\frac{r}{2}-\frac14}e^{-\frac{\tau}{4}}d\tau\over  \tau-t}
$$
\begin{equation}\label{sob-lag-sb-7.63}
\le c(\alpha,r)n^\frac34
\int_{Q_{33}^3}{\tau^{\alpha-\frac{r}{2}-\frac54}e^{-\frac{\tau}{4}}d\tau\over  \tau-t}
\le c(\alpha,r)e^{-\frac{3\theta_n}{8}}\le c(\alpha,r)e^{-\frac{3n}{2}}.
\end{equation}
Из \eqref{sob-lag-sb-7.59} -- \eqref{sob-lag-sb-7.63} находим
\begin{equation}\label{sob-lag-sb-7.64}
J_{33}\le c(\alpha,r)\le\left[\ln\sqrt{t\theta_n} +\left(\frac{\theta_n}{t}\right)^\frac{\alpha-r}{2}
\left(1+\ln\frac{\theta_n/t-1}{\theta_n/(2t)+\sqrt{1/(t\theta_n)}-1}\right)\right],
\end{equation}
 а из \eqref{sob-lag-sb-7.52}, \eqref{sob-lag-sb-7.58} и \eqref{sob-lag-sb-7.64}, в свою очередь, получаем

\begin{equation}\label{sob-lag-sb-7.65}
J_{3}\le c(\alpha,r)\left[\ln\sqrt{t\theta_n} +\left(\frac{\theta_n}{t}\right)^\frac{\alpha-r}{2}
\ln\frac{\theta_n/t-1}{\theta_n/(2t)+\sqrt{1/(t\theta_n)}-1}\right].
\end{equation}

 Оценим $J_1$ при $t\in G_2$. Воспользовавшись равенствами \eqref{sob-lag-sb-7.28} и \eqref{sob-lag-sb-7.4} мы можем записать
\begin{equation}\label{sob-lag-sb-7.66}
J_1\le c(\alpha)(J_{11}+J_{12}+J_{13}),
\end{equation}
где
\begin{equation}\label{sob-lag-sb-7.67}
J_{11}=t^{\frac{r}{2}+\frac14}e^{-t/2}|\hat L_{n-r}^\alpha(t)|\int_{Q_1}\tau^{\alpha-\frac{r}{2}-\frac14}e^{-\tau/2}|\hat L_{n-r}^\alpha(\tau)|d\tau,
\end{equation}
\begin{equation}\label{sob-lag-sb-7.68}
J_{12}=nt^{\frac{r}{2}+\frac14}e^{-\frac{t}{2}}|\hat L_{n-r+1}^\alpha(t)-\hat L_{n-r-1}^\alpha(t)|\int_{Q_1}{\tau^{\alpha-\frac{r}{2}-\frac14}e^{-\frac{\tau}{2}}\over \tau-t}|\hat L_{n-r}^\alpha(\tau)|d\tau \end{equation}
\begin{equation}\label{sob-lag-sb-7.69}
J_{13}=nt^{\frac{r}{2}+\frac14}e^{-\frac{t}{2}}|\hat L_{n-r}^\alpha(t)|\int_{Q_1}{\tau^{\alpha-\frac{r}{2}-\frac14}e^{-\frac{\tau}{2}}\over \tau-t}|\hat L_{n-r+1}^\alpha(\tau)-\hat L_{n-r-1}^\alpha(\tau)|d\tau.
\end{equation}
Чтобы оценить величину $J_{11}$ представим ее в следующем виде
\begin{equation}\label{sob-lag-sb-7.70}
J_{11}=J_{11}^1+J_{11}^2,
\end{equation}
в котором $(k=1,2)$
\begin{equation}\label{sob-lag-sb-7.71}
J_{11}^k=t^{\frac{r}{2}+\frac14}e^{-t/2}|\hat L_{n-r}^\alpha(t)|\int_{Q_{1}^k}\tau^{\alpha-\frac{r}{2}-\frac14}e^{-\tau/2}|\hat L_{n-r}^\alpha(\tau)|d\tau,
\end{equation}
где $Q_{1}^1=(0,1/\theta_n)$, $Q_{1}^2=(1/\theta_n,t-\sqrt{t/\theta_n})$. Обратимся к к неравенству \eqref{sob-lag-sb-2.13}, тогда из  \eqref{sob-lag-sb-7.71} находим
$$
J_{11}^1\le c(\alpha,r)\theta_n^{-\frac14}t^\frac{r-\alpha}{2}\int_0^{1/\theta_n}\tau^{\alpha-\frac{r}{2}-\frac14}\theta_n^{\frac\alpha2}d\tau=
$$
\begin{equation}\label{sob-lag-sb-7.72}
{c(\alpha,r)\over \alpha-\frac{r}{2}+\frac34}\theta_n^{-\frac14}t^\frac{r-\alpha}{2}\theta_n^{\frac\alpha2}\theta_n^{-\alpha+\frac{r}{2}-\frac34}\le
c(\alpha,r)\left(t\theta_n\right)^\frac{r-\alpha}{2}\theta_n^{-1}\le c(\alpha,r)\theta_n^{-\frac12},
\end{equation}
$$
J_{11}^2\le c(\alpha,r)\theta_n^{-\frac14}t^\frac{r-\alpha}{2}\int_{1/\theta_n}^{t-\sqrt{t/\theta_n}}
\tau^{\alpha-\frac{r}{2}-\frac14}\theta_n^{-\frac14}\tau^{-\frac{\alpha}{2}-\frac14}d\tau=
$$
$$
c(\alpha,r)\theta_n^{-\frac12}t^\frac{r-\alpha}{2}\int_{1/\theta_n}^{t-\sqrt{t/\theta_n}}
\tau^{\frac{\alpha-r}{2}-\frac12}d\tau=
$$
$$
{c(\alpha,r)\over\frac{\alpha-r}{2}+\frac12}\theta_n^{-\frac12}t^\frac{r-\alpha}{2}
\left[
\left(t-\sqrt{\frac{t}{\theta_n}}\right)^{\frac{\alpha-r}{2}+\frac12}-\left(\frac{1}{\theta_n}\right)^{\frac{\alpha-r}{2}+\frac12}\right]\le
$$
\begin{equation}\label{sob-lag-sb-7.73}
c(\alpha,r)(t/\theta_n)^{\frac{1}{2}},
\end{equation}

Из \eqref{sob-lag-sb-7.72} и \eqref{sob-lag-sb-7.73} имеем
\begin{equation}\label{sob-lag-sb-7.74}
J_{11}\le c(\alpha,r)\left[(t/\theta_n)^{\frac{1}{2}}+\theta_n^{-\frac12}\right].
\end{equation}
Оценим $J_{12}$. С этой целью имеем
\begin{equation}\label{sob-lag-sb-7.75}
J_{12}=J_{12}^1+J_{12}^2,
\end{equation}
в котором ($k=1,2$)
\begin{equation}\label{sob-lag-sb-7.76}
J_{12}^k=nt^{\frac{r}{2}+\frac14}e^{-\frac{t}{2}}|\hat L_{n-r+1}^\alpha(t)-\hat L_{n-r-1}^\alpha(t)|\int_{Q_1^k}{\tau^{\alpha-\frac{r}{2}-\frac14}e^{-\frac{\tau}{2}}\over |\tau-t|}|\hat L_{n-r}^\alpha(\tau)|d\tau,
\end{equation}
где $Q_1^k$ имеет тот же смысл, что в \eqref{sob-lag-sb-7.71}. Обратимся снова к к неравенству \eqref{sob-lag-sb-2.13}, тогда из \eqref{sob-lag-sb-7.74} выводим

$$
J_{12}^1\le c(\alpha,r)nt^{\frac{r}{2}+\frac14}\theta_n^{-\frac34}t^{-\frac{\alpha}{2}+\frac14}\int_0^{1/\theta_n}
{\tau^{\alpha-\frac{r}{2}-\frac14}\over t-\tau}\theta_n^{\frac{\alpha}{2}}d\tau
$$
 $$
 \le c(\alpha,r)t^{\frac{r-\alpha}{2}+\frac12}\frac{1}{t}\theta_n^{\frac{\alpha}{2}+\frac14}\theta_n^{-\alpha+\frac{r}{2}-\frac34}
 $$
 \begin{equation}\label{sob-lag-sb-7.77}
= c(\alpha,r)t^{\frac{r-\alpha}{2}-\frac12}\theta_n^{\frac{r-\alpha}{2}-\frac12}\le c(\alpha,r)(t\theta_n)^{\frac{r-\alpha}{2}-\frac12},
\end{equation}
$$
J_{12}^2\le c(\alpha,r)nt^{\frac{r}{2}+\frac14}\theta_n^{-\frac34}t^{-\frac{\alpha}{2}+\frac14}
\int\limits_{1/\theta_n}^{t-\sqrt{t/\theta_n}}
{\tau^{\alpha-\frac{r}{2}-\frac14}\over t-\tau}\theta_n^{-\frac{1}{4}}\tau^{-\frac{\alpha}{2}-\frac14}d\tau
$$
$$
\le c(\alpha,r)t^{\frac{r-\alpha}{2}+\frac12}\int\limits_{1/\theta_n}^{t-\sqrt{t/\theta_n}}
{\tau^{\frac{\alpha-r}{2}-\frac12}\over t-\tau}d\tau
= c(\alpha,r)\int\limits_{1/(t\theta_n)}^{1-\sqrt{1/(t\theta_n)}}{x^{\frac{\alpha-r}{2}-\frac12}\over 1-x}dx
$$
$$
\le c(\alpha,r)\int\limits_{1/(t\theta_n)}^{1/3}x^{\frac{\alpha-r}{2}-\frac12}dx+
c(\alpha,r)\int\limits_{1/3}^{1-\sqrt{1/(t\theta_n)}}{dx\over 1-x}
$$
 \begin{equation}\label{sob-lag-sb-7.78}
\le c(\alpha,r)\left(1+\ln\left(\frac23\sqrt{t\theta_n}\right)\right).
\end{equation}
Из \eqref{sob-lag-sb-7.75}, \eqref{sob-lag-sb-7.77} и \eqref{sob-lag-sb-7.78} находим
 \begin{equation}\label{sob-lag-sb-7.79}
J_{12}\le c(\alpha,r)\left(1+\ln\left(\sqrt{t\theta_n}\right)\right).
\end{equation}

 Переходя к оценке $J_{13}$, запишем
 \begin{equation}\label{sob-lag-sb-7.80}
J_{13}=J_{13}^1+J_{13}^2,
\end{equation}
в котором ($k=1,2$)
\begin{equation}\label{sob-lag-sb-7.81}
J_{13}^k=nt^{\frac{r}{2}+\frac14}e^{-\frac{t}{2}}|\hat L_{n-r}^\alpha(t)|\int_{Q_1^k}{\tau^{\alpha-\frac{r}{2}-\frac14}e^{-\frac{\tau}{2}}\over |\tau-t|}|\hat L_{n-r+1}^\alpha(\tau)-\hat L_{n-r-1}^\alpha(\tau)|d\tau.
\end{equation}
 Из  \eqref{sob-lag-sb-2.13} и \eqref{sob-lag-sb-7.81} имеем
$$
J_{13}^1\le c(\alpha,r)nt^{\frac{r}{2}+\frac14}\theta_n^{-\frac14}t^{-\frac{\alpha}{2}-\frac14}\int_0^{1/\theta_n}
{\tau^{\alpha-\frac{r}{2}-\frac14}\over t-\tau}\theta_n^{\frac{\alpha}{2}-1}d\tau
$$
\begin{equation}\label{sob-lag-sb-7.82}
\le c(\alpha,r)t^\frac{r-\alpha}{2}\theta_n^{\frac{\alpha}{2}-\frac14}\frac{1}{t}\int_0^{1/\theta_n}
\tau^{\alpha-\frac{r}{2}-\frac14}d\tau
\le c(\alpha,r)(t\theta_n)^{\frac{r-\alpha}{2}-1},
\end{equation}
$$
J_{13}^2\le c(\alpha,r)nt^{\frac{r}{2}+\frac14}\theta_n^{-\frac14}t^{-\frac{\alpha}{2}-\frac14}
\int\limits_{1/\theta_n}^{t-\sqrt{t/\theta_n}}
{\tau^{\alpha-\frac{r}{2}-\frac14}\over t-\tau}\theta_n^{-\frac{3}{4}}\tau^{-\frac{\alpha}{2}+\frac14}d\tau
$$
$$
\le c(\alpha,r)t^{\frac{r-\alpha}{2}}\int\limits_{1/\theta_n}^{t-\sqrt{t/\theta_n}}
{\tau^{\frac{\alpha-r}{2}}\over t-\tau}d\tau=c(\alpha,r)\int\limits_{1/(t\theta_n)}^{1-\sqrt{1/(t\theta_n)}}
{x^{\frac{\alpha-r}{2}}\over 1-x}dx
$$
\begin{equation}\label{sob-lag-sb-7.83}
\le c(\alpha,r)\left(1+\ln\left(\sqrt{t\theta_n}\right)\right),
\end{equation}

Из  \eqref{sob-lag-sb-7.80} -- \eqref{sob-lag-sb-7.83} получаем
\begin{equation}\label{sob-lag-sb-7.84}
J_{13}\le c(\alpha,r)\left(1+\ln\left(\sqrt{t\theta_n}\right)\right),
\end{equation}
Оценки  \eqref{sob-lag-sb-7.66},  \eqref{sob-lag-sb-7.74}, \eqref{sob-lag-sb-7.79}, \eqref{sob-lag-sb-7.84}, взятые вместе дают
\begin{equation}\label{sob-lag-sb-7.85}
J_{1}\le c(\alpha,r)\left(1+\ln\left(t\theta_n\right)\right).
\end{equation}
Собирая оценки \eqref{sob-lag-sb-7.27}, \eqref{sob-lag-sb-7.42}, \eqref{sob-lag-sb-7.65} и \eqref{sob-lag-sb-7.85}, мы приходим к следующему  неравенству
$$
\lambda_{r,n}^\alpha\le c(\alpha,r)\left[\ln\sqrt{t\theta_n} +\left(\frac{\theta_n}{t}\right)^\frac{\alpha-r}{2}
\ln\frac{\theta_n/t-1}{\theta_n/(2t)+\sqrt{1/(t\theta_n)}-1}\right]
$$
\begin{equation}\label{sob-lag-sb-7.86}
\le c(\alpha,r)[\ln(n+1)+(n/t)^\frac{\alpha-r}{2}]\quad (t\in G_2).
\end{equation}
Тем самым оценка \eqref{sob-lag-sb-6.11} доказана.

Докажем \eqref{sob-lag-sb-6.12}. Пусть $t\in G_3=[\frac{1}{2}\theta_n,\frac32\theta_n]$. Воспользуемся представлением \eqref{sob-lag-sb-7.27} и оценим $J_k$ $(1\le k\le3)$. Что касается величины  $J_2$, то для нее верна оценка \eqref{sob-lag-sb-7.42}. Поэтому остается оценить $J_k$  для $k=1$  и  $k=3$. Для $k=3$ мы воспользуемся оценкой \eqref{sob-lag-sb-7.43}. Чтобы оценить величину $J_{31}$ представим ее в следующем виде
\begin{equation}\label{sob-lag-sb-7.87}
J_{31}=J_{311}+J_{312},
\end{equation}
в котором $(k=1,2)$
  \begin{equation}\label{sob-lag-sb-7.88}
J_{31k}=t^{\frac{r}{2}+\frac14}e^{-t/2}|\hat L_{n-r}^\alpha(t)|\int_{Q_{31k}}\tau^{\alpha-\frac{r}{2}-\frac14}e^{-\tau/2}|\hat L_{n-r}^\alpha(\tau)|d\tau,
\end{equation}
где $Q_{311}=(t+\sqrt{t/\theta_n},3\theta_n/2+\sqrt{t/\theta_n})$,  $Q_{312}=(3\theta_n/2+\sqrt{t/\theta_n},\infty)$.
Обратимся к  неравенству \eqref{sob-lag-sb-2.13}, тогда из  \eqref{sob-lag-sb-7.88} находим
$$
J_{311}\le c(\alpha,r){\theta_n^{-\frac12-\alpha}t^{\frac{r}{2}+\frac14}\over(\theta_n^{1/3}+|t-\theta_n|)^{1/4}}
\int_{Q_{311}}{\tau^{\alpha-\frac{r}{2}-\frac14}d\tau\over(\theta_n^{1/3}+|\tau-\theta_n|)^{1/4}}
$$
$$
\le {c(\alpha,r)\theta_n^{-\frac12}\over(\theta_n^{1/3}+|t-\theta_n|)^{1/4}}
\int_{Q_{311}}{d\tau\over(\theta_n^{1/3}+|\tau-\theta_n|)^{1/4}}
$$
$$
\le {c(\alpha,r)\theta_n^{-\frac12}\over(\theta_n^{1/3}+|t-\theta_n|)^{1/4}}
\int\limits_{\theta_n}^{\frac32\theta_n+\sqrt{t/\theta_n}}{d\tau\over(\tau+ \theta_n^{1/3}-\theta_n)^{1/4}}
$$
\begin{equation}\label{sob-lag-sb-7.89}
\le c(\alpha,r)\left({\theta_n\over\theta_n^{1/3}+|t-\theta_n|}\right)^\frac14,
\end{equation}
$$
J_{312}\le c(\alpha,r){\theta_n^{-\alpha}t^{\frac{r}{2}+\frac14}\over(\theta_n^{1/3}+|t-\theta_n|)^{1/4}}
\int_{Q_{312}}\tau^{\alpha-\frac{r}{2}-\frac14}e^{-\tau/4}d\tau
$$
$$
\le {c(\alpha,r)\over(\theta_n^{1/3}+|t-\theta_n|)^{1/4}}
\int\limits_{3\theta_n/2+\sqrt{t/\theta_n}}^\infty(\tau/\theta_n)^{\alpha}e^{-\tau/4}d\tau\le $$
$$
c(\alpha,r)
\int\limits_{3\theta_n/2}^\infty(\tau/\theta_n)^{\alpha}e^{-\tau/4}d\tau=
c(\alpha,r)\theta_n
\int\limits_{3/2}^\infty\tau^{\alpha}e^{-\theta_n\tau/4}d\tau\le
$$
\begin{equation}\label{sob-lag-sb-7.90}
c(\alpha,r)\theta_ne^{-3\theta_n/16}
\int\limits_{3/2}^\infty\tau^{\alpha}e^{-\theta_n\tau/8}d\tau\le c(\alpha,r).
\end{equation}
Из \eqref{sob-lag-sb-7.87} -- \eqref{sob-lag-sb-7.90} выводим
\begin{equation}\label{sob-lag-sb-7.91}
J_{31}\le c(\alpha,r)\left({\theta_n\over\theta_n^{1/3}+|t-\theta_n|}\right)^\frac14\quad (t\in G_3, \quad r-\frac12<\alpha<r+\frac12).
\end{equation}

Переходя к оценке величины $J_{32}$, представим ее в виде
\begin{equation}\label{sob-lag-sb-7.92}
J_{32}=J_{321}+J_{322},
\end{equation}
в котором $(k=1,2)$
\begin{equation}\label{sob-lag-sb-7.93}
J_{32k}=nt^{\frac{r}{2}+\frac14}e^{-\frac{t}{2}}|\hat L_{n-r+1}^\alpha(t)-\hat L_{n-r-1}^\alpha(t)|\int_{Q_{31k}}{\tau^{\alpha-\frac{r}{2}-\frac14}e^{-\frac{\tau}{2}}\over \tau-t}|\hat L_{n-r}^\alpha(\tau)|d\tau,
\end{equation}
 Обратимся  к неравенству \eqref{sob-lag-sb-2.14}, тогда из  \eqref{sob-lag-sb-7.93} находим
$$
J_{321}\le c(\alpha,r)nt^{\frac{r}{2}+\frac14}t^{-\frac\alpha2}\theta_n^{-\frac34}[\theta_n^{1/3}+|t-\theta_n|]^\frac14\times $$
$$
\int\limits_{Q_{311}}{\tau^{\alpha-\frac{r}{2}-\frac14}\theta_n^{-\frac\alpha2}[\theta_n(\theta_n^{\frac13}+|\tau-\theta_n|)]^{-\frac14}\over \tau-t}d\tau\le
$$
\begin{equation}\label{sob-lag-sb-7.94}
c(\alpha,r)\int\limits_{t+\sqrt{t/\theta_n}}^{\frac{3}{2}\theta_n+\sqrt{t/\theta_n}}{[\theta_n^{1/3}+|t-\theta_n|]^\frac14\over
[\theta_n^{1/3}+|\tau-\theta_n|]^\frac14}
{d\tau\over \tau-t}\le c(\alpha,r)\ln(n+1).
\end{equation}
Чтобы убедиться в справедливости \eqref{sob-lag-sb-7.94} покажем, что
$$
A=\int\limits_{t+\sqrt{t/\theta_n}}^{\frac{3}{2}\theta_n+\sqrt{t/\theta_n}}{[\theta_n^{1/3}+|t-\theta_n|]^\frac14\over
[\theta_n^{1/3}+|\tau-\theta_n|]^\frac14} {d\tau\over \tau-t}\le c(\alpha)\ln(n+1).
$$
С этой целью рассмотрим два случая: $1)\, \theta_n/2\le t\le \theta_n-2\theta_n^\frac13$; $2)\, \theta_n-2\theta_n^\frac13\le t\le 3\theta_n/2$. Во-втором из этих случаев имеем
 $$
{\theta_n^{1/3}+|t-\theta_n|\over \theta_n^{1/3}+|\tau-\theta_n|}\le 3 \quad (\theta_n-2\theta_n^\frac13\le t\le \tau),
$$
поэтому
$$
A\le 3 \ln{\frac{3}{2}\theta_n+\sqrt{t/\theta_n}-t\over\sqrt{t/\theta_n} }\le
3\ln \left(1+{\frac{3}{2}\theta_n-t\over\sqrt{t/\theta_n}} \right)\le c(\alpha)\ln(n+1).
$$
Если же $\theta_n/2\le t\le \theta_n-2\theta_n^\frac13$, то мы можем записать
$$
A=\int\limits_{t+\sqrt{t/\theta_n}}^{\theta_n-\theta_n^\frac13+\sqrt{t/\theta_n}}+
\int\limits_{\theta_n-\theta_n^\frac13+\sqrt{t/\theta_n}}^{\theta_n+\theta_n^\frac13+
\sqrt{t/\theta_n}}+
\int\limits_{\theta_n+\theta_n^\frac13+\sqrt{t/\theta_n}}^{3\theta_n/2+\sqrt{t/\theta_n}}=A_1+A_2+A_3.
$$
Поскольку в рассматриваемом случае $2\theta_n^\frac13\le\theta_n-t$, то
$$
A_2\le\int\limits_{\theta_n-\theta_n^\frac13+\sqrt{t/\theta_n}}^{\theta_n+
\theta_n^\frac13+\sqrt{t/\theta_n}}
{[\frac32(\theta_n-t)]^\frac14\over[\theta_n^{1/3}+|\tau-\theta_n|]^\frac14} {d\tau\over \tau-t}\le
$$
$$
\left[\frac32(\theta_n-t)\right]^\frac14\theta_n^{-\frac1{12}}\ln {\theta_n+\theta_n^\frac13+\sqrt{t/\theta_n}-t\over\theta_n-\theta_n^\frac13+
\sqrt{t/\theta_n}-t}=
$$
$$
\left[\frac32(\theta_n-t)\right]^\frac14\theta_n^{-\frac1{12}}\ln\left(1+ {2\theta_n^\frac13\over\theta_n-\theta_n^\frac13+\sqrt{t/\theta_n}-t}\right)\le
$$
$$
\left[\frac32(\theta_n-t)\right]^\frac14\theta_n^{-\frac1{12}}
{2\theta_n^\frac13\over\theta_n-t-\theta_n^\frac13}\le
\left[\frac32(\theta_n-t)\right]^\frac14{2\theta_n^\frac14\over\theta_n-t-\frac12(\theta_n-t)}
\le
$$
$$
\left(\frac32(\theta_n-t)\right)^\frac14{4\theta_n^\frac14\over\theta_n-t}\le 4\left(\frac32\right)^\frac14
\left({\theta_n\over(\theta_n-t)^3}\right)^\frac14\le c\left({\theta_n\over(2\theta_n^\frac13)^3}\right)^\frac14\le c,
$$
$$
A_1\le\int\limits_{t+\sqrt{t/\theta_n}}^{\theta_n-\theta_n^\frac13+\sqrt{t/\theta_n}}
{[\frac32(\theta_n-t)]^\frac14\over[\theta_n^{1/3}+|\tau-\theta_n|]^\frac14} {d\tau\over \tau-t}\le
$$
$$
\le\int\limits_{t+\sqrt{t/\theta_n}}^{\theta_n-\theta_n^\frac13+\sqrt{t/\theta_n}}
{[\frac32(\theta_n-t)]^\frac14\over(\theta_n-\tau)^\frac14} {d\tau\over \tau-t}\le
c\int\limits_{t+\sqrt{t/\theta_n}}^{\theta_n-\theta_n^\frac13+\sqrt{t/\theta_n}}
{-d{\theta_n-\tau\over\theta_n-t}
\over\left(1-{\theta_n-\tau\over\theta_n-t}\right)
\left({\theta_n-\tau\over\theta_n-t}\right)^\frac14}
$$
$$
\le c\int\limits^{\theta_n-t-\sqrt{t/\theta_n}\over
\theta_n-t}_{\theta_n^\frac13-\sqrt{t/\theta_n}\over\theta_n-t}{d\tau\over(1-\tau)\tau^{1/4}}
\le c(\alpha)\ln(n+1),
$$
$$
A_3\le\int\limits^{3\theta_n/2+\sqrt{t/\theta_n}}_{\theta_n+\theta_n^\frac13+\sqrt{t/\theta_n}}
{[\frac32(\theta_n-t)]^\frac14\over[\theta_n^{1/3}+|\tau-\theta_n|]^\frac14} {d\tau\over \tau-t}\le
$$
$$
c\int\limits^{3\theta_n/2+\sqrt{t/\theta_n}}_{\theta_n+\theta_n^\frac13+\sqrt{t/\theta_n}}
{(\tau-t)^\frac14d\tau\over(\tau-t)(\tau-\theta_n)^\frac14}\le
c\int\limits^{3\theta_n/2+\sqrt{t/\theta_n}}_{\theta_n+\theta_n^\frac13+\sqrt{t/\theta_n}}
{d\tau\over(\tau-t)^{\frac34}(\tau-\theta_n)^\frac14}\le
$$
$$
c\int\limits^{3\theta_n/2+
\sqrt{t/\theta_n}}_{\theta_n+\theta_n^\frac13+\sqrt{t/\theta_n}}
{d\tau\over(\tau-\theta_n)}\le c(\alpha)\ln(n+1).
$$
Из полученных оценок для $A_i$ $(i=1,2,3)$ вытекает, что $A\le c(\alpha)\ln(n+1)$ и тем самым доказана справедливость оценки \eqref{sob-lag-sb-7.94}.

Далее из \eqref{sob-lag-sb-7.93} и неравенства  \eqref{sob-lag-sb-2.14} имеем ($t\in G_3$)
$$
J_{322}\le c(\alpha,r)nt^{\frac{r}{2}+\frac14}t^{-\frac\alpha2}\theta_n^{-\frac34}[\theta_n^{1/3}+|t-\theta_n|]^\frac14
\int_{Q_{312}}{\tau^{\alpha-\frac{r}{2}-\frac14}e^{-\frac\tau4}d\tau\over\theta_n^{\frac{\alpha}2} (\tau-t)}
$$
$$
 \le c(\alpha,r)\theta_n^{\frac14-\alpha}[\theta_n^{1/3}+
 |t-\theta_n|]^\frac14\int\limits_{\frac{3}{2}\theta_n+\sqrt{t/\theta_n}}^\infty
 {\tau^{\alpha}e^{-\frac\tau4}d\tau\over \tau-t}\le
 $$
  $$
  c(\alpha,r)\theta_n^{\frac14-\alpha}[\theta_n^{1/3}+
 |t-\theta_n|]^\frac14\int\limits_{\frac{3}{2}\theta_n+\sqrt{3/2}}^\infty
 \tau^{\alpha}e^{-\frac\tau4}d\tau\le
 $$
 \begin{equation}\label{sob-lag-sb-7.95}
  c(\alpha,r) \int\limits_{\frac{3}{2}\theta_n}^\infty
 \tau^{\alpha}e^{-\frac\tau4}d\tau   \le c(\alpha,r)\quad (r-\frac12<\alpha<r+\frac12).
 \end{equation}
 Из \eqref{sob-lag-sb-7.92} -- \eqref{sob-lag-sb-7.95} мы замечаем, что
\begin{equation}\label{sob-lag-sb-7.96}
J_{32}\le c(\alpha,r)\ln(n+1).
\end{equation}
Комбинируя методы, которые привели к оценкам \eqref{sob-lag-sb-7.91} и  \eqref{sob-lag-sb-7.96},  нетрудно доказать также, что из \eqref{sob-lag-sb-7.46} вытекает оценка $(r-\frac12<\alpha<r+\frac12)$
\begin{equation}\label{sob-lag-sb-7.97}
J_{33}\le c(\alpha,r)\left[\ln(n+1)+\left({\theta_n\over\theta_n^{1/3}+|t-\theta_n|}\right)^\frac14\right]\quad (t\in G_3).
\end{equation}
Сопоставляя оценки \eqref{sob-lag-sb-7.91}, \eqref{sob-lag-sb-7.96}, \eqref{sob-lag-sb-7.97} с \eqref{sob-lag-sb-7.43}, имеем $(r-\frac12<\alpha<r+\frac12)$

\begin{equation}\label{sob-lag-sb-7.98}
J_{3}\le c(\alpha,r)\left[\ln(n+1)+\left({\theta_n\over\theta_n^{1/3}+|t-\theta_n|}\right)^\frac14\right]\quad (t\in G_3).
\end{equation}

Наконец, почти дословно повторяя рассуждения, которые привели нас к оценке \eqref{sob-lag-sb-7.98}, из равенства \eqref{sob-lag-sb-7.28} можно доказать, что

\begin{equation}\label{sob-lag-sb-7.99}
J_{1}\le c(\alpha,r)\left[\ln(n+1)+\left({\theta_n\over\theta_n^{1/3}+|t-\theta_n|}\right)^\frac14\right]\quad (t\in G_3, \quad r-\frac12<\alpha<r+\frac12).
\end{equation}

Объединяя оценки \eqref{sob-lag-sb-7.98},  \eqref{sob-lag-sb-7.99}, \eqref{sob-lag-sb-7.42} и  сопоставляя их с равенством \eqref{sob-lag-sb-7.27}, приходим к \eqref{sob-lag-sb-6.12}.

Нам осталось  доказать  оценку \eqref{sob-lag-sb-6.13}. С этой целью мы обратимся непосредственно к равенству \eqref{sob-lag-sb-6.8}, из которого находим

  $$
\lambda_{r,n}^\alpha(t)\le t^{\frac r2+\frac14}(e^{-t}\mathcal{K}_{n-r}^\alpha(t,t))^{\frac12}\int\limits_0^\infty \tau^{\alpha-\frac r2-\frac14}(e^{-\tau}\mathcal{K}_{n-r}^\alpha(\tau,\tau))^\frac12d\tau.
$$
Положим $(k=1,2,3)$
$$
I_k=\int_{E_k}\tau^{\alpha-\frac r2-\frac14}(e^{-\tau}\mathcal{K}_{n-r}^\alpha(\tau,\tau))^\frac12d\tau,
$$
где $E_1=[0,\frac1{\theta_n}]$, $E_2=[\frac1{\theta_n},3\theta_n/2]$, $E_3=[3\theta_n/2,\infty)$. Тогда
$$
\lambda_{r,n}^\alpha(t)\le t^{\frac r2+\frac14}(e^{-t}\mathcal{K}_{n-r}^\alpha(t,t))^{\frac12}(I_1+I_2+I_3).
$$
Из \eqref{sob-lag-sb-2.15} для $t\ge 3\theta_n/2$   имеем
$$
t^{\frac r2+\frac14}(e^{-t}\mathcal{K}_{n-r}^\alpha(t,t))^{\frac12}\le c(\alpha,r)n^\frac{1-\alpha}{2}t^{\frac r2+\frac14}e^{-t/4}.
$$
Кроме того
$$
I_1=\int_{E_1}\tau^{\alpha-\frac r2-\frac14}(e^{-\tau}\mathcal{K}_{n-r}^\alpha(\tau,\tau))^\frac12d\tau\le c(\alpha,r)n^{\frac{r-\alpha}{2}-\frac14}
$$
и, в силу леммы 7.1
$$
I_2=\int_{E_2}\tau^{\alpha-\frac r2-\frac14}(e^{-\tau}\mathcal{K}_{n-r}^\alpha(\tau,\tau))^\frac12d\tau\le
$$
$$
c(\alpha,r)n^\frac14\int_{E_2}\tau^{\alpha-\frac r2-\frac14}\tau^{-\frac{\alpha}{2}-\frac14}d\tau
\le c(\alpha,r)n^{\frac{\alpha-r}{2}+\frac34}.
$$
Наконец,
$$
I_3=\int_{E_3}\tau^{\alpha-\frac r2-\frac14}(e^{-\tau}\mathcal{K}_{n-r}^\alpha(\tau,\tau))^\frac12d\tau\le
$$
$$
\le c(\alpha,r)n^\frac{1-\alpha}{2}\int_{E_3}\tau^{\alpha-\frac r2-\frac14}e^{-\tau/4}d\tau \le c(\alpha,r).
$$
Собирая полученные оценки, находим
$
\lambda_{r,n}^\alpha(t)\le c(\alpha,r)n^{-\frac{r}{2}+\frac54}t^{\frac r2+\frac14}e^{-\frac{t}{4}}.
$
Тем самым оценка \eqref{sob-lag-sb-6.13} доказана.