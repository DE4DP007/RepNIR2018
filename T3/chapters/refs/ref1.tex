
Исследована задача о сходимости сумм Фурье по ультрасферическим полиномам Якоби в простанстве Лебега с переменным показателем. Найдены необходимые и достаточные условия на переменный показатель $p(x)>1$, которые гарантируют равномерную ограниченность последовательности  сумм Фурье $S_n^{\alpha,\alpha}(f)$ ($n=0,1,\ldots$) по ультрасферическим полиномам Якоби $P_k^{\alpha,\alpha}(x)$
в весовом пространстве Лебега $L_\mu^{p(x)}([-1,1])$ с весом $\mu=\mu(x)=(1-x^2)^\alpha$, где $\alpha>-1/2$. Случай $\alpha=-1/2$ рассмотрен отдельно. Показано, что для равномерной ограниченности  последовательности  сумм Фурье-Чебышева $S_n^{-\frac12,-\frac12}(f)$ ($n=0,1,\ldots$) в пространстве $L_\mu^{p(x)}([-1,1])$, где $\mu=\mu(x)=(1-x^2)^{-\frac12}$ достаточно и, в определенном смысле,  необходимо, чтобы переменный показатель $p$ подчинялся условию Дини -- Липшица  вида $|p(x)-p(y)|\le \frac{d}{-\ln|x-y|}$, где $|x-y|\le\frac12$, $x,y\in[-1,1]$, $d>0$, а также условию $p(x)>1$ для всех $x\in[-1,1]$.
Аналогичная задача решена для общих полиномов Якоби $P_k^{\alpha,\beta}(x)$ при $\alpha,\beta>-1/2$.



Исследованы вопросы, связанные с прямыми и обратными теоремами теории приближений функций в весовых пространствах Лебега  с переменным показателем. В частности, показано, что семейство сдвигов функции Стеклова вида
$$s_{\lambda,\tau}(f)(x) = \lambda \int_{x-\frac{1}{2\lambda}+\tau}^{x+\frac{1}{2\lambda}+\tau} f(t)dt.$$
равномерно ограничено в весовых пространствах Лебега с переменным показателем $L_{2\pi, w}^{p(x)}$, где $w = w(x)$ --- весовая функция, удовлетворяющая аналогу известного условия Макенхоупта.
Этот результат может быть использован при конструировании модуля гладкости в весовом пространстве Лебега с переменным показателем и доказательстве прямых теорем в этих пространствах.
%будет использоваться при получении оценки величины наилучшего приближения функции через наилучшее приближение ее производной.






На основе тригонометрических сумм Фурье $S_n(f,x)$ и классических средних Валле Пуссена
$$
_1V_{n,m}(f,x)= \frac1n\sum\nolimits_{l=m}^{m+n-1}S_l(f,x)
$$
 в отчетном году сконструированы повторные средние Валле Пуссена ${}_2 V_{n,m}(f,x)$ и более общие средние Валле Пуссена ${}_{l+1}V_{n,m}(f,x)$
 следующим образом
 $$
_2V_{n,m}(f,x)= \frac1n\sum\nolimits_{k=m}^{m+n-1}{}_1V_{n,k}(f,x),
$$
$$
{}_{l+1}V_{n,m}(f,x)= \frac1n\sum\nolimits_{k=m}^{m+n-1} {}_{l}V_{n,k}(f,x)\quad(l\ge1).
$$
Исследованы аппроксимативные свойства повторных средних ${}_2 V_{n,m}(f,x)$ на классах кусочно-гладких периодических функций. Показано, что локальные аппроксимативные свойства повторных средних ${}_2V_{n,m}(f,x)$ на порядок лучше чем у классических средних Валле Пуссена и на два порядка --- чем у сумм Фурье.

Отдельно исследованы локальные аппроксимативные свойства повторных средних Валле Пуссена ${}_2 V_{n,m}(f,x)$ для кусочно-гладких непрерывных непериодических функций. На основе средних ${}_2 V_{n,m}(f,x)$ и перекрывающих преобразований сконструированы операторы, осуществляющие   приближения непрерывных (вообще говоря, непериодических) функций и исследованы их аппроксимативные свойства.




%%%%%%%%%%%%%%%%%%%%%%%%%%%%%%
%%%%%%%%%%%%%%%%%%%%%%%%%%%%%%


Получены оценки скорости сходимости сплайнов по трехточечным рациональным интерполянтам через различные структурные характеристики приближаемой функции:
\begin{itemize}
  \item
  в случае равномерных сеток узлов – через модуль непрерывности третьего порядка;
  \item
  для непрерывно дифференцируемых функций с выбором узлов сетки – через вариацию и через модуль изменения производных первого и второго порядков.
\end{itemize}

Для произвольных сеток узлов построены гладкие сплайны по трехточечным рациональным интерполянтам, полюсы которых зависят от узлов и одного свободного параметра. Доказано, что последовательности таких сплайнов и последовательности их производных первого и второго порядков обладают свойством безусловной сходимости соответственно в классах непрерывных функций и функций с непрерывными производными первого и второго порядков. Получены оценки скорости сходимости относительно расстояний между узлами.

Для непрерывных и непрерывно дифференцируемых функций (до второго порядка) через модуль непрерывности и модуль изменения получены оценки скорости сходимости сплайнов по трехточечным рациональным интерполянтам. Найдены условия на сетки узлов для отсутствия и для наличия явления Гиббса при приближении сплайнами по трехточечным рациональным интерполянтам функций, непрерывных на данном отрезке, кроме точки разрыва первого рода со скачком. Получены условия выпуклой интерполяции сплайнами по трехточечным рациональным интерполянтам.


%%%%%%%%%%%%%%%%%%%%%%%%%%
%%%%%%%%%%%%%%%%%%%%%%%%%%
%%%%%%%%%%%%%%%%%%%%%%%%%%



Изучены аппроксимативные свойства частичных сумм ряда Фурье по модифицированным полиномам Мейкснера $M_{n,N}^\alpha(x)=M_n^\alpha(Nx)$ $(n=0, 1, \dots)$, которые при $\alpha>-1$ образуют ортогональную систему на сетке $\Omega_{\delta}=\{0, \delta, 2\delta, \ldots\}$, где $\delta=\frac{1}{N}$, $N>0$ с весом $w(x)=e^{-x}\frac{\Gamma(Nx+\alpha+1)}{\Gamma(Nx+1)}$. Основное внимание уделено получению верхней оценки для функции Лебега указанных частичных сумм.






%%%%%%%%%%%%%%%%%%%%%%%%%%
%%%%%%%%%%%%%%%%%%%%%%%%%%
%%%%%%%%%%%%%%%%%%%%%%%%%%

%
%
%Исследованы аппроксимативные свойства частичных сумм специальных рядов для \linebreak функций из пространств Соболева с переменным показателем. Полученные результаты показывают, что применение пространств с переменным показателем позволяет учитывать существенно переменное поведение производных аппроксимируемой функции при оценке точности приближения заданной гладкой функции частичными суммами специального ряда.
%
%Получена оценка скорости приближения функций из пространств Лебега и Соболева с переменным показателем средними Валле Пуссена тригонометрических сумм Фурье в метрике этих пространств.
%
%Хорошо известно, что имеет место равномерная сходимость полиномов Бернштейна в пространстве $C[0,1]$. Однако для суммируемой функции, вообще говоря, это не так. Кроме того, не для всякой суммируемой функции можно определить полином Бернштейна. Введенные Л.В. Канторовичем в работе \cite{shtn1} полиномы представляют собой аналог полиномов Бернштейна для более широкого класса суммируемых функций.  В отчетном году исследовались вопросы сходимости операторов Бернштейна -- Канторовича в пространствах Лебега с переменным показателем. Получены условия на переменный показатель, при которых последовательность вышеназванных операторов равномерно ограничена в пространствах Лебега с переменным показателем.
%
%Исследован вопрос о равномерной ограниченности семейства сдвигов функций Стеклова в весовых пространствах Лебега с переменным показателем. Получены условия на вес, при которых будет иметь место равномерная ограниченность упомянутого семейства.






