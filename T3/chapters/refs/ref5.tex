Широко известно о существовании трудностей численной реализации обращения изображения по Лапласу. Одним из примеров является изменение граничного условия в автомодельной задаче извлечения сухого тепла пород.
В текущем году показано, что если температура на границе нарастает с начального значения по экспоненциальному закону к некоторому постоянному значению, то на отрицательной действительной оси параметра преобразования появляется полюс, интегралы по отрицательной действительной оси становятся несобственными, и растут погрешности вычислений. Предложено решение новой задачи о росте температуры термальной воды, фонтанирующей во времени с постоянной скоростью.
Эта задача является контактной и  включает как температуру горной породы с геотермическим градиентом, так и температуру жидкости в трещине. Она смоделирована по схеме Ловерье, решена в нестационарной постановке с применением преобразования Лапласа и приведена к универсальному виду в безразмерных переменных. При этом отмечаются особенности, вносимые геотермальным градиентом и температурным фронтом воды в процедуру обращения изображения по Лапласу.
