\section{Численное дифференцирование с помощью специальных операторов со свойством <<прилипания>> на основе полиномов Чебышева. Задача идентификации}\label{sect-identification}\label{sect-2.2}
\textit{
На основе полиномов Чебышева, ортогональных на равномерной сетке, сконструированы специальные полиномиальные операторы, которые могут быть эффективно использованы как для «сглаживания» ошибок в наблюдениях исходного сигнала, так и для решения задачи одновременного приближения дифференцируемой функции и нескольких ее производных. Изучены аппроксимативные свойства указанных операторов, получены оценки функции Лебега.
С помощью указанных выше специальных операторов исследована задача об идентификации параметров линейной неинвариантной (по времени) системы в случае, когда входной и выходной сигналы заданы в узлах равномерной сетки. Построены алгоритмы численной реализации этих операторов с помощью быстрых дискретных преобразований, основанных на полиномах Чебышева I-го и II-го рода.}





\subsection{Постановка задачи}

\textit{ Рассмотрена задача об идентификации параметров линейной неинвариантной (по времени) системы вида
\begin{equation*}
x^{(r)}(t)=\sum_{\nu=0}^{r-1}a_\nu(t)x^{(\nu)}(t)+\sum_{\mu=0}^s b_\mu(t)y^{(\mu)}(t),
\end{equation*}
где неизвестные переменные коэффициенты $a_\nu(t)$ $(\nu=0,\ldots,r-1)$ и $b_\mu(t)$ $(\mu=0,\ldots,s)$ представляют собой алгебраические полиномы  заданной степени $m$. Ставится задача  найти неизвестные переменные коэффициенты $a_\nu(t)$ $(\nu=0,\ldots,r-1)$ и $b_\mu(t)$ $(\mu=0,\ldots,s)$ экспериментальным путем.
Нами рассмотрен часто встречающийся на практике случай, когда заданы значения сигналов $x(t)$ и  $y(t)$ в узлах равномерной сетки $\Omega_N=\{t_j=-1+jh\}_{j=0}^{N-1}$, где $h=\frac2{N-1}$.
Для решения этой задачи на основе полиномов Чебышева, ортогональных на равномерной сетке, сконструированы полиномиальные операторы, которые могут быть эффективно использованы как для <<сглаживания>> ошибок в наблюдениях исходного сигнала $f(t)$ в узлах сетки $\Omega_N$, так и для решения задачи одновременного приближения дифференцируемой функции $f(t)$ и ее нескольких производных. При численной реализации этих операторов используются быстрые дискретные преобразования, основанные на полиномах Чебышева I рода $C_k(t) = \cos (k \arccos t)$.
}

Рассмотрим линейную систему, у которой выходной сигнал $x=x(t)$ и входной сигнал  $y=y(t)$
связаны между собой  равенством
\begin{equation}\label{idprmeq1.2.21.1}
x^{(r)}(t)=\sum_{\nu=0}^{r-1}a_\nu(t)x^{(\nu)}(t)+\sum_{\mu=0}^s b_\mu(t)y^{(\mu)}(t),
\end{equation}
где неизвестные переменные коэффициенты $a_\nu(t)$ $(\nu=0,\ldots,r-1)$ и $b_\mu(t)$ $(\mu=0,\ldots,s)$ представляют собой алгебраические полиномы  заданной степени $m$. Будем  считать, что функции $x=x(t)$ и  $y=y(t)$ заданы на отрезке $[-1,1]$ и непрерывно дифференцируемы там $r$-раз и $s$-раз соответственно. Ставится задача  найти неизвестные переменные коэффициенты
$a_\nu(t)$ $(\nu=0,\ldots,r-1)$ и $b_\mu(t)$ $(\mu=0,\ldots,s)$ экспериментальным путем. Такую задачу часто называют  \textit{идентификацией} параметров системы. Методы и подходы к решению этой задачи существенно зависят от того, что именно мы знаем о входном и выходном сигналах  $x=x(t)$ и  $y=y(t)$. В настоящей работе рассматривается часто встречающийся на практике случай, когда заданы значения сигналов $x(t)$ и  $y(t)$ в узлах равномерной сетки $\Omega_N=\{t_j=-1+jh\}_{j=0}^{N-1}$, где $h=\frac2{N-1}$.
Будем исходить из предположения, что значения $x(t_j)$ и  $y(t_j)$ получены экспериментально в результате наблюдений и, следовательно, <<зашумлены>>. Другими словами, вместо точных значений сигналов, которые мы обозначим через $\tilde x(t_j)$ и  $\tilde y(t_j)$,   нам заданы их приближения $ x_j=x(t_j)=\tilde x(t_j)+\eta_j$ и  $ y_j=y(t_j)=\tilde y(t_j)+\xi_j $, где $\eta_j$ и $\xi_j$ -- случайные погрешности наблюдений, которые удовлетворяют условиям
\begin{equation}\label{idprmeq1.2.21.2}
E[\eta_i\eta_j]=\sigma_x^2\delta_{ij},  \quad E[\xi_i\xi_j]=\sigma_y^2\delta_{ij},
 \end{equation}
где $E(X)$ -- математическое ожидание случайной величины $X$, $\sigma_x^2$, $\sigma_y^2$ -- положительные числа, $\delta_{ij}$ -- символ Кронекера.

Основной (и наиболее трудный) вопрос, который возникает при решении поставленной задачи, заключается в том, чтобы найти в заданной точке $t\in[-1,1]$ численные значения производных $x^{(\nu)}(t)$ $(\nu=1,\ldots,r)$ и   $y^{(\mu)}(t)$  $(\mu=1,\ldots,s)$, исходя из дискретной информации $ x_j=x(t_j)$,   $ y_j=y(t_j) $,  $0\le j\le N-1$. Дело в том, что из-за присутствия в измерениях $ x_j=x(t_j)=\tilde x(t_j)+\eta_j$ и  $ y_j=y(t_j)=\tilde y(t_j)+\xi_j $ случайных погрешностей $\eta_j$ и $\xi_j$ обычные методы численного дифференцирования, основанные на применении интерполяционных полиномов и сплайн-функций,  могут оказаться непригодными для решения поставленной задачи. Требуется предварительная обработка заданной дискретной информации $ x_j=x(t_j)$,   $ y_j=y(t_j) $,  $0\le j\le N-1$, путем ее <<сглаживания>>. Один из наиболее часто применяемых методов сглаживания дискретных данных, как известно, базируется на использовании полиномиального метода наименьших квадратов, который, в свою очередь, тесно связан с полиномами Чебышева, ортогональными на дискретной сетке $\Omega_N=\{t_j=-1+jh\}_{j=0}^{N-1}$. Остановимся на этом вопросе более подробно. Обозначим через $\hat T_{n,N}(t)$ $(0\le n\le N-1)$ полиномы Чебышева, образующие на сетке $\Omega_N$ ортонормированную систему с весом $2/N$, т.е.
\begin{equation}\label{idprmeq1.2.21.3}
\frac2N\sum_{j=0}^{N-1}\hat T_{n,N}(t_j)\hat T_{m,N}(t_j)=\delta_{nm}.
\end{equation}
Эти полиномы и некоторые их обобщения были введены впервые в работах П.Л.Чебышева \cite{idprm35,idprm36,idprm37} в  связи задачей сглаживания наблюдений и в настоящее время  находят многочисленные приложения как в математической статистике (в связи с методом наименьших квадратов), так и во многих других областях. В задаче сглаживания наблюдений полиномы Чебышева возникают следующим образом. Предположим, что нам заданы измерения
$x_j=x(t_j)=\tilde x(t_j)+\eta_j$ $(0\le j\le N-1)$  и требуется найти алгебраический полином $S_{n,N}(t)$, который минимизирует величину
$$
J(a_0,\ldots,a_n)=\frac2N\sum_{j=0}^{N-1}[x_j-p_n(t_j)]^2
$$
среди всех алгебраических полиномов $p_n(t)=a_0+a_1t+\cdots+a_nt^n$ степени $n\le N-1$. П.Л.Чебышев предложил искать такой полином в виде
$$
S_{n,N}(t)=\sum_{k=0}^n b_k\hat T_{k,N}(t)
$$
и показал, что для искомого оптимального полинома коэффициенты $b_k$ принимают вид
\begin{equation}\label{idprmeq1.2.21.4}
b_k=\hat x_k=\frac2N\sum_{j=0}^{N-1}x_j\hat T_{n,N}(t_j).
\end{equation}
Таким образом, полином
\begin{equation}\label{idprmeq1.2.21.5}
S_{n,N}(t)=\sum_{k=0}^n \hat x_k\hat T_{k,N}(t),
\end{equation}
реализующий метод наименьших квадратов, представляет собой сумму Фурье дискретной функции, принимающей значения $x_j$ в точках $t_j$ $(0\le j\le N-1)$. Одним из способов <<сглаживания>> наблюдений $x_j=x(t_j)$ $(0\le j\le N-1)$ является замена значений $x_j=x(t_j)$ соответствующими приближенными значениями $S_{n,N}(t_j)$. Более того, исходную функцию $x(t)$, заданную на $[-1,1]$, можно заменить (приближенно) суммой Фурье $S_{n,N}(t)$. В работе \cite{idprmmnk}  было показано, что если $n=O(\sqrt{N})$, то  $S_{n,N}(t)=S_{n,N}(x,t)$ имеет достаточно хорошие аппроксимативные свойства в пространстве непрерывных на $[-1,1]$ функций $x=x(t)$.  Другими словами, $S_{n,N}(t)$ приближает функцию $x(t)$
достаточно хорошо при любом $t\in[-1,1]$. В то же время производные $S^{(\nu)}_{n,N}(t)$ приближают производные $x^{(\nu)}(t)$ значительно хуже, если приближают  вообще. Поэтому частичные суммы  $S_{n,N}(t)$ не могут быть рекомендованы в качестве аппарата одновременного приближения функций $x(t)$  и $y(t)$ и их производных $x^{(\nu)}(t)$ и $y^{(\nu)}(t)$ в рассматриваемой задаче идентификации параметров из \eqref{idprmeq1.2.21.1}. Требуется сконструировать альтернативные суммам Фурье $S_{n,N}(t)$ операторы $\sigma_{n,N}(f)=\sigma_{n,N}(f,t)$, которые действуют в пространстве непрерывных на $[-1,1]$ функций $f=f(t)$, используют в качестве исходной информации значения  $f(t_j)$  $(0\le j\le N-1)$ и которые могут быть эффективно использованы для <<сглаживания>> ошибок в наблюдениях $f(t_j)$  $(0\le j\le N-1)$ и для  решения задачи одновременного приближения дифференцируемой функции $f(x)$ и её нескольких  производных. В настоящей работе (\S 3) предпринята попытка сконструировать такие операторы на основе уже упомянутых выше  полиномов Чебышева $\hat T_{k,N}(t)$ и их обобщений, также введенных в работе Чебышева \cite{idprm37}.

Следующий шаг на пути к решению поставленной задачи идентификации, предпринятый в настоящей работе, заключается в том, что от уравнения
\eqref{idprmeq1.2.21.1} мы переходим к двойственному уравнению путем представления всех функций (включая и производные), фигурирующих в \eqref{idprmeq1.2.21.1}, в виде рядов по  полиномам  Чебышева первого рода $C_n(t)=\cos(n\arccos t)$.  А именно, пусть
\begin{equation}\label{idprmeq1.2.21.6}
  x^{(\nu)}(t)=\sum_{k=0}^\infty \hat x_{\nu,k}C_k(t),\quad y^{(\mu)}(t)=\sum_{k=0}^\infty \hat y_{\mu,k}C_k(t),
\end{equation}
\begin{equation}\label{idprmeq1.2.21.7}
a_\nu(t)=\sum_{j=0}^m \hat a_{\nu,j}C_j(t),\quad b_\mu(t)=\sum_{j=0}^m \hat b_{\mu,j}C_j(t).
\end{equation}
Подставляя эти значения в \eqref{idprmeq1.2.21.1}, имеем
\begin{multline}\label{idprmeq1.2.21.8}
\sum_{k=0}^\infty \hat x_{r,k}C_k(t)=\sum_{\nu=0}^{r-1} \sum_{k=0}^\infty\sum_{j=0}^m \hat a_{\nu,j} \hat x_{\nu,k}C_j(t)C_k(t)+\\
\sum_{\mu=0}^s \sum_{k=0}^\infty\sum_{j=0}^m \hat b_{\mu,j} \hat y_{\mu,k}C_j(t)C_k(t).
\end{multline}
С другой стороны, так как $2C_j(t)C_k(t)=C_{j+k}(t)+C_{|k-j|}$, то равенство \eqref{idprmeq1.2.21.8} можно переписать так

\begin{multline}\label{idprmeq1.2.21.9}
\sum_{k=0}^\infty\hat x_{r,k}C_{k}(t)= \sum_{\nu=0}^{r-1} \sum_{k=0}^\infty\sum_{j=0}^m \hat a_{\nu,j} \hat x_{\nu,k}(C_{j+k}(t)+C_{|k-j|}(t))+\\
\sum_{\mu=0}^s \sum_{k=0}^\infty\sum_{j=0}^m \hat b_{\mu,j} \hat y_{\mu,k}(C_{j+k}(t)+C_{|k-j|}(t)).
\end{multline}

Далее имеем
\begin{equation}\label{idprmeq1.2.21.10}
\sum_{k=0}^\infty\sum_{j=0}^m \hat a_{\nu,j} \hat x_{\nu,k}C_{j+k}(t)= \sum_{n=0}^\infty C_n(t)\sum_{j=0}^{\min\{n,m\}}\hat a_{\nu,j} \hat x_{\nu,n-j},
\end{equation}
\begin{multline}\label{idprmeq1.2.21.11}
\sum_{k=0}^\infty\sum_{j=0}^m \hat a_{\nu,j} \hat x_{\nu,k}C_{|j-k|}(t)= \sum_{n=0}^\infty C_n(t)\sum_{j=0}^m\hat a_{\nu,j}\hat x_{\nu,n+j}+%\\
\sum_{n=1}^m C_n(t)\sum_{j=n}^m\hat a_{\nu,j}\hat x_{\nu,j-n},
\end{multline}
\begin{equation}\label{idprmeq1.2.21.12}
\sum_{k=0}^\infty\sum_{j=0}^m \hat b_{\mu,j} \hat y_{\mu,k}C_{j+k}(t)= \sum_{n=0}^\infty C_n(t)\sum_{j=0}^{\min\{n,m\}}\hat b_{\mu,j} \hat y_{\mu,n-j},
\end{equation}
\begin{multline}\label{idprmeq1.2.21.13}
\sum_{k=0}^\infty\sum_{j=0}^m \hat b_{\mu,j} \hat y_{\mu,k}C_{|j-k|}(t)= \sum_{n=0}^\infty C_n(t)\sum_{j=0}^m\hat b_{\mu,j}\hat y_{\mu,n+j}+%\\
\sum_{n=1}^m C_n(t)\sum_{j=n}^m\hat b_{\mu,j}\hat y_{\mu,j-n}.
\end{multline}

Сопоставляя равенства \eqref{idprmeq1.2.21.9} -- \eqref{idprmeq1.2.21.13} между собой, нетрудно записать алгоритм для получения элементов матрицы $U=\{u_{il}\}_{1\le i\le \infty,1\le l\le L}$ с числом столбцов, равным $L=(r+s+1)(m+1)$  и бесконечным числом строк, с помощью которой можно записать бесконечную систему линейных уравнений относительно неизвестных коэффициентов $\hat{a}_{i,j}$ и $\hat{b}_{k,l}$:
\begin{equation}\label{idprmeq1.2.21.14}
U\cdot V=X_r,
\end{equation}
где $V$ -- вектор-столбец, для которого транспонированный вектор $V'$ имеет вид
\begin{equation}\label{idprmeq1.2.21.15}
V'= (\hat a_{0,0},\ldots,\hat a_{0,m},\ldots, \hat a_{r-1,0},\ldots,\hat a_{r-1,m},\hat b_{0,0},\ldots,\hat b_{0,m},\ldots, \hat b_{s,0},\ldots,\hat b_{s,m} ),
\end{equation}
$X_r$ -- последовательность-столбец коэффициентов Фурье -- Чебышева функции $x^{(r)}(t)$.

Обозначим через $U_{N,L}$ подматрицу матрицы $U$ вида $U_{N,L}=\{u_{il}\}_{1\le i\le N,1\le l\le L}$, причем будем считать, что $N\ge L$.
Вместо бесконечной системы  \eqref{idprmeq1.2.21.14} будем рассматривать её конечную подсистему
\begin{equation}\label{idprmeq1.2.21.16}
U_{N,L}\cdot V=X_{r,N},
\end{equation}
 где $X_{r,N}$ -- вектор-столбец, составленный из коэффициентов $\hat x_{r,k}$ $(0\le k\le N-1)$. Поскольку, как было отмечено выше, мы предполагаем $N\ge L$, то вполне может случиться так, что система \eqref{idprmeq1.2.21.16} не разрешима. Поэтому ставится задача о нахождении квази-решения
 системы \eqref{idprmeq1.2.21.16}. Если под квази-решением понимать вектор $V$, при котором левая часть уравнения \eqref{idprmeq1.2.21.16} наименее отклоняется от ее правой части в евклидовой метрике $R_2$, то можно показать, что такое решение имеет вид
 \begin{equation}\label{idprmeq1.2.21.17}
 V=(U'_{N,L}U_{N,L})^{-1}U'_{N,L}X_{r,N}.
 \end{equation}
 Если таким способом нам удастся найти численные значения элементов вектора $V$, то тогда будут найдены приближенно и искомые переменные коэффициенты  $a_\nu(t)$, $b_\mu(t)$ посредством равенств \eqref{idprmeq1.2.21.7}. Другими словами, будет решена задача идентификации рассматриваемой временно неинвариантной линейной системы.

Как уже отмечалось выше, одна из основных проблем, возникающих при решении поставленной задачи, состоит в конструировании упомянутых выше операторов $\sigma_{n,N}(f)=\sigma_{n,N}(f,t)$. При решении этой проблемы нам понадобится ряд свойств полиномов Чебышева, ортогональных на равномерной сетке.

\subsection{ Некоторые сведения о полиномах Чебышева,\\ ортогональных на равномерной  сетке}

 Пусть $N$ -- натуральное, $\alpha$, $\beta$ -- произвольные  числа. Положим
\begin{equation}\label{siTpeq1.2.22.1}
\rho(x)=\rho(x;\alpha,\beta,N)={\Gamma(x+\beta+1)
\Gamma(N-x+\alpha)\over \Gamma(x+1)\Gamma(N-x)},
\end{equation}
\begin{equation}\label{siTpeq1.2.22.2}
T_n^{\alpha,\beta}(x,N)={(-1)^n\over n!(N-1)^{[n]}\rho(x)}\Delta^n
\left\{\rho(x)(x-N-\alpha)^{[n]}x^{[n]}\right\},
\end{equation}
 где $\Delta^nf(x)$ -- конечная разность $n$-го порядка функции
     $f(x)$ в точке $x$, т.е. $\Delta^0f(x)=f(x)$,
$\Delta^1f(x)=\Delta f(x)=f(x+1)-f(x)$, $\Delta^nf(x)=\Delta
\Delta^{n-1}f(x)$ $(n\ge1)$, $a^{[0]}=1$,
$a^{[k]}=a(a-1)\cdots(a-k+1)$ при $k\ge1$. Для каждого $0\le n\le
N-1$ равенство \eqref{siTpeq1.2.22.2} определяет \cite{idprm35,idprm36,idprm37,idprm38} алгебраический полином степени $n$,
     для которого
$$
T_n^{\alpha,\beta}(N-1,N)={n+\alpha\choose n},\qquad
T_n^{\alpha,\beta}(0,N)=(-1)^n{n+\beta\choose n}.
$$
Полиномы допускают  следующее явное представление
\begin{equation}\label{siTpeq1.2.22.3}
T_n^{\alpha,\beta}(x,N)=(-1)^n{\Gamma(n+\beta+1)\over
n!}\sum_{k=0}^n(-1)^k {n^{[k]}(n+\alpha+\beta+1)_kx^{[k]}\over
\Gamma(k+\beta+1) k!(N-1)^{[k]}}.
\end{equation}
Если $\alpha$,
$\beta>-1$, то полиномы $T_n^{\alpha,\beta}(x,N)\quad (0\le n\le
N-1)$ образуют ортогональную  с весом $\rho(x)$ (см.\eqref{siTpeq1.2.22.1}) систему  на множестве
$\Omega_N=\{0,1,\ldots,N-1\}$, точнее
\begin{equation}\label{siTpeq1.2.22.4}
\sum_{x\in\Omega_N}\mu(x)T_n^{\alpha,\beta}(x,N)T_m^{\alpha,\beta}(x,N)
=h_{n,N}^{\alpha,\beta}\delta_{nm},
\end{equation}
где  $\delta_{nm}$ -- символ Кронекера,
\begin{multline}\label{siTpeq1.2.22.5}
\mu(x)=\mu(x;\alpha,\beta,N)={\Gamma(N)2^{\alpha+\beta+1} \over
\Gamma(N+\alpha+\beta+1)}\rho(x)=\\
{\Gamma(N)2^{\alpha+\beta+1}\over \Gamma(N+\alpha+\beta+1)}
{\Gamma(x+\beta+1)\Gamma(N-x+\alpha)\over \Gamma(x+1)\Gamma(N-x)}.
\end{multline}
 \begin{equation}\label{siTpeq1.2.22.6}
h_{n,N}^{\alpha,\beta}={(N+n+\alpha+\beta)^{[n]}\over
(N-1)^{[n]}}{\Gamma(n+\alpha+1)\Gamma(n+\beta+1)
2^{\alpha+\beta+1}\over
n!\Gamma(n+\alpha+\beta+1)(2n+\alpha+\beta+1)}.
\end{equation}
 При $n=0$  произведение $(\alpha+\beta+1)\Gamma(\alpha+ \beta+1)$
следует заменить на $\Gamma(\alpha+\beta+2)$. Для $0\le n\le N-1$
положим
 \begin{equation}\label{siTpeq1.2.22.7}
  \tau_n^{\alpha,\beta}(x)=\tau_n^{\alpha,\beta}(x,N)=
\left\{h_{n,N}^{\alpha,\beta}\right\}^{-1/2}
T_n^{\alpha,\beta}(x,N).
\end{equation}
 Очевидно, если $0\le n,m\le
N-1$, то
\begin{equation}\label{siTpeq1.2.22.8} \sum_{x=0}^{N-1}\mu(x)\tau_n^{\alpha,\beta}(x,N)
\tau_m^{\alpha,\beta}(x,N)=\delta_{nm}.
\end{equation}
Другими словами, многочлены $\tau_n^{\alpha,\beta}(x,N)$ $(0\le n\le N-1)$
образуют ортонормированную с весом $\mu(x)$ систему на
$\Omega_N$.

Формула Кристоффеля--Дарбу  для многочленов Чебышева
$T_n^{\alpha,\beta}(x)=T_n^{\alpha,\beta}(x,N)$ имеет следующий вид:
$$
\mathcal{K}_{n,N}^{\alpha,\beta}(x,y)=\sum_{k=0}^n\tau_k^{\alpha,\beta}(x)\tau_k^{\alpha,\beta}(y)=
\sum_{k=0}^n{T_k^{\alpha,\beta}(x)T_k^{\alpha,\beta}(y)\over
h_{k,N}^{\alpha,\beta}} $$ $$ ={(N-1)^{[n+1]}\over
(N+n+\alpha+\beta)^{[n]}} {2^{-\alpha-\beta-1}\over
2n+\alpha+\beta+2} {\Gamma(n+2)\Gamma(n+\alpha+\beta+2)\over
\Gamma(n+\alpha+1)\Gamma(n+\beta+1)}\times
$$
\begin{equation}\label{siTpeq1.2.22.9}
{T_{n+1}^{\alpha,\beta}(x)T_n^{\alpha,\beta}(y)-
T_n^{\alpha,\beta}(x)T_{n+1}^{\alpha,\beta}(y)\over x-y}.
\end{equation}
Поскольку $\Delta a^{[k]}=ka^{[k-1]}$, то из \eqref{siTpeq1.2.22.3} находим
 \begin{multline}\label{siTpeq1.2.22.10}
(n+1)T_{n+1}^{\alpha,\beta}(x,N)+(n+\beta+1)T_n^{\alpha, \beta}(x,N)=\\
{2n+\alpha+\beta+2\over N-1}xT_n^{\alpha,\beta+1}(x-1,N-1).
 \end{multline}
 Из равенства $\mu(N-1-x;\beta,\alpha,N)=\mu(x;\alpha,\beta,N)$, непосредственно
вытекающего из  соотношения ортогональности \eqref{siTpeq1.2.22.4}, следует,
что при $\alpha,\beta>-1$
 \begin{equation}\label{siTpeq1.2.22.11}
T_n^{\alpha,\beta}(x,N)=(-1)^nT_n^{\beta,\alpha}(N-1-x,N).
 \end{equation}
  Поскольку обе части этого равенства аналитичны
относительно $\alpha$ и $\beta$, то оно справедливо для произвольных
$\alpha$ и $\beta$. Из \eqref{idprmeq1.2.21.10} и \eqref{idprmeq1.2.21.11} имеем также следующее
     равенство
\begin{multline}\label{siTpeq1.2.22.12}
(n+\alpha+1)T_n^{\alpha,\beta}(x,N)-(n+1)T_{n+1}^{\alpha,\beta}(x,N)=\\
{2n+\alpha+\beta+2\over   N-1}(N-1-x)T_n^{\alpha+1,
\beta}(x,N-1).
\end{multline}
 Полагая в \eqref{siTpeq1.2.22.9} $y=N-1$, имеем:
  $$
\mathcal{K}_{n,N}^{\alpha,\beta}(x,N-1)=
{T_n^{\alpha,\beta}(x)-{n+1\over n+\alpha+1}
T_{n+1}^{\alpha,\beta}(x)\over N-1-x}
     $$
     $$
\times {(N-1)^{[n+1]}\over (N+n+\alpha+\beta)^{[n]}}
{2^{-\alpha-\beta-1}\over 2n+\alpha+\beta+2}
{(n+\alpha+1)\Gamma(n+\alpha+\beta+2)\over
\Gamma(\alpha+1)\Gamma(n+\beta+1)} $$
\begin{equation}\label{siTpeq1.2.22.13}
={(N-2)^{[n]}\Gamma(n+\alpha+\beta+2)2^{-\alpha-\beta-1}\over
(N+n+\alpha+\beta)^{[n]}\Gamma(\alpha+1)\Gamma(n+\beta+1)}
T_n^{\alpha+1,\beta}(x,N-1).
\end{equation}
 Последнее выражение
является следствием равенства \eqref{siTpeq1.2.22.12}. Аналогично, полагая в \eqref{siTpeq1.2.22.9}
$y=0$ и учитывая \eqref{siTpeq1.2.22.11}, получаем:
\begin{equation}\label{siTpeq1.2.22.14} \mathcal{K}_{n,N}^{\alpha,\beta}(x,0)=(-1)^n
{(N-2)^{[n]}\Gamma(n+\alpha+\beta+2)2^{-\alpha-\beta-1}\over
(N+n+\alpha+\beta)^{[n]}\Gamma(\beta+1)\Gamma(n+\alpha+1)}
T_n^{\alpha,\beta+1}(x-1,N-1).
\end{equation}
  Непосредственно из явной формулы \eqref{siTpeq1.2.22.3} мы можем вывести следующее
     полезное равенство
\begin{equation}\label{siTpeq1.2.22.15}
\Delta^m T_n^{\alpha,\beta}(x,N)={(n+\alpha+\beta+1)_m\over
(N-1)^{[m]}} T_{n-m}^{\alpha+m,\beta+m}(x,N-m),
\end{equation}
где $(a)_0=1$, $(a)_k=a(a+1)\cdots(a+k-1)$ при $k\ge1$. Если $\beta$
такое целое число, что $-n\le\beta\le-1$, то из \eqref{siTpeq1.2.22.15} выводим также
\begin{equation}\label{siTpeq1.2.22.16}
T_n^{\alpha,\beta}(x,N)={(n+\beta)!\over n!}
{(n+\alpha)^{[-\beta]}x^{[-\beta]}\over (N-1)^{[-\beta]}}
T_{n+\beta}^{\alpha,-\beta}(x+\beta,N+\beta),
 \end{equation}
 а если $\alpha$ и $\beta$ -- целые, $-n\le\beta\le-1$, $-(n+\beta)\le\alpha\le-1$,
$N\ge2$, то \begin{equation}\label{siTpeq1.2.22.17} T_n^{\alpha,\beta}(x,N)={(-1)^\alpha
x^{[-\beta]}(N-x-1)^{[-\alpha]}\over
(N-1)^{[-\beta]}(N-1+\beta)^{[-\alpha]}}
T_{n+\alpha+\beta}^{-\alpha,-\beta}(x+\beta,N+\alpha+\beta). \end{equation}
Разностная формула Родрига \eqref{siTpeq1.2.22.1} допускает следующее
обобщение
     $$
\rho(x+m;\alpha,\beta,N+m)T_n^{\alpha,\beta}(x+m,N+m)= $$
\begin{equation}\label{siTpeq1.2.22.18}
{(-1)^m\over n^{[m]}(N)_m}\Delta^m
\left\{\rho(x;\alpha+m,\beta+m,N)T_{n-m}^{\alpha+m,\beta+m}(x,N)\right\},
 \end{equation}
 которое, впрочем, непосредственно вытекает из \eqref{siTpeq1.2.22.1}.
Заменяя здесь  $m$ на $\nu$,  $\alpha$ и $\beta$  на $m-\nu$, $n$ на
$k+\nu-m$, мы можем также записать
$$\Delta^\nu\{(x+1)_m(N-x)_mT_{k-m}^{m,m}(x,N)\}=
$$\begin{equation}\label{siTpeq1.2.22.19}(-1)^\nu(k+\nu-m)^{[\nu]}(N+\nu-1)^{[\nu]}(x+1+\nu)_{m-\nu}(N-x)_{m-\nu}T_{k+\nu-m}^{m-\nu,m-\nu}(x+\nu,N+\nu).\end{equation}
 Если в равенстве \eqref{siTpeq1.2.22.18}  мы заменим $\alpha$, $\beta$ и $n$,
     соответственно, на   $\alpha-m$, $\beta-m$ и $k+m$, то придем
     к формуле
\begin{equation}\label{siTpeq1.2.22.20}
\Delta^m T_{k+m}^{\alpha-m,\beta-m}(x,N)={(k+\alpha+\beta)^{[m]}
\over (N-1)^{[m]}}T_k^{\alpha,\beta}(x,N-m).
\end{equation}



Пусть $a,\alpha>-1$, $0\le n\le N-1$. Тогда \cite{idprm39}
$$
{T_n^{a,a}(x,N)\over T_n^{a,a}(N-1,N)}=\sum_{j=0}^{[n/2]}
{n!(\alpha+1)_{n-2j}(n+2a+1)_{n-2j}(1/2)_j(a-\alpha)_j
\over (n-2j)!(2j)!(a+1)_{n-2j}(n-2j+2\alpha+1)_{n-2j}}$$
\begin{equation}\label{siTpeq1.2.22.21}
 \times{1\over (n-2j+a+1)_j(n-2j+\alpha+3/2)_j}
  {T_{n-2j}^{\alpha,\alpha}(x,N)\over
T_{n-2j}^{\alpha,\alpha}(N-1,N)},
    \end{equation}
  где $[n/2]$--целая часть числа $n/2$.


Мы введем здесь двух-индексные полиномы Чебышева
$T_{k,M}^{\alpha,\beta}(x)$ и $\hat T_{k,M}^{\alpha,\beta}(x)$  с
помощью следующих равенств $(0\le k\le M-1)$:
\begin{equation}\label{siTpeq1.2.22.22}
T_{k,M}^{\alpha,\beta}(x)=T_k^{\alpha,\beta}\left({M-1\over2}(1+x),M\right),
\hat T_{k,M}^{\alpha,\beta}(x)=\tau_k^{\alpha,\beta}\left({M-1\over2}(1+x),M\right).
\end{equation}
 В случае целых $\alpha$ и $\beta$ в \cite{idprmasympPropsAndWeightEst, idprmasympProps, idprmasympCheb} установлен
следующий результат. Пусть $P_n^{\alpha,\beta}(x)$-- полином Якоби,
для которого $P_n^{\alpha,\beta}(1)={n+\alpha\choose n }$,
  $a>0$. Тогда имеет место асимптотическая формула
\begin{equation}\label{siTpeq1.2.22.23}
 T_{n,N}^{\alpha,\beta}(t)= P_n^{\alpha,\beta}(t)+ v_{n,N}^{\alpha,\beta}(t),
\end{equation}
 для остаточного члена $v_{n,N}^{\alpha,\beta}(t)$
 которой при $1\le n\le aN^{1/2}$, $\delta>0$ справедлива оценка
\begin{equation}\label{siTpeq1.2.22.24} |v_{n,N}^{\alpha,\beta}(t)|  \le c \sqrt{n\over N} \left[|1-t|^{1/2}+{1\over
n}\right]^{-\alpha-{1\over2}} \left[|1+t|^{{1\over2}}+{1\over
n}\right]^{-\beta-{1\over2}},
\end{equation}
где
$c=c(\alpha,\beta,a,\delta)$, $-1-\delta/ n^2\le t\le 1+\delta/
n^2$.

Далее, пусть $j_1,j_2$ -- фиксированные целые
числа, $\alpha,\beta$ -- неотрицательные целые числа, $1\le n\le
aN^{{1\over2}}$ $(a>0)$, $\delta\ge0$, $|\tau|\le\delta$, $-1\le t
\le1$.  Тогда из \eqref{siTpeq1.2.22.23} и \eqref{siTpeq1.2.22.24} с учетом известных весовых оценок
для полиномов Якоби $P_n^{\alpha,\beta}(x)$ непосредственно
выводится следующая оценка:
\begin{multline}\label{siTpeq1.2.22.25}
\left|T_n^{\alpha,\beta}\left[{N+j_1\over2}(1+t)+\tau,
N+j_2\right]\right|  \le \\
cn^{-{1\over2}}\left[(1-t)^{{1\over2}}+{1\over
n}\right]^{-\alpha-{1\over2}} \left[(1+t)^{1\over2}+{1\over
n}\right]^{-\beta-{1\over2}},
\end{multline}
 где $c=c(a,\alpha,\beta,j_1,j_2,\delta)$.

Имеет место  рекуррентная формула:
$T_0^{\alpha,\beta}(x,N)=1$, $ T_1^{\alpha,\beta}(x,N)= x(\alpha+\beta+2)/(N-1)-\beta-1$,
 \begin{equation}\label{siTpeq1.2.22.26}
T_n^{\alpha,\beta}(x)=
(\kappa_nx-\sigma_n)T_{n-1}^{\alpha,\beta}(x)-\delta_nT_{n-2}^{\alpha,\beta}(x),
\end{equation}
где

$$\begin{array}{crl}\kappa_n&=&{(2n+\alpha+\beta-1)(2n+\alpha+\beta)\over
n(n+\alpha+\beta)(N-n)},\\\\\sigma_n&=&
\kappa_n\left(\frac{(\beta^2-\alpha^2)(\alpha+\beta+2N)}{4(2n+\alpha+\beta-2)
(2n+\alpha+\beta)}+\frac{\alpha-\beta+2N-2}4\right),\\\\\delta_n&=&{(n+\alpha-1)(n+\beta-1)(2n+\alpha+\beta)(N+n+\alpha+\beta-1)\over
n(n+\alpha+\beta)(2n+\alpha+\beta-2)(N-n)}.\end{array}
$$

Для полиномов $\tau_n^{\alpha,\beta}(x,N)$ (см.\eqref{siTpeq1.2.22.7}) справедлива следующая
рекуррентная формула
$$
\tau_0^{\alpha,\beta}(x,N)=\left[\frac{\Gamma(\alpha+\beta+2)}{2^{\alpha+\beta+1}\Gamma(\alpha+1)\Gamma(\beta+1)}\right]^\frac12,
$$
$$
\tau_1^{\alpha,\beta}(x,N)=\left\{\frac{(N-1)\Gamma(\alpha+\beta+2)(\alpha+\beta+3)}{(N+\alpha+\beta+1)
\Gamma(\alpha+2)\Gamma(\beta+2)2^{\alpha+\beta+1}}\right\}^\frac12\left[x\frac{\alpha+\beta+2}{N-1}-\beta-1\right],
$$
\begin{equation}\label{siTpeq1.2.22.27}
\tau_n^{\alpha,\beta}(x)=
(\hat\kappa_nx-\hat\sigma_n)\tau_{n-1}^{\alpha,\beta}(x)-\hat\gamma_n\tau_{n-2}^{\alpha,\beta}(x),
\end{equation}
где
$$
\hat\kappa_n=(2n+\alpha+\beta)\left[\frac{(2n+\alpha+\beta-1)(2n+\alpha+\beta+1)}{(N+n+\alpha+\beta)
(N-n)n(n+\alpha)(n+\beta)(n+\alpha+\beta)}\right]^\frac12,
$$
$$
\hat\sigma_n=\hat\kappa_n\left(\frac{(\beta^2-\alpha^2)(\alpha+\beta+2N)}{4(2n+\alpha+\beta-2)
(2n+\alpha+\beta)}+\frac{\alpha-\beta+2N-2}4\right),
$$
$$
\hat\gamma_n=\frac{2n+\alpha+\beta}{2n+\alpha+\beta-2}\left[\frac{N+n+\alpha+\beta-1}{N+n+\alpha+\beta}
\,\frac{N-n+1}{N-n}\,\frac{n-1}n\,\frac{n+\alpha-1}{n+\alpha}\,\frac{n+\beta-1}
{n+\beta}\right.\times
$$
$$
\times\left.\frac{n+\alpha+\beta-1}{n+\alpha+\beta}\,\frac{2n+\alpha+\beta+1}{2n+\alpha+\beta-3}\right]^\frac12.
$$





Введем полиномы $(n=0,1,\ldots,N-1)$
\begin{multline}\label{siTpeq1.2.22.28}
 \tilde T_n^{\alpha,\beta}(t)=\tilde
T_{n,N}^{\alpha,\beta}(t)=\\
{2^n(N-1)!\over(2N+\alpha+\beta)^n(N-n-1)!}
T_n^{\alpha,\beta}\left[{2N+\alpha+\beta\over4}(1+t)-{\beta+1\over 2},N\right],
\end{multline}

для которых рекуррентная формула принимает следующий вид: $\tilde T_0^{\alpha,\beta}(t)=1$,

 $\tilde T_1^{\alpha,\beta}(t)=\frac12(\alpha+\beta+2)t+\frac12(\alpha-\beta)$,
\begin{equation}\label{siTpeq1.2.22.29}
\tilde
T_n^{\alpha,\beta}(t)= (\alpha_nt-\beta_n)\tilde
T_{n-1}^{\alpha,\beta}(t)-\gamma_n\left[1-\left({2n+\alpha+
\beta-2\over2N+\alpha+\beta}\right)^2\right]\tilde T_{n-2}^{\alpha,\beta}(t),
\end{equation}
где
 $$
 \begin{array}{crl}\alpha_n&=&{(2n+\alpha+\beta-1)(2n+\alpha+\beta)\over
2n(n+\alpha+\beta)},\\\\\beta_n&=&\alpha_n\frac{\beta^2-\alpha^2}{(2n+\alpha+\beta-2)
(2n+\alpha+\beta)},\\\\\gamma_n&=&
\alpha_n{2(n+\alpha-1)(n+\beta-1)\over(2n+\alpha+\beta-2)(2n+\alpha+\beta-1)}.\end{array}
$$

\subsection{О преобразовании Фурье--Чебышева на равномерной сетке}


 Пусть $\Omega=\{0,1,\cdots,N-1\}$, $\mu(x)=\mu(x;\alpha,\beta,N)$ -- весовая
функция, определенная равенством \eqref{siTpeq1.2.22.5}, $f=f(x)$ -- дискретная функция, заданная на $\Omega$. В приложениях полиномов Чебышева $\tau_n^{\alpha,\beta}(x,N)$, в том числе и тех, о которых шла речь выше,
приходится переходить от функции $f$ к ее дискретному преобразованию Фурье -- Чебышева $\hat f(n)=\hat f_n=\hat f_n(\alpha,\beta)$ $(n\in\Omega)$ по формуле
     \begin{equation}\label{FTchbeq1.2.23.1}
\hat f_n=\sum_{t\in\Omega}f(t)\tau_n^{\alpha,\beta}(t,N)\mu(t).
    \end{equation}
Обратное дискретное преобразование Фурье -- Чебышева дает:
    \begin{equation}\label{FTchbeq1.2.23.2}
f(x)=\sum_{n=0}^{N-1}\hat f_n\tau_n^{\alpha,\beta}(x,N)\quad (\alpha,\beta>-1).
    \end{equation}
Основная проблема, возникающая при численной реализации преобразований \eqref{FTchbeq1.2.23.1} и \eqref{FTchbeq1.2.23.2},
заключается в необходимости достаточно быстрого и устойчивого вычисления значения многочленов
$\tau_n^{\alpha,\beta}(x,N)$. Мы здесь остановимся на двух случаях: 1) $\alpha=\beta=0$; 2)
 $\alpha=\beta=-1/2$.

Начнем со случая $\alpha=\beta=0$. Для сокращения записи, положим $\tau_n^0(x)=\tau_n^{0,0}(x,N)$.
Тогда из рекуррентной формулы \eqref{siTpeq1.2.22.27} для  $\alpha=\beta=0$ имеем:
$$
\tau_0^0(x)=2^{-1/2},\quad\tau_1^0(x)=\left({3\over 2(N^2-1)}\right)^{1/2} (2x-N+1),
     $$
\begin{equation}\label{FTchbeq1.2.23.3}
\tau_n^0(x)=X_n(2x-N+1)\tau_{n-1}^0(x)-Y_n\tau_{n-2}^0(x), \,\,\,\,\,n\ge2,
\end{equation}
 где
 $$
X_n={1\over n}\left({(2n-1)(2n+1)\over N^2-n^2}\right)^{1/2},\quad
     Y_n={n-1\over n}\left({(2n+1)(N^2-(n-1)^2)\over (2n-3)(N^2-n^2)} \right)^{1/2}.
 $$
Кроме того напомним, что
\begin{equation}\label{FTchbeq1.2.23.4}
 \tau_n^0(N-1-x)=(-1)^n\tau_n^0(x).
 \end{equation}
  Если $n$ существенно меньше $N$, то формулы
\eqref{FTchbeq1.2.23.3} и \eqref{FTchbeq1.2.23.4} дают устойчивый метод для вычисления значений полиномов $\tau_n^0(x)$ при
$x\in[0,N-1]$. Достаточно, например, считать, что $n\le \min\{N-1,\quad 5N^{1/2}\}$. Однако, если
$n$ близко к $N$, то использование этих формул при больших $N$ (например, если $n>70$ и $N=100$)
дает катастрофический рост погрешности вычислений при стремлении  $x$ к нулю или к $N-1$.  Эту
трудность можно  избежать различными приемами. Более подробно этот вопрос мы рассмотрим позже.

     Переходя к случаю $\alpha=\beta=-1/2$, заметим, что для
$v_j=\mu(j;-1/2,-1/2,N)$ из \eqref{siTpeq1.2.22.5} имеем:
    \begin{equation}\label{FTchbeq1.2.23.5}
v_j={\Gamma(j+1/2)\Gamma(N-j-1/2)\over\Gamma(j+1) \Gamma(N-j)}.
    \end{equation}
Так как $\Gamma(z+1)=\Gamma(z)z$, то
    \begin{equation}\label{FTchbeq1.2.23.6}
v_{j+1}=v_j{(j+0.5)(N-1-j)\over (N-j-3/2)(j+1)}\quad (0\le j\le N-2).
   \end{equation}
Кроме того, пользуясь для целого $m\ge1$ формулой
   $$
{\Gamma(m+1/2)\over \Gamma(m+1)}=\pi^{1/2}\prod_{k=1}^m
                                  \left(1-{0.5\over k}\right),
   $$
из \eqref{FTchbeq1.2.23.5} при $j=\left[{N-1\over2}\right]$ имеем:
\begin{equation}\label{FTchbeq1.2.23.7}
v_j=\pi\begin{cases}\prod\limits_{k=1}^j\left(1-{0.5\over k}\right)^2, \text{если $N-1$
четно,}\\ {0.5+j\over j+1}\prod\limits_{k=1}^j\left(1-{0.5\over k}\right)^2, \text{если
$N-1$ нечетно}.\end{cases}
\end{equation}
 Для многочленов $\tau_n^{-1/2}(x)=\tau_n^{-1/2,-1/2}(x,N)$ находим:
$$
\tau_0^{-1/2}(x)=\pi^{-1/2},\tau_1^{-1/2}(x)=\left({2\over \pi(N-1)N}\right)^{1/2}(2x-N+1),
$$
\begin{equation}\label{FTchbeq1.2.23.8}
\tau_n^{-1/2}(x)=\tilde X_n(2x-N+1)\tau_{n-1}^{-1/2}(x)- \tilde Y_n\tau_{n-2}^{-1/2}(x),
\end{equation}
где
$$ \tilde X_n={2\over ((N+n-1)(N-n))^{1/2}},
 \quad \tilde Y_n=\left({(N+n-2)(N-n+1)\over
(N-n)(N+n-1)}\right)^{1/2}2^{\varphi(n)},
$$
а $\varphi(2)=1/2$, $\varphi(n)=0$ при $n\ge3$.


Через $S_{n,N}(f,x)=S_{n,N}^{\alpha,\beta}(f,x)$ обозначим частичную сумму конечного ряда Фурье -- Чебышева \eqref{FTchbeq1.2.23.2} вида
 \begin{equation}\label{FTchbeq1.2.23.7}
S_{n,N}(f,x)=\sum_{k=0}^{n}\hat f_k\tau_k^{\alpha,\beta}(x,N)\quad (\alpha,\beta>-1).
    \end{equation}
Как хорошо известно из теории евклидовых пространств, $S_{n,N}(f,x)$ доставляет минимум величине
$$
J=\sum_{j=0}^{N-1}[f(j)-p_n(j)]^2\mu(j;\alpha,\beta,N)
$$
среды   всех алгебраических полиномов $p_n(x)$ степени $n$.  При этом
\begin{equation}\label{FTchbeq1.2.23.8}
\sum_{j=0}^{N-1}[f(j)-S_{n,N}(f,j)]^2\mu(j;\alpha,\beta,N)=\sum_{k=n+1}^{N-1}\hat f_k^2.
\end{equation}

\subsection{Некоторые специальные ряды по полиномам Чебышева,\\
 ортогональным на равномерной сетке }


     Пусть $r$ и $N$ -- натуральные числа. Рассмотрим дискретную
     функцию $d(x)$, заданную на сетке $\bar \Omega_{N+2r}=
\{-r,-r+1,\ldots,-1,0,1,\ldots,N-1,N,\ldots,N-1+r\}$.
     Положим
   \begin{equation}\label{sTpseq1.2.24.1}
F(x)=d(x-r),\qquad x\in \Omega_{N+2r},
 \end{equation}
     \begin{equation}\label{sTpseq1.2.24.2}
     b(x)=b(x;r,N)=\Delta^rF(x).
     \end{equation}
Дискретная функция  $b(x)$ определена на сетке $\Omega_{N+r}$ и,
     следовательно, ее можно разложить в конечный ряд Фурье
по ортогональной системе полиномов Чебышева
$\{T_k^{0,0}(x,N+r)\}_{k=0}^{N-1+r}$:
     \begin{equation}\label{sTpseq1.2.24.3}
 b(x)=\sum_{k=0}^{N-1+r}d_{r,k}T_k^{0,0}(x,N+r),
     \end{equation}
  где

  \begin{equation}\label{sTpseq1.2.24.4}
 d_{r,k}    =d_{r,k}(N+r)={2\over (N+r)h_{k,N+r}^{0,0}}
     \sum_{t\in\Omega_{N+r}}b(t)T_k^{0,0}(t,N+r).
    \end{equation}

\textit{Смешанный ряд} функции $d=d(x)$ по полиномам  Чебышева
$\{T_k^{\alpha,\beta}(x,N+r)\}_{k=0}^{N-1+r}$ при $\alpha=\beta=0$
имеет вид (см. \cite{idprmapproxYn, idprmapproxByCheb, idprmgreenBook})
\begin{equation}\label{sTpseq1.2.24.5}
 d(x)= \mathcal{D}_{2r-1,N}(d,x)+\mathcal{T}_{r,N}(d,x),
\end{equation}
 где
  $$\mathcal{D}_{2r-1,N}(x)=\mathcal{D}_{2r-1,N}(d,x)=
$$
\begin{equation}\label{sTpseq1.2.24.6}
\sum_{i=1}^r(-1)^{i-1}{(x+1)_r(N-x)_r\over
(i-1)!(r-i)!(N+i)_r}\left[{d(-i)\over x+i}+{d(N-1+i)\over
N-1+i-x}\right],
\end{equation}
\begin{equation}
\label{sTpseq1.2.24.7}
\mathcal{T}_{r,N}(d,x)= {(-1)^r
(x+1)_r(N-x)_r\over (N-1+r)^{[r]}} \sum_{k=r}^{N-1+r} {d_{r,k}\over
k^{[r]}}T_{k-r}^{r,r}(x,N),
\end{equation}
$x\in\bar\Omega_{N+2r}=\{-r,\ldots,-1,0,1,\ldots,N-1,N,\ldots,N-1+r\}$.

Рассмотрим некоторые разностные свойства ряда \eqref{sTpseq1.2.24.5}, которые нам
понадобятся в дальнейшем. Применим равенства \eqref{sTpseq1.2.24.5} -- \eqref{sTpseq1.2.24.7} к функции
$\partial(x)=\Delta^\nu d(x-\nu)$, заданной на $\bar\Omega_{N+\nu+2(r-\nu)}=
\{-r+\nu,\ldots,-1,0,1,\ldots,N-1,N,\ldots,N-1+r\}$. Это дает
\begin{equation}
\label{sTpseq1.2.24.8}
\partial(x)=\Delta^\nu d(x-\nu)= \mathcal{D}_{2(r-\nu)-1,N+\nu}(\partial,x)+\mathcal{T}_{r-\nu,N+\nu}(\partial,x),
\end{equation}
где
$$
\mathcal{D}_{2(r-\nu)-1,N+\nu}(\partial,x)=
$$
\begin{equation}
\label{sTpseq1.2.24.9}
\sum_{i=1}^{r-\nu}(-1)^{i-1}{(x+1)_{r-\nu}(N+\nu-x)_{r-\nu}\over
(i-1)!(r-\nu-i)!(N+\nu+i)_{r-\nu}}\left[{\partial(-i)\over
x+i}+{\partial(N+\nu-1+i)\over N+\nu-1+i-x}\right],
\end{equation}
\begin{multline}
\label{sTpseq1.2.24.10}
\mathcal{T}_{r-\nu,N+\nu}(\partial,x)=\\
{(-1)^{r-\nu}
(x+1)_{r-\nu}(N+\nu-x)_{r-\nu}\over (N-1+r)^{[r-\nu]}}
\sum_{k=r-\nu}^{N-1+r} {\partial_{r-\nu,k}\over  k^{[r-\nu]} }
T_{k-r+\nu}^{r-\nu,r-\nu}(x,N+\nu),
\end{multline}
С другой стороны, заметим, что
$$
\partial_{r-\nu,k}=
{2\over (N+r)h_{k,N+r}^{0,0}}\sum_{t=0}^{N+r-1}\Delta^{r-\nu}\partial(t-r+\nu)T^{0,0}_k(t,N+r)=
$$
\begin{equation}\label{sTpseq1.2.24.11}
{2\over (N+r)h_{k,N+r}^{0,0}}\sum_{t=0}^{N+r-1}\Delta^rd(t-r)T^{0,0}_k(t,N+r)=d_{r,k},
\end{equation}
поэтому \eqref{sTpseq1.2.24.9}  можно переписать еще так
$$\mathcal{T}_{r-\nu,N+\nu}(\partial,x)= {(-1)^{r-\nu}
(x+1)_{r-\nu}(N+\nu-x)_{r-\nu}\over (N-1+r)^{[r-\nu]}}
\sum_{k=r-\nu}^{N-1+r} {d_{r,k}\over k^{[r-\nu]}}
T_{k-r+\nu}^{r-\nu,r-\nu}(x,N+\nu).
$$
Равенство \eqref{sTpseq1.2.24.8} в развернутом виде принимает теперь следующий вид

$$\partial(x)=\Delta^\nu d(x-\nu)= \mathcal{D}_{2(r-\nu)-1,N+\nu}(\partial,x)+
$$
$$
(-1)^{r-\nu}
{(x+1)_{r-\nu}(N+\nu-x)_{r-\nu}\over (N-1+r)^{[r-\nu]}}
\sum_{k=r-\nu}^{N-1+r} {\partial_{r-\nu,k}\over
k^{[r-\nu]}}T_{k-r+\nu}^{r-\nu,r-\nu}(x,N+\nu)=
$$
$$
\mathcal{D}_{2(r-\nu)-1,N+\nu}(\partial,x)+
$$
\begin{equation}
\label{sTpseq1.2.24.12}
(-1)^{r-\nu}
{(x+1)_{r-\nu}(N+\nu-x)_{r-\nu}\over (N-1+r)^{[r-\nu]}}
\sum_{k=r-\nu}^{N-1+r} {d_{r,k}\over k^{[r-\nu]}}
T_{k-r+\nu}^{r-\nu,r-\nu}(x,N+\nu).
\end{equation}
Далее, взяв конечные разности  порядка $\nu$ от обеих частей равенства \eqref{sTpseq1.2.24.5} и
учитывая \eqref{siTpeq1.2.22.19}, имеем
%fixed
$$
\partial(x)=\Delta^\nu d(x-\nu)=\Delta^\nu\mathcal{D}_{2r-1,N}(d,x-\nu)+
$$
\begin{equation}
\label{sTpseq1.2.24.13}
{(-1)^{r-\nu}
(x+1)_{r-\nu}(N+\nu-x)_{r-\nu}\over (N-1+r)^{[r-\nu]}}
\sum_{k=r}^{N-1+r} {d_{r,k}\over
k^{[r-\nu]}}T_{k-r+\nu}^{r-\nu,r-\nu}(x,N+\nu).
\end{equation}
Сопоставляя \eqref{sTpseq1.2.24.12} с \eqref{sTpseq1.2.24.13}, мы замечаем, что
$$
\Delta^\nu\mathcal{D}_{2r-1,N}(d,x-\nu)=\mathcal{D}_{2(r-\nu)-1,N+\nu}(\partial,x)+
$$
\begin{equation}
\label{sTpseq1.2.24.14}
{(-1)^{r-\nu}
(x+1)_{r-\nu}(N+\nu-x)_{r-\nu}\over (N-1+r)^{[r-\nu]}}
\sum_{k=r-\nu}^{r-1} {d_{r,k}\over k^{[r-\nu]}}
T_{k-r+\nu}^{r-\nu,r-\nu}(x,N+\nu).
\end{equation}




\subsection{Операторы $\mathcal{Y}_{n+2r,N}(d)=\mathcal{Y}_{n+2r,N}(d,x)$}



Рассмотрим частичную сумму ряда \eqref{sTpseq1.2.24.5} следующего вида
\begin{equation}
\label{Ynrreq1.2.25.1}
\mathcal{Y}_{n+2r,N}(d,x)=\mathcal{D}_{2r-1,N}(x)+{(-1)^r (x+1)_r(N-x)_r\over
(N-1+r)^{[r]}} \sum_{k=r}^{n+r} {d_{r,k}\over
k^{[r]}} T_{k-r}^{r,r}(x,N),
\end{equation}
представляющую собой алгебраический полином степени $n+2r$.  Отметим
некоторые важные свойства оператора $\mathcal{Y}_{n+2r,N}(d)$. Если
$p_m=p_m(x)$ -- алгебраический полином степени $m\le n+2r$, то
$\Delta^rp_{m}(x)$ имеет степень $m-r\le n+r$, поэтому из \eqref{sTpseq1.2.24.4}
вытекает, что   $(p_m)_{r,k}=0$ при $k>n+r$. Отсюда  и из равенств
\eqref{sTpseq1.2.24.5} -- \eqref{sTpseq1.2.24.7}, \eqref{Ynrreq1.2.25.1} следует, что
\begin{equation}
\label{Ynrreq1.2.25.2}
\mathcal{Y}_{n+2r,N}(p_m,x) \equiv p_m(x)\quad(m\le n+2r).
\end{equation}


С другой стороны, из равенств \eqref{Ynrreq1.2.25.1}, \eqref{sTpseq1.2.24.5} -- \eqref{sTpseq1.2.24.7} непосредственно
вытекает, что полином $\mathcal{Y}_{n+2r,N}(d,x)$ интерполирует функцию
$d(x)$  в узлах множества $A=\{-r,-r+1,\ldots,-1\}\bigcup\{N,N+1,\ldots, N-1+r\}$, т.е.  мы  имеем
\begin{equation}\label{Ynrreq1.2.25.3}
\mathcal{Y}_{n+2r,N}(d,x)=d(x) \quad (x\in A ).
\end{equation}
Далее мы имеем
$(x\in \bar\Omega_{N+2r})$
\begin{multline}
\label{Ynrreq1.2.25.4}
d(x)-\mathcal{Y}_{n+2r,N}(d,x)=\\
{(-1)^r
(x+1)_r(N-x)_r\over (N-1+r)^{[r]}} \sum_{k=n+r+1}^{N-1+r}
{d_{r,k}\over  k^{[r]}} T_{k-r}^{r,r}(x,N)=\mathcal{R}_{n,N}^r(d,x).
\end{multline}

Пусть $0\le\nu\le r$, $-r\le t\le N-1+r-\nu$ (t -- целое). Тогда
     мы можем взять конечные разности порядка $\nu$ от обеих
     частей равенства \eqref{Ynrreq1.2.25.4}, что дает

$$
    \Delta^\nu d(t)- \Delta^\nu\mathcal{Y}_{n+2r,N}(d,t)=
\Delta^\nu\mathcal{R}_{n,N}^r(t)=
$$
\begin{equation}\label{Ynrreq1.2.25.5}
{(-1)^r\over(N-1+r)^{[r]}} \sum_{k=n+r+1}^{N-1+r}{d_{r,k}\over k^{[r]}}\Delta^\nu
\{(t+1)_r(N-t)_rT_{k-r}^{r,r}(t,N)\}.
\end{equation}




Если теперь воспользуемся равенством \eqref{siTpeq1.2.22.19}, то из \eqref{Ynrreq1.2.25.5}  приходим к следующему утверждению.
\vskip 0.5cm
\begin{theorem}\label{Yn2r:t1} Пусть $0\le\nu\le r$, $-r\le t\le N-1+r-\nu$ ($t$ -- целое). Тогда имеет место равенство
    $$
    \Delta^\nu d(t)- \Delta^\nu\mathcal{Y}_{n+2r,N}(d,t)=
    $$
    \begin{equation}\label{Ynrreq1.2.25.6}
    {(-1)^{r-\nu}(t+1+\nu)_{r-\nu}(N-t)_{r-\nu} \over(N-1+r)^{[r-\nu]}}
    \sum_{k=n+r+1}^{N-1+r}{d_{r,k}\over k^{[r-\nu]}} T_{k+\nu-r}^{r-\nu,r-\nu}(t+\nu,N+\nu).
    \end{equation}
\end{theorem}


Отметим еще следующее полезное равенство
\begin{multline}\label{Ynrreq1.2.25.7}
\Delta^\nu\mathcal{Y}_{n+2r,N}(d,t-\nu)=\mathcal{D}_{2(r-\nu)-1,N+\nu}(\partial ,t)+\\
{(-1)^{r-\nu}(t+1)_{r-\nu}(N+\nu-t)_{r-\nu}\over (N-1+r)^{[r-\nu]}}
\sum_{k=r-\nu}^{n+r} {(\partial)_{r-\nu,k}\over k^{[r-\nu]}}
T_{k-r+\nu}^{r-\nu,r-\nu}(t,N+\nu)=\\
\mathcal{Y}_{n+2r-\nu,N+\nu}(\partial,t),
\end{multline}
которое вытекает из \eqref{Ynrreq1.2.25.1}, \eqref{sTpseq1.2.24.14}, \eqref{sTpseq1.2.24.11} и \eqref{siTpeq1.2.22.19}.

Пусть $p_m(t)$ -- произвольный алгебраический полином степени $m\le n+2r-\nu$. Тогда в силу свойства \eqref{Ynrreq1.2.25.2}
$$
 \mathcal{Y}_{n+2r-\nu,N+\nu}(p_m,t)\equiv p_m(t),
$$
поэтому мы можем записать
$$
p_m(t)- \mathcal{Y}_{n+2r-\nu,N+\nu}(\partial,t)=\mathcal{Y}_{n+2r-\nu,N+\nu}(p_m-\partial,t).
$$
С учетом этого факта из \eqref{Ynrreq1.2.25.7}  имеем
$$
\Delta^\nu d(t-\nu)-\Delta^\nu\mathcal{Y}_{n+2r,N}(d,t-\nu)=\partial(t)-\mathcal{Y}_{n+2r-\nu,N+\nu}(\partial,t)
$$

\begin{equation}\label{Ynrreq1.2.25.8}
=\partial(t)- p_m(t)+\mathcal{Y}_{n+2r-\nu,N+\nu}(p_m-\partial,t).
\end{equation}
Для натуральных  $N$,  $m$  и $r$  положим
$$
v=v(t)=v_{r,m,N}(t)=\frac1{N+r}\sqrt{(t+r)(N+r-1-t)}+\frac1m.
$$
Через $C_v(\bar \Omega_{N+2(r-\nu)})$ обозначим нормированное пространство дискретных  функций $b(t)$, заданных на $\bar \Omega_{N+2(r-\nu)}$,  для которых норма определяется равенством
 $$
 \|b\|_v=\max_{t\in \bar\Omega_{N+2(r-\nu)}} {|b(t)|\over v^{r-\nu}(t)},
 $$
где $0\le \nu\le r$. Обозначим через $p_{n+2r-\nu}(\partial)=p_{n+2r-\nu}(\partial)(t)$ алгебраический полином степени $n+2r-\nu$, который совпадает с функцией  $\partial(t)$ в точках множества $A=\{-r+\nu,-r+\nu+1,\ldots,-1, N+\nu, N+\nu+1, \ldots, N+r-1\}$ и среди таких полиномов осуществляет наилучшее приближение к  $\partial(t)$ в нормированном  пространстве  $C_v(\bar \Omega_{N+2(r-\nu)})$. Положим $E_{n+2r-\nu}^v(\partial)=\|\partial- p_{n+2r-\nu}(\partial)\|_v$. Тогда
из \eqref{Ynrreq1.2.25.8} имеем
 $$
|\Delta^\nu d(t-\nu) -\Delta^\nu\mathcal{Y}_{n+2r,N}(d,t-\nu)|=|\partial(t)-\mathcal{Y}_{n+2r-\nu,N+\nu}(\partial,t)| \le
$$
$$
|\partial(t)-   p_{n+2r-\nu}(\partial)(t)|+|\mathcal{Y}_{n+2r-\nu,N+\nu}(p_{n+2r-\nu}(\partial)-\partial,t)|\le
$$
\begin{equation}\label{Ynrreq1.2.25.9}
v^{r-\nu}(t)E_{n+2r-\nu}^v(\partial)+|\mathcal{Y}_{n+2r-\nu,N+\nu}(p_{n+2r-\nu}(\partial)-\partial,t)|.
\end{equation}
С другой стороны, в силу \eqref{Ynrreq1.2.25.1}
$$
\mathcal{Y}_{n+2r-\nu,N+\nu}(p_{n+2r-\nu}(\partial)-\partial,t)=\mathcal{D}_{2(r-\nu)-1,N+\nu}(p_{n+2r-\nu}(\partial)-\partial,t)+
$$
$$
{(-1)^{r-\nu}(t+1)_{r-\nu}(N+\nu-t)_{r-\nu}\over (N-1+r)^{[r-\nu]}}
\sum_{k=r-\nu}^{n+r} {(p_{n+2r-\nu}(\partial)-\partial)_{r-\nu,k}\over k^{[r-\nu]}}
T_{k-r+\nu}^{r-\nu,r-\nu}(t,N+\nu).
$$
Поскольку, очевидно, что $\mathcal{D}_{2(r-\nu)-1,N+\nu}(p_{n+2r-\nu}(\partial)-\partial,t)\equiv 0$, то это равенство можно переписать так
$$
\mathcal{Y}_{n+2r-\nu,N+\nu}(p_{n+2r-\nu}(\partial)-\partial,t)=
$$
\begin{equation}\label{Ynrreq1.2.25.10}
{(-1)^{r-\nu}(t+1)_{r-\nu}(N+\nu-t)_{r-\nu}\over (N-1+r)^{[r-\nu]}}
\sum_{k=r-\nu}^{n+r} {(p_{n+2r-\nu}(\partial)-\partial)_{r-\nu,k}\over k^{[r-\nu]}}
T_{k-r+\nu}^{r-\nu,r-\nu}(t,N+\nu).
\end{equation}

Далее имеем
$$
(p_{n+2r-\nu}(\partial)-\partial)_{r-\nu,k}=
$$
$$
{2\over (N+r)h_{k,N+r}^{0,0}}\sum_{t=0}^{N+r-1}\Delta^{r-\nu}[p_{n+2r-\nu}(\partial)(j-r+\nu)-\partial(j-r+\nu)]T^{0,0}_k(j,N+r).
$$
Применим к правой части этого равенства преобразование Абеля $r-\nu$ раз. Тогда с учетом того, что разность  $p_{n+2r-\nu}(\partial)(j)-\partial(j)$
обращается в нуль во всех точках множества $A=\{-r+\nu,-r+\nu+1,\ldots,-1, N+\nu, N+\nu+1, \ldots, N+r-1\}$, мы находим
 $$
(p_{n+2r-\nu}(\partial)-\partial)_{r-\nu,k}=
$$
$$
{(-1)^{r-\nu}2\over (N+r)h_{k,N+r}^{0,0}}\sum_{j=0}^{N+r-1}[p_{n+2r-\nu}(\partial)(j)-\partial(j)]\Delta^{r-\nu}T^{0,0}_k(j,N+r).
$$
Обратимся теперь к формуле \eqref{siTpeq1.2.22.15}, из которой находим
$$
\Delta^{r-\nu}T^{0,0}_k(j,N+r)= {(k+1)_{r-\nu}T_{k-r+\nu}^{r-\nu,r-\nu}(j,N+\nu)\over (N+r-1)^{[r-\nu]}}.
$$
Подставляя это выражение в правую часть предыдущего равенства, получим
$$
(p_{n+2r-\nu}(\partial)-\partial)_{r-\nu,k}=
$$
\begin{equation}\label{Ynrreq1.2.25.11}
{(-1)^{r-\nu}2\over (N+r)h_{k,N+r}^{0,0}}\sum_{j=0}^{N+r-1}[p_{n+2r-\nu}(\partial)(j)-\partial(j)]{(k+1)_{r-\nu}T_{k-r+\nu}^{r-\nu,r-\nu}(j,N+\nu)\over (N+r-1)^{[r-\nu]}}.
\end{equation}
Из \eqref{Ynrreq1.2.25.10} и \eqref{Ynrreq1.2.25.11} находим
$$
\mathcal{Y}_{n+2r-\nu,N+\nu}(p_{n+2r-\nu}(\partial)-\partial,t)=
$$
$$
{(t+1)_{r-\nu}(N+\nu-t)_{r-\nu}\over (N-1+r)^{[r-\nu]}}
\sum_{k=r-\nu}^{n+r} {2T_{k-r+\nu}^{r-\nu,r-\nu}(t,N+\nu)\over k^{[r-\nu]}
 (N+r)h_{k,N+r}^{0,0}}\times
 $$
$$
\sum_{t=0}^{N+r-1}[p_{n+2r-\nu}(\partial)(j)-\partial(j)]{(k+1)_{r-\nu}T_{k-r+\nu}^{r-\nu,r-\nu}(j,N+\nu)\over (N+r-1)^{[r-\nu]}}=
$$
$$
{(t+1)_{r-\nu}(N+\nu-t)_{r-\nu}\over (N-1+r)^{[r-\nu]}}{2\over N+r}\sum_{j=0}^{N+r-1}[p_{n+2r-\nu}(\partial)(j)-\partial(j)]\times
$$
$$
\sum_{k=r-\nu}^{n+r} {(k+1)_{r-\nu}T_{k-r+\nu}^{r-\nu,r-\nu}(j,N+\nu)T_{k-r+\nu}^{r-\nu,r-\nu}(t,N+\nu)\over k^{[r-\nu]}.
 (N+r)h_{k,N+r}^{0,0}}
$$
Отсюда имеем

$$
|\mathcal{Y}_{n+2r-\nu,N+\nu}(p_{n+2r-\nu}(\partial)-\partial,t)|\le
$$
$$
{2(t+1)_{r-\nu}(N+\nu-t)_{r-\nu}\over (N-1+r)^{[r-\nu]}}\sum_{j=0}^{N+r-1}{|p_{n+2r-\nu}(\partial)(j)-\partial(j)|\over v^{r-\nu}(j)}\times
$$
\begin{equation}\label{Ynrreq1.2.25.12}
v^{\nu-r}(j)\left|\sum_{k=r-\nu}^{n+r} {(k+1)_{r-\nu}T_{k-r+\nu}^{r-\nu,r-\nu}(j,N+\nu)T_{k-r+\nu}^{r-\nu,r-\nu}(t,N+\nu)\over k^{[r-\nu]}.
 (N+r)h_{k,N+r}^{0,0}}\right|.
\end{equation}
Положим
$$
\lambda_{n,N,r,\nu}(t)={(t+1)_{r-\nu}(N+\nu-t)_{r-\nu}\over (N-1+r)^{[r-\nu]}}\times
$$
\begin{equation}\label{Ynrreq1.2.25.13}
{2\over N+r}\sum_{j=0}^{N+r-1}v^{\nu-r}(j)
\left|\sum_{k=r-\nu}^{n+r} {(k+1)_{r-\nu}T_{k-r+\nu}^{r-\nu,r-\nu}(j,N+\nu)T_{k-r+\nu}^{r-\nu,r-\nu}(t,N+\nu)\over k^{[r-\nu]}.
 h_{k,N+r}^{0,0}}\right|.
\end{equation}
Тогда неравенство \eqref{Ynrreq1.2.25.12} принимает следующий  вид
\begin{equation}\label{Ynrreq1.2.25.14}
|\mathcal{Y}_{n+2r-\nu,N+\nu}(p_{n+2r-\nu}(\partial)-\partial,t)|\le E_{n+2r-\nu}^v(\partial)\lambda_{n,N,r,\nu}(t),
\end{equation}
Из \eqref{Ynrreq1.2.25.9} и \eqref{Ynrreq1.2.25.14} выводим следующий результат.

%\textbf{ Теорема 5.2}. \textit{
\begin{theorem}\label{Yn2r:t2}
Пусть $0\le\nu\le r$, $-r+\nu\le t\le N-1+r$ (t -- целое). Тогда имеет место неравенство

\begin{equation}\label{Ynrreq1.2.25.15}
{|\Delta^\nu d(t-\nu) -\Delta^\nu\mathcal{Y}_{n+2r,N}(d,t-\nu)|\over v^{r-\nu}(t)} \le E_{n+2r-\nu}^v(\partial)\left(1+{\lambda_{n,N,r,\nu}(t)\over v^{r-\nu}(t)}\right).
\end{equation}
\end{theorem}
В связи с неравенством  \eqref{Ynrreq1.2.25.15} возникает важная задача о оценке величины $\lambda_{n,N,r,\nu}(t)$  при $n,N\to\infty$.


Перейдем к исследованию более тонких аппроксимативных свойств операторов   $\mathcal{Y}_{n+2r,N}(d)=\mathcal{Y}_{n+2r,N}(d,t)$ на классах дискретных функций и их модифицированных аналогов на классах функций, непрерывных на отрезке $[-1,1]$.



\subsection{Операторы $\mathcal{X}_{n+2r,N}(f)=\mathcal{X}_{n+2r,N}(f,x)$}

Пусть функция $f=f(x)$ определена в узлах сетки
     $H_\Lambda=\{x_j=-1+{2j\over\Lambda-1}\}_{j=0}^{\Lambda-1}$,
     где $\Lambda=N+2r$. С помощью равенства
     \begin{equation}\label{Xn2req1.2.26.1}
     d(j-r)=f(x_j)\quad(j=0,1,\ldots,N+2r-1)
\end{equation}
 мы можем  на сетке
$\bar \Omega_\Lambda =  \{-r,-r+1,\ldots,-1,0,1,\ldots,N-1,N,\ldots,N-1+r\}$ определить дискретную функцию
$d=d(t)$ и для нее построить оператор $\mathcal{Y}_{n+2r,N}(d)=\mathcal{Y}_{n+2r,N}(d,t)$. Тогда
  \begin{equation}\label{Xn2req1.2.26.2}
           \mathcal{X}_{n+2r,N}(f,x)=  \mathcal{Y}_{n+2r,N}
     \left(d,{\Lambda-1\over2}(1+x)-r\right)
  \end{equation}
представляет собой алгебраический полином степени $n+2r$, для которого
         \begin{equation}\label{Xn2req1.2.26.3}
           \mathcal{X}_{n+2r,N}(f,x_j)=f(x_j), \quad 0\le j\le r-1,\quad
          N+r\le j\le N-1+2r.
          \end{equation}
В частности, если $p_m(x)$ представляет собой алгебраический полином
степени  $m\le n+2r$, то из \eqref{Xn2req1.2.26.3} следует, что
\begin{equation}\label{Xn2req1.2.26.4}
\mathcal{X}_{n+2r,N}(p_m,x)=p_m(x)
\end{equation}
тождественно. Далее, полагая $t={\Lambda-1\over2}(1+x)-r$ и сопоставляя \eqref{Ynrreq1.2.25.1} с \eqref{Xn2req1.2.26.2}, мы
можем записать
\begin{equation}\label{Xn2req1.2.26.5}
\mathcal{X}_{n+2r,N}(f,x)=\mathcal{D}_{2r-1,N}(d,t)+{(-1)^r (t+1)_r(N-t)_r\over (N-1+r)^{[r]}}
\sum_{k=r}^{n+r} {f_{r,k}\over k^{[r]}} T_{k-r}^{r,r}(t,N),
 \end{equation}
где $f_{r,k}=d_{r,k}$,
\begin{equation}\label{Xn2req1.2.26.6}
\mathcal{D}_{2r-1,N}(d,t)=
\sum_{i=1}^r(-1)^{i-1}{(t+1)_r(N-t)_r\over
(i-1)!(r-i)!(N+i)_r}\left[{f(x_{r-i})\over t+i}+{f(x_{N-1+r+i})\over
N-1+i-t}\right].
\end{equation}

Из  \eqref{Xn2req1.2.26.2} и теоремы \ref{Yn2r:t1} непосредственно вытекает справедливость следующего утверждения.

%теорема 6.1
\begin{theorem}\label{Xn2r:t1} Пусть $t={\Lambda-1\over2}(1+x)-r$, $h={2\over \Lambda-1}$, $0\le\nu\le r$, $-r\le t\le N-1+r-\nu$ (t -- целое). Тогда имеет место равенство

    $$
     \Delta^\nu_h f(x)- \Delta^\nu_h\mathcal{X}_{n+2r,N}(f,x)=
     $$
    $$
{(-1)^{r-\nu}(t+1+\nu)_{r-\nu}(N-t)_{r-\nu} \over(N-1+r)^{[r-\nu]}}
\sum_{k=n+r+1}^{N-1+r}{f_{r,k}\over k^{[r-\nu]}} T_{k+\nu-r}^{r-\nu,r-\nu}(t+\nu,N+\nu),
    $$
где $\Delta^\nu_h g(x)$ есть $\nu$ -- тая степень оператора  конечной разности $\Delta_h g(x)=g(x+h)-g(x)$  с шагом $h$.
\end{theorem}

Через $C[-1,1]$ обозначим, как обычно, пространство непрерывных
функций, определенных на $[-1,1]$. Мы можем рассмотреть  $\mathcal{X}_{n+2r,N}(f)=\mathcal{X}_{n+2r,N}(f,x)$ как линейный оператор,
действующий в $C[-1,1]$ :$f\to \mathcal{X}_{n+2r,N}(f)$. Нашей целью
является изучение аппроксимативных свойств этих операторов, другими
словами, требуется исследовать поведение разности $|f(x)-\mathcal{X}_{n+2r,N}(f,x)|$ при определенных условиях на гладкость функции
$f(x)$.

Нам понадобятся некоторые обозначения. Среди алгебраических
полиномов $p_m(x)$ степени $m$, удовлетворяющих условию
\begin{equation}\label{Xn2req1.2.26.7}
f(x_j)=p_m(x_j),\quad j\in
\{0,1,\ldots,r-1\}\bigcup\{N+r,\ldots,N+2r-1\},
\end{equation}
 через $p_m^r(f)=p_{m,N}^r(f,x)$ и $q_m^r(f)=q_{m,N}^r(f,x)$ обозначим,
соответственно, полиномы,  для которых
\begin{equation}\label{Xn2req1.2.26.8}
E_m^r(f,N)=\inf_{p_m}\max_{x\in H_\lambda}{|f(x)-p_m(x)|\over(\sqrt{1-x^2}+1/m)^r}=\max_{x\in
H_\lambda}{|f(x)-p_m^r(f,x)|\over(\sqrt{1-x^2}+1/m)^r},
\end{equation}
\begin{equation}\label{Xn2req1.2.26.9}
\mathcal{E}_m^r(f,N)=\inf_{p_m}\max_{-1\le
x\le1}{|f(x)-p_m(x)|\over(\sqrt{1-x^2}+1/m)^r}=\max_{-1\le
x\le1}{|f(x)-q_m^r(f,x)|\over(\sqrt{1-x^2}+1/m)^r}.
\end{equation}
Учитывая \eqref{Xn2req1.2.26.4}, мы имеем
\begin{equation}
\label{Xn2req1.2.26.10}
f(x)-\mathcal{X}_{n+2r,N}(f,x)=f(x)-p_{n+2r}^r(f,x)+\mathcal{X}_{n+2r,N}(p_{n+2r}^r(f)-f,x),
\end{equation}
\begin{equation}
\label{Xn2req1.2.26.11}
 f(x)-\mathcal{X}_{n+2r,N}(f,x)=f(x)-q_{n+2r}^r(f,x)+\mathcal{X}_{n+2r,N}(q_{n+2r}^r(f)-f,x).
\end{equation}
 Сопоставляя \eqref{Xn2req1.2.26.10} и \eqref{Xn2req1.2.26.11} с \eqref{Xn2req1.2.26.3}, \eqref{Xn2req1.2.26.5}  и \eqref{Xn2req1.2.26.6}, мы замечаем, что
 \begin{equation}\label{Xn2req1.2.26.12}
\mathcal{X}_{n+2r,N}(p_{n+2r}^r(f)-f,x)={(-1)^r (t+1)_r(N-t)_r\over
(N-1+r)^{[r]}} \sum_{k=r}^{n+r} (p_{n+2r}^r(f)-f)_{r,k}
{T_{k-r}^{r,r}(t,N)\over k^{[r]}},
\end{equation}
 где (конечная разность $\Delta^r$ берется по переменной $j$ )
$$(p_{n+2r}^r(f)-f)_{r,k}= {2\over (N+r)h_{k,N+r}^{0,0}}
\sum_{j\in\Omega_{N+r}}T_k^{0,0}(j,N+r)\Delta^r(p_{n+2r}^r(f,x_j)-f(x_j))=
$$
$$ {2\over (N+r)h_{k,N+r}^{0,0}}
\sum_{j=0}^{N+r-1}T_k^{0,0}(j,N+r)\Delta^r(p_{n+2r}^r(f,x_j)-f(x_j)).
$$
 Отсюда, после $r$-кратного преобразования Абеля, учитывая равенства \eqref{Xn2req1.2.26.7},  находим
 $$
 (p_{n+2r}^r(f)-f)_{r,k}=
{(-1)^r2\over (N+r)h_{k,N+r}^{0,0}}
\sum_{j=0}^{N-1}\Delta^rT_k^{0,0}(j,N+r)(p_{n+2r}^r(f,x_{j+r})-f(x_{j+r})),
$$
 Воспользуемся  теперь формулой \eqref{siTpeq1.2.22.15}. Тогда последнее  равенство приобретет следующий\linebreak вид
$$
(p_{n+2r}^r(f)-f)_{r,k}= {(-1)^r2\over (N+r)h_{k,N+r}^{0,0}}
\sum_{j=0}^{N-1}{(k+1)_rT_{k-r}^{r,r}(j,N)\over (N+r-1)^{[r]}}
(p_{n+2r}^r(f,x_{j+r})-f(x_{j+r})).
 $$
 Подставляя это выражение в \eqref{Xn2req1.2.26.12}, мы получаем
  $$
  \mathcal{X}_{n+2r,N}(p_{n+2r}^r(f)-f,x)= {
(t+1)_r(N-t)_r\over (N-1+r)^{[r]}}{2\over N+r}\times
$$
$$
\sum_{j=0}^{N-1}(p_{n+2r}^r(f,x_{j+r})-f(x_{j+r}))\sum_{k=0}^n
{(k+r+1)_rT_k^{r,r}(j,N)T_k^{r,r}(t,N)\over
(k+r)^{[r]}(N+r-1)^{[r]}h_{k+r,N+r}^{0,0}}
$$
С другой стороны, учитывая \eqref{siTpeq1.2.22.6}, заметим,что
$$
(N+r)(N+r-1)^{[r]}h_{k+r,N+r}^{0,0}{(N+r-1)^{[r]}(k+r)^{[r]}\over(k+r+1)_r}=
$$
$$
(N+r)(N+r-1)^{[r]}{(N+k+2r)^{[k+r]}\over
(N+r-1)^{[k+r]}}{2\over 2k+2r+1}{(N+r-1)^{[r]}(k+r)^{[r]}\over(k+r+1)_r}=
$$
$$
N{(N+2r)^{[2r]}\over2^{2r}}h_{k,N}^{r,r},
$$
поэтому, принимая во внимание \eqref{siTpeq1.2.22.15}, предыдущее выражение принимает
окончательно следующий вид
$$
\mathcal{X}_{n+2r,N}(p_{n+2r}^r(f)-f,x)=
$$
$$ { (t+1)_r(N-t)_r2^{2r}\over (N+2r)^{[2r]}}{2\over
N} \sum_{j=0}^{N-1}(p_{n+2r}^r(f,x_{j+r})-f(x_{j+r}))\sum_{k=0}^n
{T_k^{r,r}(j,N)T_k^{r,r}(t,N)\over h_{k,N}^{r,r}}=
$$
 \begin{equation}\label{Xn2req1.2.26.13}
 {(t+1)_r(N-t)_r2^{2r}\over (N+2r)^{[2r]}}{2\over N}
\sum_{j=0}^{N-1}(p_{n+2r}^r(f,x_{j+r})-f(x_{j+r}))D_{n,N}^{r,r}(j,t).
\end{equation}
 Совершенно аналогично мы выводим
 $$
 \mathcal{X}_{n+2r,N}(q_{n+2r}^r(f)-f,x)=
 $$
 \begin{equation}\label{Xn2req1.2.26.14}
 {(t+1)_r(N-t)_r2^{2r}\over(N+2r)^{[2r]}}{2\over N}
\sum_{j=0}^{N-1}(q_{n+2r}^r(f,x_{j+r})-f(x_{j+r}))D_{n,N}^{r,r}(j,t).
\end{equation}
 Если мы примем во внимание \eqref{Xn2req1.2.26.8} и \eqref{Xn2req1.2.26.9}, то из \eqref{Xn2req1.2.26.13} и \eqref{Xn2req1.2.26.14} можем вывести следующие оценки:
 $$
 |\mathcal{X}_{n+2r,N}(p_{n+2r}^r(f)-f,x)|\le
E_{n+2r}^r(f,N)\times
$$
\begin{equation}\label{Xn2req1.2.26.15}
 {|(t+1)_r(N-t)_r|2^{2r}\over(N+2r)^{[2r]}}{2\over N}
\sum_{j=0}^{N-1}\left(\sqrt{1-x_{j+r}^2}+{1\over
n+2r}\right)^r\left|D_{n,N}^{r,r}(j,t)\right|,
\end{equation}
$$|\mathcal{X}_{n+2r,N}(q_{n+2r}^r(f)-f,x)|\le \mathcal{E}_{n+2r}^r(f,N)\times
$$
\begin{equation}\label{Xn2req1.2.26.16}
{|(t+1)_r(N-t)_r|2^{2r}\over (N+2r)^{[2r]}}{2\over N}
\sum_{j=0}^{N-1}\left(\sqrt{1-x_{j+r}^2}+{1\over
n+2r}\right)^r\left|D_{n,N}^{r,r}(j,t)\right|.
\end{equation}
Из \eqref{Xn2req1.2.26.10}, \eqref{Xn2req1.2.26.11}, \eqref{Xn2req1.2.26.15} и \eqref{Xn2req1.2.26.16} мы получаем следующие оценки
$$
{|f(x)-\mathcal{X}_{n+2r,N}(f,x)|\over(\sqrt{1-x^2}+{1\over
n+2r})^{r-1/2}}\le {|f(x)-p_{n+2r}^r(f,x)|\over(\sqrt{1-x^2}+{1\over
n+2r})^{r-1/2}}+
$$
$$
E_{n+2r}^r(f,N)
{|(t+1)_r(N-t)_r|2^{2r}\over(\sqrt{1-x^2}+{1\over n+2r})^{r-1/2}
(N+2r)^{[2r]}}\times
$$
$$
{2\over N}\sum_{j=0}^{N-1}\left(\sqrt{1-x_{j+r}^2}+{1\over
n+2r}\right)^r\left|D_{n,N}^{r,r}(j,t)\right|,
$$
 $$
 {|f(x)-\mathcal{X}_{n+2r,N}(f,x)|\over(\sqrt{1-x^2}+{1\over n+2r})^{r-1/2}}\le
{|f(x)-q_{n+2r}^r(f,x)|\over(\sqrt{1-x^2}+{1\over n+2r})^{r-1/2}}+
$$
$$\mathcal{E}_{n+2r}^r(f,N)
{|(t+1)_r(N-t)_r|2^{2r}\over(\sqrt{1-x^2}+{1\over n+2r})^{r-1/2}
(N+2r)^{[2r]}}\times
$$
\begin{equation}\label{Xn2req1.2.26.18}
{2\over N}
\sum_{j=0}^{N-1}\left(\sqrt{1-x_{j+r}^2}+{1\over
n+2r}\right)^r\left|D_{n,N}^{r,r}(j,t)\right|,
\end{equation}
Отсюда, с учетом \eqref{Xn2req1.2.26.8} и \eqref{Xn2req1.2.26.9} имеем
$$
{|f(x)-\mathcal{X}_{n+2r,N}(f,x)|\over(\sqrt{1-x^2}+{1\over n+2r})^{r}}\le
$$
\begin{equation}\label{Xn2req1.2.26.19}
E_{n+2r}^r(f,N)\left(1
+{I_{n,N}^r(x)|(t+1)_r(N-t)_r|2^{2r}\over(\sqrt{1-x^2}+{1\over
n+2r})^r (N+2r)^{[2r]}}\right)\quad (x\in H_\Lambda),
\end{equation}
$$
{|f(x)-\mathcal{X}_{n+2r,N}(f,x)|\over(\sqrt{1-x^2}+{1\over
n+2r})^{r}}\le
 $$
\begin{equation}\label{Xn2req1.2.26.20}
\mathcal{E}_{n+2r}^r(f,N)\left(1
+{I_{n,N}^r(x)|(t+1)_r(N-t)_r|2^{2r}\over(\sqrt{1-x^2}+{1\over
n+2r})^r (N+2r)^{[2r]}}\right) \quad (x\in [-1,1]),
\end{equation}
 где ($t={\Lambda-1\over2}(1+x)-r$, $\Lambda=N+2r$ )
 \begin{equation}\label{Xn2req1.2.26.21}
 I_{n,N}^r(x)=
{2\over N} \sum_{j=0}^{N-1}\left(\sqrt{1-x_{j+r}^2}+{1\over
n+2r}\right)^r\left|D_{n,N}^{r,r}(j,t)\right|.
\end{equation}
Неравенства \eqref{Xn2req1.2.26.19} и \eqref{Xn2req1.2.26.20} сводят задачу об оценке разности
$|f(x)-\mathcal{X}_{n+2r,N}(f,x)|$ к исследованию поведения величины
$I_{n,N}^r(x)$ при $n,N\to \infty$, аналогично тому,  как задача об
оценке отклонения частичных сумм Фурье от исходной функции сводится
к исследованию  поведения их функции Лебега. В работе \cite{idprm99} установлен  следующий результат.

%теорема 6.2
\begin{theorem}\label{Xn2r:t2} Пусть $r\ge 1$, $a>0$, $1\le n\le a\sqrt{N}
$, $-1\le x \le1$. Тогда имеет место оценка
$$
I_{n,N}^r(x)\le
c(r,a)\left(\sqrt{1-x^2}+{1\over
n}\right)^{-r}\left(\left(\sqrt{1-x^2}+\frac1n\right)^{-1/2}+\ln(n\sqrt{1-x^2}+1)\right).
$$
\end{theorem}
Из теоремы \ref{Xn2r:t2} и неравенств \eqref{Xn2req1.2.26.19} и \eqref{Xn2req1.2.26.20} вытекает

\begin{corollary}\label{Xn2r:col1} Пусть $r\ge 1$, $a>0$, $1\le n\le
a\sqrt{N} $, $-1\le x \le1$. Тогда
$${|f(x)-\mathcal{X}_{n+2r,N}(f,x)|\over(\sqrt{1-x^2}+{1\over n+2r})^{r-1/2}}\le
$$
\begin{equation}\label{Xn2req1.2.26.22}
 c(r,a)E_{n+2r}^r(f,N)\left(1 + \left(\sqrt{1-x^2}+{1\over
n}\right)^{1/2}\ln(n\sqrt{1-x^2}+1)\right) \quad (x\in
H_\Lambda),
\end{equation}
 $$
 {|f(x)-\mathcal{X}_{n+2r,N}(f,x)|\over(\sqrt{1-x^2}+{1\over n+2r})^{r-1/2}}\le
$$
\begin{equation}
\label{Xn2req1.2.26.23}
c(r,a)\mathcal{E}_{n+2r}^r(f,N)\left(1 + \left(\sqrt{1-x^2}+{1\over
n}\right)^{1/2}\ln(n\sqrt{1-x^2}+1)\right) \quad (x\in [-1,1]).
\end{equation}
\end{corollary}

Пусть, по-прежнему,  $d(j)=f(x_{j+r})$ $(j\in \bar \Omega_\Lambda)$.  Для $t\in \{-r+\nu,\ldots,-1,0,\ldots,N-1,N,\ldots,N+r-1\}$ положим $d_\nu=d_\nu(t)=d(t)=f(x_{t+r})$ и введем оператор $\mathcal{X}^\nu_{n+2r-\nu,N+\nu}(f)=\mathcal{X}^\nu_{n+2r-\nu,N+\nu}(f,x)$, полагая для
 $ t={\Lambda-1\over2}(1+x)-r$
$$
\mathcal{X}^\nu_{n+\nu+2(r-\nu),N+\nu}(f,x)=\mathcal{Y}_{n+\nu+2(r-\nu),N+\nu}(d_\nu,t)=
\mathcal{D}_{2(r-\nu)-1,N+\nu}(d_\nu,t)+
$$
\begin{equation}\label{Xn2req1.2.26.24}
{(-1)^{r-\nu}(t+1)_{r-\nu}(N+\nu-t)_{r-\nu}\over (N-1+r)^{[r-\nu]}}
\sum_{k=r-\nu}^{n+r} {(d_\nu)_{r-\nu,k}\over k^{[r-\nu]}}
T_{k-r+\nu}^{r-\nu,r-\nu}(t,N+\nu),
\end{equation}
где в силу \eqref{sTpseq1.2.24.4}
$$
 (d_\nu)_{r-\nu,k}    ={2\over (N+r)h_{k,N+r}^{0,0}}
     \sum_{t\in\Omega_{N+r}}\Delta^{r-\nu}d_\nu(t-r+\nu)T_k^{0,0}(t,N+r)=
    $$
     \begin{equation}\label{Xn2req1.2.26.25}
     {2\over (N+r)h_{k,N+r}^{0,0}}
     \sum_{t\in\Omega_{N+r}}\Delta^{r-\nu}_hf(x_{t+\nu})T_k^{0,0}(t,N+r)=f^\nu_{r-\nu,k},
    \end{equation}

$$
\mathcal{D}_{2(r-1)-1,N+\nu}(d_\nu,t)=
$$
\begin{equation}\label{Xn2req1.2.26.26}
\sum_{i=1}^{r-\nu}(-1)^{i-1}{(t+1)_{r-\nu}(N-t)_{r-\nu}\over
(i-1)!(r-\nu-i)!(N+\nu+i)_{r-\nu}}\left[{f(x_{r-i})\over t+i}+{f(x_{N+\nu-1+r+i})\over
N+\nu-1+i-t}\right].
\end{equation}



Из определения \eqref{Xn2req1.2.26.24} следует, что оператор
$$
\mathcal{X}^\nu_{n+\nu+2(r-\nu),N+\nu}(f)=\mathcal{X}^\nu_{n+\nu+2(r-\nu),N+\nu}(f,x)
$$
является проектором на пространство алгебраических полиномов $p_m(x)$ степени $m\le n+2r-\nu$, т.е.
\begin{equation}\label{Xn2req1.2.26.27}
\mathcal{X}^\nu_{n+\nu+2(r-\nu),N+\nu}(p_m,x)\equiv p_m(x)\quad ( m\le n+2r-\nu).
\end{equation}
Кроме того, имеют место следующие равенства
\begin{equation}\label{Xn2req1.2.26.27}
\mathcal{X}^\nu_{n+\nu+2(r-\nu),N+\nu}(f,x_j)=f(x_j), \nu\le j\le r-1, N+r+\nu\le j\le N+2r-1.
\end{equation}



Положим $\psi(x)=\Delta^\nu_hf(x-\nu h)$ и рассмотрим функцию  $\partial(t)=\Delta^\nu d(t-\nu)=\Delta^\nu_hf(x_{t-\nu+r})=\psi(x_{t+r})$, для которой  в силу \eqref{Ynrreq1.2.25.8} мы можем записать
$$
\Delta_h^\nu f(x_{j-\nu+r})-\Delta_h^\nu\mathcal{X}_{n+2r,N}(f,x_{j-\nu+r})=
$$
$$
\Delta^\nu d(j-\nu)-\Delta^\nu\mathcal{Y}_{n+2r,N}(d,j-\nu)=\partial(j)-\mathcal{Y}_{n+\nu+2(r-\nu),N+\nu}(\partial,j)=
$$
$$
\psi(x_{j+r})-\mathcal{X}^\nu_{n+\nu+2(r-\nu),N+\nu}(\psi,x_{j+r})
$$
или, что то же,
\begin{equation}\label{Xn2req1.2.26.28}
\Delta_h^\nu f(x_{j-\nu})-\Delta_h^\nu\mathcal{X}_{n+2r,N}(f,x_{j-\nu})=
\psi(x_{j})-\mathcal{X}^\nu_{n+\nu+2(r-\nu),N+\nu}(\psi,x_{j}).
\end{equation}


Через $\mathcal{P}_m^{r,\nu}$ обозначим пространство алгебраических полиномов  $p_m(x)$ степени $m$, удовлетворяющих условию
\begin{equation}\label{Xn2req1.2.26.29}
\psi(x_j)=p_m(x_j),\quad j\in
\{\nu,\ldots,r-1\}\bigcup\{N+r+\nu,\ldots,N+2r-1\},
\end{equation}
а через  $q_m^{r,\nu}(\psi)=q_{m,N}^{r,\nu}(\psi,x)$ обозначим полином из $\mathcal{P}_m^{r,\nu}$,
 для которого

\begin{multline}\label{Xn2req1.2.26.30}
 E_m^{r,\nu}(\psi,N)=\inf_{p_m\in\mathcal{P}_m^{r,\nu}}\max_{j\in \Omega_\Lambda}{|\psi(x_j)-p_m(x_j)|\over\left(\sqrt{1-x_j^2}+\frac1m\right)^{r-\nu}}=\\
 \max_{j\in \Omega_\Lambda}{|\psi(x_j)-q_m^{r,\nu}(\psi,x_j)|\over\left(\sqrt{1-x_j^2}+\frac1m\right)^{r-\nu}}.
\end{multline}

Тогда, учитывая \eqref{Xn2req1.2.26.26}, мы имеем
\begin{multline}\label{Xn2req1.2.26.31}
\psi(x)-\mathcal{X}^\nu_{n+\nu+2(r-\nu),N+\nu}(\psi,x)=\\
\psi(x)-q_{m,N}^{r,\nu}(\psi,x)+\mathcal{X}^\nu_{n+\nu+2(r-\nu),N+\nu}(q_{m,N}^{r,\nu}(\psi)-\psi,x).
\end{multline}
Далее заметим, что если в равенстве  \eqref{Xn2req1.2.26.26} функцию $f(x)$ заменить функцией $q_{m,N}^{r,\nu}(\psi,x)-\psi(x)$,  то в силу
 \eqref{Xn2req1.2.26.29} будем  иметь $\mathcal{D}_{2(r-1)-1,N+\nu}(d_\nu,t)\equiv0$, поэтому из \eqref{Xn2req1.2.26.24} находим
$$
\mathcal{X}^\nu_{n+\nu+2(r-\nu),N+\nu}(q_{m,N}^{r,\nu}(\psi)-\psi,x)=
$$
\begin{equation}\label{Xn2req1.2.26.32}
{(-1)^{r-\nu}(t+1)_{r-\nu}(N+\nu-t)_{r-\nu}\over (N-1+r)^{[r-\nu]}}
\sum_{k=r-\nu}^{n+r} {(q_{m,N}^{r,\nu}(\psi)-\psi)_{r-\nu,k}^\nu\over k^{[r-\nu]}}
T_{k-r+\nu}^{r-\nu,r-\nu}(t,N+\nu),
\end{equation}
где в силу \eqref{Xn2req1.2.26.25}
$$
 (q_{m,N}^{r,\nu}(\psi)-\psi)_{r-\nu,k}^\nu=
 $$
 \begin{equation}\label{Xn2req1.2.26.33}
    {2\over (N+r)h_{k,N+r}^{0,0}}
     \sum_{j=0}^{N+r-1}T_k^{0,0}(j,N+r)\Delta^{r-\nu}(q_{m,N}^{r,\nu}(\psi,x_{j+\nu})-\psi(x_{j+\nu})),
 \end{equation}
причем конечная разность $\Delta^{r-\nu}$ берется по переменной $t$. Применим к правой части равенства \eqref{Xn2req1.2.26.33}
преобразование Абеля $r-\nu$-раз, тогда в силу равенств \eqref{Xn2req1.2.26.29}, которым удовлетворяет полином $p_m(x)=q_{m,N}^{r,\nu}(\psi,x)$, получим
  $$
 (q_{m,N}^{r,\nu}(\psi)-\psi)_{r-\nu,k}^\nu=
 $$
 \begin{equation}\label{Xn2req1.2.26.34}
    {2(-1)^{r-\nu}\over (N+r)h_{k,N+r}^{0,0}}
     \sum_{j=0}^{N+r-1}(q_{m,N}^{r,\nu}(\psi,x_{j+r})-\psi(x_{j+r}))\Delta^{r-\nu}T_k^{0,0}(j,N+r).
 \end{equation}
 Отсюда с учетом равенства \eqref{siTpeq1.2.22.15} находим
$$
 (q_{m,N}^{r,\nu}(\psi)-\psi)_{r-\nu,k}^\nu=
 $$
 \begin{equation}\label{Xn2req1.2.26.35}
    {2(-1)^{r-\nu}\over (N+r)h_{k,N+r}^{0,0}}
     \sum_{j=0}^{N+\nu-1}{(k+1)_{r-\nu}T_{k-r+\nu}^{r-\nu,r-\nu}(j,N+\nu)\over (N+r-1)^{[r-\nu]}}(q_{m,N}^{r,\nu}(\psi,x_{j+r})-\psi(x_{j+r})).
 \end{equation}
Подставляя это выражение в \eqref{Xn2req1.2.26.32}, мы получаем
$$
\mathcal{X}^\nu_{n+\nu+2(r-\nu),N+\nu}(q_{m,N}^{r,\nu}(\psi)-\psi,x)=
{(t+1)_{r-\nu}(N+\nu-t)_{r-\nu}\over (N-1+r)^{[r-\nu]}}
\sum_{k=r-\nu}^{n+r} {2\over (N+r)k^{[r-\nu]}h_{k,N+r}^{0,0}}\times
$$
$$
        \sum_{j=0}^{N+\nu-1}{(k+1)_{r-\nu}T_{k-r+\nu}^{r-\nu,r-\nu}(j,N+\nu)\over (N+r-1)^{[r-\nu]}}(q_{m,N}^{r,\nu}(\psi,x_{j+r})-\psi(x_{j+r}))
T_{k-r+\nu}^{r-\nu,r-\nu}(t,N+\nu)=
$$
$$
{2(t+1)_{r-\nu}(N+\nu-t)_{r-\nu}\over(N+r) (N-1+r)^{[r-\nu]}}\sum_{j=0}^{N+\nu-1}(q_{m,N}^{r,\nu}(\psi,x_{j+r})-\psi(x_{j+r}))\times
$$
\begin{equation}\label{Xn2req1.2.26.36}
\sum_{k=0}^{n+\nu} {(k+r-\nu+1)_{r-\nu}T_{k}^{r-\nu,r-\nu}(j,N+\nu)T_{k}^{r-\nu,r-\nu}(t,N+\nu)\over (k+r-\nu)^{[r-\nu]}h_{k+r-\nu,N+r}^{0,0}(N+r-1)^{[r-\nu]}}.
\end{equation}
С другой стороны, учитывая \eqref{siTpeq1.2.22.6}, заметим,что
$$
(N+r)(N+r-1)^{[r-\nu]}h_{k+r-\nu,N+r}^{0,0}{(N+r-1)^{[r-\nu]}(k+r-\nu)^{[r-\nu]}\over(k+r-\nu+1)_{r-\nu}}=
$$
$$
(N+r)(N+r-1)^{[r-\nu]}{(N+\nu+k+2(r-\nu))^{[k+r-\nu]}\over
(N+r-1)^{[k+r-\nu]}}\times
$$
$$
{2\over 2k+2(r-\nu)+1}{(N+r-1)^{[r-\nu]}(k+r-\nu)^{[r-\nu]}\over(k+r-\nu+1)_{r-\nu}}=
$$
$$
(N+\nu){(N+\nu+2(r-\nu))^{[2(r-\nu)]}\over2^{2(r-\nu)}}h_{k,N+\nu}^{r-\nu,r-\nu},
$$
поэтому, принимая во внимание \eqref{siTpeq1.2.22.6}, предыдущее выражение принимает
окончательно следующий вид
$$
\mathcal{X}^\nu_{n+\nu+2(r-\nu),N+\nu}(q_{m,N}^{r,\nu}(\psi)-\psi,x)=
$$
$$
{2^{2(r-\nu)}(t+1)_{r-\nu}(N+\nu-t)_{r-\nu}\over (N+\nu+2(r-\nu))^{2[r-\nu]}}\sum_{j=0}^{N+\nu-1}(q_{m,N}^{r,\nu}(\psi,x_{j+r})-\psi(x_{j+r}))\times
$$
$$
\sum_{k=0}^{n+\nu} {T_{k}^{r-\nu,r-\nu}(j,N+\nu)T_{k}^{r-\nu,r-\nu}(t,N+\nu)\over h_{k,N+\nu}^{r-\nu,r-\nu}}=
$$
\begin{equation}\label{Xn2req1.2.26.37}
{2^{2(r-\nu)}(t+1)_{r-\nu}(N+\nu-t)_{r-\nu}\over (N+\nu+2(r-\nu))^{2[r-\nu]}}\sum_{j=0}^{N+\nu-1}(q_{m,N}^{r,\nu}(\psi,x_{j+r})-\psi(x_{j+r}))D_{n+\nu,N+\nu}^{r-\nu,r-\nu}(j,t).
\end{equation}
 Если мы примем во внимание \eqref{Xn2req1.2.26.30}, то для $m\le n+2r-\nu $ из \eqref{Xn2req1.2.26.37} можем вывести следующую  оценку:
\begin{multline}\label{Xn2req1.2.26.38}
 |\mathcal{X}^\nu_{n+\nu+2(r-\nu),N+\nu}(q_{m,N}^{r,\nu}(\psi)-\psi,x)|\le \\
 E_m^{r,\nu}(\psi,N){|(t+1)_{r-\nu}(N+\nu-t)_{r-\nu}|2^{2(r-\nu)+1}\over(N+\nu+2(r-\nu)^{[2(r-\nu)]}(N+\nu)} \times \\
\times\sum_{j=0}^{N+\nu
-1}\left(\sqrt{1-x_{j+r}^2}+{1\over m}\right)^{r-\nu}\left|D_{n+\nu,N+\nu}^{r-\nu,r-\nu}(j,t)\right|.
\end{multline}
Положим $m=n+2r-\nu$, тогда из  \eqref{Xn2req1.2.26.31} и \eqref{Xn2req1.2.26.38} находим
$$
|\psi(x)-\mathcal{X}^\nu_{n+\nu+2(r-\nu),N+\nu}(x)|\le
$$
\begin{equation}\label{Xn2req1.2.26.39}
|\psi(x)-q_{m,N}^{r,\nu}(\psi,x)| +{|(t+1)_{r-\nu}(N+\nu-t)_{r-\nu}|2^{2(r-\nu)}\over(N+\nu+2(r-\nu)^{[2(r-\nu)]}}E_m^{r,\nu}(\psi,N)I_{n,N}^{r,\nu}(x),
\end{equation}
где $t=\frac{\Lambda-1}{2}(1+x)-r$,
\begin{equation}\label{Xn2req1.2.26.40}
 I_{n,N}^{r,\nu}(x)=\frac{2}{N+\nu}
\sum_{j=0}^{N+\nu-1}\left(\sqrt{1-x_{j+r}^2}+{1\over m}\right)^{r-\nu}\left|D_{n+\nu,N+\nu}^{r-\nu,r-\nu}(j,t)\right|.
\end{equation}
Неравенство \eqref{Xn2req1.2.26.39} сводит задачу об оценке отклонения полинома $\mathcal{X}^\nu_{n+\nu+2(r-\nu),N+\nu}(x)$ от функции $\psi(x)$
 к вопросу об оценке величины $I_{n,N}^{r,\nu}(x)$. Но этот вопрос, по сути,  был уже рассмотрен в теореме \ref{Xn2r:t2} и мы  можем здесь отметить следующий результат.

%\textbf{ Теорема 6.2A.} \textit{
\begin{theorem}\label{Xn2r:t3}
Пусть $r\ge 1$, $0\le \nu\le r-1$, $a>0$, $1\le n\le a\sqrt{N}
$, $-1\le x \le1$. Тогда имеет место оценка
$$
I_{n,N}^{r,\nu}(x)\le
c(r,a)\left(\sqrt{1-x^2}+{1\over
n}\right)^{-r+\nu}\left(\left(\sqrt{1-x^2}+\frac1n\right)^{-1/2}+\ln(n\sqrt{1-x^2}+1)\right).
$$
\end{theorem}

Из теоремы \ref{Xn2r:t3} с учетом равенства \eqref{Xn2req1.2.26.28} и неравенства \eqref{Xn2req1.2.26.39} мы выводим

%\textbf{ Следствие 6.2.} \textit{
\begin{corollary}\label{Xn2r:col2}
Пусть $r\ge 1$, $0\le \nu\le r-1$, $a>0$, $1\le n\le a\sqrt{N}
$. Тогда имеет место оценка
$$
{|\Delta_h^\nu f(x_{j-\nu})-\Delta_h^\nu\mathcal{X}_{n+2r,N}(f,x_{j-\nu})|\over\left(\sqrt{1-x_{j-\nu}^2}+{1\over
n}\right)^{r-\nu-\frac12}}\le
$$
\begin{equation}\label{Xn2req1.2.26.41}
c(r,a)E_m^{r,\nu}(\psi,N)\left(1+\left(\sqrt{1-x_{j-\nu}^2}+\frac1n\right)^{1/2}\ln\left(n\sqrt{1-x_{j-\nu}^2}+1\right)\right).
\end{equation}
\end{corollary}
\subsection{Численное дифференцирование\\ посредством операторов $\mathcal{X}_{n+2r,N}(f)$ }
Пусть $1\le r$, $2\le N$ -- натуральные числа, функция $f=f(x)$ -- дифференцируема на $[-1,1]$ $r$-раз. Далее, пусть заданы значения $f(x_j)$  в узлах сетки  $H_\Lambda=\{x_j=-1+{2j\over\Lambda-1}\}_{j=0}^{\Lambda-1}$,  где $\Lambda=N+2r$. Ставится задача найти приближенные значения производных $f^{(\nu)}(x_j)$ при $\nu=1,\ldots, r$ посредством операторов  $\mathcal{X}_{n+2r,N}(f)$ . Идея применения операторов
$\mathcal{X}_{n+2r,N}(f)$  в задаче численного дифференцирования функций возникает в связи с результатами, отмеченными выше, в частности, с     оценкой \eqref{Xn2req1.2.26.41}. Дело в том, что оценке \eqref{Xn2req1.2.26.41} можно придать несколько иной вид, поделив ее правую и левую  части на $h^\nu=(\frac{2}{\Lambda-1})^\nu$, а именно
$$
{|\frac1{h^{\nu}}\Delta_h^\nu\left[ f(x_{j-\nu})-\mathcal{X}_{n+2r,N}(f,x_{j-\nu})\right]|\over\left(\sqrt{1-x_{j-\nu}^2}+{1\over
n}\right)^{r-\nu-\frac12}}\le
$$
\begin{equation}\label{ndXn2req1.2.27.1}
c(r,a)E_m^{r,\nu}(\frac1{h^\nu}\psi,N)\left(1+\left(\sqrt{1-x_{j-\nu}^2}+\frac1n\right)^{1/2}\ln\left(n\sqrt{1-x_{j-\nu}^2}+1\right)\right).
\end{equation}
Теперь воспользуемся известным свойством конечной разности, согласно которому в интервале $(x_{j-\nu},x_j)$ найдется такая точка $y_j=x_{j-\nu}+ \theta_j \nu h$ ($0<\theta<1$), для которой $\frac1{h^\nu}\Delta_h^\nu g(x_{j-\nu})= g^{(\nu)}(y_j)$. Применяя это равенство к функции $f(x_{j-\nu})-\mathcal{X}_{n+2r,N}(f,x_{j-\nu})$, из \eqref{ndXn2req1.2.27.1}  получим
$$
{| f^{(\nu)}(y_j)-\mathcal{X}^{(\nu)}_{n+2r,N}(f,y_j)|\over\left(\sqrt{1-x_{j-\nu}^2}+{1\over
n}\right)^{r-\nu-\frac12}}\le
$$

\begin{equation}\label{ndXn2req1.2.27.2}
c(r,a)E_m^{r,\nu}(\frac1{h^\nu}\psi,N)\left(1+\left(\sqrt{1-x_{j-\nu}^2}+\frac1n\right)^{1/2}\ln\left(n\sqrt{1-x_{j-\nu}^2}+1\right)\right),
\end{equation}
где $y_j=x_{j-\nu}+ \theta_j \nu h$ ($0<\theta<1$) для всех $j$ таких, что $\nu\le j \le N+2r-1$. Нетрудно заметить, что если мы в оценке \eqref{ndXn2req1.2.27.2} заменим точки  $x_j$ на $y_j=x_{j-\nu}+ \theta_j \nu h$, то она останется справедливой (возможно с другой константой $c(r,a)$) и примет следующий вид
$$
{| f^{(\nu)}(y_j)-\mathcal{X}^{(\nu)}_{n+2r,N}(f,y_j)|\over\left(\sqrt{1-y_j^2}+{1\over n}\right)^{r-\nu-\frac12}}\le
$$
\begin{equation}\label{ndXn2req1.2.27.3}
c(r,a)E_m^{r,\nu}(\frac1{h^\nu}\psi,N)\left(1+\left(\sqrt{1-y_j^2}+\frac1n\right)^{1/2}\ln\left(n\sqrt{1-y_j^2}+1 \right)\right),
\end{equation}
где $r\ge 1$, $0\le \nu\le r-1$, $a>0$, $1\le n\le a\sqrt{N}$. Кроме того, заметим, что $\frac1{h^\nu}\psi(x_{j})=\frac1{h^\nu}\Delta^\nu f(x_{j-\nu})=f^{(\nu)}(z_j)$, где $z_j=x_{j-\nu}+\eta_j\nu h$ ($0<\eta_j<1$).

Оценка \eqref{ndXn2req1.2.27.3} показывает, что операторы $\mathcal{X}_{n+2r,N}(f)$ успешно могут быть использованы в задаче одновременного приближения функций и их нескольких производных. Для приложений важно то, что конструкция операторов $\mathcal{X}_{n+2r,N}(f)$ базируется на массиве  значений функции $f(x)$ в узлах равномерной сетки   $H_\Lambda=\{x_j=-1+\frac{2j}{\Lambda-1}\}$. С другой стороны, с точки зрения численной реализации $\mathcal{X}_{n+2r,N}(f,x)$, когда   $x$ пробегает некоторую достаточно густую сетку $Q\subset[-1,1]$, возникает неудобство, заключающееся в том, что не удается непосредственно использовать алгоритм быстрого дискретного преобразования Фурье для одновременного вычисления всех значений
$\mathcal{X}_{n+2r,N}(f,x)$ для $x\in Q$. Мы рассмотрим эту задачу более подробно в случае, когда конечное множество $Q$ состоит из всех нулей
полинома Чебышева $C_M(x)=\cos(M\arccos x)$, т.е. $Q=Q_M=\{t_i=\cos\frac{(2i+1)\pi}{2M}\}_{i=0}^{M-1}$. Для этого представим полином $\mathcal{X}_{n+2r,N}(f,x)$ в следующем виде
\begin{equation}\label{ndXn2req1.2.27.4}
\mathcal{X}_{n+2r,N}(f,x)=\frac{a_0}{2}+\sum_{k=1}^{n+2r}a_kC_k(x),
\end{equation}
где
\begin{equation}\label{ndXn2req1.2.27.5}
a_k =\frac{2}{n+2r+1}\sum_{j=0}^{n+2r}\mathcal{X}_{n+2r,N}(f,\cos\tau_j)\cos k\tau_j,
\end{equation}
 $\tau_j= \frac{(2j+1)\pi}{2(n+2r+1)}$. Справедливость равенства \eqref{ndXn2req1.2.27.4} вытекает из следующего соотношения ортогональности для полиномов Чебышева $C_k(x)$ $(0\le k\le n+2r)$:
\begin{equation}\label{ndXn2req1.2.27.6}
\frac{2}{n+2r+1}\sum_{j=0}^{n+2r}C_k(\cos\tau_j)C_l(\cos\tau_j)=\begin{cases} 0, \text{если $k\ne l$,}\\
2, \text{если $k=l=0$,}\\1,\text{если $k=l>0$}.\end{cases}
\end{equation}
Заметим, что для численного нахождения значений $\mathcal{X}_{n+2r,N}(f,\cos\tau_j)$, фигурирующих в формуле \eqref{ndXn2req1.2.27.5}, мы можем воспользоваться равенством \eqref{Xn2req1.2.26.5}, которое слегка преобразуем следующим образом. Используя ортогональность при $k\ge r$  полинома $T_k^{0,0}(j,N+r)$ к полиному $\mathcal{D}_{2r-1,N}(d,j-r)$, из \eqref{sTpseq1.2.24.1} -- \eqref{sTpseq1.2.24.4} имеем
  $$
 f_{r,k}=d_{r,k}   ={2\over (N+r)h_{k,N+r}^{0,0}}
     \sum_{j\in\Omega_{N+r}}T_k^{0,0}(j,N+r)\Delta^rd(j-r)=
     $$
     $$
     {2\over (N+r)h_{k,N+r}^{0,0}}
     \sum_{j\in\Omega_{N+r}}T_k^{0,0}(j,N+r)\Delta^r[d(j-r)-\mathcal{D}_{2r-1,N}(d,j-r)].
    $$
 Отсюда, воспользовавшись преобразованием Абеля $r$-раз и учитывая равенство \eqref{siTpeq1.2.22.15}, находим
  $$
f_{r,k}={(-1)^r2\over (N+r)h_{k,N+r}^{0,0}}
     \sum_{j\in\Omega_N}[d(j)-\mathcal{D}_{2r-1,N}(d,j)]\Delta^rT_k^{0,0}(j,N+r)=
$$
$$
{(-1)^r2\over (N+r)h_{k,N+r}^{0,0}}{(k+1)_r\over (N+r-1)^{[r]}}
     \sum_{j\in\Omega_N}[d(j)-\mathcal{D}_{2r-1,N}(d,j)] T_{k-r}^{r,r}(j,N).
$$
откуда, полагая
$$
\bar f_{r,k}=
{2\over (N+r)}\sum_{j\in\Omega_N}[d(j)-\mathcal{D}_{2r-1,N}(d,j)] T_k^{r,r}(j,N),
$$
получаем
$$
{f_{r,k+r}\over (k+r)^{[r]}} ={(-1)^r\bar f_{r,k}\over h^{0,0}_{k+r,N+r}}{(k+r+1)_r\over (k+r)^{[r]} (N+r-1)^{[r]}}=
$$
$$
{(-1)^r\bar f_{r,k}(N+r-1)^{[k+r]}\over(N+2r+k)^{[k+r]}}{(k+r+1)_r\over (k+r)^{[r]} (N+r-1)^{[r]}}{2k+2r+1\over2}=
$$
$$
{(-1)^r\bar f_{r,k}(N+r-1)^{[r]}\over(N+2r)^{[r]}}{(N-1)^{[k]}\over(N+2r+k)^{[k]}}{2^{2r}\over(N+r-1)^{[r]}}
   {k!(k+2r)!\over (k+r)!^2}{2k+2r+1\over2^{2r+1}}=
$$
$$
\bar f_{r,k}{(-1)^r2^{2r}\over(N+2r)^{[r]}}{1\over h^{r,r}_{k,N}}.
$$
С учетом этого равенства из  \eqref{Xn2req1.2.26.5} мы выводим
$$
\mathcal{X}_{n+2r,N}(f,x)=\mathcal{D}_{2r-1,N}(d,t)+{2^{2r}(t+1)_r(N-t)_r\over (N-1+r)^{[r]}(N+2r)^{[r]}}
\sum_{k=0}^n \bar f_{r,k}{T_k^{r,r}(t,N)\over h^{r,r}_{k,N}}.
 $$
Далее, если мы введем обозначение
\begin{equation}\label{ndXn2req1.2.27.7}
\hat f_{r,k}={\bar f_{r,k}\over( h^{r,r}_{k,N})^\frac12}=\sum_{j\in\Omega_N}[d(j)-\mathcal{D}_{2r-1,N}(d,j)] \tau_k^{r,r}(j,N),
\end{equation}
то окончательно получим следующее представление $(t=\frac{N+2r-1}{2}(1+x)-r)$
$$
\mathcal{X}_{n+2r,N}(f,x)=\mathcal{D}_{2r-1,N}(d,t)+{2^{2r+1}(t+1)_r(N-t)_r\over N(N+r)^{[r]}(N+2r)^{[r]}}
\sum_{k=0}^n \hat f_{r,k}\tau_k^{r,r}(t,N)=
 $$
\begin{equation}\label{ndXn2req1.2.27.8}
\mathcal{D}_{2r-1,N}(d,t)+\mu(t;r,r,N)\sum_{k=0}^n \hat f_{r,k}\tau_k^{r,r}(t,N),
\end{equation}
где $\mu(t;r,r,N)$ -- весовая функция, определенная равенством \eqref{siTpeq1.2.22.5}. Для численного нахождения значений полинома $\mathcal{X}_{n+2r,N}(f,x)$ с помощью равенства \eqref{ndXn2req1.2.27.8} можно использовать рекуррентное соотношение
\eqref{siTpeq1.2.22.27}, которое в случае целых $\alpha=\beta=r$ принимает следующий вид
$$
\tau_0^{r,r}(t,N)=\left[\frac{(2r+1)!}{2^{2r+1}r!^2}\right]^\frac12,
$$
$$
\tau_1^{r,r}(t,N)=\left\{\frac{(N-1)(2r+1)!(2r+3)}{(N+2r+1)r!^22^{2r+1}}\right\}^\frac12\left[\frac{2t}{N-1}-1\right],
$$

\begin{equation}\label{ndXn2req1.2.27.9}
\tau_n^{r,r}(t)=
\hat\kappa_n(t-\frac{N-1}{2})\tau_{n-1}^{r,r}(t)-\hat\gamma_n\tau_{n-2}^{r,r}(t),
\end{equation}
где
$$
\hat\kappa_n=2\left[\frac{(2n+2r-1)(2n+2r+1)}{(N+n+2r)
(N-n)n(n+2r)}\right]^\frac12,
$$

$$
\hat\gamma_n=\left[\frac{N+n+2r-1}{N+n+2r}
\frac{N-n+1}{N-n}\frac{n-1}n\frac{n+2r-1}{n+2r}\frac{2n+2r+1}{2n+2r-3}\right]^\frac12.
$$



Если элементы массива $\{X_j=\mathcal{X}_{n+2r,N}(f,\cos\tau_j)\}_{j=0}^{n+2r}$ уже найдены, то мы можем найти массив  коэффициентов $a_k$ $(0\le k\le n+2r)$ из равенств
$$
\mathcal{X}_{n+2r,N}(f,\cos\tau_j)=\frac{a_0}{2}+\sum_{k=1}^{n+2r}a_k\cos k\tau_j \quad(0\le j\le n+2r),
$$
получающихся из \eqref{ndXn2req1.2.27.4} путем подстановки  $\cos\tau_j$ вместо $x$. Для этого достаточно к массиву $\{X_j\}_{j=0}^{n+2r}$ применить прямое быстрое дискретное косинус-преобразование Фурье.





Вернемся теперь к вопросу о  вычислении значений полинома $\mathcal{X}_{n+2r,N}(f,x)$ в узлах сетки $Q_M=\{t_i=\cos\frac{(2i+1)\pi}{2M}\}_{i=0}^{M-1}$ при  $M\ge n+2r$. Полагая для $n+2r+1\le k\le M-1$  $a_k=0$, в силу \eqref{ndXn2req1.2.27.4} мы можем записать
\begin{equation}\label{ndXn2req1.2.27.10}
\mathcal{X}_{n+2r,N}(f,t_i)=\frac{a_0}{2}+\sum_{k=0}^{M-1}a_kC_k(t_i)=\frac{a_0}{2}+\sum_{k=0}^{M-1}a_k\cos \frac{k(2i+1)\pi}{2M}.
\end{equation}
Поэтому для получения массива значений полинома  $\mathcal{X}_{n+2r,N}(f,x)$ в узлах сетки $Q_M=\{t_i=\cos\frac{(2i+1)\pi}{2M}\}_{i=0}^{M-1}$
достаточно применить к массиву коэффициентов $\{a_k\}_{k=0}^{M-1}$ обратное быстрое дискретное косинус--преобразование Фурье.

Перейдем к вопросу о вычислении на сетке $Q_M=\{t_i=\cos\frac{(2i+1)\pi}{2M}\}_{i=0}^{M-1}$ производных от полинома $\mathcal{X}_{n+2r,N}(f,x)$.
Представим  $\mathcal{X}^{(\nu)}_{n+2r,N}(f,x)$  в следующем виде

\begin{equation}\label{ndXn2req1.2.27.11}
\mathcal{X}^{(\nu)}_{n+2r,N}(f,x)=\frac{a^\nu_0}{2}+\sum_{k=1}^{n+2r-\nu}a^\nu_kC_k(x),
\end{equation}
где
\begin{equation}\label{ndXn2req1.2.27.12}
a^\nu_k =\frac{2}{n+2r+1}\sum_{j=0}^{n+2r}\mathcal{X}^{(\nu)}_{n+2r,N}(f,\cos\tau_j)\cos k\tau_j,
\end{equation}
Чтобы найти значения $\mathcal{X}^{(\nu)}_{n+2r,N}(f,\cos\tau_j)$ мы  вновь обратимся к равенству \eqref{ndXn2req1.2.27.4} и запишем
\begin{equation}\label{ndXn2req1.2.27.13}
\mathcal{X}^{(\nu)}_{n+2r,N}(f,x)=\sum_{k=\nu}^{n+2r}a_kC^{(\nu)}_k(x),
\end{equation}
где $C_k^{(\nu)}(x)$ -- $\nu$-- тая производная полинома Чебышева $C_k(x)$. В связи с \eqref{ndXn2req1.2.27.11} возникает задача о выводе рекуррентных соотношений между $C_{k-2}^{(\nu)}(x)$,  $C_{k-1}^{(\nu)}(x)$ и $C_k^{(\nu)}(x)$.
 С этой целью воспользуемся рекуррентной формулой для
ультрасферических полиномов Якоби \cite{idprm9}:
    \begin{equation}\label{ndXn2req1.2.27.14}
     P_k^{\alpha,\alpha}(x)=
     {(2k+2\alpha-1)(k+\alpha)\over(k+2\alpha)k}
     xP_{k-1}^{\alpha,\alpha}(x)-
{(k+\alpha-1)(k+\alpha)\over(k+2\alpha)k}
P_{k-2}^{\alpha,\alpha}(x)
     \end{equation}
   и  равенством
         \begin{equation}\label{ndXn2req1.2.27.15}
     \left(C_{k+\nu}(x)\right)^{(\nu)}=
     {(2^{k+\nu}(k+\nu)!)^2(k+\nu)_\nu\over (2(k+\nu))!2^\nu}
     P_k^{\nu-1/2,\nu-1/2}(x).
          \end{equation}
Сопоставляя \eqref{ndXn2req1.2.27.11} с \eqref{ndXn2req1.2.27.12}, мы приходим к следующим
соотношениям: для $k<\nu$ $\left(C_k(x)\right)^{(\nu)}=0$; кроме
того, для $\nu\ge1$ $\left(C_\nu(x)\right)^{(\nu)}=2^{\nu-1}\nu!$,
$\left(C_{\nu+1}(x)\right)^{(\nu)}=2^\nu(\nu+1)!x$ и
 \begin{equation}\label{ndXn2req1.2.27.16}
\left(C_k(x)\right)^{(\nu)}={k\over k-\nu}2x
\left(C_{k-1}(x)\right)^{(\nu)}-{k(k+\nu-2)\over
(k-\nu)(k-2)}\left(C_{k-2}(x)\right)^{(\nu)}\quad (k>\nu+1).
 \end{equation}
Если элементы массива $\{X^\nu_j=\mathcal{X}^{(\nu)}_{n+2r,N}(f,\cos\tau_j)\}_{j=0}^{n+2r}$ уже найдены, то мы можем найти массив  коэффициентов $a^\nu_k$ $(0\le k\le n+2r-\nu)$ из равенств
$$
X^\nu_j=\frac{a^\nu_0}{2}+\sum_{k=1}^{n+2r-\nu}a^\nu_k\cos k\tau_j \quad(0\le j\le n+2r),
$$
получающихся из \eqref{ndXn2req1.2.27.8} путем подстановки  $\cos\tau_j$ вместо $x$. Для этого достаточно к массиву $\{X^\nu_j\}_{j=0}^{n+2r}$ применить прямое быстрое дискретное косинус-преобразование Фурье.



Рассмотрим теперь вопрос о вычислении значений полинома $\mathcal{X}^{(\nu)}_{n+2r,N}(f,x)$ в узлах сетки $Q_M=\{t_i=\cos\frac{(2i+1)\pi}{2M}\}_{i=0}^{M-1}$ при  $M\ge n+2r-\nu$. Из \eqref{ndXn2req1.2.27.8}, полагая для $n+2r-\nu+1\le k\le M-1$  $a^\nu_k=0$,  мы можем вывести
\begin{equation}\label{ndXn2req1.2.27.17}
\mathcal{X}^{(\nu)}_{n+2r,N}(f,t_i)=\frac{a^\nu_0}{2}+\sum_{k=1}^{M-1}a^\nu_k\cos \frac{k(2i+1)\pi}{2M},
\end{equation}
поэтому для получения массива значений полинома  $\mathcal{X}^{(\nu)}_{n+2r,N}(f,x)$ в узлах сетки \\ $Q_M=\{t_i=\cos\frac{(2i+1)\pi}{2M}\}_{i=0}^{M-1}$ достаточно применить к массиву коэффициентов  $\{a^\nu_k\}_{k=0}^{M-1}$ обратное быстрое дискретное косинус-преобразование Фурье.

