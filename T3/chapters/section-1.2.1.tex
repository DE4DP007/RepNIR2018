\section{Аппроксимативные свойства специальных рядов со свойством прилипания по ультрасферическим полиномам.} \label{sect-2.1}

\textit{
Предельные ультрасферические ряды
$$
\sum_{k=0}^\infty f_k^{-1}\hat P_k^{-1}(x),
$$
были введены в работах \cite{shii1,shii2} %warning
 совсем недавно путем предельного перехода \linebreak $f_k^{-1}\hat P_k^{-1}(x)=\lim_{\alpha\to-1} f_k^{\alpha}\hat P_k^{\alpha}(x)$,
где $\sum_{k=0}^\infty f_k^{\alpha}\hat P_k^{\alpha}(x)$ -- ряд Фурье-Якоби функции $f\in C[-1,1]$ по ортонормированным ультрасферическим полиномам Якоби $\hat P_n^\alpha(x)=\hat P_n^{\alpha,\alpha}(x)$ ($\alpha>-1$).
Пользуясь явным видом  ряда $\sum_{k=0}^\infty f_k^{-1}\hat P_k^{-1}(x)$, установленным нами в \cite{shii1},  в настоящей параграфе рассмотрены новые более  общие специальные  ряды по  ультрасферическим  полиномам Якоби их аппроксимативные свойства. Показано, что эти   ряды, как аппарат приближения непрерывных функций выгодно  отличается от рядов Фурье по полиномам Якоби, обладая в тоже время простой конструкцией, допускающей в важных частных
 случаях  применение быстрого преобразования Фурье для численной реализации частичных  сумм рассматриваемых рядов.
}

%\subsection{ Введение}
Представление функций в виде рядов по тем или иным ортонормированным системам с целью последующего их приближения
частичными суммами выбранного ортогонального ряда является, пожалуй, одним из самых часто применяемых подходов в теории приближений и ее приложениях. Наряду с задачами математической физики, для решения которых указанный подход является традиционным, появились и продолжают появляться все новые важные задачи, для решения которых также все чаще применяются методы, основанные на представлении функций(сигналов) в виде рядов по подходящим ортонормированным системам (см., например, \cite{dedus3, pash4, arush5, tref6, tref7, muku8, malvarSign}). При этом часто возникает такая ситуация, когда функция (сигнал, временной ряд, изображение и.т.д) $f=f(t)$ задана на достаточно длинном промежутке $[0,T]$ и нам требуется разбить этот промежуток на части $[a_j,a_{j+1}]$ $(j=0,1,\ldots,m)$, рассмотреть отдельные фрагменты функции определенные на этих частичных отрезках, представить их в виде рядов по выбранной ортонормированной системе и аппроксимировать каждый такой фрагмент частичными суммами соответствующего ряда. Такая ситуация является типичной для задач, связанных с решением нелинейных дифференциальных уравнений численно-аналитическими методами \cite{pash4, tref6}, обработкой временных рядов и изображений и других \cite{arush5,tref6,tref7}, в которых
возникает необходимость разбить заданный ряд данных на части,
аппроксимировать каждую часть и заменить приближенно исходный
временный ряд (изображение) функцией, полученной в результате
<<пристыковки>> функций, аппроксимирующих отдельные части. Но тогда в
местах <<стыка>> возникают нежелательные разрывы (артефакты) (см.\cite{muku8}), которые искажают общий вид временного ряда (изображения). Такая
картина непременно возникает при использовании для приближения
<<кусков>> исходной функции сумм Фурье по классическим
ортонормированным системам. Не является исключением и случай
приближения суммами Фурье $S_n^\alpha(f,x)$ по ультрасферическим
полиномам Якоби $\hat P_n^{\alpha}(x)=\hat P_n^{\alpha,\alpha}(x)$
при $\alpha>-1$. Указанные выше разрывы в точках <<стыка>> возникают
из-за того, что суммы Фурье $S_n^\alpha(f,x)$ не всегда совпадают с
исходной функцией $f(x)$ в точках $x=-1$ и $x=1$. С другой стороны,
проанализировав асимптотические свойства полиномов $\hat
P_n^{\alpha}(x)(x)$ вблизи концов $\pm1$,  можно заметить,
что суммы Фурье $S_n^\alpha(f,x)$ по этим полиномам имеют тенденцию
стремиться к $f(x)$ в точках $x=\pm1$ при $\alpha\to -1$, т.е. $S_n^{-1}(f,\pm1)=f(\pm1)$,
 где $S_n^{-1}(f,x)=\lim_{\alpha\to-1}S_n^\alpha(f,x)$.
Вопрос о том, что собой представляет $S_n^{-1}(f,x)$ и нельзя ли использовать $S_n^{-1}(f,x)$ в качестве
 альтернативного суммам Фурье-Якоби $S_n^{\alpha}(f,x)$ аппарата приближения непрерывных и гладких функций
 $f(x)$, был рассмотрен в работе \cite{shii1}, в которой были  исследованы
аппроксимативные свойства предельных сумм $S_n^{-1}(f,x)$ и, в частности, подробно доказано, что $S_n^{-1}(f,\pm1)=f(\pm1)$ для любой непрерывной на $[-1,1]$ функции $f(x)$. Следует сразу отметить, что полиномы $\hat P_n^{-1}(x)=\lim_{\alpha\to-1}
\hat P_n^\alpha(x)$ не образуют ортонормированной системы на
$[-1,1]$, более того, в частности, $\hat
P_1^{-1}(x)\equiv0$. Тем не менее, как было  показано в \cite{shii1},
$S_n^{-1}(f,x)$ является проектором на подпространство
алгебраических полиномов степени $n$. Для функции Лебега
$\Lambda_n(x)$ частичных сумм $S_n^{-1}(f,x)$ в \cite{shii1} получена
оценка $\Lambda_n(x)\le c\ln(2+n\sqrt{1-x^2})$ $(-1\le x\le1)$,
которая показывает, что $S_n^{-1}(f)=S_n^{-1}(f,x)$ как аппарат
приближения непрерывных и гладких функций в метрике пространства
$C[-1,1]$ не уступает суммам Фурье-Чебышева по полиномам Чебышева
первого рода $T_n(x)=\cos(n\arccos x)$. Более того, вблизи концов $\pm 1$ $S_n^{-1}(f,x)$ приближают функцию $f(x)$
лучше, чем суммы  Фурье-Чебышева.

Отправляясь от результатов, полученных в работе \cite{shii1}, в данном параграфе вводятся {\em специальные ряды} по ультрасферическим полиномам Якоби и исследуются их аппроксимативные свойства.


Пусть $\alpha>-1$, $\mu^\alpha(x)=(1-x^2)^\alpha$,
$\hat P_n^\alpha(x)$ $(n=0,1,\ldots)$ ортонормированная с весом $\mu^\alpha(x)$ система ультрасферических полиномов
Якоби, т.е.
\begin{equation}\label{eq.1.2.4.1.1}
\int\limits\limits_{-1}^1\mu^\alpha(x)\hat P_n^{\alpha}(x)
\hat P_k^{\alpha}(x)dx=\delta_{nk}.
\end{equation}

Для произвольной непрерывной  на $[-1,1]$ функции $f(x)$ мы можем определить коэффициенты Фурье-Якоби
\begin{equation}\label{FYSerKoeffs}
f_k^\alpha=\int\limits_{-1}^1f(t)\hat
P_k^{\alpha}(t)\mu^{\alpha}(t)dt\,\, (k=0,1,\ldots),
\end{equation}
ряд Фурье-Якоби
\begin{equation}\label{eq.1.2.4.1.3}
f\sim \sum _{k=0}^\infty f_k^\alpha \hat P_k^\alpha(x)
\end{equation}
и сумму Фурье-Якоби

\begin{equation}\label{intreq.1.2.4.1.4}
S_n^\alpha(f,x)= \sum _{k=0}^n f_k^\alpha \hat P_k^\alpha(x).
\end{equation}
Однако, если $\alpha=-1$, то  функция $\mu^{-1}(t)$ становится неинтегрируемой на $(-1,1)$ и, как следствие, равенство \eqref{FYSerKoeffs} теряет смысл. Но тем не менее, как было показано в работе \cite{shii1}, можно рассмотреть предельный ряд
\begin{equation}\label{LimSerPS}
f\sim \sum_{k=0}^\infty f_k^{-1}\hat P_k^{-1}(x),
\end{equation}
 в котором

$$
f_k^{-1}\hat P_k^{-1}(x)=\lim_{\alpha\to-1}f_k^{\alpha}\hat P_k^{\alpha}(x).
$$
Частичную сумму ряда \eqref{LimSerPS} обозначим через $S_n^{-1}(f,x)$, т.е.
$$
S_n^{-1}(f,x)= \sum _{k=0}^n f_k^{-1} \hat P_k^{-1}(x).
$$
В \cite{shii1} доказано, что
\begin{equation}\label{eq.1.2.4.1.6}
S_n^{-1}(f,x)=c_f(x)+(1-x^2)\sum_{k=0}^{n-2}\hat g_k\hat P_k^1(x),
\end{equation}
где
\begin{equation}\label{eq.1.2.4.1.7}
c_f(x)={f(-1)+f(1)\over 2}+{f(1)-f(-1)\over
2}x,\quad
g(x)=f(x)-c_f(x),
\end{equation}
\begin{equation}\label{eq.1.2.4.1.8}
\hat g_k=\int\limits_{-1}^1g(t)\hat P_k^{1}(t)dt.
\end{equation}
Обозначим через $C[-1,1]$ пространство функций $f=f(x)$, заданных и непрерывных на $[-1,1]$, для которых норма  $\|f\|$  определяется обычным образом, т.е. $\|f\|=\max_{-1\le x\le1}|f(x)|$.
Пусть $E_n(f)$ означает наилучшее приближение функции $f\in C[-1,1] $
алгебраическими полиномами $p_n(x)$ степени $n$. Из результатов, полученных в \cite{shii1} вытекает, что
  \begin{equation}\label{estimLimSer}
 |f(x)-S_n^{-1}(f,x)|\le cE_n(f)(1+\ln(1+n\sqrt{1-x^2})).
 \end{equation}
Из \eqref{estimLimSer}, в свою очередь, следует, что если $\lim_{n\to\infty}E_n(f)\ln n=0$, то $S_n^{-1}(f,x)$ сходится к  $f(x)$ равномерно на $[-1,1]$ и, следовательно,
\begin{equation}\label{eqtimLimSer1}%\label{eq.1.2.4.1.10}
f(x)=c_f(x)+(1-x^2)\sum_{k=0}^\infty\hat g_k\hat P_k^1(x).
\end{equation}
Для $x\in (-1,1)$ равенство \eqref{eqtimLimSer1} можно переписать так
\begin{equation}\label{eqtimLimSer2}%\label{eq.1.2.4.1.11}
F(x)={f(x)-c_f(x)\over 1-x^2}=\sum_{k=0}^\infty\hat g_k\hat P_k^1(x).
\end{equation}
Правая часть равенства \eqref{eqtimLimSer2} представляет собой ряд Фурье-Якоби функции $F(x)$ по ортонормированной системе полиномов Якоби $\hat P_k^1(x)$ $(k=0,1,\ldots)$ и, следовательно,
\begin{equation}\label{eqtimLimSer3}%\label{eq.1.2.4.1.12}
\hat g_k=\int\limits_{-1}^1F(t)(1-t^2)\hat P_k^1(t)dt
=\int\limits_{-1}^1g(t)\hat P_k^1(t)dt.
\end{equation}
Равенство \eqref{eqtimLimSer3} показывает, что предельный ряд \eqref{eqtimLimSer1} может быть получен путем разложения функции $F(x)={f(x)-c_f(x)\over 1-x^2}$
в ряд Фурье по полиномам Якоби $\hat P_k^1(x)$ $(k=0,1,\ldots)$. Кроме того, равенство \eqref{eqtimLimSer2} допускает следующее обобщение предельного ряда \eqref{eqtimLimSer1}. А именно, пусть $\alpha>0$,  $\hat P_k^\alpha(x)$ $(k=0,1,\ldots)$ -- ортонормированная система полиномов Якоби с весом $\mu^\alpha(x)$. Тогда для функции $F(x)={f(x)-c_f(x)\over 1-x^2}$ мы можем определить коэффициенты Фурье-Якоби
\begin{equation}\label{eqtimLimSer3}%\label{eq.1.2.4.1.13}
F_k^\alpha=\int\limits_{-1}^1F(t)(1-t^2)^\alpha\hat P_k^\alpha(t)dt=
\int\limits_{-1}^1g(t)(1-t^2)^{\alpha-1}\hat P_k^\alpha(t)dt=\hat g_k^\alpha.
\end{equation}
и ряд Фурье-Якоби
$$
F(x)\sim \sum_{k=0}^\infty\hat g_k^\alpha\hat P_k^\alpha(x).
$$
Предположим, что этот ряд сходится при $x\in (-1,1)$ и имеет место равенство
\begin{equation}\label{eqtimLimSer4}%\label{eq.1.2.4.1.14}
F(x)= \sum_{k=0}^\infty\hat g_k^\alpha\hat P_k^\alpha(x),
\end{equation}
которое можно переписать так
\begin{equation}\label{eqtimLimSer5}%\label{eq.1.2.4.1.15}
f(x)=c_f(x)+(1-x^2) \sum_{k=0}^\infty\hat g_k^\alpha\hat P_k^\alpha(x).
\end{equation}
Ряд \eqref{eqtimLimSer5} будем называть {\em специальным рядом} по полиномам Якоби $\hat P_k^\alpha(x)$. В частности, при $\alpha=1$ специальный ряд \eqref{eqtimLimSer5} совпадает с предельным рядом \eqref{eqtimLimSer1}.

Другой важный частный случай специального ряда \eqref{eqtimLimSer5} мы получим при  $\alpha=1/2$, а именно,
\begin{equation}\label{eqtimLimSer6}%\label{eq.1.2.4.1.16}
f(x)=c_f(x)+(1-x^2) \sum_{k=0}^\infty\hat g_k^{1/2}\hat P_k^{1/2}(x).
\end{equation}
Положим $x=\cos\theta$, $a_\Phi(\theta)=c_f(\cos\theta)$, $\Phi(\theta)=f(\cos\theta)$. Тогда из \eqref{eqtimLimSer6} получим
\begin{equation}\label{eqtimLimSer7}%\label{eq.1.2.4.1.17}
 \Phi(\theta)=a_\Phi(\theta)+\sin^2\theta \sum_{k=0}^\infty\hat g_k^{1/2}\hat P_k^{1/2}(\cos\theta).
\end{equation}
Поскольку
\begin{equation}\label{eqtimLimSer8}%\label{eq.1.2.4.1.18}
\hat P_k^{1/2}(\cos\theta)=\sqrt{\frac2\pi}{\sin(k+1)\theta\over\sin\theta},
\end{equation}
то
\begin{equation}\label{eqtimLimSer9}%\label{eq.1.2.4.1.19}
\hat g_k^{1/2}=
\int\limits_{-1}^1g(t)(1-t^2)^{-1/2}\hat P_k^{1/2}(t)dt=\sqrt{\frac2\pi}
\int\limits_{0}^\pi g(\cos\tau){\sin(k+1)\tau\over\sin\tau}d\tau.
\end{equation}
%WARNING номер ссылки странный 323-44 в тексте
Из \eqref{eqtimLimSer7}-\eqref{eqtimLimSer9} находим
\begin{equation}\label{eqtimLimSer10}
 \Phi(\theta)=a_\Phi(\theta)+\sin\theta \sum_{k=1}^\infty\varphi_k\sin k \theta,
\end{equation}
где
\begin{equation}\label{eqtimLimSer11}
\varphi_k=\sqrt{\frac2\pi}\hat g_{k-1}^{1/2}={\frac2\pi}
\int\limits_{0}^\pi \varphi(\tau){\sin k\tau\over\sin\tau}d\tau,
\end{equation}
\begin{equation}\label{eqtimLimSer12}
\varphi(\theta)=\Phi(\theta)-a_\Phi(\theta)=\Phi(\theta)-{\Phi(0)+\Phi(\pi)\over2}-
{\Phi(0)-\Phi(\pi)\over2}\cos\theta.
\end{equation}

Обозначим через $\sigma_n^\alpha(f,x)$
частичную сумму ряда \eqref{eqtimLimSer5} (см.\eqref{apsseq.1.2.4.3.1}). Как показано в \S 3 $\sigma_n^\alpha(f,x)$ является проектором на подпространство $H^n$ всех алгебраических полиномов $p_n(x)$ степени не выше $n$, важной особенностью которого является тот факт, что $\sigma_n^\alpha(f,x)$
на концах отрезка $[-1,1]$ совпадает с исходной функцией $f(x)$, т.е.
\begin{equation}\label{eqtimLimSer13}
\sigma_n^\alpha(f,\pm1)=f(\pm1).
\end{equation}
Это свойство операторов $\sigma_n^\alpha(f)=\sigma_n^\alpha(f,x)$ имеет важное значение в задачах, связанных с обработкой временных рядов и изображений.  Другое важное свойство операторов $\sigma_n^\alpha(f)=\sigma_n^\alpha(f,x)$ заключается в том, что при $1/2\le \alpha <3/2$, как это показано в настоящем параграфе (см. следствие \ref{apss:col1} ), имеет место оценка
\begin{equation}\label{eqtimLimSer14}%\label{eq.1.2.4.1.24}
|f(x)-\sigma_n^\alpha(f,x)|\le c(\alpha)E_n(f)(1+\ln(1+n\sqrt{1-x^2})),
\end{equation}
где $E_n(f)$ -- наилучшее приближение функции $f\in C[-1,1]$ алгебраическими полиномами\emph{} $p_n\in H^n$. Оценка \eqref{eqtimLimSer14} и свойство \eqref{eqtimLimSer13}  показывают, что операторы $\sigma_n^\alpha(f)$ как
аппарат приближения обладают весьма привлекательными аппроксимативными свойствами в пространстве $C[-1,1]$.

Для сравнения установленных в настоящей работе аппроксимативных свойств операторов  $\sigma_n^\alpha(f)=\sigma_n^\alpha(f,x)$ с хорошо известными аппроксимативными свойствами сумм Фурье $S_n^\beta(f,x)$ по ультрасферическим полиномам $\hat P_n^\beta(x)$  отметим, что из результатов, полученных в работе \cite{badCow}, при $-1<\beta<-1/2$ вытекает оценка

\begin{equation}\label{eqtimLimSer15}%\label{eq.1.2.4.1.25}
|f(x)-S_n^\beta(f,x)|\le c(\beta)E_n(f)(1+\ln(1+n\sqrt{1-x^2})).
\end{equation}
Сопоставляя \eqref{eqtimLimSer14} с \eqref{eqtimLimSer15}, мы замечаем, что аппроксимативные свойства частичных сумм специального ряда $\sigma_n^\alpha(f,x)$ при $1/2\le \alpha <3/2$ и $-1<x<1$ аналогичны аппроксимативным свойствам сумм Фурье $S_n^\beta(f,x)$ при $-1<\beta <-1/2$, за исключением   того факта, что $\sigma_n^\alpha(f,x)$ при $x=\pm1$ совпадает с $f(x)$, а $S_n^\beta(f,x)$ этим свойством не обладает. Отметим также, что аппроксимативные свойства операторов $\sigma_n^{3/2}(f,x)$ аналогичны аппроксимативным свойствам сумм Фурье-Чебышева $S_n^{-1/2}(f,x)$ в том смысле, что
\begin{equation}\label{eqtimLimSer16}%\label{eq.1.2.4.1.26}
|f(x)-\sigma_n^{3/2}(f,x)|\le cE_n(f)\ln(n+1),
\end{equation}
\begin{equation}\label{eqtimLimSer17}%\label{eq.1.2.4.1.27}
|f(x)-S_n^{-1/2}(f,x)|\le cE_n(f)\ln(n+1).
\end{equation}
Оценка \eqref{eqtimLimSer16} установлена в следствии  3.1 (см.\S 3.1), а оценка \eqref{eqtimLimSer17} хорошо известна. В то же время, как уже отмечалось, $\sigma_n^{3/2}(f,\pm1)=f(\pm1)$, а $S_n^{-1/2}(f,x)$ этим важным свойством не обладает.

Отметим, наконец, еще одно важное обстоятельство, касающееся численной реализации операторов $\sigma_n^\alpha(f,x)$ ( $1/2 \le \alpha<3/2$)  и $S_n^\beta(f,x)$ ( $-1< \beta<-1/2$),  обладающих аналогичными аппроксимативными свойствами(см.\eqref{eqtimLimSer14} и \eqref{eqtimLimSer15}). В семействе операторов  $\sigma_n^\alpha(f,x)$ ( $1/2\le \alpha<3/2$) существует
оператор $\sigma_n^{1/2}(f,x)$, для численной реализации которого в силу равенства \eqref{eqtimLimSer11} можно использовать алгоритм быстрого дискретного преобразования Фурье. Что же касается семейства операторов Фурье-Якоби
$S_n^\beta(f,x)$ ( $-1< \beta<-1/2$), то быстрое дискретное  преобразование Фурье для вычисления коэффициентов Фурье-Якоби $f_k^\beta$ при $-1< \beta<-1/2$ (см.\eqref{FYSerKoeffs} и \eqref{intreq.1.2.4.1.4}), насколько известно, не разработано.

Основной целью настоящей параграфа является исследование аппроксимативных свойств операторов $\sigma_n^\alpha(f,x)$ ( $1/2\le \alpha\le3/2$). Нам понадобится для этого ряд свойств полиномов Якоби, которые для удобства собраны в следующем параграфе.



\subsection{ Некоторые сведения о полиномах Якоби}


Полиномы Якоби можно определить \cite{Sego} с помощью следующего явного вида
\begin{equation}\label{Ykbeq.1.2.4.2.1}%\label{eq.1.2.4.2.1}
P_n^{\alpha,\beta}(x) ={n+\alpha\choose n}
\sum_{k=0}^n {(-n)_k(n+\alpha+\beta+1)_k\over k!(\alpha+1)_k}\left({1-x\over2}\right)^k.
\end{equation}
 Ниже нам понадобятся следующие свойства полиномов Якоби:

{\em симметрия}
$$
      P_n^{\alpha,\beta}(-x)=(-1)^nP_n^{\beta,\alpha}(x);
$$
{\em равенство}
 \begin{equation}\label{Ykbeq.1.2.4.2.2}%\label{eq.1.2.4.2.2}
(1-x)P_n^{\alpha+1,\beta}(x)={2\over2n+\alpha+\beta+2} \left[(n+
\alpha+1)P_n^{\alpha,\beta}(x)-
(n+1)P_{n+1}^{\alpha,\beta}(x)\right];
\end{equation}
{\em ортогональность при $\alpha,\beta>-1$}
\begin{equation}\label{Ykbeq.1.2.4.2.3}%\label{eq.1.2.4.2.3}
\int\limits\limits_{-1}^1\kappa(x)P_n^{\alpha,\beta}(x)
P_k^{\alpha,\beta}(x)dx
= h_n^{\alpha,\beta}\delta_{nk},
\end{equation}
где  $ \kappa(x)=\kappa^{\alpha,\beta}(x)=(1-x)^\alpha(1+x)^\beta$,
\begin{equation}\label{Ykbeq.1.2.4.2.4}%\label{eq.1.2.4.2.4}
 h_n^{\alpha,\beta} =
{\Gamma(n+\alpha+1)\Gamma(n+\beta+1)2^{\alpha+\beta+1} \over
n!\Gamma(n+\alpha+\beta+1)(2n+\alpha+\beta+1)}\,\,(n\ge0);
\end{equation}
{\em формула Кристоффеля-Дарбу }
$$\mathcal{K}_n^{\alpha,\beta}(x,y)=
\sum_{k=0}^n{P_k^{\alpha,\beta}(x)P_k^{\alpha,\beta}(y)\over
h_k^{\alpha,\beta}}=
$$
\begin{equation}\label{Ykbeq.1.2.4.2.5}%\label{eq.1.2.4.2.5}
 {2^{-\alpha-\beta}\over 2n+\alpha+\beta+2}
{\Gamma(n+2)\Gamma(n+\alpha+\beta+2)\over
\Gamma(n+\alpha+1)\Gamma(n+\beta+1)}
 {P_{n+1}^{\alpha,\beta}(x)P_n^{\alpha,\beta}(y)-
P_n^{\alpha,\beta}(x)P_{n+1}^{\alpha,\beta}(y)\over x-y};
\end{equation}


{\em весовая оценка} $(1\le x\le1)$
\begin{equation}\label{Ykbeq.1.2.4.2.6}%\label{eq.1.2.4.2.6}
\sqrt{n}\left| P_n^{\alpha,\beta}(x)\right|\le c(\alpha,\beta)
\left(\sqrt{1-x}+{1\over n}\right)^{-\alpha-{1\over2}}
\left(\sqrt{1+x}+{1\over n}\right)^{-\beta-{1\over2}},
\end{equation}
где здесь и всюду в дальнейшем $c,c(\alpha),c(\alpha,\beta),c(\alpha,\beta,\ldots,\gamma)$ означают
положительные числа, зависящие лишь от указанных параметров, вообще
говоря, различные в разных местах;

{\em оценка для интеграла, содержащего полином Якоби }
\begin{equation}\label{Ykbeq.1.2.4.2.7}%\label{eq.1.2.4.2.7}
\int_0^1(1-x)^\gamma|P_n^{\alpha,\beta}(x)|dx\asymp
 \begin{cases}
 n^{\alpha-2\gamma-2},&\text{$2\gamma<\alpha-3/2$},\\
n^{-1/2}\ln n ,&\text{$2\gamma=\alpha-3/2$},\\
n^{-1/2},&\text{$2\gamma>\alpha-3/2$},
\end{cases}
\end{equation}
где $a_n\asymp b_n$  означает, что найдутся такие положительные постоянные $c_1$, $c_2$, что $c_1\le {a_n\over b_n}\le c_2$  $(n=1,2,\ldots)$.



\subsection{ Аппроксимативные свойства специальных рядов по  ультрасферическим полиномам}

 Через $C[-1,1]$ обозначим нормированное пространство непрерывных
 функций $f=f(x)$, определенных на $[-1,1]$, для которых норма
 определяется обычным образом: $\|f\|=\sup\{|f(x)|:x\in[-1,1]\}$. Рассмотрим
для $f\in C[-1,1]$ специальный ряд \eqref{eqtimLimSer6} и его частичную сумму
 \begin{equation}\label{apsseq.1.2.4.3.1}%\label{eq.1.2.4.3.1}
\sigma_n^\alpha(f,x)=c_f(x)+(1-x^2) \sum_{k=0}^{n-2}\hat g_k^\alpha\hat P_k^\alpha(x)\quad(\alpha>0).
\end{equation}
Из самого определения специального ряда \eqref{eqtimLimSer6} следует, что $\sigma_n^\alpha(f)=\sigma_n^\alpha(f,x)$  является проектором на пространство $H^n$, состоящее из алгебраических полиномов $p_n(x)$ степени $n$, т.е. если $p_n(x)\in H^n$, то
\begin{equation}\label{apsseq.1.2.4.3.2}
\sigma_n^\alpha(p_n,x)\equiv p_n(x).
\end{equation}
В самом деле, если $p_n\in H^n$, то $p_n(\pm1)=c_{p_n}(\pm1)$ и поэтому
$$
q_{n-2}(x)={p_n(x)-c_{p_n}(x)\over1-x^2}
$$
представляет собой алгебраический полином степени $n-2$. Следовательно, для $F(x)=q_{n-2}(x)$ имеет место равенство \eqref{eqtimLimSer4}, в котором все коэффициенты $\hat g_k^\alpha$ с $k>n-2$ обращаются в нуль, а это равносильно равенству \eqref{apsseq.1.2.4.3.2}. Пусть $f(x)\in C[-1,1]$, $p_n(x)\in H^n$. Тогда в силу \eqref{apsseq.1.2.4.3.2} мы можем записать
\begin{equation}\label{apsseq.1.2.4.3.3}
f(x)-\sigma_n^\alpha(f,x)= f(x)-p_n(x)+\sigma_n^\alpha(p_n-f,x).
\end{equation}
Обозначим через $E_n^\pm(f)$  наилучшее приближение функции $f(x)\in C[-1,1]$ алгебраическими полиномами $p_n\in H^n$, которые в точках $\pm1$ совпадают с $f(x)$:
\begin{equation}\label{apsseq.1.2.4.3.4}
E_n^\pm(f)=\inf_{p_n\in H^n\atop p_n(\pm1)=f(\pm1)}\|f-p_n\|. \end{equation}
Пусть для $p_n^*\in H^n$ будет $E_n^\pm(f)=\|f-p_n^*\|$ и $ p_n*(\pm1)=f(\pm1)$. Тогда из \eqref{apsseq.1.2.4.3.3} и \eqref{apsseq.1.2.4.3.4} имеем
\begin{equation}\label{apsseq.1.2.4.3.5}
|f(x)-\sigma_n^\alpha(f,x)|\le  |f(x)-p_n^*(x)|+|\sigma_n^\alpha(p_n^*-f,x)|\le E_n^\pm(f)+|\sigma_n^\alpha(p_n^*-f,x)|. .
\end{equation}
С другой стороны, так как $ p_n*(\pm1)-f(\pm1)=0$, то из \eqref{apsseq.1.2.4.3.1} следует, что
\begin{equation}\label{apsseq.1.2.4.3.6}
\sigma_n^\alpha(p_n^*-f,x)=
(1-x^2)\int\limits_{-1}^1(p_n^*(t)-f(t))(1-t^2)^{\alpha-1}\mathcal{K}_{n-2}^\alpha(x,t)dt,
 \end{equation}
где $(P_k^\alpha(x)=P_k^{\alpha,\alpha}(x))$
\begin{equation}\label{apsseq.1.2.4.3.7}
\mathcal{K}_{n-2}^\alpha(x,t)=\sum_{k=0}^{n-2}{P_k^\alpha(t)P_k^\alpha(x)\over h_k^{\alpha,\alpha}}.
\end{equation}
Из \eqref{apsseq.1.2.4.3.6} и \eqref{apsseq.1.2.4.3.7} имеем
\begin{equation}\label{apsseq.1.2.4.3.8}
|\sigma_n^\alpha(p_n^*-f,x)|\le E_n^\pm(f)(1-x^2)\int\limits_{-1}^1(1-t^2)^{\alpha-1}|\mathcal{K}_{n-2}^\alpha(x,t)|dt.
\end{equation}
Сопоставляя \eqref{apsseq.1.2.4.3.5} с \eqref{apsseq.1.2.4.3.8}, получаем
\begin{equation}\label{apsseq.1.2.4.3.9}
|f(x)-\sigma_n^\alpha(f,x)|\le E_n^\pm(f)(1+\Lambda_n^\alpha(x)),
\end{equation}
\begin{equation}\label{apsseq.1.2.4.3.10}
\Lambda_n^\alpha(x)=(1-x^2)\int\limits_{-1}^1(1-t^2)^{\alpha-1}|\mathcal{ K}_{n-2}^\alpha(x,t)|dt.
\end{equation}
В связи с неравенством \eqref{apsseq.1.2.4.3.9} возникает задача об оценке величины $\Lambda_n^\alpha(x)$. При рассмотрении этой задачи понадобится ряд
вспомогательных утверждений.

\begin{lemma}\label{apss:lemma1} Пусть $1/2<\alpha$, $\alpha\ne\frac32$, $-1< x< 1$. Тогда имеет место оценка
$$
\Lambda_n^\alpha(x)\le c(\alpha)\left[1+\ln(1+n\sqrt{1-x^2})+(1-x^2)^{3/4-\alpha/2}\right].
$$
\end{lemma}
\begin{lemma}\label{apss:lemma2} Имеет место оценка
$$\Lambda_n^{\frac32}(x)\le c\ln(n+1)\quad(-1\le x\le1).$$
\end{lemma}

\begin{lemma}\label{apss:lemma3} Имеет место оценка
$$\Lambda^\frac12_n(x)\le c(1+\ln(n\sqrt{1-x^2}+1))\quad(-1<x<1).$$
\end{lemma}

\begin{theorem}\label{apss:t1} Пусть $-1\le x\le1$. Тогда имеют место оценки
$$\Lambda^\alpha_n(x)\le c(\alpha)[1+\ln(n\sqrt{1-x^2}+1)]\quad(\frac12\le\alpha<\frac32),$$
$$\Lambda^{\frac32}_n(x)\le c\ln(n+1).$$
\end{theorem}

\noindent Доказательство этой теоремы вытекает из лемм \ref{apss:lemma1}--\ref{apss:lemma3}.

\begin{corollary}\label{apss:col1} Пусть $f(x)\in C[-1,1]$, $E_n(f)$ -- наилучшее приближение $f(x)$ в $[-1,1]$ алгебраическими полиномами степени $n$. Тогда имеют место оценки
$$|f(x)-\sigma_n^\alpha(x)|\le c(\alpha)E_n(f)(1+\ln(n\sqrt{1-x^2}+1))\quad(\frac12\le\alpha<\frac32),$$
$$|f(x)-\sigma_n^{\frac32}(x)|\le c E_n(f)\ln(n+1).$$
\end{corollary}

Убедимся сначала, что $E^\pm_n(f)\le2E_n(f)$. В самом деле, обозначим через $p_n^*(x)$ алгебраический полином наилучшего приближения к функции f, для которого $\|p_n^*-f\|=E_n(f)$. Тогда $|p_n^*(\pm1)-f(\pm1|\le E_n(f)$ и, следовательно, $|c_{p_n^*-f}(x)|\le E_n(f)$ для всех $x\in [-1,1]$. Поэтому для алгебраического полинома  $q_n(x)=p_n^*(x)-c_{p_n^*-f}(x)$ имеем $q_n(\pm1)=f(\pm1)$  и  $|q_n(x)-f(x)|\le |p_n^*(x)-f(x)|+|c_{p_n^*-f}(x)|\le 2E_n(f)$. Тем самым неравенство $E^\pm_n(f)\le2E_n(f)$ доказано. С учетом этого неравенства  доказательство  следствия \ref{apss:col1} непосредственно вытекает из \eqref{apsseq.1.2.4.3.9}, \eqref{apsseq.1.2.4.3.10} и теоремы \ref{apss:t1}.

Рассмотрим частичную сумму ряда \eqref{eqtimLimSer10} следующего вида
\begin{equation}\label{apsseq.1.2.4.3.79}S_n(\Phi,\theta)=a_\Phi(\theta)+\sin\theta\sum\limits_{k=1}^{n-2}\varphi_k\sin k\theta\end{equation}
и оценим величину $|\Phi(\theta)-S_n(\Phi,\theta)|$. С этой целью обозначим через $E^\pi_n(\Phi)$ -- наилучшее равномерное приближение функции $\Phi(\theta)$ тригонометрическими полиномами $T_n(\theta)$ порядка $n$, для которых
\begin{equation}\label{apsseq.1.2.4.3.80}
\Phi(k\pi)=T_n(\Phi,k\pi),k\in\mathbb{Z},
\end{equation}
$\tau_\Phi^n$ -- пространство всех тригонометрических полиномов $T_n(\theta)$ порядка $n$, удовлетворяющих условию \eqref{apsseq.1.2.4.3.80}. Пусть для $T_n(\Phi)\in\tau_\Phi^n$ имеет место равенство
\begin{equation}\label{apsseq.1.2.4.3.81}\|\Phi-T_n(\Phi)\|=\max_{\theta\in\mathbb{R}}|\Phi(\theta)-T_n(\Phi,\theta)|=E^\pi_n(\Phi)\end{equation}
Из конструкции ряда \eqref{eqtimLimSer10} (или из \eqref{apsseq.1.2.4.3.79}) не трудно заметить, что $S_n(\Phi)=S_n(\Phi,\theta)$ является проектором на пространство четных тригонометрических полиномов порядка $n$ поэтому $S_n(T_n(\Phi,\theta))=T_n(\Phi,\theta)$. Отсюда следует, что
$$|\Phi(\theta)-S_n(\Phi,\theta)|=|\Phi(\theta)-T_n(\Phi,\theta)+S_n(T_n(\Phi)-\Phi,\theta)|\le$$
\begin{equation}\label{apsseq.1.2.4.3.82}E_n^\pi(\Phi)+E_n^\pi(\Phi)L_n^\pi(\theta)=E_n^\pi(\Phi)(1+L_n^\pi(\theta)),\end{equation}
где
\begin{equation}\label{apsseq.1.2.4.3.83}L_n^\pi(\theta)=\frac{|\sin\theta|}{\pi}\int\limits_0^\pi\frac{|R_n(\theta,\tau)|}{\sin\tau}d\tau,\end{equation}
где функция $R_n(\theta,\tau)=2\sum\limits_{k=1}^{n-1}\sin k\theta\sin k\tau$. Из \eqref{apsseq.1.2.4.3.10} и \eqref{eqtimLimSer8} при $\alpha=\frac12$ имеем \begin{equation}\label{apsseqLambda12}
\Lambda^{\frac12}_n(\cos\theta)=\frac2\pi\sin\theta\int\limits_0^\pi\frac1{\sin\tau}\left|\sum\limits_{k=1}^{n-1}\sin k\theta\sin k\tau\right|d\tau.
\end{equation}
Сопоставляя \eqref{apsseqLambda12} с \eqref{apsseq.1.2.4.3.83}, мы замечаем, что $L_n^\pi(\theta)=\Lambda^{\frac12}_n(\cos\theta)$, поэтому в силу леммы 3.3
\begin{equation}\label{apsseq.1.2.4.3.84}L_n^\pi(\theta)\le c(1+\ln(n\sin\theta+1))\quad(0\le\theta\le\pi).\end{equation}
Из \eqref{apsseq.1.2.4.3.82} и \eqref{apsseq.1.2.4.3.84} мы выводим

\begin{corollary}\label{apss:col2} Пусть $\Phi(\theta)$ -- четная $2\pi$-периодическая непрерывная функция. Тогда имеет место оценка
\begin{equation}\label{apsseq.1.2.4.3.85}|\Phi(\theta)-S_n(\Phi,\theta)|\le c E_n^{\pi}(\Phi)(1+\ln(n|\sin\theta|+1)).\end{equation}
\end{corollary}

\begin{corollary}\label{apss:col3} Пусть $\frac12\le\alpha\le\frac32$. Тогда
\begin{equation}\label{apsseq.1.2.4.3.86}\Lambda_n^{\alpha}=\max_{-1\le x\le1}\Lambda_n^\alpha(x)\asymp\ln(n+1),\end{equation}
другими словами, найдутся такие положительные числа $c_1(\alpha)$ и $c_2(\alpha)$, что
\begin{equation}\label{apsseq.1.2.4.3.87}c_1\ln(n+1)\le\Lambda_n^\alpha\le c_2\ln(n+1).\end{equation}

\end{corollary}

\begin{theorem} Пусть  $\alpha>3/2$, $0<\delta\le1$
Тогда для $-1+\frac\delta{n^2}\le x\le 1-\frac\delta{n^2}$ имеет место оценка
$$
\Lambda_n^\alpha(x)\le c(\alpha)\left[\ln(1+n\sqrt{1-x^2})+\left(\sqrt{1-x^2}+\frac1n\right)^{3/2-\alpha}\right].
$$
\end{theorem}
Утверждение этой теоремы непосредственно вытекает из леммы \ref{apss:lemma1} и равенства $\Lambda_n^\alpha(x)=\Lambda_n^\alpha(-x)$.

\begin{theorem} Пусть  $\alpha>3/2$, $0<\delta\le\frac 1{\alpha+1}$.
Тогда для $1-\frac1{(\alpha+1)n^2}\le x\le 1-\frac\delta{n^2}$ имеет место оценка
\begin{equation}\label{eq.1.2.4.3.90}
c_1(\alpha,\delta)n^{\alpha-3/2}\le\Lambda_n^\alpha(x)\le c_2(\alpha,\delta)n^{\alpha-3/2}.
\end{equation}
\end{theorem}

