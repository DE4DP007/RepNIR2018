\chapter{О базисности системы  полиномов Якоби в весовом пространстве Лебега с переменным показателем}

Для того чтобы сформулировать основной результат, полученный нами в отчетном году, понадобятся некоторые понятия и определения, касающиеся весовых пространств Лебега $L^{p(x)}_\mu([-1,1])$ с переменным показателем $p(x)$ и классических ортогональных полиномов Якоби.


\section{Некоторые сведения о пространстве Лебега с переменным показателем }

Пусть $E$ произвольное
множество, на котором задана мера Лебега $m$, $p=p(x)$ -- неотрицательная $m$-измеримая  функция, заданная на $E$. Через $L^{p(x)}_m(E)$ обозначим пространство $m$-измеримых функций $f=f(x)$, заданных на $E$,  для которых конечен интеграл Лебега $\int_E|f(x)|^{p(x)}m(dx)$.  Если $p=p(x)$ существенно ограничена на $E$, то, как показано в \cite{ShIIBJWShar4}, $L^{p(x)}_m(E)$ является линейным топологическим пространством, в котором при дополнительном условии $1\le p(x)\le \overline{p}<\infty$ можно ввести норму
\begin{equation}\label{2.1}
\|f\|_{p(\cdot)}(E)=\inf\{\alpha>0: \int_E\left|\frac{f(x)}{\alpha}\right|^{p(x)}m(dx)\le1\},
\end{equation}
которая превращает $L^{p(x)}_m(E)$ в банахово пространство. Если $E \subset \mathbb{R}^n$, $m(dx)=w(x)dx$, мы будем писать  $L^{p(x)}_w(E)$ вместо $L^{p(x)}_m(E)$ и называть $L^{p(x)}_w(E)$ весовым пространством Лебега с переменным показателем $p(x)$ и весом $w=w(x)$. Если к тому же $w(x)\equiv 1$, то будем писать $L^{p(x)}(E)$ вместо $L^{p(x)}_1(E)$.

Если $1<p(x)<\infty$, то можно определить новый переменный показатель $p'(x)=p(x)/(p(x)-1)$ и ввести соответствующее пространство $L^{p'(x)}_m(E)$. В работе \cite{ShIIBJWShar4} было показано, что если $1< p(x)\le \overline{p}(E)<\infty$, то пространство $(L^{p(x)}_m(E))'$, состоящее из непрерывных линейных функционалов $F$, заданных в $L^{p(x)}_m(E)$,  совпадает с линейной оболочкой пространства $(L^{p(x)}_m(E))'$ в том смысле, что произвольный элемент $F\in(L^{p(x)}_m(E))'$ имеет вид
\begin{equation}\label{2.2}
F(f)=\int_E f(x)g(x)m(dx), \quad\frac{g}{\alpha}\in L^{p'(x)}_m(E), \quad\alpha\ne0.
\end{equation}
В частности, если $1<\underline{p}(E)\le p(x)\le \overline{p}(E)\le\infty$, то и $p'$ обладает этим свойством и поэтому $(L^{p(x)}_m(E))'=L^{p'(x)}_m(E)$ и, как следствие, пространство $L^{p(x)}_m(E)$ рефлексивно. Этот результат позволяет ввести в $L^{p(x)}_m(E)$
другие нормы, эквивалентные исходной норме $\|f\|_{p(\cdot)}(E)$ (см.\eqref{2.1}). Итак, пусть $1<\underline{p}(E)\le p(x)\le \overline{p}(E)<\infty$, тогда сопряженная функция $p'(x)=p(x)/(p(x)-1)$  обладает, как уже отмечалось, такими же свойствами, т.е. $1<\underline{p'}(E)\le p'(x)\le \overline{p'}(E)<\infty$, поэтому мы можем ввести в рассмотрение следующую норму для $f\in L^{p(x)}_m(E)$:
\begin{equation}\label{2.3}
\|f\|^*_{p(\cdot)}(E)=\sup_{g\in L^{p'(x)}_{m}(E), \atop\|g\|_{p'(\cdot)}(E)\le1}\int_Ef(x)g(x)m(dx),
\end{equation}
для которой имеет место (см. \cite{ShIIBJWShar3}) неравенство
\begin{equation}\label{2.4}
\|f\|^*_{p(\cdot)}(E)\le 2\|f\|_{p(\cdot)}(E).
\end{equation}
Пусть $p$ и $q$ два переменных показателя, заданных на $E$, для которых $1\le p(x)\le q(x)\le \overline{q}(E)<\infty$. Тогда для $f\in L^{q(x)}_m(E)$  имеет место \cite{ShIIBJWShar2,ShIIBJWShar3,ShIIBJWShar6} неравенство
\begin{equation}\label{2.5}
\|f\|_{p(\cdot)}(E)\le c(p,q)\|f\|_{q(\cdot)}(E),
\end{equation}
где здесь и всюду  в дальнейшем через $c, c(p), c(p,q),\ldots $ обозначаются положительные
числа, зависящие лишь от указанных параметров.

Далее, пусть $E_1$  и $E_2$ два измеримых множества с соответствующими мерами Лебега $m_1$ и $m_2$, причем $m_i(E_i)<\infty, i=1,2$. Пусть $f=f(x,t)$ -- измеримая функция, заданная на декартовом произведении $E_1\times E_2$. Тогда имеет место  \cite{ShIIBJWShar2,ShIIBJWShar3,ShIIBJWShar6} неравенство
\begin{equation}\label{2.6}
\|\int_{E_2}f(\cdot,t)m_2(dt)\|_{p(\cdot)}(E_1)\le 2\int_{E_2}\|f(\cdot,t)\|_{p(\cdot)}(E_1)m_2(dt).
\end{equation}




\section{Условие Дини-Липшица и его следствия}
В работе \cite{ShIIBJWShar1} на примере системы Хаара впервые был рассмотрен вопрос о сходимости в пространстве $L^{p(x)}([0,1])$   сумм Фурье по ортонормированным системам функций. В связи с этой задачей в \cite{ShIIBJWShar1} было показано, что для базисности в $L^{p(x)}([0,1])$ системы Хаара $\{\chi_n(x)\}$ достаточно и, в определенном смысле, необходимо, чтобы переменный показатель $p(x)$ подчинялся условию Дини -- Липшица вида ($x,y\in[0,1]$, $d>0$)
\begin{equation}\label{3.1}
|p(x)-p(y)|\le \frac{d}{-\ln|x-y|}, \quad |x-y|\le\frac12.
\end{equation}
Обозначим через $\mathcal{P}_{2\pi}$ множество всех  $2\pi$-периодических переменных показателей  $p>1$, для которых имеет место неравенство \eqref{3.1}.  В работе \cite{ShIIBJWShar5} было доказано, что условие $p\in\mathcal{P}_{2\pi}$  является достаточным и необходимым в определенном смысле для равномерной ограниченности некоторых семейств операторов свертки в пространстве $L^{p(x)}_{2\pi}$, где $L^{p(x)}_{2\pi}$ -- пространство $2\pi$-периодических измеримых функций, для которых определена норма $\|f\|_{p(\cdot)}=\|f\|_{p(\cdot)}([0,2\pi])$. В работе \cite{ShIIBJWShar2} было доказано, что тригонометрическая система является базисом в пространстве $L^{p(x)}_{2\pi}$ тогда и только тогда, когда переменный показатель $p>1$ удовлетворяет условию  \eqref{3.1}.
 В \cite{ShIIBJWSharap1} было показано, что условие \eqref{3.1} необходимо и достаточно для базисности системы полиномов Лежандра в пространстве $L^{p(x)}([-1,1])$ при дополнительном условии, которое состоит в том, что для  переменного показателя  $p$ найдутся отрезки вида $[-1,-1+\delta_1]$ и $[1-\delta_2,1]$, на  которых $p$ сохраняет постоянные значения, причем $p(\pm1)\in(\frac43,4)$.

В \cite{ShIIBJWDiening,ShIIBJWCruz-Uribe}  доказано (см. также цитированную в \cite{ShIIBJWDiening,ShIIBJWCruz-Uribe} литературу), что при соблюдении условия \eqref{3.1} и некоторого дополнительного условия на $p(x)$ при $|x|\to\infty$ максимальная функция Харди -- Литлвуда $M(f)$ является элементом пространства $L^{p(x)}(\mathbb{R})$ вместе с $f$. Сформулируем этот результат более точно, следуя    \cite{ShIIBJWCruz-Uribe}.
Пусть $\Omega$ -- открытое множество в $\mathbb{R}^n$, $f$ -- локально интегрируемая функция, заданная на $\Omega$. Тогда мы можем определить максимальную функцию Харди -- Литлвуда $M(f)=M(f)(x)$, полагая
\begin{equation}\label{3.2}
M(f)(x)=\sup_{x\in B}\frac{1}{|B|}\int_{B\cap\Omega}|f(y)|dy,
\end{equation}
где верхняя грань берется по всем открытым шарам $B\in \mathbb{R}^n $, содержащим точку $x$, $|B|$ -- объем шара $B$. Хорошо известно, что если постоянное число $p$  удовлетворяет условию $1<p<\infty$, то $M(f)$ вместе с $f$ принадлежит $L^p(\Omega)$, причем $\|M(f)\|_p(\Omega)\le c(p)\|f\|_p(\Omega)$. В  \cite{ShIIBJWCruz-Uribe} (см. также \cite{ShIIBJWDiening}) показано, что этот результат допускает обобщение на пространства $L^{p(x)}(\Omega)$, если переменный показатель $p>1$ подчиняется
$n$-мерному аналогу условия \eqref{3.1} и дополнительному условию
\begin{equation}\label{3.3}
|p(x)-p(y)|\le \frac{c}{\ln(e+|x|^n)}, \quad x,y\in\Omega,\quad |y|>|x|,
\end{equation}
где для $x=(x_1,\ldots,x_n)\in \mathbb{R}^n$ принято обозначение $|x|=(x_1^2+\ldots+x_n^2)^{1/2}$.
А именно, пусть задано открытое множество $\Omega\subset\mathbb{R}^n$, $p:\Omega\to [0,\infty)$ -- переменный показатель, для которого $1<\underline{p}\le p(x)\le\overline{p}<\infty$. Предположим, что   $p$ удовлетворяет условиям  \eqref{3.1} и \eqref{3.3} для $x,y\in\Omega$. Тогда имеет место оценка
\begin{equation}\label{3.4}
\|M(f)\|_{p(\cdot)}(\Omega)\le c(p)\|f\|_{p(\cdot)}(\Omega).
\end{equation}


Это очень важный результат, который находит себе многочисленные приложения. В качестве одного из следствий оценки \eqref{3.4} отметим свойство ограниченности в $L^{p(x)}(\mathbb{R}^n)$ некоторых сингулярных интегралов и преобразований Кальдерона -- Зигмунда (см.  \cite{ShIIBJWCruz-Uribe,ShIIBJWDiening} и цитированную там литературу). Нам понадобится свойство ограниченности в $L^{p(x)}(\mathbb{R})$ преобразования Гильберта, которое мы рассмотрим более подробно в следующем параграфе.


\section{Преобразование Гильберта в $L^{p(x)}(\mathbb{R})$}
Пусть $p>1$, $f\in L^{p}(\mathbb{R})$. Тогда мы можем определить преобразование Гильберта $H(f)=H(f)(x)$ следующим образом:
\begin{equation}\label{4.1}
H(f)(x)=\int_{\mathbb{R}}\frac{f(t)dt}{t-x},
\end{equation}
 где интеграл в \eqref{4.1} понимается в смысле главного значения Коши, т.е. \linebreak
$
H(f)(x)=\lim_{\varepsilon\to0}H_\varepsilon(f)(x),\quad H_\varepsilon(f)(x)=\int_{|t-x|>\varepsilon}\frac{f(t)dt}{t-x}.
$
Хорошо известно, что функция $Hf(x)$ конечна для почти всех $x\in\mathbb{R}$. Более того, если $f\in L^p(\mathbb{R})$, где $p=const>1$, то из известной теоремы Рисса  следует, что
$\|H(f)\|_p(\mathbb{R})\le c(p)\|f\|_p(\mathbb{R})$. Как это показано в \cite{ShIIBJWCruz-Uribe,ShIIBJWDiening}, эта оценка допускает обобщение на случай, когда $f\in L^{p(x)}(\mathbb{R})$, если переменный показатель  $p:\mathbb{R}\to[0,\infty)$ удовлетворяет условиям \eqref{3.1} и \eqref{3.3}. Другими словами, если переменный показатель  $p:\mathbb{R}\to[0,\infty)$ удовлетворяет условиям, которые обеспечивают справедливость оценки \eqref{3.4}, то имеет место оценка
\begin{equation}\label{4.2}
\|H(f)\|_{p(\cdot)}(\mathbb{R})\le c(p)\|f\|_{p(\cdot)}(\mathbb{R}).
\end{equation}
Через $\mathcal{ P}(\mathbb{R})$ обозначим множество переменных показателей $p>1$, заданных на вещественной оси  $\mathbb{R}$, для каждого из которых найдется такое число $\Delta>1$, что $p(x)=p(\Delta)$ при $|x|\ge \Delta$, а на отрезке $E=[-\Delta,\Delta]$ $p$ удовлетворяет условию \eqref{3.1}.
Очевидно, что если $p\in \mathcal{ P}(\mathbb{R})$, то условие \eqref{3.3} также выполнено, поэтому для такого переменного показателя справедлива оценка \eqref{4.2}.
Рассмотрим  преобразование Гильберта следующего вида
\begin{equation}\label{4.3}
H^{a,b}f=H^{a,b}f(x)=\int_{a}^{b}\frac{f(t)dt}{t-x}.
\end{equation}
Пусть переменный показатель $p$ удовлетворяет на $[a,b]$ условию \eqref{3.1}. Тогда мы можем продолжить $p=p(x)$ на все $\mathbb{R}$ следующим образом. Пусть $A=1 +\max\{|a|,|b|\}$, $p(x)=3/2$ при $|x|\ge A$, а на отрезки $[-A,a]$ и $[b,A]$ продолжим $p(x)$ линейно и непрерывно. Тогда нетрудно убедиться в том, что  продолженная функция $p$ удовлетворяет условиям \eqref{3.1} и \eqref{3.3} на всем $\mathbb{R}$. С  другой стороны, если функция $f\in L^{p(x)}([a,b])$, то можно ее продолжить на все $\mathbb{R}$, полагая $f(x)\equiv0$ при $x\notin[a,b]$ и рассмотреть преобразование Гильберта
 \begin{equation*}
Hf=Hf(x)=\int_\mathbb{R}\frac{f(t)dt}{t-x}=\int_{a}^{b}\frac{f(t)dt}{t-x}=H^{a,b}f(x).
\end{equation*}
Отсюда и из \eqref{4.2} выводим
\begin{equation}\label{4.4}
\|H^{a,b}(f)\|_{p(\cdot)}(\mathbb{R})=\|H(f)\|_{p(\cdot)}(\mathbb{R})\le c(p)\|f\|_{p(\cdot)}(\mathbb{R})=c(p)\|f\|_{p(\cdot)}([a,b]).
\end{equation}






 \section{Некоторые сведения о полиномах Якоби}
 Для произвольных действительных $\alpha$ и $\beta$ полиномы Якоби  $P_n^{\alpha,\beta}(x)$ можно определить \cite{ShIIBJWSege}  с помощью формулы Родрига
 \begin{equation}\label{7.1}
P_n^{\alpha,\beta}(x) = {(-1)^n\over2^nn!}{1\over\mu(x)}{d^n\over
dx^n} \left\{\mu(x)\sigma^n(x)\right\},
\end{equation}
где $\alpha,\beta$ -- произвольные действительные числа, $\mu(x)=
\mu(x;\alpha,\beta) =
(1-x)^\alpha(1+x)^\beta,\,\sigma(x)=1-x^2$. Если
$\alpha,\beta>-1$, то полиномы Якоби образуют ортогональную
систему с весом $\mu(x)$, т.е.
\begin{equation}\label{7.2}
\int_{-1}^1P_n^{\alpha,\beta}(x)P_m^{\alpha,\beta}(x)\rho(x)dx =
h_n^{\alpha,\beta}\delta_{nm},
\end{equation}
где
\begin{equation}\label{7.3}
h_n^{\alpha,\beta} =
{\Gamma(n+\alpha+1)\Gamma(n+\beta+1)2^{\alpha+\beta+1} \over
n!\Gamma(n+\alpha+\beta+1)(2n+\alpha+\beta+1)}.
\end{equation}

Пусть $-1<\alpha, \beta$ -- произвольные вещественные числа,
$$
s(\theta)=s^{\alpha,\beta}(\theta)=\pi^{-\frac12}
\left(\sin\frac{\theta}{2}\right)^{-\alpha-\frac12}
\left(\cos\frac{\theta}{2}\right)^{-\beta-\frac12},
$$
$$
\lambda_n=n+\frac{\alpha+\beta+1}{2}, \quad\gamma=-
\left(\alpha+\frac{1}{2}\right)\frac{\pi}{2}.
$$
Тогда для $0<\theta<\pi$ имеет место асимптотическая формула
\begin{equation}\label{7.4}
P_n^{\alpha,\beta}(\cos\theta)=n^{-\frac12}s(\theta)
\left(\cos(\lambda_n\theta+\gamma)+
\frac{v_n(\theta)}{n\sin\theta}\right),
\end{equation}
в которой для функции $v_n(\theta)=v_n(\theta;\alpha,\beta)$
справедлива оценка
\begin{equation}\label{7.5}
|v_n(\theta)|\le c(\alpha,\beta,\delta)\quad
 \left(0<\frac{\delta}{n}\le\theta\le\pi-\frac{\delta}{n}\right).
\end{equation}
Из \eqref{7.4} и  \eqref{7.5} непосредственно выводим следующие оценки

\begin{equation}\label{7.6}
|P_n^{\alpha,\beta}(\cos\theta)|\le c(\alpha,\beta,\gamma)(n+1)^\alpha, \quad |\theta|\le \frac{\gamma}{n+1},
\end{equation}

\begin{equation}\label{7.7}
\sqrt{n+1}(1-x)^{\frac\alpha2+\frac14}(1+x)^{\frac\beta2+\frac14}
|P_n^{\alpha,\beta}(x)|\le c(\alpha,\beta)\,\,\, (-1\le x\le1,\alpha,\beta\ge-1/2).
\end{equation}
 Указанные оценки вместе с  формулой Кристоффеля -- Дарбу
$$
 K_n^{\alpha,\beta}(x,y)=
\sum_{k=0}^n{P_k^{\alpha,\beta}(x)P_k^{\alpha,\beta}(y)\over
h_k^{\alpha,\beta}}={2^{-\alpha-\beta}\over
2n+\alpha+\beta+2} {\Gamma(n+2)\Gamma(n+\alpha+\beta+2)\over
\Gamma(n+\alpha+1)\Gamma(n+\beta+1)}
 $$
\begin{equation}\label{7.8}
\times  {P_{n+1}^{\alpha,\beta}(x)P_n^{\alpha,\beta}(y)-
P_n^{\alpha,\beta}(x)P_{n+1}^{\alpha,\beta}(y)\over x-y}
\end{equation}
играют основополагающую роль при изучении аппроксимативных свойств  сумм Фурье по полиномами Якоби.


%Отметим еще следующие свойства полиномов Якоби:
%
% \begin{equation}\label{7.9}
%P_n^{\alpha,\beta}(-x) = (-1)^nP_n^{\beta,\alpha}(x),
%\end{equation}
%$$
%(1-x)P_n^{\alpha+1,\beta}(x)=
%$$
%\begin{equation}\label{7.10}
%{2\over2n+\alpha+\beta+2}
%\left[(n+ \alpha+1)P_n^{\alpha,\beta}(x)-
%(n+1)P_{n+1}^{\alpha,\beta}(x)\right].
%\end{equation}
%Пусть $T_n(x)=\cos(n\arccos x)$ -- полином Чебышева первого рода. Тогда \cite{Sege}
%\begin{equation}\label{7.11}
%P_n^{-\frac{1}{2},-\frac{1}{2}}(x)=\frac{(2n)!}{2^{2n}{n!}^2}T_n(x).
%\end{equation}



\section{О равномерной ограниченности и сходимости сумм Фурье по ультрасферическим полиномам Якоби в весовом пространстве Лебега с переменным показателем }
Пусть $\alpha,\beta>-1$, $\mu=\mu(x)=\mu(x;\alpha,\beta) =(1-x)^\alpha(1+x)^\beta$, $L^{p(x)}_\mu([-1,1])$ -- пространство Лебега с переменным показателем $p=p(x)$ и весом $\mu$. Если $p\ge1$ и $f\in L^{p(x)}_\mu([-1,1])$, то мы можем определить коэффициенты Фурье -- Якоби
\begin{equation}\label{8.1}
f_k^{\alpha,\beta}=\frac{1}{h_k^{\alpha,\beta}}\int_{-1}^1f(t)P_k^{\alpha,\beta}(t)
\mu(t)dt
\end{equation}
и сумму Фурье -- Якоби
\begin{equation}\label{8.2}
S_n^{\alpha,\beta}(f)=S_n^{\alpha,\beta}(f,x)=\sum_{k=0}^nf_k^{\alpha,\beta}P_k^{\alpha,\beta}(x).
\end{equation}
Основная цель предпринятого исследования заключается в том, чтобы найти условия на переменный показатель $p=p(x)$ и весовую функцию $\mu(x) = (1-x)^\alpha (1+x)^\beta$, которые гарантируют сходимость при $n\to\infty$ сумм Фурье $S_n^{\alpha,\beta}(f)$ по полиномам Якоби $P_k^{\alpha,\beta}(x)$ к функции $f\in L^{p(x)}_\mu([-1,1])$ по норме пространства $L^{p(x)}_\mu([-1,1])$. Другими словами, рассматривается задача о базисности в банаховом пространстве $L^{p(x)}_\mu([-1,1])$ системы ортонормированных полиномов Якоби $p_k^{\alpha,\beta}(x)=(h_k^{\alpha,\beta})^{-1/2}P_k^{\alpha,\beta}(x)$ $(k=0,1,\ldots)$, ограничившись при этом случаем $\alpha, \beta\ge-1/2$.
С этой целью  вводится класс переменных показателей $\mathcal{ P}(-1,1)$, удовлетворяющих следующим условиям:

$A)$ $p$ обладает свойством \eqref{3.1} на $[-1,1]$;

$B)$  $\underline{p}([-1,1])=\min_{x\in [-1,1]}p(x)>1$;

$C)$ для $p$ найдутся такие  положительные (сколь угодно малые) числа $\delta_i=\delta_i(p)$ $(i=1,2)$, что $p(x)=p(-1)$ при $x\in[-1,-1+\delta_1]$ и $p(x)=p(1)$ при $x\in [1-\delta_2,1]$.


Показано, что условия $A)$, $B)$ и $C)$, взятые вместе, обеспечивают равномерную ограниченность в  $L^{p(x)}_\mu([-1,1])$ последовательности линейных операторов $S_n^{\alpha,\beta}(f)$ $(n=0,1,\ldots)$ при соблюдении дополнительного условия \eqref{8.13}.  При этом отметим, что  $A)$ и $B)$ являются также и необходимыми условиями для равномерной ограниченности в $L^{p(x)}_\mu([-1,1])$ последовательности линейных операторов $S_n^{\alpha,\beta}(f)$ $(n=0,1,\ldots)$. Что же касается условия $C)$, то вопрос о его необходимости в случае $\alpha,\beta>-1/2$ остается открытым, а когда $\alpha=\beta=-1/2$  оно становится излишним.
Сформулируем основной результат, полученного нами в отчетном году. А именно, имеет место следующая





\begin{theorem}
Пусть $\alpha,\beta>-1/2$, $\mu=\mu(x)=(1-x)^\alpha(1+x)^\beta$, $p\in\mathcal{ P}(-1,1)$,
$4\frac{\alpha+1}{2\alpha+3}<p(1)<4\frac{\alpha+1}{2\alpha+1}$, $4\frac{\beta+1}{2\beta+3}<p(-1)<4\frac{\beta+1}{2\beta+1}$.
Тогда суммы Фурье -- Якоби $S_n^{\alpha,\beta}(f)$ $(n=0,1,\ldots)$ равномерно ограничены в весовом пространстве
Лебега $L_\mu^{p(x)}([-1,1])$ с переменным показателем $p$. Другими словами, найдется такое положительное число $c(\alpha,\beta,p)$, зависящее только от указанных параметров $\alpha$, $\beta$ и $p$, что для произвольной функции $f\in L_\mu^{p(x)}([-1,1])$ имеет место оценка
\begin{equation}\label{8.91}
\|S_n^{\alpha,\beta}(f)\|_{p(\cdot),\mu}([-1,1])\le c(\alpha,\beta,p)\|f\|_{p(\cdot),\mu}([-1,1]).
\end{equation}
\end{theorem}
%
%
%\begin{proof}
%  Поскольку сумма Фурье -- Якоби $\|S_n^{\alpha,\beta}(f)$ представляет собой линейный оператор в пространстве $L_\mu^{p(x)}([-1,1])$, то, очевидно,  достаточно проверить справедливость оценки \eqref{8.91} для таких функций $f\in L_\mu^{p(x)}([-1,1])$, для которых  $\|f\|_{p(\cdot),\mu}([-1,1])\le1$. Но в этом случае мы можем воспользоваться неравенством \eqref{8.8} и леммой 8.2. Это дает
%  \begin{equation}\label{8.92}
%\|S_n^{\alpha,\beta}(f)\|_{p(\cdot),\mu}([0,1])\le c(\alpha,\beta,p).
%\end{equation}
%С другой стороны, если  мы определим функцию $g$ равенством $g(x)=f(-x)$, то с помощью равенства \eqref{7.9} нетрудно проверить, что
%\begin{equation}\label{8.93}
%S_n^{\alpha,\beta}(f,-x)=S_n^{\beta,\alpha}(g,x).
%\end{equation}
%Введем новый переменный показатель $s=s(x)=p(-x)$, тогда в силу \eqref{8.93}
%\begin{equation}\label{8.94}
%\int_{-1}^0|S_n^{\alpha,\beta}(f,x)|^{p(x)}\mu(x)dx=\int_0^1|S_n^{\beta,\alpha}(g,x)|^{s(x)}\mu(x)dx.
%\end{equation}
%Поскольку переменный показатель $s$ вместе с $p$ удовлетворяет всем условиям леммы 8.2, то имеет место оценка
%\begin{equation}\label{8.95}
%\int_0^1|S_n^{\beta,\alpha}(g,x)|^{s(x)}\mu(x)dx\le c(\beta,\alpha,s).
%\end{equation}
%Из \eqref{8.94} и \eqref{8.95} находим
%$\int_{-1}^0|S_n^{\alpha,\beta}(а,x)|^{p(x)}\mu(x)dx\le c(\alpha,\beta,p)$,
%а это неравенство эквивалентно неравенству
%\begin{equation}\label{8.96}
%\|S_n^{\alpha,\beta}(f)\|_{p(\cdot),\mu}([-1,0])\le c(\alpha,\beta,p).
%\end{equation}
%Сопоставляя \eqref{8.96} с \eqref{8.92}, убеждаемся в справедливости утверждения теоремы 1.
%\end{proof}



Доказательство этой теоремы базируется на ряде вспомогательных утверждений (леммы 1.1 -- 1.4), связанных с преобразованиями частичных сумм ряда Фурье -- Якоби $S_n^{\alpha,\beta}(f,x)$.
В первую очередь воспользовавшись формулой   Кристоффеля -- Дарбу \eqref{7.8}, перепишем \eqref{8.2} следующим образом
\begin{equation}\label{8.3}
S_n^{\alpha,\beta}(f,x)=\int_{-1}^1K^{\alpha,\beta}_n(x,y)f(y)\mu(y)dy.
\end{equation}

\begin{lemma}\label{l1}
Если $x\neq y$, то ядро Кристоффеля -- Дарбу $K^{\alpha,\beta}_n(x,y)$ допускает следующее представление
\begin{equation}\label{8.4}
K^{\alpha,\beta}_n(x,y)=K^{\alpha,\beta}_{n1}(x,y)+K^{\alpha,\beta}_{n2}(x,y),
\end{equation}
где
\begin{equation}\label{8.5}
K^{\alpha,\beta}_{n1}(x,y)=-\gamma_n(\alpha,\beta)
\frac{(1-x)P_n^{\alpha+1,\beta}(x)P_n^{\alpha,\beta}(y)}{x-y},
\end{equation}
\begin{equation}\label{8.6}
K^{\alpha,\beta}_{n2}(x,y)=\gamma_n(\alpha,\beta)
\frac{(1-y)P_n^{\alpha+1,\beta}(y)P_n^{\alpha,\beta}(x)}{x-y},
\end{equation}
где $\gamma_n(\alpha,\beta)=O(n)$ $(n\to\infty)$.
\end{lemma}
%Доказательство леммы \ref{l1} непосредственно вытекает из \eqref{7.8} и \eqref{7.10}.

С учетом леммы \ref{l1} равенство \eqref{8.3} можно переписать в следующем виде
$$
S_n^{\alpha,\beta}(f,x)=\gamma_n(\alpha,\beta)P_n^{\alpha,\beta}(x)\int_{-1}^1
\frac{(1-y)P_n^{\alpha+1,\beta}(y)f(y)\mu(y)dy}{x-y}
$$
\begin{equation}\label{8.7}
-\gamma_n(\alpha,\beta)(1-x)P_n^{\alpha+1,\beta}(x)\int_{-1}^1
\frac{P_n^{\alpha,\beta}(y)f(y)\mu(y)dy}{x-y},
\end{equation}
где интегралы понимаются в смысле главных значений Коши. В дальнейшем %(до теоремы 2)
мы рассмотрим случай, когда $\alpha,\beta>-1/2$, $\mu(x)=(1-x)^\alpha(1-x)^\beta$, $x\in[0,1)$.
 Поскольку функция $f(x)(1-x)^\alpha(1-x)^\beta$ интегрируема на $(-1,1)$, то в этом случае из \eqref{8.7} c учетом весовой оценки \eqref{7.7} для почти всех
 $x\in[0,1]$ выводим
 $$
(1-x)^{\frac{\alpha}{p(x)}}|S_n^{\alpha,\beta}(f,x)|\le
$$
$$
c(\alpha,\beta)\sqrt{n+1}(1-x)^{u(x)}\left|\int_{-1}^1
\frac{(1-y)\mu(y) P_n^{\alpha+1,\beta}(y)f(y)dy}{x-y}\right|
$$
$$
+c(\alpha,\beta)\sqrt{n+1}(1-x)^{v(x)}\left|\int_{-1}^1
\frac{\mu(y)P_n^{\alpha,\beta}(y)f(y)dy}{x-y}\right|\le
$$
\begin{equation}\label{8.8}
\sigma_1(f,x)+\sigma_2(f,x)+\sigma_3(f,x)+\sigma_4(f,x),
\end{equation}
где
\begin{equation}\label{8.9}
\sigma_1(f,x)=c(\alpha,\beta)\sqrt{n+1}(1-x)^{u(x)}\left|\int_{-1}^{-\frac12}
\frac{(1-y)\mu(y) P_n^{\alpha+1,\beta}(y)f(y)dy}{x-y}\right|,
\end{equation}
\begin{equation}\label{8.10}
\sigma_2(f,x)=c(\alpha,\beta)\sqrt{n+1}(1-x)^{u(x)}\left|\int_{-\frac12}^{1}
\frac{(1-y)\mu(y) P_n^{\alpha+1,\beta}(y)f(y)dy}{x-y}\right|,
\end{equation}
\begin{equation}\label{8.11}
\sigma_3(f,x)=c(\alpha,\beta)\sqrt{n+1}(1-x)^{v(x)}\left|\int_{-1}^{-\frac12}
\frac{\mu(y) P_n^{\alpha,\beta}(y)f(y)dy}{x-y}\right|,
\end{equation}
\begin{equation}\label{8.12}
\sigma_4(f,x)=c(\alpha,\beta)\sqrt{n+1}(1-x)^{v(x)}\left|\int_{-\frac12}^{1}
\frac{ P_n^{\alpha,\beta}(y)\mu(y)f(y)dy}{x-y}\right|,
\end{equation}
а $u(x)=\frac{\alpha}{p(x)}-\frac{\alpha}{2}-\frac14$, $v(x)=\frac{\alpha}{p(x)}-\frac{\alpha}{2}+\frac14$. Мы рассмотрим вопрос об оценке норм
$\|\sigma_i(f)\|_{p(\cdot)}([0,1])$ $(i=1,2,3,4)$ при $p \in \mathcal{ P}(-1,1)$.
Следующее утверждение является основополагающим:

\begin{lemma}\label{l2}
Пусть $\alpha,\beta>-1/2$, $p\in\mathcal{ P}(-1,1)$, $\mu=\mu(x)=(1-x)^\alpha(1-x)^\beta$, $f\in L_\mu^{p(x)}([-1,1])$, $\|f\|_{p(\cdot)}([-1,1])\le1$,
\begin{equation}\label{8.13}
4\frac{\alpha+1}{2\alpha+3}<p(1)<4\frac{\alpha+1}{2\alpha+1},\quad
4\frac{\beta+1}{2\beta+3}<p(-1)<4\frac{\beta+1}{2\beta+1}.
\end{equation}
Тогда
\begin{equation}\label{8.14}
\|\sigma_i(f)\|_{p(\cdot)}([0,1])\le c(\alpha,\beta,p)\quad (i=1,2,3,4).
\end{equation}
\end{lemma}
%Доказательству этой леммы мы предпошлем некоторые вспомогательные ут-\linebreak верждения.

\begin{lemma}\label{l3}
Пусть $0<\gamma<1$. Тогда имеет место неравенство
\begin{equation}\label{8.15}
|a^\gamma-1|\le\frac{2|a-1|}{a^{1-\gamma}+1},\quad  0\le a<\infty.
\end{equation}
\end{lemma}
%\begin{proof}
%В самом деле, если $a\ge1$, то нужно проверить, что $(a^\gamma-1)(a^{1-\gamma}+1)<2(a-1)$ или, что то же, $1-a\le a^{1-\gamma}-a^{\gamma} $. Но это неравенство справедливо, так как $a^{\gamma}\le a$ и  $1\le a^{1-\gamma}$. Если же $0<a<1$, то нужно проверить,  что $(1-a^\gamma)(a^{1-\gamma}+1)<2(1-a)$ или, что то же,  $ a^{1-\gamma}-a^{\gamma}\le 1-a$. Это неравенство следует из того, что $a^{1-\gamma}<1$ и $a^\gamma>a$. Итак неравенство \eqref{8.15} справедливо при любом $a\ge0$.
%\end{proof}
Из леммы 1.3 непосредственно вытекает
\begin{lemma}
Пусть $-1<a,b<1$, $0<\gamma<1$. Тогда
\begin{equation}\label{8.16}
\frac{1}{|a-b|}\left|\left( \frac{1-b}{1-a}\right)^\gamma -1\right|\le \frac{2}{(1-a)^\gamma((1-a)^{1-\gamma}+(1-b)^{1-\gamma})}.
\end{equation}
\end{lemma}


%
%\section*{Доказательство леммы \ref{l2}.}
%
%Пусть $p\in\mathcal{ P}(-1,1)$, тогда  найдутся такие  положительные числа $\delta_s=\delta_s(p)$ $(s=1,2)$, что $p(x)=p(-1)$ при $x\in[-1,-1+\delta_1]$ и $p(x)=p(1)$ при $x\in [1-\delta_2,1]$. Аналогичными свойствами обладает и сопряженная функция $q(x)=p(x)/(p(x)-1)$. Перейдем к доказательству оценки \eqref{8.14} для $i=1$. Из \eqref{8.9} с учетом \eqref{7.7} для $0\le x<1$  имеем
%$$
%\sigma_1(f,x)\le c(\alpha)(1-x)^{u(x)}\int_{-1}^{-\frac12}\mu(y)
%(1+y)^{-\beta/2-1/4} |f(y)|dy=
%$$
%$$ c(\alpha)(1-x)^{u(x)}\left[\int_{-1}^{-1+\delta_1}+\int_{-1+\delta_1}^{-1/2}\right]\mu(y)
%(1+y)^{-\beta/2-1/4} |f(y)|dy
%$$
%\begin{equation}\label{8.17}
%\le\sigma_{11}(f,x)+ \sigma_{12}(f,x),
%\end{equation}
%где
%\begin{equation}\label{8.18}
%\sigma_{11}(f,x)= c(\alpha,\beta)(1-x)^{u(x)}\int\limits_{-1}^{-1+\delta_1}
%(1+y)^{-\frac\beta2-\frac14+\frac\beta{q(-1)}}(1-y^2)^\frac\beta{p(-1)} |f(y)|dy,
%\end{equation}
%\begin{equation}\label{8.19}
%\sigma_{12}(f,x)= c(\alpha,\beta)(1-x)^{u(x)}\int_{-1+\delta_1}^{-1/2}|f(y)|dy.
%\end{equation}
%Используя неравенство Гельдера, из \eqref{8.18} находим
%$$
%\sigma_{11}(f,x)\le c(\alpha,\alpha)(1-x)^{u(x)}\left(\int_{-1}^{-1+\delta_1}
%(1+y)^{-q(-1)(\frac\beta2+\frac14)+\beta}dy\right)^\frac{1}{q(-1)}\times
%$$
%\begin{equation}\label{8.20}
%\left(\int_{-1}^{-1+\delta_1}(1-y^2)^\beta |f(y)|^{p(-1)}dy \right)^\frac{1}{p(-1)}.
%\end{equation}
%Первый из интегралов в \eqref{8.20} конечен, так как в силу  \eqref{8.13} $\beta-q(-1)(\frac\beta2+\frac14)>-1$. Что касается второго из интегралов в \eqref{8.20},
%то он равен $\|f\|_{p(\cdot),\mu}([-1,-1+\delta_1])$, поэтому он допускает оценку
%$\|f\|_{p(\cdot),\mu}([-1,-1+\delta_1])\le \|f\|_{p(\cdot),\mu}([-1,1])\le1$.
%Отсюда и из \eqref{8.20} находим
%\begin{equation}\label{8.21}
%\sigma_{11}(f,x)\le c(p,\alpha,\beta)(1-x)^{u(x)}.
%\end{equation}
%С другой стороны, для $\sigma_{12}(f,x)$ из \eqref{8.19} имеем
%$$
%\sigma_{12}(f,x)\le c(p,\alpha,\beta)(1-x)^{u(x)}
%\int_{-1+\delta_1}^{-\frac12}|f(y)|dy
%$$
%$$
%\le
%c(p,\alpha,\beta)(1-x)^{u(x)}\|f\|_{p(\cdot)}([-1,-1+\delta_1])
%$$
%\begin{equation}\label{8.22}
% \le c(p,\alpha,\beta)(1-x)^{u(x)}\|f\|_{p(\cdot),\mu}([-1,1])\le c(p,\alpha,\beta)(1-x)^{u(x)}.
%\end{equation}
%Из \eqref{8.17}, \eqref{8.21} и \eqref{8.22} получаем
%\begin{equation}\label{8.23}
%\sigma_{1}(f,x)\le c(p,\alpha,\beta)(1-x)^{u(x)}\quad (0\le x<1).
%\end{equation}
%Совершенно аналогично из \eqref{8.11} выводим
%\begin{equation}\label{8.24}
%\sigma_{3}(f,x)\le c(p,\alpha,\beta)(1-x)^{v(x)}\quad (0\le x<1).
%\end{equation}
%Далее, из \eqref{8.23} имеем
%$$
%\int_0^1(\sigma_1(f,x))^{p(x)}dx=\int_0^{1-\delta_2}(\sigma_1(f,x))^{p(x)}dx+
%\int_{1-\delta_2}^1(\sigma_1(f,x))^{p(1)}dx\le
%$$
%$$
%c(\alpha,\beta,p)+c(\alpha,\beta,p)\int_{1-\delta_2}^1(1-x)^{p(1)u(1)}dx=
%$$
%$$
%c(\alpha,\beta,p)+c(\alpha,\beta,p)\int_{1-\delta_2}^1
%(1-x)^{p(1)(\frac{\alpha}{p(1)}-\frac{\alpha}{2}-\frac14)}dx=
%$$
%\begin{equation}\label{8.25}
%c(\alpha,\beta,p)+c(\alpha,\beta,p)\int_{1-\delta_2}^1
%(1-x)^{\alpha-{p(1)}(\frac{\alpha}{2}+\frac14)}dx\le c(\alpha,\beta,p),
%\end{equation}
%так как из \eqref{8.13} следует, что $p(1)<4(\alpha+1)/(2\alpha+1)$ и, стало быть $\alpha-{p(1)}(\frac{\alpha}{2}+\frac14)>-1$. Из \eqref{8.25} выводим $\|\sigma_1(f)\|_{p(\cdot)}([0,1])\le c(\alpha,\beta,p)$, и аналогично, из \eqref{8.24} выводим
%$\|\sigma_3(f)\|_{p(\cdot)}([0,1])\le c(\alpha,\beta,p)$. Таким образом, оценка \eqref{8.14} для $i=1$ и $i=3$ доказана.
%
%Перейдем к случаю $i=2$. Положим
%\begin{equation}\label{8.26}
%A_n(y)=\sqrt{n+1}(1+y)^\beta P_n^{\alpha+1,\beta}(y)(1-y)^{\frac{\alpha}{2}+\frac34},
%\end{equation}
%тогда из \eqref{8.10} имеем
%$$
%\sigma_2(f,x)=c(\alpha,\beta)(1-x)^{u(x)}\left|\left(\int_{-\frac12}^{1-\delta_2}+
%\int_{1-\delta_2}^1\right)
%\frac{A_n(y)(1-y)^{\frac{\alpha}{2}+\frac14} f(y)dy}{x-y}\right|
%$$
%\begin{equation}\label{8.27}
%\le \sigma_{21}(f,x)+\sigma_{22}(f,x),
%\end{equation}
%где $\sigma_{21}(f,x)=c(\alpha,\beta)(1-x)^{u(x)}|\int_{-\frac12}^{1-\delta_2}\mathcal{ F}|$, $\sigma_{22}(f,x)=c(\alpha,\beta)(1-x)^{u(x)}|\int_{1-\delta_2}^1\mathcal{ F}|$, а $\mathcal{ F}$ -- выражение, фигурирующее под знаком интеграла в левой части неравенства \eqref{8.27}. Положим $\varphi_n(y)=A_n(y)(1-y)^{\frac{\alpha}{2}+\frac14}$ и заметим, что в силу \eqref{7.7} найдется постоянная $c(\alpha,\beta,p)$, для которой
%\begin{equation}\label{8.28}
%|\varphi_n(y)|\le c(\alpha,\beta,p)\quad (-1/2\le y\le 1-\delta_2/2).
%\end{equation}
%Запишем $\sigma_{21}(f,x)$ в виде
%\begin{equation}\label{8.29}
%\sigma_{21}(f,x)=c(\alpha,\beta)(1-x)^{u(x)}\left|\int_{-\frac12}^{1-\delta_2}
%\frac{\varphi_n(y) f(y)dy}{x-y}\right|
%\end{equation}
%и рассмотрим два случая: a) $0\le x\le 1-\delta_2/2$; b) $ 1-\delta_2/2< x<1$.
% Если  $0\le x\le 1-\delta_2/2$, то из \eqref{4.3} и \eqref{8.29} имеем
%\begin{equation}\label{8.30}
%\sigma_{21}(f,x)\le c(\alpha,\beta,p)\left|\int_{-\frac12}^{1-\delta_2}
%\frac{\varphi_n(y) f(y)dy}{x-y}\right|=c(\alpha,\beta,p)|H^{-1/2,1-\delta_2}g(x)|,
%\end{equation}
%где $g(y)=\varphi_n(y) f(y)$. Обратимся к оценке \eqref{4.4}, которая с учетом \eqref{8.28}
%дает
%$$
%\|H^{-\frac12,1-\delta_2}g\|_{p(\cdot)}(\mathbb{R})\le c(\alpha,\beta,p)\|g\|_{p(\cdot)}([-1/2,1-\delta_2])
%$$
%$$\le c(\alpha,\beta,p)\|f\|_{p(\cdot)}([-1/2,1-\delta_2])\le c(\alpha,\beta,p)\|f\|_{p(\cdot),\mu}([-1/2,1-\delta_2])$$
%\begin{equation}\label{8.31}
%\le c(\alpha,\beta,p)\|f\|_{p(\cdot),\mu}([-1,1])\le c(\alpha,\beta,p).
%\end{equation}
%Из \eqref{8.30} и \eqref{8.31} имеем
%\begin{equation}\label{8.32}
%\|\sigma_{21}(f)\|_{p(\cdot)}([0,1-\delta_2/2])\le c(\alpha,\beta,p).
%\end{equation}
% Если же $ 1-\delta_2/2< x<1$, то
%\begin{equation}\label{8.33}
%\sigma_{21}(f,x)=c(\alpha,\beta,p)(1-x)^{u(1)}\int_{-\frac12}^{1-\delta_2}
%|\varphi_n(y) f(y)|dy.
%\end{equation}
%С другой стороны, из \eqref{2.5}  и \eqref{8.28} имеем
%$$
%\int_{-\frac12}^{1-\delta_2}|g(y)|dy\le c(p)\|g\|_{p(\cdot)}([-1/2,1-\delta_2])\le
%$$
%\begin{equation}\label{8.34}
%c(\alpha,\beta,p)\|f\|_{p(\cdot),\mu}([-1/2,1-\delta_2])\le c(\alpha,\beta,p).
%\end{equation}
%Из \eqref{8.33} и  \eqref{8.34} находим (см. \eqref{8.25})
%$$
%\int_{1-\delta_2/2}^1(\sigma_{21}(f,x))^{p(x)}dx =
%\int_{1-\delta_2/2}^1(\sigma_{21}(f,x))^{p(1)}dx\le
%$$
%\begin{equation}\label{8.35}
%c(\alpha,\beta,p)\int_{1-\delta_2/2}^1(1-x)^{p(1)u(1)}dx\le c(\alpha,\beta,p).
%\end{equation}
% Соединяя оценки \eqref{8.32} и \eqref{8.35}, мы можем записать
%\begin{equation}\label{8.36}
% \|\sigma_{21}(f)\|_{p(\cdot)}([0,1])\le c(\alpha,\beta,p).
% \end{equation}
%Перейдем к оценке номы функции $\sigma_{22}(f)$. Рассмотрим два случая: 1) $0\le x\le 1-\delta_2$; 2) $ 1-\delta_2< x<1$.
% Если  $0\le x\le 1-\delta_2$, то из  \eqref{8.27} имеем
%\begin{equation}\label{8.37}
%\sigma_{22}(f,x)\le c(\alpha,\beta,p)\left|\int\limits_{1-\delta_2}^1
%\frac{A_n(y)(1-y)^{\frac\alpha2+\frac14} f(y)dy}{x-y}\right|\le\sigma_{221}(f,x)+\sigma_{222}(f,x),
%\end{equation}
%где $\sigma_{221}(f,x)= c(\alpha,\beta,p)\left|\int_{1-\delta_2}^{1-\delta_2/2}\mathcal{ F}\right|$, $\sigma_{222}(f,x)= c(\alpha,\beta,p)\left|\int_{1-\delta_2/2}^1\mathcal{ F}\right|$.
%Положим $g(y)=\varphi_n(y)f(y)$ и воспользуемся обозначением
%\eqref{4.3}, тогда из \eqref{8.37} имеем
%$
%\sigma_{221}(f,x)=c(\alpha,\beta,p)|H^{1-\delta_2,1-\delta_2/2}g(x)|,
%$
%поэтому в силу \eqref{4.4} и \eqref{8.28}
%$$
%\|\sigma_{221}(f)\|_{p(\cdot)}([0,1-\delta_2/2])\le c(\alpha,\beta,p)\|g\|_{p(\cdot)}([1-\delta_2,1-\delta_2/2])\le
%$$
%\begin{equation}\label{8.38}
%c(\alpha,\beta,p)\|f\|_{p(\cdot),\mu}([1-\delta_2,1-\delta_2/2])\le c(\alpha,\beta,p)\|f\|_{p(\cdot),\mu}([-1,1])\le c(\alpha,\beta,p).
%\end{equation}
%Далее, из \eqref{7.7} и \eqref{8.26} следует, что
%\begin{equation}\label{8.39}
%|A_n(y)|\le c(\alpha,\beta)\quad (-1/2\le y<1),
%\end{equation}
%поэтому из \eqref{8.37} имеем
%$$
%\sigma_{222}(f,x)\le c(\alpha,\beta,p)\int\limits_{1-\delta_2/2}^1
%(1-y)^{\frac\alpha2+\frac14-\alpha/p(1)}(1-y^2)^{\alpha/p(1)} |f(y)|dy\le
%$$
%\begin{equation}\label{8.40}
%c(\alpha,\beta,p)\|f\|_{p(\cdot),\mu}([1-\delta_2/2,1])\left(\int_{1-\delta_2/2}^1
%(1-y)^{(\frac\alpha2+\frac14-\frac\alpha{p(1)})q(1)}\right)^{\frac1{q(1)}},
%\end{equation}
%где $q(1)=p(1)/(p(1)-1)$. Последний интеграл конечен, так как в силу \eqref{8.13} $(\frac\alpha2+\frac14-\frac\alpha{p(1)})q(1)>-1$.
%В самом деле, это неравенство эквивалентно неравенству $4(\alpha+1)/(2\alpha+5)<p(1)$, котрое, очевидно, вытекает из \eqref{8.13}. Таким образом, из \eqref{8.40} выводим
%$
%\sigma_{222}(f,x)\le c(\alpha,\beta,p)\|f\|_{p(\cdot),\mu}([1-\delta_2/2,1])
%\le c(\alpha,\beta,p)\|f\|_{p(\cdot),\mu}([-1,1])\le c(\alpha,\beta,p),\quad (0\le x\le 1-\delta_2),
%$
%откуда, в свою очередь, находим
%\begin{equation}\label{8.41}
%\|\sigma_{222}(f)\|_{p(\cdot)}([0,1-\delta_2])\le c(\alpha,\beta,p).
%\end{equation}
%Оценки \eqref{8.37}, \eqref{8.38} и \eqref{8.41}, взятые вместе дают
%\begin{equation}\label{8.42}
%\|\sigma_{22}(f)|_{p(\cdot)}([0,1-\delta_2])\le c(\alpha,\beta,p).
%\end{equation}
%Рассмотрим случай $1-\delta_2\le x<1$ и заметим, что в этом случае $p(x)=p(1)$, $u(x)=u(1)$, поэтому мы можем представить $\sigma_{22}(f,x)$ в следующем виде (см.\eqref{8.27})
%$$
%\sigma_{22}(f,x)=c(\alpha,\beta)\left|\int\limits_{1-\delta_2}^{1}
%\frac{A_n(y)(1-y)^{\frac{\alpha}{p(1)}} f(y)}{x-y}\left(\frac{1-y}{1-x}\right)^{\frac\alpha2+\frac14-\frac\alpha{p(1)}}dy\right|
%$$
%\begin{equation}\label{8.43}
%\le U(f,x)+V(f,x),
%\end{equation}
%где
%\begin{equation}\label{8.44}
%U(f,x)=c(\alpha,\beta)\left|\int_{1-\delta_2}^{1}
%\frac{A_n(y)(1-y)^{\frac{\alpha}{p(1)}} f(y)}{x-y}dy\right|,
%\end{equation}
%\begin{equation}\label{8.45}
%V(f,x)=c(\alpha,\beta)\int_{1-\delta_2}^{1}
%\frac{|A_n(y)|(1-y)^{\frac{\alpha}{p(1)}} |f(y)|}{|x-y|}\left|\left(\frac{1-y}{1-x}\right)^{\frac\alpha2+\frac14-\frac\alpha{p(1)}}-1\right|dy.
%\end{equation}
%Рассмотрим сначала $U(f,x)$. Положим  $\psi(y)=A_n(y)(1-y)^{\frac{\alpha}{p(1)}}f(y)$ и воспользуемся определением \eqref{4.3}, тогда из \eqref{8.44} имеем
%$U(f,x)=c(\alpha,\beta)|H^{1-\delta_2,1}\psi(x)|$, поэтому из оценок \eqref{4.4} и \eqref{8.39}
%находим
%$$
%\|U(f)\|_{p(\cdot)}([1-\delta_2,1])\le c(\alpha,\beta, p)\|\psi\|_{p(\cdot)}([1-\delta_2,1])\le
%$$
%\begin{equation}\label{8.46}
% c(\alpha,\beta, p)\|f\|_{p(\cdot),\mu}([1-\delta_2,1])\le c(\alpha,\beta,p).
%\end{equation}
%Чтобы оценить $\|V(f)\|_{p(\cdot)}([1-\delta_2,1])$ обратимся к соотношениям \eqref{2.3}, \eqref{2.4} и \eqref{8.45}, из которых, полагая  $\gamma=\alpha/2+1/4-1/p(1)$,
%\begin{equation}\label{8.47}
%K_\gamma(x,y)=\frac{1}{|x-y|}\left|\left(\frac{1-y}{1-x}\right)^\gamma-1\right|,
%\end{equation}
%находим
%\begin{equation}\label{8.48}
%\|V(f)\|_{p(\cdot)}([1-\delta_2,1])\le c(\alpha,\beta) \sup_{g}\int\limits_{1-\delta_2}\int\limits_{1-\delta_2}|g(x)\psi(y)|K_\gamma(x,y)dxdy,
%\end{equation}
%где верхняя грань берется по всем $g\in L^{q(1)}([1-\delta_2,1])$, для которых $\|g\|_{p(\cdot)}([1-\delta_2,1])\le 1$. Оценим двойной интеграл $I$ из \eqref{8.48}. Имеем $$
%I= \int_{1-\delta_2}^1\int_{1-\delta_2}^1|\psi(y)|K_\gamma^\frac1{p(1)}(x,y)
%\left(\frac{1-y}{1-x}\right)^\frac1{p(1)q(1)}\times
%$$
%$$
%|g(x)|K_\gamma^\frac1{q(1)}(x,y)
%\left(\frac{1-y}{1-x}\right)^{-\frac1{p(1)q(1)}} dxdy\le
%$$
%\begin{equation}\label{8.49}
%\left(\int_{1-\delta_2}^1|g(x)|^{q(1)}F_1(x)dx\right)^\frac1{q(1)}
%\left(\int_{1-\delta_2}^1|\psi(y)|^{p(1)}F_2(y)dy\right)^\frac1{p(1)},
%\end{equation}
%где
%\begin{equation}\label{8.50}
%F_1(x)=\int\limits_{1-\delta_2}^1K_\gamma(x,y)
%\left(\frac{1-x}{1-y}\right)^{\frac1{p(1)}} dy,\,\, F_2(y)=\int\limits_{1-\delta_2}^1K_\gamma(x,y)
%\left(\frac{1-y}{1-x}\right)^{\frac1{q(1)}} dx.
%\end{equation}
%Покажем, что функции $F_1$ и $F_2$ ограничены на $[1-\delta_2,1]$. С этой целью заметим, что из \eqref{8.13}   при $\alpha>-1/2$ для $\gamma=\frac\alpha2+\frac14-\frac\alpha{p(1)}$ вытекают неравенства $0<\gamma<\frac1{p(1)}$, поэтому мы можем воспользоваться леммой 8.3 (неравенство \eqref{8.16} ), из которой и из \eqref{8.50} имеем
%$$
%F_1(x)\le 2\int_{1-\delta_2}^1\frac{(1-x)^{\frac{1}{p(1)}-\gamma}dy}{
%(1-y)^\frac{1}{p(1)}((1-y)^{1-\gamma}+(1-x)^{1-\gamma})}\le
%$$
%$$
% 2\int_{1-\delta_2}^x\frac{(1-x)^{\frac{1}{p(1)}-\gamma}dy}{
%(1-y)^{\frac{1}{p(1)}+1-\gamma}}+2\int_x^1\frac{(1-x)^{\frac{1}{p(1)}-\gamma}dy}{
%(1-y)^\frac{1}{p(1)}(1-x)^{1-\gamma}}\le
%$$
%\begin{equation}\label{8.51}
%2(1-x)^{\frac{1}{p(1)}-\gamma}\frac{(1-x)^{\gamma-\frac{1}{p(1)}}}{\frac{1}{p(1)}-\gamma}+
%2(1-x)^{\frac{1}{p(1)}-1}\frac{(1-x)^{1-\frac{1}{p(1)}}}{1-\frac{1}{p(1)}}\le c(\alpha,p),
%\end{equation}
%$$
%F_2(x)\le 2\int_{1-\delta_2}^1\frac{(1-y)^{\frac{1}{q(1)}}dx}{
%(1-x)^{\gamma+\frac{1}{q(1)}}((1-y)^{1-\gamma}+(1-x)^{1-\gamma})}\le
%$$
%$$
% 2\int_{1-\delta_2}^y\frac{(1-y)^{\frac{1}{q(1)}}dx}{
%(1-x)^{\frac{1}{q(1)}+1}}+2\int_y^1\frac{(1-y)^{\frac{1}{q(1)}-1+\gamma}dx}{
%(1-x)^{\frac{1}{q(1)}+\gamma}}\le
%$$
%\begin{equation}\label{8.52}
%2(1-y)^{\frac{1}{q(1)}}\frac{(1-y)^{-\frac{1}{q(1)}}}{1/q(1)}+
%2(1-y)^{\frac{1}{q(1)}-1+\gamma}\frac{(1-y)^{1-\frac{1}{q(1)}-\gamma}}{1-\frac{1}{q(1)}-\gamma}
%\le c(\alpha,p).
%\end{equation}
%Из \eqref{8.39}, \eqref{8.49}, \eqref{8.51} и  \eqref{8.52}  следует, что
%$$
%I\le c(\alpha,p)\|g\|_{q(1)}([1-\delta_2,1])\|\psi\|_{p(1)}([1-\delta_2,1])\le
%c(\alpha,\beta,p)\|A_n\mu^\frac{1}{p(1)}f\|_{p(1)}([1-\delta_2,1])
%$$
%\begin{equation}\label{8.53}
%\le c(\alpha,\beta,p)\|f\|_{p(\cdot),\mu}([1-\delta_2,1])\le c(\alpha,\beta,p)\|f\|_{p(\cdot),\mu}([-1,1]),
%\end{equation}
%Оценки \eqref{8.48} и \eqref{8.53}, взятые вместе дают
%\begin{equation}\label{8.54}
%\|V(f)\|_{p(\cdot)}([1-\delta_2,1])\le c(\alpha,\beta,p).
%\end{equation}
%Из \eqref{8.46} и \eqref{8.54} и неравенства \eqref{8.43} выводим
%\begin{equation}\label{8.55}
%\|\sigma_{22}(f)\|_{p(\cdot)}([1-\delta_2,1])\le c(\alpha,\beta,p).
%\end{equation}
%а из   \eqref{8.42} и \eqref{8.55} получаем
%\begin{equation}\label{8.56}
%\|\sigma_{22}(f)\|_{p(\cdot)}([0,1])\le c(\alpha,\beta,p).
%\end{equation}
%Сопоставляя \eqref{8.36} и \eqref{8.56} с \eqref{8.27}, убеждаемся в  справедливости утверждения леммы 8.2 при $i=2$.
%Перейдем к доказательству утверждения леммы 8.2 для $i=4$. Положим
%\begin{equation}\label{8.57}
%B_n(y)=\sqrt{n+1}(1+y)^\beta(1-y)^{\frac{\alpha}{2}+\frac14}P_n^{\alpha,\beta}(y),
%\end{equation}
%\begin{equation}\label{8.58}
%\Phi_n(y)=B_n(y)(1-y)^{\frac{\alpha}{2}-\frac14}
%\end{equation}
%и заметим, что в силу \eqref{7.7}
%\begin{equation}\label{8.59}
%|B_n(y)|\le c(\alpha,\beta)\quad(-1/2\le y<1),
%\end{equation}
%\begin{equation}\label{8.60}
%|\Phi_n(y)|\le c(\alpha,\beta,p)\quad(-1/2\le y<1-{\delta_2}/{2}).
%\end{equation}
%Из \eqref{8.12} и \eqref{8.57} имеем
%\begin{equation}\label{8.61}
%\sigma_4(f,x)=c(\alpha,\beta)(1-x)^{v(x)}\left|\int_{-\frac12}^{1}
%\frac{\Phi_n(y)f(y)dy}{x-y}\right|\le\sigma_{41}(f,x)+\sigma_{42}(f,x),
%\end{equation}
%где $v(x)=\frac{\alpha}{p(x)}-\frac{\alpha}{2}+\frac14$,
%\begin{equation}\label{8.62}
%\sigma_{41}(f,x)=c(\alpha,\beta)(1-x)^{v(x)}\left|\int_{-\frac12}^{1-\delta_2}
%\frac{\Phi_n(y)f(y)dy}{x-y}\right|,
%\end{equation}
%\begin{equation}\label{8.63}
%\sigma_{42}(f,x)=c(\alpha,\beta)(1-x)^{v(x)}\left|\int_{1-\delta_2}^1
%\frac{\Phi_n(y)f(y)dy}{x-y}\right|.
%\end{equation}
%Оценим $|\sigma_{41}(f,x)|$. С этой целью рассмотрим два случая: 1) $0\le x\le1-\delta_2/2$; 2) $1-\delta_2/2\le x<1$. Если $0\le x\le1-\delta_2/2$, то из \eqref{8.62} и \eqref{4.3} следует, что
%\begin{equation}\label{8.64}
%\sigma_{41}(f,x)\le c(\alpha,\beta,p)\left|\int_{-\frac12}^{1-\delta_2}
%\frac{\Phi_n(y)f(y)dy}{x-y}\right|=c(\alpha,\beta,p)\left|H^{-\frac12,1-\delta_2}g(x)\right|,
%\end{equation}
%где $g(y)=\Phi_n(y)f(y)$, если же  $1-\delta_2/2\le x<1$, то
%\begin{equation}\label{8.65}
%\sigma_{41}(f,x)\le c(\alpha,\beta, p)(1-x)^{v(1)}\int_{-\frac12}^{1-\delta_2}
%|\Phi_n(y)f(y)|dy.
%\end{equation}
%Обратимся к оценке \eqref{4.4}, тогда, полагая $g(y)=\Phi_n(y)f(y)$ и учитывая \eqref{8.57} -- \eqref{8.60},  мы можем записать
%$$
%\|H^{-\frac12,1-\delta_2}g\|_{p(\cdot)}(\mathbb{R})\le c(\alpha,\beta,p)\|g\|_{p(\cdot)}([-1/2,1-\delta_2])\le
%$$
%\begin{equation}\label{8.66}
%c(\alpha,\beta,p)\|f\|_{p(\cdot)}([-1/2,1-\delta_2])\le c(\alpha,\beta,p)\|f\|_{p(\cdot),\mu}([-1/2,1-\delta_2]).
%\end{equation}
%Из \eqref{8.64} и \eqref{8.66} выводим
%\begin{equation}\label{8.67}
%\|\sigma_{41}(f)\|_{p(\cdot)}([0,1-\delta_2])\le c(\alpha,\beta,p)\|f\|_{p(\cdot),\mu}([-1,1])\le c(\alpha,\beta,p).
%\end{equation}
%С другой стороны, в силу \eqref{2.5} и \eqref{8.60}
%$$
%\int\limits_{-\frac12}^{1-\delta_2}|g(y)|dy\le c(p)\|g\|_{p(\cdot)}([-\frac12,1-\delta_2])
%$$
%$$\le c(\alpha,\beta,p)\|f\|_{p(\cdot),\mu}([-\frac12,1-\delta_2])\le c(\alpha,\beta,p),$$
%поэтому из \eqref{8.65} для $p\in\mathcal{ P}(-1,1)$ имеем
%$$\int\limits_{1-\delta_2}^{1}(\sigma_{41}(f,x))^{p(x)}dx\le c(\alpha,\beta,p)\int\limits_{1-\delta_2}^{1}(1-x)^{v(1)p(1)}dx=$$
%\begin{equation}\label{8.68}
%\frac{c(\alpha,\beta,p)\delta_2^{v(1)p(1)+1}}{v(1)p(1)+1}\le c(\alpha,\beta,p),
%\end{equation}
%так как из \eqref{8.13} вытекает, что $v(1)>0$. Чтобы убедиться в этом заметим, что из условия \eqref{8.13} имеем $4(\alpha+1)/(2\alpha+1)>p(1)$. С другой строны, если $\alpha>1/2$, то
%$4\alpha/(2\alpha-1)>4(\alpha+1)/(2\alpha+1)$ и, стало быть, $4\alpha/(2\alpha-1)>p(1)$ или, что то же, $1/p(1)>(2\alpha-1)/4\alpha$ и отсюда $\alpha/p(1)>\alpha/2-1/4$, а это означает, что $v(1)=v_\alpha(1)=-\alpha/2+1/4+\alpha/p(1)>0$. Если же $-1/2<\alpha\le1/2$, то $v_\alpha(1)>0$   каково бы ни было $p(1)>0$. В самом деле, прямая $l(\alpha)=v_\alpha(1)$ в точках $\alpha=-1/2$ и $\alpha=0$ принимает значения $l(-1/2)=1/2-1/(2p(1))>0$ и $l(0)=1/4$, соответственно, поэтому для любого $\alpha>-1/2$ имеем $v_\alpha(1)>0$.
%
%Из \eqref{8.67} и \eqref{8.68} находим
%\begin{equation}\label{8.69}
%\|\sigma_{41}(f)\|_{p(\cdot)}([0,1])\le c(\alpha,\beta,p).
%\end{equation}
%Оценим $|\sigma_{42}(f,x)|$, для чего рассмотрим два случая: 1)
%$0\le x \le1-\delta_2$; 2) $1-\delta_2\le x<1$. Если $0\le x \le1-\delta_2$, то согласно \eqref{8.63}
%\begin{equation}\label{8.70}
%\sigma_{42}(f,x)=c(\alpha)(1-x)^{v(x)}\left|\int_{1-\delta_2}^1
%\frac{\Phi_n(y)f(y)dy}{x-y}\right|\le \sigma_{421}(f,x)+\sigma_{422}(f,x),
%\end{equation}
% где
% \begin{equation}\label{8.71}
%\sigma_{421}(f,x)=c(\alpha,\beta,p)\left|\int_{1-\delta_2}^{1-\delta_2/2}
%\frac{\Phi_n(y)f(y)dy}{x-y}\right|,
%\end{equation}
%\begin{equation}\label{8.72}
%\sigma_{422}(f,x)=c(\alpha,\beta,p)\left|\int_{1-\delta_2/2}^1
%\frac{\Phi_n(y)f(y)dy}{x-y}\right|.
%\end{equation}
%Положим $r_n=r_n(y)=\Phi_n(y)f(y)$ и воспользуемся определением \eqref{4.3}, тогда можем переписать \eqref{8.71} так $\sigma_{421}(f,x)=c(\alpha,\beta,p)\left|H^{1-\delta_2,1-\delta_2/2}r_n(x)\right|$,
%поэтому в силу \eqref{4.4} и \eqref{8.60}
%$$
%\|\sigma_{421}(f)\|_{p(\cdot)}([0,1-\delta_2])\le c(\alpha,\beta,p)\|r_n\|_{p(\cdot)}([1-\delta_2,1-\delta_2/2])\le
%$$
%\begin{equation}\label{8.73}
%c(\alpha,\beta,p)\|f\|_{p(\cdot)}([1-\delta_2,1-\delta_2/2])\le c(\alpha,\beta,p)\|f\|_{p(\cdot),\mu}([-1,1])\le c(\alpha,\beta,p).
%\end{equation}
%Далее, в силу \eqref{8.59}, \eqref{8.72} и с учетом того, что $\|f\|_{p(\cdot),\mu}([-1,1])\le1$, а из левого из неравенств \eqref{8.13} вытекает $(\frac\alpha2-\frac14-\frac\alpha{p(1)})q(1)+1>0$, мы можем записать
%$$
%\sigma_{422}(f,x)\le c(\alpha,\beta,p)\int_{1-\delta_2/2}^1(1-y)^{\frac\alpha2-\frac14-\frac\alpha{p(1)}}
%(1-y^2)^\frac{\alpha}{p(1)}|f(y)|dy\le
%$$
%$$
%c(\alpha,\beta,p)\|f\|_{p(\cdot),\mu}([-1,1])
%\left(\int_{1-\delta_2/2}^1(1-y)^{(\frac\alpha2-\frac14-\frac\alpha{p(1)})q(1)}
%\right)^{\frac{1}{q(1)}}\le c(\alpha,\beta,p),
%$$
%поэтому
%\begin{equation}\label{8.74}
%\|\sigma_{421}(f)\|_{p(\cdot)}([0,1-\delta_2])\le c(\alpha,\beta,p).
%\end{equation}
%Оценки  \eqref{8.70},  \eqref{8.72} и  \eqref{8.74}, взятые вместе, дают
%\begin{equation}\label{8.75}
%\|\sigma_{42}(f)\|_{p(\cdot)}([0,1-\delta_2])\le c(\alpha,\beta,p).
%\end{equation}
%Рассмотрим случай $1-\delta_2\le x<1$.  Поскольку в этом случае $p(x)=p(1)$ и, следовательно, $v(x)=v(1)$, то мы можем записать
%\begin{equation}\label{8.76}
%\sigma_{42}(f,x)=c(\alpha)\left|\int_{1-\delta_2}^1
%\frac{B_n(y)(1-y)^{\frac{\alpha}{p(1)}}f(y)dy}{x-y}\left(\frac{1-x}{1-y}\right)^{v(1)}\right|\le
%G(f,x)+Q_(f,x),
%\end{equation}
%где
%\begin{equation}\label{8.77}
%G(f,x)=c(\alpha)\left|\int_{1-\delta_2}^1
%\frac{B_n(y)(1-y)^{\frac{\alpha}{p(1)}}f(y)dy}{x-y}\right|,
%\end{equation}
%\begin{equation}\label{8.78}
%Q(f,x)=c(\alpha)\left|\int_{1-\delta_2}^1
%\frac{B_n(y)(1-y)^{\frac{\alpha}{p(1)}}f(y)dy}{x-y}
%\left[\left(\frac{1-x}{1-y}\right)^{v(1)}-1\right]\right|.
%\end{equation}
%Рассмотрим $G(f,x)$. С этой целью положим $F(y)=B_n(y)(1-y)^\frac\alpha{p(1)}f(y)$ и обратимся к определению  \eqref{4.3}, тогда из \eqref{8.77} имеем $G(f,x)=c(\alpha)|H^{1-\delta_2,1}F|$, поэтому из оценки \eqref{4.4} с учетом \eqref{8.59} находим
%$$
%\|G(f)\|_{p(\cdot)}([1-\delta_2,1])\le c(\alpha,\beta,p)\|F\|_{p(\cdot)}([1-\delta_2,1])\le c(\alpha,\beta,p)\|f\|_{p(\cdot)}([1-\delta_2,1])
%$$
%\begin{equation}\label{8.79}
%\le c(\alpha,\beta,p)\|f\|_{p(\cdot)}([-1,1])\le c(\alpha,\beta,p).
%\end{equation}
%Далее нам понадобятся следующие неравенства: 1) $0<v(1)$; 2) $v(1)<1/q(1)$. Первое из них мы уже проверяли выше, поэтому остается убедиться в справедливости второго. Для этого заметим, что  в силу первого из неравенств
%\eqref{8.13} имеем $4(\alpha+1)/(2\alpha+3)<p(1)$, отсюда $(2\alpha+3)/4>(\alpha+1)/p(1)$, т.е. $-\alpha/2-3/4<-\alpha/p(1)-1/p(1)$ или, что то же $v_\alpha(1)<1-1/p(1)=1/q(1)$, что и требовалось доказать. Итак, мы проверили справедливость неравенств
%\begin{equation}\label{8.80}
%0<v(1)=v_\alpha(1)=-\frac\alpha2+\frac14+\frac\alpha{p(1)}<\frac{1}{q(1)}\quad (\alpha>-1/2).
%\end{equation}
%Оценим норму $\|Q(f)\|_{p(\cdot)}([1-\delta_2,1)$. Положим $\gamma=v_\alpha(1)$,
% \begin{equation}\label{8.81}
%K_\gamma(x,y)=\frac{1}{|x-y|}\left|\left(\frac{1-x}{1-y}\right)^{\gamma}-1\right|.
%\end{equation}
%Тогда с учетом \eqref{8.78} мы можем записать
%\begin{equation}\label{8.82}
%\|Q(f)\|_{p(\cdot)}([1-\delta_2,1])\le c(\alpha) \sup_{g}\int\limits_{1-\delta_2}\int\limits_{1-\delta_2}|g(x)Z(y)|K_\gamma(x,y)dxdy,
%\end{equation}
%где $Z(y)=B_n(y)(1-y)^\frac{\alpha}{p(1)}$, а  верхняя грань берется по всем $g\in L^{q(1)}([1-\delta_2,1])$, для которых $\|g\|_{p(\cdot)}([1-\delta_2,1])\le 1$. Оценим двойной интеграл $J$ из \eqref{8.82}. Имеем $$
%J= \int\limits_{1-\delta_2}\int\limits_{1-\delta_2}|F(y)|K_\gamma^\frac1{p(1)}(x,y)
%\left(\frac{1-y}{1-x}\right)^\frac1{p(1)q(1)}\times
%$$
%$$
%|g(x)|K_\gamma^\frac1{q(1)}(x,y)
%\left(\frac{1-x}{1-y}\right)^{\frac1{p(1)q(1)}} dxdy\le
%$$
%\begin{equation}\label{8.83}
%\left(\int\limits_{1-\delta_2}|g(x)|^{q(1)}F_3(x)dx\right)^\frac1{q(1)}
%\left(\int\limits_{1-\delta_2}|Z(y)|^{p(1)}F_4(y)dy\right)^\frac1{p(1)},
%\end{equation}
%где
%\begin{equation}\label{8.84}
%F_3(x)=\int\limits_{1-\delta_2}K_\gamma(x,y)
%\left(\frac{1-x}{1-y}\right)^{\frac1{p(1)}} dy,
%\end{equation}
%\begin{equation}\label{8.85}
% F_4(y)=\int\limits_{1-\delta_2}K_\gamma(x,y)
%\left(\frac{1-y}{1-x}\right)^{\frac1{q(1)}} dx.
%\end{equation}
%Покажем, что функции $F_3$ и $F_4$ ограничены на $(1-\delta_2,1)$.
%Поскольку для $\gamma=v_\alpha(1)$ справедливы неравенства \eqref{8.80}, то мы можем обратимся к лемме 8.4. Тогда из  \eqref{8.81}, \eqref{8.84}, \eqref{8.85} и \eqref{8.16} имеем
%$$
% F_3(x)\le \int_{1-\delta_2}^1 \left(\frac{1-x}{1-y}\right)^\frac{1}{p(1)}\frac{2dy}
% {(1-y)^\gamma((1-y)^{1-\gamma}+(1-x)^{1-\gamma})}<
%$$
%$$
%2(1-x)^\frac{1}{p(1)} \int_{1-\delta_2}^1\frac{(1-y)^{-\frac{1}{p(1)}-\gamma}dy}{(1-y)^{1-\gamma}+(1-x)^{1-\gamma}}\le
%$$
%$$
%2(1-x)^\frac{1}{p(1)} \int_{1-\delta_2}^x(1-y)^{-\frac{1}{p(1)}-1}dy+
%2(1-x)^{\frac{1}{p(1)}-1+\gamma} \int_x^1(1-y)^{-\frac{1}{p(1)}-\gamma}dy<
%$$
%$$
%2p(1)(1-x)^\frac{1}{p(1)}(1-x)^{-\frac{1}{p(1)}}+2(1-x)^{\frac{1}{p(1)}-1+\gamma}
%\frac{(1-x)^{-\frac{1}{p(1)}-\gamma+1}}{-\frac{1}{p(1)}-\gamma+1}=
%$$
%\begin{equation}\label{8.86}
% 2p(1)+\frac{2}{\frac1{q(1)}-\gamma}\le c(\alpha,\beta,p),
% \end{equation}
%$$
% F_4(y)\le \int_{1-\delta_2}^1 \left(\frac{1-y}{1-x}\right)^\frac{1}{q(1)}\frac{2dx}
% {(1-y)^\gamma((1-y)^{1-\gamma}+(1-x)^{1-\gamma})}<
%$$
%$$
%2(1-y)^{-\frac{1}{q(1)}} \int_{1-\delta_2}^y(1-x)^{-\frac{1}{p(1)}-1+\gamma}dx+
%2(1-y)^{\frac{1}{q(1)}-1} \int_y^1(1-x)^{-\frac{1}{q(1)}}dx<
%$$
%\begin{equation}\label{8.87}
%\frac{2(1-y)^{\frac{1}{q(1)}-\gamma}}{\frac{1}{q(1)}-\gamma}
%\left[(1-y)^{-\frac{1}{q(1)}+\gamma}-\delta_2^{-\frac{1}{q(1)}+\gamma}\right]+
%2(1-y)^{\frac{1}{q(1)}-1}\frac{(1-y)^{\frac{1}{p(1)}}}{1/p(1)}\le c(\alpha,\beta,p).
%\end{equation}
%Из  \eqref{8.59},  \eqref{8.83}, \eqref{8.86}, \eqref{8.87}  имеем
%$$
%J\le c(\alpha,\beta,p)\|g\|_{q(1)}([1-\delta_2,1])\|Z\|_{p(1)}([1-\delta_2,1])\le
%$$
%$$
%c(\alpha,\beta,p)\|B_n\mu^\frac{1}{p(1)}f\|_{p(1)}([1-\delta_2,1])\le
%c(\alpha,\beta,p)\|f\|_{p(\cdot),\mu}([-1,1])\le c(\alpha,\beta,p).
%$$
%Сопоставляя эту оценку с  \eqref{8.82},  выводим
%\begin{equation}\label{8.88}
%\|Q(f)\|_{p(\cdot)}([1-\delta_2,1])\le c(\alpha,\beta,p).
%\end{equation}
%Оценки  \eqref{8.79} и \eqref{8.88} и неравенство \eqref{8.76} дают
%\begin{equation}\label{8.89}
%\|\sigma_{42}(f)\|_{p(\cdot)}([1-\delta_2,1])\le c(\alpha,\beta,p).
%\end{equation}
%а из   \eqref{8.75} и \eqref{8.89} получаем
%\begin{equation}\label{8.90}
%\|\sigma_{42}(f)\|_{p(\cdot)}([0,1])\le c(\alpha,\beta,p).
%\end{equation}
%Сопоставляя \eqref{8.69} и \eqref{8.90} с \eqref{8.61}, убеждаемся в  справедливости утверждения леммы 8.2 при $i=4$. Итак, лемма 8.2 полностью доказана.







\begin{corollary}
Если выполнены условия теоремы 1.1,  то  система ортонормированных  полиномов  Якоби $\{p_n^{\alpha,\beta}(x)\}_{n=0}^\infty$ с $\alpha,\beta>-1/2$ является базисом пространства  $L^{p(x)}_\mu([-1,1])$, где $\mu=\mu(x)=(1-x)^\alpha(1+x)^\beta$, и, как следствие, $\|f-S_n^{\alpha,\beta}(f)\|_{p(\cdot),\mu}([-1,1])\to0$ при $n\to\infty$ для произвольной функции $f\in L^{p(x)}_\mu([-1,1])$.
\end{corollary}



Рассмотрим отдельно случай, когда $\alpha=-1/2$, который не входит в теорему 1.1. Хорошо известно (см., напр. \cite{ShIIBJWSege}), что сумма Фурье -- Якоби  $S_n^{\alpha,\alpha}(f)$ при $\alpha=-1/2$ представляет собой сумму Фурье -- Чебышева по полиномами Чебышева $T_n(x)=\cos(n\arccos x)$ $(n=0,1,\ldots)$. Это обстоятельство позволяет доказать равномерную ограниченность сумм Фурье -- Чебышева $S_n^{-1/2,-1/2}(f)$, $n=0,1,\ldots$  в пространстве $L_\mu^{p(x)}([-1,1])$ с $\mu(x)=(1-x^2)^{-\frac12}$ когда переменный показатель $p=p(x)>1$ подчиняется на $[-1,1]$ лишь условию \eqref{3.1}.  Причем это условие является, в определенном смысле, также и необходимым. Чтобы сформулировать соответствующий окончательный результат введем  одно обозначение. Обозначим через $\mathcal{ P}^\alpha$ класс переменных показателей $p>1$, удовлетворяющих на $[-1,1]$ следующему условию:
\begin{equation}\label{8.97}
|p(x)-p(y)|\left(\ln\frac{2}{|x-y|}\right)^\alpha\le d\quad(\alpha,d>0, x,y\in[-1,1]).
\end{equation}

\begin{theorem}
Пусть  $\mu=\mu(x)=(1-x^2)^{-\frac12}$. Тогда суммы Фурье-Чебышева $S_n^{-\frac12,-\frac12}(f)$ $(n=0,1,\ldots)$
равномерно ограничены в весовом пространстве Лебега $L_\mu^{p(x)}([-1,1])$ с произвольным переменным показателем $p\in \mathcal{ P}^\alpha$ тогда и только тогда, когда $\alpha\ge1$. Другими словами, найдется такое положительное число $c(\alpha,p)$, зависящее только от указанных параметров $\alpha$ и $p\in \mathcal{ P}^\alpha$,     что для произвольной функции $f\in L_\mu^{p(x)}([-1,1])$ имеет место оценка
\begin{equation}\label{8.98}
\|S_n^{-\frac12,-\frac12}(f)\|_{p(\cdot),\mu}([-1,1])\le c(\alpha,p)\|f\|_{p(\cdot),\mu}([-1,1]).
\end{equation}
Если же $0<\alpha<1$, то найдется переменный показатель $p_\alpha\in \mathcal{P}^\alpha$ и $f_\alpha\in L_\mu^{p_\alpha(x)}([-1,1])$,
для которых
\begin{equation}\label{8.99}
\|S_n^{-\frac12,-\frac12}(f_\alpha)\|_{p_\alpha(\cdot),\mu}([-1,1])\to\infty\quad(n\to\infty).
\end{equation}
\end{theorem}









\begin{corollary}
  Если выполнены условия теоремы 1.2,  то система полиномов Чебышева первого рода $\{p_n^{-\frac12,-\frac12}(x)\}_{n=0}^\infty$  является базисом пространства  $L^{p(x)}_\mu([-1,1])$, где $\mu=\mu(x)=(1-x^2)^{-\frac12}$.
\end{corollary}
