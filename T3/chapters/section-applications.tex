\section{Пакеты программ}
\subsection{Обработка и сжатие временных рядов}

%Реальные изображения, используемые в промышленных системах технического зрения, дефектоскопии, мониторинга процессов и т.д. сложны для автоматического анализа и обладают множеством свойств, определяемых не только условиями формирования изображения, но и методами его последующей обработки, а также целями использования извлекаемой из него информации.


В современном мире непрерывно возрастающая роль информационных технологий во многом определяет актуальность поиска новых методов, повышающих эффективность процессов обработки и передачи информации. Происходящие в настоящее время в обществе процессы информатизации характеризуются широким применением вычислительных устройств (в том числе мобильных) в различных сферах. При этом в каждой области применения ВУ предъявляются свои специфические требования к составу, структуре и организации обработки данных. %Особенно ярко эти требования выражены в области цифровой обработки сигналов (ЦОС), методы и средства которой, в настоящее время находят широкое применение.


Представление функций в виде рядов по тем или иным ортонормированным системам с целью последующего их приближения частичными суммами выбранного ортогонального ряда является, пожалуй, одним из самых часто применяемых подходов в теории приближений и ее приложениях. Наряду с задачами математической физики, для решения которых указанный подход является традиционным, появились и продолжают появляться все новые важные задачи, для решения которых также все чаще применяются методы, основанные на представлении функций(сигналов) в виде рядов по подходящим ортонормированным системам. При этом часто возникает такая ситуация, когда функция (сигнал, временной ряд, изображение и т.д) задана на достаточно длинном промежутке и нам требуется разбить этот промежуток на части, рассмотреть отдельные фрагменты функции определенные на этих частичных отрезках, представить их в виде рядов по выбранной ортонормированной системе и аппроксимировать каждый такой фрагмент частичными суммами соответствующего ряда. Такая ситуация является типичной для задач, связанных с решением нелинейных дифференциальных уравнений численно-аналитическими методами, обработкой временных рядов, изображений, аудио и видео файлов и других, в которых возникает необходимость разбить заданный ряд данных на части, аппроксимировать каждую часть и заменить приближенно исходный временный ряд (изображение) функцией, полученной в результате <<пристыковки>> функций, аппроксимирующих отдельные части. Но тогда в местах <<стыка>> возникают нежелательные разрывы (артефакты), которые искажают общий вид временного ряда (изображения).

Рассмотрена задача о конструировании аппроксимирующих операторов специального вида, обладающих тем важным свойством, что в окрестностях граничных точек отрезка аппроксимации приближает исходную функцию значительно лучше, чем на внутри отрезка. При этом на всем отрезке приближение предложенными специальными операторами осуществляется не хуже, чем приближение дискретным косинус-преобразованием Фурье.

Полученные в указанных пунктах фундаментальные результаты нашли практическую реализацию в разработанном пакете программ для обработки и сжатия временных рядов, звука и изображений на основе предельных и специальных дискретных рядов по полиномам, ортогональным на равномерных и неравномерных сетках.
В качестве исходных данных программа принимает массив одномерных или двумерных данных. Пользователю предоставляется возможность выбора оператора для осуществления обработки, задания параметров, а также уровня сжатия. Программа возвращает результат преобразования исходных данных, а также погрешность приближения. Кроме того, есть возможность визуализации при помощи построения графиков исходных ряда данных и аппроксимации, отрисовки исходного и восстановленного изображений, воспроизведения исходного и восстановленного звукового сигнала.



\subsection{Численное дифференцирование}\label{sect-numdiff}

Если для интегрируемой на отрезке $[a,b]$ функции $f(x)$ задана ее первообразная $F(x)$, то определенный интеграл от этой функции может быть вычислен по формуле Ньютона-Лейбница
$$
\int\limits_{a}^{b} f(x)\,dx= \Bigl.{F(x)}\Bigr|_{a}^{b}= F(b)-F(a)
$$,
где $F'(x)=f(x)$. Однако во многих случаях возникают большие трудности, связанные с нахождением первообразной, или эта задача не может быть решена элементарными способами. Например, в элементарных функциях не выражается интеграл $\textstyle{\int\limits_{1}^{2} \dfrac{dx}{\ln x}}$.
Кроме того, в вычислительной практике часто требуется находить значения производных и определенных интегралов от сеточных функций, заданных в общем случае на равномерной (или в более общем случае неравномерной) сетке
$\Omega_N= \bigl\{x_0,x_1,\ldots,x_N\bigr\},\quad x_{i+1}= x_i+h_{i+1},\quad i=\overline{0,N-1},\quad h_{i+1}= x_{i+1}-x_i$.
В связи с этим в численном анализе имеется специальный математический аппарат численного дифференцирования, отличный от соответствующего аппарата математического анализа.

Рассмотрена задача о конструировании аппроксимирующих операторов специального вида, которые характеризуются тем, что могут быть эффективно использованы для решения задачи одновременного приближения дифференцируемой функции и нескольких ее производных.

Одним из прикладных аспектов указанных исследований стала разработка пакета программ для осуществления численного дифференцирования на основе специальных дискретных рядов по полиномам Чебышева, ортогональным на равномерных сетках. Приложение принимает в качестве входных данных временной ряд, либо аналитически заданную функцию, после чего осуществляется построение специального дискретного ряда с выбранными пользователем параметрами. Пользователь имеет возможность как просмотреть числовые значения, так и оценить результат аппроксимации наглядно на графиках.


\subsection{Идентификация параметров линейных систем}

%С помощью специальных операторов исследована задача об идентификации параметров линейной неинвариантной (по времени) системы в случае, когда входной и выходной сигналы заданы в узлах равномерной сетки. Построены алгоритмы численной реализации этих операторов с помощью быстрых дискретных преобразований, основанных на полиномах Чебышева I-го и II-го рода. Разработан пакет программ для численного решения задачи идентификации параметров линейных систем на основе полиномов Чебышева первого и второго рода, а также полиномов Чебышева, ортогональных  на равномерных сетках.


Исследуется задача об идентификации параметров линейной неинвариантной (по времени) системы вида
\begin{equation*}
x^{(r)}(t)=\sum_{\nu=0}^{r-1}a_\nu(t)x^{(\nu)}(t)+\sum_{\mu=0}^s b_\mu(t)y^{(\mu)}(t),
\end{equation*}
где неизвестные переменные коэффициенты $a_\nu(t)$ $(\nu=0,\ldots,r-1)$ и $b_\mu(t)$ $(\mu=0,\ldots,s)$ представляют собой алгебраические полиномы  заданной степени $m$. Ставится задача  найти неизвестные переменные коэффициенты $a_\nu(t)$ $(\nu=0,\ldots,r-1)$ и $b_\mu(t)$ $(\mu=0,\ldots,s)$ экспериментальным путем.
Был рассмотрен часто встречающийся на практике случай, когда заданы значения сигналов $x(t)$ и  $y(t)$ в узлах равномерной сетки $\Omega_N=\{t_j=-1+jh\}_{j=0}^{N-1}$, где $h=\frac2{N-1}$.
Для решения этой задачи на основе полиномов Чебышева, ортогональных на равномерной сетке, сконструированы полиномиальные операторы, которые могут быть эффективно использованы как для <<сглаживания>> ошибок в наблюдениях исходного сигнала $f(t)$ в узлах сетки $\Omega_N$, так и для решения задачи одновременного приближения дифференцируемой функции $f(t)$ и нескольких ее производных. При численной реализации этих операторов используются быстрые дискретные преобразования, основанные на полиномах Чебышева I рода $C_k(t) = \cos (k \arccos t)$.


С использованием модулей программного комплекса, описанного в \S\ref{sect-numdiff}, был разработан пакет программ для численного решения задачи идентификации параметров линейных систем на основе полиномов Чебышева первого и второго рода, а также полиномов Чебышева, ортогональных  на равномерных сетках. Приложение принимает в качестве входного и выходного сигналов временные ряды, либо аналитически заданные функции, после чего осуществляется построение специальных дискретного ряда с выбранными пользователем параметрами. Далее осуществляется решение задачи по описанному выше алгоритму. Пользователь имеет возможность как просмотреть числовые значения, так и оценить результат аппроксимации наглядно на графиках.



\subsection{Анализ передаточной функции}

Разработан пакет компьютерных программ для исследования сложных динамических систем путем анализа пространственно-временных изменений передаточных функций между параметрами взаимосвязанных процессов, заданными в виде временных рядов.

В программе используется метод относительных амплитуд отклика (ОАО, см. \cite{taimazov}), численная реализация которого включает в себя следующие шаги:
%ERROR
\begin{enumerate}[1) ]
\item
удаление трендовой составляющей;
\item
гармонический анализ введенных временных рядов с использованием оконного преобразования Фурье;
\item
определение с заданной дискретизацией относительных амплитуд когерентных гармоник (отношение амплитуд отклика и динамического воздействия) и составление временного ряда ОАО;
\item
гармонический и корреляционный (с другими геофизическими времен-ными рядами) анализ полученного ряда ОАО, выявление в нем динамических характеристик исследуемой системы.
\end{enumerate}

Программный комплекс может быть использован, например, для совместной обработки гидрогеодинамических и барометрических наблюдений.
С помощью этих программ произведена обработка данных атмосферного давления и уровня подземных вод на станциях Айды, Каспийск и Серебряковка. Предполагается, что полученные результаты могут быть использованы для анализа изменения напряженно-деформированного состояния земной коры.

