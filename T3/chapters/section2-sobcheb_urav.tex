\chapter{Полиномы, ортогональные по Соболеву, ассоциированные с полиномами Чебышева первого рода и задача Коши для ОДУ }

%Рассмотрены полиномы $T_{r,n}(x)$ $(n=0,1,\ldots)$, порожденные многочленами Чебышева $T_{n}(x)=\cos( n\arccos x)$, образующие ортонормированную систему по Соболеву относительно скалярного произведения
%следующего вида $<f,g>=\sum_{\nu=0}^{r-1}f^{(\nu)}(-1)g^{(\nu)}(-1)+\int_{-1}^{1}f^{(r)}(t)g^{(r)}(x)\mu(x)dx$,
%где $\mu(x)=\frac2\pi(1-x^2)^{-\frac12}$. Для $T_{r,n}(x)$ $(n=0,1,\ldots)$  установлена связь с многочленами Чебышева $T_{n}(x)$ и получены явные представления, которые могут быть использованы при вычислении значений полиномов $T_{r,n}(x)$ и исследовании их асимптотических свойств. Показано, что суммы Фурье по полиномам $T_{r,n}(x)$ $(n=0,1,\ldots)$ является удобным и весьма эффективным инструментом приближенного решения задачи Коши для систем обыкновенных дифференциальных уравнений (ОДУ).



%%%%%%%%%%%%%%%%%%%%%%%%%%%%%
%%%%%%%%%%%%%%%%%%%%%%%%%%%%%
%%%%%%%%%%%%%%%%%%%%%%%%%%%%%





\section{Введение}
Интерес к теории полиномов, ортогональных относительно скалярных произведений, в которых присутствуют одна или несколько точек  с дискретными массами,  в последнее время интенсивно растет  (см. \cite{sobcheb_urav-KwonLittl1,sobcheb_urav- KwonLittl2,sobcheb_urav-MarcelAlfaroRezola,sobcheb_urav-IserKoch,sobcheb_urav-Meijer,sobcheb_urav-Lopez1995,
sobcheb_urav-MarcelXu,sobcheb_urav-Shar2016}  и цитированную там литературу).
 Это новое направление принято обозначать словами: <<полиномы, ортогональные по Соболеву>>. Не ослабивающее внимание специалистов  к этому направлению теории ортогональных полиномов можно объяснить в том числе и тем обстоятельством, что ряды Фурье по полиномам, ортогональным по Соболеву, оказались естественным и весьма удобным инструментом для представления решений  дифференциальных (разностных) уравнений. Это можно показать, в частности, на примере  задачи Коши для  дифференциального уравнения
\begin{equation}\label{sobcheb-urav-1.1}
F(x,y,y',\ldots,y^{(r)})=0
 \end{equation}
с начальными условиями $y^{(k)}(-1)=y_k$, $k=0,1,\ldots,r-1$.  Наряду с различными сеточными методами, для решения этой задачи часто применяют так называемые спектральные методы
\cite{sobcheb_urav-Tref1,sobcheb_urav-Tref2,sobcheb_urav-SolDmEg,sobcheb_urav-Pash,sobcheb_urav-Arush2010,
sobcheb_urav-Arush2013,sobcheb_urav-Arush2014,sobcheb_urav-Lukom2016}. Напомним, что суть спектрального метода решения задачи Коши  для ОДУ \eqref{sobcheb-urav-1.1} заключается в том, что в первую очередь искомое решение $y(x)$ представляется в виде ряда Фурье
\begin{equation}\label{sobcheb-urav-1.2}
 y(x)=\sum_{k=0}^\infty \hat y_k\psi_k(x)
 \end{equation}
по подходящей ортонормированной системе $\{\psi_k(x)\}_{k=0}^\infty$ (чаще всего в качестве $\{\psi_k(x)\}_{k=0}^\infty$ используют    тригонометрическую систему, ортогональные полиномы, вэйвлеты, корневые функции того или иного дифференциального оператора  и некоторые другие). На втором этапе осуществляется подстановка вместо $y(x)$ ряда \eqref{sobcheb-urav-1.2} в уравнение \eqref{sobcheb-urav-1.1}. Это приводит к системе уравнений относительно неизвестных коэффициентов $\hat y_k$ ($k=0,1,\ldots$). На третьем этапе требуется решить эту систему с учетом начальных условий  $y^{(k)}(-1)=y_k$, $k=0,1,\ldots,r-1$ исходной задачи Коши.
Одна из основных трудностей, которая возникает на этом этапе, состоит в том, чтобы
выбрать такой ортонормированный базис $\{\psi_k(x)\}_{k=0}^\infty$, для которого искомое решение $y(x)$ уравнения \eqref{sobcheb-urav-1.1}, представленное в виде ряда  \eqref{sobcheb-urav-1.2}, удовлетворяло бы начальным условиям $y^{(k)}(-1)=y_k$, $k=0,1,\ldots,r-1$. Более того, поскольку в результате решения системы уранений относительно неизвестных коэффициентов $\hat y_k$  будет найдено только конечное их число с $k=0,1,\ldots, n$, то весьма важно, чтобы частичная сумма ряда \eqref{sobcheb-urav-1.2} вида $ y_n(x)=\sum_{k=0}^n\hat y_k\psi_k(x)$,
 будучи приближенным решением рассматриваемой задачи Коши, также удовлетворяла начальным условиям $y_n^{(k)}(-1)=y_k$, $k=0,1,\ldots,r-1$. Покажем, что  базис $\{\psi_k(x)=T_{r,k}(x)\}_{k=0}^\infty$, состоящий из полиномов
$T_{r,k}(x)$, ортонормированных по Соболеву относительно скалярного произведения
\begin{equation}\label{sobcheb-urav-1.3}
<f,g>=\sum_{\nu=0}^{r-1}f^{(\nu)}(-1)g^{(\nu)}(-1)+\int_{-1}^{1}f^{(r)}(t)g^{(r)}(x)\mu(x)dx,
\end{equation}
где $\mu(x)=\frac2\pi(1-x^2)^{-\frac12}$ и порожденных многочленами Чебышева  $T_{k}(x)$ посредством равенств
   \begin{equation}\label{sobcheb-urav-1.4}
T_{r,k}(x) =\frac{(x+1)^k}{k!}, \quad k=0,1,\ldots, r-1,
\end{equation}
  \begin{equation}\label{sobcheb-urav-1.5}
 T_{r,r}(x) =\frac{(x+1)^r}{\sqrt{2}r!},\quad T_{r,r+n}(x) =\frac{1}{(r-1)!}\int\limits_{-1}^x(x-t)^{r-1}T_{n}dt, \quad n=1,\ldots.
\end{equation}
   обладает указанными свойствами. С этой целью заметим, что ряд Фурье \eqref{sobcheb-urav-1.2}, в случае, когда $\{\psi_k(x)=T_{r,k}(x)\}_{k=0}^\infty$, приобретает, как показано ниже, следующий вид
   \begin{equation}\label{sobcheb-urav-1.6}
y(x)= \sum_{k=0}^{r-1} y^{(k)}(-1)\frac{(x+1)^k}{k!}+ \sum_{k=r}^\infty \hat y_{r,k}T_{r,k}(x),
\end{equation}
где
  \begin{equation}\label{sobcheb-urav-1.7}
 \hat y_{r,k}=\int_{-1}^1 y^{(r)}(t)T_{k-r}(t)\mu(t)dt.
\end{equation}
С другой стороны, из определения \eqref{sobcheb-urav-1.5} вытекает, что $(T_{r,k}(-1))^{(\nu)}=0$ для всех $0\le\nu\le r-1$, поэтому функция $y(x)$, представленная в виде ряда \eqref{sobcheb-urav-1.6}, так и частичная  сумма этого ряда вида
 \begin{equation}\label{sobcheb-urav-1.8}
y_n(x)= \sum_{k=0}^{r-1} y^{(k)}(-1)\frac{(x+1)^k}{k!}+ \sum_{k=r}^n \hat y_{r,k}T_{r,k}(x)
\end{equation}
удовлетворяют начальным условиям задачи Коши для уравнения \eqref{sobcheb-urav-1.1}.

Таким образом, полиномы, ортогональные по Соболеву относительно скалярного произведения \eqref{sobcheb-urav-1.3}, тесно связаны с задачей Коши для уравнения \eqref{sobcheb-urav-1.1}.

         Следует отметить, что смешанные ряды вида \eqref{sobcheb-urav-1.6}, ассоциированные с классическими ортогональными полиномами, являлись основным объектом исследования работ  \cite{sobcheb_urav-Shar11} -- \cite{sobcheb_urav-Shar18}.
Полиномы $T_{r,k}(x)$ (и их обобщения), определямые равенствами типа \eqref{sobcheb-urav-1.4} и \eqref{sobcheb-urav-1.5}, в этих работах существенно использовались в качестве вспомогательного аппарата при исследовании вопросов сходимости смешанных рядов вида \eqref{sobcheb-urav-1.6},  не отмечая при  этом свойство их ортогональности по Соболеву относительно скалярных произведений типа \eqref{sobcheb-urav-1.3}. В настоящей работе, напротив, смешанные ряды будут исталковываться как ряды Фурье по полиномам, ортогональным по Соболеву. С точки зрения геометрии гильбертовых пространств общая идея, которая лежит в основе построения смешанных рядов, заключается в следующем. Предположим, что система функций  $\left\{\varphi_k(x)\right\}$ ортонормирована  на $(a,b)$  c весом   $\rho(x)$, т.е.
 \begin{equation}\label{sobcheb-urav-1.9}
\int\limits_a^b\varphi_k(x)\varphi_l(x)\rho(x)dx=\delta_{kl},
\end{equation}
где $\delta_{kl}$ -- символ Кронекера. Через $L^p_\rho(a,b)$ обозначим пространство  функций $f(x)$, измеримых  на  $(a,b)$, для которых
 \begin{equation*}
\int\limits_a^b|f(x)|^p\rho(x)dx<\infty.
\end{equation*}
Если $\rho(x)\equiv1$, то будем писать $L^p_\rho(a,b)=L^p(a,b)$ и $L(a,b)=L^1(a,b)$.
Из \eqref{sobcheb-urav-1.9} следует, что $\varphi_k(x)\in L^2_\rho(a,b)$ $(k=0,1,\ldots)$. Мы добавим к этому условию еще одно, считая, что $\varphi_k(x)\in L(a,b)$ $(k=0,1,\ldots)$. Тогда мы можем определить следующие порожденные системой $\{\varphi_k(x)\}$ функции
 \begin{equation}\label{sobcheb-urav-1.10}
\varphi_{r,r+k}(x) =\frac{1}{(r-1)!}\int\limits_a^x(x-t)^{r-1}\varphi_{k}(t)dt, \quad k=0,1,\ldots.
\end{equation}
 Кроме того, определим конечный набор функций
  \begin{equation}\label{sobcheb-urav-1.11}
\varphi_{r,k}(x) =\frac{(x-a)^k}{k!}, \quad k=0,1,\ldots, r-1.
\end{equation}

 Из \eqref{sobcheb-urav-1.10} и \eqref{sobcheb-urav-1.11} следует, что для п.в. $x\in (a,b)$
 \begin{equation}\label{sobcheb-urav-1.12}
(\varphi_{r,k}(x))^{(\nu)} =\begin{cases}\varphi_{r-\nu,k-\nu}(x),&\text{если $0\le\nu\le r-1$, $r\le k$,}\\
\varphi_{k-r}(x),&\text{если  $\nu=r\le k$,}\\
\varphi_{r-\nu,k-\nu}(x),&\text{если $\nu\le k< r$,}\\
0,&\text{если $k< \nu\le r-1$}.
  \end{cases}
\end{equation}
Через $W^r_{L^p_\rho(a,b)}$ обозначим пространство Соболева $W^r_{L^p_\rho(a,b)}$, состоящее из функций $f(x)$, непрерывно дифференцируемых на $[a,b]$ $r-1$ раз, причем $f^{(r-1)}(x)$ абсолютно непрерывна на $[a,b]$  и $f^{(r)}(x)\in L^p_\rho(a,b)$.
Скалярное произведение в пространстве $W^r_{L^2_\rho(a,b)}$ определим с помощью равенства
\begin{equation}\label{sobcheb-urav-1.13}
<f,g>=\sum_{\nu=0}^{r-1}f^{(\nu)}(a)g^{(\nu)}(a)+\int_{a}^{b} f^{(r)}(t)g^{(r)}(t)\rho(t) dt.
\end{equation}
Тогда, пользуясь определением функций  $\varphi_{r,k}(x)$ (см. \eqref{sobcheb-urav-1.10} и \eqref{sobcheb-urav-1.11}) и равенством  \eqref{sobcheb-urav-1.12} нетрудно увидеть(см. \ref{sobcheb-uravtheo1}),  что система $\{\varphi_{r,k}(x)\}_{k=0}^\infty$ является ортонормированной в пространстве $W^r_{L^2_\rho(a,b)}$.  Мы будем называть систему $\{\varphi_{r,k}(x)\}_{k=0}^\infty$ \textit{ортонормированной по Соболеву } относительно скалярного произведения \eqref{sobcheb-urav-1.13} и  \textit{порожденной} ортонормированной системой $\{\varphi_{k}(x)\}_{k=0}^\infty$.
В дальнейшем будет показано (см.\S2),  что ряд Фурье функции $f(x)\in W^r_{L^2_\rho(a,b)}$ по системе  $\{\varphi_{r,k}(x)\}_{k=0}^\infty$ имеет смешанный характер, а, более точно, имеет следующий вид
  \begin{equation}\label{sobcheb-urav-1.14}
f(x)\sim \sum_{k=0}^{r-1} f^{(k)}(a)\frac{(x-a)^k}{k!}+ \sum_{k=r}^\infty \hat f_{r,k}\varphi_{r,k}(x),
\end{equation}
где
  \begin{equation}\label{sobcheb-urav-1.15}
 \hat f_{r,k}=\int\limits_a^b f^{(r)}(t) \varphi^{(r)}_{r,k}(t)\rho(t)dt=\int\limits_a^b f^{(r)}(t) \varphi_{k-r}(t)\rho(t)dt,
\end{equation}
поэтому ряд  вида \eqref{sobcheb-urav-1.14} будем (следуя
\cite{sobcheb_urav-Shar11,sobcheb_urav-Shar12,sobcheb_urav-Shar13,sobcheb_urav-Shar14,sobcheb_urav-Shar15,
sobcheb_urav-Shar16})  называть \textit{смешанным рядом} по  системе $\{\varphi_{k}(x)\}_{k=0}^\infty$, считая это название условным и сокращенным обозначением полного названия: <<\textit{ряд Фурье по системе  $\{\varphi_{r,k}(x)\}_{k=0}^\infty$, ортонормированной по Соболеву, порожденной ортонормированной системой $\{\varphi_{k}(x)\}_{k=0}^\infty$}>>.



В цитированных выше   работах  \cite{sobcheb_urav-Shar11,sobcheb_urav-Shar12,sobcheb_urav-Shar13,sobcheb_urav-Shar14,sobcheb_urav-Shar15,
sobcheb_urav-Shar16,sobcheb_urav-Shar17,sobcheb_urav-Shar18} (см. также  \cite{sobcheb_urav-Shar19})   были рассмотрены некоторые частные случаи ортонормированных систем функций вида $\{\varphi_{r,k}(x)\}_{k=0}^\infty$, порожденных классическими ортонормированными системами Якоби, Лежандра, Чебышева, Лагерра и Хаара. С другой стороны, как уже отмечалось, в последние годы интенсивное развитие получила  теория полиномов, ортогональных относительно различных скалярных произведений соболевского типа (полиномы, ортогональные по Соболеву). Скалярные произведения соболевского типа характеризуются тем, что они включают в себя слагаемые, которые <<контролируют>>, поведение соответствующих ортогональных полиномов  в нескольких заданных точках числовой оси. Например, в некоторых случаях оказывается так, что полиномы, ортогональные по Соболеву на интервале $(a,b)$, могут иметь нули, совпадающие с одним или с обоими концами этого интервала. Это обстоятельство имеет важное значение для некоторых приложений, в которых требуется, чтобы значения  частичных сумм ряда Фурье функции $f(x)$ по рассматриваемой системе ортогональных полиномов совпали в концах интервала $(a,b)$ со значениями $f(a)$ и $f(b)$.  Заметим, что обычные ортогональные с положительным на  $(a,b)$ весом полиномы этим важным свойством не обладают. Скалярное произведение \eqref{sobcheb-urav-1.13}, рассматриваемое в настоящей работе, имеет одну особую точку, а именно, точку $a$, в окрестности которой <<контролируется>>, поведение функций $\varphi_{r,k}(x)$, ортогональных по Соболеву и порожденных исходной ортонормированной системой $\{\varphi_{k}(x)\}_{k=0}^\infty$ посредством равенства \eqref{sobcheb-urav-1.10}.


Отметим некоторые важные свойства смешанного ряда \eqref{sobcheb-urav-1.14}, непосредственно вытекающие из \eqref{sobcheb-urav-1.12}. Первое из них связано с дифференциальным свойством смешанного ряда, а именно, если $r>1$, то в результате почленного дифференцирования смешанного ряда \eqref{sobcheb-urav-1.14} мы получим смешанный ряд для производной $f'(x)$, соответствующий случаю, когда вместо $r$ фигурирует $r-1$, другими словами
\begin{equation}\label{sobcheb-urav-1.16}
f'(x)\sim  \sum_{k=1}^\infty f'_{r-1,k-1}\varphi_{r-1,k-1}(x)=\sum_{k=1}^\infty (\hat f_{r,k}\varphi_{r,k}(x))'.
\end{equation}
Второе свойство связано c почленным интегрированием с переменным верхним пределом и имеет вид
\begin{equation}\label{sobcheb-urav-1.17}
\int\limits_a^xf'(t)dt\sim \sum_{k=1}^\infty f'_{r-1,k-1}\int\limits_a^x\varphi_{r-1,k-1}(t)dt=\sum_{k=1}^\infty \hat f_{r,k}\varphi_{r,k}(x).
\end{equation}
Важное значение имеет свойство  смешанного ряда \eqref{sobcheb-urav-1.14}, котрое заключается в том, что его частичная сумма вида
\begin{equation}\label{sobcheb-urav-1.18}
Y_{r,N}(f,x)=\sum_{k=0}^{r-1} f^{(k)}(a)\frac{(x-a)^k}{k!}+ \sum_{k=r}^{N} \hat f_{r,k}\varphi_{r,k}(x)
\end{equation}
 при   $r\le N$  совпадает с исходной функцией $f(x)$   в точке $x=a$ $r$-кратно , т.е.
\begin{equation}\label{sobcheb-urav-1.19}
(Y_{r,N}(f,x))^{(\nu)}_{x=a}=f^{(\nu)}(a)\quad (0\le\nu\le r-1).
\end{equation}
Кроме того, из \eqref{sobcheb-urav-1.12} и \eqref{sobcheb-urav-1.18} следует, что $(0\le\nu\le r-1)$
\begin{equation}\label{sobcheb-urav-1.20}
 Y_{r,N}^{(\nu)}(f,x)=\sum_{n=0}^{r-1-\nu} f^{(n+\nu)}(a)\frac{(x-a)^n}{n!}+ \sum_{n=r-\nu}^{N-\nu} f_{r-\nu,n}^{(\nu)}\varphi_{r-\nu,n}(x)=Y_{r-\nu,N-\nu}(f^{(\nu)},x),
 \end{equation}
отсюда, в свою очередь, выводим $(0\le\nu\le r-2)$
 $$
f^{(\nu)}(x)-Y_{r,N}^{(\nu)}(f,x)= \frac{1}{(r-\nu-2)!}\int_a^x (x-t)^{r-\nu-2}(f^{(r-1)}(t)-Y_{r,N}^{(r-1)}(f,t))dt=
$$
  \begin{equation}\label{sobcheb-urav-1.21}
\frac{1}{(r-\nu-2)!}\int_a^x (x-t)^{r-\nu-2}(f^{(r-1)}(t)-Y_{1,N-r+1}(f^{(r-1)},t))dt.
 \end{equation}


  В \cite{sobcheb_urav-Shar11,sobcheb_urav-Shar12,sobcheb_urav-Shar13,sobcheb_urav-Shar14,sobcheb_urav-Shar15,
sobcheb_urav-Shar16,sobcheb_urav-Shar17,sobcheb_urav-Shar18}, воспользовавшись равенствами типа \eqref{sobcheb-urav-1.20} и \eqref{sobcheb-urav-1.21}, было показано, что частичные суммы смешанных рядов по классическим ортогональным полиномами, в отличие от сумм Фурье по этим же полиномам, успешно могут быть использованы в задачах, в которых требуется одновременно приближать дифференцируемую функцию и ее несколько производных. Как было показано выше, такие задачи непосредственно возникают, например, в связи с решением краевых задач для дифференциальных уравнений спектральными методами.  Основное внимание  в \cite{sobcheb_urav-Shar11,sobcheb_urav-Shar12,sobcheb_urav-Shar13,sobcheb_urav-Shar14,sobcheb_urav-Shar15,
sobcheb_urav-Shar16}  уделялось исследованию аппроксимативных свойств смешанных рядов по ультрасферическим полиномам Якоби  $P_n^{\alpha,\alpha}(x)$, тогда как в работе \cite{sobcheb_urav-Shar18} были найдены условия на параметры $\alpha$ и $\beta$, которые обеспечивают равномерную сходимость смешанных рядов по общим полиномам Якоби $P_n^{\alpha,\beta}(x)$ с $\alpha,\beta>-1$.

   В связи с отмеченными выше и рядом других  задач, в которых полиномы, ортогональные по Соболеву, предстают перед нами как естественный и эффективный инструмент для их решения, возникают важные задачи об изучении различных  свойств самых этих полиномов. Наиболее трудными из них, как это и бывает в теории ортогональных полиномов, являются задачи, связанные с их асимптотическим поведением.   В связи с этой проблемой  отметим  работу
\cite{sobcheb_urav-Lopez1995}, в которой, используя  идеи и технику А.\,А. Гончара \cite{sobcheb_urav-Gonchar1975}, исследована задача о сравнительной асимптотике полиномов, ортогональных относительно скалярного произведения типа  Соболева с дискретными массами.

Одной из целей настоящей статьи является получение различных выражений для полиномов $T_{r,k}(x)$, ортогональных по Соболеву относительно скалярного произведения \eqref{sobcheb-urav-1.3} и порожденных полиномами  $T_{k}(x)$ посредством  равенства \eqref{sobcheb-urav-1.5}. Эти результаты могут быть использованы при вычислении значений полиномов  $T_{r,k}(x)$ и изучении асимптотических свойств полиномов $T_{r,k}(x)$ при $x\in[-1,1]$ и $n\to\infty$. А это, в свою очередь, позволит исследовать аппроксимативные свойства частичных сумм \eqref{sobcheb-urav-1.8} ряда Фурье \eqref{sobcheb-urav-1.6} по полиномам $T_{r,k}(x)$ для гладких и аналитических функций $y=y(x) $(решений дифференциального уравнения \eqref{sobcheb-urav-1.1}).

Прежде, чем  перейти к исследованию свойств полиномов $T_{r,k}(x)$, мы рассмотрим некоторые общие свойства  функций $\varphi_{r,k}(x)$, $(k=0,1,\ldots)$, определенных равенствами \eqref{sobcheb-urav-1.10} и \eqref{sobcheb-urav-1.11},  ортонормированных по Соболеву относительно скалярного произведения \eqref{sobcheb-urav-1.13}.

%\section{Некоторые результаты общего характера }
%Рассмотрим сначала задачу о полноте в $W^r_{L^2_\rho(a,b)}$ системы $\{\varphi_{r,k}(x)\}_{k=0}^\infty$, состоящей из функций, определенных равенствами   \eqref{sobcheb-urav-1.10} и \eqref{sobcheb-urav-1.11}.
%\begin{theorem}\label{sobcheb-uravtheo1}
% Предположим, что    функции $\varphi_k(x)$ $(k=0,1,\ldots)$ образуют полную в $L^2_\rho(a,b)$ ортонормированную   c весом   $\rho(x)$ систему на  $(a,b)$. Тогда система $\{\varphi_{r,k}(x)\}_{k=0}^\infty$, порожденная системой $\{\varphi_{k}(x)\}_{k=0}^\infty$ посредством равенств \eqref{sobcheb-urav-1.10} и \eqref{sobcheb-urav-1.11}, полна  в $W^r_{L^2_\rho(a,b)}$ и ортонормирована относительно скалярного произведения \eqref{sobcheb-urav-1.13}.
%\end{theorem}
%%\begin{proof}
%%Из равенства \eqref{sobcheb-urav-1.12} следует, что если $r\le k$ и $0\le\nu\le r-1$, то  $(\varphi_{r,k}(x))^{(\nu)}_{x=a}=0$, поэтому
%%в силу \eqref{sobcheb-urav-1.9}, \eqref{sobcheb-urav-1.12},  имеем
%%$$
%%<\varphi_{r,k},\varphi_{r,l}>= \int_{a}^b(\varphi_{r,k}(x))^{(r)}(\varphi_{r,l}(x))^{(r)}\rho(x) dx=
%%$$
%%\begin{equation}\label{2.1}
%%    \int_{a}^b\varphi_{k-r}(x)\varphi_{l-r}(x)\rho(x) dx=\delta_{kl},\quad k,l\ge r ,
%%  \end{equation}
%%  а из \eqref{sobcheb-urav-1.11} и \eqref{sobcheb-urav-1.12} имеем
%%\begin{equation}\label{2.2}
%%  <\varphi_{r,k},\varphi_{r,l}>=\sum_{\nu=0}^{r-1}(\varphi_{r,k}(x))^{(\nu)}|_{x=a}(\varphi_{r,l}(x))^{(\nu)}|_{x=a}=\delta_{kl},\quad k,l< r.
%%  \end{equation}
%%  Очевидно также, что
%%  \begin{equation}\label{2.3}
%%  <\varphi_{r,k},\varphi_{r,l}>=0,\quad \text{если}\quad k< r\le l\quad \text{или} \quad l< r\le k.
%%  \end{equation}
%% Это означает, что функции  $\varphi_{r,k}(t)\, (k=0,1,\ldots) $ образуют   в $W^r_{L^2_\rho(a,b)}$ ортонормированную  систему относительно скалярного произведения \eqref{sobcheb-urav-1.13}.  Остается убедиться в ее полноте в $W^r_{L^2_\rho(a,b)}$. С этой целью покажем, что если для некоторой функции $f=f(x)\in W^r_{L^2_\rho(a,b)}$ и для  всех $k=0,1,\ldots$ справедливы равенства $<f,\varphi_k>=0$, то $f(x)\equiv0$. В самом деле, если $k\le r-1$, то  $<f,\varphi_{r,k}>=f^{(k)}(a)$, поэтому с учетом того, что $<f,\varphi_{r,k}>=0$,  для нашей функции  $f(x)$ формула Тейлора приобретает вид
%%\begin{equation}\label{2.4}
%%f(x)={1\over (r-1)!}\int\limits_{a}^x(x-t)^{r-1} f^{(r)}(t)dt.
%%     \end{equation}
%%С другой стороны, для всех $k\ge r$ имеем
%%$$
%% 0= <f,\varphi_{r,k}>=\int_{a}^bf^{(r)}(x) (\varphi_{r,k}(x))^{(r)}\rho(x) dx=
%%  \int_{a}^b f^{(r)}(x)\varphi_{k-r}(x) \rho(x) dx .
%%$$
%%Отсюда и из того, что $\varphi_m(x)$ ($m=0,1,\ldots$)  образуют в $L^2_{\rho}(a,b)$ полную ортонормированную систему имеем $f^{(r)}(x)=0$ почти всюду на $(a,b)$. Поэтому   $f(x)\equiv0$. Теорема 1 доказана.
%%\
%%\end{proof}
%
%\begin{theorem}\label{sobcheb-uravtheo2}
%Предположим, что  $ \frac{1}{\rho(x)}\in L(a,b) $, а  функции $\varphi_k(x)$ $(k=0,1,\ldots)$  образуют полную в $L^2_\rho(a,b)$ ортонормированную   c весом   $\rho(x)$ систему на $(a,b)$, $\{\varphi_{r,k}(x)\}_{k=0}^\infty$ -- система, ортонормированная в $W^r_{L^2_\rho(a,b)}$ относительно скалярного произведения \eqref{sobcheb-urav-1.13},  порожденная системой $\{\varphi_{k}(x)\}_{k=0}^\infty$ посредством равенств \eqref{sobcheb-urav-1.10} и \eqref{sobcheb-urav-1.11}.
%Тогда, если $f(x)\in W^r_{L^2_\rho(a,b)}$, то ряд Фурье (смешанный ряд) \eqref{sobcheb-urav-1.14} сходится к функции $f(x)$ равномерно относительно $x\in[a,b]$.
% \end{theorem}
%%\begin{proof}
%%Обозначим через $S_n(f^{(r)})=S_n(f^{(r)},x)$ частичную сумму ряда Фурье функции $f^{(r)}(x)\in L^2_\rho(a,b) $ по системе $\{\varphi_k(x)\}_{k=0}^n$, т.е.
%%\begin{equation}\label{2.5}
%%S_n(f^{(r)},x)=\sum_{k=0}^n f_{r,k+r}\varphi_k(x),
%%     \end{equation}
%%где коэффициенты $f_{r,k+r}$ $(k=0,1,\ldots)$ определены равенством \eqref{sobcheb-urav-1.15}.  Из условий теоремы 2 следует, что при $n\to\infty$
%% \begin{equation}\label{2.6}
%%\|f^{(r)}-S_n(f^{(r)})\|_{L^2_\rho(a,b)}\to0.
%% \end{equation}
%%Далее, обозначим через $ Y_{n+r}(f,x)$ частичную сумму смешанного ряда \eqref{sobcheb-urav-1.14} следующего вида
%% \begin{equation}\label{2.7}
%% Y_{n+r}(f,x)= \sum_{k=0}^{r-1} f^{(k)}(a)\frac{(x-a)^k}{k!}+ \sum_{k=r}^{n+r} \hat f_{r,k}\varphi_{r,k}(x),
%%\end{equation}
%%и запишем формулу Тейлора
%%\begin{equation}\label{2.8}
%% f(x)= \sum_{k=0}^{r-1} f^{(k)}(a)\frac{(x-a)^k}{k!}+ {1\over (r-1)!}\int\limits_{a}^x(x-t)^{r-1} f^{(r)}(t)dt.
%%\end{equation}
%%Из \eqref{2.7} и \eqref{2.8} имеем
%%\begin{equation}\label{2.9}
%% f(x)-Y_{n+r}(f,x)=  {1\over (r-1)!}\int\limits_{a}^x(x-t)^{r-1} f^{(r)}(t)dt-\sum_{k=r}^{n+r} \hat f_{r,k}\varphi_{r,k}(x).
%%\end{equation}
%%Обратимся к равенству \eqref{sobcheb-urav-1.10}, тогда \eqref{2.9} можно переписать так
%%$$
%%f(x)-Y_{n+r}(f,x)=
%%$$
%%$$
%%  {1\over (r-1)!}\int\limits_{a}^x(x-t)^{r-1} f^{(r)}(t)dt-{1\over (r-1)!}\int\limits_{a}^x(x-t)^{r-1}\sum_{k=r}^{n+r} \hat f_{r,k}\varphi_{k-r}(t)dt=
%%$$
%%\begin{equation}\label{2.10}
%% {1\over (r-1)!}\int\limits_{a}^x(x-t)^{r-1}[f^{(r)}(t)-S_n(f^{(r)},t)]dt .
%%\end{equation}
%%Из \eqref{2.10} и неравенства Гельдера имеем
%%$$
%%|f(x)-Y_{n+r}(f,x)|\le
%%$$
%%\begin{equation}\label{2.11}
%% {1\over (r-1)!} \left(\int\limits_{a}^b\frac{|x-t|^{2(r-1)}}{\rho(t)}dt\right)^\frac12\left(\int\limits_{a}^b [f^{(r)}(t)-S_n(f^{(r)},t)]^2\rho(t)dt\right)^\frac12.
%% \end{equation}
%%Сопоставляя \eqref{2.6} с \eqref{2.11}, убеждаемся в справедливости теоремы 2.
%%
%%\end{proof}
%
%%\section{Некоторые сведения о полиномах Якоби}
%%При изучении свойств полиномов, ортогональных по Соболеву, порожденных полиномами Чебышева первого рода $T_n(x)=\cos(n\arccos x)$, нам понадобится ряд свойств классических полиномов Якоби.  Для произвольных действительных $\alpha$ и $\beta$ полиномы Якоби  $P_n^{\alpha,\beta}(x)$ можно определить \cite{sobcheb_urav-Sege}  с помощью формулы Родрига
%% \begin{equation}\label{sobcheb-urav-3.1}
%%P_n^{\alpha,\beta}(x) = {(-1)^n\over2^nn!}{1\over\rho(x)}{d^n\over
%%dx^n} \left\{\rho(x)\sigma^n(x)\right\},
%%\end{equation}
%%где $\alpha,\beta$ -- произвольные действительные числа, $\rho(x)=
%%\rho(x;\alpha,\beta) =
%%(1-x)^\alpha(1+x)^\beta,\,\,\sigma(x)=1-x^2$. Если
%%$\alpha,\beta>-1$, то полиномы Якоби образуют ортогональную
%%систему с весом $\rho(x)$, т.е.
%%\begin{equation}\label{sobcheb-urav-3.2}
%%\int_{-1}^1P_n^{\alpha,\beta}(x)P_m^{\alpha,\beta}(x)\rho(x)dx =
%%h_n^{\alpha,\beta}\delta_{nm},
%%\end{equation}
%%где
%%\begin{equation}\label{sobcheb-urav-3.3}
%%h_n^{\alpha,\beta} =
%%{\Gamma(n+\alpha+1)\Gamma(n+\beta+1)2^{\alpha+\beta+1} \over
%%n!\Gamma(n+\alpha+\beta+1)(2n+\alpha+\beta+1)}.
%%\end{equation}
%%Нам понадобятся еще следующие свойства полиномов Якоби~\cite{sobcheb_urav-Sege}, \cite{sobcheb_urav-Gasper}:
%%
%%
%%\begin{equation}\label{sobcheb-urav-3.4}
%%{d\over dx}P_n^{\alpha,\beta}(x) =
%%{1\over2}(n+\alpha+\beta+1)P_{n-1}^{\alpha+1,\beta+1}(x),
%%\end{equation}
%%\begin{equation}\label{sobcheb-urav-3.5}
%%{d^\nu\over dx^\nu}P_n^{\alpha,\beta}(x) =
%%{(n+\alpha+\beta+1)_\nu\over2^\nu} P_{n-\nu}^{\alpha+\nu,\beta+\nu}(x),
%%\end{equation}
%%где $(a)_0=1$, $(a)_\nu=a(a+1)\dots(a+\nu-1)$, $a^{[0]}=1$,
%%
%% \begin{equation}\label{sobcheb-urav-3.6}
%% {n\choose l}P_n^{\alpha,-l}(x)= {n+\alpha\choose
%%l}\left({x+1\over2}\right)^lP_{n-l}^{\alpha,l}(x),
%%     \quad 1\le l \le n,
%%\end{equation}
%%\begin{equation}\label{sobcheb-urav-3.7}
%%P_n^{\alpha,\beta}(t) ={n+\alpha\choose n}
%%\sum_{k=0}^n{(-n)_k(n+\alpha+\beta+1)_k\over k!(\alpha+1)_k}
%%\left({1-t\over 2}\right)^k,
%%\end{equation}
%%\begin{equation}\label{sobcheb-urav-3.8}
%%(1-x)^\alpha(1+x)^\beta P_n^{\alpha,\beta}(x)
%%={(-1)^m\over2^mn^{[m]}}{d^m\over dx^m}
%%\left\{(1-x)^{m+\alpha}(1+x)^{m+\beta} P_{n-m}^{m+\alpha,m+\beta}(x)
%%\right\},
%%\end{equation}
%%где $k^{[0]}=1$, $k^{[r]}=k(k-1)\dots(k-r+1)$,
%%\begin{equation}\label{sobcheb-urav-3.9}
%%P_n^{\alpha,\beta}(-1)=(-1)^n {n+\beta\choose n},\quad P_n^{\alpha,\beta}(1)= {n+\alpha\choose n},
%%\end{equation}
%%
%%$$
%%{P_n^{a,a}(x)\over P_n^{a,a}(1)}=\sum_{j=0}^{[n/2]}
%%   { n!(\alpha+1)_{n-2j}(n+2a+1)_{n-2j}(1/2)_j(a-\alpha)_j
%%    \over (n-2j)!(2j)!(a+1)_{n-2j}(n-2j+2\alpha+1)_{n-2j}}
%%     $$
%%\begin{equation}\label{sobcheb-urav-3.10}
%%    \times{1\over (n-2j+a+1)_j(n-2j+\alpha+3/2)_j}
%%     {P_{n-2j}^{\alpha,\alpha}(x)\over
%%P_{n-2j}^{\alpha,\alpha}(1)},
%%\end{equation}
%%где $[b]$ -- целая часть числа $b$.
%%
%%
%%\begin{lemma} Пусть $\alpha>-1$, $k,r$ -- целые, $r\ge1$,
%%     $k\ge r+1$. Тогда
%%     $$
%%P_{k+r}^{\alpha-r,\alpha-r}(x)=\sum_{j=0}^r\lambda_j^\alpha
%%P_{k+r-2j}^{\alpha,\alpha}(x),
%%     $$
%%где
%%     $$
%% \lambda_j^\alpha=\lambda_j^\alpha(r,k)=
%%{(-1)^j(k-r+2\alpha+1)_{k+r-2j}(1/2)_jr^{[j]}
%%     (\alpha+k)^{[j]}
%%\over(k+r-2j+2\alpha+1)_{k+r-2j} (k+r-2j+\alpha+3/2)_j(2j)!}.
%%     $$
%%\end{lemma}
%%%\begin{proof}
%%%Предположим сначала, что
%%%$\alpha-r>-1$, тогда, полагая
%%%     $a=\alpha-r$, мы можем воспользоваться формулой \eqref{sobcheb-urav-3.10}.
%%% Поскольку  при $j\ge r+1$ выполняется равенство $(a-\alpha)_j=(-r)_j=0$, то из \eqref{sobcheb-urav-3.10} мы имеем
%%%     $$
%%%     P_{k+r}^{\alpha-r,\alpha-r}(x)=\sum_{j=0}^r\lambda_j^\alpha
%%%P_{k+r-2j}^{\alpha,\alpha}(x),
%%%     $$
%%%где
%%%     $$
%%%\lambda_j^\alpha={P_{k+r}^{\alpha-r,\alpha-r}(1)\over
%%%     P_{k+r-2j}^{\alpha,\alpha}(1)}
%%%{(k+r)!(\alpha+1)_{k+r-2j}(k-r+2\alpha+1)_{k+r-2j}\over
%%%     (k+r-2j)!(2j)!(\alpha-r+1)_{k+r-2j}}\times
%%%     $$
%%%     $$
%%%     {(1/2)_j(-r)_j\over(k+r-2j+2\alpha+1)_{k+r-2j}(k-2j+\alpha+1)_j
%%%     (k+r-2j+\alpha+3/2)_j}=
%%%     $$
%%%     $$
%%%{(-1)^j(k-r+2\alpha+1)_{k+r-2j}(1/2)_jr^{[j]}(\alpha+k)^{[j]}
%%%\over(k+r-2j+2\alpha+1)_{k+r-2j}
%%%(k+r-2j+\alpha+3/2)_j(2j)!}.
%%%     $$
%%%     Отсюда следует справедливость утверждения леммы 3.1 в случае
%%%     $\alpha>r-1$. Но поскольку  $P_{k+r}^{\alpha-r,\alpha-r}(x)$,
%%% $\lambda_j^\alpha$ и $P_{k+r-2j}^{\alpha,\alpha}(x)$ представляют
%%%собой аналитические функции относительно $\alpha$, то утверждение
%%%леммы 3.1 вытекает из уже доказанного случая.
%%%\end{proof}
%%
%%\begin{lemma} Пусть  $k,r$ -- целые, $r\ge1$,
%%     $k\ge r+1$. Тогда
%%     $$
%%P_{k+r}^{-\frac12-r,-\frac12-r}(x)=\sum_{j=0}^r\lambda_j^{-\frac12}(k,r)
%%P_{k+r-2j}^{-\frac12,-\frac12}(x),
%%     $$
%%где
%%$$
%% \lambda_j^{-\frac12}(k,r)=
%%{(-1)^j(k-r)_{k+r-2j}(1/2)_jr^{[j]}
%%     (k-1/2)^{[j]}
%%\over(k+r-2j)_{k+r-2j} (k+r-2j+1)_j(2j)!}=
%%$$
%%\begin{equation}\label{sobcheb-urav-3.11}
%%(-1)^j{((k+r-2j)!)^22^{2k+2r-4j}\over(2(k+r-2j))!}
%%{(2k)!\over(k!)^22^{2k+2r}}{r^{[j]}\over j!}
%%{k^{[r+1]}\over(k+r-j)^{[r+1]}}.
%%\end{equation}
%%\end{lemma}
%%%\begin{proof}
%%%Чтобы убедиться в справедливости утверждения леммы 3.2 достаточно в лемме 3.1 взять $\alpha=-\frac12$.
%%%\end{proof}
%%
%%Пусть $T_n(x)=\cos(n\arccos x)$ -- полином Чебышева первого рода. Тогда \cite{sobcheb_urav-Sege}
%%\begin{equation}\label{sobcheb-urav-3.12}
%%P_n^{-\frac{1}{2},-\frac{1}{2}}(x)=\frac{1\cdot3\cdot\ldots\cdot(2n-1)}
%%{2\cdot4\cdot\ldots\cdot2n}T_n(x)=\frac{(2n)!}{2^{2n}{n!}^2}T_n(x);
%%\end{equation}
%%Имеет место следующая
%%
%%\begin{lemma}\label{sobcheb-uravlemma3}
%%Пусть  $k,r$ -- целые, $r\ge1$,
%%     $k\ge r+1$. Тогда
%% $$
%%P_{k+r}^{-\frac12-r,-\frac12-r}(x)={(2k)!\over(k!)^22^{2k+2r}}\sum_{j=0}^r{(-1)^j\over j!}
%%{r^{[j]}k^{[r+1]}\over(k+r-j)^{[r+1]}}T_{k+r-2j}(x).
%%     $$
%%\end{lemma}
%%%\begin{proof}
%%%Утверждение леммы 3.3 непосредственно вытекает из леммы 3.2 и равенств \eqref{sobcheb-urav-3.12} и \eqref{sobcheb-urav-3.11}.
%%%\end{proof}
%%
%%Пусть $\alpha$ и $\beta$ -- произвольные вещественные числа,
%%$$
%%s(\theta)=\pi^{-\frac12}\left(\sin\frac{\theta}{2}\right)^{-\alpha-\frac12}
%%\left(\cos\frac{\theta}{2}\right)^{-\beta-\frac12},
%%$$
%%$$
%%\lambda_n=n+\frac{\alpha+\beta+1}{2}, \quad\gamma=-
%%\left(\alpha+\frac{1}{2}\right)\frac{\pi}{2}.
%%$$
%%Тогда для $0<\theta<\pi$ имеет место асимптотическая формула
%%\begin{equation}\label{sobcheb-urav-3.13}
%%P_n^{\alpha,\beta}(\cos\theta)=n^{-\frac12}s(\theta)
%%\left(\cos(\lambda_n\theta+\gamma)+
%%\frac{v_n(\theta)}{n\sin\theta}\right),
%%\end{equation}
%%в которой для функции $v_n(\theta)=v_n(\theta;\alpha,\beta)$
%%справедлива оценка
%%\begin{equation}\label{sobcheb-urav-3.14}
%%|v_n(\theta)|\le c(\alpha,\beta,\delta)\quad
%% \left(0<\frac{\delta}{n}\le\theta\le\pi-\frac{\delta}{n}\right).
%%\end{equation}
%


%\section{Ортогональные по Соболеву полиномы, порожденные полиномами Якоби}
%Из равенства \eqref{sobleg-2.2} следует, что если $\alpha,\beta>-1$, то полиномы
%\begin{equation}\label{sobcheb-urav-4.1}
%p_n^{\alpha,\beta}(x)={P_n^{\alpha,\beta}(x)\over\sqrt{ h_n^{\alpha,\beta}}}\quad(n=0,1,\ldots)
%\end{equation}
%образуют ортонормированную  в $L_\kappa^2(-1,1)$ с весом $\rho(x)=(1-x)^\alpha(1+x)^\beta$ систему. Как хорошо известно \cite{sobcheb_urav-Sege}, система полиномов Якоби \eqref{sobcheb-urav-4.1}полна в $L_\kappa^2(-1,1)$.   Эта система порождает на $[-1,1]$ систему полиномов  $p_{r,k}^{\alpha,\beta}(x)=\varphi_{r,k}(x)$ $(k=0,1,\ldots)$, определенных равенствами \eqref{sobcheb-urav-1.10} и \eqref{sobcheb-urav-1.11} с $a=-1$, $\varphi_k(x)=p_{k}^{\alpha,\beta}(x)$. Из теоремы \ref{sobcheb-uravtheo1} непосредственно вытекает
%\begin{corollary}
%Пусть $a=-1$, $\alpha,\beta>-1$. Тогда система полиномов $\{p_{r,k}^{\alpha,\beta}(x)\}_{k=0}^\infty$, порожденная системой ортонормированных полиномов Якоби \eqref{sobcheb-urav-4.1} посредством равенств \eqref{sobcheb-urav-1.10} и \eqref{sobcheb-urav-1.11}, полна  в $W^r_{L^2_\rho(-1,1)}$ и ортонормирована относительно скалярного произведения \eqref{sobcheb-urav-1.13}.
%\end{corollary}
%
%Ряд Фурье \eqref{sobcheb-urav-1.14} для системы   $\{p_{r,k}^{\alpha,\beta}(x)\}_{k=0}^\infty$ приобретает вид
%\begin{equation}\label{sobcheb-urav-4.2}
%f(x)\sim \sum_{k=0}^{r-1} f^{(k)}(-1)\frac{(x+1)^k}{k!}+ \sum_{k=r}^\infty \hat f_{r,k}p_{r,k}^{\alpha,\beta}(x),
%\end{equation}
%где
%  \begin{equation}\label{sobcheb-urav-4.3}
% \hat f_{r,k}=\int_{-1}^1 f^{(r)}(t)p_{k-r}^{\alpha,\beta}(t)\rho(t)dt.
%\end{equation}
%
%\begin{corollary}
%Пусть $-1<\alpha,\beta<1$. Тогда, если $f(x)\in W^r_{L^2_\rho(-1,1)}$, то ряд Фурье (смешанный ряд) \eqref{sobcheb-urav-4.2} сходится к функции $f(x)$ равномерно относительно $x\in[-1,1]$.
%\end{corollary}
%%\begin{proof}
%%Заметим, что если $-1<\alpha,\beta<1$, то $\frac{1}{\rho(x)}\in L(-1,1)$, где $\rho(x)=(1-x)^\alpha(1+x)^\beta$. Поэтому утверждение следствия 2 вытекает из теоремы 2 и следствия 1.
%%\end{proof}
%
%В следующей теореме установлен явный вид полиномов $p_{r,k}^{\alpha,\beta}(x)$ при $k\ge r$, который играет  важную роль при исследовании асимптотических свойств полиномов $p_{r,n+r}^{\alpha,\beta}(x)$ в окрестности точки $x=-1$.
%
%
%
%\begin{theorem} Для произвольных $\alpha, \beta>-1$ и $n\ge0$
%имеет место следующее равенство
%\begin{equation}\label{sobcheb-urav-4.4}
%p_{r,n+r}^{\alpha,\beta}(x)=\frac{(-1)^n2^{r}}{\sqrt{ h_n^{\alpha,\beta}}}
%{n+\beta\choose n}
%\sum_{k=0}^n{(-n)_k(n+\alpha+\beta+1)_k\over (\beta+1)_k(k+r)!}
%\left({1+x\over 2}\right)^{k+r}.
%\end{equation}
%\end{theorem}
%%\begin{proof}
%%Воспользуемся равенством \eqref{sobcheb-urav-3.7} и запишем
%%$$
%%P_{n}^{\alpha,\beta}(t)=(-1)^nP_{n}^{\beta,\alpha}(-t)=(-1)^n{n+\beta\choose n}
%%\sum_{k=0}^n{(-n)_k(n+\alpha+\beta+1)_k\over k!(\beta+1)_k}
%%\left({1+t\over 2}\right)^k,
%%$$
%%поэтому, в силу \eqref{sobcheb-urav-4.1} имеем
%%\begin{equation}\label{4.5}
%%p_{n}^{\alpha,\beta}(t)=\frac{(-1)^n}{\sqrt{ h_n^{\alpha,\beta}}}{n+\beta\choose n}
%%\sum_{k=0}^n{(-n)_k(n+\alpha+\beta+1)_k\over k!(\beta+1)_k}
%%\left({1+t\over 2}\right)^k.
%%\end{equation}
%%С другой стороны в силу формулы Тейлора
%%\begin{equation}\label{4.6}
%%\left({1+x\over 2}\right)^{k+r}={(k+r)^{[r]}\over 2^r(r-1)!}\int\limits_{-1}^x(x-t)^{r-1} \left({1+t\over 2}\right)^{k}dt.
%%\end{equation}
%%Сопоставляя \eqref{4.5} и \eqref{4.6} с определением
%%\begin{equation}\label{4.7}
%%  p_{r,n+r}^{\alpha,\beta}(x) =\frac{1}{(r-1)!}\int\limits_{-1}^x(x-t)^{r-1}p_{n}^{\alpha,\beta}(t)dt, \quad n=1,\ldots,
%%\end{equation}
%% убеждаемся в справедливости утверждения теоремы 3.
%%\end{proof}



