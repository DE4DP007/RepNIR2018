\begin{thebibliography}{999}
\bibitem{Haar-Tcheb-Shar11}

 Шарапудинов И.И. Приближение функций с переменной гладкостью суммами Фурье Лежандра // Математический сборник. 2000. Том 191, вып. 5. С. 143-160.



\bibitem{Haar-Tcheb-Shar13}

 Шарапудинов И.И. Смешанные ряды по ортогональным полиномам // Махачкала. Издательство Дагестанского научного центра. 2004.



\bibitem{Haar-Tcheb-Shar15}

 Шарапудинов И.И. Аппроксимативные свойства смешанных рядов по полиномам Лежандра на классах $W^r$ // Математический сборник. 2006. Том 197, вып. 3. С. 135--154.



\bibitem{Haar-Tcheb-Shar16}

 Шарапудинов И.И. Аппроксимативные свойства средних типа Валле-Пуссена частичных сумм смешанных рядов по полиномам Лежандра // Математические заметки. 2008. Том 84, вып. 3. С. 452--471



\bibitem{Haar-Tcheb-Shar18}

 Шарапудинов И.И., Шарапудинов Т.И. Смешанные ряды по полиномам Якоби и Чебышева и их дискретизация // Математические заметки. 2010. Том 88, вып. 1. С. 116--147.



\bibitem{Haar-Tcheb-Shar19}

 Шарапудинов И.И.,  Муратова Г.Н. Некоторые свойства r-кратно интегрированных рядов по системе Хаара // Изв. Сарат. ун-та. Нов. сер. Сер. Математика. Механика. Информатика. 2009. Том 9, вып. 1. С. 68--76.



\bibitem{Haar-Tcheb-IserKoch}

 Iserles A., Koch P.E., Norsett S.P. and J.M. Sanz-Serna. On polynomials  orthogonal  with respect  to certain Sobolev inner products // J. Approx. Theory. 1991. Vol. 65. P. 151--175.



\bibitem{Haar-Tcheb-MarcelAlfaroRezola }

 Marcellan F., Alfaro M. and Rezola M.L. Orthogonal polynomials on Sobolev spaces: old and new directions // Journal of Computational and Applied Mathematics. 1993. Vol. 48. P. 113--131.



\bibitem{Haar-Tcheb-Meijer}

 Meijer H.G. Laguerre polynomials generalized to a certain discrete Sobolev inner product space // J. Approx. Theory. 1993. Vol. 73. P. 1--16.



\bibitem{Haar-Tcheb-KwonLittl1}

 Kwon K.H. and Littlejohn L.L. The orthogonality of the Laguerre polynomials $\{L_n^{(-k)}(x)\}$ for positive integers $k$ // Ann. Numer. Anal. 1995. N 2. P. 289--303.



\bibitem{Haar-Tcheb-KwonLittl2}

 Kwon K.H. and Littlejohn L.L. Sobolev orthogonal polynomials and second-order differential equations // Rocky Mountain J. Math. 1998. Vol. 28. P. 547--594.



\bibitem{Haar-Tcheb-MarcelXu}

 Marcellan F. and Yuan Xu. ON SOBOLEV ORTHOGONAL POLYNOMIALS // arXiv: 6249v1 [math.C.A] 25 Mar 2014. P. 1--40.



\bibitem{Haar-Tcheb-Lopez1995}

 Lopez G. Marcellan F. Vanassche W. Relative Asymptotics for Polynomials Orthogonal with Respect to a Discrete Sobolev Inner-Product // Constr. Approx.1995. Vol. 11. N 1. P. 107--137.



\bibitem{Haar-Tcheb-Gonchar1975}

 Гончар А.А. О сходимости аппроксимаций Паде для некоторых классов мероморфных функций // Математический сборник. 1975. Т. 97(139), вып. 4(8). C. 607--629.



\bibitem{Haar-Tcheb-Tref1}

 Trefethen L.N. Spectral methods in Matlab // Philadelphia. SIAM. 2000.



\bibitem{Haar-Tcheb-Tref2}

 Trefethen L.N. Finite difference and spectral methods for ordinary and partial differential equation // Cornell University. 1996.



\bibitem{Haar-Tcheb-SolDmEg}

Солодовников В.В., Дмитриев А.Н., Егупов Н.Д. Спектральные методы расчета и проектирования систем управления // Москва: Машиностроение. 1986.



\bibitem{Haar-Tcheb-Pash}

 Пашковский С. Вычислительные применения многочленов и рядов Чебышева // Москва: Наука. 1983.



\bibitem{Haar-Tcheb-Arush2010}

 Арушанян О.Б., Волченскова Н.И., Залеткин С.Ф. Приближенное решение обыкновенных дифференциальных уравнений с использованием рядов Чебышева // Сиб. электрон. матем. изв. 1983. Вып. 7. С. 122--131.



\bibitem{Haar-Tcheb-Arush2013}

Арушанян О.Б., Волченскова Н.И., Залеткин С.Ф. Метод решения задачи Коши для обыкновенных дифференциальных уравнений с использованием рядов Чебышeва // Выч. мет. программирование. 2013. Т. 14, вып. 2. С. 203--214.



\bibitem{Haar-Tcheb-Arush2014}

 Арушанян О.Б., Волченскова Н.И., Залеткин С.Ф. Применение рядов Чебышева для интегрирования обыкновенных дифференциальных уравнений // Сиб. электрон. матем. изв. 2014. Вып. 11. С. 517--531.



\bibitem{Haar-Tcheb-Lukom2016}

 Лукомский Д.С., Терехин П.А. Применение системы Хаара к численному решению задачи Коши для линейного дифференциального уравнения первого порядка // Материалы 18-й международной Саратовской зимней школы «Современные проблемы теории функций и их приложения». Саратов. ООО «Издательство «Научная книга». 2016. С. 171--173.



\bibitem{Haar-Tcheb-MMG2016}

 Магомед-Касумов М.Г. Приближенное решение обыкновенных дифференциальных уравнений с использованием смешанных рядов по системе Хаара // Материалы 18-й международной Саратовской зимней школы «Современные проблемы теории функций и их приложения». Саратов. ООО «Издательство «Научная книга». 2016. С. 176--178.



\bibitem{Haar-Tcheb-KashSaak}

 Кашин Б.С., Саакян А.А. Ортогональные ряды. Москва: АФЦ. 1999.



\bibitem{Haar-Tcheb-Sege}

 Сеге Г. Ортогональные многочлены. Москва: Физматгиз. 1962.



\bibitem{Haar-Tcheb-Gasper}

 Gasper G. Positivity and special function // Theory and appl. Spec. Funct. Edited by Richard A. Askey. 1975. P. 375--433.



\bibitem{Haar-Tcheb-Muckenhoupt}

 Muckenhoupt B. Mean convergence of Jacobi series // Proc. Amer. Math. Soc. 1969. Vol. 23. N 2 P. 306--310.

%====== Section 2 =========




\bibitem{sob-jac-discrete-Shar12}

 Шарапудинов И.И. Аппроксимативные свойства операторов $\mathcal{Y}_{n+2r}(f)$ и их дискретных аналогов // Математические заметки. 2002. Т. 72, вып. 5. С. 765--795.



\bibitem{sob-jac-discrete-Shar17}

 Шарапудинов И.И. Смешанные ряды по ультрасферическим полиномам // Математический сборник. 2003. Т. 194, вып. 3. С. 115--148.






\bibitem{sob-jac-discrete-Sege}

 Сеге Г. Ортогональные многочлены. Москва: Физматгиз. 1962.



\bibitem{sob-jac-discrete-Timan}

 Тиман А.Ф. Теория приближения функций действительного переменного. Москва: Физматгиз, 1962.



\bibitem{sob-jac-discrete-Tel}

Теляковский С.А. Две теоремы о приближении функций
алгебраическими многочленами // Математический сборник. 1966. Т. 70, вып. 2. С. 252--265.



\bibitem{sob-jac-discrete-Gop}

 Гопенгауз И.З. К теореме А.Ф. Тимана о приближении
функций многочленами на конечном отрезке // Математические заметки. 1967. Т. 1, вып. 2. C. 163--172.



\bibitem{meixner-7}

 Шарапудинов И.И. Приближение дискретных функций и многочлены Чебышева, ортогональные на равномерной сетке // Мат. заметки. 2000. Т. 67, вып. 3. С. 460--470. DOI: 10.4213/mzm858



\bibitem{meixner-8}

 Шарапудинов И.И. Приближение функций с переменной гладкостью суммами Фурье- Лежандра // Мат. сборник. 2000. Т. 191, вып. 5. С. 143--160. DOI: 10.4213/sm480



\bibitem{meixner-10}

 Шарапудинов И.И. Смешанные ряды по ультрасферическим полиномам и их аппроксимативные свойства // Мат. сборник. 2003. Т. 194, вып 3. С. 115--148. DOI: 10.4213/sm723



\bibitem{meixner-11}

 Шарапудинов И.И. Смешанные ряды по ортогональным полиномам. Махачкала: Дагестан. науч. центр РАН. 2004. С. 1--176.



\bibitem{meixner-12}

 Шарапудинов И.И. Смешанные ряды по полиномам Чебышева, ортогональным на равномерной сетке // Мат. заметки. 2005. т. 78. вып. 3. C. 442--465. DOI: 10.4213/mzm2599



\bibitem{meixner-14}

 Шарапудинов Т.И. Аппроксимативные свойства смешанных рядов по полиномам Чебышева, ортогональным на равномерной сетке // Вестник Дагестанского научного центра РАН. 2007. Т. 29. C. 12--23.



\bibitem{meixner-18}

 Шарапудинов И.И. Системы функций, ортогональных по Соболеву, порожденные ортогональными функциями // Современные проблемы теории функций и их приложения. Материалы 18-й международной Саратовской зимней школы. 2016. C. 329--332.



\bibitem{meixner-19}

 Trefethen L.N. Spectral methods in Matlab. Fhiladelphia: SIAM. 2000.



\bibitem{meixner-20}

 Trefethen L.N. Finite difference and spectral methods for ordinary and partial differential equation. Cornell University. 1996.



\bibitem{meixner-22}

 Шарапудинов И.И. Многочлены, ортогональные на дискретных сетках. Издательство Даг. гос. пед. ун-та. Махачкала. 1997.



\bibitem{meixner-23}

 Gasper G. Positiviti and special function // Theory and appl. Spec. Funct. Edited by Richard A.Askey. 1975. P. 375--433.





\bibitem{sob-tcheb-difference-IserKoch}


Iserles A., Koch P.E., Norsett S.P. and Sanz-Serna J.M. On polynomials  orthogonal  with respect  to certain Sobolev inner products //
J. Approx. Theory. 1991. Vol. 65.  P. 151--175.





\bibitem{sob-tcheb-difference-MarcelAlfaroRezola}


Marcellan F., Alfaro M. and Rezola M.L.
Orthogonal polynomials on Sobolev spaces: old and new directions //
Journal of Computational and Applied Mathematics. North-Holland. 1993. Vol. 48. P. 113--131.





\bibitem{sob-tcheb-difference-Meijer}


Meijer H.G. Laguerre polynimials generalized to a certain discrete Sobolev inner product space // J. Approx. Theory. Vol. 73. 1993. P. 1--16.




\bibitem{sob-tcheb-difference-KwonLittl1}


Kwon K.H. and Littlejohn L.L.
The orthogonality of the Laguerre polynomials $\{L_n^{(-k)}(x)\}$ for positive integers $k$ //
Ann. Numer. Anal. 1995. N 2. P. 289--303.



\bibitem{sob-tcheb-difference-KwonLittl2}


 Kwon K.H. and Littlejohn L.L. Sobolev orthogonal polynomials and second-order differential equations // Rocky Mountain J. Math. 1998. Vol. 28. P. 547--594.



\bibitem{sob-tcheb-difference-MarcelXu}


Marcellan F. and Yuan Xu.
On Sobolev orthogonal polynomials // arXiv: 6249v1 [math.C.A] 25 Mar 2014. 2014. P. 1--40.



\bibitem{sob-tcheb-difference-Shar9}


 Шарапудинов И.И.
Приближение дискретных функций и многочлены Чебышева, ортогональные на равномерной сетке //
Математические заметки. 2000. Т. 67, вып. 3. С. 460--470.



\bibitem{sob-tcheb-difference-Shar1}


Шарапудинов И.И. Приближение функций с переменной гладкостью суммами Фурье Лежандра //
Математический сборник. 2000. Т. 191, вып. 5. С. 143--160.




\bibitem{sob-tcheb-difference-Shar2}


Шарапудинов И.И.
Аппроксимативные свойства операторов $\mathcal{Y}_{n+2r}(f)$ и их дискретных аналогов //
Математические заметки. 2002 Т. 72, вып. 5. С. 765--795.



\bibitem{sob-tcheb-difference-Shar3}


Шарапудинов И.И. Смешанные ряды по ультрасферическим полиномам и их аппроксимативные свойства //
Математический сборник. 2003. Т. 194, вып. 3. С. 115--148.



\bibitem{sob-tcheb-difference-Shar4}


Шарапудинов И.И.
Смешанные ряды по ортогональным полиномам. Издательство Дагестанского научного центра. Махачкала. 2004. С. 1--176.




\bibitem{sob-tcheb-difference-Shar11}


Шарапудинов И.И. Смешанные ряды по полиномам Чебышева, ортогональным на равномерной сетке // Математические заметки. 2005 Т. 78, вып. 3. С. 442--465.




\bibitem{sob-tcheb-difference-Shar5}


Шарапудинов И.И.
Аппроксимативные свойства смешанных рядов по полиномам Лежандра на классах $W^r$,
Математический сборник. 2006. Т. 197, вып. 3. С. 135--154.



\bibitem{sob-tcheb-difference-Shar6}


Шарапудинов И.И.
Аппроксимативные свойства средних типа Валле-Пуссена частичных сумм смешанных рядов по полиномам Лежандра //
Математические заметки. 2008. Т. 84, вып. 3. С. 452--471.



\bibitem{sob-tcheb-difference-Shar7}


 Шарапудинов И.И.,  Муратова Г.Н. Некоторые свойства r-кратно интегрированных рядов по системе Хаара // Изв. Сарат. ун-та. Нов. сер. Сер. Математика. Механика. Информатика. 2009. Т. 9, вып. 1. С. 68--76.



\bibitem{sob-tcheb-difference-Shar8}


Шарапудинов И.И., Шарапудинов Т.И.
Смешанные ряды по полиномам Якоби и Чебышева и их дискретизация //
Математические заметки. 2010. Т. 88, вып. 1.  С. 116--147.




\bibitem{sob-tcheb-difference-SharII}


Шарапудинов И.И.
Системы функций, ортогональных по Соболеву, порожденные ортогональными функциями //
Современные проблемы теории функций и их приложения.  Материалы 18-й международной Саратовской зимней школы. 2016. C. 329--332.



\bibitem{sob-tcheb-difference-Cheb1}


Чебышев П.Л. О непрерывных дробях // Полн. собр. соч. Изд. АН СССР. Москва. 1947. Т. 2. С. 103--126.



\bibitem{sob-tcheb-difference-Cheb2}


Чебышев П.Л. Об одном новом ряде // Полн. собр. соч. Изд. АН СССР. Москва. 1947. Т. 2. С. 236--238.



\bibitem{sob-tcheb-difference-Cheb3}


Чебышев П.Л.
Об интерполировании по способу наименьших квадратов //
Полн. собр. соч. Изд. АН СССР. Москва. 1947. Т. 2. С. 314--334.



\bibitem{sob-tcheb-difference-Cheb4}


Чебышев П.Л. Об интерполировании // Полн. собр. соч. Изд. АН СССР. 1947. Москва. Т. 2. С. 357--374.



\bibitem{sob-tcheb-difference-Cheb5}


Чебышев П.Л.
Об интерполировании величин  равноотстоящих (1875) //
Полн. собр. соч. Изд. АН СССР. Москва. 1947. Т. 2. С. 66--87.



\bibitem{sob-tcheb-difference-SolDmEg}


Солодовников В.В., Дмитриев А.Н., Егупов Н.Д.
Спектральные методы расчета и проектирования систем управления. Москва: Машиностроение. 1986.




\bibitem{sob-tcheb-difference-Tref1}


Trefethen L.N. Spectral methods in Matlab. 2000. SIAM. Fhiladelphia.



\bibitem{sob-tcheb-difference-Shar16}


Шарапудинов И.И.
Асимптотические свойства и весовые оценки для ортогональных многочленов Чебышева–Хана,
// Математический сборник. 1991. Т. 182, вып. 3. С. 408--420.



\bibitem{ogrValle}


Шарапудинов И.И.
Об ограниченности в $C[-1,1]$ средних Валле-Пуссена для дискретных сумм Фурье–Чебышева,
// Математический сборник. 1996. Т. 187, вып. 1. С. 143--160.




\bibitem{sob-tcheb-difference-Shar17}


Шарапудинов И.И.
Об асимптотике многочленов Чебышева, ортогональных на конечной системе точек //
Вестник МГУ. Серия 1. 1992. Т. 1. С. 29--35.



\bibitem{sob-tcheb-difference-Shar18}


Шарапудинов И.И. Многочлены, ортогональные на дискретных сетках,
Издательство Даг. гос. пед. ун-та. Махачкала. 1997.



\bibitem{sob-tcheb-difference-MagKas}


Магомед-Касумов М.Г.
Приближенное решение обыкновенных дифференциальных уравнений с использованием смешанных рядов по системе Хаара //
Современные проблемы теории функций и их приложения. Материалы 18-й международной Саратовской зимней школы.  27 января -- 3 февраля 2016, С. 176--178.



\bibitem{sob-lag-sb-Meijer}


 Meijer H.G.  Laguerre polynomials generalized to a certain discrete Sobolev inner product space // J. Approx. Theory. 1993. Vol. 73. P. 1--16.



\bibitem{sob-lag-sb-Sege}

 Сеге Г. Ортогональные многочлены. Москва:Физматгиз. 1962.



\bibitem{sob-lag-sb-AskeyWaiger}

 Askey R., Wainger S. Mean convergence of expansions in Laguerre and Hermite series // Amer. J. Mathem. 1965. Vol. 87. P. 698--708.

%
%
%
\bibitem{Ram14}


Гаджиева З.Д. Смешанные ряды по полиномам Мейкснера. Кандидатская диссертация - Саратов. Саратовский гос. ун-т. 2004.


%
\bibitem{sob-leg-MarcelXu}


Marcellan F. and Yuan Xu. On Sobolev orthogonal polynomials // Expositiones Mathematicae. 2015. Vol. 33. N 3. P. 308--352.


\bibitem{sob-leg-Shar2016}

 Шарапудинов И.И. Системы функций, ортогональные по Соболеву, порожденные ортогональными функциями // Материалы 18-й международной Саратовской зимней школы «Современные проблемы теории функций и их приложения». Саратов: ООО «Издательство «Научная книга». 2016. С. 329--332.


\bibitem{sob-leg-SolDmEg}

 Солодовников В.В., Дмитриев А.Н., Егупов Н.Д. Спектральные методы расчета и проектирования систем управления. Москва: Машиностроение. 1986.


\bibitem{sob-leg-Pash}

Пашковский С. // Вычислительные применения многочленов и рядов Чебышева. Москва: Наука. 1983. С. 143--160.


\bibitem{sob-leg-SHII}

 Шарапудинов И.И. Некоторые специальные ряды по общим полиномам Лагерра и ряды Фурье по полиномам Лагерра, ортогональным по Соболеву // Дагестанские электронные математические известия. 2015. Т. 4.


\bibitem{sob-leg-TEL}

 Теляковский С.А. Две теоремы о приближении функций алгебраическими многочленами // Математический сборник. 1966. Т. 70, вып. 2. С. 252--265.


\bibitem{sob-leg-GOP}

 Гопенгауз И.З. К теореме А.Ф. Тимана о приближении функций многочленами на конечном отрезке // Математические заметки. 1967. Т. 1, вып. 2. С. 163--172.


\bibitem{sob-leg-OSK}

 Осколков К.И. К неравенству Лебега в равномерной метрике и на множестве полной меры // Математические  заметки. 1975. Т. 18, вып. 4. С. 515--526.


\bibitem{sob-leg-sharap1}

 Sharapudinov I.I. On the best approximation and polinomial of the least quadratic deviation // Analysis Mathematica. 1983. Vol. 9. N 3. P. 223--234.


\bibitem{sob-leg-sharap2}

 Шарапудинов И.И. О наилучшем приближении и суммах Фурье-Якоби // Математические заметки. 1983. Т. 34, вып. 5. С. 651--661.


\bibitem{sob-leg-sharap3}

 Шарапудинов И.И. Некоторые специальные ряды по ультрасферическим полиномам и их аппроксимативные свойства // Изв. РАН. Сер. матем. 1983. Т. 78, вып. 5. С. 201--224.


\bibitem{ark-1}

  Алберг Дж., Нилсон Э., Уолш Дж.. Теория сплайнов и ее приложения  М.: Мир. 1972.



\bibitem{ark-2}

  Стечкин С.Б., Субботин Ю.Н. Добавления к книге Дж. Алберг, Э. Нилсон, Дж. Уолш.
 Теория сплайнов и ее приложения.  М.: Мир. 1972.



\bibitem{ark-3}

 Стечкин С.Б., Субботин Ю.Н.  Сплайны в вычислительной математике. М.:  Наука. 1976.



\bibitem{ark-4}

  Корнейчук Н.П.  Сплайны в теории приближения. М.:  Наука. 1984.




\bibitem{ark-5}

  Малоземов В.Н., Певный А.Б. Полиномиальные сплайны. Л.: Изд-во ЛГУ. 1986.



\bibitem{ark-6}

 Завьялов Ю.С., Квасов Б.И., Мирошниченко В.Л.  Методы сплайн-функций.  М.: Наука. 1980.




\bibitem{ark-7}

 Schaback R. Spezielle rationale Splinefunktionen //  J. Approx. Theory. Academic Press, Inc. 1973. V.7. N 2. P. 281--292.



\bibitem{ark-8}

  Duan Q., Djidjeli K., Price W.G., Twizell E.H.  Weighted rational  cubic spline interpolation and its application // J. of Computational and Applied Mathematics. 2000. V. 117. N 2. P. 121--135.



\bibitem{ark-9}

  Hussain M.Z., Sarfraz M., Shaikh T.S.  Shape preserving rational cubic spline for positive and convex data // Egyptian Informatics Journal. 2011. V. 12. P. 231--236.



\bibitem{ark-10}

  Edeoa A., Gofeb G., Tefera T.  Shape preserving $C^2$ rational cubic spline interpolation //  American Scientific Research Journal for Engineering, Technology and Sciences. 2015. V. 12. N 1. P. 110--122.



\bibitem{ark-11}

Рамазанов А.-Р.К., Магомедова В.Г. Сплайны по рациональным интерполянтам // Дагестанские
электронные математические известия. 2015. Вып. 4. С. 22--31.



\bibitem{ark-12}

 Рамазанов А.-Р.К., Магомедова В.Г. Интерполяционные рациональные сплайны // Вестник ДГУ. Серия 1. Естеств. науки. 2016.
Т. 31, вып. 2. С. 35--40.



\bibitem{ark-13}

 Рамазанов А.-Р.К., Магомедова В.Г. Сплайны по четырехточечным рациональным интерполянтам // Труды Института матем. и механики УрО РАН. 2016. Т. 22, вып. 4. С. 233--246.



\bibitem{sob-lag-smj-Pash}

 С. Пашковский. Вычислительные применения многочленов и рядов Чебышева: Пер. с польск. // Москва: Наука. 1983.


\bibitem{sob-lag-smj-fiht2}

 Фихтенгольц Г.М. Курс дифференциального и интегрального исчисления, т. 2 // Москва: Физматлит. 2001. 810 с.




\bibitem{sms1}

 Meyer Y. Ondelettes et Operateurs // Paris. Hermann. 1990. Vol. I–III.



\bibitem{sms2}

 Daubechies L. Ten Lectures on Wavelets // CBMS-NSF Regional Conference Series in Applied Mathematics Proceedings. 1992. Vol. 61. Philadelphia. PA: SIAM. 357 p. ISBN 0-89871-274-2. DOI: 10.1137/1.9781611970104.



\bibitem{sms3}

 Chui C.K. An Introduction to  Wavelets // Boston. Academic Press. 1992. 271 p. ISBN 0-12-174584-8.



\bibitem{sms4}

 Chuii C.K. Mhaskar H.N. On Trigonometric Wavelets // Constructive Approximation, 1993. Vol. 9. N 2--3. P. 167--190. DOI: 10.1007/BF01198002.



\bibitem{sms5}

 Kilgore T., Prestin J. Polynomial wavelets on an interval // Constructive Approximation, 1996. Vol. 12. N 1. P. 95--110. DOI: 10.1007/BF02432856.



\bibitem{sms6}

 Fischer B., Prestin J. Wavelet based on orthogonal polynomials // Mathematics of computation. 1997. Vol. 66. N 220. P. 1593--1618. DOI: 10.1090/S0025-5718-97-00876-4.



\bibitem{sms7}

 Fischer B., Themistoclakis W. Orthogonal polynomial wavelets // Numerical Algorithms. 2002. Vol. 30. Issue 1. P. 37--58. DOI: 10.1023/A:1015689418605.



\bibitem{sms8}

 Capobiancho M.R., Themistoclakis W. Interpolating polynomial wavelet on $[-1,1]$ // Advanced in Computational Mathematics. 2005. Vol. 23. N 4. P. 353--374. DOI: 10.1007/s10444-004-1828-2.



\bibitem{sms9}

 Dao-Qing Dai, Wei Lin. Orthonormal polynomial wavelets on the interval // Proceedings of the American Mathematical Society. 2005. Vol. 134. N 5. P. 1383--1390. DOI: 10.1090/S0002-9939-05-08088-3.



\bibitem{sms10}

 Mohd F., Mohd I. Orthogonal Functions Based on Chebyshev Polynomials // Matematika. 2011. Vol. 27. N 1. P. 97--107.



\bibitem{sms11}

 Султанахмедов М.С. Аппроксимативные свойства вейвлет-рядов Чебышева второго рода  // Владикавказский математический журнал. 2015. Том 17, вып. 3. С. 56--64.



\bibitem{sms112}

 Султанахмедов М.С. Специальные вейвлеты на основе полиномов Чебышева второго рода  //
Изв. Сарат. ун-та. Нов. сер. Сер. Математика. Механика. Информатика. 2016. Т. 16, вып. 1. C. 34--41.




\bibitem{sms12}

 Шарапудинов И.И. Предельные ультрасферические ряды и их аппроксимативные свойства // Математические заметки. 2013. Т. 94, вып. 2. С. 295--309. DOI: 10.4213/mzm10292.



\bibitem{smsshti1}

 Шарапудинов T.И.
Конечные предельные ряды по полиномам Чебышева, ортогональным на равномерных сетках
// Изв. Сарат. ун-та. Нов. сер. Сер. Математика. Механика. Информатика. 2013. Т. 13, вып. 1(2). C. 104--108.




\bibitem{sms14}

 Яхнин Б.М. О функциях Лебега разложений в ряды по полиномам Якоби для случаев $\alpha=\beta=\frac12, \alpha=\beta=-\frac12, \alpha=\frac12, \beta=-\frac12$ // Успехи математических наук. 1958. Т. 13, вып. 6(84). C. 207--211.



\bibitem{sms15}

 Яхнин Б.М. Приближение функций класса $Lip_\alpha$ частными суммами ряда Фурье по многочленам Чебышева второго рода // Изв. вузов. Матем. 1963. Вып. 1, C. 172--178.



\bibitem{sms17}

 Тиман А.Ф. Теория приближения функций действительного переменного // Москва: Физматгиз. 1960. 626 с.



\bibitem{sms18}

 Шарапудинов И.И. О наилучшем приближении и суммах Фурье–Якоби // Математические заметки, 1983. Т. 34, вып. 5. С. 651--661. DOI: 10.1007/BF01157445.



\bibitem{ramazanov-2}

  Стечкин С.Б., Субботин Ю.Н. Добавления к книге Дж. Алберг, Э. Нилсон, Дж. Уолш.
 Теория сплайнов и ее приложения.  М.: Мир. 1972.



\bibitem{ramazanov-3}

 Стечкин С.Б., Субботин Ю.Н.  Сплайны в вычислительной математике. М.:  Наука. 1976.



\bibitem{ramazanov-4}

Корнейчук Н.П.  Сплайны в теории приближения. М.:  Наука. 1984.




\bibitem{ramazanov-6}

Завьялов Ю.С., Квасов Б.И., Мирошниченко В.Л.  Методы сплайн-функций.  М.: Наука. 1980.




\bibitem{fiht_diff_v3}

Фихтенгольц Г.М. Курс дифференциального и интегрального исчисления. Москва: Физматлит. Т. 3. 2001. С. 662.

\end{thebibliography}
