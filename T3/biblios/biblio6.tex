\begin{thebibliography}{1} %% здесь библиографический список

%\bibitem{filosofyNewestdict}
%{Грицанов} А.А.~и др.
%\newblock {\em Новейший философский словарь}.
%\newblock Мн.: Книжный Дом., 2003.




\bibitem{valle-pussen-2-NIK}Никольский С.М. О некоторых методах приближения тригонометрическими суммами // Изв. АН СССР. Сер. матем., 1940. Т. 4, вып.  6. С. 509--520.


\bibitem{valle-pussen-2-EFIM}Ефимов А.В. О приближении периодических функций суммами Валле-Пуссена // Изв. АН СССР. Сер. матем. 1959.  Т. 23, вып. 5. С. 737--770.

\bibitem{valle-pussen-2-TEl}Теляковский С.А. О приближении дифференцируемых функций линейными средними их рядов Фурье // Изв. АН СССР. Сер. матем. 1960. Т. 24, вып. 2. С. 213--242.





\bibitem{valle-pussen-2-Zhuk} Жук В.В. Аппроксимация периодических функций. Ленинград. 1982.


\bibitem{valle-pussen-2-mmg} Магомед-Касумов М.Г. Аппроксимативные свойства классических средних Валле-Пуссена для кусочно гладких функций
// Вестник Дагестанского научного центра РАН. 2014. Т. 54. С. 5--12.






\bibitem{valle-pussen-2-Shar7} Шарапудинов И.И. Аппроксимативные свойства средних Валле-Пуссена на классах типа Соболева с переменным показателем // Вестник Дагестанского научного центра РАН. 2012. Вып. 45. С. 5–13.

\bibitem{valle-pussen-2-Shar8} Шарапудинов И.И.  Приближение гладких функций в  $L_{2\pi}^{p(x)}$ средними Валле-Пуссена // Известия Саратовского университета. Математика. Механика. Информатика. 2012. Т. 13, вып.  1. Часть 1, С. 45--49.




\bibitem{valle-pussen-2-Shar6}Шарапудинов И.И. Приближение функций в $L^{p(x)}_{2\pi}$ тригонометрическими полиномами // Известия РАН: Серия математическая. 2013, Т. 77, вып. 2. С. 197--224.




\bibitem{valle-pussen-2-Zigmund} Зигмунд А. Тригонометрические ряды. Т. 1. Москва: Мир. 1965.




\bibitem{akm-1}Свами М., Тхуласираман К. Графы, сети и алгоритмы. Москва. Мир. 1982.

\bibitem{akm-2}
Визинг В.Г. Об оценке хроматического класса $p$-графа //
Дискретный анализ. Сб. науч. тр. 1964. Т. 3. Новосибирск. Ин-т математики СО АН СССР. -- С. 25--30.

\bibitem{akm-3} Holyer I. The NP-completeness of edge-coloring //
 SIAM J. Comput. 1981. V. 10. N 4. -- P. 718--720.

\bibitem{akm-4}
Ловас Л., Пламмер М. Прикладные задачи теории графов. Теория паросочетаний в математике, физике, химии: Пер. с англ. Москва. Мир. 1998.

\bibitem{akm-5}
Асратян А.С., Камалян Р.Р. Интервальные раскраски ребер мультиграфа // Прикладная математика. Т. 5. Ереван.  Изд-во Ереванского ун-та. 1987.  С. 25--34.

\bibitem{akm-6}
~Магомедов А.М. Непрерывное расписание для специализированных процессоров без отношения предшествования //
Вестник Московского Энергетического Института, сер. Автоматика, выч.техника, информатика. 2009. Т. 5. С. 14--17.

\bibitem{akm-7}
Магомедов А.М., Сапоженко А.А.
Условия существования непрерывных расписаний длительности пять // Вестник МГУ, сер. Вычислительная математика и кибернетика. Т. 34. Вып. 1. 2010. С. 39--44.

\bibitem{akm-8}
Магомедов А.М., Магомедов Т.А. О приложении алгоритма вычисления подграфа максимальной плотности к задаче оптимизации расписания // Матем. заметки. 2013. Т. 93. Вып. 2. С. 313--315.

\bibitem{akm-9}
Магомедов А.М. Непрерывное учебное расписание с $m,\ m-2$ или 2 уроками у преподавателей // Дискретная математика. 2012. Т. 24. Вып. 2. С. 37--45.
\bibitem{akm-3Tanaev}
Танаев В.С., Сотсков Ю.Н., Струсевич В.А. Теория расписаний. Многостадийные системы. Москва. Наука. 1989. 328~с.

\bibitem{akm-8Seva}
Севастьянов С.В. Об интервальной раскрашиваемости ребер двудольного графа // Методы дискретного анализа. 1990. Т.50. С.61--72.

\bibitem{akm-9Kamal}
Камалян Р.Р. Интервальные раскраски полных двудольных графов и деревьев // Препринт ВЦ АН АрмССР, Ереван. 1989. 11~c.

\bibitem{akm-10Geri}
Гэри М, Джонсон Д. Вычислительные машины и труднорешаемые задачи.
М:~Мир, 1982. 416~с.

\bibitem{akm-11Aho}  Ахо А., Хопкркофт Дж., Ульман Дж. Построение и анализ вычислительных алгоритмов.
М:~Мир. 1979. 536~с.

\bibitem{akm-12Petersen} Petersen J. Die Theorie der regularen graphs // Acta Mathematica, 1891. 15. P. 193-–220.

\bibitem{akm-13Tucker} Tucker A. Chapter 2: Covering Circuits and Graph Colorings // Applied Combinatorics. 5th. Hoboken: John Wiley \& sons. 2006. P. 49.

\bibitem{akm-14Lawler}
 Lawler E.L. Combinatorial Optimization: Networks and Matroids.
New York: Holt, Rinehart and Winston. 1976.

\bibitem{akm-15Giaro} Giaro K. Compact task scheduling on dedicated processors with no waiting period (in Polish) // PhD thesis, Technical University of Gdansk, IETI Faculty, Gdansk. 1999.








\bibitem{akm-Uch}Магомедов А.М. Упражнения по компьютерной графике (учебное пособие).  Махачкала. Юнион-Сервис. 2016. 60 c.
\bibitem{akm-DEMI} Магомедов А.М. Элиминация перебора двудольных графов на 15 вершинах // Дагестанские электрические математические известия (ДЭМИ). 2016. Т. 5. С. 20--25.

\bibitem{akm-artIRJ} Магомедов А.М. О раскрашиваемости двудольных графов специального вида // International research journal (Международный научно-исследовательский журнал). 2016. №9 (51). Часть 2, сентябрь. Екатеринбург.  –- С. 128-131. Импакт-фактор (5-летний) 0.208, РИНЦ 2227 – 6017.
\bibitem{akm-artDMA}
Магомедов А.М. Цепочечные структуры в задачах о расписаниях // Прикладная дискретная математика. 2016. №3(33). C. 67-77.  ISSN 2311-2263 (Online), ISSN 2071-0410 (Print). Импакт-фактор за 2015 – 0,315.
\bibitem{akm-artDMath}
Магомедов А.М., Магомедов Т.А. Последовательное разбиение ребер двудольного графа на паросочетания // Дискретная математика. 2016. Т. 28. Выпуск 1. С. 78–86 (Mi dm1358) ринц 2305-3143, импакт-фактор Math-Net.ru 0,326.
\bibitem{akm-artScool}
Магомедов А.М., Лугуев Т.С. Задача 5480 // Математика в школе. 2016. № 8. С. 65.


\bibitem{akm-tez1MCH}
Магомедов А.М. Перечисление топологических сортировок дуг ациклического ориентированного графа // Материалы межд. научной конференции <<Мухтаровские чтения>> <<Актуальные проблемы математики и смежные вопросы>>, 8 апреля 2016г. 114 с. (с. 46-47).
\bibitem{akm-tez2MCH}
Магомедов А.М., Алибекова П.Х. Согласование списка формул // Материалы межд. научной конференции <<Мухтаровские чтения>> <<Актуальные проблемы математики и смежные вопросы>>, 8 апреля 2016г. -- C. 114 (с. 47-48).
\bibitem{akm-tez3IT}
Магомедов А.М., Алибекова П.Х. Вещественные типы Delphi в примерах // Дагестан - IT - 2016: сборник материалов II - Всероссийской научно-практической конференции / под общ. ред. профессора М.А. Сурхаева. - Махачкала: Деловой мир, 2016. 184 с. (С. 61--65).
\bibitem{akm-tez4IT}
Магомедов А.М., Алибекова П.Х. Слабо-избыточное перечисление двудольных графов специального вида // Дагестан - IT - 2016: сборник материалов II - Всероссийской научно-практической конференции / под общ. ред. профессора М.А. Сурхаева. - Махачкала: Деловой мир, 2016. – 184 с. (с. 136-138).


\bibitem{akm-prog0}
Магомедов А.М. Свидетельство № 2016612506 от 29 февраля 2016 г. о государственной регистрации программы для ЭВМ “Инициализация перечисления разбиений прямоугольной области”. Заявка № 2015663321, дата поступления 31 Декабря 2015 г., дата государственной регистрации в Реестре программ для ЭВМ 29 февраля 2016 г.

\bibitem{akm-prog00}
Магомедов А.М. Свидетельство № 2016612507 от 29 февраля 2016 г. о государственной регистрации программы для ЭВМ “Согласование взаимно-рекуррентных соотношений”. Заявка № 2015663320, дата поступления 31 декабря 2015 г., дата государственной регистрации в Реестре программ для ЭВМ 29 февраля 2016 г.

\bibitem{akm-prog1}
Магомедов А.М., Якубов А.З. Свидетельство о государственной регистрации программы для ЭВМ «Разбиение множества ребер двудольного графа на непродолжимые реберно-пересекающиеся пути» № 2016618204. Заявка № 2016615333, дата поступления 24 мая 2016 г. Дата государственной регистрации в Реестре программ для ЭВМ 22 июля 2016 г.

\bibitem{akm-prog2}
Магомедов А.М. Свидетельство о государственной регистрации программы для ЭВМ «Ретроспективный прогноз» № 2016618194. Заявка № 2016615291, дата поступления 24 мая 2016. Дата государственной регистрации в Реестре программ для ЭВМ 22 июля 2016 г.

\bibitem{akm-prog3}
Магомедов А.М. Свидетельство о государственной регистрации программы для ЭВМ «Процедура заливки фигур с ограничивающим цветом» № 2016660829. Заявка № 2016618297, дата поступления 29 июля 2016 г. Дата государственной регистрации в Реестре программ для ЭВМ 22 сентября 2016 г.


\bibitem{akm-prog4}
Магомедов А.М. Свидетельство о государственной регистрации программы для ЭВМ «Реберная раскраска двудольных графов» № 2016660830. Заявка № 2016618295, дата поступления 29 июля 2016 г. Дата государственной регистрации в Реестре программ для ЭВМ 22 сентября 2016 г.

\bibitem{akm-prog5}
Магомедов А.М. Генерация графов и их раскраска методом жадного алгоритма // представлено 20.08.2016, ожидается получение Свидетельства.

\bibitem{smsshti1}

 Шарапудинов T.И.
Конечные предельные ряды по полиномам Чебышева, ортогональным на равномерных сетках
// Изв. Сарат. ун-та. Нов. сер. Сер. Математика. Механика. Информатика, 2013. Т.~13. Вып.~1(2). C.~104–-108.

\bibitem{sms11}
Султанахмедов М.С. Аппроксимативные свойства вейвлет-рядов Чебышева второго рода  // Владикавказский математический журнал, 2015. Том 17. Вып. 3, С.~56-64.



\bibitem{sms112}
Султанахмедов М.С. Специальные вейвлеты на основе полиномов Чебышева второго рода  //
Изв. Сарат. ун-та. Нов. сер. Сер. Математика. Механика. Информатика, 2016. Т.~16. Вып.~1. C.~34–-41.


\bibitem{sms12}

 Шарапудинов И.И. Предельные ультрасферические ряды и их аппроксимативные свойства // Математические заметки, 2013. Т.~94. Вып.~2. С.~295–-309. DOI: 10.4213/mzm10292.


\bibitem{sms13}

 Шарапудинов И.И. Некоторые специальные ряды по ультрасферическим полиномам и их аппроксимативные свойства // Известия РАН. Серия математическая. Т.~78,	Вып.~5,	C.~201–-224.  DOI: 10.4213/im8117.


\end{thebibliography}
