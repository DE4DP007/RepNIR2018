\begin{thebibliography}{999}
\bibitem{Haar-Tcheb-Shar11}
 И.И. Шарапудинов. Приближение функций с переменной гладкостью суммами Фурье Лежандра // Математический сборник. 2000. Том 191. Вып. 5. С.143-160.


\bibitem{Haar-Tcheb-Shar13}
 И.И. Шарапудинов. Смешанные ряды по ортогональным полиномам // Махачкала. Издательство Дагестанского научного центра. 2004.


\bibitem{Haar-Tcheb-Shar15}
 И.И. Шарапудинов. Аппроксимативные свойства смешанных рядов по полиномам Лежандра на классах $W^r$ // Математический сборник. 2006. Том 197. Вып. 3. С. 135–154.


\bibitem{Haar-Tcheb-Shar16}
 И.И. Шарапудинов. Аппроксимативные свойства средних типа Валле-Пуссена частичных сумм смешанных рядов по полиномам Лежандра // Математические заметки. 2008. Том 84. Вып. 3. С. 452-471


\bibitem{Haar-Tcheb-Shar18}
 И.И. Шарапудинов, Т.И. Шарапудинов. Смешанные ряды по полиномам Якоби и Чебышева и их дискретизация // Математические заметки. 2010. Том 88. Вып. 1. С. 116-147.


\bibitem{Haar-Tcheb-Shar19}
 И.И. Шарапудинов,  Г.Н. Муратова. Некоторые свойства r-кратно интегрированных рядов по системе Хаара // Изв. Сарат. ун-та. Нов. сер. Сер. Математика. Механика. Информатика. 2009. Том 9. Вып. 1. С. 68-76.


\bibitem{Haar-Tcheb-IserKoch}
 A. Iserles, P.E. Koch, S.P. Norsett and J.M. Sanz-Serna. On polynomials  orthogonal  with respect  to certain Sobolev inner products // J. Approx. Theory. 1991. Vol. 65. p. 151-175.


\bibitem{Haar-Tcheb-MarcelAlfaroRezola }
 F. Marcellan, M. Alfaro and M.L. Rezola Orthogonal polynomials on Sobolev spaces: old and new directions // Journal of Computational and Applied Mathematics. 1993. Vol. 48. p. 113-131.


\bibitem{Haar-Tcheb-Meijer}
 H.G. Meijer Laguerre polynimials generalized to a certain discrete Sobolev inner product space // J. Approx. Theory. 1993. Vol. 73. p. 1-16.


\bibitem{Haar-Tcheb-KwonLittl1}
 K.H. Kwon and L.L. Littlejohn. The orthogonality of the Laguerre polynomials $\{L_n^{(-k)}(x)\}$ for positive integers $k$ // Ann. Numer. Anal. 1995. \textnumero 2. p.289-303.


\bibitem{Haar-Tcheb-KwonLittl2}
 K.H. Kwon and L.L. Littlejohn. Sobolev orthogonal polynomials and second-order differential equations // Rocky Mountain J. Math. 1998. Vol. 28. p. 547-594.


\bibitem{Haar-Tcheb-MarcelXu}
 F. Marcellan and Yuan Xu. ON SOBOLEV ORTHOGONAL POLYNOMIALS // arXiv: 6249v1 [math.C.A] 25 Mar 2014. p. 1-40.


\bibitem{Haar-Tcheb-Lopez1995}
 Lopez G. Marcellan F. Vanassche W. Relative Asymptotics for Polynomials Orthogonal with Respect to a Discrete Sobolev Inner-Product // Constr. Approx.1995. Vol. 11:1. p. 107-137.


\bibitem{Haar-Tcheb-Gonchar1975}
 А. А. Гончар. О сходимости аппроксимаций Паде для некоторых классов мероморфных функций // Математический сборник. 1975. Вып. 97(139):4(8). C. 607-629.


\bibitem{Haar-Tcheb-Tref1}
 L.N. Trefethen. Spectral methods in Matlab // Philadelphia. SIAM. 2000.


\bibitem{Haar-Tcheb-Tref2}
 L.N. Trefethen. Finite difference and spectral methods for ordinary and partial differential equation // Cornell University. 1996.


\bibitem{Haar-Tcheb-SolDmEg}
 В.В. Солодовников, А.Н. Дмитриев, Н.Д. Егупов. Спектральные методы расчета и проектирования систем управления // Москва. Машиностроение. 1986.


\bibitem{Haar-Tcheb-Pash}
 С. Пашковский. Вычислительные применения многочленов и рядов Чебышева // Москва. Наука. 1983.


\bibitem{Haar-Tcheb-Arush2010}
 О. Б. Арушанян, Н. И. Волченскова, С. Ф. Залеткин. Приближенное решение обыкновенных дифференциальных уравнений с использованием рядов Чебышева // Сиб. электрон. матем. изв. 1983. Вып. 7. С. 122-131.


\bibitem{Haar-Tcheb-Arush2013}
 О. Б. Арушанян, Н. И. Волченскова, С. Ф. Залеткин. Метод решения задачи Коши для обыкновенных дифференциальных уравнений с использованием рядов Чебышeва // Выч. мет. программирование. 2013. Вып. 14:2. С. 203-214.


\bibitem{Haar-Tcheb-Arush2014}
 О. Б. Арушанян, Н. И. Волченскова, С. Ф. Залеткин. Применение рядов Чебышева для интегрирования обыкновенных дифференциальных уравнений // Сиб. электрон. матем. изв. 2014. Вып. 11. С. 517-531.


\bibitem{Haar-Tcheb-Lukom2016}
 Д.С. Лукомский, П.А. Терехин. Применение системы Хаара к численному решению задачи Коши для линейного дифференциального уравнения первого порядка // Материалы 18-й международной Саратовской зимней школы «Современные проблемы теории функций и их приложения». Саратов. ООО «Издательство «Научная книга». 2016. С. 171-173.


\bibitem{Haar-Tcheb-MMG2016}
 М.Г. Магомед-Касумов. Приближенное решение обыкновенных дифференциальных уравнений с использованием смешанных рядов по системе Хаара // Материалы 18-й международной Саратовской зимней школы «Современные проблемы теории функций и их приложения». Саратов. ООО «Издательство «Научная книга». 2016. С. 176-178.


\bibitem{Haar-Tcheb-KashSaak}
 Б.С. Кашин, А.А. Саакян. Ортогональные ряды // Москва. АФЦ. 1999.


\bibitem{Haar-Tcheb-Sege}
 Г. Сеге. Ортогональные многочлены // Москва. Физматгиз. 1962.


\bibitem{Haar-Tcheb-Gasper}
 G. Gasper. Positivity and special function // Theory and appl.Spec.Funct. Edited by Richard A.Askey. 1975. p. 375-433.


\bibitem{Haar-Tcheb-Muckenhoupt}
 B. Muckenhoupt. Mean convergence of Jacobi series // Proc.Amer. Math. Soc. 1969. Vol. 23. \textnumero 2 p. 306-310.

%====== Section 2 =========



\bibitem{sob-jac-discrete-Shar12}
 И.И. Шарапудинов, Аппроксимативные свойства операторов $\mathcal{Y}_{n+2r}(f)$ и их дискретных аналогов // Математические заметки, 2002, т. 72, вып. 5, стр. 765-795.


\bibitem{sob-jac-discrete-Shar17}
 И.И. Шарапудинов, Смешанные ряды по ультрасферическим полиномам // Математический сборник, 2003, т. 194, вып. 3, стр. 115-148.





\bibitem{sob-jac-discrete-Sege}
 Г. Сеге, Ортогональные многочлены. Москва: Физматгиз, 1962.


\bibitem{sob-jac-discrete-Timan}
 А.Ф. Тиман, Теория приближения функций действительного переменного. Москва: Физматгиз, 1962.


\bibitem{sob-jac-discrete-Tel}
 С.А.Теляковский, Две теоремы о приближении функций
алгебраическими многочленами // Математический сборник, 1966, т. 70, вып. 2, стр. 252-265.


\bibitem{sob-jac-discrete-Gop}
 И.З. Гопенгауз, К теореме А.Ф.Тимана о приближении
функций многочленами на конечном отрезке // Математические заметки, 1967, т. 1, вып. 2, стр. 163-172.


\bibitem{meixner-1}
 Iserles~A., Koch~P.~E., Norsett~S.~P. and Sanz-Serna~J.~M. On polynomials orthogonal with respect to certain Sobolev inner products, // J. Approx. Theory., 1991, vol. 65, pp. 151-175.


\bibitem{meixner-2}
 Marcellan~F., Alfaro~M. and Rezola~M.~L. Orthogonal polynomials on Sobolev spaces: old and new directions. // North-Holland, 1993, vol. 48, 25 Mar 2014, pp. 113-131.


\bibitem{meixner-3}
 Meijer~H.~G. Laguerre polynimials generalized to a certain discrete Sobolev inner product space // J. Approx. Theory, 1993, vol. 73, pp. 1-16.


\bibitem{meixner-4}
 Kwon~K.~H.  and Littlejohn~L.~L. The orthogonality of the Laguerre polynomials $\{L_n^{(-k)}(x)\}$ for positive integers $k$ // Ann. Numer. Anal., 1995, vol. 2, pp. 289-303.


\bibitem{meixner-5}
 Kwon~K.~H. and Littlejohn~L.~L. Sobolev orthogonal polynomials and second-order differential equations // Ann. Numer. Anal., 1998, vol. 28, pp. 547-594.


\bibitem{meixner-6}
 Marcellan~F. and Yuan Xu ON SOBOLEV ORTHOGONAL POLYNOMIALS // arXiv: 6249v1 [math.C.A] 25 Mar 2014, pp. 1-40.


\bibitem{meixner-7}
 Шарапудинов~И.~И. Приближение дискретных функций и многочлены Чебышева, ортогональные на равномерной сетке // Мат. заметки, 2000, т. 67, №~3, С.~460-470. DOI: 10.4213/mzm858


\bibitem{meixner-8}
 Шарапудинов~И.~И. Приближение функций с переменной гладкостью суммами Фурье- Лежандра // Мат. сборник, 2000, т. 191, №~5, С.~143-160. DOI: 10.4213/sm480


\bibitem{meixner-9}
 Шарапудинов~И.~И. Аппроксимативные свойства операторов $\mathcal{Y}_{n+2r}(f)$ и их дискретных аналогов // Мат. заметки, 2002, т. 72, №~5, С.~765-795. DOI: 10.4213/mzm466


\bibitem{meixner-10}
 Шарапудинов~И.~И. Смешанные ряды по ультрасферическим полиномам и их аппроксимативные свойства // Мат. сборник, 2003, т. 194, №~3, С.~115-148. DOI: 10.4213/sm723


\bibitem{meixner-11}
 Шарапудинов~И.~И. Смешанные ряды по ортогональным полиномам. Махачкала: Дагестан. науч. центр РАН, 2004. С.~1-176.


\bibitem{meixner-12}
 Шарапудинов~И.~И. Смешанные ряды по полиномам Чебышева, ортогональным на равномерной сетке // Мат. заметки, 2005, т. 78, №~3, C.~442-465. DOI: 10.4213/mzm2599


\bibitem{meixner-13}
 Шарапудинов~И.~И. Аппроксимативные свойства смешанных рядов по полиномам Лежандра на классах $W^r$ // Математический сборник, 2006, т. 197, №~3, C.~135-154. DOI: 10.4213/sm1539


\bibitem{meixner-14}
 Шарапудинов~Т.~И. Аппроксимативные свойства смешанных рядов по полиномам Чебышева, ортогональным на равномерной сетке // Вестник Дагестанского научного центра РАН, 2007, т. 29, C.~12–23.


\bibitem{meixner-15}
 Шарапудинов~И.~И. Аппроксимативные свойства средних типа Валле-Пуссена частичных сумм смешанных рядов по полиномам Лежандра // Мат. заметки, 2008, т. 84, №~3, C.~452-471. DOI: 10.4213/mzm5541


\bibitem{meixner-16}
 Шарапудинов~И.~И., Муратова~Г.~Н. Некоторые свойства r-кратно интегрированных рядов по системе Хаара // Изв. Сарат. ун-та. Нов. Сер. Математика. Механика. Информатика, 2009, т. 9, №~1, C.~68-76.


\bibitem{meixner-17}
 Шарапудинов~И.~И., Шарапудинов~Т.~И. Смешанные ряды по полиномам Якоби и Чебышева и их дискретизация // Мат. заметки, 2010, т. 88, №~1, C.~116-147. DOI: 10.4213/mzm6607


\bibitem{meixner-18}
 Шарапудинов~И.~И. Системы функций, ортогональных по Соболеву, порожденные ортогональными функциями // Современные проблемы теории функций и их приложения.  Материалы 18-й международной Саратовской зимней школы, 2016, C.329-332.


\bibitem{meixner-19}
 Trefethen~L.~N. Spectral methods in Matlab. Fhiladelphia: SIAM, 2000.


\bibitem{meixner-20}
 Trefethen~L.~N. Finite difference and spectral methods for ordinary and partial differential equation. Cornell University, 1996.


\bibitem{meixner-21}
 Магомед-Касумов~Р.~Г. Приближенное решение обыкновенных дифференциальных уравнений с использованием смешанных рядов по системе Хаара // Современные проблемы теории функций и их приложения.  Материалы 18-й международной Саратовской зимней школы, 2016, C.176-178.


\bibitem{meixner-22}
 Шарапудинов~И.~И. Многочлены, ортогональные на дискретных сетках. Издательство Даг. гос. пед. ун-та. Махачкала, 1997.


\bibitem{meixner-23}
 Gasper~G. Positiviti and special function // Theory and appl.Spec.Funct. Edited by Richard A.Askey, 1975, 375-433.




\bibitem{sob-tcheb-difference-IserKoch}

A. Iserles, P.E. Koch, S.P. Norsett and J.M. Sanz-Serna,
On polynomials  orthogonal  with respect  to certain Sobolev inner products,
J. Approx. Theory, 65, 1991, pp. 151-175.




\bibitem{sob-tcheb-difference-MarcelAlfaroRezola}

F. Marcellan, M. Alfaro and M.L. Rezola,
Orthogonal polynomials on Sobolev spaces: old and new directions,
Journal of Computational and Applied Mathematics, North-Holland, vol. 48, 1993, pp. 113 -- 131.




\bibitem{sob-tcheb-difference-Meijer}

H.G. Meijer,
Laguerre polynimials generalized to a certain discrete Sobolev inner product space,
J. Approx. Theory, vol. 73, 1993, pp. 1-16.



\bibitem{sob-tcheb-difference-KwonLittl1}

K.H. Kwon and L.L. Littlejohn,
The orthogonality of the Laguerre polynomials $\{L_n^{(-k)}(x)\}$ for positive integers $k$,
Ann. Numer. Anal., № 2, 1995, pp. 289 -- 303.


\bibitem{sob-tcheb-difference-KwonLittl2}

 K.H. Kwon and L.L. Littlejohn,
Sobolev orthogonal polynomials and second-order differential equations, Rocky Mountain J. Math.,
vol. 28, 1998, pp. 547 –- 594.


\bibitem{sob-tcheb-difference-MarcelXu}

F. Marcellan and Yuan Xu,
On Sobolev orthogonal polynomials,
arXiv: 6249v1 [math.C.A] 25 Mar 2014, 2014, pp. 1-40.


\bibitem{sob-tcheb-difference-Shar9}

 И.И. Шарапудинов,
Приближение дискретных функций и многочлены Чебышева, ортогональные на равномерной сетке,
Математические заметки, т. 67, \textnumero 3, 2000, с. 460 -- 470.


\bibitem{sob-tcheb-difference-Shar1}

И.И. Шарапудинов,
Приближение функций с переменной гладкостью суммами Фурье Лежандра,
Математический сборник, т. 191, \textnumero 5, 2000, с. 143 -- 160.



\bibitem{sob-tcheb-difference-Shar2}

И.И. Шарапудинов,
Аппроксимативные свойства операторов $\mathcal{Y}_{n+2r}(f)$ и их дискретных аналогов,
Математические заметки, т. 72, \textnumero 5, 2002, с. 765 -- 795.


\bibitem{sob-tcheb-difference-Shar3}

И.И. Шарапудинов,
Смешанные ряды по ультрасферическим полиномам и их аппроксимативные свойства,
Математический сборник, т. 194, \textnumero 3, 2003, с. 115 -- 148.


\bibitem{sob-tcheb-difference-Shar4}

И.И. Шарапудинов,
Смешанные ряды по ортогональным полиномам,
Издательство Дагестанского научного центра, Махачкала, 2004, с. 1 -- 176.



\bibitem{sob-tcheb-difference-Shar11}

И.И. Шарапудинов,
Смешанные ряды по полиномам Чебышева, ортогональным на равномерной сетке,
Математические заметки, т. 78, \textnumero 3, 2005, с. 442 -- 465.



\bibitem{sob-tcheb-difference-Shar5}

И.И. Шарапудинов,
Аппроксимативные свойства смешанных рядов по полиномам Лежандра на классах $W^r$,
Математический сборник, т. 197, \textnumero 3, 2006, с. 135 -- 154.


\bibitem{sob-tcheb-difference-Shar6}

И.И. Шарапудинов,
Аппроксимативные свойства средних типа Валле-Пуссена частичных сумм смешанных рядов по полиномам Лежандра,
Математические заметки, т. 84, \textnumero 3, 2008, с. 452 -- 471.


\bibitem{sob-tcheb-difference-Shar7}

 И.И. Шарапудинов,  Г. Н. Муратова,
Некоторые свойства r-кратно интегрированных рядов по системе Хаара,
Изв. Сарат. ун-та. Нов. сер. Сер. Математика. Механика. Информатика, vol. 9, \textnumero 1, 2009, с. 68 -- 76.


\bibitem{sob-tcheb-difference-Shar8}

И.И. Шарапудинов, Т. И. Шарапудинов,
Смешанные ряды по полиномам Якоби и Чебышева и их дискретизация,
Математические заметки, т. 88, \textnumero 1, 2010, с. 116 -- 147.



\bibitem{sob-tcheb-difference-SharII}

И.И. Шарапудинов,
Системы функций, ортогональных по Соболеву, порожденные ортогональными функциями,
Современные проблемы теории функций и их приложения.  Материалы 18-й международной Саратовской зимней школы,
2016, с. 329 -- 332.


\bibitem{sob-tcheb-difference-Cheb1}

П.Л. Чебышев,
О непрерывных дробях, Полн.собр.соч., Изд.АН СССР, Москва, т. 2, 1947, с. 103 -- 126.


\bibitem{sob-tcheb-difference-Cheb2}

П.Л. Чебышев,
Об одном новом ряде, Полн.собр.соч., Изд.АН СССР, Москва, т. 2, 1947, с. 236 -- 238.


\bibitem{sob-tcheb-difference-Cheb3}

П.Л. Чебышев,
Об интерполировании по способу наименьших квадратов,
Полн.собр.соч., Изд.АН СССР, Москва, т. 2, 1947, с. 314 -- 334.


\bibitem{sob-tcheb-difference-Cheb4}

П.Л. Чебышев,
Об интерполировании, Полн.собр.соч., Изд.АН СССР, Москва, т. 2, 1947, с. 357 -- 374.


\bibitem{sob-tcheb-difference-Cheb5}

П.Л. Чебышев,
Об интерполировании величин  равноотстоящих (1875),
Полн.собр.соч., Изд.АН СССР, Москва, т. 2, 1947, с. 66 -- 87.


\bibitem{sob-tcheb-difference-SolDmEg}

В.В. Солодовников, А.Н. Дмитриев, Н.Д. Егупов,
Спектральные методы расчета и проектирования систем управления, 1986, Машиностроение, Москва.



\bibitem{sob-tcheb-difference-Tref1}

L.N. Trefethen,
Spectral methods in Matlab, 2000, SIAM, Fhiladelphia.


\bibitem{sob-tcheb-difference-Tref2}

L.N. Trefethen,
Finite difference and spectral methods for ordinary and partial differential equation,
1996, Cornell University.


\bibitem{sob-tcheb-difference-Shar16}

И.И. Шарапудинов,
Асимптотические свойства и весовые оценки для ортогональных многочленов Чебышева–Хана,
// Математический сборник, т. 182, \textnumero 3, 1991, с. 408 -- 420.


\bibitem{ogrValle}

И.И. Шарапудинов,
Об ограниченности в $C[-1,1]$ средних Валле-Пуссена для дискретных сумм Фурье–Чебышева,
// Математический сборник, т. 187, \textnumero 1, 1996, с. 143 -- 160.



\bibitem{sob-tcheb-difference-Shar17}

И.И. Шарапудинов,
Об асимптотике многочленов Чебышева, ортогональных на конечной системе точек,
Вестник МГУ. Серия 1, т. 1, 1992, с. 29 -- 35.


\bibitem{sob-tcheb-difference-Shar18}

И.И. Шарапудинов,
Многочлены, ортогональные на дискретных сетках,
Издательство Даг. гос. пед. ун-та. Махачкала, 1997.


\bibitem{sob-tcheb-difference-MagKas}

М.Г. Магомед-Касумов,
Приближенное решение обыкновенных дифференциальных уравнений с использованием смешанных рядов по системе Хаара,
Современные проблемы теории функций и их приложения. Материалы 18-й международной Саратовской зимней школы.  27 января -- 3 февраля 2016, c. 176 -- 178.


\bibitem{sob-lag-sb-Meijer}

 H.G. Meijer, Laguerre polynimials generalized to a certain,  Laguerre polynimials generalized to a certain discrete Sobolev inner product space // J. Approx. Theory, 1993, vol. 73, pp. 1-16.


\bibitem{sob-lag-sb-Shar14}
 И.И. Шарапудинов, Смешанные ряды по полиномам Чебышева, ортогональным на равномерной сетке // Математические заметки, 2005, т. 78, вып. 3, стр. 442-465.


\bibitem{sob-lag-sb-Sege}
 Г. Сеге, Ортогональные многочлены. Москва:Физматгиз. 1962.


\bibitem{sob-lag-sb-AskeyWaiger}
 R. Askey, S. Wainger, Mean convergence of expansions in Laguerre and Hermite series // Amer. J. Mathem., 1965, vol. 87, pp. 698-708.

%
%
%\bibitem{Ram1}

%Area~I., Godoy~E., Marcellan~F. Inner products involving differences: the Meixner--Sobolev polynomials // J. Difference Equations Appl. 6 (2000), pp. 1-31.
%
%
%\bibitem{Ram2}

%Marcellan~F., Xu~Y. On Sobolev orthogonal polynomials // Expositiones Mathematicae, 33, №~3, 2015, pp.308-352.
%
%
%\bibitem{Ram3}

%Fernandez~L., Teresa E. Perez, Miguel A. Pinar, Xu~Y. Weighted Sobolev orthogonal polynomials on the unit ball // Journal of Approximation Theory, 171, 2013, pp. 84–104.
%
%
%\bibitem{Ram4}

%Antonia M. Delgado, Fernandez~L., Doron S. Lubinsky, Teresa E. Perez, Miguel A. Pinar. Sobolev orthogonal polynomials on the unit ball via outward normal derivatives // Journal of Mathematical Analysis and Applications, 440, №~2, 2016, pp. 716–740.
%
%
%\bibitem{Ram5}

%Fernandez~L., Marcellan~F., Teresa E. Perez, Miguel A. Pinar, Xu~Y. Sobolev orthogonal polynomials on product domains // Journal of Computational and Applied Mathematics, 284, 2015, pp. 202–215.
%
%
%\bibitem{Ram6}

%Lopez~G., Marcellan~F., Vanassche~W. Relative asymptotics for polynomials orthogonal with respect to a discrete Sobolev inner-product // Constr. Approx., 11, №~1, 1995, pp. 107-137.
%
%
%\bibitem{Ram7}

%Iserles~A., Koch~P.~E., Norsett~S.~P., Sanz-Serna~J.~M. On polynomials  orthogonal  with respect  to certain Sobolev inner products,J. Approx. Theory. 65 (1991), pp. 151--175.
%
%
%\bibitem{Ram8}

%Marcellan~F., Alfaro~M., Rezola~M.L. Orthogonal polynomials on Sobolev spaces: old and new directions. North-Holland, V. 48, 1993, 25 Mar 2014, pp. 113--131.
%
%
%\bibitem{Ram9}

%Meijer~H.G. Laguerre polynimials generalized to a certain discrete Sobolev inner product space // J. Approx. Theory, 73, 1993, pp. 1--16.
%
%
%\bibitem{Ram10}

%Kwon~K.~H., Littlejohn~L.L. The orthogonality of the Laguerre polynomials $\{L_n^{(-k)}(x)\}$ for positive integers $k$ // Ann. Numer. Anal. 2, 1995. Pp. 289--303.
%
%
%\bibitem{Ram11}

%Kwon~K.H., Littlejohn~L.L.  Sobolev orthogonal polynomials and second-order differential equations // Ann. Numer. Anal. 28, 1998. Pp. 547--594.
%
%
%\bibitem{Ram12}

%Шарапудинов~И.И. Аппроксимативные свойства операторов $Y_{n+2r}(f)$ и их дискретных аналогов // Мат. заметки, 72, №~5, 2002, C.~765-795. DOI: 10.4213/mzm466
%
%
%\bibitem{Ram13}

%Шарапудинов~И.И. Смешанные ряды по ультрасферическим полиномам и их аппроксимативные свойства // Матем. сб., 194:3 (2003), С.~115--148. DOI: 10.4213/sm723


\bibitem{Ram14}

Гаджиева~З.~Д. Смешанные ряды по полиномам Мейкснера. Кандидатская диссертация - Саратов. Саратовский гос. ун-т. 2004.


%\bibitem{Ram16}

%Шарапудинов~И.И. Смешанные ряды по полиномам Чебышева, ортогональным на равномерной сетке // Мат. заметки, 78, №~3, 2005, C.~442-465. DOI: 10.4213/mzm2599
%
%
%\bibitem{Ram18}

%Шарапудинов~И.И. Аппроксимативные свойства смешанных рядов по полиномам Лежандра на классах $W^r$ //   Математический сборник, 197, №~3, 2006, C.~135--154. DOI: 10.4213/sm1539
%
%
%\bibitem{Ram19}

%Шарапудинов~И.И. Аппроксимативные свойства средних типа Валле-Пуссена частичных сумм смешанных рядов по полиномам Лежандра // Мат. заметки, 84, №~3, 2008, C.~452-471. DOI: 10.4213/mzm5541
%
%
%\bibitem{Ram20}

%Шарапудинов~И.И., Гаджиева~З.Д. Полиномы, ортогональные по Соболеву, порожденные многочленами Мейкснера, Изв. Сарат. ун-та. Нов. сер. Сер. Математика. Механика. Информатика {16}(3), (2016), C.~310--321.



\bibitem{sob-leg-MarcelXu}

F. Marcellan and Yuan Xu. On Sobolev orthogonal polynomials // Expositiones Mathematicae, 2015. Vol. 33. Issue 3. P. 308-352

\bibitem{sob-leg-Shar2016}
 И.И. Шарапудинов. Системы функций, ортогональные по Соболеву, порожденные ортогональными функциями // Материалы 18-й международной Саратовской зимней школы «Современные проблемы теории функций и их приложения». Саратов: ООО «Издательство «Научная книга». 2016. С. 329-332.

\bibitem{sob-leg-Tref1}
 L.N. Trefethen. Spectral methods in Matlab. Fhiladelphia: SIAM. 2000.

\bibitem{sob-leg-SolDmEg}
 В.В. Солодовников, А.Н. Дмитриев, Н.Д. Егупов. Спектральные методы расчета и проектирования систем управления. Москва: Машиностроение. 1986.

\bibitem{sob-leg-Pash}
 С. Пашковский // Вычислительные применения многочленов и рядов Чебышева. Москва: Наука. 1983. С. 143-160.

\bibitem{sob-leg-Shar17}
 И.И. Шарапудинов. Смешанные ряды по ультрасферическим полиномам и их аппроксимативные свойства // Математический сборник, 2003. Т. 194. Вып. 3. С. 115-148.

\bibitem{sob-leg-Gasper}
 G. Gasper. Positiviti and special function // Theory and appl.Spec.Funct. Edited by Richard A.Askey, 1975. P. 375-433.

\bibitem{sob-leg-SHII}
 И.И. Шарапудинов. Некоторые специальные ряды по общим полиномам Лагерра и ряды Фурье по полиномам Лагерра, ортогональным по Соболеву // Дагестанские электронные математические известия, 2015. Т. 4.

\bibitem{sob-leg-TEL}
 С. А. Теляковский. Две теоремы о приближении функций алгебраическими многочленами // Математический сборник, 1966. Т. 70. Вып. 2. С. 252-265.

\bibitem{sob-leg-GOP}
 И. З. Гопенгауз. К теореме А. Ф. Тимана о приближении функций многочленами на конечном отрезке // Математические  заметки, 1967. Т. 1. Вып. 2. С. 163-172.

\bibitem{sob-leg-OSK}
 К. И. Осколков. К неравенству Лебега в равномерной метрике и на множестве полной меры // Математические  заметки, 1975. Т. 18. Вып. 4. С. 515-526.

\bibitem{sob-leg-sharap1}
 I.I. Sharapudinov. On the best approximation and polinomial of the least quadratic deviation // Analysis Mathematica, 1983. Vol. 9. Issue 3. P. 223-234.

\bibitem{sob-leg-sharap2}
 И.И. Шарапудинов. О наилучшем приближении и суммах Фурье-Якоби // Математические заметки, 1983. Т. 34. Вып. 5. С. 651-661.

\bibitem{sob-leg-sharap3}
 И.И. Шарапудинов. Некоторые специальные ряды по ультрасферическим полиномам и их аппроксимативные свойства // Изв. РАН. Сер. матем., 1983. Т. 78. Вып. 5. С. 201-224.

\bibitem{ark-1}
  Дж.~Алберг, Э.~Нилсон, Дж.~Уолш. Теория сплайнов и ее приложения  М.: Мир. 1972.


\bibitem{ark-2}
  С.Б.~Стечкин, Ю.Н.~Субботин. Добавления к книге Дж.~Алберг, Э.~Нилсон, Дж.~Уолш.
 Теория сплайнов и ее приложения.  М.: Мир. 1972.


\bibitem{ark-3}
 С.Б.~Стечкин, Ю.Н.~Субботин.  Сплайны в вычислительной математике. М.:  Наука. 1976.


\bibitem{ark-4}
  Н.П.~Корнейчук.  Сплайны в теории приближения. М.:  Наука. 1984.



\bibitem{ark-5}
  В.Н.~Малоземов, А.Б.~Певный. Полиномиальные сплайны. Л.: Изд-во ЛГУ. 1986.


\bibitem{ark-6}
 Ю.С.~Завьялов, Б.И.~Квасов, В.Л.~Мирошниченко.  Методы сплайн--функций.  М.: Наука. 1980.



\bibitem{ark-7}
 R.~Schaback.  Spezielle rationale Splinefunktionen //  J. Approx. Theory. Academic Press, Inc. 1973. V.7:2.
P. 281--292.


\bibitem{ark-8}
  Q.~Duan, K.~Djidjeli, W.G.~Price, E.H.~Twizell.  Weighted rational  cubic spline interpolation
 and its application // J. of Computational  and Applied Mathematics. 2000. V. 117:2. P. 121--135.


\bibitem{ark-9}
  M.Z.~Hussain, M.~Sarfraz, T.S.~Shaikh.  Shape preserving rational cubic spline for positive
and convex data //  Egyptian Informatics Journal. 2011. V.  12. P. 231--236.


\bibitem{ark-10}
  A.~Edeoa, G.~Gofeb, T.~Tefera.  Shape preserving $C^2$ rational cubic spline interpolation //
 American Scientific Research Journal for Engineering, Technology and Sciences. 2015. V. 12:1. P. 110--122.


\bibitem{ark-11}
 А.-Р.К.~Рамазанов, В.Г.~Магомедова. Сплайны по рациональным интерполянтам // Дагестанские
электронные математические известия. 2015. Вып. 4. С. 22-31.


\bibitem{ark-12}
 А.-Р.К.~Рамазанов, В.Г.~Магомедова. Интерполяционные рациональные сплайны // Вестник ДГУ. Серия 1. Естеств. науки. 2016.
Т. 31. Вып. 2. С. 35--40.


\bibitem{ark-13}
 А.-Р.К.~Рамазанов, В.Г.~Магомедова. Сплайны по четырехточечным рациональным интерполянтам //
Труды Института матем. и механики УрО РАН. 2016. Т. 22, \No 4. С. 233-246.


\bibitem{sob-lag-smj-Pash}
 С. Пашковский. Вычислительные применения многочленов и рядов Чебышева: Пер. с польск. // Москва, Наука, 1983

\bibitem{sob-lag-smj-fiht2}
 Г.М. Фихтенгольц. Курс дифференциального и интегрального исчисления, т. 2 // Москва, Физматлит, 2001, стр. 810.



\bibitem{sms1}
 Meyer Y. Ondelettes et Operateurs // Paris. Hermann. 1990. Vol.~I–III.


\bibitem{sms2}
 Daubechies L. Ten Lectures on Wavelets // CBMS-NSF Regional Conference Series in Applied Mathematics Proceedings,
Vol. 61. Philadelphia, PA: SIAM, 1992, 357~p. ISBN 0-89871-274-2. DOI: 10.1137/1.9781611970104.


\bibitem{sms3}
 Chui C.K. An Introduction to  Wavelets // Boston. Academic Press. 1992. 271~p. ISBN 0-12-174584-8.


\bibitem{sms4}
 Chuii C.K. Mhaskar H.N. On Trigonometric Wavelets // Constructive Approximation, 1993. Vol.~9. Issue~2--3. pp.~167--190. DOI: 10.1007/BF01198002.


\bibitem{sms5}
 Kilgore T., Prestin J. Polynomial wavelets on an interval // Constructive Approximation, 1996. Vol.~12, Issue~1. pp.~95--110. DOI: 10.1007/BF02432856.


\bibitem{sms6}
 Fischer B., Prestin J. Wavelet based on orthogonal polynomials // Mathematics of computation, 1997. Vol.~66, Num.~220. pp.~1593--1618. DOI: 10.1090/S0025-5718-97-00876-4.


\bibitem{sms7}
 Fischer B., Themistoclakis W. Orthogonal polynomial wavelets // Numerical Algorithms, 2002. Vol.~30, Issue~1. pp.~37--58. DOI: 10.1023/A:1015689418605.


\bibitem{sms8}
 Capobiancho M.R., Themistoclakis W. Interpolating polynomial wavelet on $[-1,1]$ // Advanced in Computational Mathematics, 2005. Vol.~23, Issue~4. pp.~353–-374. DOI: 10.1007/s10444-004-1828-2.


\bibitem{sms9}
 Dao-Qing Dai, Wei Lin. Orthonormal polynomial wavelets on the interval // Proceedings of the American Mathematical Society, 2005. Vol.~134, Issue~5. pp.~1383–1390. DOI: 10.1090/S0002-9939-05-08088-3.


\bibitem{sms10}
 Mohd F., Mohd I. Orthogonal Functions Based on Chebyshev Polynomials // Matematika, 2011. Vol.~27, Num.~1, pp.~97-–107.


\bibitem{sms11}
 Султанахмедов М.С. Аппроксимативные свойства вейвлет-рядов Чебышева второго рода  // Владикавказский математический журнал, 2015. Том 17. Вып. 3, С.~56-64.


\bibitem{sms112}
 Султанахмедов М.С. Специальные вейвлеты на основе полиномов Чебышева второго рода  //
Изв. Сарат. ун-та. Нов. сер. Сер. Математика. Механика. Информатика, 2016. Т.~16. Вып.~1. C.~34–-41.



\bibitem{sms12}
 Шарапудинов И.И. Предельные ультрасферические ряды и их аппроксимативные свойства // Математические заметки, 2013. Т.~94. Вып.~2. С.~295–-309. DOI: 10.4213/mzm10292.


\bibitem{smsshti1}
 Шарапудинов T.И.
Конечные предельные ряды по полиномам Чебышева, ортогональным на равномерных сетках
// Изв. Сарат. ун-та. Нов. сер. Сер. Математика. Механика. Информатика, 2013. Т.~13. Вып.~1(2). C.~104–-108.



\bibitem{sms14}
 Яхнин Б.М. О функциях Лебега разложений в ряды по полиномам Якоби для случаев $\alpha=\beta=\frac12, \alpha=\beta=-\frac12, \alpha=\frac12, \beta=-\frac12$ // Успехи математических наук, 1958. Т.~13. Вып.~6(84). C.~207-–211.


\bibitem{sms15}
 Яхнин Б.М. Приближение функций класса $Lip_\alpha$ частными суммами ряда Фурье по многочленам Чебышева второго рода // Изв. вузов. Матем., 1963. Вып.~1, C.~172-–178.


\bibitem{sms17}
 Тиман А.Ф. Теория приближения функций действительного переменного // Москва. Физматгиз. 1960. 626~с.


\bibitem{sms18}
 Шарапудинов И.И. О наилучшем приближении и суммах Фурье–Якоби // Математические заметки, 1983. Т.~34. Вып.~5. С.~651-–661. DOI: 10.1007/BF01157445.


\bibitem{ramazanov-2}
  С.\,Б.~Стечкин, Ю.\,Н.~Субботин. Добавления к книге Дж.~Алберг, Э.~Нилсон, Дж.~Уолш.
 Теория сплайнов и ее приложения.  М.: Мир. 1972.


\bibitem{ramazanov-3}
 С.\,Б.~Стечкин, Ю.\,Н.~Субботин.  Сплайны в вычислительной математике. М.:  Наука. 1976.


\bibitem{ramazanov-4}
  Н.\,П.~Корнейчук.  Сплайны в теории приближения. М.:  Наука. 1984.



\bibitem{ramazanov-6}
 Ю.\,С.~Завьялов, Б.\,И.~Квасов, В.\,Л.~Мирошниченко.  Методы сплайн--функций.  М.: Наука. 1980.



\bibitem{fiht_diff_v3}
Г.М. Фихтенгольц. Курс дифференциального и интегрального исчисления, т. 3. Москва: Физматлит, 2001. С. 662.

\end{thebibliography}
