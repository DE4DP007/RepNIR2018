\Conclusion

В отчетном году продолжены исследования эллиптических уравнений второго порядка. Для недивергентных эллиптических уравнений второго порядка, коэффициенты которых локально периодичны (с малым периодом) по одной из переменных выведены усредненные уравнения.

В задаче Римана -- Гильберта для системы уравнений Бельтрами доказано свойство гельдеровости решения, а для обобщенных уравнений Бельтрами получены оценки усреднения решения.

В случае уравнения Бельтрами с периодическим коэффициентом, зависящим от малого параметра, построено асимптотическое разложение решения задачи Римана – Гильберта и оценена невязка.


За отчетный период были исследованы вопросы глобальной экспоненциональной $p$-устой\-чи\-вос\-ти $(2 \le p < \infty)$ систем линейных дифференциальных уравнений Ито с запаздываниями
специального вида, используя теорию положительно обратимых матриц.
Кроме того, были исследованы вопросы асимптотической
$p$-устойчивости ($2 \le p < \infty $) тривиального решения
относительно начальных данных для линейной однородной импульсной
системы дифференциальных уравнений Ито с линейными запаздываниями
методом вспомогательных или модельных уравнений. Получены достаточные условия устойчивости в терминах параметров исследуемых систем. Результаты исследований опубликованы в работах \cite{kad11,kad12,kad13,kad14}.

Для задачи Дирихле для нелинейного дифференциального уравнения с $p(x)$-лапласиа\-ном конструируются верхнее и нижнее решения путем склеивания на двух участках.
Построенные верхнее и нижнее решения позволяют не только обосновать существование слабого решения, но и оценить решение сверху и  снизу.




