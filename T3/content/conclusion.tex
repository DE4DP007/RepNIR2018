\Conclusion



%%%%%%%%%%%%%%%%%%%%%%%
% Сиражудинов

В отчетном году сотрудниками ОМИ продолжены исследования вопросов $G$--схо-\linebreak димости %некоторых классов 
эллиптических дифференциальных операторов. %второго порядка на плоскости с комплекснозначными коэффициентами.
Установлена $G$-компактность некоторых классов недивергентных эллиптических операторов второго порядка с комплекснозначными коэффициентами.

Кроме того, продолжается изучение задач усреднения эллиптических операторов. 
В литературе известны два метода исследования различных аспектов усреднения недивергентных эллиптических операторов: метод интегрального тождества и метод асимптотических разложений. Эти методы различаются по математической технике и опираются на оценки типа оценок острого угла. При изучении скорости сходимости решения исходной задачи (зависящей от малого параметра $\varepsilon$) к решению усредненной задачи, предпочтительней второй метод.
В отчетном году асимптотическими методами получены оценки погрешности усреднения порядка $O(\varepsilon)$, где $\varepsilon$ -- малый параметр, периодической 
для недивергентных эллиптических операторов второго порядка и периодической задачи для уравнения Бельтрами.

%%%%%%%%%%%%%%%%%%%%%%%
% Кадиев

Рассматривались вопросы моментной устойчивости решений относительно начальных данных для линейных дифференциальных уравнений Ито с последействием высоких порядков. Исследование проводится методом модельных уравнений с использованием теории положительно обратимых матриц. Получены достаточные условия устойчивости в терминах параметров исследуемых систем.

Исследовалась $p$-устойчивость $(2 \le p < \infty)$ систем линейных дифференциальных уравнений Ито с запаздываниями специального вида, используя теорию положительно обратимых матриц. Для этого применяется идеи и методы, разработанная Н.В. Азбелевым и его учениками для исследования вопросов устойчивости для детерминированных функционально-дифференциальных уравнений. Получены достаточные условия $p$-устойчивости $(2 \le p < \infty)$ исследуемых систем в терминах положительной обратимости матрица, построенной по исходной системе. Проверена  выполнимость этих условий для конкретных уравнений.


%%%%%%%%%%%%%%%%%%%%%%%
% АЭИ

Изучались вопросы 
единственности положительного радиально-симметричного решения задачи Дирихле в кольцевой
области. 
Для одного класса нелинейных дифференциальных уравнений второго порядка доказано существование и единственность такого решения. 
Разработан численный метод его построения.









%%%%%%%%%%%%%%%%%%%%%%%
% Меджидов

В области восстановления функций по неполным данным решена задача восстановления функции, сосредоточенной в полосе на плоскости, заданной своими интегралами с полиномиальными весами вдоль ветвей гипербол одного семейства. Неизвестная функция при этом предполагается дважды непрерывно дифференцируемой, а ее поиск сводится к решению интегрального уравнения типа Вольтерра относительно преобразования Фурье для нее.

Для так называемого $V$-преобразования Радона получена формула обращения интегрального преобразования функции на семействе ломаных в круге. $V$-преобразова-\linebreak ние используется в практических задачах для моделирования ослабления интенсивности луча, падающего под некоторым углом на границу круга после отражения с тем же углом.

Ещё одна формула получена для определения функции, заданной интегралами на одном двухпараметрическом семействе ломаных в круге.  Угол падения луча может меняться в сколь угодно малом угловом диапазоне.












%==========================
%%%%%%%%%%%%%%%%%%%%%%%%
%% OLDS BELOW
%
%
%В отчетном году продолжены исследования эллиптических уравнений второго порядка. Для недивергентных эллиптических уравнений второго порядка, коэффициенты которых локально периодичны (с малым периодом) по одной из переменных выведены усредненные уравнения.

%В задаче Римана -- Гильберта для системы уравнений Бельтрами доказано свойство гельдеровости решения, а для обобщенных уравнений Бельтрами получены оценки усреднения решения.
%
%В случае уравнения Бельтрами с периодическим коэффициентом, зависящим от малого параметра, построено асимптотическое разложение решения задачи Римана – Гильберта и оценена невязка.
%
%
%За отчетный период были исследованы вопросы глобальной экспоненциональной $p$-устой\-чи\-вос\-ти $(2 \le p < \infty)$ систем линейных дифференциальных уравнений Ито с запаздываниями
%специального вида, используя теорию положительно обратимых матриц.
%Кроме того, были исследованы вопросы асимптотической
%$p$-устойчивости ($2 \le p < \infty $) тривиального решения
%относительно начальных данных для линейной однородной импульсной
%системы дифференциальных уравнений Ито с линейными запаздываниями
%методом вспомогательных или модельных уравнений. Получены достаточные условия устойчивости в терминах параметров исследуемых систем. Результаты исследований опубликованы в работах \cite{kad11,kad12,kad13,kad14}.
%
%Для задачи Дирихле для нелинейного дифференциального уравнения с $p(x)$-лапласиа\-ном конструируются верхнее и нижнее решения путем склеивания на двух участках.
%Построенные верхнее и нижнее решения позволяют не только обосновать существование слабого решения, но и оценить решение сверху и  снизу.




