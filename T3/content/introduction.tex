\Introduction




В последние десятилетия получила развитие теория $G$--сходимости и усреднения дифференциальных
операторов. Многие задачи математической
физики приводят к изучению вопроса $G$--сходимости дифференциальных
операторов с частными производными. Такие вопросы возникают в теории упругости,
электродинамике и других разделах физики и механики. Вопросам
$G$--сходимости и усреднения дифференциальных
операторов посвящено много работ и монографий (см. монографию В.\,В. Жикова и др.
\cite{smm_ZhKO} и имеющуюся там литературу).

$G$--сходимость дивергентных эллиптических операторов второго порядка изучена
С. Спаньоло и Е. Де Джорджи \cite{smm_Dg1, smm_Dg2, smm_Dg3}.
$G$--сходимость дивергентных эллиптических операторов высокого порядка изучена в
работах Жикова В. В. и др. (см. \cite{smm_ZhKO} и имеющуюся там литературу).
Вопросам $G$--сходимости и усреднению недивергентных эллиптических операторов
посвящены работы \cite{smm_ZhS, smm_ZhS1, smm_Sir1,smm_Sir}.
Оценки погрешности усреднения  дивергентных эллиптических уравнений  получены в работе  \cite{smm_Sus} М.\,Ш. Бирмана и
  Т.\,А.~Суслиной.
Изложение различных аспектов операторных оценок усреднения для дивергентных
операторов можно найти в обзорной работе В.\,В. Жикова,
С.\,Е. Пастуховой  \cite{smm_VP}.

В отчётном году учеными ОМИ обнаружены $G$-компактные классы недивергентных эллиптических операторов второго порядка с комплекснозначными коэффициентами.
Кроме того, получены оценки погрешности усреднения для недивергентных
эллиптических операторов второго порядка и периодической задачи для уравнения Бельтрами.


%%%%%%%%%%%%%%%%%%%%%%%%%%
%%%%%%%%%%%%%%%%%%%%%%%%%%
%Кадиев
%=================================

Вторым направлением исследований, проводившихся в отчётный период, было исследование вопросов моментной устойчивости решений для линейных дифференциальных уравнений Ито с запаздываниями высоких порядков и систем линейных дифференциальных уравнений Ито с запаздываниями специального вида на основе теории положительно обратимых матриц.

Устойчивость решений систем со случайными параметрами ранее изучалась в достаточно большом количестве работ (см., напр., \cite{kri-bib-1, kri-bib-2, kri-bib-3, kri-bib-4}). При этом чаще всего изучение стохастической устойчивости проводится методами, основанными на функционалах Ляпунова -- Красовского -- Разумихина.
Во многих случаях это вызывает серьёзные трудности, в связи с чем признаки устойчивости удаётся обнаруживать лишь для сравнительно узких классов стохастических дифференциальных уравнений.

Ранее при исследовании вопросов устойчивости для линейных дифференциальных уравнений Ито с запаздываниями первого порядка нами было показано, что 
более эффективным оказывается применение идей и методов, разработанных Н.В. Азбелевым и его учениками для исследования вопросов устойчивости детерминированных линейных функционально-дифференциальных уравнений (\cite{kri-bib-5, kri-bib-6, kri-bib-7, kri-bib-8}).
%
Метод основывается на преобразовании, целью которого является получение интегрального уравнения, для которого проще исследовать нужные свойства решений.
На случай стохастических функционально-дифференциальных уравнений этот метод распространен в работах \cite{kri-bib-9, kri-bib-10, kri-bib-11, kri-bib-12, kri-bib-13, kri-bib-14, kri-bib-15}.

При изучении литературы по данной тематике, нам не удалось обнаружить работы, в которых были бы исследованы вопросы устойчивости для уравнений Ито с последействием более высоких порядков.
Нами была предпринята попытка устранить этот пробел. 

Мы исследовали вопросы 
моментной устойчивости решений относительно начальных данных для линейных дифференциальных уравнений Ито с последействием высоких порядков
и
экспоненциальной $p$-устойчивости $(2 \le p < \infty)$ решений для систем линейных дифференциальных уравнений Ито с запаздываниями высоких порядков. 
При этом применяются методы исследования вопросов устойчивости для систем линейных дифференциальных уравнений Ито с последействием первого порядка, построенных по исходным уравнениям, принципы  $W$-метода и теория положительно обратимых матриц.
Отличие от классического $W$-метода состоит в том, что каждое уравнение системы преобразуется независимо от остальных, а каждая компонента решения оценивается отдельно.

Такой подход в сочетании со специальным видом исходных уравнений и специальным способом построения по исходным уравнениям систем линейных уравнений Ито с последействием первого порядка позволяет получить эффективные признаки устойчивости исходных уравнений в терминах параметров этих уравнений.
Отметим, что были использованы идеи работы \cite{kri-bib-16}, примененные при исследовании экспоненциальной устойчивости детерминированных линейных дифференциальных уравнений с запаздываниями высоких порядков.


%%%%%%%%%%%%%%%%%%%%%%%%%%
%%%%%%%%%%%%%%%%%%%%%%%%%%
%ЭИ

Ещё одно направление исследований связано с положительным решением задачи Коши. Этому вопросу посвящён ряд
работ российских и зарубежных математиков, в большинстве из которых изучаются в основном
вопросы существования положительного решения, его поведение,
априорные оценки и т.д.
Единственность положительного решения при этом изучается сравнительно мало.
В 2018 г. в ОМИ доказано существование и единственность положительного радиально-симметричного решения задачи Дирихле в кольцевой
области для одного класса нелинейных дифференциальных уравнений второго порядка и разработан численный метод его построения.

%Положительным решениям задачи Коши посвящен ряд
%работ российских и зарубежных математиков, в большинстве из которых изучаются в основном
%вопросы существования положительного решения, его поведение,
%априорные оценки и т.д.
%Единственность положительного решения при этом изучается сравнительно мало.
%В 2018 г. в ОМИ доказано существование и единственность положительного радиально-симметричного решения задачи Дирихле в кольцевой
%области для одного класса нелинейных дифференциальных уравнений второго порядка и разработан численный метод его построения.


%%%%%%%%%%%%%%%%
%%% MZG

Наконец, сотрудниками отдела решена задача восстановления функции, сосредоточенной в полосе на плоскости, заданной своими интегралами с весами вдоль ветвей гипербол одного семейства. Функция ищется в классе дважды непрерывно дифференцируемых функций. Решение задачи сведено к решению интегрального уравнения типа Вольтерра относительно преобразования Фурье неизвестной функции. В качестве весовых функций выступают многочлены.


Получена формула обращения интегрального преобразования функции на семействе ломаных в круге – так называемого $V$-преобразования Радона. Это преобразование моделирует ослабление интенсивности луча, падающего под некоторым углом на границу круга после отражения с тем же углом.
Получена формула для определения функции, заданной интегралами на одном двухпараметрическом семействе ломаных в круге, когда один из параметров – угол падения луча – меняется в сколь угодно малом угловом диапазоне.






%%%%%%%%%%%%%%%%%%%%%%%%%%
%%%%%%%%%%%%%%%%%%%%%%%%%%
%==================================
%
%\textbf{НИЖЕ СТАРЬЁ}
%
%
%
%Метод усреднения дифференциальных операторов, основанный на асимптотических
%разложениях по малому параметру, широко используется  в математической и
%физической литературе. Этот метод позволяет помимо теоремы усреднения
%получить оценки разности точного решения и его приближений. Нами впервые
%к усреднению обобщенного уравнения Бельтрами привлечены асимптотические
%методы, что позволило получить теорему усреднения и оценки порядка
%$O(\sqrt{\varepsilon})$  разности точного решения и его приближений
%в нормах пространств Лебега и Соболева. Эти оценки получены
%асимптотическими методами при минимальных предположениях гладкости
%на данные задачи: 1) коэффициенты --- измеримые ограниченные
%$\varepsilon$-периодические функции; 2) граница области из класса $C^2$; 3) правая часть из пространства Соболева $W_2^1$ (см. \cite{1, 2, 4}).
%Получены операторные оценки усреднения обобщенных уравнений Бельтрами.
%Изучены вопросы гельдеровости решений задачи Римана-Гильберта для
%обобщенных систем уравнений Бельтрами при минимальных
%условиях на данные задачи (см.\, \cite{3, 5, 6, 7, 8, 9, 10}).
%%end SMM
%
%Вопросам устойчивости решений систем со случайными параметрами
%посвящено большое количество работ. Достаточно полный их список
%приведен в монографиях \cite{kad1,kad2,kad3,kad4}. В этих работах для исследования
%вопросов стохастической устойчивости, в основном, применялся
%традиционный метод, основанный на функционалах
%Ляпунова -- Красовского -- Разумихина. Однако применение этих методов во
%многих случаях встречало серьёзные трудности. Поэтому эффективные
%признаки устойчивости обычно удавалось доказывать лишь для
%сравнительно узких классов стохастических дифференциальных уравнений
%с последействием. С другой стороны, в теории устойчивости
%детерминированных функционально--дифференциальных уравнений высокую
%эффективность показал <<W-метод>>, т.е. метод преобразования
%исходного уравнения с помощью вспомогательного уравнения,
%разработанный Н.В. Азбелевым и его учениками. Результатом таких
%преобразований является получение более простого уравнения, для
%которого легко выяснить наличие для его решений исследуемых свойств.
%
%Для нелинейных дифференциальных уравнений Ито с последействием
%вопросы устойчивости мало изучены. В работах \cite{kad5}, \cite{kad6} изучались
%вопросы локальной устойчивости решений нелинейных стохастических
%дифференциальных уравнений с последействием с помощью <<W-метода>>. В
%случае линейных уравнений локальная устойчивость решений и
%глобальная устойчивость решений эквивалентны, а в нелинейном случае
%из глобальной устойчивости решения следует локальная устойчивость
%этого же решения, а обратное неверно. Кроме того, в случае линейных
%уравнений из локальной устойчивости некоторого решения уравнения
%следует локальная устойчивость любого решения этого же уравнения, а в
%случае нелинейных уравнений этот факт не имеет место.
%Исследуются вопросы  глобальной
%экспоненциональной $p$-устойчи\-вос\-ти $(2 \le p < \infty )$ систем
%нелинейных дифференциальных уравнений Ито с запаздываниями
%специального вида, применяя идеи <<W-метода>> и  используя теорию
%положительно обратимых матриц. Отличие в том, что преобразуется
%отдельно каждое уравнение системы в отдельности и сначала
%оценивается каждая компонента решения. Такой подход и специальный
%вид уравнения позволяет получить новые результаты не только для
%нелинейных уравнений, но и для линейных уравнений, как частный
%случай нелинейных уравнений. При проведении исследований использован
%подход работы \cite{kad7}, примененный для исследования глобальной
%экспоненциальной устойчивости систем детерминированных нелинейных
%дифференциальных уравнений с запаздываниями.
%% end KRI
%
%
%
%Изучены вопросы единственности при $ n \geq 2$ положительного решения задачи Дирихле
%в шаре
%$ S=\left\{x\in R^n:\vert x \vert\ < 1 \right\} $
%с границей $\Gamma $
%для уравнения
%$$
%\Delta_{p} u+a(|x|){\vert u \vert}^q=0,\ x \in S,
%$$
%где $ \Delta_{p} u=div(\vert \nabla u\vert^{p-2}\nabla u),
%1<p \le2, q>1 $-константы, $a(t)-$ непрерывная неотрицательная
%при $t\geq 0 $ функция.
%Результаты
%относительно единственности положительного радиально-симметричного
%решения решения, полученные ранее в случае
%$a(|x|)=|x|^m, m\geq 0 $, обобщены на более общий случай $ a(|x|).$
%
%Доказаны существование и
%единственность положительного решения краевой задачи
%$$
%D_{0+}^{\alpha} u(t)+f(t,u(t) )=0,0<t<1,
%$$
%$$
%u(0)=u(1)=0
%$$
%в случае, когда $f(t,u)$ имеет степенной рост по $u$, а также предложен
%численный метод его построения.
%%end EI
%
%
%
%
%
%
%
%
%
%
%
%
%
%
%
%
%%%%%%%%%%%%%%%%%%%%%%%%%%%%
%%ТЕКСТ ВВЕДЕНИЯ ДЛЯ ТЕМЫ 3%
%%%%%%%%%%%%%%%%%%%%%%%%%%%%
%
%Кроме того, рассмотрена задача о восстановлении векторного поля по данным его поперечного лучевого преобразования в ограниченном угловом диапазоне на плоскости. Поперечное лучевое преобразование $P^\bot$ векторного поля $f=(f_1,f_2)$ на евклидовой плоскости $\mathbb R^2$ определяется формулой
%
%\begin{equation}\label{med-eq1}
%P^\bot f(\xi,s)=\int\limits_{-\infty}^\infty\left<\xi,f(s\xi+t\eta)\right>dt,
%\end{equation}
%$\xi\in\mathbb R^2, s\in\mathbb R$. Интегрирование ведется вдоль прямой $l: x=s\xi+t\eta$ с направляющим вектором $\eta=(\eta_1, \eta_2),  \xi=(\xi_1,\xi_2)=\eta^\bot$ -- ортогональный вектор. Преобразования вида \eqref{med-eq1} возникают в физике плазмы, в фотоэластичности, в акустической томографии, в задачах визуализации крови в человеческом теле путем измерений ультразвуковых сигналов (см. \cite{med-metka1,med-metka2}).
%Установлению свойств поперечного преобразования, получению формул его обращения, а также алгоритмов обращения  в случае, когда функция  $P^\bot f(\xi,s)$ задана на всем цилиндре $\mathbb Z=\mathbb S^1\times\mathbb R$, посвящено много работ (см., например, \cite{med-metka2} и процитированную там литературу). Практически обоснованным является решение задачи обращения в случае, когда функция  $P^\bot f(\xi,s)$ задана не на всем цилиндре $\mathbb Z$, а лишь на некоторой его части. Мы предполагаем, что вектор $\xi$ меняется в произвольно малом угловом диапазоне. Однозначное решение задачи оказывается возможным в классе полей, имеющих финитные преобразования Фурье.
%
%Исследован вопрос о восстановлении функции по ее интегралам вдоль ломаных одного семейства на плоскости.
%Пусть $\omega=(\omega_1,\omega_2), \theta=(\theta_1, \theta_2)$ -- линейно независимые единичные векторы на плоскости, $x=(x_1, x_2)\in\mathbb R^2$ -- произвольная точка. Символом $\Gamma_{\omega, \theta}(x)$ обозначим объединение лучей с вершиной в точке $x$ и направляющими векторами $\omega$ и $\theta$ соответственно:
%
%$$\Gamma_{\omega, \theta}(x)=\Gamma_{\omega}(x)\cup \Gamma_{\theta}^-(x),$$
%где $\Gamma_{\omega}(x)=\{\xi=(\xi_1,\xi_2))\in\mathbb R^2:\xi=x+\theta t, t\geq 0\}$; знак "$-$" $\,$  в символе $\Gamma_{\theta}^-(x)$ означает, что луч  $\Gamma_{\theta}(x)$ проходится в направлении, противоположном направлению вектора $\theta$ . При фиксированных $\omega$ и $\theta$ $\,\Gamma_{\omega, \theta}(x)$ -- двухпараметрическое семейство ломаных, зависящих от координат вершины $x=(x_1,x_2)$.
%
%Рассмотрим множество криволинейных интегралов 1-го рода вдоль $\Gamma_{\omega, \theta}(x)$
%\begin{equation}
%\label{med-PifagorTh1}
%g_\rho(x;\omega,\theta)=\int\limits_{\Gamma_{\omega, \theta}(x)} \rho(x,\xi)f(\xi)d\xi=\int\limits_{\Gamma_{\omega}(x)} \rho(x,\xi)f(\xi)d\xi-\int\limits_{\Gamma_{\theta}(x)} \rho(x,\xi)f(\xi)d\xi,
%\end{equation}
%где $\rho$ -- некоторая весовая функция.
%
%
%Задача состоит в восстановлении финитной функции $f(x)$ по известным интегралам \eqref{med-PifagorTh1}, в которых $\omega$ и $\theta$ -- фиксированные векторы.
%
%В пространстве четной размерности задача определения функции по ее интегралам вдоль конических поверхностей с единичной весовой функцией решена в работе А. Бегматова \cite{med-metka3}. Сформулируем этот результат.
%
%Пусть $\{\Gamma(x,y)\}$, где $x\in \mathbb R^{n-1}, \, y\in\mathbb R, \, y\geq0,\, n=2m>2,$ -- семейство конусов с вершинами в точках $(x,y)$:
%
%$$\sum\limits_{k=1}^{n-1}(x_k-\xi_k)^2=(y-\eta)^2, \quad 0\leq\eta\leq y,$$
%где $\xi=(\xi_1,\ldots,\xi_{n-1})\in\mathbb R^{n-1},\, \eta\in\mathbb R, \, \eta\geq0$. Через $U$ обозначим класс функций $f(x,y)$, которые всюду имеют все непрерывные частные производные до $n$-го порядка включительно и финитны с носителем в слое
%$$\Omega=\{(x,y): x\in \mathbb R^n,\, y\in (o,h), \, h<+\infty\}.$$
%
%Тогда решение уравнения
%
%$$\int\limits_{\Gamma(x,y)}f(\xi,\eta)ds=g(x,y)$$
%в классе $U$ единственно, причем имеет место представление
%
%\begin{equation}
%\label{med-PifagorTh0}
%f(x,y)=C\left(\frac{\partial^2}{\partial y^2}-\Delta_x\right)^m\int\limits_0^y g(x,y)d\eta,\end{equation}
%где $\Delta_x$ -- оператор Лапласа.
%
%
%В статье \cite{med-metka4}  вводится в рассмотрение  двухпараметрическое семейство симметричных относительно вертикальной оси ломаных $(\xi_1,\xi_2)=(x\pm r \sin\varphi, r \cos\varphi)$ с вершинами $M(x,0)$ на оси $O\xi$, где $r\geq0,\, 0<\varphi<\pi/2$. Соответствующее интегральное преобразование имеет вид
%
%$$g(x,\tau)=\int\limits_0^\infty f(x\pm r\sin\varphi, r\cos\varphi)dr,$$
%где $\tau=\tg\varphi$. Формула обращения, которую можно понимать как композицию свертки и обратной проекции, имеет вид
%
%$$f(x,y)=\frac{1}{2\pi^2}\int\limits_0^\infty\frac{d\tau}{\sqrt{1+\tau^2}}\left(p.v.\int\limits_\mathbb R d\xi\left( \frac{g'(\xi,\tau)}{\xi-x-y\tau}+\frac{g'(\xi,\tau)}{\xi-x+y\tau}\right)\right);$$
%внутренний интеграл понимается в смысле главного значения, а штрих означает производную по переменной $\xi$.
%
%Первым шагом решим задачу обращения интегрального преобразования
%\begin{equation}
%\label{med-PifagorTh2}
%g_\rho(x;\omega)=\int\limits_{\Gamma_\omega(x)}\rho(x,\xi)f(\xi)d\xi,\end{equation}
%в котором интегралы берутся вдоль лучей, параллельных вектору $\omega$, а весовая функция -- квазимногочлен вида
%
%\begin{equation}
%\label{med-PifagorTh3}
%\rho(x,\xi)=(x_1-\xi_1)^n_1(x_2-\xi_2)^n_2e^{\left<a,x-\xi\right>},\end{equation}
%где $a=(a_1,a_2), \, \left<a,x-\xi\right>=a_1(x_1-\xi_1)+a_2(x_2-\xi_2)$.
%
%Преобразования вида \eqref{med-PifagorTh1}, \eqref{med-PifagorTh2} с различными весовыми функциями -- частными случаями функции \eqref{med-PifagorTh3} -- изучены в работах \cite{med-metka5, med-metka6}. Применением преобразований Фурье и Лапласа для них получены формулы обращения в классе гладких финитных функций с носителями, лежащими в некоторой полосе на плоскости.
%
%
%
%
%








%%%%%%%%%%%%%%%%%%%%%%%%%%%%%%%%%%%%%%%%%%%%%%%
%%%%%%%% ПРОШЛЫЙ ГОД %%%%%%%%%%%%%%%%%%%%%%%%%%
%%%%%%%%%%%%%%%%%%%%%%%%%%%%%%%%%%%%%%%%%%%%%%%

%
%Исследования дифференциальных уравнений и вариационных проблем,
%включающих $p(x)$-лап\-ласиан,  вызывают особый интерес в последние
%годы. Ряд публикаций российских и зарубежных математиков
%посвящен этим проблемам (см., например \cite{Abd16} -- \cite{Abd17} и др.). Повышенный интерес к этой теме связан с тем, что некоторые важные физические явления
%могут быть смоделированы с помощью  уравнений с $p(x)$-лапласианом. В частности,
%свойства электрореологической жидкости описываются эллиптическими и параболическими
%уравнениями с $p(x)$-лапласианом \cite{Abd13} -- \cite{Abd15}.  Электрореологические жидкости
%-- жидкости, которые могут резко менять свою вязкость под
%действием электрических полей. Такие жидкости чрезвычайно перспективны
%с практической точки зрения (уже сегодня они нашли
%применение в космической технике, биомеханике и биомедицине).
%Эксперименты показывают, что электрореологический эффект
%в своей основе связан с электростатическим взаимодействием частиц
%и динамикой изменения структуры размещения дисперсных частиц
%под действием электрического поля и градиентов скоростей деформации.
%Примером природной электрореологической жидкости является слизь,
%вырабатываемая обычными слизнями. По данным World Scientist,
%у слизня сетчатого (Deroceras reticulatum) впервые обнаружили
%особое свойство выделяемой слизи твердеть под действием
%электрического поля, что характерно для особого класса
%<<умных>>  материалов -- электрореологических жидкостей,
%которые уже сейчас предполагаются к использованию при разработке
%новых транспортных средств в труднодоступных местах
%(шахтах, пещерах, космосе), гидрозатворах, механически
%нагруженных подшипниках, гидродинамических устройствах и т.д.
%
%
%
%В отчетном году для задачи Дирихле для нелинейного дифференциального уравнения с $p(x)$-лапласиа\-ном конструируются верхнее и нижнее решения путем склеивания на двух участках.
%Построенные верхнее и нижнее решения позволяют не только обосновать существование слабого решения, но и оценить решение сверху и  снизу.
%
%
%
%
%Во многих областях естествознания  возникают задачи, связанные с изучением свойств сильно неоднородных сред периодической структуры и
%происходящих в них процессов. Значительную часть таких сред составляют композитные материалы типа войлока,
%линейные размеры неоднородностей которых существенно меньше линейных размеров области и существенно больше
%размеров молекул --- настолько, что к этим средам можно применить уравнения механики сплошных сред. Для описания процессов в композитах предложено много методов. Одним из эффективных методов исследования является метод асимптотических разложений. В данном отчете излагаются результаты изучения вопросов
%гельдеровости решения задачи Римана – Гильберта для уравнения Бельтрами с периодическим коэффициентом, зависящим от малого параметра, и конструирования асимптотического разложения для него. Кроме того, исследуется задача усреднения недивергентного эллиптического оператора второго порядка в области на плоскости в случае, когда коэффициенты локально периодичны по одной из переменных.









