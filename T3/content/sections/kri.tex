

%Кадиев Р.И.
\chapter{Экспоненциальная $p$-устойчивость решений линейных \\ДУ Ито второго порядка с запаздыванием}




%\section{Введение и постановка задачи}
%
%Вопросам устойчивости решений систем со случайными параметрами посвящено большое количество работ.
%Достаточно полный их список приведен в монографиях \cite{kri-bib-1, kri-bib-2, kri-bib-3, kri-bib-4}.
%В основном, в этих работах исследование стохастической устойчивости проводится традиционными методами, основанными на функционалах Ляпунова-Красовского-Разумихина.
%Однако применение этих методов во многих случаях встречает серьёзные трудности.
%Поэтому эффективные признаки устойчивости обычно удается доказывать лишь для сравнительно узких классов стохастических дифференциальных уравнений.
%С другой стороны, в теории устойчивости детерминированных функционально-дифференциальных уравнений высокую эффективность показал  $W$-метод, т.е. метод преобразования исходного уравнения с помощью вспомогательного уравнения, разработанный Н.В. Азбелевым и его учениками \cite{kri-bib-5, kri-bib-6, kri-bib-7, kri-bib-8}.
%Целью такого преобразования является получение интегрального уравнения, для которого проще исследовать нужные свойства решений.
%На случай стохастических функционально-дифференциальных уравнений этот метод распространен в работах \cite{kri-bib-9, kri-bib-10, kri-bib-11, kri-bib-12, kri-bib-13, kri-bib-14, kri-bib-15}.
%
%Для линейных дифференциальных уравнений Ито с последействием более высоких порядков вопросы устойчивости изучены недостаточно.
%Нам не известны работы других авторов, в которых исследуются вопросы устойчивости для уравнений Ито с последействием более высоких порядков.
%
%В настоящей работе исследуются вопросы экспоненциальной p-устойчивости $(2 \le p < \infty)$ решений для линейных дифференциальных уравнений Ито с запаздываниями второго порядка.
%При этом применяются методы исследования вопросов устойчивости для линейных дифференциальных уравнений Ито с последействием высоких порядков исследованием вопросов устойчивости для систем линейных дифференциальных уравнений Ито с последействием первого порядка, построенных по исходным уравнениям, принципы  $W$-метода и теория положительно обратимых матриц.
%Отличие от классического $W$-метода состоит в том, что каждое уравнение системы преобразуется независимо от остальных, а каждая компонента решения оценивается отдельно.
%Такой подход, в сочетании со специальным видом исходных уравнений, специальным способом построения по исходным уравнениям систем линейных уравнений Ито с последействием первого порядка, позволяет получить эффективные признаки устойчивости исходных уравнений в терминах параметров этих уравнений.
%В статье использованы идеи работы \cite{kri-bib-16}, примененные при исследовании экспоненциальной устойчивости детерминированных линейных дифференциальных уравнений с запаздываниями высоких порядков.

\section{Предварительные сведения и объект исследования}

Пусть:  $(\Omega, \Im, (\Im_t)_{t \ge 0}, P)$ --- стохастический базис;  $k^n$ --- линейное пространство  $n$-мерных $\Im_0$-измеримых случайных величин;  $B_i, i=1, \dots, m$ --- скалярные независимые стандартные винеровские процессы; $1 \le p < \infty$;  $c_p$ --- положительное число, зависящее от $p$ (см. \cite[с. 65]{kri-bib-17}) и используемое в оценке \eqref{kri-eq-2}; $E$ --- символ математического ожидания; $|.|$ --- норма в $R^n$; $\|.\|$ --- норма $n \times m$-матрицы, согласованная с нормой в $R^m$; $\|.\|_X$ --- норма в нормированном пространстве  $X$; $\mu$ --- мера Лебега на $[0, \infty )$.
$\overline E$ единичная матрица размерности  $n\times n$.

Пусть $B=(b_{ij})_{i,j=1}^m$ --- $m\times m$-матрица.
Матрица  $B$ называется неотрицательной, если  $b_{ij} \ge 0$, $i,j=1, \dots ,m$, и положительной, если $b_{ij} > 0$, $i, j=1, \dots, m$.

\begin{definition}
\cite{kri-bib-18} Матрица  $B=(b_{ij})_{i,j=1}^m$ называется $M$-матрицей, если $b_{ij}\le 0$ при  $i,j=1,\dots,m$ и $i\neq j$ и выполнено одно из следующих условий:

\begin{itemize}
	\item[-]
	для матрицы  $B$ существует положительная обратная матрица $B^{-1}$;
	
	\item[-]
	диагональные миноры матрицы  $B$ положительны.
\end{itemize}
\end{definition}

\begin{lemma}
\cite{kri-bib-18} Матрица $B$ является $M$-матрицей, если $b_{ij}\le 0$ при $i,j=1,\dots,m$ и $i\neq j$, а также выполнено одно из следующих условий:

\begin{itemize}
	\item[-]
	$b_{ii}\ge \overset m{\underset{j=1,i\neq j}{\sum }}|b_{ij}|$, $i=1,\dots,m$;
	
	\item[-]
	$b_{jj}\ge \overset m{\underset{i=1,i\neq j}{\sum }}|b_{ij}|$, $j=1,\dots,m$;
	
	\item[-]
	существуют положительные числа $\xi_i$, $i=1,\dots,m$ такие, что \\ $\xi_ib_{ii}\ge \overset m{\underset{j=1,i\neq j}{\sum}}\xi_j|b_{ij}|$, $i=1,\dots,m$;
	
	\item[-]
	существуют положительные числа $\xi_i$, $i=1,\dots,m$ такие, что \\ $\xi_jb_{jj}\ge \overset m{\underset{j=1,i\neq j}{\sum}}\xi_j|b_{ij}|$, $j=1,\dots,m$;
\end{itemize}
\end{lemma}

Объектом исследований является дифференциальное уравнение Ито второго порядка вида

\begin{equation}\label{kri-eq-1}
	dx'(t) = \left[-a_{00}(t)x(t) - a_{10}(t)x'(t) + \sum_{j=1}^{m_0}c_{j0}(t)x(h_{j0}(t))\right]dt +
\end{equation}
\begin{equation*}
	\sum_{i=1}^{m} \left[-a_{0i}(t)x(t)-a_{1i}(t)x'(t)+\sum_{j=1}^{m_i}c_{ji}(t)x(h_{ji}(t))\right] \quad (t \ge 0),
\end{equation*}

с начальными условиями

\begin{equation}\label{kri-eq-1a}
	x(t)=\varphi (t)\quad(t\ge 0),
\end{equation}

\begin{equation}\label{kri-eq-1b}
	x^{(j)}(0)=b_{j+1}, \quad j=0,1,
\end{equation}

где:

\begin{enumerate}
	\item[1.]
	$a_{ij}\le 0,$ $i=0,1,$ $j=0,\dots,m$ - измеримые по Лебегу	функции, заданные на $[0,\infty [$ и  $0<\overline{a_{\mathit{i0}}}\le a_{\mathit{i0}}(t)\le A_{\mathit{i0}}$ $\mu$-почти всюду для некоторых положительных чисел $\overline{a_{\mathit{i0}}},A_{\mathit{i0}}$ при $i=0,1$, $|a_{ij}(t)|\le A_{ij}$ $\mu$-почти всюду для некоторого положительного числа  $A_{ij}$ при $i=0,1$, $j=1,\dots,m$, $|c_{ij}(t)|\le C_{ij}$ $\mu$-почти всюду для некоторого положительного числа  $C_{ij}$ при $i=0,...,m$, $j=0,...,m_i$;
	
	\item[2.]
	$h_{ij}$, $i=0,...,m$, $j=0,...,m_i$ - измеримые по Лебегу функции, заданные на $[0,\infty [$ такие, что $0 \le t-h_{ij}(t)\le \tau_{ij}$ $\mu$-почти всюду для некоторого положительного числа  $\tau_{ij}$ при $i=0,...,m$, $j=0,...,m_i$;
	
	\item[3.]
	$\varphi - \Im_0$-измеримый скалярный случайный процесс, заданный на $[\sigma ,0[$, где \linebreak $\sigma =\text{max}\left\{\tau_{ij}, i=0,...,m, j=0,\dots,m_i\right\}$;
	
	\item[4.]
	$b_j$, $j=1,2$ $\Im_0$-измеримые скалярные случайные величины.
\end{enumerate}

В дальнейшем нам понадобятся следующие обозначения:

\begin{itemize}
	\item[-]
	$b=\text{col}(b_1,b_2)$;
	
	\item[-]
	$k_p^n=\{\alpha :\alpha \in k^n(T),\|\alpha \|_{k_p^n}\overset{\text{def}}{=}\left(E|\alpha
	|^p\right)^{1/p}<\infty \}$.
\end{itemize}

Под решением задача \eqref{kri-eq-1}, \eqref{kri-eq-1a}, \eqref{kri-eq-1b} понимается любой случайный процесс $x(t)$, $t\in ]-\infty ,+\infty[$, прогрессивно измеримый и непрерывно дифференцируемый при  $t\ge 0$, удовлетворяющий условиям \eqref{kri-eq-1a} и \eqref{kri-eq-1b}, а также уравнению

\begin{equation}\label{kri-eq-1-1}
	x'(t) = b_1+ \int_{0}^{1} \left[-a_{00}(s)x(s) - a_{10}(s)x'(s) + \sum_{j=1}^{m_0}c_{j0}(s)x(h_{j0}(s))\right]ds +
\end{equation}
\begin{equation*}
	\sum_{i=1}^{m} \int_{0}^{1} \left[-a_{0i}(t)x(t)-a_{1i}(t)x'(t)+\sum_{j=1}^{m_i}c_{ji}(t)x(h_{ji}(t))\right]dB_i(t) \quad (t \ge 0),
\end{equation*}

где первый интеграл понимается в смысле Лебега, а второй интеграл в смысле Ито.

Отметим, что задача \eqref{kri-eq-1}, \eqref{kri-eq-1a}, \eqref{kri-eq-1b} при сделанных предположениях имеет единственное решение в силу результатов работы \cite{kri-bib-10}. Обозначим через $x(t,b,\varphi )$ - решение задача \eqref{kri-eq-1}, \eqref{kri-eq-1a}, \eqref{kri-eq-1b}, т.е. $x(t,b,\varphi)=\varphi$ при $t<0$ и $x(0,b,\varphi)=b_1$, $x'(0,b,\varphi)=b_2$.

\begin{definition}
Будем говорить, что уравнения \eqref{kri-eq-1} \textit{экспоненциально } $p$\textit{{}-устойчи-\linebreak вым}
относительно начальных данных (или просто \textit{экспоненциально } $p$\textit{{}-устойчивым}), если при некоторых
положительных постоянных  $K, \lambda$ справедливо неравенство

\begin{equation*}
(E|x(t,b,\varphi )|^p)^{1/p}\le K\left(\|b\|_{k_p^2}+ \mathrm{vrai} \sup\limits_{t<0}(E|\varphi
(t)|^p)^{1/p}\right)\mathrm{exp}\{-\mathit{\lambda t}\}\text{  }(t\geq 0).
\end{equation*}
\end{definition}

Исследование экспоненциальной  $p${}-устойчивости \ уравнения \eqref{kri-eq-1} будет производиться исследованием вопросов
экспоненциальной  $p${}-устойчивости \ для системы линейных уравнений Ито с последействием построенная по уравнению
\eqref{kri-eq-1}. Традиционно для этого по уравнению \eqref{kri-eq-1} с начальными условиями \eqref{kri-eq-1a}, \eqref{kri-eq-1b} строят систему следующего вида

\begin{equation*}
	x'_1(t) = x_2(t)\quad(t\ge0)
\end{equation*}
\begin{equation}\label{kri-eq-2}
	dx_2(t) = \left[-a_{00}(t)x_1(t) - a_{10}(t)x_2(t) + \sum_{j=1}^{m_0}c_{j0}(t)x_1(h_{j0}(t))\right]dt +
\end{equation}
\begin{equation*}
	\sum_{i=1}^{m} \left[-a_{0i}(t)x_1(t)-a_{1i}(t)x_2(t)+\sum_{j=1}^{m_i}c_{ji}(t)x_1(h_{ji}(t))\right]dB_i(t) \quad (t \ge 0),
\end{equation*}
с начальными условиями
\begin{equation}\label{kri-eq-2a}
	x_1(t)=\varphi (t)\quad(t\ge 0),
\end{equation}
\begin{equation}\label{kri-eq-2b}
	x_j(t)=b_j, j=1,2,
\end{equation}
и при этом используется факт совпадения первой компоненты решения задачи \eqref{kri-eq-2}, \eqref{kri-eq-2a}, \eqref{kri-eq-2b} с решением задачи \eqref{kri-eq-1}, \eqref{kri-eq-1a},
\eqref{kri-eq-1b}. Следовательно, экспоненциальная  $p$-\linebreak устойчивость уравнения \eqref{kri-eq-1} эквивалентна экспоненциальной
$p${}-устойчивости по первой компоненте системы \eqref{kri-eq-2}. Вопросы экспоненциальной  $p${}-устойчивости по части переменных
для систем линейных уравнения Ито с последействием более общего вида изучены в работах \cite{kri-bib-19, kri-bib-20}. Кроме того,
экспоненциальная  $p${}-устойчивость уравнения \eqref{kri-eq-1} будет следовать из экспоненциальной  $p${}-устойчивости системы \eqref{kri-eq-2}.
Исследование экспоненциальной  $p${}-устойчивости системы \eqref{kri-eq-2} можно провести эффективно, используя метод
вспомогательных уравнений и эти результаты можно применить для определения экспоненциальной  $p${}-устойчивости
уравнения \eqref{kri-eq-1}.

Использование системы \eqref{kri-eq-2} для анализа экспоненциальной  $p${}-устойчивости \linebreak уравнения \eqref{kri-eq-1} не эффективно, так как первое
уравнение этой системы не содержат информацию о параметрах уравнения \eqref{kri-eq-1}. В работе \cite{kri-bib-16} для анализа экспоненциальной
устойчивости детерминированных линейных дифференциальных уравнений с последействием высоких порядков используя
положительную обратимость матриц предложен другой подход построения вспомогательных систем по исходным уравнениям. Этот
подход оказался более эффективным чем традиционный подход. Нами исследована экспоненциальная  $p${}-устойчивость
уравнения \eqref{kri-eq-1} используя идеи и приемы работы \cite{kri-bib-16}.

Рассмотрим систему линейных уравнений Ито с запаздыванием следующего вида

\begin{equation*}
	x_1'(t)=-qx_1(t)+x_2(t)(t\ge 0),
\end{equation*}
\begin{equation}\label{kri-eq-3}
	\text{dx}_2(t)=\left[-(a_{00}(t)-(a_{10}(t)-q)q)x_1(t)-(a_{10}(t)-q)x_2(t)+\overset{m_0}{\underset{j=1}{\sum
	}}c_{\mathit{j0}}(t)x_1(h_{\mathit{j0}}(t))\right]\text{dt}+
\end{equation}
\begin{equation*}
\overset
	m{\underset{i=1}{\sum }}\left[-(a_{0i}(t)-a_{1i}(t)q)x_1(t)-a_{1i}(t)x_2(t)+\overset{m_i}{\underset{j=1}{\sum
	}}c_{\text{ji}}(t)x_1(h_{\text{ji}}(t))\right]\mathit{dB}_i(t)\text{  }(t\ge
	0),
\end{equation*}
где  $q$ - некоторое положительное число, а остальные параметры
определены в уравнении \eqref{kri-eq-1}, с начальными условиями
\begin{equation}\label{kri-eq-3a}
	x_1(t)=\varphi (t)\text{
	}(t\ge
	0),
\end{equation}
\begin{equation}\label{kri-eq-3b}
	x_j(t)=b_j,\text{
	}j=1,2,
\end{equation}

Нетрудно убедиться, что первая компонента решения задачи \eqref{kri-eq-3}, \eqref{kri-eq-3a}, \eqref{kri-eq-3b} совпадает с решением задачи \eqref{kri-eq-1}, \eqref{kri-eq-1a}, \eqref{kri-eq-1b}.
Следовательно, экспоненциальная  $p${}-\linebreak устойчивость уравнения \eqref{kri-eq-1} будет следовать из экспоненциальной
$p${}-устойчивости системы \eqref{kri-eq-3}. В дальнейшем для системы \eqref{kri-eq-3} будут исследованы вопросы экспоненциальной
$p${}-устойчивости.


\begin{lemma}
 Пусть  $f(s)$\ {}- скалярный случайный процесс, интегрируемый по винеровскому процессу  $B(s)$
\textsubscript{\ }на отрезке  $[0,t]$. Тогда справедливо неравенство

\begin{equation}\label{kri-eq-4}
	\left(E\left| \int\limits_{0}^{t} f(s)\text{dB}(s)\right|^{2p}\right)^{1/2p}\le
	c_p\left(E\left(\int\limits_{0}^{t} |f(s)|^2\text{ds}\right)^p\right)^{1/2p},
\end{equation}
где  $c_p$\ {}- некоторое число, зависящее от  $p$.
\end{lemma}
Справедливость неравенства \eqref{kri-eq-2} следует из неравенства \eqref{kri-eq-4} работы \cite[стр. 65]{kri-bib-18}, где приведено и конкретное выражение для
$c_p$.

\begin{lemma}
Пусть  $g(s)$\ {}- скалярная функция на  $[0,\infty [$, квадрат которой локально суммируем,
$f(s)$ {}- скалярный случайный процесс такой, что  $\underset{s\ge 0} \sup \left(E|f(s)|^{2p}\right)^{1/2p} <\infty $. Тогда справедливы следующие неравенства

\begin{equation}\label{kri-eq-5}
	\underset{t\ge 0}\sup\left(E\left|\int\limits_{0}^{t} g(s)f(s)\text{ds}\right|^{2p}\right)^{1/2p}\le \underset{t\ge 0}\sup\left(\int\limits_{0}^{t} |g(s)|\text{ds}\right)\underset{s\ge
		0}\sup\left(E|f(s)|^{2p}\right)^{1/2p},
\end{equation}

\begin{equation}\label{kri-eq-6}
	\underset{t\ge 0}\sup\left(E\left| \int\limits_{0}^{t} (g(s))^2(f(s))^2\text{ds}\right|^p\right)^{1/2p}\le \underset{t\ge 0}\sup\left(\int\limits_{0}^{t} 	(g(s))^2\text{ds}\right)^{1/2}\underset{s\ge
		0}\sup\left(E|f(s)|^{2p}\right)^{1/2p}.
\end{equation}
\end{lemma}
Справедливость неравенств \eqref{kri-eq-5} и \eqref{kri-eq-6} доказана в работе \cite{kri-bib-21}.

\section{Основной результат}

В силу предположений существуют положительные числа  $\gamma _i,\text{  }i=0,1,...,m$ \ такие, что для системы \eqref{kri-eq-3}
$|(a_{00}(s)+(a_{10}(s)-q)|\le \gamma _0,\text{  }|(a_{0i}(s)+a_{1i}(s)q)|\le \gamma _i,\text{  }i=1,...,m.$ \ Пусть
$Z=\left(\begin{matrix}z_{11}&z_{12}\\z_{21}&z_{22}\end{matrix}\right)$, где  $z_{11}=1$,  $z_{12}=-\frac 1 q$,
$z_{21}=-\frac{\overset{m_0}{\underset{j=1}{\sum }}c_{\mathit{j0}}}{\overline{a_{10}}-q}-\frac{c_p\overset
m{\underset{i=1}{\sum }}\left[\gamma _i\overline{y_1}+\overset{m_i}{\underset{j=1}{\sum
}}c_{\text{ji}}\right]}{\sqrt{\overline{a_{10}}-q}}$, \  $z_{22}=1-\frac{\gamma
_0}{\overline{a_{10}}-q}-\frac{c_p\overset m{\underset{i=1}{\sum }}A_{1i}}{\sqrt{\overline{a_{10}}-q}}.$

\begin{theorem}
Если существует положительное число  $q$ \ такое, что матрица  $Z$ является  $M${}-матрицей, то
уравнение \ \eqref{kri-eq-1} экспоненциально  $2p${}-устойчиво.
\end{theorem}

\begin{proof}
Систему \eqref{kri-eq-3} с условиями \eqref{kri-eq-3a} \ запишем в следующем виде

\begin{equation*}
	\overline{x_1}'(t)=-q\overline{x_1}(t)+\overline{x_2}(t)\text{   }(t\ge 0),
\end{equation*}
\begin{equation*}
	d\overline{x_2}(t)=[-(a_{00}(t)-(a_{10}(t)-q)q)\overline{x_1}(t)-(a_{10}(t)-q)\overline{x_2}(t)+
\end{equation*}
\begin{equation}\label{kri-eq-7}
	\overset{m_0}{\underset{j=1}{\sum }}c_{\mathit{j0}}(t)(\overline{x_1}(h_{\mathit{j0}}(t))+\overline{\varphi
	}(h_{\mathit{j0}}(t)))]\text{dt}+
\end{equation}
\begin{equation*}
	\overset m{\underset{i=1}{\sum
	}}[-(a_{0i}(t)-a_{1i}(t)q)\overline{x_1}(t)-a_{1i}(t)\overline{x_2}(t)+
\end{equation*}
\begin{equation*}
	\overset{m_i}{\underset{j=1}{\sum
	}}c_{\text{ji}}(t)(\overline{x_1}(h_{\text{ji}}(t))+\overline{\varphi
	}(h_{\text{ji}}(t)))]\mathit{dB}_i(t)\text{  }(t\ge 0),
\end{equation*}
где  $\overline{x_i}$\ --- неизвестный скалярный случайный процесс на
$\text{  }]-\infty ,+\infty [$ такой, что  $\overline{x_i}(t)=0$ при
$t<0$ \ и  $\overline{\varphi }(t)$\ --- известный скалярный случайный процесс на $\text{  }]-\infty ,+\infty
[$ такой, что  $\overline{\varphi }(t)=\varphi (t)$ \ при  $t\in [\sigma ,0[$ \ и  $\overline{\varphi
}(t)=0$ \ когда  $t$ не принадлежит полуинтервалу  $[\sigma ,0[$. Обозначим через
$\overline x(t,b,\varphi )$ решение системы \eqref{kri-eq-7}, удовлетворяющее условию \eqref{kri-eq-3b}.
Очевидно, что решение задачи \eqref{kri-eq-7}, \eqref{kri-eq-3b} при  $t\ge 0$ совпадает с решением задачи \eqref{kri-eq-3}, \eqref{kri-eq-3a}, \eqref{kri-eq-3b}.

Если в системе \eqref{kri-eq-7} сделать замену  $\overline{x_i}(t)=\text{exp}\{-\mathit{\lambda t}\}y_i(t)$, где  $y_i(t)$\ ---
неизвестный скалярный случайный процесс на  $\text{  }]-\infty ,+\infty [$ \ такой, что $y_i(t)=0$ при  $t<0$,
$0<\lambda <\text{min}\{q,A_{10}-q\}$ \ для  $i=1,2$, то получится система

\begin{equation*}
	y'_{1}(t)=(\lambda -q)y_1(t)+y_2(t)\text{   }(t\ge 0),
\end{equation*}
\begin{equation*}	
	\text{dy}_2(t)=[-(a_{00}(t)-(a_{10}(t)-q)q)y_1(t)+(\lambda -(a_{10}(t)-q))y_2(t)+
\end{equation*}
\begin{equation}\label{kri-eq-8}	
	\overset{m_0}{\underset{j=1}{\sum }}c_{\mathit{j0}}(t)\text{exp}\{\mathit{\lambda t}\}(\text{exp}\{-\mathit{\lambda
		h}_{\mathit{j0}}(t)\}y_1(h_{\mathit{j0}}(t))+\overline{\varphi
	}(h_{\mathit{j0}}(t)))]\text{dt}+
\end{equation}
\begin{equation*}
	\overset m{\underset{i=1}{\sum
	}}[-(a_{0i}(t)-a_{1i}(t)q)y_1(t)-a_{1i}(t)y_2(t)+
\end{equation*}
\begin{equation*}
	+\overset{m_i}{\underset{j=1}{\sum
	}}c_{\text{ji}}(t)\text{exp}\{\mathit{\lambda t}\}(\text{exp}\{-\mathit{\lambda
		h}_{\text{ji}}(t)\}y_1(h_{\mathit{j0}}(t)+\overline{\varphi
	}(h_{\text{ji}}(t)))]\mathit{dB}_i(t)\text{  }(t\ge 0),
\end{equation*}

Положив  $\eta (t)=(a_{10}(t)-q)-\lambda $, с учетом условий \ \eqref{kri-eq-3b} систему \eqref{kri-eq-8} можно переписать в следующем виде:

\begin{equation*}
	y_1(t)=\text{exp}\{-(q-\lambda )t\}b_1+\int\limits_{0}^{t}\text{exp}\{-(q-\lambda
	)(t-s)\}y_2(s)\text{ds}\text{   }(t\ge 0),
\end{equation*}
\begin{equation*}		
	y_2(t)=\text{exp}\left\{-\int\limits_{0}^{t}\eta (s)\text{ds}\right\}b_2+\int\limits_{0}^{t}\text{exp}\left\{-\int\limits_{s}^{t}\eta (\zeta )\mathit{d\zeta
	}\right\}[-(a_{00}(s)-(a_{10}(s)-q)q)y_1(s)+
\end{equation*}	
\begin{equation}\label{kri-eq-9}
	\overset{m_0}{\underset{j=1}{\sum
	}}c_{\mathit{j0}}(s)\text{exp}\{\mathit{\lambda s}\}(\text{exp}\{-\mathit{\lambda
		h}_{\mathit{j0}}(s)\}(y_1(h_{\mathit{j0}}(s))+\overline{\varphi
	}(h_{\mathit{j0}}(s)))]\text{ds}+
\end{equation}	
\begin{equation*}
	\overset m{\underset{i=1}{\sum }}\int\limits_{0}^{s} \text{exp}\left\{-\int\limits_{s}^{t}\eta (\zeta )\mathit{d\zeta
	}\right\}[-(a_{0i}(s)-a_{1i}(s)q)y_1(s)-a_{1i}(s)y_2(s)+
\end{equation*}	
\begin{equation*}
	+\overset{m_i}{\underset{j=1}{\sum
	}}c_{\text{ji}}(s)\text{exp}\{\mathit{\lambda s}\}(\text{exp}\{-\mathit{\lambda
		h}_{\text{ji}}(s)\}(y_1(h_{\mathit{j0}}(s)+\overline{\varphi
	}(h_{\text{ji}}(s)))]\mathit{dB}_i(s)\text{  }(t\ge 0),
\end{equation*}

В дальнейшем будем пользоваться обозначениями \  $\overline{y_i}=\sup\limits_{t\geq0} \left(E|y_i(t)|^{2p}\right)^{1/2p},i=1,2$, \  $\overline{\overline{\varphi
}}=\sup\limits_{t<0} \left(E|\varphi (t)|^{2p}\right)^{1/2p}$ \ и неравенствами \eqref{kri-eq-2}, \eqref{kri-eq-3}, \eqref{kri-eq-4}.

Из первого \ уравнения системы \eqref{kri-eq-9} получим

\begin{equation*}
\overline{y_1}\le \|b_1\|_{k_{2p}^1}+\frac 1{q-\lambda }\overline{y_2}.
\end{equation*}
Из последнего уравнения системы \eqref{kri-eq-9} с учетом предыдущих обозначений и неравенств \eqref{kri-eq-2}, \eqref{kri-eq-3}, \eqref{kri-eq-4} получим

\begin{equation*}
  \overline{y_2} \leq \|b_2\|_{k_{2p}^1} + \left[ \gamma_0 \overline{y_2} + \sum\limits_{j=1}^{m_0} c_{j0} \exp\{\lambda\tau_{j0}\}(\overline{y_1}+\overline{\overline{\varphi}}) \right]
\sup\limits_{t\geq0} \int\limits_{0}^{e} \exp\left\{-\int\limits_{s}^{t} \eta(\zeta)d\zeta\right\}ds +
\end{equation*}
\begin{equation}\label{kri-eq-10}
c_p \sum\limits_{i=1}^{m} \left[\gamma_1\overline{y_1}+A_{1i}\overline{y_2}+\sum\limits_{j=1}^{m_i} c_{ji}\exp \{\lambda\tau_{ji}\}(\overline{y_1}+\overline{\overline{\varphi}})\right]
\sup\limits_{t\geq0} \left(\int\limits_{0}^{t}\exp\left\{ -2\int\limits_{s}^{t} \eta(\zeta)d\zeta \right\} ds \right)^{1/2}
\end{equation}
%
Так как

\begin{equation*}
\sup\limits_{t\ge 0}\int\limits_{0}^{t}\text{exp}\left\{-\int\limits_{s}^{t}\eta (\zeta )\mathit{d\zeta }\right\}\text{ds}=\sup\limits_{t\geq 0} \int\limits_{s}^{t}
\left[\text{exp}\left\{-\int\limits_{s}^{t}\eta (\zeta
)\mathit{d\zeta }\right\}\eta (s)\right]/\eta (s)\text{ds}\le \frac 1{\overline{a_{10}}-q-\lambda }
\end{equation*}
и

\begin{equation*}
\sup\limits_{t\ge 0}\left(\int\limits_{0}^{t}\text{exp}\left\{-2\int\limits_{s}^{t}\eta (\zeta )\mathit{d\zeta
}\right\}\text{ds}\right)^{1/2}=
\end{equation*}
\begin{equation*}
\sup\limits_{t\ge 0}\left(\int\limits_{0}^{t}\left[\text{exp}\left\{-2\int\limits_{s}^{t}\eta (\zeta )\mathit{d\zeta }\right\}2\eta
(s)\right]/(2\eta (s))\text{ds}\right)^{1/2}\le \frac 1{\sqrt{\overline{a_{10}}-q-\lambda
}},
\end{equation*}
то из оценки \eqref{kri-eq-10} получаем

\begin{equation*}
\overline{y_2} \leq \|b_2\|_{k^1_{2p}} + \frac{\gamma_0\overline{y_2}+\sum\limits_{j=1}^{m_0} c_{j0}\exp \{\lambda\tau_{j0} \} \overline{y_1}}{\overline{a_{10}}-q-\lambda} +
\end{equation*}
\begin{equation}\label{kri-eq-11}
\frac{c_p \sum\limits_{i=1}^{m} \left[ \gamma_i\overline{y_1}+A_{1i}\overline{y_2}+\sum\limits_{j=1}^{m_i} c_{ji}\exp\{\lambda\tau_{ji} \} \overline{y_1} \right]}{\sqrt{\overline{a_{10}}-q-\lambda}} + M(\lambda)\overline{\overline{\varphi}},
\end{equation}
где
\begin{equation*}
M(\lambda )=\frac{\overset{m_0}{\underset{j=1}{\sum }}c_{\mathit{j0}}\text{exp}\{\text{$\lambda \tau
		$}_{\mathit{j0}}\}}{\overline{a_{10}}-q-\lambda }+ \frac{c_p \sum\limits_{i=1}^{m}\left[\sum\limits_{j=1}^{m_i} c_{ji}\exp\{\lambda\tau_{ji} \}\right]}{\sqrt{\overline{a_{10}}-q-\lambda}}.
\end{equation*}

\begin{equation*}
M(\lambda )=\frac{\overset{m_0}{\underset{j=1}{\sum }}c_{\mathit{j0}}\text{exp}\{\text{$\lambda \tau
$}_{\mathit{j0}}\}}{\overline{a_{10}}-q-\lambda }+c_p\overset m{\underset{i=1}{\sum }}.
\end{equation*}
Обозначим теперь  $y(t)=\text{col}(y_1(t),y_2(t))$,  $\overline
y=\text{col}(\overline{y_1},\overline{y_2})$,  $\overline{M(\lambda
)}=\text{col}(0,M(\lambda ))$,  $e=\text{col}(1,1)$,  $Z(\lambda
)=\left(\begin{matrix}z_{11}(\lambda )&z_{12}(\lambda )\\z_{21}(\lambda )&z_{22}(\lambda )\end{matrix}\right)$, \ где
$z_{11}(\lambda )=1$,  $z_{12}(\lambda )=-\frac 1{q-\lambda }$,  $z_{21}(\lambda
)=-\frac{\overset{m_0}{\underset{j=1}{\sum }}c_{\mathit{j0}}\text{exp}\{\text{$\lambda \tau
$}_{\mathit{j0}}\}}{\overline{a_{10}}-q-\lambda }-c_p\overset m{\underset{i=1}{\sum }}.$, \  $z_{22}(\lambda
)=1-\frac{\gamma _0}{\overline{a_{10}}-q-\lambda }-\frac{c_p\overset m{\underset{i=1}{\sum
}}A_{1i}}{\sqrt{\overline{a_{10}}-q-\lambda }}.$

Тогда из оценок \eqref{kri-eq-11} \ с учетом, что  $\|b_i\|_{k_{2p}^1}\le \|b\|_{k_{2p}^2},\text{  }i=1,2$ получаем

\begin{equation}\label{kri-eq-12}
	Z(\lambda
	)\overline y\le e\|b\|_{k_{2p}^2}+\overline{M(\lambda )}\overline{\overline{\varphi
	}}.
\end{equation}

Очевидно также, что  $Z(0)=Z$. В силу условий теоремы матрица  $Z$ является  $M$-\linebreak матрицей, а тогда при достаточно
малых  $\lambda $ \ матрица  $Z(\lambda )$ \ также является  $M${}-матрицей, а значит существует  $\lambda =\lambda _0$
\ такое, что \  $Z(\lambda _0)$ $\lambda$-положительно обратима. Тогда из неравенства \eqref{kri-eq-12} получаем

\begin{equation}\label{kri-eq-13}
	|\overline y|\le K(\|b\|_{k_{2p}^2}+\overline{\overline{\varphi
	}}),
\end{equation}
где  $K=||Z(\lambda _0)^{-1}||\text{max}\{1,M(\lambda _0)\}.$

Поскольку  $x(t,b,\varphi )=\text{exp}-\beta ty(t)$  и   $\underset{t\ge
0}\sup\left(E|y(t)|^{2p}\right)^{1/2p}=|\overline y|$,  $\overline{\overline{\varphi
}}=\begin{matrix}\sup\limits_{t<0}\end{matrix}\left(E|\varphi (t)|^{2p}\right)^{1/2p}$, то из неравенства \eqref{kri-eq-13} следует,
что существуют положительные числа  $\lambda =\lambda _0$  и \linebreak  $K=||Z(\lambda _0)^{-1}||\text{max}\{1,M(\lambda _0)\}$
такие, что для решения  $x(t,b,\varphi )$ \ задачи \eqref{kri-eq-1}, \eqref{kri-eq-1a}, \eqref{kri-eq-1b} \ справедливо неравенство

\begin{equation*}
(E|x(t,b,\varphi )|^{2p})^{1/2p}\le
K\left(\|b\|_{k_{2p}^2}+\mathrm{vrai} \sup\limits_{t<0}(E|\varphi(t)|^{2p})^{1/2p}\right)\text{exp}\{-\lambda _0 t\}\text{  }(t\ge 0).
\end{equation*}
\end{proof}

\section{Следствие основного результата}

\begin{corollary}
Если существует положительное число  $q$ \ такое, что для \ уравнения \eqref{kri-eq-1} имеет место
неравенство

\begin{equation}\label{kri-eq-14}
	1-\frac{\gamma _0}{\overline{a_{10}}-q}-\frac{c_p\overset m{\underset{i=1}{\sum
		}}A_{1i}}{\sqrt{\overline{a_{10}}-q}}-\frac 1 q\left[\frac{\overset{m_0}{\underset{j=1}{\sum
		}}c_{\mathit{j0}}}{\overline{a_{10}}-q}+\frac{c_p\overset m{\underset{i=1}{\sum }}\left[\gamma
		_i\overline{y_1}+\overset{m_i}{\underset{j=1}{\sum
		}}c_{\text{ji}}\right]}{\sqrt{\overline{a_{10}}-q}}\right]>0,
\end{equation}
то оно экспоненциально  $2p${}-устойчиво.
\end{corollary}

Справедливость следствия следует из теоремы в силу того, что при выполнении условий следствия матрица  $Z$
будет  $M${}-матрицей, так как все диагональные миноры матрицы  $Z$
положительны.

Используя следствие можно получить достаточные условия экспоненциальной  $2p$-устойчивости для конкретных уравнений
вида \eqref{kri-eq-1} в терминах параметров этих уравнений. 