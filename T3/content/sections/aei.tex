


\chapter{ Численный метод построения  радиально-симметричного
 решения задачи Дирихле  в кольцевой области для  одного класса нелинейных
  дифференциальных уравнений  второго порядка}


%Аннотация


%Доказано существование и единственность положительного
%радиально-симметричного решения задачи Дирихле в кольцевой
%области для одного класса нелинейных дифференциальных
%уравнений второго порядка и разработан численный метод его построения.










%%%%%%%%%%%%%%%%%%%%
%%%%%%%%%%%%%%%%%%%%
%%%%%%%%%%%%%%%%%%%%


\section{Постановка задачи}

           Пусть $ D={x\in R^2: 1<r<2}-$  кольцевая область,
$ r=|x| $.    Рассмотрим задачу Дирихле
\begin{equation}\label{aei1_1}
\Delta u+r^kv^p=0, \qquad 1<r<2,
\end{equation}
\begin{equation}\label{aei1_2}
u|_{r=1}=u|_{r=2}=0.
\end{equation}
Здесь $ k \geq 0, p>1$ --- вещественные числа.

Очевидно, $ u \equiv 0$ -- тривиальное решение задачи \eqref{aei1_1}, \eqref{aei1_2}. Под
положительным решением задачи \eqref{aei1_1}, \eqref{aei1_2} понимается функция $ u \in
C^2(\overline {D})$, положительная в $D$, удовлетворяющая
уравнению \eqref{aei1_1} и граничным условиям\eqref{aei1_2}.

Положительным решениям уравнений вида \eqref{aei1_1} посвящено много
работ российских и зарубежных математиков
(см., например, \cite{aeiL_1, aeiL_2, aeiL_3, aeiL_4, aeiL_5, aeiL_6, aeiL_7, aeiL_8, aeiL_9, aeiL_10, aeiL_11, aeiL_12}). 
Во многих из них изучаются в основном
вопросы существования положительного решения, его поведение,
априорные оценки и другие. Публикаций, посвященных
единственности положительного решения задачи Дирихле
для уравнений вида \eqref{aei1_1} с $ p>1$, сравнительно мало.
Задача Дирихле для уравнения вида \eqref{aei1_1} в кольцевой области
изучалась также в \cite{aeiL_1} и \cite{aeiL_11}. Но, в них доказано существование
по крайней мере одного положительного радиально-симметричного
решения. Доказательство единственности положительного решения
 задачи Дирихле для уравнений вида \eqref{aei1_1} представляет
значительные трудности, которые можно объяснить наличием
тривиального решения $ u \equiv 0 $  и тем, что $ p>1 $.
Единственность положительного радиально-симметричного решения
задачи Дирихле в шаровой области для  уравнения \eqref{aei1_1}  без
ограничений на $ p>1 $ при $ n=2 $  и с некоторым
ограничением на $ p>1 $  при $ n>2 $  доказана в работах
автора \cite{aeiL_12} и  \cite{aeiL_13} соответственно.

В настоящем разделе эти результаты существования и единственности при $n=2$ распространяются на кольцевую область. В этом случае не требуются
никакие ограничения на показатель $ p>1 $. Кроме того разработан численный метод построения положительного радиально симметричного решения задачи Дирихле \eqref{aei1_1} 
, \eqref{aei1_2}.






%%%%%%%%%%%%%%%%%%%%
%%%%%%%%%%%%%%%%%%%%
%%%%%%%%%%%%%%%%%%%%





\section{ Существование и единственность положительного радиально-симметричного решения}

Докажем, что задача \eqref{aei1_1}
, \eqref{aei1_2} имеет единственное положительное
радиально-симметричное решение $ u \in C^2(\overline {D}) $
при любом  $ p>1 $.

 \subsection{Вспомогательные предложения}

 Рассмотрим уравнение
\begin{equation}\label{aei1_3}
v^{\prime\prime}+r_0^{k+2}e^{(k+2)t}v^p, \quad t>0 
\end{equation}
с начальными условиями
\begin{equation}\label{aei1_4}
v(0)=0, \quad v^{\prime}(0)=r_0,
\end{equation}
где $ r_0$ --- произвольное положительное число.

Из уравнения \eqref{aei1_3} следует, что $ v^{\prime\prime} \leq 0 $
при $ t \geq 0, $  т.е. функция $ v(t) $  выпукла вверх
при $ t \geq 0 $ . Поэтому в силу $ v^{\prime}(0)=r_0 $
при любом $ r_0 >0 $  существуют $ t_0 $ и $ \delta_0 $
такие, что $ \delta_0=v(t_0)>v(t)>0 $  при $ t \in (0,t_0). $

\begin{lemma}\label{aeiR_1}
\textit{ При любом $ r_0>0 $  в \eqref{aei1_3}, \eqref{aei1_4} существует
единственное число $ t^*>0 $  ,  зависящее лишь от $ k,p $ и
$ r_0 $   такое, что задача Коши \eqref{aei1_3}, \eqref{aei1_4} имеет единственное
решение $ v \in C^2[0,t^*] $  такое, что $ v(t)>0 $  при
$ t \in (0,t^*), \quad v(t^*)=0. $ }
\end{lemma}
  
\begin{proof}
 
Интегрируя два раза уравнение \eqref{aei1_3} 
 с учетом начальных условий \eqref{aei1_4} , получим

\begin{equation}\label{aei1_5}
v(t)=r_0t-{r_0}^{k+2} \int_0^t(t-s)e^{(k+2)s}v^p(s)ds.
\end{equation}
Из уравнения \eqref{aei1_3}  следует, что $ v^{\prime\prime}(t)<0 $ в тех
точках,где $ v(t) \neq 0 $ . 
Следовательно, положительное решение
задачи \eqref{aei1_3} , \eqref{aei1_4}  выпукло вверх. Как отмечено выше, существуют
положительные числа $ t_0 $ и $ \delta_0 $ такие, что $
\delta_0=v(t_0)>v(t)>0 $  при $ t \in (0,t_0)$. Предположим
противное, т.е. $ v(t)>0 $  при всех $ t>0 $ . Пусть $ t_1$ --
некоторое число. Из положительности и выпуклости вверх $ v(t) $
следует, что

\begin{equation}\label{aei1_6}
 v(\frac{t_0+t_1}{2})
\geq \frac{v(t_0)}{2}+\frac{v(t_1)}{2}>\frac{\delta_0)}{2} 
\end{equation}
и $ v(s) \geq \frac {s}{t}v(t) $ для любого $ s>0 $ .
Поэтому из \eqref{aei1_5} получаем
$$
v(t) \leq r_0t-\frac{r_0^{k+2}v^p(t)}{t^p}
\int_0^t(t-s)e^{(k+2)s}s^pds \leq
$$
$$
 \leq r_0t-\frac{r_0^{k+2}v^p(t)}{t^p}
 \int_0^t(t-s)s^pds=
  r_0t-\frac{r_0^{k+2}t^2}{(p+1)(p+2)}v^p(t).
$$
Отсюда имеем
$$
v(t)+\frac{r_0^{k+2}t^2}{(p+1)(p+2)}v^p(t) \leq r_0t.
$$
Следовательно,
\begin{equation}\label{aei1_7}
v(t)
\leq {\left [\frac {(p+1)(p+2)}{r_0^{k+1}t}\right ]}^
{\frac{1}{p}}.   
\end{equation}
Отсюда, полагая $ t=t_0,  $ получаем
$$
t_0 \leq \frac {(p+1)(p+2)}{r_0^{k+1}t\delta_0^p}
$$
При $ t=\frac{t_0+t_1}{2} $ из \eqref{aei1_7} в силу \eqref{aei1_6} имеем
$$
\frac {\delta_0}{2}<
{\left [\frac {(p+1)(p+2)}{r_0^{k+1}(t_0+t_1)}\right ]}^
{\frac{1}{p}}.
$$
Полагая здесь
$$
t_1=\frac {2^p(p+1)(p+2)}{r_0^{k+1}t\delta_0^p}-t_0>t_0,
$$
имеем $ \frac{\delta_0}{2}<\frac{\delta_0}{2}$. 
Получили противоречие. Следовательно, существует точка $ t^*, $
в которой $ v(t) $ обращается в нуль. В силу выпуклости
$ v(t) $  вверх точка $ t^* $  единственная. Кроме того,
$ v(t)>0 $   при $ t \in (0,t^*) $. Из \eqref{aei1_5} следует, что
 при $ 0 \leq t \leq t^* $  справедливо неравенство
$ 0 \leq v(t) \leq t, $ т.е. решение задачи Коши \eqref{aei1_3}, \eqref{aei1_4}
ограничено на отрезке $ [0,t^*] $. Поэтому $ v \in C^2[0,t^*]$
\end{proof}

\begin{lemma}\label{aeiR_2}
\textit{ Функция $ t^*(r_0) $ , определенная в лемме1,
является непрерывной монотонно убывающей по $ r_0 $ , причем, $
\displaystyle \lim_{r_0 \to +\infty}t^{*}(r_0)=0 $  и $
\displaystyle \lim_{r_0 \to 0}t^{*}(r_0)=+\infty.$}
\end{lemma}
\begin{proof}
Пусть $ v(t,r_0) $ решение 3адачи Коши \eqref{aei1_3},
\eqref{aei1_4}. Обозначим $ w(t,r_0)=\frac {\partial v(t,r_0)}{\partial r_0} $. 
Дифференцируя по второму аргументу $ r_0 $ уравнение \eqref{aei1_3},
начальные условия \eqref{aei1_4} и равенство $ v(t^*,r_0)=0 $, получим
\begin{equation}\label{aei1_8}
w^{\prime\prime}+(k+2)r_0^{k+1}e^{(k+2)t}v^p+pr_0^{k+2}e^{(k+2)t}v^{p-1}w=0
\end{equation}
\begin{equation}\label{aei1_9}
w(0)=0, \quad w{ \prime}(0)=0,
\end{equation}
\begin{equation}\label{aei1_10}
w(t^*)=0.
\end{equation}
Здесь  для сокращения записи приняты обозначения

$$
w^{\prime\prime},\quad v,\quad w(0),\quad w^{\prime}(0),\quad w(t^*)
$$
 вместо
 $$ \quad \frac {\partial^2w(t,r_0)}{\partial r_0^2},
\quad v(t,r_0),\quad w(t,r_0),\quad w(0,r_0),\quad \frac {\partial
w(0,r_0)}{\partial r_0},\quad w(t^*,r_0)
$$
соответственно.

Полагая в \eqref{aei1_5} $ t=t^* $  и учитывая, что $ v(t^*)=0 $ , имеем

$$
t^*=r_0^{k+1}\int_0^{t^*}(t^*-s)e^{(k+2)s}v^p(s)ds
$$
или
\begin{equation}\label{aei1_11}
r_0^{k+1}=\frac{t^*}{\int_0^{t^*}(t^*-s)e^{(k+2)s}v^p(s)ds}.
\end{equation}
Отсюда следует, что функция $
\Phi(r_0,t^*)=r_0^{k+1}-\frac{t^*}{\int_0^{t^*}(t^*-s)e^{(k+2)s}v^p(s)ds}$
непрерывна при всех $ r_0>0 $ и $ t^*>0 $ и $ \frac{\partial
\Phi(r_0,t^*)}{\partial r_0}>0, $ т.е. при постоянном $ t^* $
функция $ \Phi(r_0,t^*)$ монотонно возрастает по $ r_0 $. Тогда по
известной теореме о неявной функции ( см. \cite{aeiL_15}, стр.449) существует
непрерывная неявная функция $ t*(r_0). $  Дифференцируя равенство
\eqref{aei1_11} по $ t^* $, получим
$$
(k+1)r_0^k \frac{dr_0}{dt^*}= \frac
{\int_0^{t^*}(t^*-s)e^{(k+2)s}v^p(s)ds- t^*
\int_0^{t^*}e^{(k+2)s}v^p(s)ds} { \left (
\int_0^{t^*}(t^*-s)e^{(k+2)s}v^p(s)ds \right )^2}=
$$
$$
=-\frac {\int_0^{t^*}se^{(k+2)s}v^p(s)ds}{ \left (
\int_0^{t^*}(t^*-s)e^{(k+2)s}v^p(s)ds \right )^2}.
$$
Отсюда следует, что $ \frac{dr_0}{dt^*} <0 $, т.е. $ r_0 $ убывает с
возрастанием $ t^* $. Тогда, очевидно, убывает и функция $ t^*(r_0)
$ с возрастанием $ r_0 $. Из \eqref{aei1_5} следует, что при $ 0 \leq t \leq
t^* $ справедливо неравенство $ v(t) \leq r_0t$. 
Тогда из \eqref{aei1_11} имеем
$$
{r_0}^{k+1} \geq \frac{1}{\int_0^{t^*}e^{(k+2)s}{r_0}^ps^pds}
$$
Отсюда получим
\begin{equation}\label{aei1_12}
{r_0}^{k+1+p} \geq \frac{1}{\int_0^{t^*}e^{(k+2)s}s^pds} >
\frac{p+1}{e^{(k+2)t^*} (t^*)^{p+1}}. 
\end{equation}
В силу монотонного убывания  функции $ t^*(r_0) $ отсюда следует
$$
\displaystyle \lim_{r_0 \to +\infty}t^{*}(r_0)=0, \displaystyle
\lim_{r_0 \to 0}t^{*}(r_0)=+\infty.
$$
\end{proof}

 \subsection{Единственность положительного радиально-симметричного решения}

Радиально-симметричное решение задачи \eqref{aei1_1}, \eqref{aei1_2} удовлетворяет
уравнению
\begin{equation}\label{aei1_13}
u^{\prime\prime}+\frac{u^{\prime}}{r}+r^ku^p=0, \quad 1<r<2
\end{equation}
и краевым условиям
\begin{equation}\label{aei1_14}
u(1)=u(2)=0. 
\end{equation}

\begin{theorem}
\textit{ При любых $ k \geq 0$, $p>1 $ задача \eqref{aei1_1}, \eqref{aei1_2} имеет
единственное положительное радиально-симметричное решение.}
\end{theorem}

\begin{proof}
С помощью преобразования Ц.На \cite{aeiL_14}

\begin{equation}\label{aei1_15}
\left \{
\begin{array}{l}
 r =A^{\alpha}s, \\
 u =A^{\beta}v,
\end{array} \right.
\end{equation}
где $ \alpha,\beta- $ некоторые вещественные числа, $ A$  -- числовой
параметр, уравнение \eqref{aei1_13} приводится к виду
$$
A^{\beta-2\alpha}v^{\prime\prime}+\frac{A^{\beta-2\alpha}v^{\prime}}{s}+A^{k\alpha+p\beta}s^k|v|^p=0.
$$
Выберем здесь показатели   равными между собой
\begin{equation}\label{aei1_16}
\beta-2\alpha=k\alpha+p\beta.
\end{equation}
Тогда получим уравнение
$$
v^{\prime\prime}+\frac{v^{\prime}}{s}+s^kv^p=0.
$$
Уравнение \eqref{aei1_13} оказалось инвариантным относительно преобразования
\eqref{aei1_15}.

Обозначим через $ A $  недостающее начальное условие
$$
u^{\prime}(1)=A.
$$
В координатах \eqref{aei1_15} это условие примет вид
$$
A^{\beta-\alpha}v^{\prime}(A^{-\alpha})=A.
$$
Положив здесь
\begin{equation}\label{aei1_17}
\beta-\alpha=1,
\end{equation}
получим
$$
v^{\prime}(A^{-\alpha})=1.
$$
Условие $ u(1)=0 $ в координатах \eqref{aei1_15}  примет вид $
v(A^{-\alpha})=0$.

Таким образом, $ v $  является решением задачи Коши
\begin{equation}\label{aei1_18}
v^{\prime\prime}+\frac{v^{\prime}}{s}+s^kv^p=0, \quad s>r_0,
\end{equation}
\begin{equation}\label{aei1_19}
v(r_0)=0, \quad v^{\prime}(r_0)=1,
\end{equation}
где
\begin{equation}\label{aei1_20}
r_0=A^{-\alpha}. 
\end{equation}
Из \eqref{aei1_16} и \eqref{aei1_17} параметры $\alpha $   и $ \beta $  определяются однозначно
$$
\alpha=\frac{1-p}{k+p+1}, \quad \beta=\frac{k+2}{k+p+1}.
$$
Сделаем замену
\begin{equation}\label{aei1_21}
t=\ln{\frac{s}{r_0}}. 
\end{equation}
Задача Коши \eqref{aei1_18}, \eqref{aei1_19} после этой замены приводится к задаче \eqref{aei1_3},
\eqref{aei1_4}.

 По лемме \ref{aeiR_1} существует единственное число $ t_1^*>0 $  такое, что задача \eqref{aei1_3},
\eqref{aei1_4} имеет единственное
 решение $ v \in C^2[0,t_1^*] $   такое, что $ v(t_1^*)=0,v(t)>0 $  при $ t \in (0,t_1^*)$ .
 Следовательно, в силу \eqref{aei1_21} существует единственная точка $ s_1^*=r_0e^{t_1^*}$
  такая, что на отрезке $ [r_0,s_1^*] $  задача \eqref{aei1_18},
\eqref{aei1_19} имеет единственное решение $ v \in C^2[r_0,s_1^*] $  и
\begin{equation}\label{aei1_22}
v(s_1^*)=0.
\end{equation}
В соответствии с формулами \eqref{aei1_15} значению $ s=r_0 $  соответствует
значение $ r_1=A^{\alpha }r_0 $в силу \eqref{aei1_20}, а значению $ s=s_1^*-$
значение $ r_2=A^{\alpha}s_1^*=\frac{s_1^*}{r_0}$. 
Выберем параметр
$ r_0 $ так, чтобы $ r_2=\frac{s_1^*}{r_0}=e^{t_1^*(r_0)}=2 $. 
Отсюда имеем
\begin{equation}\label{aei1_23}
t_1^*(r_0)=\ln(2).
\end{equation}
Так как по лемме \ref{aeiR_2} $ t_1^*(r_0)$  непрерывная убывающая функция и $
\displaystyle \lim_{r_0 \to 0}t_1^*(r_0)=+\infty$, $\displaystyle \lim_{r_0 \to +\infty}t_1^*(r_0)=0 $,  то уравнение
\eqref{aei1_23} имеет единственное решение $ r_0=r_0^*$. Тогда
$$
\left (\frac{2}{s_1^*}\right )^{\frac{1}{\alpha}}=\left
(\frac{2}{r_0}e^{-t_1^*}\right )^{\frac{1}{\alpha}}.
$$

В силу \eqref{aei1_15}, \eqref{aei1_20}  и \eqref{aei1_22}
\begin{equation}\label{aei1_24}
\begin{cases}
 u(1)=A^{\beta}v(A^{-\alpha})= A^{\beta}v(r_0)=0,\\
 u(2) =A^{\beta}v(2A^{-\alpha})=A^{\beta}v(s_1^*)=0.
\end{cases}      
\end{equation}
Так как по лемме \ref{aeiR_1} задача Коши  \eqref{aei1_3}, \eqref{aei1_4} имеет единственное
положительное решение $ v \in C^2[0,t^*] $  и $ v(0)=v(t^*)=0 $, то
в силу однозначности отображений \eqref{aei1_15}, \eqref{aei1_21} во множестве
положительных чисел и в силу \eqref{aei1_24}  задача \eqref{aei1_13}, \eqref{aei1_14} имеет
единственное положительное решение. Следовательно, задача \eqref{aei1_1}, \eqref{aei1_2}
имеет единственное  положительное радиально-симметричное решение.
\end{proof}



%%%%%%%%%%%%%%%
%%%%%%%%%%%%%%%
%%%%%%%%%%%%%%%




\section{ Численный метод построения решения}
Введем обозначение
\begin{equation}\label{aei1_25}
c_0=\left [\frac{p+1}{2^{k+2}(\ln2)^{p+1}}\right ]^{\frac{1}{k+p+1}}.
\end{equation}
Тогда справедлива

 \begin{lemma}\label{aeiR_3}
 \textit{ Если $ r_0> c_0, $ то $ t^* >\ln2,$ где $ t^* -$ точка, в которой  решение задачи Коши \eqref{aei1_3}, \eqref{aei1_4} обращается в нуль по лемме \ref{aeiR_1} при $ r_0=c_0 $. }
 \end{lemma}

\begin{proof}
Из  \eqref{aei1_12} при $ t^*=\ln2 $ имеем
$$
{r_0}^{k+1+p} >\frac{p+1}{2^{(k+2)} (\ln)^{p+1}}.
$$
Отсюда следует
$$
r_0>\left [\frac{p+1}{2^{k+2}(\ln2)^{p+1}}\right ]^{\frac{1}{k+p+1}}\equiv c_0.
$$
Так как $t^*(r_0) -$ убывающая функция от $+\infty $ до $-\infty$ по лемме \ref{aeiR_2}, то $ \ln2 < t^*(c_0)$.

\end{proof}

С помощью полученных результатов положительное радиально-симметричное решение задачи Дирихле \eqref{aei1_1}, \eqref{aei1_2} можно построить неитерационным методом по алгоритму, состоящему из следующих шагов
\bigskip

\centerline\textit{ $ r_0:=c_0- $ константа, определенная формулой \eqref{aei1_25}}.

\underline{\textbf{ Шаг 1. }}
\textit{ Решаем задачу коши
\begin{equation*} %\label{1.6}
v^{\prime\prime}+r_0^{k+2}e^{(k+2)t} {\vert v \vert}^p,t>0
\end{equation*}
 с начальными условиями
\begin{equation*} %\label{1.7}
v(0)=0, v^{\prime}(0)=r_0,
\end{equation*}
методом Рунге-Кутта 4-го порядка с некоторым малым шагом $ h,$ начиная с $ t=0,$ до тех пор, пока по лемме \ref{aeiR_1} найдем точку $ t^*$ такую, что $ v(t^*)=0.$  После этого проверяем условие:}

$$
|t^*-\ln2| \leq\varepsilon.
$$
\textit{Если да, то  запоминаем $ r_0 $ и переходим к \textbf{ Шагу 2}, если нет, то $ r_0:=r_0+\delta$ и переходим к \textbf{ Шагу 1}.  (Здесь $\varepsilon, \delta - $ достаточно малые числа). }

\underline{\textbf{ Шаг 2. }}
$$
A:=\left (r_0 \right )^{\frac {k+p+1} {p-1}}.
$$
\textit{ Решаем задачу Коши
$$
u^{\prime \prime}+\frac{u^{\prime}}{r} +r^ku^p=0, 1<r<2,
$$
$$
u(1)=u(2)=0,
$$
начиная с $r=1$ до $r=2.$}


Таким образом полученное решение есть положительное радиально-симметричное решение задачи Дирихле \eqref{aei1_1}, \eqref{aei1_2}.
При $k=0$, $p=2 $ по приведенной схеме алгоритма получено следующее решение
положительное решение
\vspace{0.5cm}


 \centerline{\begin{tabular}{ | l | l | }
\hline
r & u  \\ \hline
1. 00 & 0.000  \\
1.10 & 3.931 \\
1.20 & 7.351 \\
1.30 & 9.970 \\
1.40 & 11.446 \\
1.50 & 11.577 \\
1.60 & 10.426 \\
1.70 & 8.304 \\
1.80 & 5.633 \\
1.90 & 2.788 \\
2.00 & 0.000 \\
\hline
\end{tabular}}

