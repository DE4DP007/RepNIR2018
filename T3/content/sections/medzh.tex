\chapter{Меджидов З.Г.}

\section{Научные результаты}

%1. 
Рассмотрим двухпараметрическое семейство равнобочных гипербол 
\begin{equation*}
\Gamma \left(x,y\right)=\left\{\left(\xi ,\eta \right):\left(\eta -c\right)^2-(\xi -x)^2=(c-y)^2,\quad 0{\leq}\eta
{\leq}y{\leq}h\right\},\quad c>h>0,
\end{equation*}
на евклидовой плоскости  $R^2$, параметризованных абсциссами центров  $x$ и полуосями  $c-y$. Центры гипербол лежат на
прямой  $\eta =c$. В полосу 
\begin{equation*}
\Omega _h=\{\left(\xi ,\eta \right):-{\infty}<\xi <{\infty},\quad 0{\leq}\eta {\leq}h\}
\end{equation*}
попадает лишь часть нижней ветви  $\eta =c-\sqrt{\left(c-y\right)^2-\left(\xi -x\right)^2}$ \ каждой гиперболы.

Решается задача определения неизвестной функции  $f(x,y)$ \ по заданным интегралам вдоль частей нижних ветвей гипербол
семейства  $\Gamma \left(x,y\right)$, попадающих в полосу  $\Omega _h$:

\begin{equation}\label{medzh-eq-1}
\int _{\Gamma \left(x,y\right){\cap}\Omega _h}^{}\left(c-\eta \right)^nf\left(\xi ,\eta \right)\mathit{d\xi
}=g\left(x,y\right),\quad n{\geq}1.
\end{equation}
Функция  $f(x,y)$ \ ищется в классе непрерывных финитных функций, носители которых лежат в полосе  $\Omega _h$. 

Доказана следующая 

\textbf{Теорема 1}. \textit{Пусть функция } $g\left(x,y\right)$ \textit{\ известна для всех } $\left(x,y\right)$
\textit{\ из полосы } $\Omega _h$\textit{. Тогда уравнение \eqref{medzh-eq-1} в классе дважды непрерывно дифференцируемых функций с
	носителем в полосе } $\Omega _h$ \textit{\ имеет единственное решение.}

%2. 
Пусть  $\widehat  f(\lambda ,\eta )$ \ и  $\widehat  g\left(\lambda ,y\right)$\ – одномерные преобразования Фурье
искомой функции  $f(\xi ,\eta )$ \ и определенной по формуле \eqref{medzh-eq-1} функции  $g(x,y)$ \ по первому аргументу. Получены
явные формулы для преобразования Фурье  $\widehat  f\left(\lambda ,y\right)$: 


\bigskip

\begin{equation*}
\widehat  f\left(\lambda ,y\right)=\frac 1{\pi \left(c-y\right)^{n+1}}\frac{{\partial}}{{\partial}y}\int
_0^y\frac{\left(c-\eta \right)\mathit{ch}\left(\lambda \sqrt{\left(c-\eta
		\right)^2-\left(c-y\right)^2}\right)}{\sqrt{\left(c-\eta \right)^2-\left(c-y\right)^2}}\widehat  g\left(\lambda ,\eta
\right)\mathit{d\eta }.
\end{equation*}
\begin{equation*}
\widehat  f\left(\lambda ,y\right)=\frac 1{\pi \left(c-y\right)^n}\int _0^y\frac{\mathit{ch}\left(\lambda
	\sqrt{\left(c-\tau \right)^2-\left(c-y\right)^2}\right)}{\sqrt{\left(c-\tau \right)^2-\left(c-y\right)^2}}\widehat 
g_{\tau }'\left(\lambda ,\tau \right)d\tau .
\end{equation*}
%3. 
Пусть  $S_R=\left\{\left(x_1,x_2\right){\in}R^2:x_1^2+x_2^2{\leq}R^2\right\}$\ – круг радиуса  $R$ \ с центром в
начале координат на евклидовой плоскости. В круге  $S_R$ \ рассмотрим двухпараметрическое семейство ломаных 

\begin{equation*}
\Gamma \left(\beta ,d\right)=L_d{\cup}L,0{\leq}\beta {\leq}2\pi ,0{\leq}d{\leq}R,
\end{equation*}
где 

\begin{equation*}
L_d=\left\{\left(x_1,x_2\right)=\left(R-s\right)\omega ,\quad 0{\leq}s{\leq}d\right\},\quad \omega
=(\cos \beta ,\sin \beta ),
\end{equation*}
\begin{equation*}
L=\left\{\left(x_1,x_2\right)=\left(R-d\right)\omega -s\left(\cos \left(\beta +\theta \right),\sin \left(\beta +\theta
\right)\right),\right.
\end{equation*}
\begin{equation*}
\left.0{\leq}s{\leq}\left(R-d\right)\sin \theta +\sqrt{R^2-\left(R-d\right)^2\sin ^2\theta }\right\};
\end{equation*}
здесь  $\theta $\ – фиксированный острый угол.

Решена задача: по заданным интегралам 

\begin{equation*}
\mathit{Vf}\left(\beta ,d\right)=\int _{\Gamma \left(\beta ,d\right)}^{}f(x)\mathit{ds},
\end{equation*}
от непрерывной функции  $f(x)$ \ по ломаным  $\Gamma \left(\beta ,d\right)$, определить эту функцию.

Данная задача решена при условии, что значения функции  $\mathit{Vf}\left(\beta ,d\right)$ \ известны лишь для значений
аргумента  $\beta $ \ из сколь угодно малого диапазона.

Пусть
\begin{equation*}
\mathit{Rf}\left(\beta ,r\right)=\int _{x{\cdot}\omega =r}^{}f(x)\mathit{dx}
\end{equation*}
-- преобразование Радона функции  $f(x)$,  $\widehat {\mathit{Rf}}$\ – его преобразование Фурье по второму аргументу.

\textbf{Теорема 2.} \textit{Пусть функция}  $\mathit{Vf}\left(\beta ,d\right)$ \ \textit{известна для значений
	аргументов}  $\left(\beta ,d\right),\left|\beta \right|<\alpha _0$ \ и  $\left|\pi -\beta \right|<\alpha
_0,0{\leq}d{\leq}2R$. \textit{Тогда преобразование Фурье } $\widetilde f$ \textit{\ функции } $f$\textit{,
	сосредоточенной внутри круга } $S_{R\sin \theta }$\textit{, восстанавливается по формуле}

% TODO Исправить формулу
\begin{equation*}
\widetilde f\left(y_1,y_2\right)=i\exp \left(R\sqrt{K^2y_2^2-y_1^2}\right){\cdot}\left(\int_{-\inf}^{-K\left|y_2\right|}\frac{\sin R \sqrt{z^2-K^2y^2}}{\pi(z - y_1)}\sigma\widehat{Rf}(\chi, \sigma)dz\right.
\end{equation*}
\begin{equation*}
\left.-\int _{K\left|y_2^{}\right|}^{{\infty}}\frac{\sin R\sqrt{z^2-K^2y_2^2}}{\pi \left(z-y_1\right)}\sigma \widehat
{\mathit{Rf}}\left(\chi ,\sigma \right)\mathit{dz}\right)
\end{equation*}
\textit{где } $K=\mathit{ctg}\alpha _0$\textit{, } $\sigma =\sqrt{z^2+y_2^2}$\textit{,} $\chi =\mathit{arctg}\frac{y_2}
z$\textit{, у квадратного корня выбрана ветвь, у которой } $\Re \sqrt{K^2y_2^2-y_1^2}>0$\textit{.}