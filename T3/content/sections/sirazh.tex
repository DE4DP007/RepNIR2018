
%Сиражудинов М.М.

\chapter{О G-компактности некоторых классов эллиптических операторов второго порядка}

%Аннотация
%
%В данном разделе
%рассматриваются вопросы $G$--сходимости некоторых классов недивергентных
%эллиптических операторов
%второго порядка на плоскости с комплекснозначными коэффициентами.


%\section{Введение}

$G$--сходимость недивергентных дифференциальных операторов --- это, иначе, слабая сходимость
соответствующих обратных операторов. Поэтому по понятным причинам в задачах
$G$--сходимости,
кроме корректной разрешимости краевых задач, требуются также оценки решений,
равномерные относительно любого оператора. Для недивергентных эллиптических
операторов и систем (к которым относятся и рассматриваемые здесь операторы)
такого рода оценки мало изучены, поэтому $G$--сходимость
недивергентных операторов изучена не столь детально как для дивергентных операторов.
Вопросам $G$--сходимости и усреднению недивергентных эллиптических операторов
посвящены работы \cite{smm_ZhS, smm_ZhS1, smm_Sir1,smm_Sir}.

\begin{list}{}{\leftmargin=10pt \itemsep=3pt}
\item\textbf{ Мы будем придерживаться следующих обозначений:}
\item $\mathbb R^2$ --- плоскость, $x\cdot y=x_1y_1+x_2y_2$ --- скалярное произведение.
\item $Q\subset \mathbb R^2$ --- ограниченная односвязная область.
\item $\overline{Q}$ --- замыкание области $Q$.
\item $\partial_{\bar z}=2^{-1}(\mathcal{D}_1+i\,\mathcal{D}_2)$,\quad $\partial_z=2^{-1}(\mathcal{D}_1-i\,\mathcal{D}_2)$,
\quad $\mathcal{D}_j=\frac{\partial}{\partial x_j}$,\quad
$j=1,\,2$.
\item $i$ --- мнимая единица.
\item $L_2(Q;\mathbb C)$ --- пространство Лебега комплекснозначных
квадратично суммируемых
функций. \textit{ Символ $\mathbb C$ }(\textit{ здесь и далее})
\textit{ в обозначении пространства означает также, что это
пространство есть линейное пространство над полем действительных чисел $\mathbb R$}.
Скалярное произведение в $L_2(Q;\mathbb C)$ дается равенством:
$$
(u,v)_{L_2(Q;\mathbb C)}=\text
{Re}\,\int_Qu\bar v\,dx,\qquad u,\,v\in L_2(Q;\mathbb C),
$$
где $\bar v$ --- комплексно сопряженная $v$ функция.

Соответствие
$L_2(Q;\mathbb C)\ni u=u_1+iu_2\mapsto (u_1,u_2)=U\in(L_2(Q))^2$ есть
изоморфизм, причем $(U,V)_{(L_2(Q))^2}=(u,v)_{L_2(Q;\mathbb C)}$.
\item $W_p^k(Q)$ ($k\in\mathbb{N}$, $1\leqslant p<\infty$) --- обычное
пространство Соболева; $\overset{\!\!\circ}{W_p^{\smash[t]k}}(Q)$ --- подпространство
$W_p^k(Q)$, состоящее из элементов с нулевыми следами на границе.
\item $W_p^k(Q;\mathbb C)$ --- пространство Соболева комплекснозначных
функций.
\item $\rightharpoonup$ --- знак слабой сходимости в соответствующем пространстве.
\item В случае, когда это не вызывает недоразумений, как
уравнение, так и оператор краевой задачи обозначаем одним и тем же
символом.
\end{list}



%\section{Формулировка результатов}

Пусть $Q$  --- ограниченная односвязная область плоскости с гладкой
(класса $C^1$) границей, $W_0(Q;\, \mathbb{C})$ --- подпространство $W_2^2(Q;\, \mathbb{C})$,
причем $C_0^\infty(Q;\, \mathbb{C})\subset W_0(Q;\, \mathbb{C})$.
И пусть $\Lambda: W_0(Q;\, \mathbb{C})\to L_2(Q;\, \mathbb{C})$ ---
обратимый дифференциальный оператор (вида \eqref{smm1_1} ниже) с постоянными коэффициентами.
Обозначим через  $\mathcal{A}(\nu_0,\,\nu_1)=\mathcal{A}(\nu_0,\,\nu_1;\,Q)$
класс операторов, действующих из $W_0(Q;\, \mathbb{C})$ в $L_2(Q;\,\mathbb{C})$,  вида
 \begin{equation}\label{smm1_1}
     Au=\mu_1\partial_{z\,\bar{z}}^2 u+\mu_2\partial_{z\,z}^2 u+
  \mu_3\partial_{\bar{z}\,\bar{z}}^2 u +\mu_4\partial_{z\,\bar{z}}^2 \bar{u}+
  \mu_5\partial_{z\,z}^2 \bar{u}+\mu_6\partial_{\bar{z}\,\bar{z}}^2 \bar{u},\quad
  u\in W_0(Q;\,\mathbb{C}),
 \end{equation}
где $\mu_j$ $(j=1,2,\dots,6)$ --- измеримые комплекснозначные функции,
принадлежащие $L_2(Q;\,\mathbb{C})$ и удовлетворяющие в любой гладкой односвязной
подобласти $Q_1\subset Q$ условиям
 \begin{equation}\label{smm1_2}
      \left\|\Lambda u\right\|^2_{L_2(Q_1;\,\mathbb{C})}\leqslant \nu_0\,
      \text{Re}\, \int\limits_{Q_1}Au\, \overline{\Lambda\,u} dx,\\
 \end{equation}
\begin{equation}\label{smm1_3}
   \text{Re}\,\int\limits_{Q_1} Au\,\Lambda v dx\leqslant\nu_1 \left(\text{Re}\, \int\limits_{Q_1} Au\,
    \Lambda u dx\right)^{\frac12}\cdot \left\| \Lambda v\right\|_{L_2(Q_1;\,\mathbb{C})},\quad
\forall u,v\in W_0(Q;\,\mathbb{C}),
\end{equation}
где $\nu_0$, $\nu_1>0$ ---  постоянные.
\begin{theorem}  \label{smm1_teor1}
Пусть $A\in \mathcal{A}(\nu_0,\nu_1; Q)$ --- произвольный оператор. Тогда задача $Au=f$,
$u\in W_0(Q;\,\mathbb{C})$, однозначно разрешима для любого $f\in L_2(Q;\,\mathbb{C})$.
Причем имеют место оценки
$$
\lambda_0\left\| u\right\|_{W_2^2(Q_1;\,\mathbb{C})}\leqslant
 \left\| Au \right\|_{L_2(Q_1;\,\mathbb{C})} \leqslant
 \lambda_1 \left\| u\right\|_{W_2^2(Q_1;\,\mathbb{C})}
 \quad (\forall  u\in W_0(Q_1;\,\mathbb{C})) ,
$$
где $\lambda_0$, $\lambda_1$ --- константы, зависящие только от $\nu_0$, $\nu_1$.
\end{theorem}



Рассмотрим неоднородную задачу
$$
\left\{\begin{array}{l}
  Au=f\in L_2(Q;\,\mathbb{C}), \\[1mm]
u-v\in W_0(Q;\,\mathbb{C})
\end{array}\right.
    $$
где $v$ --- любой фиксированный элемент $W_2^2(Q)$.
Эта задача однозначно разрешима. Действительно, это
следует из разрешимости задачи $Az=f-Av$, $z\in W_0(Q;\,\mathbb{C})$.

%Дадим понятие G-сходимости.

\begin{definition}
Скажем, что последовательность  операторов $\{A_k\}$ из класса
$A(\nu_0,\nu_1; Q)$ G-сходится в области $Q$ к оператору $A \in A(\nu_0,	\nu_1; Q)$ (обозначение:
$A_k\overset{G}\longrightarrow A$), если имеет место слабая сходимость $ A_k^{-1}\rightharpoonup A^{-1}$. Иначе говоря, для любого $f\in L_2(Q;\,\mathbb{C})$ последовательность решений $u_k\in W_0(Q;\,\mathbb{C})$ задачи $A_ku_k=f$, $u_k\in W_0(Q;\,\mathbb{C})$ слабо сходится в $W_2^2(Q;\,\mathbb{C})$ к решению задачи $Au=f$, $u\in W_0(Q;\,\mathbb{C})$.
\end{definition}


Справедлива следующая теорема о $G$-компактности.
\begin{theorem} \label{smm1_teor2}
    Из любой последовательности $\{A_k\}\subset \mathcal{A}(\nu_0,\,\nu_1;\,Q)$
    можно выделить $G$-сходя-\linebreak щуюся подпоследовательность.
\end{theorem}
$G$-сходимость обладает свойством локальности, а именно имеет место
\begin{theorem} \label{smm1_teor3}
     Пусть $A_k\overset{G}{\to}A$  в области $Q$ и пусть $Q_1$ --- любая
     фиксированная односвязная гладкая подобласть $Q$.
     Тогда $A_k\overset{G}\longrightarrow A$ и в области $Q_1$.
\end{theorem}

Из $G$-сходимости следует сходимость "произвольных"\     решений. Точнее, имеет место
\begin{theorem}\label{teor4}
 Пусть $A_ku_k=f_k$, $f_k\to f$ в $L_2(Q;\,\mathbb{C})$, $u_k\rightharpoonup u $
 в $W_2^2(Q;\,\mathbb{C})$ и пусть $A_k\overset{G}\longrightarrow A$ в области $Q$.
 Тогда $Au=f$.
\end{theorem}

Доказательства этих теорем см. в \cite{smm_DZh}.






%%%%%%%%%%%%%%%%%%%%%%
%%%%%%%%%%%%%%%%%%%%%%
%%%%%%%%%%%%%%%%%%%%%%





\chapter{Оценки погрешности усреднения периодической задачи для уравнения Бельтрами}


%Аннотация
%
%
%В литературе известны два метода изучения различных аспектов усреднения недивергентных эллиптических операторов: метод интегрального тождества и метод асимптотических разложений. Эти методы различаются по математической технике и опираются на оценки типа оценок острого угла. При изучении скорости сходимости решения исходной задачи (зависящей от малого параметра $\varepsilon$) к решению усредненной задачи, предпочтительней второй метод. В работе асимптотическими методами получены оценки погрешности усреднения порядка $O(\varepsilon)$, где $\varepsilon$ – малый параметр, периодической задачи для уравнения Бельтрами.
%


%\section{Формулировка результатов}

Рассмотрим периодическую задачу:
\begin{equation}\label{smm2_1}
   \left\{\begin{array}{l}
A_\varepsilon u_\varepsilon\equiv\partial_{\bar{z}}u_\varepsilon+\mu^{\varepsilon}\,\partial_z u_\varepsilon
=f-\left<f\,\overline{p^\varepsilon}\right>,\quad f\in L_2(\square), \\[3mm]
   u_\varepsilon\in W^1_2(\square),\qquad \left<u\right>=0,
\end{array}\right.
\end{equation}
где   $\mu^\varepsilon=\mu(\varepsilon^{-1}x)$, $\mu(x)=\mu(x_1,x_2)$
 --- измеримая ограниченная комплекснозначная периодическая
 (периода $1$ по каждой переменной)
функция, удовлетворяющая условию эллиптичности
$
    \textrm{ vrai\,sup}_{x\in \mathbb{R}^n}\left|\mu(x)\right|\leqslant k_0 <1,
    %\mathop\textrm{ vrai\,sup}_{x\in \mathbb{R}^n}\left|\mu(x)\right|\leqslant k_0 <1,
$
 $k_0>0$ --- постоянная эллиптичности; $p^\varepsilon=p(\varepsilon^{-1}x)$,
$p=p(x)$, $\left\langle p\right\rangle=1$, --- базисный вектор ядра $A^\ast: L_2(\square)\to W^{-1}_2(\square)$, $A^* p=-\partial_z p-\partial_{\overline z} (\overline\mu p)$,      $p\in L_2 (\square )$.
Отметим, что ядро оператора $A^\ast$ одномерно (см. \cite{smm_ZhKO}).
Здесь
и ниже $\left\langle g\right\rangle$ --- среднее значение периодической функции, $\varepsilon=1/n$, $n\in \mathbb{N}$,
 --- малый параметр. Без такого ограничения на $\varepsilon$ решение задачи
 \eqref{smm2_1} может не быть периодической (периода $1$) функцией.
%
 Периодическая
 задача \eqref{smm2_1} однозначно разрешима для любой правой части
 $f\in L_2(\square)$ (см. \cite{smm_SDZ}).

 Пусть функция $f$ из правой части \eqref{smm2_1} принадлежит пространству $W_2^1(Q)$
и пусть $u_\varepsilon$ --- решение задачи \eqref{smm2_1}.
В качестве первого приближения к решению $u_\varepsilon$  задачи  \eqref{smm2_1} возьмем   функцию
$
u_1^\varepsilon(x)=u^0(x)+\varepsilon N(\varepsilon^{-1}x)\partial_zu^0(x)$,
где $N$ --- периодическое решение задачи на ячейке периодов
$AN\equiv \partial_{\bar z}N+\mu\partial_z N=\mu^0-\mu(x)$,
$N\in W_2^{1}(\square)$, $\left\langle N\right\rangle=0$; $u^0$ --- решение усредненной задачи: $A_0u^0\equiv \partial_{\overline z}u^0+\left\langle\mu\overline p\right\rangle\partial_zu^0=f-\left\langle f\right\rangle$, $u^0\in W_2^1(\square)$,
$\left\langle u^0\right\rangle=0$. Справедлива следующая

\smallskip

\begin{theorem}
\textit{Пусть функция  $f$ из  правой части задачи \eqref{smm2_1} принадлежит пространству $W^1_2(\square)$,  тогда имеют место оценки
$$
\|u_\varepsilon-u_1^\varepsilon\|_{W^1_2(\square)}\leqslant c\,\varepsilon\,\|f\|_{W^1_2(\square)},\quad
\|u_\varepsilon-u^0\|_{L_2(\square)}\leqslant c\,\varepsilon\,\|f\|_{W^1_2(\square)},
$$
где $c>0$ --- постоянная, зависящая только от постоянной эллиптичности $k_0$.}
\end{theorem}

Доказательство теоремы см. в \cite{smm_SS}.





%%%%%%%%%%%%%%%%%%%%%%
%%%%%%%%%%%%%%%%%%%%%%
%%%%%%%%%%%%%%%%%%%%%%




\chapter{Оценки погрешности усреднения недивергентных
эллиптических операторов второго порядка}

%\section{Формулировка результатов}

Рассмотрим задачу Дирихле
\begin{equation}\label{smm3_1}
  A_\varepsilon u_\varepsilon\equiv\sum\limits_{i,j=1}^na_{ij}^\varepsilon(x)\,
\frac{\partial^2u_\varepsilon}{\partial x_i\partial x_j}=f\in
L_2(Q), \qquad
   u_\varepsilon\in W_{2,0}^2(Q),
\end{equation}
где $\varepsilon>0$ --- малый параметр,  $a_{ij}^\varepsilon(x)=a_{ij}
(\varepsilon^{-1}x)$,
$i,j=1,\dots,n$, $W_2^{2,0}(Q)=W_2^2(Q)\cap \overset{\!\circ}{W_2^{\smash[t]1}}(Q)$,
$Q$ --- ограниченная выпуклая область
$\mathbb{R}^n$ c гладкой границей, $a_{ij}(x)=a_{ji}(x)$. Коэффициенты
 $a_{ij}(x)$  --- измеримые ограниченные
периодические
функции, удовлетворяющие условию равномерной эллиптичности
 и условию Кордеса во всем пространстве $\mathbb{R}^n$:
 \begin{equation}\label{smm3_2}
\nu\,|\xi|^2\leq\sum_{i,j=1}^na_{ij}(x)\,\xi_i\xi_j\leq\nu^{-1}\,|\xi|^2,
\qquad x,\,\xi\in\mathbb{R}^n,
\end{equation}
\vspace{-3mm}
\begin{equation}\label{smm3_3}
(n-1+\mu)\sum_{i,j=1}^na_{ij}^2(x)\leq\Big(\sum_{i=1}^na_{ii}(x)\Big)^2,
\qquad x\in \mathbb{R}^n,
\end{equation}
где $\nu, \mu<1$ --- положительные постоянные.

Решение задачи \eqref{smm3_1} сходится в $W_2^1(Q)$ к решению усредненной задачи
\begin{equation}\label{smm3_4}
A_0u_0\equiv \sum_{i,j=1}^na_{ij}^0\frac{\partial^2u_0}{\partial x_i
\partial x_j}=f\in L_2(Q),\quad u_0\in W_{2,0}^2(Q),
\end{equation}
 причем коэффициенты усредненного оператора \eqref{smm3_4} постоянные, определяемые  равенством $a_{ij}^0=\left<pa_{ij}\right>$,
$i,j=1,\dots,n$, где $p\in L_2(\square)$, ($p>0$, $\left\langle p\right\rangle=1$) --- базисный вектор ядра оператора, сопряженного к оператору периодической задачи:
$$
A u\equiv\sum\limits_{i,j=1}^na_{ij}\,
\frac{\partial^2u}{\partial x_i\partial x_j}=f\in
L_2(\square), \qquad
   u\in W_{2}^2(\square),
$$
$\square$ --- ячейка периодов.
Справедлива следующая

\begin{theorem}
Пусть правая часть $f$ задачи Дирихле \eqref{smm3_1} принадлежит пространству \linebreak $W^2_2(Q)$,  $Q$ --- выпуклая область с гладкой \textrm{(}класса $C^4$\textrm{)} границей, тогда имеют место оценки
\begin{equation}\label{smm3_5}
\|u_\varepsilon-u_1^\varepsilon\|_{W^2_2(Q)}\leqslant c\sqrt{\varepsilon}\,\|f\|_{W^2_2(Q)},\qquad
\|u_\varepsilon-u^0\|_{L_2(Q)}\leqslant c\sqrt{\varepsilon}\,\|f\|_{W^2_2(Q)},
\end{equation}
где $c>0$ --- постоянная, независящая от $\varepsilon$ и $f$.
\end{theorem}

Здесь $u_1^\varepsilon(x)$ --- первое приближение  к решению $u_\varepsilon$  задачи  \eqref{smm3_1}, оно определяется формулой
$$
u_1^\varepsilon(x)=u_0(x)+\varepsilon^2\sum_{i,j=1}^nN_{ij}(y)\frac{\partial^2u_0(x)}
{\partial x_i\partial x_j},\quad y=\varepsilon^{-1}x,
$$
где $N_{ij}(x)$, ($i,j=1,\dots,n$), --- решение периодической задачи
\begin{align*}
&AN_{km}\equiv\sum_{i,j=1}^na_{ij}(x)\frac{\partial^2N_{km}(x)}{\partial x_i\partial x_j}=
a_{km}^0-a_{km}(x),\\
&N_{km}\in W^2_2(\square),\quad \left<N_{km}\right>=0,\quad k,m=1,\dots,n.
\end{align*}

Следует отметить, что для малых размерностей ($n=2,3,4$) постоянная $c$ в
\eqref{smm3_5} зависит только от постоянной эллиптичности $\nu$ и постоянной $\mu$
из условия Кордеса \eqref{smm3_2}. Если размерность больше четырех, то $c$ зависит
еще от модулей непрерывности коэффициентов.
Эти результаты анонсированы в заметке \cite{smm_SM}. 