\Referat %Реферат отчёта, не более 1 страницы

Отчет содержит Х~с., Х~источников.%, 8~таблиц, 12~иллюстраций.

\bigskip
\textbf{ Ключевые
	слова:}
ДИФФЕРЕНЦИАЛЬНЫЕ УРАВНЕНИЯ; ЭЛЛИПТИЧЕСКИЕ ОПЕРАТОРЫ; СИСТЕМА НАВЬЕ-СТОКСА; $G$-СХОДИМОСТЬ; $G$-КОМПАКТНОСТЬ; УСРЕДНЕНИЯ ОПЕРАТОРОВ; НЕЛИНЕЙНЫЕ СИСТЕМЫ ДУ;
ПРЕОБРАЗОВАНИЕ РАДОНА;
$p$-ЛАПЛАСИАН; $p$-УСТОЙЧИВОСТЬ; ЗАДАЧА ДИРИХЛЕ; УРАВНЕНИЯ ИТО; УРАВНЕНИЯ С ПОСЛЕДЕЙСТВИЕМ;  УРАВНЕНИЯ БЕЛЬТРАМИ; ПОЛОЖИТЕЛЬНОЕ РЕШЕНИЕ; РАДИАЛЬНО-СИММЕТРИЧНОЕ РЕШЕНИЕ; ДРОБНЫЕ ПРОИЗВОДНЫЕ; РАЗНО-\linebreakСТНЫЕ УРАВНЕНИЯ;  ТЕОРИЯ УСТОЙЧИВОСТИ; МОМЕНТНАЯ УСТОЙЧИВОСТЬ; МЕТОД МОДЕЛЬНЫХ УРАВНЕНИЙ; ТЕОРЕМЫ ТИПА БОЛЯ-ПЕРРОНА; ЛУЧЕВОЕ.


%дифференциальные уравнения Ито, устойчивость решений, положительная обратимость матриц

\bigskip

Настоящий отчёт содержит итоги работы за 2017 год Отдела математики и информатики ДНЦ РАН по теме
<<Асимптотические методы усреднения недивергентных дифференциальных операторов. Исследование вопросов моментной устойчивости и устойчивости по части переменных для дифференциальных уравнений Ито с импульсными воздействиями и разностных уравнений Ито. Исследование вопросов существования и единственности решений краевых задач для нелинейных эллиптических уравнений с $p$- и $p(x)$-лапласианом. Лучевое преобразование векторных и тензорных полей и некоторые его обобщения>>
%осуществлению фундаментальных научных исследований в соответствии с
из Программы фундаментальных научных исследований государственных академий наук на 2013–2020 годы.


%%%%%%%%%%%%%%%%
%%%%%%%%%%%%%%%%
=====================



1) Получены оценки погрешности усреднения задачи Дирихле для недивергентных эллиптических операторов второго порядка, удовлетворяющих условию Кордеса на разброс собственных значений матрицы коэффициентов. При этом коэффициенты уравнения --- $\varepsilon$-периодические функции, где $\varepsilon$ --- малый параметр.

2) Получены оценки погрешности усреднения периодической задачи для уравнения Бельтрами, с $\varepsilon$-периодическим коэффициентом, где $\varepsilon=1/n$  --- малый параметр.

3) Выделены G-компактные классы недивергентных эллиптических операторов второго порядка с комплекснозначными коэффициентами.




%%%%%%%%%%%%%%%%
%%%%%%%%%%%%%%%%
=====================




За отчетный период были исследованы вопросы моментной устойчивости решений для линейных дифференциальных уравнений Ито с
запаздываниями высоких порядков и систем линейных дифференциальных уравнений Ито с запаздываниями специального вида на
основе теории положительно обратимых матриц. Для этого применяются идеи и методы, разработанные Н.В.Азбелевым и его
учениками для исследования вопросов устойчивости детерминированных линейных функционально-дифференциальных уравнений,
которые ранее были применены нами для исследований вопросов устойчивости для линейных дифференциальных уравнений Ито с
запаздываниями первого порядка. Получены достаточные условия моментной устойчивости исследуемых уравнений в терминах
\ в терминах положительной обратимости матриц, построенных по исходным уравнениям. Проверяется выполнимость этих
условий для конкретных уравнений. Для простоты остановимся на изложении результатов в случае линейных дифференциальных
уравнений Ито с запаздываниями второго порядка. Результаты исследований опубликованы в следующих работах,
опубликованных за отчетный период.




%%%%%%%%%%%%%%%%
%%%%%%%%%%%%%%%%
=====================




Метод усреднения дифференциальных операторов, основанный на асимптотических разложениях по малому параметру, широко используется  в математической и физической литературе. Этот метод позволяет помимо теоремы усреднения получить оценки разности точного решения и его приближений. Нами впервые к усреднению обобщенного уравнения Бельтрами привлечены асимптотические методы, что позволило получить теорему усреднения и оценки порядка $O(\sqrt{\varepsilon})$ разности точного решения и его приближений в нормах пространств Лебега и Соболева.


Эти оценки получены асимптотическими методами при минимальных предположениях гладкости на данные задачи:

\begin{itemize}
	\item
	коэффициенты --- измеримые ограниченные $\varepsilon$-периодические функции;
	\item
	граница области из класса $C^2$;
	\item
	правая часть из пространства Соболева $W_2^1$.
\end{itemize}

Получены операторные оценки усреднения обобщенных уравнений Бельтрами. Изучены вопросы гельдеровости решений задачи Римана-Гильберта для обобщенных систем уравнений Бельтрами при минимальных условиях на данные задачи.




%%%%%%%%%%%%%%% KRI


Применяя метод модельных уравнений, исследованы вопросы моментной устойчивости решений по части переменных относительно начальных данных для линейных импульсных систем  дифференциальных уравнений Ито с последействием.
%Получены достаточные условия устойчивости в терминах параметров исследуемых систем.
Получены достаточные условия устойчивости решений стохастических дифференциальных уравнений в терминах параметров этих уравнений.  Эти результаты могут быть применены при исследованию на  устойчивость развитие различных процессов биологии, физики, химии, экономики, подверженных случайным воздействиям.


Изучены вопросы  глобальной экспоненциональной $p$-устойчивости $(2 \le p < \infty)$ систем линейных дифференциальных уравнений Ито с запаздываниями специального вида, используя теорию положительно обратимых матриц. Для этого применяется идеи и методы, разработанная Н.В. Азбелевым и его учениками для исследования вопросов устойчивости для детерминированных функционально--дифференциальных уравнений. Получены достаточные условия глобальной экспоненциональной $p$-устойчивости $(2 \le p < \infty)$ систем нелинейных  дифференциальных уравнений Ито с запаздываниями в терминах положительной обратимости матрицы, построенной по исходной системе. Проверена  выполнимость этих условий для конкретных уравнений.


Исследованы вопросы моментной устойчивости решений  относительно начальных данных и допустимости пар пространств для линейных дифференциально--разностных уравнений Ито. Исследование проведено методом вспомогательных или модельных уравнений.







%%%%%%%%%%%%%%% AEI

Исследованы вопросы существования, единственности, построения решения численными методами решения задачи Дирихле для одного нелинейного дифференциального уравнения второго порядка с $p$-лапласианом.


Доказано существование и единственность решения двухточечной краевой задачи для одного семейства нелинейных обыкновенных дифференциальных уравнений четвертого порядка.


Исследованы вопросы существования, единственности, построения решения численными методами решения одной нелинейной двухточечной краевой задачи для одного нелинейного обыкновенного дифференциального уравнения с дробными производными.




%%%%%%%%%%%%%%% MZG

Введены в рассмотрение два двухпараметрических семейства ломаных на плоскости. Решена задача восстановления функции по ее интегралам вдоль этих ломаных, когда весовая функция – квазимногочлен. Для частных случаев весовых функций получены формулы обращения. В общем случае доказана единственность решения поставленной задачи. Результаты применены к доказательству единственности задачи интегральной геометрии с возмущением.


Доказана единственность восстановления функции, суммируемой в полосе на плоскости, заданной своими интегралами вдоль дуг двухпараметрических кривых второго порядка с весом, аналитическим по части переменных.

Доказаны формулы для определения неизвестного векторного поля на плоскости, заданного своим поперечным лучевым преобразованием в ограниченном угловом диапазоне. В первой формуле используется интегральная формула интерполяции функции с ограниченным спектром. Восстанавливается преобразование Фурье дивергенции неизвестного поля. Во второй формуле интерполяция производится по дискретным значениям функции. Восстанавливаются координатные функции потенциальной части искомого поля. Векторное поле ищется в классе вектор-функций, сосредоточенных в некоторой полосе, достаточно быстро убывающих на бесконечности и имеющих непрерывные вторые производные.


%Рассмотрена система функций $\mathcal{\psi}_{r,n}(x)$ $(r=1,2,\ldots, n=0,1,\ldots)$ ортонормированная по Соболеву относительно скалярного произведения  вида\linebreak $\langle f,g\rangle=\sum_{k=0}^{r-1}\Delta^kf(0)\Delta^kg(0)+
%\sum_{j=0}^\infty\Delta^rf(j)\Delta^rg(j)\rho(j)$,
%порожденная заданной ортонормированной системой функций $\mathcal{\psi}_{n}(x)$ $( n=0,1,\ldots)$. Показано, что ряды и суммы Фурье по системе
%$\mathcal{\psi}_{r,n}(x)$ $(r=1,2,\ldots, n=0,1,\ldots)$ является удобным и весьма эффективным инструментом приближенного решения задачи Коши для разностных уравнений.
%
%Для системы полиномов $l_{r,n}^{\alpha}(x)$ ($r$-натуральное число, $n=0, 1, \ldots$), ортонормированной относительно скалярного произведения типа Соболева (полиномы, ортонормированные по Соболеву) следующего вида $\langle f,g\rangle=\sum_{\nu=0}^{r-1}f^{(\nu)}(0)g^{(\nu)}(0)+\int_{0}^{\infty} f^{(r)}(x)g^{(r)}(x)\rho(x) dx$ и порожденной классическими ортонормированными полиномами Лагерра, получены рекуррентные соотношения, которые могут быть использованы для изучения различных свойств этих полиномов и вычисления их значений при любых $x$ и $n$.
%
%
%
%
%
%
%
%
%
%%%%%%%%%%%%%%%%%
%%%%%%%%%%%%%%%%%
%%%%%%%%%%%%%%%%%
%
%
%Рассмотрена задача о конструировании полиномов $s_{r,n}^\alpha(x)$, порожденных полиномами Шарлье $s_n^\alpha(x)$ и ортонормированных относительно скалярного произведения типа Соболева вида
%$  \langle f,g \rangle = \sum_{k=0}^{r-1} \Delta^k f(0) \Delta^k g(0) + \sum_{j=0}^\infty \Delta^r f(j) \Delta^r g(j) \rho(j) $, где $ \rho(x)=\alpha^x e^{-\alpha}/\Gamma(x+1)$.
%Показано, что система полиномов $s_{r,n}^\alpha(x)$, порожденная полиномами Шарлье, полна в гильбертовом пространстве $W^r_{l_\rho}$, состоящем из дискретных функций, заданных на сетке $\Omega=\{0,1,\ldots\}$, в котором введено скалярное произведение $\langle f,g \rangle$. Найдена явная формула вида $ s_{r,k+r}^{\alpha}(x) = \sum_{l=0}^{k} b_l^r x^{[l+r]} $, в которой $x^{[m]} = x(x-1)\ldots(x-m+1)$. Установлена связь полиномов $s_{r,n}^\alpha(x)$ с порождающими их ортонормированными классическими полиномами Шарлье $s_n^\alpha(x)$ вида $	s_{r,k+r}^{\alpha}(x)= U_k^r \left[s_{k+r}^{\alpha}(x) - \sum_{\nu=0}^{r-1} V_{k,\nu}^r x^{[\nu]}\right]$, в которой для чисел $U_k^r$, $V_{k,\nu}^r$ найдены явные выражения. 