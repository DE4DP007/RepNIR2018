\Referat %Реферат отчёта, не более 1 страницы

Отчет содержит Х~с., Х~источников.%, 8~таблиц, 12~иллюстраций.

\bigskip
\textbf{ Ключевые
	слова:}
дифференциальные уравнения;
эллиптические операторы;
%система навье-стокса;
$G$-сходи\-мость; %$G$-компактность;
усреднения операторов;
%нелинейные системы ДУ;
преобразование радона;
$p$-лапласиан;
%$p$-устойчивость;
задача Дирихле;
уравнения Ито;
уравнения с последействием;
уравнения Бельтрами;
положительное решение;
радиально-симметричное решение;
%дробные производные;
%разно-\linebreakстные уравнения;
теория устойчивости;
%моментная устойчивость;
метод модельных уравнений.
%; теоремы типа боля-перрона.

%\textbf{ Ключевые слова}: положительное решение, задача Дирихле, кольцевая область, нелинейное дифференциальное уравнение, единственность.
%дифференциальные уравнения Ито, устойчивость решений, положительная обратимость матриц

\bigskip



Настоящий отчёт содержит итоги работы Отдела математики и информатики ДНЦ РАН по теме № 0202-2017-0002  Плана федеральных научных исследований на 2018 год. 
В отчетном году продолжены исследования избранных вопросов математической физики и теории дифференциальных уравнений (ДУ).

В ряде задач математической физики широко используется так называемый метод усреднения дифференциальных операторов, основанный на асимптотических разложениях по малому параметру. Этот метод позволяет получить оценки разности точного решения и его приближений.
В отчетном году асимптотическими методами получены оценки погрешности усреднения периодической задачи для уравнения Бельтрами, с $\varepsilon$-периодическим коэффициентом, где $\varepsilon=1/n$  --- малый параметр.
Получены оценки погрешности усреднения задачи Дирихле для недивергентных эллиптических операторов второго порядка, удовлетворяющих условию Кордеса на разброс собственных значений матрицы коэффициентов. При этом коэффициенты уравнения --- $\varepsilon$-периодические функции, где $\varepsilon$ --- вновь малый параметр.
Продолжены исследования  $G$--сходимости недивергентных дифференциальных операторов, выделены $G$-компактные классы недивергентных эллиптических операторов второго порядка с комплекснозначными коэффициентами.








Проведены исследования вопросов моментной устойчивости решений для линейных ДУ Ито с
запаздываниями высоких порядков и систем линейных ДУ Ито с запаздываниями специального вида на
основе теории положительно обратимых матриц. Для этого применяются идеи и методы, разработанные Н.В. Азбелевым и его
учениками (так называемые метод модельных уравнений). 
Получены достаточные условия моментной устойчивости исследуемых уравнений в терминах в терминах положительной обратимости матриц, построенных по исходным уравнениям.
Результаты могут найти применение при исследовании развития различных процессов, подверженных случайным воздействиям, на  устойчивость. 
Проверена выполнимость этих условий для конкретных уравнений. 








%Изучался также вопрос единственности положительного решения задачи Коши. 
В 2018 г. в ОМИ доказано существование и единственность положительного радиально-симметричного решения задачи Дирихле в кольцевой области для одного класса нелинейных дифференциальных уравнений второго порядка и разработан численны метод его построения.














\newpage

%%%%%%%%%%%%%%%%%%%%%%%%
%%%%% OLD STYLE OTCHYOT BELOW  %%%%
=====================








Настоящий отчёт содержит итоги работы за 2018 год Отдела математики и информатики ДНЦ РАН по теме
<<Асимптотические методы усреднения недивергентных дифференциальных операторов. Исследование вопросов моментной устойчивости и устойчивости по части переменных для дифференциальных уравнений Ито с импульсными воздействиями и разностных уравнений Ито. Исследование вопросов существования и единственности решений краевых задач для нелинейных эллиптических уравнений с $p$- и $p(x)$-лапласианом. Лучевое преобразование векторных и тензорных полей и некоторые его обобщения>>
%осуществлению фундаментальных научных исследований в соответствии с
из Программы фундаментальных научных исследований государственных академий наук на 2013–2020 годы.


%%%%%%%%%%%%%%%%
%%%%%%%%%%%%%%%%
%%%Сиражудинов М.М.

В ряде задач математической физики широко используется так называемый метод усреднения дифференциальных операторов, основанный на асимптотических разложениях по малому параметру. Этот метод позволяет получить оценки разности точного решения и его приближений.
Ранее исследователями ОМИ впервые к усреднению обобщенного уравнения Бельтрами привлечены асимптотические методы, что позволило получить теорему усреднения и оценки порядка $O(\sqrt{\varepsilon})$ разности точного решения и его приближений в нормах пространств Лебега и Соболева.
%%%%%%%%%
В отчетном году асимптотическими методами получены оценки погрешности усреднения периодической задачи для уравнения Бельтрами, с $\varepsilon$-периодическим коэффициентом, где $\varepsilon=1/n$  --- малый параметр.

Кроме того, нами были получены оценки погрешности усреднения задачи Дирихле для недивергентных эллиптических операторов второго порядка, удовлетворяющих условию Кордеса на разброс собственных значений матрицы коэффициентов. При этом коэффициенты уравнения --- $\varepsilon$-периодические функции, где $\varepsilon$ --- вновь малый параметр.




Продолжены исследования  $G$--сходимости недивергентных дифференциальных операторов. $G$--сходимость --- это, иначе, слабая сходимость
соответствующего обратного оператора, а поэтому в задачах $G$--сходимости, кроме корректной разрешимости краевых задач, требуются также оценки решений,
равномерные относительно любого оператора. Нами выделены $G$-компактные классы недивергентных эллиптических операторов второго порядка с комплекснозначными коэффициентами.




%%%%%%%%%%%%%%%%
%%%%%%%%%%%%%%%%
%%%Кадиев Р.И.



В отчетном периоде проведены исследования вопросов моментной устойчивости решений для линейных дифференциальных уравнений Ито с
запаздываниями высоких порядков и систем линейных дифференциальных уравнений Ито с запаздываниями специального вида на
основе теории положительно обратимых матриц. Для этого применяются идеи и методы, разработанные Н.В.Азбелевым и его
учениками для исследования вопросов устойчивости детерминированных линейных функционально-дифференциальных уравнений,
которые ранее были применены нами для исследований вопросов устойчивости для линейных дифференциальных уравнений Ито с
запаздываниями первого порядка.

Получены достаточные условия моментной устойчивости исследуемых уравнений в терминах положительной обратимости матриц, построенных по исходным уравнениям.
Результаты могут найти применение при исследовании развития различных процессов, подверженных случайным воздействиям, на  устойчивость.
Проверена выполнимость этих условий для конкретных уравнений.
%Для простоты остановимся на изложении результатов в случае линейных дифференциальных уравнений Ито с запаздываниями второго порядка.
Результаты исследований опубликованы в центральной печати.




%%%%%%%%%%%%%%%%
%%%%%%%%%%%%%%%%
%%%%%%%%%%%%%%% AEI





Положительным решениям задачи Коши посвящен ряд
работ российских и зарубежных математиков, в большинстве из которых изучаются в основном
вопросы существования положительного решения, его поведение,
априорные оценки и т.д.
Единственность положительного решения при этом изучается сравнительно мало.
В 2018 г. в ОМИ доказано существование и единственность положительного радиально-симметричного решения задачи Дирихле в кольцевой
области для одного класса нелинейных дифференциальных уравнений второго порядка и разработан численны метод его построения.






%%%%%%%%%%%%%%%%
%%%%%%%%%%%%%%%%
=====================






%%%%%%%%%%%%%%% MZG

Введены в рассмотрение два двухпараметрических семейства ломаных на плоскости. Решена задача восстановления функции по ее интегралам вдоль этих ломаных, когда весовая функция – квазимногочлен. Для частных случаев весовых функций получены формулы обращения. В общем случае доказана единственность решения поставленной задачи. Результаты применены к доказательству единственности задачи интегральной геометрии с возмущением.


Доказана единственность восстановления функции, суммируемой в полосе на плоскости, заданной своими интегралами вдоль дуг двухпараметрических кривых второго порядка с весом, аналитическим по части переменных.

Доказаны формулы для определения неизвестного векторного поля на плоскости, заданного своим поперечным лучевым преобразованием в ограниченном угловом диапазоне. В первой формуле используется интегральная формула интерполяции функции с ограниченным спектром. Восстанавливается преобразование Фурье дивергенции неизвестного поля. Во второй формуле интерполяция производится по дискретным значениям функции. Восстанавливаются координатные функции потенциальной части искомого поля. Векторное поле ищется в классе вектор-функций, сосредоточенных в некоторой полосе, достаточно быстро убывающих на бесконечности и имеющих непрерывные вторые производные.

