\begin{thebibliography}{111}


\bibitem{ZhKO}
Жиков В.В., Козлов С.М., Олейник О.А. \emph{Усреднение дифференциальных
операторов}. М.: Наука, 1993.

\bibitem{Dg1}
Spagnolo S. \emph{Sul limite delle solutioni di problemi di Cauchy
relativi all\'{}equazione del calore}// Ann. Scuola Norm. Sup. Pisa, CI. Sci. 1967. V.\,21.
P.\,657--699.

\bibitem{Dg2}
Spagnolo S. \emph{Sulla convergenza di solutioni di equazioni paraboliche ed ellittiche
} // Ann. Scuola Norm. Sup. Pisa, CI. Sci. 1968. V.\,22.
P.\,577--597.

\bibitem{Dg3}
De Giorgi E., Spagnolo S. \emph{Sulla convergenza degli integrali dell \'{}energia
per operatori ellittici del 2 ordine} // Boll. Un. Mat. Ital. 1973. V.\,8, №4. P.\,391--411.

\bibitem{ZhS}
Жиков В.В., Сиражудинов М.М. \emph{О $G$--компактности одного класса недивергентных
эллиптических операторов второго порядка}// Изв. АН СССР, серия матем.
1981. Т.\,45, № 4. С.\,718--733.

\bibitem{ZhS1}
Жиков В.В., Сиражудинов М.М. \emph{Усреднение недивергентных эллптических и
параболических операторов второго порядка и стабилизация решений задачи Коши}//
Мат. сб. 1981. Т.\,116, № 2. С.\,166--186.

\bibitem{Sir1}
Сиражудинов М.М. \emph{$G$--сходимость и усреднение некоторых недивергентных
эллиптических операторов высокого порядка}// Дифференц. уравнения. 1983.
Т.19. № 11. С.\,1949--1956.
1981. Т.\,45, № 4. С.\,718--733.

\bibitem{Sir}
Сиражудинов М.М.\emph{О $G$--компактности одного класса эллиптических систем первого
порядка}// Дифференц. уравнения. 1990. Т.26. № 2. С.\,298--305.

\bibitem{Sus}
  Бирман М.\,Ш.,  Суслина Т.\,А.
\emph{Периодические дифференциальные операторы второго порядка.
Пороговые свойства и усреднения}//
Алгебра и анализ. 2003. Т. 15, №5. С. 1--108


\bibitem{VP}
  Жиков В.\,В,  Пастухова С.\,Е.
 \emph{Об операторных оценках в теории усреднения}//
Успехи матем. наук. 2016. Т. 71, №3. С. 27--122.


\bibitem{DZh}
Сиражудинов М.\,М., Джамалудинова С.\,П. \emph{О G-компактности некоторых классов эллиптических операторов второго порядка}// Даг. электронные математ. известия, 2018, №10.  http://mathreports.ru/

\bibitem{SDZ} Сиражудинов М.\,М. \emph{ Асимптотический метод усреднения обобщенных операторов Бельтрами}~//  Матем. сборник. 2017. Т.~208, №~4. С.~87–110.

\bibitem{SS} Сиражудинов М.\,М. \emph{Оценки погрешности усреднения периодической за-дачи для уравнения Бельтрами}// Вестник ДГУ, 2018, №4. http://vestnik.dgu.ru/

\bibitem{SM}	Сиражудинов М.\,М. \emph{Оценки погрешности усреднения недивергентных эллиптических уравнений второго порядка}// Междунар. конфер. по диф-фер. урав. и динам. сист. г. Суздаль 6-11 июля 2018 г. http://www.mathnet.ru/ConfLogos/1278/Programma main.pdf



%%%%%%%%%%%%%%%%%%
%%%%%%%%%%%%%%%%%%
%%%%%%%%%%%%%%%%%%



	\bibitem{kri-bib-1}
	Колмановский В.Б., Носов В.Р. Устойчивость и периодические режимы регулируемых систем с последействием. М.: Наука, 1981.
	
	\bibitem{kri-bib-2}
	Царьков Е.Ф. Случайные возмущения дифференционально-функциональных уравнений. Рига: Зинате, 1989.
	
	\bibitem{kri-bib-3}
	Mao X. Stochastic Differential Equations and Applications /Chichester: Horwood Publishing Itd.1997.360p.
	
	\bibitem{kri-bib-4}
	Mohammed S.--E.F. Stochastic Functional Differential Equations With Memory. Theory, Examples and Applications // Proceeding of The Sixth on Stochastic Analysis. Geilo. Norway. 1996. P.1-91.{\textbackslash}
	
	\bibitem{kri-bib-5}
	Азбелев Н. В., Березанский Л. М., Симонов П. М., Чистяков А. В. Устойчивость линейных систем с последствием. 1 // Дифференц. уравнения. 1987. Т. 28. № 5 С. 745-754.
	
	\bibitem{kri-bib-6}
	Березанский Л. М. Развитие W-метода Н. В. Азбелева в задачах устойчивости решений линейных функционально-дифференциальных уравнений // Дифференц. уравнения. 1986. Т. 22. № 5 С. 739-750
	
	\bibitem{kri-bib-7}
	Азбелев Н. В., Максимов В. П., Рахматуллина Л. Ф. Введение в теорию функционально-дифференциальных уравнений. М.: Наука, 1991. 280 с.
	
	\bibitem{kri-bib-8}
	L. Berezansky, E. Braverman On exponential dichotomy, Bohl Perron type theorems and stability of difference equations // J. Math. Annal. Appl. no 304 (2005), pp. 511-530.
	
	\bibitem{kri-bib-9}
	Kadiev R.I. and Ponosov A.V. Stability of stochastic functional differential equations and the W-transform // Electron J. Diff. Eqns. 2004. V.2004. N 92. P. 1-36.
	
	\bibitem{kri-bib-10}
	Кадиев Р.И., Поносов А.В. Устойчивость решений линейных импульсных систем дифференциальных уравнений //Дифференц. уравнения. 2007. Т. 43. N 7. С. 879-885.
	
	\bibitem{kri-bib-11}
	Kadiev R.I., Ponosov A.V. Exponential stability of linear stochastic differential equations with bounded delay and the W-transform // E. J. Qualitative Theory of Diff. Eq. 2008. N 23. P. 1-14.
	
	\bibitem{kri-bib-12}
	Кадиев Р.И. Поносов А.В. Устойчивость решений линейных импульсных систем дифференциальных уравнений Ито с ограниченными запаздываниями // Дифференц. уравнения. Минск. Т.46. N 4, 2010.С.486-498.
	
	\bibitem{kri-bib-13}
	R. Kadiev, A. Ponosov. Stability of impulsive stochastic differential linear functional	equations with linear delays // J. of Abstract Differential Equations and Applications. V. 2. N. 2, 2012. P. 7-25.
	
	\bibitem{kri-bib-14}
	R. Kadiev, A. Ponosov. The W-transform in stability analysis for stochastic linear functional difference equations // J. Mathem. Analysis and Appl. V. 389. N 2. 2012. P. 1239-1250.
	
	\bibitem{kri-bib-15}
	Кадиев Р.И. Устойчивость решений нелинейных функционально-дифференциальных уравнений с импульсными воздействиями по линейному приближению. // Дифференц. уравнения. Минск. Т.49. N 8. 2013. С.963-970.
	
	\bibitem{kri-bib-16}
	Berezansky L., Braverman E., Idels L. Nev Global Exponential Stability Criteria for	Nonlinear Delay Differential Systems with Applications to BAM Neural Networks. // Applied Mathematics and Computation, 243 (2014), pp. 899-910.
	
	\bibitem{kri-bib-17}
	Липцер Р.Ш., Ширяев А.Н. Теория мартингалов. М.: Наука, 1986.
	
	\bibitem{kri-bib-18}
	A. Berman and R. Plemmons, Nonnegative Matrices in the Mathematical Sciences, Computer Science and Applied Mathematics, Academic Press, New York-London, 1979.
	
	\bibitem{kri-bib-19}
	Кадиев Р.И. К вопросу об устойчивости по начальным данным по части переменных решений линейных импульсных	дифференциальных уравнений Ито с последействием // Вестник Тамбовского университета. 2015, том 20, № 5, с.1190-1194.
	
	\bibitem{kri-bib-20}
	Кадиев Р.И., Шахбанова З.И. Устойчивость по части переменных функционально-дифференциальных уравнений со случайными параметрами. // Вестник ДГУ. Естественные науки. Вып. 1. Махачкала. 2008. С. 21-26.
	
	\bibitem{kri-bib-21}
	Кадиев Р.И., Шахбанова З.И. Применение теории положительно обратимых матриц при исследовании устойчивости решений
	систем линейных дифференциальных уравнений Ито. // Вестник ДГУ. Естественные науки. Вып. 1. Махачкала. 2012. С.
	30-36.






%%%%%%%%%%%%%%%%%%
%%%%%%%%%%%%%%%%%%
%%%%%%%%%%%%%%%%%%






\bibitem {nelinanaliz}
 С. Bandle, Man Kam Kwong. Semilinear elliptic problems in annular domains.
  ~// Journal of applied Mathematics and Physics(ZAMP).1989. v.~40. p. ~245--247.
\bibitem {nelinanaliz}
 Vieri Benci anl Donato Fortinato. Some nonlinear elliptic problems
with asymptotic conditions. ~// Nonlinear Analysis. Theory, Methods
and Applications.  1979. v.~3. № 2. p. ~157--173.
\bibitem {nelinanaliz}
 Похожаев С.И. Об одной задаче Овсянникова. ~// ПМТФ. 1989. №2. C. ~5--10.
\bibitem {nelinanaliz}
Галахов Е.И. Положительные решения квазилинейного эллиптического
уравнения. ~// Математические заметки. 2005. т.78. вып.2. с.
~202--211.
\bibitem {nelinanaliz}
 Kuo-Shung Cheng and Jenn-Tsann Lin. On the elliptic equations
$ \Delta u=K(x)u^{\alpha} $   and $ \Delta u=K(x)\exp^{2u} $  .
 ~//Transactions of American mathematical society. 1987. v.~304.
№ 2. p.~633--668.
\bibitem {nelinanaliz}
 Gidas B., Spruck  J.  Global and local behavior of positive solutions
of nonlineare elliptic equations.//  Communications on Pure and
Applied Mathematics,  1982. V.~ 4. P.~525--598.
\bibitem {nelinanaliz}
Похожаев С.И. О целых радиальных решениях некоторых квазилинейных
эллиптических уравнений. ~// Математический сборник. 1992. т.83.
№ 11. с. ~3--18.
\bibitem {nelinanaliz}
Dancer E. Norman, Shi Junping. Uniqueness and nonexistence of
positive solutions to semipositive problems. ~// London Math. Soc.
2006. v. ~38. № 6.  p. ~1033--1044.
\bibitem {nelinanaliz}
Kavano Nichiro, Satsuma Junkichi, Youtsutani Shoji. On the Positive
Solution of an Emden-Type Elliptic Equation. ~// Proc. Jap. Acad.
1985.  Ser A. v. ~61. № 6. p. ~186--189.
\bibitem {nelinanaliz}
Jiang Ju. On radially  symmetric solutions to singular nonlinear
Dirichlet problems. ~// Nonlinear Anal. Theory, Methods and
Applications. 1995. v. ~24.  p. ~159--163.
\bibitem {nelinanaliz}
 Абдурагимов Э.И. Положительное решение двухточечной краевой задачи для
одного нелинейного оду четвертого порядка. ~//Дагестанский
математический сборник. 2005. Т. ~1. C. ~7--12.
\bibitem {nelinanaliz}
 Абдурагимов Э.И. О положительном радиально-симметричном решении
 задачи Дирихле для одного нелинейного  уравнения и численном методе
 его получения. ~// Изв. вузов. Математика. 1997.№ 5. С. ~3--6.
\bibitem {nelinanaliz}
Абдурагимов Э.И. О единственности положительного
радиально-симметричного решения задачи Дирихле в шаре для одного
нелинейного дифференциального уравнения второго порядка. ~// Изв.
вузов. Математика. 2008. № 12. С. ~3-6.
\bibitem {nelinanaliz}
 На Ц. Вычислительные методы решения прикладных граничных задач.
 М.: Мир, 1982. 296~с.
\bibitem {analiz}
Фихтенгольц Г.М. Курс дифференциального и интегрального исчисления.
М.:Наука, 1970. 608~с.



    %%%%%%%%%%%%%%%%%%
    %%%%%%%%%%%%%%%%%%
    %%%%%%%%%%%%%%%%%%







%\bibitem{sobleg-Shar11}
%{Шарапудинов И.И.} Приближение функций с переменной гладкостью суммами Фурье Лежандра // Мат. сборник,
%191(5), 2000. С. 143--160.
%
%
%\bibitem{sobleg-Shar12}
%{Шарапудинов И.И.} Аппроксимативные свойства операторов $\mathcal{ Y}_{n+2r}(f)$ и их дискретных аналогов // Мат. заметки, 72(5), 2002. С.765--795.
%
%
%\bibitem{sobleg-Shar13}
%{Шарапудинов И.И.} Смешанные ряды по ортогональным полиномам. Издательство Дагестанского научного центра.
%Махачкала 2004. С.1--176.
%
%
%\bibitem{sobleg-Shar15}
%{Шарапудинов И.И.}
%Аппроксимативные свойства смешанных рядов по полиномам Лежандра на классах $W^r$ //
%Мат. сборник, 97(3), 2006. С. 135--154.
%
%
%\bibitem{sobleg-Shar16}
%{Шарапудинов И.И.}
%Аппроксимативные свойства средних типа Валле-Пуссена частичных сумм смешанных рядов по полиномам Лежандра // Мат. заметки, 84(3), 2008. С.452--471.
%
%
%\bibitem{sobleg-Shar17}
%{Шарапудинов И.И.}
% Смешанные ряды по ультрасферическим полиномам и их аппроксимативные свойства
%// Мат. сборник, 194(3), 2003. С. 115--148.
%
%
%\bibitem{sobleg-Shar18}
%{Шарапудинов И.И., Шарапудинов Т.И.}
% Смешанные ряды по полиномам Якоби и Чебышева и их дискретизация
%// Мат. заметки, 88(1), 2010. С. 116--147.
%
%
%\bibitem{sobleg-sharap3}
%{Шарапудинов И.И.}
% Некоторые специальные ряды по ультрасферическим полиномам и их аппроксимативные свойства
%// Изв. РАН. Сер. матем. 78(5), 2014. С. 201--224.
%
%
%\bibitem{sobleg-SHII}
%{Шарапудинов И.И.}
% Некоторые специальные ряды по общим полиномам Лагерра и ряды Фурье по полиномам Лагерра, ортогональным по Соболеву
%// Дагестанские электронные математические известия. 2015. Вып. 4.
%
%
%\bibitem{sobleg-Sege}
%{Сеге Г.} Ортогональные многочлены. Физматгиз. Москва. 1962.
%
%
%\bibitem{sobleg-Gasper}
%{Gasper G.}
% Positiviti and special function
%// Theory and appl.Spec.Funct. Edited by Richard A.Askey. 1975. Pp. 375--433.
%
%
%\bibitem{sobleg-KwonLittl1}
%{Kwon K.H., Littlejohn L.L.}
% The orthogonality of the Laguerre polynomials $\{L_n^{(-k)}(x)\}$ for positive integers $k$
%// Ann. Numer. Anal. Iss. 2. 1995. Pp. 289--303.
%
%
%\bibitem{sobleg-KwonLittl2}
%{Kwon K.H., Littlejohn L.L.}
% Sobolev orthogonal polynomials and second-order differential equations
%// Rocky Mountain J. Math. Vol. 28. 1998. Pp. 547--594.
%
%
%\bibitem{sobleg-MarcelAlfaroRezola}
%{Marcellan F. , Alfaro M., Rezola M.L.} Orthogonal polynomials on Sobolev spaces: old and new directions
%// Journal of Computational and Applied Mathematics. Vol. 48. 1993. Pp. 113--131.
%
%
%\bibitem{sobleg-IserKoch}
%{ Iserles A., Koch P.E., Norsett S.P., Sanz-Serna J.M.}
% On polynomials  orthogonal  with respect  to certain Sobolev inner products
%// J. Approx. Theory, 65. 1991. Pp. 151--175.
%
%
%\bibitem{sobleg-Meijer}
%{Meijer H.G.} Laguerre polynimials generalized to a certain.
%Laguerre polynimials generalized to a certain discrete Sobolev inner product space
%// J. Approx. Theory, 73. 1993. Pp. 1--16.
%
%
%\bibitem{sobleg-Lopez1995}
%{Lopez G. Marcellan F. Vanassche W.}
% Relative Asymptotics for Polynomials Orthogonal with Respect to a Discrete Sobolev Inner-Product
%// Constr. Approx. 11:1. 1995. Pp. 107--137.
%
%
%\bibitem{sobleg-MarcelXu}
%{Marcellan F., Xu Y.}
% On Sobolev orthogonal polynomials
%// Expositiones Mathematicae, 33(3). 2015. Pp. 308--352.
%
%
%\bibitem{sobleg-Shar2016}
%И.И. Шарапудинов
% Системы функций, ортогональные по Соболеву, порожденные ортогональными функциями
%// Материалы 18-й международной Саратовской зимней школы «Современные проблемы теории функций и их приложения». 2016. С. 329--332.
%
%
%\bibitem{sobleg-Tref1}
%{Trefethen  L.N.} Spectral methods in Matlab. Fhiladelphia. SIAM. 2000.
%
%
%\bibitem{sobleg-Tref2}
%{Trefethen  L.N.}
%Finite difference and spectral methods for ordinary and partial differential equation. Cornell University. 1996.
%
%
%\bibitem{sobleg-SolDmEg}
%{Солодовников В.В., Дмитриев А.Н., Егупов Н.Д.}
%Спектральные методы расчета и проектирования систем управления. Машиностроение. Москва. 1986.
%
%
%\bibitem{sobleg-Pash}
%{Пашковский С.} Вычислительные применения многочленов и рядов Чебышева. Наука. Москва. 1983. С. 143--160.
%
%
%\bibitem{sobleg-MMG2016}
%{Магомед-Касумов М.Г.}
% Приближенное решение обыкновенных дифференциальных уравнений с использованием смешанных рядов по системе Хаара
%// Материалы 18-й международной Саратовской зимней школы «Современные проблемы теории функций и их приложения». 2016. С. 176--178.
%
%
%\bibitem{sobleg-Gonchar1975}
%{Гончар А.А.}
% О сходимости аппроксимаций Паде для некоторых классов мероморфных функций
%// Мат. сборник, 97(139):4(8), 1975. С. 607--629.
%
%
%\bibitem{sobleg-TEL}
%{Теляковский С.А.}
% Две теоремы о приближении функций алгебраическими многочленами
%// Мат. сборник, 70(2), 1966. С.252--265.
%
%
%\bibitem{sobleg-GOP}
%{Гопенгауз И.З.}
% К теореме А. Ф. Тимана о приближении функций многочленами на конечном отрезке
%// Мат. заметки, 1(2), 1967. С. 163--172.
%
%
%\bibitem{sobleg-OSK}
%{Осколков К.И.}
% К неравенству Лебега в равномерной метрике и на множестве полной меры
%// Мат.  заметки, 18(4), 1975. С. 515--526.
%
%
%\bibitem{sobleg-sharap1}
%{Sharapudinov I.I.}
% On the best approximation and polinomial of the least quadratic deviation
%// Analysis Mathematica, 9(3), 1983. Pp. 223--234.
%
%
%\bibitem{sobleg-sharap2}
%{Шарапудинов И.И.}
% О наилучшем приближении и суммах Фурье-Якоби
%//Мат. заметки, 34(5), 1983. С. 651--661.
%
%
%\bibitem{sobleg-Timan}
%{Тиман А.Ф.} Теория приближения функций действительного переменного. Физматгиз, Москва. 1960.
%%%
%%% end lit. section2-sobleg
%
%
%\bibitem{laplas-Shar13}
%{Шарапудинов И.И.}
%Смешанные ряды по ортогональным полиномам // Издательство Дагестанского научного центра. Махачкала. 2004. Стр. 1--176.
%
%
%\bibitem{laplas-Shar14}
%{Шарапудинов И.И.}
%Смешанные ряды по полиномам Чебышева, ортогональным на равномерной сетке // Математические заметки. 2005. Т. 78. Вып. 3. Стр. 442–-465.
%
%
%\bibitem{laplas-Shar11}
%{Шарапудинов И.И.}
%Специальные ряды по полиномам Лагерра и их аппроксимативные свойства // Сибирский математический журнал. 2017. Т. 58. Вып. 2. Стр. 440--467.
%
%
%\bibitem{laplas-Meijer}
%{Meijer H.G.}
%Laguerre polynimials generalized to a certain discrete Sobolev inner product space // J. Approx. Theory. 1993. Vol. 73. Pp. 1--16.
%
%
%\bibitem{laplas-MarcelXu}
%{Marcellan F., Yuan Xu}
%ON SOBOLEV ORTHOGONAL POLYNOMIALS. arXiv: 6249v1 [math.C.A] 25 Mar 2014. Pp. 1--40
%
%
%\bibitem{laplas-MarcelVanash}
%{Lopez G., Marcellan F., Van Assche W.}
%Relative asymptotics for polynomials orthogonal with respect to a discrete Sobolev inner product // Constr. Approx. 1995. Vol. 11. Issue 1. Pp. 107--137.
%
%
%\bibitem{laplas-Sege}
%{Сеге Г.}
%Ортогональные многочлены. Москва. Физматгиз. 1962.
%
%
%\bibitem{laplas-AskeyWaiger}
%{Askey R., Wainger S.}
%Mean convergence of expansions in Laguerre and Hermite series // Amer. J. Mathem. 1965. Vol. 87. Pp. 698--708.
%
%
%\bibitem{laplas-DitPrud}
%{Диткин В.А., Прудников А.П.}
%Операционное исчисление. Москва. Высшая школа. 1975.
%
%
%\bibitem{laplas-KrylovSkob}
%{Крылов В.И., Скобля Н.С.}
%Методы приближенного преобразования Фурье и обращения преобразования Лапласа. Москва. Наука. 1974.
%%%
%%% end lit. section2-laplas
%
%
%\bibitem{equ102-Shar20}
%Шарапудинов И.И. Ортогональные  по Соболеву системы, порожденные ортогональными функциями // Изв. РАН. Сер. Математическая. 2018. Том. 82. (Принята к печати)
%
%
%\bibitem{equ102-Tref1}
%{Trefethen  L.N.}
%Spectral methods in Matlab. Fhiladelphia. SIAM. 2000.
%
%
%\bibitem{equ102-Arush2014}
%{Арушанян О.Б., Волченскова Н.И., Залеткин С.Ф.}
%Применение рядов Чебышева для интегрирования обыкновенных дифференциальных уравнений // Сиб. электрон. матем. изв. 2014. Вып. 11. Стр. 517--531.
%
%
%\bibitem{equ102-Lukom2016}
%{Лукомский Д.С., Терехин П.А.}
%Применение системы Хаара к численному решению задачи Коши для линейного дифференциального уравнения первого порядка // Материалы 18-й международной Саратовской зимней школы «Современные проблемы теории функций и их приложения». Саратов. ООО «Издательство «Научная книга». 2016. Стр. 171--173.
%
%
%\bibitem{equ102-DiffUr2017}
%{Шарапудинов И.И., Магомед-Касумов М.Г.}
%О представлении решения задачи Коши  рядом Фурье  по полиномам, ортогональным по  Соболеву, порожденным многочленами Лагерра. Дифференциальные уравнения. 2017 (принята к печати)
%
%
%\bibitem{equ102-KashSaak}
%{Кашин Б.С., Саакян А.А.}
%Ортогональные ряды. Москва. АФЦ 1999.
%
%
%\bibitem{equ102-Shar19}
%{Шарапудинов И.И., Муратова Г.Н.}
%Некоторые свойства r-кратно интегрированных рядов по системе Хаара // Изв. Сарат. ун-та. Нов. сер. Сер. Математика. Механика. Информатика. 2009. Т. 9. Вып. 1. Стр. 68 -- 76
%
%
%\bibitem{equ102-Faber}
%{G. Faber}
%Ober die Orthogonalfunktionen des Herrn Haar // Jahresber. Deutsch. Math. Verein. 1910. Vol. 19. Pp. 104--112.
%
%
%\bibitem{equ102-Shar25}
%{Шарапудинов И.И.}
%Асимптотические свойства полиномов, ортогональных по Соболеву, порожденных полиномами Якоби // Дагестанские электронные математические известия. 2016. Вып. 6.	Стр. 1–-24.
%
%
%\bibitem{equ130-Shar13}
%{Шарапудинов И.И.}
%Смешанные ряды по ортогональным полиномам. Издательство Дагестанского научного центра. Махачкала. 2004. С. 1--176.
%
%
%\bibitem{equ130-Tref1}
%{Trefethen L.N.}
%Spectral methods in Matlab. SIAM. Philadelphia. 2000.
%
%
%\bibitem{equ130-DiffUr2017}
%{Шарапудинов И.И., Магомед-Касумов М.Г.}
%О представлении решения задачи Коши  рядом Фурье  по полиномам, ортогональным по  Соболеву, порожденным многочленами Лагерра // Дифференциальные уравнения. 2017. (принята к печати)
%
%
%
%
%
%
%
%
%\bibitem{equ130-Sege}
%Сеге Г. Ортогональные многочлены. Москва. Физматгиз. 1962.
%%%
%%% end lit. section2-equ130
%
%
%\bibitem{sobcheb_urav-Arush2010}
%{Арушанян О.Б., Волченскова Н.И., Залеткин С.Ф.}
%Приближенное решение обыкновенных дифференциальных уравнений с использованием рядов Чебышева // Сиб. электрон. матем. 1983. изв. Вып. 7. Стр. 122–-131
%
%
%\bibitem{sobcheb_urav-Arush2013}
%{Арушанян О.Б., Волченскова Н.И., Залеткин С.Ф.}
% Метод решения задачи Коши для обыкновенных дифференциальных уравнений с использованием рядов Чебышeва // Выч. мет. программирование. 2013. Вып. 14:2. Стр. 203-214.
%
%
%\bibitem{sobcheb_urav-fiht2}
%{Фихтенгольц Г.М.}
%Курс дифференциального и интегрального исчисления // Физматлит. Москва. 2001. Т. 2. Стр. 810.
%
%
%\bibitem{ramis-Ram1}
%{Шарапудинов~И.И.} Многочлены, ортогональные на сетках. Махачкала, Изд-во Даг. гос. пед. ун-та. 1997.		
%
%
%\bibitem{ramis-shGadj}
%{Шарапудинов И.И., Гаджиева З.Д.}
%Полиномы, ортогональные по Соболеву, порожденные многочленами Мейкснера // Изв. Сарат. ун-та. Нов. сер. Сер. Математика. Механика. Информатика,
%2016. Т.16. Вып. 3. С. 310--321.
%
%
%\bibitem{ramis-shGadjGadjMir}
%{Шарапудинов И.И., Гаджиева З.Д., Гаджимирзаев Р.М.}
%Разностные уравнения и полиномы, ортогональные по Соболеву, порожденные многочленами Мейкснера //
%Владикавказский Мат. журнал, 2017. Т.19. Вып. 2. С. 58--72.
%
%
%\bibitem{ramis-Shar9}
%{Шарапудинов~И.И.}
%Приближение дискретных функций и многочлены Чебышева, ортогональные на равномерной сетке //
%Мат. заметки, 2000. Т. 67. Вып. 3. С. 460--470.
%
%
%\bibitem{ramis-SharT1}
%{Шарапудинов~Т.И.}
%Аппроксимативные свойства смешанных рядов по полиномам Чебышева, ортогональным на равномерной сетке //
%Вестник Дагестанского научного центра РАН, 2007. Т. 29. С. 12-–23.
%
%
%\bibitem{ramis-SharII}
%{Шарапудинов~И.И.}
%Системы функций, ортогональных по Соболеву, порожденные ортогональными функциями //
%Современные проблемы теории функций и их приложения.  Материалы 18-й международной Саратовской зимней школы. 2016. С. 329--332.
%
%
%\bibitem{ramis-Gadz12}
%{Fernandez~L., Teresa E. Perez, Miguel A. Pinar, Xu~Y.} Weighted Sobolev orthogonal polynomials on the unit ball~// Journal of Approximation Theory, 171, 2013, pp.~84--104.
%
%
%\bibitem{ramis-Gadz13}
%{Antonia~M. Delgado, Fernandez~L., Doron~S. Lubinsky, Teresa~E. Perez, Miguel~A. Pinar.} Sobolev orthogonal polynomials on the unit ball via outward normal derivatives~// Journal of Mathematical Analysis and Applications, 440, №~2, 2016, pp.~716--740.
%
%
%\bibitem{ramis-Gadz14}
%{Fernandez~L., Marcellan~F., Teresa~E. Perez, Miguel~A. Pinar, Xu~Y.} Sobolev orthogonal polynomials on product domains~// Journal of Computational and Applied Mathematics, 284, 2015, pp.~202--215.
%
%
%\bibitem{ramis-Gadz16}
%{Шарапудинов~И.~И., Шарапудинов~Т.~И.} Полиномы, ортогональные по Соболеву, порожденные многочленами Чебышева, ортогональными на сетке~// Изв. вузов. Матем., 2017, №~8, 67--79.
%
%
%\bibitem{ramis-Gadz17}
%{Гаджимирзаев~Р.~М.} Ряды Фурье по полиномам Мейкснера, ортогональным по Соболеву~// Изв. Сарат. ун-та. Нов. сер. Сер. Математика. Механика. Информатика, 16:4 (2016), 388--395.
%
%
%\bibitem{ramis-Gadz1}
%{Шарапудинов~И.~И., Гаджиева~З.~Д., Гаджимирзаев~Р.~М.} Системы функций, ортогональных относительно скалярных произведений типа Соболева с дискретными массами, порожденных классическими ортогональными системами~// Дагестанские электронные математические известия. 2016. Вып.~6. С.~31--60.
%
%
%\bibitem{ramis-Gadz3}
%{Сеге~Г.} Ортогональные многочлены. М.: Физматгиз, 1962.
%%%
%%% end lit. ramis
%
%\bibitem{charlier-Shar3}
%Meijer~H.~G. Laguerre polynomials generalized to a certain discrete Sobolev inner product space~// J. Approx. Theory. 1993. Vol.~73. Iss.~1. Pp.~1--16.
%
%\bibitem{charlier-Shar8}
%Шарапудинов~И.~И. Смешанные ряды по ортогональным полиномам. Махачкалаю. Изд-во ДНЦ РАН. 2004.
%
%\bibitem{charlier-Shar10}
%Бейтмен~Г., Эрдейи~А. Высшие трансцендентные функции. Том 2. Москва. Наука. 1974.
%
%\bibitem{charlier-Shar11}
%Ширяев~А.~Н. Вероятность-1. Москва. Изд-во МЦНМО. 2007.

\end{thebibliography}
