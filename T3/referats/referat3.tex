\Referat %Реферат отчёта, не более 1 страницы

Отчет содержит Х~с., Х~источников.%, 8~таблиц, 12~иллюстраций.

 \bigskip
 \textbf{ Ключевые
  слова:}
  ДИФФЕРЕНЦИАЛЬНЫЕ УРАВНЕНИЯ; ЭЛЛИПТИЧЕСКИЕ ОПЕРАТОРЫ; СИСТЕМА НАВЬЕ-СТОКСА; $G$-СХОДИМОСТЬ; $G$-КОМПАКТНОСТЬ; УСРЕДНЕНИЯ ОПЕРАТОРОВ; НЕЛИНЕЙНЫЕ СИСТЕМЫ ДУ;
  ПРЕОБРАЗОВАНИЕ РАДОНА;
  $p$-ЛАПЛАСИАН; $p$-УСТОЙЧИВОСТЬ; ЗАДАЧА ДИРИХЛЕ; УРАВНЕНИЯ ИТО; УРАВНЕНИЯ С ПОСЛЕДЕЙСТВИЕМ;  УРАВНЕНИЯ БЕЛЬТРАМИ; ПОЛОЖИТЕЛЬНОЕ РЕШЕНИЕ; РАДИАЛЬНО-СИММЕТРИЧНОЕ РЕШЕНИЕ; ДРОБНЫЕ ПРОИЗВОДНЫЕ; РАЗНО-\linebreakСТНЫЕ УРАВНЕНИЯ;  ТЕОРИЯ УСТОЙЧИВОСТИ; МОМЕНТНАЯ УСТОЙЧИВОСТЬ; МЕТОД МОДЕЛЬНЫХ УРАВНЕНИЙ; ТЕОРЕМЫ ТИПА БОЛЯ-ПЕРРОНА; ЛУЧЕВОЕ.

 \bigskip

Настоящий отчёт содержит итоги работы за 2017 год Отдела математики и информатики ДНЦ РАН по теме
<<Асимптотические методы усреднения недивергентных дифференциальных операторов. Исследование вопросов моментной устойчивости и устойчивости по части переменных для дифференциальных уравнений Ито с импульсными воздействиями и разностных уравнений Ито. Исследование вопросов существования и единственности решений краевых задач для нелинейных эллиптических уравнений с $p$- и $p(x)$-лапласианом. Лучевое преобразование векторных и тензорных полей и некоторые его обобщения>>
%осуществлению фундаментальных научных исследований в соответствии с
из Программы фундаментальных научных исследований государственных академий наук на 2013–2020 годы.

\input chapters/refs/ref3.tex


