\Referat %Реферат отчёта, не более 1 страницы

Отчет содержит 55~с., 53~источника.

 \bigskip
 \textbf{ Ключевые
  слова:}
  ФУНКЦИОНАЛЬНЫЕ ПРОСТРАНСТВА ЛЕБЕГА И СОБОЛЕВА С ПЕРЕМЕННЫМ ПОКАЗАТЕЛЕМ; ВЕСОВЫЕ ПРОСТРАНСТВА ЛЕБЕГА С ПЕРЕМЕННЫМ ПОКАЗАТЕЛЕМ; ТЕОРИЯ ПРИБЛИЖЕНИЙ; УЛЬТРАСФЕРИЧЕСКИЕ ПОЛИНОМЫ ЯКОБИ; СПЕЦИАЛЬНЫЕ РЯДЫ ПО КЛАССИЧЕСКИМ ОРТОГОНАЛЬНЫМ ПОЛИНОМАМ; НАИЛУЧШЕЕ ПРИБЛИЖЕНИЕ И СКОРОСТЬ СХОДИМОСТИ; ПОВТОРНЫЕ СРЕДНИЕ ВАЛЛЕ ПУССЕНА; ОРТОГОНАЛЬНЫЕ ФУНКЦИИ; ОПЕРАТОР СТЕКЛОВА; АНАЛОГИ ТЕОРЕМЫ ДЖЕКСОНА.

 \bigskip

Настоящий отчёт содержит итоги работы за 2017 год Отдела математики и информатики ДНЦ РАН по теме
<<Функциональные пространства с переменным показателем и их приложения. Некоторые вопросы теории приближений полиномами, рациональными функциями, сплайнами и вейвлетами>>
%осуществлению фундаментальных научных исследований в соответствии с
из Программы фундаментальных научных исследований государственных академий наук на 2013–2020 годы.

\input chapters/refs/ref1.tex



%
%
%\textbf{Отчет содержит:}  190 страниц,  13 иллюстраций,  5 таблиц,  197 источников.
%
%
%\textbf{Ключевые слова:} функциональные пространства Лебега и Соболева с переменным показателем; весовые пространства; теория приближений; ортогональные полиномы; предельные и специальные ряды; специальные дискретные ряды; наилучшее приближение и скорость сходимости; функция Лебега; неравномерные сетки; эллиптические операторы; средние Валле--Пуссена; уравнения Ито; уравнения Бельтрами; теория расписаний; лучевое преобразование Радона.
