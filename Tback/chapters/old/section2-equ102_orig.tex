\chapter{Ортогональные по Соболеву системы функций и задача Коши для ОДУ}



%Рассмотрены системы функций $\mathcal{ \varphi}_{r,n}(x)$ $(r=1,2,\ldots, n=0,1,\ldots)$,
% ортонормированные по Соболеву относительно скалярного произведения  вида $\langle f,g\rangle=\sum_{\nu=0}^{r-1}f^{(\nu)}(a)g^{(\nu)}(a)+\int_{a}^{b}f^{(r)}(t)g^{(r)}(x)\rho(x)dx$,
%порожденные заданной ортонормированной системой функций $\mathcal{ \varphi}_{n}(x)$ $( n=0,1,\ldots)$.  Показано, что ряды и суммы Фурье по системе $\mathcal{ \varphi}_{r,n}(x)$ $(r=1,2,\ldots, n=0,1,\ldots)$ являются удобным и весьма эффективным инструментом приближенного решения задачи Коши для обыкновенных дифференциальных уравнений (ОДУ).
%


%%%%%%%%%%%%%%%%%%%%%%%%%%%%%%%%%%% 
%%%%%%%%%%%%%%%%%%%%%%%%%%%%%%%%%%%
%%%%%%%%%%%%%%%%%%%%%%%%%%%%%%%%%%%




\section{Введение}
В настоящей работе мы продолжаем рассмотрение систем функций, ортогональных относительно скалярных произведений, в которых присутствуют одна или несколько точек  с дискретными массами. Интерес к таким системам  в последнее время интенсивно растет  (см. \cite{Shar2016} -- \cite{Shar13}  и цитированную там литературу).
 Это новое направление принято обозначать словами: "Функции, ортогональные по Соболеву". Возросшее  внимание специалистов  к этому направлению теории ортогональных систем можно объяснить в том числе и тем обстоятельством, что ряды Фурье по полиномам (и функциям), ортогональным по Соболеву, оказались естественным и весьма удобным инструментом для представления решений  дифференциальных (разностных) уравнений. Это можно показать, в частности, на примере  задачи Коши для  дифференциального уравнения
\begin{equation}\label{1.1}
F(x,y,y',\ldots,y^{(r)})=0
 \end{equation}
с начальными условиями $y^{(k)}(a)=y_k$, $k=0,1,\ldots,r-1$.  Наряду с различными сеточными и аппроксимационно-аналитическими методами, для решения этой задачи часто применяют так называемые спектральные методы \cite{Tref1} -- \cite{Shar18}, суть которых  заключается в том, что на первом этапе искомое решение $y(x)$ задачи Коши для уравнения \eqref{1.1} представляется в виде ряда Фурье
\begin{equation}\label{1.2}
 y(x)=\sum_{k=0}^\infty \hat y_k\psi_k(x)
 \end{equation}
по подходящей ортонормированной системе $\{\psi_k(x)\}_{k=0}^\infty$. На втором этапе осуществляется подстановка вместо $y(x)$ ряда \eqref{1.2} в уравнение \eqref{1.1}. Это приводит к системе уравнений (вообще говоря, бесконечной) относительно неизвестных коэффициентов $\hat y_k$ ($k=0,1,\ldots$). На третьем этапе требуется решить эту систему с учетом начальных условий  $y^{(k)}(a)=y_k$, $k=0,1,\ldots,r-1$ исходной задачи Коши.
Одна из основных трудностей, которая возникает на этом этапе, состоит в том, чтобы
выбрать такой ортонормированный базис $\{\psi_k(x)\}_{k=0}^\infty$, для которого искомое решение $y(x)$ уравнения \eqref{1.1}, представленное в виде ряда  \eqref{1.2}, удовлетворяло бы начальным условиям $y^{(k)}(a)=y_k$, $k=0,1,\ldots,r-1$. Более того, поскольку в результате решения системы уравнений относительно неизвестных коэффициентов $\hat y_k$  будет найдено только конечное их число с $k=0,1,\ldots, n$, то весьма важно, чтобы частичная сумма ряда \eqref{1.2} вида $ y_n(x)=\sum_{k=0}^n\hat y_k\psi_k(x)$,
 будучи приближенным решением рассматриваемой задачи Коши, также удовлетворяла начальным условиям $y_n^{(k)}(a)=y_k$, $k=0,1,\ldots,r-1$. Как это было  показано в \cite{Shar20},  базис $\{\psi_k(x)= \varphi_{r,n}(x)\}_{k=0}^\infty$, состоящий из функций $\varphi_{r,n}(x)$, ортонормированных по Соболеву относительно скалярного произведения
\begin{equation}\label{1.3}
<f,g>=\sum_{\nu=0}^{r-1}f^{(\nu)}(a)g^{(\nu)}(a)+\int_{a}^{b}f^{(r)}(t)g^{(r)}(x)\rho(x)dx,
\end{equation}
 обладает указанными свойствами. Ниже нам понадобятся некоторые факты, установленные в \cite{Shar20}.  Предположим, что система функций  $\left\{\varphi_k(x)\right\}$ ортонормирована  на $(a,b)$  c весом   $\rho(x)$, т.е.
 \begin{equation}\label{1.4}
\int_a^b\varphi_k(x)\varphi_l(x)\rho(x)dx=\delta_{kl},
\end{equation}
где $\delta_{kl}$ -- символ Кронекера. Через $L^p_\rho(a,b)$ обозначим пространство  функций $f(x)$, измеримых  на  $(a,b)$, для которых
 \begin{equation*}
\int_a^b|f(x)|^p\rho(x)dx<\infty.
\end{equation*}
Если $\rho(x)\equiv1$, то будем писать $L^p_\rho(a,b)=L^p(a,b)$ и $L(a,b)=L^1(a,b)$.
Из \eqref{1.4} следует, что $\varphi_k(x)\in L^2_\rho(a,b)$ $(k=0,1,\ldots)$. Мы добавим к этому условию еще одно, считая, что $\varphi_k(x)\in L(a,b)$ $(k=0,1,\ldots)$. Тогда, следуя  \cite{Shar20}, мы можем определить следующие порожденные системой $\{\varphi_k(x)\}$ функции
 \begin{equation}\label{1.5}
\varphi_{r,r+k}(x) =\frac{1}{(r-1)!}\int_a^x(x-t)^{r-1}\varphi_{k}(t)dt, \quad k=0,1,\ldots,
\end{equation}

  \begin{equation}\label{1.6}
\varphi_{r,k}(x) =\frac{(x-a)^k}{k!}, \quad k=0,1,\ldots, r-1.
\end{equation}

 Из \eqref{1.5} и \eqref{1.6} следует, что для п.в. $x\in (a,b)$
 \begin{equation}\label{1.7}
(\varphi_{r,k}(x))^{(\nu)} =\begin{cases}\varphi_{r-\nu,k-\nu}(x),&\text{если $0\le\nu\le r-1$, $r\le k$,}\\
\varphi_{k-r}(x),&\text{если  $\nu=r\le k$,}\\
\varphi_{r-\nu,k-\nu}(x),&\text{если $\nu\le k< r$,}\\
0,&\text{если $k< \nu\le r-1$}.
  \end{cases}
\end{equation}
Через $W^r_{L^p_\rho(a,b)}$ обозначим пространство Соболева, состоящее из функций $f(x)$, непрерывно дифференцируемых на $[a,b]$ $r-1$ раз, причем $f^{(r-1)}(x)$ абсолютно непрерывна на $[a,b]$  и $f^{(r)}(x)\in L^p_\rho(a,b)$.
Скалярное произведение в пространстве $W^r_{L^2_\rho(a,b)}$ определим с помощью равенства \eqref{1.3}. Тогда, пользуясь определением функций  $\varphi_{r,k}(x)$ (см. \eqref{1.5} и \eqref{1.6}) и равенством  \eqref{1.7}, нетрудно увидеть (см.\cite{Shar20}),  что система $\{\varphi_{r,k}(x)\}_{k=0}^\infty$ является ортонормированной в пространстве $W^r_{L^2_\rho(a,b)}$. Следуя \cite{Shar20}, мы будем называть систему $\{\varphi_{r,k}(x)\}_{k=0}^\infty$ \textit{ ортонормированной по Соболеву } относительно скалярного произведения \eqref{1.3} и  \textit{ порожденной} ортонормированной системой $\{\varphi_{k}(x)\}_{k=0}^\infty$.
В \cite{Shar20} показано,  что ряд Фурье функции $f(x)\in W^r_{L^2_\rho(a,b)}$ по системе  $\{\varphi_{r,k}(x)\}_{k=0}^\infty$ имеет смешанный характер, а именно:
  \begin{equation}\label{1.8}
f(x)\sim \sum_{k=0}^{r-1} f^{(k)}(a)\frac{(x-a)^k}{k!}+ \sum_{k=r}^\infty \hat f_{r,k}\varphi_{r,k}(x),
\end{equation}
где
  \begin{equation}\label{1.9}
 \hat f_{r,k}=\int_a^b f^{(r)}(t) \varphi^{(r)}_{r,k}(t)\rho(t)dt=\int_a^b f^{(r)}(t) \varphi_{k-r}(t)\rho(t)dt,
\end{equation}
поэтому ряд  вида \eqref{1.8} будем  называть \textit{смешанным рядом} по  системе $\{\varphi_{k}(x)\}_{k=0}^\infty$, считая это название условным и сокращенным обозначением полного названия: <<\textit{ряд Фурье по системе  $\{\varphi_{r,k}(x)\}_{k=0}^\infty$, ортонормированной по Соболеву, порожденной ортонормированной системой $\{\varphi_{k}(x)\}_{k=0}^\infty$}>>.





\section{Некоторые результаты общего характера }

Отметим некоторые важные свойства смешанного ряда \eqref{1.8}, непосредственно вытекающие из \eqref{1.7}:
\begin{equation}\label{2.1}
f'(x)\sim \sum_{k=1}^\infty (\hat f_{r,k}\varphi_{r,k}(x))'= \sum_{k=1}^\infty f'_{r-1,k-1}\varphi_{r-1,k-1}(x).
\end{equation}
\begin{equation}\label{2.2}
\int_a^xf'(t)dt\sim \sum_{k=1}^\infty f'_{r-1,k-1}\int_a^x\varphi_{r-1,k-1}(t)dt=\sum_{k=1}^\infty \hat f_{r,k}\varphi_{r,k}(x).
\end{equation}
Важное значение имеет свойство  смешанного ряда \eqref{1.8}, которое заключается в том, что его частичная сумма вида
\begin{equation}\label{2.3}
Y_{r,N}(f,x)=\sum_{k=0}^{r-1} f^{(k)}(a)\frac{(x-a)^k}{k!}+ \sum_{k=r}^{N} \hat f_{r,k}\varphi_{r,k}(x)
\end{equation}
 при   $r\le N$  совпадает с исходной функцией $f(x)$   в точке $x=a$ $r$-кратно , т.е.
\begin{equation}\label{2.4}
(Y_{r,N}(f,x))^{(\nu)}_{x=a}=f^{(\nu)}(a)\quad (0\le\nu\le r-1).
\end{equation}
 В дальнейшем нам понадобятся  некоторые  свойства системы $\{\varphi_{r,k}(x)\}_{k=0}^\infty$, состоящей из функций, определенных равенствами   \eqref{1.5} и \eqref{1.6}, установленные в работе \cite{Shar20}.

 \textbf{ Теорема A.} \textit{  Предположим, что    функции $\varphi_k(x)$ $(k=0,1,\ldots)$ образуют полную в $L^2_\rho(a,b)$ ортонормированную   c весом   $\rho(x)$ систему на  $(a,b)$. Тогда система $\{\varphi_{r,k}(x)\}_{k=0}^\infty$, порожденная системой $\{\varphi_{k}(x)\}_{k=0}^\infty$ посредством равенств \eqref{1.5} и \eqref{1.6}, полна  в $W^r_{L^2_\rho(a,b)}$ и ортонормирована относительно скалярного произведения \eqref{1.3}.}


\textbf{ Теорема B.} \textit{
Предположим, что  $ \frac{1}{\rho(x)}\in L(a,b) $, а  функции $\varphi_k(x)$ $(k=0,1,\ldots)$  образуют полную в $L^2_\rho(a,b)$ ортонормированную   c весом   $\rho(x)$ систему на $(a,b)$, $\{\varphi_{r,k}(x)\}_{k=0}^\infty$ -- система, ортонормированная в $W^r_{L^2_\rho(a,b)}$ относительно скалярного произведения \eqref{1.8},  порожденная системой $\{\varphi_{k}(x)\}_{k=0}^\infty$ посредством равенств \eqref{1.5} и \eqref{1.6}.
Тогда если $f(x)\in W^r_{L^2_\rho(a,b)}$, то ряд Фурье (смешанный ряд) \eqref{1.8} сходится к функции $f(x)$ равномерно относительно $x\in[a,b]$.}

\begin{corollary}
Предположим, что  $ \frac{1}{\rho(x)}\in L(a,b) $, а  функции $\varphi_k(x)$ $(k=0,1,\ldots)$  образуют полную в $L^2_\rho(a,b)$ ортонормированную   c весом   $\rho(x)$ систему на $(a,b)$, $\{\varphi_{1,k}(x)\}_{k=0}^\infty$ -- система, ортонормированная в $W^1_{L^2_\rho(a,b)}$ относительно скалярного произведения \eqref{1.8} с $r=1$,  порожденная системой $\{\varphi_{k}(x)\}_{k=0}^\infty$ посредством равенств \eqref{1.5} и \eqref{1.6}.
Тогда если числовая последовательность $\{c_k\}_{k=0}^\infty$ такова, что $\sum_{k=0}^\infty c_k^2<\infty$, то  ряд  $\sum_{k=0}^\infty c_k\varphi_{1,k+1}(x)$ сходится равномерно на $[a,b]$ к функции $\eta(x)=\int_a^x g(t)dt$ и  $\eta(x)\in W^r_{L^2_\rho(a,b)}$, где $g(x)=\sum_{k=0}^\infty c_k\varphi_{k}(x)$.
\end{corollary}
%\begin{proof}
%Если $\sum_{k=0}^\infty c_k^2<\infty$ и $g(x)=\sum_{k=0}^\infty c_k\varphi_{k}(x)$, то $g(x)\in L^2_\rho(a,b)$, поэтому из условия $ \frac{1}{\rho(x)}\in L(a,b)$ следует, что $\int_a^x g(t)dt\in W^1_{L^2_\rho(a,b)}$. С другой стороны, имеем
%$$
%\left|\int_a^x g(t)dt-\sum\nolimits_{k=0}^N c_k\varphi_{1,k+1}(x)\right|=
%\left|\int_a^x [g(t)dt-\sum\nolimits_{k=0}^N c_k\varphi_{k}(t)]dt\right|\le
%$$
%$$
%\int_a^x |g(t)dt-\sum\nolimits_{k=0}^N c_k\varphi_{k}(t)|dt\le
%\int_a^b |g(t)dt-\sum\nolimits_{k=0}^N c_k\varphi_{k}(t)|dt\le
%$$
%$$
%\left(\int_a^b \frac{dt}{\rho(t)}\right)^\frac12
%\left(\int_a^b (g(t)dt-\sum\nolimits_{k=0}^Nc_k\varphi_{k}(t))^2\rho(t)dt\right)^\frac12\to 0
%$$
%при $N\to\infty$.  Следствие 1 доказано.
%
%
%
%
%\end{proof}
%
%



\vskip 0.2cm



\section{О представлении решения задачи Коши для ОДУ рядом Фурье по функциям $\varphi_{r,n}(x)$}
В настоящем разделе мы рассмотрим задачу о приближении решения задачи Коши для ОДУ  суммами  Фурье по системе $\{\varphi_{r,n}(x)\}_{n=0}^\infty$, ортогональной по Соболеву и порожденной ортонормированной системой функций $\{\varphi_{n}(x)\}_{n=0}^\infty$ посредством равенств \eqref{1.5} и \eqref{1.6} с $a=0$, $b=1$.
 Полученные ниже (теорема 1) результаты можно перенести на системы дифференциальных уравнений вида
$y'(x)=f(x,y), \quad y(0)=y_0$, где $f=(f_1, \ldots, f_m)$, $y=(y_1, \ldots, y_m)$. Но для простоты выкладок мы ограничимся рассмотрением задачи Коши вида
\begin{equation}\label{3.1}
y'(x)=f(x,y), \quad y(0)=y_0,
\end{equation}
 в которой функцию   $f(x,y)$  будем считать непрерывной в некоторой замкнутой  области $\bar G$ переменных $(x,y)$, содержащей точку $(0,y_0)$. Кроме того, мы будем  считать, что  $[0,1]\times\mathbb{R}\subset\bar G$. Это требование не сужает дальнейшие рассмотрения, так как, не ограничивая в общности,  мы можем, в случае необходимости, продолжить функцию $f(x,y)$ по переменной $y$ на всё $\mathbb{R}$, сохраняя свойство ее подчиненности  нижеследующему условию Липшица \eqref{3.3}. Например, если область $\bar G$ такова, что  прямая в $\mathbb{R}^{2}$ вида $(x,ty)$ ($t\in\mathbb{R}$) для каждого $x\in[0,1]$ и $y\in\mathbb{R}$ пересекается с границей области $\bar G$ не более, чем в двух (граничных для $\bar G$) точках $(x,y')$ и $(x,y'')$, то функцию $f(x,y)$ можно непрерывно продолжить   на $[0,1]\times\mathbb{R}$, считая ее  постоянной на лучах, выходящих из точек  $(x,y')$ и $(x,y'')$ в противоположных направлениях вдоль прямой $(x,ty)$ ($t\in\mathbb{R}$).

  Требуется аппроксимировать с заданной точностью  функцию $y=y(x)$, определенную на $[0,1]$, которая является решением задачи Коши \eqref{3.1}.
Будем считать, что весовая функция $\rho(x)$ интегрируема на $(0,1)$, а система $\{\varphi_{n}(x)\}_{n=0}^\infty$ удовлетворяет условиям теоремы \textbf{ B}, а порожденная система $\{\varphi_{1,n}(x)\}_{n=0}^\infty$ -- условиям $(0\le x\le 1)$
\begin{equation}\label{3.2}
\delta_\varphi(x)=\sum_{k=1}^{\infty}(\varphi_{1,k}(x))^2<\infty,\quad
\kappa_{\varphi}=\left(\int_0^1\sum_{k=1}^{\infty}
(\varphi_{1,k}(t))^2\rho(t)dt\right)^{\frac12}<\infty.
\end{equation}
Кроме того, мы предположим, что по переменной $y$ функция $f(x,y)$ удовлетворяет условию Липшица
 \begin{equation}\label{3.3}
|f(x,y')-f(x,y'')|\le \lambda|y'-y''|, \quad 0\le x \le 1.
\end{equation}
Через $m$ обозначим наименьшее натуральное число, для которого $h\lambda\kappa_\varphi<1$, где $h=1/m$. Если, в частности, $\lambda\kappa_\varphi<1$, то $m=1$. Полагая $x=t/m$, отобразим линейно отрезок $[0,m]$ на $[0,1]$. Относительно новой переменной $t\in [0,m]$ уравнение \eqref{3.1} принимает следующий вид
\begin{equation}\label{3.4}
\eta'(t)=hf(ht,\eta(t)), \quad \eta(0)=y_0,\quad 0\le t\le m,
\end{equation}
где $h=1/m$, $\eta(t)=y(ht)$. Мы можем представить отрезок $[0,m]$ в виде объединения отрезков $[l,l+1]$ $(l=0,1,\ldots m-1)$ и  решать поставленную задачу Коши для уравнения \eqref{3.4} сначала на $[0,1]$, а затем, используя найденное начальное значение $\eta(1)$,  решать её на $[1,2]$ и так далее. Мы ограничимся рассмотрением этой задачи для отрезка $[0,1]$. Поскольку, по предположению, функция $f(x,y)$ непрерывна в области $\bar G$, то из \eqref{3.4} следует, что  функция $\eta'(t)$ непрерывна на $[0,1]$ и, следовательно, $\eta\in W_{L_\rho^2(0,1)}^1$, поэтому в силу теоремы \textbf{ B}  мы можем представить  функцию $\eta(t)$ в виде равномерно сходящегося на $[0,1]$ ряда Фурье по порожденной системе $\{\varphi_{1,n}(t)\}_{n=0}^\infty$:
\begin{equation}\label{3.5}
\eta(t)= \eta(0)+ \sum\nolimits_{k=1}^\infty \hat \eta_{1,k}\varphi_{1,k}(t),
\end{equation}
где
  \begin{equation}\label{3.6}
\hat \eta_{1,k}=\int_{0}^1 \eta'(t)\varphi_{k-1}(t)\rho(t)dt\quad(k\ge1).
\end{equation}
Наша цель состоит в том, чтобы сконструировать итерационный процесс для нахождения приближенных значений коэффициентов $c_k=m\hat \eta_{1,k+1}$ $(k=0,1,\ldots)$. Для этого обратимся к соотношениям \eqref{1.7} и \eqref{2.1}, которые вместе с \eqref{3.5} дают
\begin{equation}\label{3.7}
\eta'(t)=  \sum\nolimits_{k=0}^\infty \hat \eta_{1,k+1}\varphi_k(t),
\end{equation}
где равенство понимается в том смысле, что ряд в правой части равенства \eqref{3.7} сходится к $\eta'$ в метрике пространства $L^2_{\rho}(0,1)$. Положим $q(t)=f(ht,\eta(t))=m\eta'(t)$ и заметим, что в силу  \eqref{3.6} (см. также \eqref{3.7}) коэффициенты Фурье функции $q=q(t)$ по системе  $\{\varphi_{n}(t)\}_{n=0}^\infty$ имеют вид
\begin{equation}\label{3.8}
 c_k(q)=\int_{0}^1 q(t)\varphi_{k}(t)\rho(t)dt=m\hat \eta_{1,k+1} \quad (k\ge0).
\end{equation}
С учетом этих равенств мы можем переписать \eqref{3.5} в следующем виде
\begin{equation}\label{3.9}
\eta(t)= \eta(0)+ h\sum\nolimits_{k=0}^\infty c_k(q){\varphi}_{1,k+1}(t).
\end{equation}
Из  \eqref{3.8} и \eqref{3.9}, в свою очередь, выводим следующие соотношения
\begin{equation}\label{3.10}
c_k(q)=\int_{0}^1f\left[ht,\eta(0)+ h\sum\nolimits_{j=0}^\infty c_j(q)\varphi_{1,j+1}(t)\right]\varphi_k(t)\rho(t) dt,\, k=0,1,\ldots.
\end{equation}
Введем в рассмотрение гильбертово пространство $l_2$, состоящее из последовательностей $C=(c_0,c_1,\ldots)$, для которых определена норма
$\|C\|=\left(\sum_{j=0}^\infty c_j^2\right)^\frac12$.  В пространстве $l_2$ рассмотрим оператор $A$, сопоставляющий точке $C\in l_2$ чиловую последовательность $C'=(c_0',c_1',\ldots)$ по правилу
\begin{equation}\label{3.11}
c_k'=\int_{0}^1f\left[ht,\eta(0)+ h\sum\nolimits_{j=0}^\infty c_j
\varphi_{1,j+1}(t)\right]\varphi_k(t)\rho(t) dt,\quad k=0,1,\ldots.
\end{equation}
Нетрудно увидеть, что $C'\in l_2$. В самом деле, из того, что $C\in l_2$, и из следствия 1 вытекает, что функция $g(t)=f[ht,\eta(0)+ h\sum\nolimits_{j=0}^\infty c_j
\varphi_{1,j+1}(t)]$ непрерывна на $[0,1]$ и, следовательно, $g(t)\in L^2_\rho(0,1)$, а отсюда и следует, что $C'\in l_2$. Из  \eqref{3.10} вытекает, что точка $C(q)=(c_0(q),c_1(q),\ldots)$ является неподвижной точкой оператора $A:l_2\to l_2$. Для того чтобы найти точку $C(q)$ методом простых итераций, достаточно показать, что оператор $A:l_2\to l_2$ является сжимающим в метрике пространства $l_2$. С этой целью рассмотрим две точки $P,Q\in l_2$, где $P=(p_0,\ldots)$, $Q=(q_0,\ldots)$, и положим $P'=A(P)$, $Q'=A(Q)$. Имеем
\begin{equation}\label{3.11}
p'_k-q'_k=\int_{0}^1F_{P,Q}(t)\varphi_k(t)\rho(t)dt,\quad k=0,1,\ldots
\end{equation}
где
\begin{multline}\label{3.12}
 F_{P,Q}(t)=f\left[ht,\eta(0)+ h\sum\nolimits_{j=0}^\infty p_j\varphi_{1,j+1}(t)\right] \\
  -f\left[ht,\eta(0)+ h\sum\nolimits_{j=0}^\infty q_j\varphi_{1,j+1}(t)\right].
\end{multline}
Из \eqref{3.11}, пользуясь неравенством Бесселя, находим
 \begin{equation}\label{3.13}
\sum\nolimits_{k=0}^\infty (p'_k-q'_k)^2\le\int_{0}^1(F_{P,Q}(t))^2\rho(t) dt.
\end{equation}
Из \eqref{3.12} и \eqref{3.3}  имеем
 \begin{equation}\label{3.15}
(F_{P,Q}(t))^2\le (h\lambda)^2   \left(\sum\nolimits_{j=0}^\infty( p_j-q_j)\varphi_{1,j+1}(t)\right)^2,
\end{equation}
откуда,  воспользовавшись неравенством Коши-Буняковского, выводим
$$
(F_{P,Q}(t))^2\le(h\lambda)^2   \sum\nolimits_{j=0}^\infty( p_j-q_j)^2\sum\nolimits_{j=0}^\infty(\varphi_{1,j+1}(t))^2.
$$
Сопоставляя \eqref{3.15} с \eqref{3.13}, находим
\begin{equation}\label{3.16}
\sum\nolimits_{k=1}^\infty (p'_k-q'_k)^2\le(h\lambda)^2 \sum\nolimits_{k=0}^\infty( p_k-q_k)^2\int_{0}^1 \sum\nolimits_{j=0}^\infty(\varphi_{1,j+1}(t))^2\rho(t) dt.
\end{equation}
Из  \eqref{3.16}  и \eqref{3.2} имеем
\begin{equation}\label{3.17}
\left(\sum\nolimits_{k=0}^\infty (p'_k-q'_k)^2\right)^\frac12\le h\kappa_\varphi\lambda \left(\sum\nolimits_{k=0}^\infty (p_k-q_k)^2\right)^\frac12. \end{equation}
Неравенство \eqref{3.17} показывает, что если $h\kappa_\varphi\lambda<1$, то оператор  $A:l_2\to l_2$ является сжимающим и, как следствие, итерационный процесс $C^{\nu+1}=A(C^{\nu})$  сходится к точке $C(q)$ при $\nu\to\infty$. Однако с точки зрения приложений важно рассмотреть конечномерный аналог оператора $A$. Мы рассмотрим оператор $A_N:\mathbb{R}^N\to \mathbb{R}^N$, cопоставляющий точке
$C_N=(c_0,\ldots,c_{N-1})\in \mathbb{R}^N $ точку  $C'_N=(c_0',\ldots,c_{N-1}')\in \mathbb{R}^N $ по правилу
\begin{equation}\label{3.18}
c_k'=\int\limits_{0}^1f\left[ht,\eta(0)+ h\sum\nolimits_{j=0}^{N-1} c_j\varphi_{1,j+1}(t)\right]\varphi_k(t)\rho(t) dt,\,k=0,1,\ldots, N-1.
\end{equation}
 Рассмотрим две точки $P_N,Q_N\in \mathbb{R}^N$, где $P_N=(p_0,p_1,\ldots,p_{N-1})$,\\   $Q_N=(q_0,q_1,\ldots,p_{N-1})$ и положим $P'_N=A_N(P_N)$, $Q'_N=A_N(Q_N)$. Дословно повторяя рассуждения, которые привели нас к неравенству \eqref{3.17}, мы получим
\begin{equation}\label{3.19}
\left(\sum\nolimits_{k=0}^{N-1} (p'_k-q'_k)^2\right)^\frac12\le h\kappa_\varphi\lambda \left(\sum\nolimits_{k=0}^{N-1} (p_k-q_k)^2\right)^\frac12.
\end{equation}
Неравенство \eqref{3.19} показывает, что если $h\kappa_\varphi\lambda<1$, то оператор  $A_N:\mathbb{R}^N\to \mathbb{R}^N$ является сжимающим и, как следствие, итерационный процесс $C_N^{\nu+1}=A_N(C_N^{\nu})$  при $\nu\to\infty$ сходится к его неподвижной точке, которую мы обозначим через  $\bar C_N(q)=(\bar c_0(q),\ldots,\bar c_{N-1}(q))$. С другой стороны, рассмотрим точку $C_N(q)=(c_0(q),\ldots,c_{N-1}(q))$, составленную из искомых коэффициентов Фурье функции $q$ по системе $\varphi$. Нам остается оценить погрешность, проистекающую в результате замены точки $C_N(q)$ точкой $\bar C_N(q)$. Другими словами, требуется оценить величину
$\|C_N(q)-\bar C_N(q)\|_N= \left(\sum_{j=0}^{N-1}(c_j(q)-\bar c_j(q))^2\right)^\frac12$. С этой целью рассмотрим точку $C'_N(q)=A_N(C_N(q))=(c_0'(q),\ldots,c_{N-1}'(q))$ и запишем
\begin{equation}\label{3.20}
\|C_N(q)-\bar C_N(q)\|_N\le \|C_N(q)- C_N'(q)\|_N+\|C_N'(q)-\bar C_N(q)\|_N.
\end{equation}
Далее, пользуясь неравенством \eqref{3.19}, имеем
$$
\|C_N'(q)-\bar C_N(q)\|_N=\|A_N(C_N(q))-A_N(\bar C_N(q))\|\le
$$
\begin{equation}\label{3.21}
h\kappa_\varphi\lambda\|C_N(q)-\bar C_N(q)\|_N.
\end{equation}
Из \eqref{3.20} и \eqref{3.21} выводим
\begin{equation}\label{3.22}
\|C_N(q)-\bar C_N(q)\|_N\le \frac1{1-h\kappa_\varphi\lambda}\|C_N(q)- C_N'(q)\|_N.
\end{equation}
Чтобы оценить норму в правой части неравенства \eqref{3.22}, заметим, что в силу неравенства Бесселя
\begin{equation}\label{3.23}
\|C_N(q)- C_N'(q)\|_N^2\le \int_{0}^1(F_{C(q),C_N(q)}(t))^2\rho(t) dt,
\end{equation}
где
\begin{multline}\label{3.24}
 F_{C(q),C_N(q)}(t)=f\left[ht,\eta(0)+ h\sum\nolimits_{j=0}^\infty c_j(q)\varphi_{1,j+1}(t)\right] \\
  -f\left[ht,\eta(0)+ h\sum\nolimits_{j=0}^{N-1}c_j(q)\varphi_{1,j+1}(t)\right].
\end{multline}
Из \eqref{3.24} и \eqref{3.3} следует, что
$$
(F_{C(q),C_N(q)}(t))^2\le \lambda^2   \left(\sum\nolimits_{j=N}^\infty hc_j(q)\varphi_{1,j+1}(t)\right)^2,
$$
отсюда с учетом \eqref{3.7} имеем
\begin{equation}\label{3.25}
(F_{C(q),C_N(q)}(t))^2\le \lambda^2   \left(\sum\nolimits_{j=N}^\infty  \hat \eta_{1,j+1}\varphi_{1,j+1}(t)\right)^2.
\end{equation}
Сопоставляя \eqref{3.25} с \eqref{3.23}, получаем
\begin{equation}\label{3.26}
\|C_N(q)- C_N'(q)\|_N^2\le \lambda^2\int_{0}^1\left(\sum\nolimits_{j=N}^\infty \hat \eta_{1,j+1} \varphi_{1,j+1}(t)\right)^2\rho(t) dt,
\end{equation}
где согласно \eqref{3.6}
\begin{equation}\label{3.27}
 \hat \eta_{1,j+1}=\int_{0}^1\eta'(t)\varphi_j(t)\rho(t)dt \quad(j=0,1,\ldots)
\end{equation}
-- коэффициенты Фурье функции $\eta'(t)=hf(ht,\eta(t))$.

Подводя итоги, из \eqref{3.22} и \eqref{3.26}  мы можем сформулировать следующий результат.
\begin{theorem} Пусть область $\bar G$ такова, что $[0,1]\times\mathbb{R}\subset \bar G$, функция $f(x,y)$ непрерывна в области $\bar G$ и удовлетворяет условию Липшица \eqref{3.3}, а $h$ и $\lambda$ удовлетворяет неравенству $h\lambda\kappa_\varphi<1$, где величина $\kappa_\varphi$ определена равенством \eqref{3.2}. Далее, пусть $l_2$ гильбертово пространство, состоящее из последовательностей $C=(c_0,\ldots)$, для которых введена норма $\|C\|=\left(\sum_{j=0}^\infty c_j^2\right)^\frac12$,   оператор $A: l_2\to l_2$ сопоставлят точке $C\in l_2$ точку $C'\in l_2$ по правилу \eqref{3.9}. Кроме того, пусть $A_N:\mathbb{R}^N\to \mathbb{R}^N$ -- конечномерный аналог оператора $A$, cопоставляющий точке $C_N=(c_0,\ldots,c_{N})\in \mathbb{R}^N $ точку  $C'_N=(c_0',\ldots,c_{N}')\in \mathbb{R}^N $ по правилу \eqref{3.18}.
Тогда операторы $A: l_2\to l_2$ и $A_N:\mathbb{R}^N\to \mathbb{R}^N$ являются сжимающими и, следовательно, существуют  их неподвижные точки $C(q)=(c_0(q),c_1(q),\ldots)=A(C(q))\in l_2$ и $\bar C_N(q)=(\bar c_0(q),\bar c_0(q),\ldots,\bar c_{N}(q))=A_N(\bar C_N(q))\in \mathbb{R}^N$, для которых имеет место неравенство
\begin{equation}\label{3.28}
\|C_N(q)-\bar C_N(q)\|_N\le \frac{\lambda \sigma_N^\varphi(\eta)}{1-h\kappa_\varphi\lambda},
\end{equation}
где
\begin{equation}\label{3.29}
\sigma_N^\varphi(\eta)=\left(\int_{0}^1\left(\sum\nolimits_{j=N+1}^\infty \hat \eta_{1,j}\varphi_{1,j}(t)\right)^2\rho(t) dt\right)^\frac12,
\end{equation}
 a $C_N(q)=(c_0(q),\ldots,c_{N-1}(q))$ -- конечная последовательность, составленная из первых $N$ компонент точки  $C(q)$, при этом в силу  \eqref{3.7} справедливо также равенство  $hC_N(q)=(\hat \eta_{1,1},\hat \eta_{1,2}, \ldots, \hat \eta_{1,N})$.
\end{theorem}


Заметим,  что величина
\begin{equation}\label{3.30}
 V_N(\eta,t)=V_N^\varphi(\eta,t)=\eta(x)- Y_{1,N}(\eta,x)
=\sum\nolimits_{j=N+1}^\infty \hat \eta_{1,j}\varphi_{1,j}(t),
\end{equation}
  фигурирующая в правой части равенства \eqref{3.29}, представляет собой остаточный член ряда Фурье функции $\eta$ по функциям $\varphi_{1,k}(t)$ $(k=0,1,\ldots)$, образующим ортонормированную систему по Соболеву относительно скалярного произведения \eqref{1.3}, порожденным  функциями $\varphi_k$ посредством равенств \eqref{1.5} и \eqref{1.6} с $r=1$, $a=0$, $b=1$.  Неравенство \eqref{3.28} непосредственно приводит к  задаче об исследовании поведения величины $\sigma_N^\varphi(\eta)=(\int_{0}^1(\eta(x)- Y_{1,N}(\eta,x))^2 dx)^\frac12$.
Другими словами, требуется исследовать задачу об оценке приближения функции $\eta$ в метрике пространства $L^2_{\rho}(0,1)$ частичными суммами   ряда Фурье \eqref{3.5} вида $Y_{1,N}(\eta,x)= \eta(0)+ \sum\nolimits_{k=1}^N \hat \eta_{1,k}\varphi_{1,k}(t).$

\section{О величинах $\delta_\varphi(x)$, $\kappa_\varphi$ и $\sigma_N^\varphi(\eta)$ для системы Хаара }

Анализ результатов, полученных в предыдущем параграфе, показывает, что решающую роль
при их доказательстве играют верхние оценки для величин $\delta_\varphi(x)$ и $\kappa_\varphi$, $\sigma_N^\varphi(\eta)$, определенных равенствами из \eqref{3.2} и \eqref{3.29}.  В настоящем параграфе мы рассмотрим вопрос об этих оценках для   системы Хаара.

При исследовании системы функций $\mathcal{ X}_{r,n}(x)$ $(n=0,1,\ldots)$, ортогональных по Соболеву, порожденных функциями  Хаара $\varphi_k(x)=\chi_{k+1}(x)$ $(k=0,1,\ldots)$ нам понадобятся следующие обозначения \cite{KashSaak}. Пусть $n=2^k+i$, $i=1,2,\ldots, 2^k$,  $k=0,1,\ldots$,
$\Delta_n=\Delta_k^i=(\frac{i-1}{2^k},\frac{i}{2^k})$, $\bar\Delta_n=[\frac{i-1}{2^k},\frac{i}{2^k}]$, ($n\ge2$), $\Delta_1=[0,1]$,
$$
 \Delta_n^+=(\Delta_k^i)^+=(\frac{i-1}{2^k},\frac{2i-1}{2^{k+1}})=\Delta_{k+1}^{2i-1}, \quad \Delta_n^-=(\Delta_k^i)^-=(\frac{2i-1}{2^{k+1}},\frac{i}{2^{k}})=\Delta_{k+1}^{2i}.
$$
Система Хаара - это система функций, $\chi=\{\chi_n\}_{n=1}^\infty(x)$, $x\in[0,1]$, в которой $\chi_1(x)=1$, a функция $\chi_n(x)$
с $2^k<n\le 2^{k+1}$, $k=0,1,\ldots$ определяется равенством
\begin{equation}\label{4.1}
\chi_n(x)=\begin{cases} 0,&\text{если $x\notin \bar\Delta_n$,}\\
2^{k/2},& \text{если $x\in \Delta_n^+$,}\\
-2^{k/2},& \text{если $x\in \Delta_n^-$.}
\end{cases}
\end{equation}
Значения в точках разрыва функции $\chi_n(x)$ выбираются так, чтобы выполнялись равенства:
$$
\chi_n(x)=\lim_{\delta\to0}\frac12[\chi_n(x+\delta)+\chi_n(x-\delta)],\quad x\in (0,1),
$$
$$
\chi_n(0)=\lim_{\delta\to0} \chi_n(\delta),\quad \chi_n(1)=\lim_{\delta\to0} \chi_n(1-\delta).
$$
Непосредственно из \eqref{4.1} вытекает следующее свойство ортогональности функций Хаара:
\begin{equation}\label{4.2}
\int_0^1\chi_n(x)\chi_m(x)dx=\delta_{nm}.
\end{equation}


Положим $\varphi_k(x)=\chi_{k+1}(x)$ $(k=0,1,\ldots)$, и для каждого натурального $r$ мы определим систему функций $\{\mathcal{ X}_{r,n}\}_{n=r}^\infty$ следующим образом:
\begin{equation}\label{4.3}
\mathcal{ X}_{r,k+r}(x) =\frac{1}{(r-1)!}\int\limits_0^x(x-t)^{r-1}\varphi_k(x)dt, \quad k=0,1, 2, \ldots.
\end{equation}
 Кроме того, определим конечный набор функций
  \begin{equation}\label{4.4}
\mathcal{ X}_{r,n}(x) =\frac{x^n}{n!}, \quad n=0,1,\ldots, r-1.
\end{equation}

  Пусть $W^r_{L^2(0,1)}$ --  пространство Соболева с  $L^2(0,1)=L^2_\rho(0,1)$, а $\rho=\rho(x) \equiv 1$. Из теоремы 1 непосредственно вытекает
 \begin{corollary}
   Система $\mathcal{ X}_r=\{\mathcal{ X}_{r,n}(x)\}_{n=0}^\infty$, порожденная системой Хаара $\{\chi_{n}(x)\}_{n=1}^\infty$ посредством равенств \eqref{4.3} и \eqref{4.4}, полна  в $W^r_{L^2(0,1)}$ и ортонормирована относительно скалярного произведения
 \begin{equation}\label{4.5}
<f,g>=\sum_{\nu=0}^{r-1}f^{(\nu)}(0)g^{(\nu)}(0)+\int_{0}^{1} f^{(r)}(t)g^{(r)}(t) dt.
\end{equation}
 \end{corollary}

Ряд Фурье функции $\eta\in W^r_{L^2(0,1)}$ по системе $\mathcal{ X}_r$ имеет следующий вид (см. \eqref{1.8})
  \begin{equation}\label{4.6}
\eta(x)\sim \sum_{n=0}^{r-1} \eta^{(n)}(0)\frac{x^n}{n!}+ \sum_{n=r}^\infty\hat \eta_{r,n}\mathcal{ X}_{r,n}(x),
\end{equation}
где
  $$
\hat \eta_{r,n}=\int_0^1 \eta^{(r)}(t)\varphi_{n-r}(t)dt=\int_0^1 \eta^{(r)}(t) \chi_{n-r+1}(t)dt.
$$
В работах \cite{Shar19} и \cite{Shar20}  получено  следующее представление для функций $\mathcal{ X}_{r,n}(x)$, определенных равенством \eqref{4.3}
 ($n=2^k+i$, $i=1,2,\ldots,2^k $, $k=0,1,\ldots$ )
  \begin{equation}\label{4.7}
 \mathcal{ X}_{r,n+r-1}(x)=\frac{2^{\frac{k}{2}}}{r!}
 \begin{cases} 0,&\text{$0\le x\le\frac{i-1}{2^k}$;}\\
 (x-\frac{i-1}{2^k})^r,&\text{$\frac{i-1}{2^k}\le x\le \frac{2i-1}{2^{k+1}}$;}\\
 (x-\frac{i-1}{2^k})^r-2(x-\frac{2i-1}{2^{k+1}})^r,&\text{$\frac{2i-1}{2^{k+1}}\le x\le \frac{i}{2^{k}}$;}\\
  (x-\frac{i-1}{2^k})^r-2(x-\frac{2i-1}{2^{k+1}})^r+(x-\frac{i}{2^{k}})^r, &\text{$\frac{i}{2^{k}}\le x\le1$}
   \end{cases}
  \end{equation}
 и, как следствие, доказано, что

   \begin{equation}\label{4.8}
 \mathcal{ X}_{r,n+r-1}(x)=\frac{2^{k/2}}{r!}\Delta^2_\frac{1}{2^{k+1}}(x-t)^r_+
 \big|_{t=\frac{i-1}{2^k}},
\end{equation}
где $\Delta^2_h g(t)$ -- конечная разность второго порядка с шагом $h$, $a_+^r=a^r, \text{ если } a\ge0 $ и $a_+^r=0, \text{ если } a<0 $. Отметим частный случай равенства \eqref{4.7} при $r=1$:
\begin{equation}\label{4.9}
 \mathcal{ X}_{1,n}(x)=2^{\frac{k}{2}}
 \begin{cases} 0,&\text{$0\le x\le\frac{i-1}{2^k}$;}\\
 x-\frac{i-1}{2^k},&\text{$\frac{i-1}{2^k}\le x\le \frac{2i-1}{2^{k+1}}$;}\\
 \frac{i}{2^{k}}-x,&\text{$\frac{2i-1}{2^{k+1}}\le x\le \frac{i}{2^{k}}$;}\\
  0, &\text{$\frac{i}{2^{k}}\le x\le1$}.
   \end{cases}
  \end{equation} Из \eqref{4.9} вытекает следующее неравенство
\begin{equation}\label{4.10}
 \mathcal{ X}_{1,n}(x)\le 2^{-\frac{k}{2}-1}, \quad 0\le x\le 1, n=2^k+i,\quad i=1,\ldots,2^k, k=0,\ldots.
  \end{equation}

Отметим, что система функций $F_0(x)=\mathcal{ X}_{1,0}(x)=1$, $F_1(x)=\mathcal{ X}_{1,1}(x)=x$,
$F_n(x)=2\max_{0\le t\le1}|\chi_{n}(t)|\mathcal{ X}_{1,n}(x), n=2,\ldots,$
введенная впервые  в работе \cite{Faber}, представляет собой классическую систему Фабера-Шаудера, являющуюся базисом Шаудера в пространстве непрерывных функций $C[0,1]$.  Однако, насколько нам известно,  система Фабера-Шаудера ранее не рассматривалась как ортогональнюю по Соболеву относительно скалярного произведения \eqref{4.5} и в этом качестве не применялась.

Используя свойства \eqref{4.7} и \eqref{4.8}, в цитированных выше работах  \cite{Shar19} и \cite{Shar20} исследованы аппроксимативные свойства частичных сумм  ряда \eqref{4.6} следующего вида
\begin{equation}\label{4.11}
 Y_{r,N}(\eta,x)=\sum_{n=0}^{r-1} \eta^{(n)}(0)\frac{x^n}{n!}+ \sum_{n=r}^{N}\hat \eta_{r,n}\mathcal{ X}_{r,n}(x)
 \end{equation}
 для функций $f$ из  классов  $W^r_{L^p(0,1)}$. В частности, в \cite{Shar19}, \cite{Shar20} получен следующий результат (см., например, теорему 3 из \cite{Shar20}). Пусть $r\ge1$, $f\in W^r_{L(0,1)}$, $n=N-r+1$.  Тогда имеет место следующая оценка
\begin{equation}\label{4.12}
|\eta^{(r-1)}(x)-Y_{r,N}^{(r-1)}(\eta,x)| \le \omega_2(\eta^{(r-1)},1/N), \quad x\in[0,1],
\end{equation}
где $\omega_2(g,\delta)=\sup_{\atop{ 0\le h\le\delta,\atop h\le x\le 1-h}}|g(x+h)+g(x-h)-2g(x)|$ -- модуль непрерывности второго порядка для функции $g$.

Из \eqref{3.29} и \eqref{4.12} непосредственно выводим
\begin{corollary}
  Пусть $\varphi=\{\varphi_k(x)=\chi_{k+1}(x)\}_{k=0}^\infty$ -- система Хаара (со сдвинутым на $1$ индексом), $\mathcal{ X}_1$ -- ассоциированная с ней система, определенная посредством равенств \eqref{4.3} и \eqref{4.4} с $r=1$.
Тогда   справедливо следующее неравенство
$$\sigma_N^{\varphi}(\eta)\le \omega_2(\eta,1/N).$$
\end{corollary}
\begin{theorem}
  Пусть $\varphi=\{\varphi_k(x)=\chi_{k+1}(x)\}_{k=0}^\infty$ -- система Хаара (со сдвинутым на $1$ индексом), $\mathcal{ X}_1$ -- ассоциированная с ней система, определенная посредством равенств \eqref{4.3} и \eqref{4.4} с $r=1$. Тогда для величин $\delta_\varphi(x)$ и $\kappa_\varphi$, определенных равенствами из \eqref{3.2}, имеют место соотношения
 $$
\delta_\varphi(x)\le 5/4,\quad\kappa_{\varphi}=\sqrt{1/2}.
$$
\end{theorem}
%\begin{proof}
%  Оценим сначала $\delta_\varphi(x)$. Прежде всего заметим, что $\mathcal{ X}_{1,1}(x)=x$, $\mathcal{ X}_{1,2}(x)=x$ при $0\le x\le1/2$  и $\mathcal{ X}_{1,2}(x)=1-x$ при $1/2\le x\le1$. Если же $n>1$, то мы можем воспользоваться равенствами \eqref{4.9} и   оценкой \eqref{4.10}. Поэтому мы можем записать
%  $$
%  \delta_\varphi(x)=(\mathcal{ X}_{1,1}(x))^2+(\mathcal{ X}_{1,2}(x))^2+\sum_{n=2}^{\infty}(\mathcal{ X}_{1,n}(x))^2\le 1+\sum_{k=1}^\infty 2^{-k-2}=1+1/4=5/4.
%  $$
%Тем самым первое утверждение теоремы 2 доказано. Рассмотрим $\kappa_\varphi$. Имеем
%$$
%  (\kappa_\varphi)^2=\int_0^1\sum_{n=1}^\infty(\mathcal{ X}_{1,n}(x))^2dx.
%$$
%Далее, пользуясь равенствами \eqref{4.9}, при $n=2^k+i$ находим
% $$
% \int_0^1(\mathcal{ X}_{1,n}(x))^2dx= 2^{k+1}\int_{\frac{i-1}{2^k}}^{\frac{2i-1}{2^{k+1}}}(x-\frac{i-1}{2^k})^2dx=\frac{1}{3}2^{-2k-2}.
% $$
% Эти равенства, взятые вместе, дают
%$$
%  (\kappa_\varphi)^2=\int_0^1x^2dx+\sum_{k=0}^\infty\sum_{i=1}^{2^k}\int_0^1(\mathcal{ X}_{1,2^k+i}(x))^2dx=\frac12.
%$$
%Теорема 2 доказана.
%\end{proof}

Из \eqref{3.28}, следствия 2 и теоремы 2 выводим
\begin{corollary}
  Пусть $\varphi=\{\varphi_k(x)=\chi_{k+1}(x)\}_{k=0}^\infty$ -- система Хаара (со сдвинутым на $1$ индексом), $\mathcal{ X}_1$ -- ассоциированная с ней система, определенная посредством равенств \eqref{4.3} и \eqref{4.4} с $r=1$.
Тогда   справедливо следующее неравенство
$$\|C_N(q)-\bar C_N(q)\|_N\le \frac{\lambda \sqrt{2}\omega_2(\eta,1/N)}{\sqrt{2}-h\lambda}.$$
\end{corollary}




\section{О величинах $\delta_\varphi(x)$, $\kappa_\varphi$ и $\sigma_N^\varphi(\eta)$ для системы косинусов }


Рассмотрим систему функций
\begin{equation}\label{5.1}
 \xi_0(x)=1,\quad \{\xi_n(x)=\sqrt{2}\cos(\pi nx)\}_{n=1}^\infty,
 \end{equation}
 которая является ортонормированной относительно скалярного произведения $<f,g>=\int_0^1f(x)g(x)dx$, т.е.
$
 <\xi_n,\xi_m>=\int_0^1\xi_n(x)\xi_m(x)dx=\delta_{nm}.
$
Соответствующая ей порожденная система $\{\xi_{1,k}\}_{k=0}^\infty$ с $r=1$ ортонормирована
относительно скалярного произведения $ <f,g>=f(0)g(0)+\int_0^1f'(x)g'(x)dx$
и имеет вид
$$
 \xi_{1,0}(x)=1,\quad \xi_{1,n}(x)=\int_0^x \xi_{n-1}(t)dt, \quad n=1,2,\ldots,
$$
т.е.
\begin{equation}\label{5.2}
 \xi_{1,0}(x)=1,\quad \xi_{1,1}(x)=x,\quad \xi_{1,n+1}(x)=\frac{\sqrt{2}}{\pi n}\sin(\pi nx),\quad n=1,2,\ldots.
\end{equation}
Ряд Фурье функции $\eta$ по системе $\{\xi_{1,k}\}_{k=0}^\infty$ имеет вид
 \begin{equation}\label{5.3}
 \eta(t)=\eta(0) +\sum\nolimits_{j=1}^\infty \hat \eta_{1,j}\xi_{1,j}(t),
\end{equation}
где
$$
\hat \eta_{1,1}=\int_{0}^1 \eta'(t)dt,\quad
\hat \eta_{1,k+1}=\sqrt{2}\int_{0}^1 \eta'(t)\cos(\pi kt)dt.
$$
\begin{theorem}
  Пусть $\xi=\{\xi_k(x)\}_{k=0}^\infty$ -- система  \eqref{5.1}, для которой  \eqref{5.2} является порожденной. Тогда для величин $\delta_{\xi}(x)$ и    $\kappa_\xi$  имеют место соотношения
 $$
\delta_{\xi}(x)\le 4/3,\quad\kappa_{\xi}=\sqrt{1/2}.
$$
\end{theorem}
%\begin{proof} Из \eqref{5.2} имеем
% $$
%\kappa_{\xi}^2=\sum_{n=1}^\infty \int_{0}^1 (\xi_{1,n}(x))^2dx=
%\int_{0}^1 x^2dx+\sum_{n=1}^\infty\frac{2}{(n\pi)^2} \int_{0}^1 \sin^2(\pi nx)dx
%$$
%$$
%=\frac13+\frac1{\pi^2}\sum_{n=1}^\infty\frac1{n^2}=\frac13+\frac1{\pi^2}\frac{\pi^2}{6}=
%\frac12,
% $$
%$$
% \delta_{\xi}(x)= x^2+\frac2{\pi^2}\sum_{n=1}^\infty\frac{\sin^2(\pi nx)}{n^2}\le
%1+ \frac2{\pi^2}\sum_{n=1}^\infty\frac{1}{n^2}\le \frac43.
%$$
% Теорема 3 доказана.
%\end{proof}

 В связи с неравенством \eqref{3.28} возникает задача об исследовании поведения остаточного члена $ V_N(\eta,t)=\eta(x)- Y_{1,N}(\eta,x)
=\sum\nolimits_{j=N+1}^\infty \hat \eta_{1,j}\xi_{1,j}(t)$  ряда Фурье \eqref{5.3}. Прежде всего мы усилим результат, полученный в теореме \textbf{ B}. А именно, мы покажем, что  условие $\eta\in W^1_{L^2(0,1)}$ можно заменить условием $\eta\in W^1_{L^1(0,1)}$. Поскольку для построения смешанного ряда \eqref{5.3} требуется, чтобы функция $\eta$ была абсолютно непрерывной на $[0,1]$, то следующая теорема носит окончательный характер.
\begin{theorem}
 Пусть $\{\varphi_{1,k}(x)=\xi_{1,k}(x)\}_{k=0}^\infty$. Тогда, если $\eta\in W^1_{L^1(0,1)}$, то ряд Фурье (смешанный ряд) \eqref{5.3} сходится к функции $\eta(x)$ равномерно относительно $x\in[0,1]$.
 \end{theorem}
%\begin{proof} С учетом равенств \eqref{5.2} мы можем переписать \eqref{5.3} в следующем виде
%\begin{equation}\label{5.4}
%\eta(t)= \eta(0)+\hat \eta_{1,1}x+\sqrt{2} \sum\nolimits_{k=1}^\infty \hat \eta_{1,k+1}\frac{\sin(\pi kx)}{\pi k},
%\end{equation}
%где
%  \begin{equation}\label{5.5}
%\hat \eta_{1,1}=\int_{0}^1 \eta'(t)dt=\eta(1)-\eta(0),
%\end{equation}
%$$
%\hat \eta_{1,k+1}=\sqrt{2}\int_{0}^1 \eta'(t)\cos(\pi kt)dt=
%$$
%$$
%\sqrt{2}\int_{0}^1 [\eta(t)-\eta(0)-(\eta(1)-\eta(0))t]'\cos(\pi kt)dt=
%$$
%\begin{equation}\label{5.6}
%\sqrt{2}\pi k\int_{0}^1 [\eta(t)-\eta(0)-(\eta(1)-\eta(0))t]\sin(\pi kt)dt.
%\end{equation}
%Из \eqref{5.4} -- \eqref{5.6} имеем
%\begin{equation}\label{5.7}
%\bar\eta(x)=\eta(x)-\eta(0)-(\eta(1)-\eta(0))x= \sum\nolimits_{k=1}^\infty b_k\sin(\pi kx),
%\end{equation}
%где
%\begin{equation}\label{5.8}
%b_k=2\int_{0}^1 \bar\eta(t)\sin(\pi kt)dt.
%\end{equation}
%Правая часть равенства \eqref{5.7} представляет собой  тригонометрический ряд Фурье функции $\bar\eta(x)=  \eta(x)-\eta(0)-(\eta(1)-\eta(0))x$, продолженной на всю  числовую ось по нечетности и $2$-периодически. Далее заметим, из абсолютной непрерывности функции $\eta(x)$ на $[0,1]$ следует, что функция $\bar\eta(x)$ абсолютно непрерывна на $[-1,1]$, поэтому ряд Фурье \eqref{5.7} равномерно на $[-1,1]$ сходится к $\bar\eta(x)$. Это равносильно тому, что ряд \eqref{5.4} или, что то же, ряд \eqref{5.3} сходится равномерно на $[0,1]$ к своей сумме $\eta(x)$. Теорема 4 доказана.
%\end{proof}

Вернемся к задаче об исследовании поведения остаточных членов ряда Фурье \eqref{5.3}   вида $V_N(\eta,t)=\sum\nolimits_{j=N+1}^\infty \hat \eta_{1,j}\xi_{1,j}(t)$, фигурирующих в правой части неравенства \eqref{3.28}. Как было показано выше, ряд \eqref{5.3} допускает его преобразование к виду \eqref{5.7}, и, соответственно, имеем
\begin{equation}\label{5.9}
V_N(\eta,t)=\eta(x)-Y_{N-1}(\eta,t)= \sum\nolimits_{k=N}^\infty b_k\sin(\pi kx),
\end{equation}
где  $Y_{N-1}(\eta,t)$ частичная сумма ряда \eqref{5.3} вида
$$
Y_{N-1}(\eta,t)= \eta(0)+\hat\eta_{1,k}\xi_{1,1}(t)+ \sum\nolimits_{k=2}^N \hat \eta_{1,k}\xi_{1,k}(t)
$$
\begin{equation}\label{5.10}
= \eta(0)+(\eta(1)-\eta(0))x+\sum\nolimits_{k=1}^{N-1} b_k\sin(\pi kx).
\end{equation}
Рассмотрим частичную сумму
\begin{equation}\label{5.11}
S_{N-1}(\bar\eta,x)= \sum\nolimits_{k=1}^{N-1} b_k\sin(\pi kx)
\end{equation}
тригонометрического ряда Фурье \eqref{5.7} для функции $\bar\eta(x)$ и заметим, что в силу \eqref{5.9} -- \eqref{5.11} имеем
\begin{equation}\label{5.12}
V_N(\eta,x)= \sum\nolimits_{k=N}^{\infty} b_k\sin(\pi kx)=\bar\eta(x)-S_{N-1}(\bar\eta,x).
\end{equation}
Отсюда мы замечаем, что задача об исследовании поведения величины $V_N(\eta,x)$ сводится к аналогичной задаче для остатка ряда Фурье
\begin{equation}\label{5.13}
R_N(\bar\eta,x)= \bar\eta(x)-S_{N-1}(\bar\eta,x).
\end{equation}
В частности, для величины $\sigma_N^\varphi(\eta)=\|V_N(\eta)\|_{L^2(0,1)}$, фигурирующей в неравенстве \eqref{3.28}, имеем
\begin{equation}\label{5.14}
\sigma_N^\xi(\eta)=\left(\int_{0}^1\left(\sum\nolimits_{j=N}^\infty \hat \eta_{1,j+1}\xi_{1,j+1}(t)\right)^2 dt\right)^\frac12=\|R_N(\bar\eta)\|_{L^2(0,1)}.
\end{equation}
Обозначим через $E_m(\bar\eta)_2$ наилучшее приближение функции $\bar\eta$ в $L^2(-1,1)$ тригонометрическими полиномами вида $T_m(x)=a_0+\sum_{k=1}^{m}a_k\cos(\pi kx)+b_k\sin(\pi kx)$. Поскольку $S_{N-1}(\bar\eta,x)$ является полиномом наилучшего приближения к функции $\bar\eta$ в $L^2(-1,1)$, то из \eqref{5.14}, с учетом того, что функция $\bar\eta$ нечетная, имеем
\begin{equation}\label{5.15}
\sigma_N^\varphi(\eta)=\frac12\|R_N(\bar\eta)\|_{L^2(-1,1)}=\frac12E_{N-1}(\bar\eta)_2.
\end{equation}
Из теоремы 3 и равенства \eqref{5.15} непосредственно вытекает
\begin{theorem}
  При соблюдении условий теоремы 1 имеют место соотношения
  \begin{equation}\label{5.16}
  \sigma_N^\varphi(\eta)=\frac12E_{N-1}(\bar\eta)_2, \quad
\|C_N(q)-\bar C_N(q)\|_N\le \frac{\lambda E_{N-1}(\bar\eta)_2} {\sqrt{2}(\sqrt{2}-h\lambda)}.
\end{equation}
\end{theorem}






\section{О величинах  $\kappa_\varphi$ и $\sigma_N^\varphi(\eta)$ для модифицированныx полиномов Чебышева первого рода}


Рассмотрим систему $\varphi=\bar T$, составленную из модифицированных полиномов Чебышева первого рода:
 \begin{equation}\label{6.1}
1,\quad \bar T_k(x)=\sqrt{2}\cos(k\arccos(2x-1)), \quad k=1,2,\ldots.
\end{equation}
 Полиномы \eqref{6.1} образуют ортонормированную   с весом  $\rho(x)=\frac1\pi(x(1-x))^{-\frac12}$ систему на $(0,1)$. Прежде, чем  исследовать полиномы $\bar T_{r,k}(x)$, порожденные системой $\bar T$, нам будет удобно рассмотреть аналогичную проблему для классических  полиномов Чебышева
 \begin{equation}\label{6.2}
\frac{1}{\sqrt{2}},\quad  T_k(x)=\cos(k\arccos(x)), \quad k=1,2,\ldots,
\end{equation}
ортонормированных на $(-1,1)$ c весом  $\mu(x)=\frac2\pi(1-x^2)^{-\frac12}$.
   Как хорошо известно \cite{Sege}, система полиномов Чебышева \eqref{6.2} полна в $L_\mu^2(-1,1)$.   Эта система порождает на $[-1,1]$ систему полиномов $T_{r,k}(x)$ $(k=0,1,\ldots)$, определенных равенствами

  \begin{equation}\label{6.3}
T_{r,k}(x) =\frac{(x+1)^k}{k!}, \quad k=0,1,\ldots, r-1,
\end{equation}

  \begin{equation}\label{6.4}
 T_{r,r}(x) =\frac{(x+1)^r}{\sqrt{2}r!},\quad T_{r,r+k}(x) =\frac{1}{(r-1)!}\int\limits_{-1}^x(x-t)^{r-1}T_k(t)dt, \, k=1,2\ldots.
\end{equation}
Из теоремы \textbf{ A} непосредственно вытекает
\begin{corollary}
  Система полиномов $\{T_{r,k}(x)\}_{k=0}^\infty$, порожденная системой ортонормированных полиномов Чебышева \eqref{6.2} посредством равенств \eqref{6.3} и \eqref{6.4}, полна  в $W^r_{L^2_\mu(-1,1)}$ и ортонормирована относительно скалярного произведения
\begin{equation}\label{6.5}
<f,g>=\sum_{\nu=0}^{r-1}f^{(\nu)}(-1)g^{(\nu)}(-1)+\int_{-1}^{1} f^{(r)}(t)g^{(r)}(t)\mu(t) dt.
\end{equation}
\end{corollary}

Ряд Фурье \eqref{1.8} для системы   $\{T_{r,k}(x)\}_{k=0}^\infty$ приобретает вид
\begin{equation}\label{6.6}
f(x)\sim \sum_{k=0}^{r-1} f^{(k)}(-1)\frac{(x+1)^k}{k!}+ \sum_{k=r}^\infty \hat f_{r,k}T_{r,k}(x),
\end{equation}
где
  \begin{equation}\label{6.7}
\hat f_{r,r}=\frac{1}{\sqrt{2}}\int_{-1}^1 f^{(r)}(t)\mu(t)dt,\quad \hat f_{r,r+j}=\int_{-1}^1 f^{(r)}(t)T_{j}(t)\mu(t)dt\quad(j\ge1).
\end{equation}

\begin{corollary}
 Если $f(x)\in W^r_{L^2_\mu(-1,1)}$, то ряд Фурье (смешанный ряд) \eqref{6.6} сходится к функции $f(x)$ равномерно относительно $x\in[-1,1]$.
\end{corollary}
%\begin{proof}
% Так как $\frac{1}{\mu(x)}\in L(-1,1)$, то утверждение следствия 5 вытекает из теоремы \textbf{ B} и следствия 4.
%\end{proof}

\begin{remark}
В работе \cite{Shar20} показано, что утверждение следствия 5 допускает  усиление, состоящее в том, что  условие $f(x)\in W^r_{L^2_\mu(-1,1)}$ можно заменить менее слабым требованием $f(x)\in W^r_{L^1_\mu(-1,1)}$, которое уже носит окончательный характер.
\end{remark}

В работах \cite{Shar20}, \cite{Shar25}, в частности, доказано
\begin{corollary} Имеют место равенства
\begin{equation}\label{6.8}
T_{1,k+1}(x)={T_{k+1}(x)\over2(k+1)}- {T_{k-1}(x)\over2(k-1)} -\frac{(-1)^k}{k^2-1}\quad (k\ge 2),
\end{equation}
\begin{equation}\label{6.9}
T_{1,0}(x)=1, \quad T_{1,1}(x)=\frac{1+x}{\sqrt{2}}, \quad T_{1,2}(x)=\frac12(x^2-1).
\end{equation}
\end{corollary}
 Вернемся теперь к системе \eqref{6.1}. Она порождает на $[0,1]$ систему полиномов $\bar T_{r,k}(x)$ $(k=0,1,\ldots)$, определенных равенствами

  \begin{equation}\label{6.10}
\bar T_{r,k}(x) =\frac{x^k}{k!}, \quad k=0,1,\ldots, r,
\end{equation}

  \begin{equation}\label{6.11}
\quad \bar T_{r,r+k}(x) =\frac{1}{(r-1)!}\int\limits_{0}^x(x-t)^{r-1}\bar T_k(t)dt, \, k=1,2\ldots.
\end{equation}
Сопоставляя \eqref{6.4} c \eqref{6.11}, при $k\ge1$ имеем
\begin{equation}\label{6.12}
\bar T_{r,r+k}(x) =\frac{\sqrt{2}}{(r-1)!}\int\limits_{0}^x(x-t)^{r-1}T_k(2t-1)dt=2^{\frac12-r}T_{r,r+k}(2x-1).
\end{equation}
Из равенств  \eqref{6.9}, \eqref{6.12} и следствия 6 выводим
\begin{corollary} Имеют место равенства
\begin{equation}\label{6.13}
\bar T_{1,k+1}(x)={\bar T_{k+1}(x)\over4(k+1)}- {\bar T_{k-1}(x)\over4(k-1)} -\frac{(-1)^k2^{-\frac12}}{k^2-1}\quad (k\ge 2),
\end{equation}
\begin{equation}\label{6.14}
\bar T_{1,0}(x)=1, \quad \bar T_{1,1}(x)=x, \quad \bar T_{1,2}(x)=\sqrt{2}x(x-1).
\end{equation}
\end{corollary}
\begin{theorem}
  Пусть $\bar T=\{\bar T_k(x)\}_{k=0}^\infty$ -- система  \eqref{6.1}, для которой  $\{\bar T_{1,k}(x)\}_{k=0}^\infty$  является порожденной посредством равенств \eqref{6.10}, \eqref{6.11} с $r=1$. Тогда   имеют место равенства
 $$
\kappa_{\bar T}=\left(\frac{11}{32}+\frac{\pi^2}{48}+\frac12\sum_{k=2}^\infty\frac{1}{(k^2-1)^2}\right)^\frac12,
$$
$$
(\sigma^{\bar T}_N(\eta))^2=
$$
$$
\left(\sum_{j=N+1}^{\infty}\hat\eta_{1,j}\frac{(-1)^j2^{-\frac12}}{(j-1)^2-1}\right)^2+
{(\hat\eta_{1,N+1})^2\over16(N-1)^2} +{(\hat\eta_{1,N+2})^2\over16N^2}
+\sum_{j=N+1}^{\infty}\frac{(\hat\eta_{1,j}-\hat\eta_{1,j+2})^2}{16j^2}.
$$
\end{theorem}
%\begin{proof}
%Пользуясь следствием 7, имеем
% \begin{equation}\label{6.15}
%\kappa^2_{\bar T}=\sum_{k=1}^{\infty}\int_{0}^{1}(\bar T_{1,k}(x))^2\rho(x)dx=
%\sum_{k=1}^{\infty}\frac1\pi\int_{0}^{1}\frac{(\bar T_{1,k}(x))^2dx}{\sqrt{x(1-x)}},
% \end{equation}
%где
%\begin{equation}\label{6.16}
%\frac1\pi\int_{0}^{1}\frac{(\bar T_{1,1}(x))^2}{\sqrt{x(1-x)}}=\frac1\pi\int_{0}^{1}\frac{x^2dx}{\sqrt{x(1-x)}}=\frac38,
% \end{equation}
%\begin{equation}\label{6.17}
%\frac1\pi\int_{0}^{1}\frac{(\bar T_{1,2}(x))^2}{\sqrt{x(1-x)}}=\frac1\pi\int_{0}^{1}\frac{x^2(1-x)^2dx}{\sqrt{x(1-x)}}=\frac{3}{64},
% \end{equation}
%\begin{equation}\label{6.18}
% \int_{0}^{1}(\bar T_{1,k+1}(x))^2\rho(x)dx=\int\limits_{0}^{1}\left[{\bar T_{k+1}(x)\over4(k+1)}- {\bar T_{k-1}(x)\over4(k-1)} -\frac{(-1)^k2^{-\frac12}}{k^2-1} \right]^2\rho(x)dx.
% \end{equation}
%С другой стороны, пользуясь ортонормированностью с весом $\rho(x)$ системы $\bar T$,  имеем ($k\ge2$)
%$$
%\int\limits_{0}^{1}\left[{\bar T_{k+1}(x)\over4(k+1)}- {\bar T_{k-1}(x)\over4(k-1)} -\frac{(-1)^k2^{-\frac12}}{k^2-1} \right]^2\rho(x)dx=
%$$
%\begin{equation}\label{6.19}
%{1\over16(k+1)^2}+ {1\over16(k-1)^2} +\frac{2^{-1}}{k^2-1} .
% \end{equation}
%Из \eqref{6.18} и \eqref{6.19}, пользуясь равенством $\sum_{n=1}^{\infty}{1\over n^2}=\pi^2/6$, находим
%$$
%\sum_{k=2}^{\infty}\int_{0}^{1}(\bar T_{1,k+1}(x))^2\rho(x)dx=\sum_{k=2}^{\infty}\left[{1\over16(k+1)^2}+ {1\over16(k-1)^2} +\frac{2^{-1}}{k^2-1}\right]
%$$
%\begin{equation}\label{6.20}
%=\frac{\pi^2}{48}-\frac{1}{16}(1+\frac14)+\frac12\sum_{k=2}^{\infty}\frac{1}{k^2-1}.
% \end{equation}
%Сопоставляя \eqref{6.15} -- \eqref{6.17} с \eqref{6.20}, убеждаемся в справедливости первого утверждения теоремы 6.
%
%
%Перейдем к рассмотрению величины
%\begin{equation}\label{6.21}
%\sigma^{\bar T}_N(\eta)=\left(\int_{0}^{1}(V_N^{\bar T}(\eta,t))^2\rho(t)dt\right)^\frac12 \quad(\rho(t)=\frac1\pi(t(1-t))^{-\frac12}),
%\end{equation}
%где $V_N^{\bar T}(\eta,t)=\sum_{j=N+1}^{\infty}\hat\eta_{1,j}\bar T_{1,j}(t)$,
%$\hat \eta_{1,k}=\int_{0}^1 \eta'(t)\bar T_{k-1}(t)\rho(t)dt$.
%Воспользовавшись равенством \eqref{6.13}, мы можем записать
%$$
%V_N^{\bar T}(\eta,t)=
%-\left(\sum_{j=N+1}^{\infty}\hat\eta_{1,j}\frac{(-1)^{j-1}2^{-\frac12}}{(j-1)^2-1}\right)\bar T_0(x)
%$$
%\begin{equation}\label{6.22}
%-{\hat\eta_{1,N+1}\over4(N-1)}\bar T_{N-1}(x)-{\hat\eta_{1,N+2}\over4N}\bar T_{N}(x)
%-\sum_{j=N+1}^{\infty}\frac{\hat\eta_{1,j}-\hat\eta_{1,j+2}}{4j}\bar T_{j}(x).
%\end{equation}
%Поскольку система \eqref{6.1} ортонормирована с весом $\rho(x)=\frac1\pi (x(1-x))^{-1/2}$, то из   \eqref{6.22} имеем
%$$
%\int_{0}^{1}(V_N^{\bar T}(\eta,t))^2\rho(t)dt=
%\left(\sum_{j=N+1}^{\infty}\hat\eta_{1,j}\frac{(-1)^{j-1}2^{-\frac12}}{(j-1)^2-1}\right)^2+
%$$
%\begin{equation}\label{6.23}
%{(\hat\eta_{1,N+1})^2\over16(N-1)^2} +{(\hat\eta_{1,N+2})^2\over16N^2}
%+\sum_{j=N+1}^{\infty}\frac{(\hat\eta_{1,j}-\hat\eta_{1,j+2})^2}{16j^2}.
%\end{equation}
%Из \eqref{6.21} и \eqref{6.23} вытекает справедливость второго утверждения теоремы 6.
%\end{proof}

Обозначим через $E^\rho_ m(f)$ -- наилучшее приближение функции $f\in L^2_\rho(0,1)$ алгебраическими полиномами степени $m$. Тогда из теоремы 6 мы можем вывести следующее утверждение.
\begin{corollary}
  Пусть $\bar T=\{\bar T_k(x)\}_{k=0}^\infty$ -- система  \eqref{6.1}, для которой  $\{\bar T_{1,k}(x)\}_{k=0}^\infty$  является порожденной посредством равенств \eqref{6.10}, \eqref{6.11} с $r=1$. Тогда   имеют место неравенства
$$
\sigma^{\bar T}_N(\eta)\le I_NE^\rho_ {N-1}(\eta'),\quad\|C_N(q)-\bar C_N(q)\|_N\le \frac{\lambda I_N}{1-h\kappa_{\bar T}\lambda}E^\rho_ {N-1}(\eta'),
$$
где
$$
I_N=\left(\frac{1}{4(N-1)^2}+\frac12\sum_{j=N}^{\infty}\frac{1}{(j^2-1)^2}\right)^\frac12.
 $$
\end{corollary}
%\begin{proof}
%Имеем
%$$
%\left(\sum_{j=N+1}^{\infty}\hat\eta_{1,j}\frac{(-1)^j2^{-\frac12}}{(j-1)^2-1}\right)^2\le
%$$
%\begin{equation}\label{6.24}
%\frac12 \sum_{j=N+1}^{\infty}(\hat\eta_{1,j})^2\sum_{j=N}^{\infty}\frac{1}{(j^2-1)^2}=(E^\rho_ {N-1}(\eta'))^2\frac12\sum_{j=N}^{\infty}\frac{1}{(j^2-1)^2},
%\end{equation}
%$$
%{(\hat\eta_{1,N+1})^2\over16(N-1)^2} +{(\hat\eta_{1,N+2})^2\over16N^2}
%+\sum_{j=N+1}^{\infty}\frac{(\hat\eta_{1,j}-\hat\eta_{1,j+2})^2}{16j^2}\le
%$$
%$$
%{3(\hat\eta_{1,N+1})^2\over16(N-1)^2} +{3(\hat\eta_{1,N+2})^2\over16(N-1)^2}+\sum_{j=N+3}^{\infty}\frac{(\hat\eta_{1,j})^2}{8j^2}
%+\sum_{j=N+3}^{\infty}\frac{(\hat\eta_{1,j})^2}{8(j-2)^2}\le
%$$
%\begin{equation}\label{6.25}
%\frac1{4(N-1)^2}\sum_{j=N+1}^{\infty}(\hat\eta_{1,j})^2=\frac{(E^\rho_ {N-1}(\eta'))^2}{4(N-1)^2}
%\end{equation}
%Первое утверждение следствия 8 вытекает из  \eqref{6.21}, \eqref{6.23} -- \eqref{6.25}, а второе выводим из первого из них,  первого утверждения теорем 1 и теоремы 6.
%\end{proof}
%














%%%%%%%%%%%%%%%%%%%%%%%%%%%%%%%%%%%
%%%%%%%%%%%%%%%%%%%%%%%%%%%%%%%%%%%
%%%%%%%%%%%%%%%%%%%%%%%%%%%%%%%%%%%



%
%\begin{thebibliography}{20}
%
%\RBibitem{Shar2016}
%\by И.И. Шарапудинов
%\paper Системы функций, ортогональные по Соболеву, порожденные ортогональными функциями
%\inbook Материалы 18-й международной Саратовской зимней школы «Современные проблемы теории функций и их приложения»
%\publ ООО «Издательство «Научная книга»
%\yr 2016
%\pages 329-332
%\publaddr Саратов
%
%
%\RBibitem{Shar20}
%\by И.\,И. Шарапудинов
%\paper  Ортогональные  по Соболеву системы, порожденные ортогональными функциями
%\inbook Изв. РАН. Сер. Математическая
%\vol 82
%\issue
%\yr 2018 (Принята к печати)
%\pages
%
%
%\RBibitem{IserKoch}
%\by A. Iserles, P.E. Koch, S.P. Norsett and J.M. Sanz-Serna
%\paper On polynomials  orthogonal  with respect  to certain Sobolev inner products
%\inbook ,  J. Approx. Theory
%\vol 65
%\issue
%\yr 1991
%\pages 151-175.
%
%\RBibitem{MarcelAlfaroRezola }
%\by F. Marcellan, M. Alfaro and M.L. Rezola
%\paper Orthogonal polynomials on Sobolev spaces: old and new directions
%\inbook Journal of Computational and Applied Mathematics
%\vol 48
%\issue
%\yr 1993
%\pages 113 -- 131.
%\publaddr North-Holland
%
%\RBibitem{Meijer}
%\by H.\,G. Meijer, Laguerre polynimials generalized to a certain
%\paper  Laguerre polynimials generalized to a certain discrete Sobolev inner product space
%\inbook ,  J. Approx. Theory
%\vol 73
%\issue
%\yr 1993
%\pages 1-16.
%
%
%\RBibitem{KwonLittl1}
%\by K.\,H. Kwon and L.\,L. Littlejohn
%\paper The orthogonality of the Laguerre polynomials $\{L_n^{(-k)}(x)\}$ for positive integers $k$
%\inbook Ann. Numer. Anal.
%\vol
%\issue 2
%\yr 1995
%\pages 289 –- 303.
%
%\RBibitem{Lopez1995}
%\by Lopez G. Marcellan F. Vanassche W.
%\paper Relative Asymptotics for Polynomials Orthogonal with Respect to a Discrete Sobolev Inner-Product
%\inbook Constr. Approx.
%\vol 11:1
%\yr 1995
%\pages 107–137
%
%
%
%\RBibitem{ KwonLittl2}
%\by K.\,H. Kwon and L.\,L. Littlejohn
%\paper Sobolev orthogonal polynomials and second-order differential equations
%\inbook Rocky Mountain J. Math.
%\vol 28
%\issue
%\yr 1998
%\pages 547 –- 594.
%
%
%\RBibitem{MarcelXu}
%\by F. Marcellan and Yuan Xu
%\paper On Sobolev orthogonal polynomials
%\inbook  Expositiones Mathematicae
%\vol 33
%\issue 3
%\yr 2015
%\pages 308--352
%
%\RBibitem{Shar17}
%\by И.\,И. Шарапудинов
%\paper Смешанные ряды по ультрасферическим полиномам и их аппроксимативные свойства
%\inbook Математический сборник
%\vol 194
%\issue 3
%\yr 2003
%\pages 115--148
%
%
%\RBibitem{Shar13}
%\by И.\,И. Шарапудинов
%\paper Смешанные ряды по ортогональным полиномам
%\inbook
%\publ Издательство Дагестанского научного центра
%\yr 2004
%\pages 1 --176
%\publaddr Махачкала
%
%
%
%
%
%
%
%
%\RBibitem{Tref1}
%\by   L.N. Trefethen
%\book Spectral methods in Matlab
%\yr 2000
%\serial
%\publ SIAM
%\publaddr Fhiladelphia
%\vol
%
%\RBibitem{Tref2}
%\by   L.N. Trefethen
%\book Finite difference and spectral methods for ordinary and partial differential equation
%\yr 1996
%\serial
%\publ Cornell University
%\publaddr
%\vol
%
%
%\RBibitem{SolDmEg}
%\by  В.В. Солодовников, А.Н. Дмитриев, Н.Д. Егупов
%\book Спектральные методы расчета и проектирования систем управления
%\yr 1986
%\serial
%\publ Машиностроение
%\publaddr Москва
%\vol
%
%\RBibitem{Pash}
%\by С. Пашковский
%\paper
%\inbook Вычислительные применения многочленов и рядов Чебышева
%\publ Наука
%\yr 1983
%\pages
%\publaddr Москва
%
%\RBibitem{Arush2014}
%\by О. Б. Арушанян, Н. И. Волченскова, С. Ф. Залеткин
%\paper Применение рядов Чебышева для интегрирования обыкновенных дифференциальных уравнений
%\inbook Сиб. электрон. матем. изв.
%\issue 11
%\yr 2014
%\pages 517-531
%
%\RBibitem{Lukom2016}
%\by Д.С. Лукомский, П.А. Терехин
%\paper Применение системы Хаара к численному решению задачи Коши для линейного дифференциального уравнения первого порядка
%\inbook Материалы 18-й международной Саратовской зимней школы «Современные проблемы теории функций и их приложения»
%\publ ООО «Издательство «Научная книга»
%\yr 2016
%\pages 171-173
%\publaddr Саратов
%
%
%\RBibitem{MMG2016}
%\by М.Г. Магомед-Касумов
%\paper Приближенное решение обыкновенных дифференциальных уравнений с использованием смешанных рядов по системе Хаара
%\inbook Материалы 18-й международной Саратовской зимней школы «Современные проблемы теории функций и их приложения»
%\publ ООО «Издательство «Научная книга»
%\yr 2016
%\pages 176-178
%\publaddr Саратов
%
%\RBibitem{DiffUr2017}
%\by И.И. Шарапудинов, М.Г. Магомед-Касумов
%\paper О представлении решения задачи Коши  рядом Фурье  по полиномам, ортогональным по  Соболеву, порожденным многочленами Лагерра \inbook Дифференциальные уравнения
%\yr 2017 (принята к печати)
%
%
%
%
%
%
%
%\RBibitem{Shar18}
%\by И.\,И. Шарапудинов, Т.\, И. Шарапудинов
%\paper Смешанные ряды по полиномам Якоби и Чебышева и их дискретизация
%\inbook Математические заметки
%\vol 88
%\issue 1
%\yr 2010
%\pages 116 -- 147
%
%
%
%\RBibitem{KashSaak}
%\by Б.\,С.~Кашин, А.\,А.~Саакян
%\book Ортогональные ряды
%\yr 1999
%\publ АФЦ
%\publaddr Москва
%
%
%
%
%\RBibitem{Shar19}
%\by И.\,И. Шарапудинов,  Г.\, Н. Муратова
%\paper  Некоторые свойства r-кратно интегрированных рядов по системе Хаара
%\inbook Изв. Сарат. ун-та. Нов. сер. Сер. Математика. Механика. Информатика
%\vol 9
%\issue 1
%\yr 2009
%\pages 68 -- 76
%
%
%
%
%\RBibitem{Faber}
%\by G. Faber
%\paper Ober die Orthogonalfunktionen des Herrn Haar
%\inbook Jahresber. Deutsch. Math. Verein.
%\vol 19
%\issue
%\yr 1910
%\pages 104 -- 112
%
%\RBibitem{Shar25}
%\by И.\,И. Шарапудинов
%\paper  Асимптотические свойства полиномов, ортогональных по Соболеву, порожденных полиномами Якоби
%\inbook Дагестанские электронные математические известия, 2016, выпуск 6,	страницы 1–24	(Mi demr26)		 	
%
%\vol
%\issue 6
%\yr 2016
%\pages 1 -- 24
%
%\RBibitem{Sege}
%\by Г. Сеге
%\paper
%\inbook Ортогональные многочлены
%\publ Физматгиз
%\yr 1962
%\pages
%\publaddr Москва
%
%
%
%
%
%
%
%
%
%
%
%\end{thebibliography}
%
