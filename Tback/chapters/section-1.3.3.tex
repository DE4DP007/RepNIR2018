\section{Численный метод построения положительного решения двухточечной краевой задачи для одного дифференциального уравнения второго порядка с дробной производной}
%\textbf{Получены новые достаточные условия существования и единственности положительного решения двухточечной краевой задачи для одного специального вида нелинейного обыкновенного дифференциального уравнения с дробными производными порядка $\alpha$}


\textit{Численными методами построено положительное решение двухточечной краевой задачи для одного нелинейного дифференциального уравнения с дробными производными. Этот же метод позволяет убедиться в единственности построенного решения.}

В последние годы возрос интерес к исследованию дифференциальных уравнений дробного порядка, в которых неизвестная функция содержится под знаком производной дробного порядка. Это обусловлено как развитием самой теории дробного интегрирования и дифференцирования, так и приложениями этой теории к различным областям науки.

Физики достаточно давно и плодотворно используют идеи дробного исчисления преимущественно во фрактальных средах. Дифференциальные уравнения дробного порядка встречаются при описании медленных и быстрых стохастических процессов, диффузии в средах с фрактальной геометрией, при изучении деформационно-прочностных свойств полимерных материалов. Полученные при этом результаты говорят о существовании мощного метода, каким является дробное исчисление при построении математических моделей в тех средах, где классическое дифференциальное исчисление не работает. Особый интерес к дробным производным проявляют гидрогеологи в связи с вопросами безопасности. хранения высокоактивных долгоживущих радиоизотопов в геологических фармациях.

В последние годы возросла интенсивность изучения дифференциальных уравнений с дробными производными (см., например, \cite{bailu,wangwang,zhangS,qiubai,caballero,changnieto,shangSQ,beyb,beybShab,aleroev}). В частности, имеются публикации, посвященные существованию положительных решений краевых задач для нелинейных дифференциальных уравнений второго порядка с дробными производными (см., например, \cite{bailu,wangwang}, \cite{qiubai,caballero,changnieto}, \cite{beyb}, \cite{beybDavud}). При этом публикаций на тему построения численными методами положительных решений краевых задач для нелинейных дифференциальных уравнений сравнительно мало, хотя такие задачи возникают на практике.

Нами предложен численный метод построения положительного решения двухточечной краевой задачи для одного нелинейного дифференциального уравнения с дробными производными. Этот же метод позволяет доказать существование и единственность.



\subsection{Предварительные сведения}

Рассмотрим двухточечную краевую задачу:
\begin{equation}D^{\alpha }_{0+}u\left(t\right)+f\left(t,u\left(t\right)\right)=0,\ \ 0<t<1,\label{EIO1}\end{equation}
\begin{equation}u\left(0\right)=u\left(1\right)=0, \label{EIO2}\end{equation}
где $1<\alpha\le2$ -- вещественное число, и  численными методами строится это решение, $D^{\alpha }_{0+}u\left(t\right)$ -- производная в смысле Римана-Лиувилля.
В дальнейшем нам понадобится


\begin{theorem}\label{AbEI3.3:th1}(см. \cite{bailu}, c.502) Пусть $f(t,u)$ непрерывно на $\left[0,1\right]\times[0,\infty )$. Предположим, что существуют две положительные константы $r_2>r_1>0$ такие, что

\begin{equation}\left(\text{У}1\right)\ \ f\left(t,u\right)\le Mr_2,\text{ для всех} \left(t,u\right)\in\left[0,1\right]\times\left[0,r_2\right];\label{EIO3}\end{equation}

\begin{equation}\left(\text{У}2\right)\ f\left(t,u\right)\ge Nr_1,\text{ для всех} \left(t,u\right)\in\left[0,1\right]\times\left[0,r_1\right],\label{EIO4}\end{equation}

где  $M\left(\alpha \right)={\left(\int^1_0{G\left(\alpha ,s,s\right)ds}\right)}^{-1},\ \ N\left(\alpha \right)={\left(\int^{\frac{3}{4}}_{\frac{1}{4}}{\gamma \left(\alpha ,s\right)G\left(\alpha ,s,s\right)ds}\right)}^{-1}$,

$$\gamma \left(\alpha ,s\right)=\left\{ \begin{array}{c}
\frac{{[\frac{3}{4}(1-s)]}^{\alpha -1}-{(\frac{3}{4}-s)}^{\alpha -1}}{{[s\left(1-s\right)]}^{\alpha -1}}, s\in \left(0,r\right], \\
\frac{1}{{(4s)}^{\alpha -1}},s\in \left[r,1\right), \end{array}
\right.$$
а функция Грина $G(\alpha ,t,s)$ имеет вид:
\begin{equation}
\label{EIO5}
G\left(\alpha ,t,s\right)=\left\{ \begin{array}{c}
\frac{{\left[t\left(1-s\right)\right]}^{\alpha -1}-{\left(t-s\right)}^{\alpha -1}}{\Gamma\left(\alpha \right)},\ 0\le s\le t\le1, \frac{{\left[t\left(1-s\right)\right]}^{\alpha -1}}{\Gamma\left(\alpha \right)}, 0 \le t \le s \le 1. \end{array}
\right.
\end{equation}
Тогда задача \eqref{EIO1}--\eqref{EIO2} имеет не менее одного положительного решения и такого, что $r_1\le \left\|u\right\|\le r_2$.
\end{theorem}

\subsection{Единственность и  численный метод построения положительного решения}

Рассмотрим численный метод построения положительного решения двухточечной краевой задачи на примере задачи
\begin{equation} \label{EIO6} {D}^{\alpha }_{0+}u\left(t\right)+u^2+\frac{\sin t}{4}+1=0,\ \ 0<t<1,\  \end{equation}
\begin{equation}u\left(0\right)=u\left(1\right)=0,\ \label{EIO7}\end{equation} где $\ \ 3/2\le \alpha \le 2.$
Справедлива

\begin{lemma}\label{AbEI3.3:lemm}Для любого $3/2\le \alpha \le 2$ функция $f\left(t,u\right)=u^2+\frac{\sin t}{4}+1$ удовлетворяет условиям теоремы \ref{AbEI3.3:th1}.
\end{lemma}

%Доказательство леммы
%\textbf{Доказательство.}  Для этого, используя теорему \ref{AbEI3.3:th2}, оценим снизу
%$$M\left(\alpha \right)={\left(\int^1_0{G\left(\alpha ,s,s\right)ds}\right)}^{-1},$$
%где  функция $\ G(\alpha ,t,s)$ определена формулой \eqref{EIO5}. Легко показать, что  $M\left(\frac{3}{2}\right)=\frac{4}{\sqrt{\pi }}.$  Докажем, что $M(\alpha )$ возрастает по $\alpha .$ Имеем
%
%$$M'\left(\alpha \right)=\Gamma'(\alpha )\int^1_0{\left[s\left(1-s\right)\right]}^{\alpha -1}ds-
%\Gamma\left(\alpha \right)\int^1_0{{\ln  \left[s\left(1-s\right)\right]\ }
%{\left[s\left(1-s\right)\right]}^{\alpha -1}ds}$$ $$=\int^1_0\left\{\Gamma'\left(\alpha \right){\left[s\left(1-s\right)\right]}^{\alpha -1}-\Gamma\left(\alpha \right){\ln  \left[s\left(1-s\right)\right]\ }
%{\left[s\left(1-s\right)\right]}^{\alpha -1}\right\}ds=$$ $$\int^1_0{\left[s\left(1-s\right)\right]}^{\alpha -1}\left(\Gamma'\left(\alpha \right)-\Gamma\left(\alpha \right){\ln  \left[s\left(1-s\right)\right]\ }\right)ds=$$ $$
%\int^1_0{{\left[s\left(1-s\right)\right]}^{\alpha -1}\Gamma\left(\alpha \right)[{\ln  \left(\alpha -1\right)\ }-{\ln  \left[s\left(1-s\right)\right]]\ }}ds.$$
%
%Так как $s\left(1-s\right)\le \frac{1}{4}$  при $s\in\left[0,1\right]$, а $\alpha -1\ge \frac{1}{2},$ то отсюда следует, что $M'\left(\alpha \right)>0.$
%
%Следовательно, функция $M\left(\alpha \right)$ возрастает, поэтому $M\left(\alpha \right)\ge \frac{4}{\sqrt{\pi }}$ при $\alpha \epsilon \left[\frac{3}{2},2\right].$
%
%При $\left(t,u\right)\in\left[0,1\right]\times[0,r_2]$ имеем
%$$f\left(t,u\right)=u^2+\frac{sint}{4}+1\le 2.25\le \frac{4}{\sqrt{\pi }}\cdot r_2,$$
%где $r_2=1,$ т.е. условие \eqref{EIO3} теоремы \ref{AbEI3.3:th1} выполняется.
%
%Теперь проверим условие \eqref{EIO4} теоремы \ref{AbEI3.3:th1}.
%
%Имеем \textit{}
%\begin{equation*}
%\frac{1}{N(\alpha )}=\int^{3/4}_{1/4}{\gamma \left(\alpha ,s\right)G\left(\alpha ,s,s\right)ds=\int^r_{1/4}{\frac{{[\frac{3}{4}\left(1-s\right)]}^{\alpha -1}-{(\frac{3}{4}-s)}^{\alpha -1}]{[s\left(1-s\right)]}^{\alpha -1}}{{\left[s\left(1-s\right)\right]}^{\alpha -1}\Gamma(\alpha )}ds+\int^{3/4}_r{\frac{1}{{(4s)}^{\alpha -1}}\frac{{[s\left(1-s\right)]}^{\alpha -1}}{\Gamma(\alpha )}ds.}}}
%\end{equation*}
%
%Так как $\frac{3}{4}\left(1-s\right)\ge \frac{3}{4}-s$ при $s\in\left[\frac{1}{4},\frac{3}{4}\right],$ то отсюда следует, что
%
%$$\frac{1}{N(\alpha )}\ge \frac{1}{4^{\alpha -1}\Gamma(\alpha )}\int^{3/4}_r{{(1-s)}^{\alpha -1}ds\ge -1}\Gamma(\alpha )4^{\alpha -1}=\frac{1}{4^{2\alpha -2}\Gamma(\alpha )}.$$
%Следовательно,
%
%$$N(\alpha )\le 4^{2\alpha -2}\Gamma(\alpha )\le {\mathop{\max }_{\frac{3}{2}\le \alpha \le 2} \left[4^{2\alpha -2}\Gamma\left(\alpha \right)\right]\equiv N_0.\ }$$
%Тогда при $\left(t,u\right)\in\left[0,1\right]\times\left[0,r_1\right],\ $ где $r_1=\frac{1}{N_0}\ $
%
%$$f\left(t,u\right)=u^2+\frac{sint}{4}+1\ge 1=N_0\frac{1}{N_0}=N_0r_1\ge N\left(\alpha \right)r_1.$$
%Следовательно, функция $f\left(t,u\right)=u^2+\frac{sint}{4}+1$ удовлетворяет условиям теоремы \ref{AbEI3.3:th1}. Лемма доказана.$\blacksquare $

Следовательно, по теореме \ref{AbEI3.3:th1} задача \eqref{EIO6}, \eqref{EIO7} имеет не менее одного положительного решения при $\frac{3}{2}\le \alpha \le 2$, причем $r_1\le \left\|u\right\|\le r_2$, где $\left\|u\right\|{=\mathop{\max }_{0\le t\le 1} u(t)\ }$.

%В дальнейшем докажем, что это решение единственно.

Положительное решение задачи \eqref{EIO6}, \eqref{EIO7} удовлетворяет уравнению
\begin{equation} \label{EIO8} u=\Phi \left(u\right),\end{equation}
где
\begin{equation}\label{EIO9}\Phi (u)=\int^1_0{G\left(t,s\right)[u^2\left(s\right)+\frac{\mathrm{\sin s}}{4}}+1]ds, \end{equation}

Нелинейное уравнение \eqref{EIO8} будем решать методом последовательных приближений:

$$ u_{k+1}\left(t\right)=\Phi (u_k),\text{ т.е.}$$

\begin{equation}
\label{EIO10}
u_{k+1}\left(t\right)=\int^1_0{G\left(t,s\right)\left[u^2_k\left(s\right)+\frac{\sin s}{4}+1\right]ds, k=0,1,2,\dots }
\end{equation}

Покажем, что $\Phi \left(u\right)$ является сжимающимся отображением на множестве

$$T=\left\{u\in C\left[0,1\right]:0\le u\left(t\right)\le 1\text{ при }t\in\left[0,1\right]\right\}.$$
Для этого достаточно показать, что $\left\|{\Phi }'(u)\right\|\le q<1.$
Легко видеть, что
$${\Phi }'\left(u\right)h=\int^1_0{2G\left(\alpha ,t,s\right)u\left(s\right)h(s)ds.}$$
Тогда при $u\in T$ имеем:
$$\left\|{\Phi }'(u)\right\|={\mathop{\max }_{0\le t\le 1} \left|\int^1_0{2G\left(\alpha ,t,s\right)u\left(s\right)ds}\right|\le 2\int^1_0{G\left(\alpha ,s,s\right)ds=\frac{2}{Г(\alpha )}\int^1_0{{[s\left(1-s\right)]}^{\alpha -1}ds\le \frac{2}{Г(\alpha )}\frac{1}{4^{\alpha -1}}.}}\ }$$
Легко проверить, что функция $Г\left(\alpha \right)4^{\alpha -1}$ при $\frac{3}{2}\le \alpha \le 2$ возрастает.
Поэтому

$$Г\left(\alpha \right)4^{\alpha -1}\ge Г\left(\frac{3}{2}\right)4^{\frac{3}{2}-1}=2\sqrt{\pi }.$$
Следовательно, $\left\|{\Phi }'(u)\right\|\le \frac{2}{2\sqrt{\pi }}=\frac{1}{\sqrt{\pi }}=q<1.$

Обозначим $G_q=\left\{u\in C\left[0,1\right],\ 0\le u\left(t\right)\le q\right\}.$ Покажем, что $\Phi (u)$ отображает множество $G_q$ в себя.  Имеем

$$0 \le \Phi \left(u\right)=\int^1_0 G\left(\alpha ,t,s\right)\left[u^2\left(s\right)+\frac{\sin s}{4}+1\right]ds\le \left(q^2+\frac{7}{32}+1\right)\int^1_0{G\left(\alpha ,s,s\right)ds}\le$$


$$\left(q^2+\frac{39}{32}\right)\frac{1}{Г\left(\alpha \right)4^{\alpha -1}}\le \left(q^2+\frac{39}{32}\right)\frac{q}{2}=\frac{(\frac{1}{\pi }+\frac{39}{32})}{2}q\le q.$$
Следовательно,$\Phi \left(u\right)$ отображает множество  $G_{q}$ в себя.
Так как $\Phi (u)$ является сжимающимся отображением на множестве T и отображает множество $G_q\subset T$ в себя, то в силу принципа сжимающихся отображений и  теоремы \ref{AbEI3.3:th1} справедлива

\begin{theorem}\label{AbEI3.3:th2} При $3/2\le \alpha \le 2$ задача \eqref{EIO6}, \eqref{EIO7} имеет единственное решение $u$$\frac{1}{N_0}\le \left\|u\right\|\le \frac{1}{\sqrt{\pi }}$, где  $N_0={\mathop{max}_{3/2\le \alpha \le 2} \left[4^{2\alpha -2}Г\left(\alpha \right)\right].\ }$
Кроме того, итерационный процесс \eqref{EIO10} сходится к этому решению и справедлива оценка погрешности:

$$\left\|u-u_k\right\|\le \frac{q^k}{1-q}{\mathop{max}_{0\le j\le 10} \left|u_1\left(t_j\right)-u_0(t_j)\right|\ }\ \text{, где } q=\frac{1}{\sqrt{\pi }}.$$
\end{theorem}
В качестве начального приближения в \eqref{EIO10} возьмем функцию $u_0\left(t\right)=\frac{t\left(1-t\right)}{2},$ которая удовлетворяет краевым условиям \eqref{EIO2}. Последующие приближения $u_{k+1}\left(t\right),\left(k=0,1,2,\dots \right)$ будем вычислять в точках $t_j=0.1j,$ $j=0,1,2,\dots 10.$ При этом интеграл в правой части \eqref{EIO10} будем вычислять по какой-либо квадратурной формуле, например, по квадратурной формуле средних прямоугольников:

\begin{equation}
\label{EIO11}
\int^b_a{f\left(x\right)dx \approx h\sum^{N-1}_{i=0}{f\left(x_i+\frac{h}{2}\right),\ где\ h=\frac{b-a}{N},\ x_i=a+ih,\ i=\overline{0,N}.}}
\end{equation}
$$\int^1_0{G\left(t_j,s\right)\left(y^2\left(s\right)+\frac{\sin s}{4}+1\right)ds \approx h\sum^{n-1}_{i=0}{G}\left(t_j,s_i\right) [y^2(s_i)+\frac{\sin s_i}{4}+1]},$$
где $s_i=0.1i+\frac{h}{2},\ \ \ \ i=\overline{0,n-1}$,$\ \ \ j=0,1,\dots 10$.

При вычислении интеграла в \eqref{EIO10} функцию $u_k\left(t\right)\ (k=0,1,2,\dots )$ заменим интерполяционным многочленом Лагранжа 10-й степени по ее значениям в узлах $t_j=0.1j$, $j=0,1,\dots 10.$

Так как $u_k\left(t\right)=\frac{t(1-t)}{2}$  при $k=0$, то интерполяционный многочлен $L_{10}(t)$ для этой функции равен $\frac{t(1-t)}{2},\ (0\le t\le 1)$ в силу единственности интерполяционного многочлена. Полагая  в \eqref{EIO10} $u_k\left(s\right)при\ k=0$ приближенно $L_{10}\left(s\right)=\frac{s(1-s)}{2}$ и, пользуясь квадратурной формулой \eqref{EIO11} с шагом h=0.01, можно вычислить (при k=0) значения правой части равенства \eqref{EIO8} при $t=t_j,\ \ j=\overline{0,10}$, т.е. можно найти значения первого приближения $u_1(t)$ при $\ t=t_j,\ \ j=\overline{0,10}$.  Далее с помощью полученных значений $u_1(t_j)$ можно аппроксимировать $u_1(t)$  интерполяционным многочленом 10-й степени. Затем при k=1 так же, как и в случае k=0, можно вычислить значения  $u_2\left(t_j\right),\ \ j=\overline{0,\ 10}$ и так далее, продолжаем вычислять $u_k\left(t_j\right)$до тех пор, пока не окажется  $\frac{q^k}{1-q}{\mathop{\max }_{0\le j\le 10} \left|u_1\left(t_j\right)-u_0\left(t_j\right)\right|\le \varepsilon ,\ \ }$ где $\varepsilon $ -- некоторое малое положительное число.

При $\varepsilon ={10}^{-4}$ нами получены следующие таблицы положительных решений задачи \eqref{EIO6}, \eqref{EIO7} для значений $\alpha \ от\ \frac{3}{2}\ до\ 2\ $ с шагом 0.1.
$$ $$
\begin{tabular}{|p{0.2in}|p{0.5in}|p{0.5in}|p{0.5in}|p{0.5in}|p{0.5in}|p{0.5in}|} \hline
\textbf{U(t)\newline \newline t=} & \textbf{при }${\mathbf \alpha }$\textbf{=1.5} & \textbf{при }${\mathbf \alpha }$\textbf{=1.6} & \textbf{при }${\mathbf \alpha }$\textbf{=1.7} & \textbf{при }${\mathbf \alpha }$\textbf{=1.8} & \textbf{при }${\mathbf \alpha }$\textbf{=1.9} & \textbf{    при }${\mathbf \alpha }$\textbf{=2.0      } \\ \hline
\textbf{0.0} & \textbf{0.0000} & \textbf{0.0000} & \textbf{0.0000} & \textbf{0.0000} & \textbf{0.0000} & \textbf{0.0000} \\ \hline
\textbf{0.1} & \textbf{0.2557} & \textbf{0.1890} & \textbf{0.1392} & \textbf{0.1020} & \textbf{0.0743} & \textbf{0.0539} \\ \hline
\textbf{0.2} & \textbf{0.3459} & \textbf{0.2753} & \textbf{0.2182} & \textbf{0.1720} & \textbf{0.1348} & \textbf{0.1052} \\ \hline
\textbf{0.3} & \textbf{0.4006} & \textbf{0.3333} & \textbf{0.2760} & \textbf{0.2275} & \textbf{0.1862} & \textbf{0.1519} \\ \hline
\textbf{0.4} & \textbf{0.4332} & \textbf{0.3720} & \textbf{0.3180} & \textbf{0.2705} & \textbf{0.2286} & \textbf{0.1924} \\ \hline
\textbf{0.5} & \textbf{0.4497} & \textbf{0.3958} & \textbf{0.3468} & \textbf{0.3024} & \textbf{0.2620} & \textbf{0.2660} \\ \hline
\textbf{0.6} & \textbf{0.4538} & \textbf{0.4077} & \textbf{0.3645} & \textbf{0.3243} & \textbf{0.2865} & \textbf{0.2521} \\ \hline
\textbf{0.7} & \textbf{0.4485} & \textbf{0.4100} & \textbf{0.3728} & \textbf{0.3373} & \textbf{0.3030} & \textbf{0.2710} \\ \hline
\textbf{0.8} & \textbf{0.4364} & \textbf{0.4050} & \textbf{0.3737} & \textbf{0.3429} & \textbf{0.3122} & \textbf{0.2831} \\ \hline
\textbf{0.9} & \textbf{0.4203} & \textbf{0.3953} & \textbf{0.3693} & \textbf{0.3428} & \textbf{0.3157} & \textbf{0.2893} \\ \hline
\textbf{1.0} & \textbf{0.0000} & \textbf{0.0000} & \textbf{0.0000} & \textbf{0.0000} & \textbf{0.0000} & \textbf{0.0000} \\ \hline
\end{tabular}
$$ $$

Приведем  соответствующие графики полученных решений.

\begin{minipage}{\linewidth}
    \centering
    \begin{minipage}{0.45\linewidth}
        \begin{figure}[H]
            \includegraphics[width=6cm]{elderhan2-graph1}
            \caption{При ${\mathbf \alpha }$ =1.5}
        \end{figure}
    \end{minipage}
    \hspace{0.05\linewidth}
    \begin{minipage}{0.45\linewidth}
        \begin{figure}[H]
            \includegraphics[width=6cm]{elderhan2-graph2}
            \caption{При ${\mathbf \alpha }$ =1.6}
        \end{figure}
    \end{minipage}
\end{minipage}

\begin{minipage}{\linewidth}
    \centering
    \begin{minipage}{0.45\linewidth}
        \begin{figure}[H]
            \includegraphics[width=6cm]{elderhan2-graph3}
            \caption{При ${\mathbf \alpha }$ =1.7}
        \end{figure}
    \end{minipage}
    \hspace{0.05\linewidth}
    \begin{minipage}{0.45\linewidth}
        \begin{figure}[H]
            \includegraphics[width=6cm]{elderhan2-graph4}
            \caption{При ${\mathbf \alpha }$ =1.8}
        \end{figure}
    \end{minipage}
\end{minipage}

\begin{minipage}{\linewidth}
    \centering
    \begin{minipage}{0.45\linewidth}
        \begin{figure}[H]
            \includegraphics[width=6cm]{elderhan2-graph5}
            \caption{При ${\mathbf \alpha }$ =1.9}
        \end{figure}
    \end{minipage}
    \hspace{0.05\linewidth}
    \begin{minipage}{0.45\linewidth}
        \begin{figure}[H]
            \includegraphics[width=6cm]{elderhan2-graph6}
            \caption{При ${\mathbf \alpha }$ =2.0}
        \end{figure}
    \end{minipage}
\end{minipage}

%При ${\mathbf \alpha }$ =1.5 При ${\mathbf \alpha }$ =1.6


%При ${\mathbf \alpha }$ =1.7 При ${\mathbf \alpha }$ =1.8


%При ${\mathbf \alpha }$ =1.9 При ${\mathbf \alpha }$ =2.0



Приведенные результаты показывают, что полученные решения удовлетворяют условиям теоремы \ref{AbEI3.3:th2}: $\left\|u\right\|\le \frac{1}{\sqrt{\pi }}$ и убывают по $\alpha $.
