
\subsection{Метод асимптотических разложений для системы уравнений Бельтрами}

\textit{
    В работе рассматривается асимптотические методы усреднения уравнения Бельтрами. Дается скорость сходимости к нулю разности точного решения и первого приближения.
}


Пусть $u_{\varepsilon }$ - решение задачи Римана-Гильберта
\begin{equation}
\label{sirM3.1_} \left\{\begin{array}{l} {A_{\varepsilon } u_{\varepsilon } \equiv \partial _{\bar{z}} u_{\varepsilon } +\mu (\varepsilon ^{-1} x)\partial _{z} u_{\varepsilon } =f,} \\ {u_{\varepsilon } \in W_{0} (Q),} \end{array}\right.
\end{equation}

где  $\mu (x)$ - периодическая функция,  $W_{0} (Q)=\left\{u\in W_{p}^{1} |\; Reu|_{\partial Q} =0\right\}$,
\begin{equation*}
\partial _{\bar{z}} =2^{-1} \left({\partial  \mathord{\left/{\vphantom{\partial  \partial x_{1} }}\right.\kern-\nulldelimiterspace} \partial x_{1} } +i{\partial  \mathord{\left/{\vphantom{\partial  \partial x_{2} }}\right.\kern-\nulldelimiterspace} \partial x_{2} } \right),  \partial _{\bar{z}} =2^{-1} \left({\partial  \mathord{\left/{\vphantom{\partial  \partial x_{1} }}\right.\kern-\nulldelimiterspace} \partial x_{1} } -i{\partial  \mathord{\left/{\vphantom{\partial  \partial x_{2} }}\right.\kern-\nulldelimiterspace} \partial x_{2} } \right).
\end{equation*}
$Q$- ограниченная односвязная область плоскости с кусочно-гладкой границей.

Первое приближение к решению $u_{\varepsilon } $ ищем в виде
\begin{equation}
u_{1}^{3} (x)=u^{0} (x)+\varepsilon u_{1} (x,y),\label{sirM3.2_}
\end{equation}
где $u_{1}(x,y)$ - периодическая функция по $y$.

Найдем результаты действия дифференциальных выражений $\partial _{\bar{z}} $ и $\partial _{z} $ на сложную функцию $\varphi (x,y)$, $y=\varepsilon ^{-1} x$. Полные производные по $z$ и $\bar{z}$ обозначим $\frac{d}{dz} $, $\frac{d}{d\bar{z}} $ соответственно. Имеем
\begin{equation*}
\frac{d\varphi }{dz} =\partial _{z} \varphi +\varepsilon ^{-1} \partial _{\xi } \varphi ,  \frac{d\varphi }{d\bar{z}} =\partial _{\bar{z}} \varphi +\varepsilon ^{-1} \partial _{\bar{\xi }} \varphi ,  \xi =y_{1} +iy_{2} .
\end{equation*}

Следовательно,
\begin{equation*}
A_{\varepsilon } =\partial _{\bar{z}} +\mu (y)\partial _{z} +\varepsilon ^{-1} (\partial _{\bar{\xi }} +\mu (y)\partial _{\varepsilon } )\equiv A_{1} +\varepsilon ^{-1} A_{2} ,
\end{equation*}
где $A_{1} =\partial _{\bar{z}} +\mu (y)\partial _{z} $, $A_{2} =\partial _{\bar{\xi }} +\mu (y)\partial _{\xi } $.
Значит,
\begin{equation*}
A_{\varepsilon } u_{1}^{\varepsilon } =(A_{1} +\varepsilon ^{-1} A_{2} )(u^{0} +\varepsilon u_{1} )=A_{1} u^{0} +A_{2} u_{1} +\varepsilon A_{1} u_{1} +\varepsilon ^{-1} A_{2} u^{0}.
\end{equation*}
Так как $A_{2} u^{0} =\partial _{\bar{\xi }} u^{0} (x)+\mu (y)\partial _{\xi } u^{0} (x)=0,$ имеем
\begin{equation}
\label{sirM3.3_} A_{\varepsilon } (u_{1}^{\varepsilon } -u_{\varepsilon } )=A_{1} u^{0} +A_{2} u_{1} +\varepsilon A_{1} u_{1} -f. \end{equation}
Приравняем к нулю коэффициент при $\varepsilon ^{0} $, получим
\begin{equation*}
0=A_{1} u^{0} +A_{2} u_{1} -f=\partial _{\bar{z}} u^{0} (x)+\mu (y)\partial _{z} u^{0} (x)+\partial _{\bar{\xi }} u_{1} (x,y)+\mu (y)\partial _{\xi } u_{1} (x,y)-f,
\end{equation*}
т.е.
\begin{equation}
\partial _{\bar{\xi }} u_{1} (x,y)+\mu (y)\partial _{\xi } u_{1} (x,y)=-\partial _{\bar{z}} u^{0} (x)+\mu (y)\partial _{z} u^{0} (x)+f(x). \label{sirM3.4_}
\end{equation}

Задача \eqref{sirM3.4_} есть периодическая задача по $y$,  $x$ - параметр.
Для разрешимости \eqref{sirM3.4_} необходимо и достаточно, чтобы правая часть была ортогональна ядру сопряженного однородного уравнения. Значит,
\begin{equation}
\left\langle \left(\partial _{\bar{z}} u^{0} (x)+\mu (y)\partial _{z} u^{0} (x)-f(x)\right)\overline{p(y)}\right\rangle =0, \label{sirM3.5_}
\end{equation}
где  $\left\langle g\right\rangle =T^{-2} \iint \nolimits _{\Omega }g\, dy_{1}  dy_{2} $ - среднее значение периодической функции $g$, $\Omega =\left[0,T\right]\times \left[0,T\right]$, среднее значение в \eqref{sirM3.5_} берется по $y$.

Из \eqref{sirM3.5_}, с учетом $\left\langle p\right\rangle =1$ (см. \cite{Sirazh3}), получим
\begin{equation}
\partial _{\bar{z}} u^{0} (x)+\left\langle \mu \bar{p}\right\rangle \partial _{z} u^{0} (x)=f(x). \label{sirM3.6_}
\end{equation}

Получили усредненное уравнение (см. \cite{Sirazh3}). К нему добавим условие $u^{0} \in W_{0} (Q;C)$. С учетом \eqref{sirM3.6_} из \eqref{sirM3.4_} получим
\begin{equation}
\partial _{\bar{\xi }} u_{1} (x,y)+\mu (y)\partial _{\xi } u_{1} (x,y)=\left(\left\langle \mu \bar{p}\right\rangle -\mu (y)\right)\partial _{z} u^{0} (x).
\label{sirM3.7_}
\end{equation}

Пусть  $N(y)$ - решение периодической задачи
\begin{equation*}
\partial _{\bar{\xi }} N+\mu \partial _{\xi } N=\left\langle \mu \bar{p}\right\rangle -\mu ,
\end{equation*}
тогда решением \eqref{sirM3.7_} будет
\begin{equation*}
u_{1} (x,y)=N(y)\partial _{z} u^{0} (x).
\end{equation*}

Следовательно, первое приближение \eqref{sirM3.2_} имеет вид
\begin{equation*}
u_{\varepsilon }^{1} (x)=u^{0} (x)+\varepsilon \, N(y)\partial _{z} u^{0} (x).
\end{equation*}

Соотношение \eqref{sirM3.3_} примет вид
\begin{equation}
A_{\varepsilon } (u_{\varepsilon }^{1} -u_{\varepsilon } )=\varepsilon A_{1} u_{1} =\varepsilon N(y)\left(\partial _{z\bar{z}}^{2} u^{0} (x)+\mu (y)\partial _{zz}^{2} u^{0} (x)\right).
\label{sirM3.8_}
\end{equation}

Поступая аналогично  \cite{JikovKozlov}, получим следующую оценку разности  $u_{1}^{\varepsilon } -u^{\varepsilon } $:
\begin{equation*}
\left\| u_{1}^{\varepsilon } -u^{\varepsilon } \right\| _{W_{p}^{1} } \le c\sqrt[{P}]{\varepsilon },
\end{equation*}
где $p>2$ - постоянная, зависящая только от $k_{0}$.

Отсюда,  в силу теоремы вложения Соболева, получим справедливость следующей теоремы:
\textbf{Теорема.  }\textit{Разность}\textit{$u_{1}^{\varepsilon } -u^{\varepsilon } $ допускает следующую оценку}
\begin{equation*}
\left\| u_{1}^{\varepsilon } -u^{\varepsilon } \right\| _{H^{\alpha } } \le c{\kern 1pt} \, \sqrt[{p}]{\varepsilon },
\end{equation*}
\textit{где  $c>0$ - постоянная, зависящая только от $k_{0} $, $f$; $H^{\alpha } $ - пространство Гельдера непрерывных функций,  $\alpha ={(p-2) \mathord{\left/{\vphantom{(p-2) p}}\right.\kern-\nulldelimiterspace} p} $.}

