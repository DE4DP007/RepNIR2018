Многие задачи математической физики приводят к изучению вопросов G--сходимости дифференциальных операторов. Такие вопросы возникают в теории упругости, электродинамике и других разделах физики и механики.

Вопросам G--сходимости  дифференциальных операторов посвящено много работ (см. монографию В. В. Жикова и др. \cite{JikovKozlov} и имеющуюся там литературу). Теория  G--сходимости  дивергентных эллиптических операторов второго порядка в общих чертах завершена.

G--сходимость дифференциальных операторов -- это, иначе, слабая сходимость соответствующих обратных операторов. Поэтому по понятным причинам в задачах G--сходимости кроме корректной разрешимости краевых задач требуются также оценки решений, равномерные относительно любого оператора. Для недивергентных эллиптических операторов и систем (к которым относятся и рассматриваемые в работе операторы) такого рода оценки мало изучены, поэтому G--сходимость недивергентных операторов также изучена не столь детально, как для дивергентных операторов.
Вопросам G--сходимости и усреднению недивергентных эллиптических операторов посвящены работы \cite{JikovSirazh1, JikovSirazh2, Sirazh1, Sirazh2, Sirazh3}.

Проведено исследование $G$--сходимости одного класса эллиптических операторов второго порядка с комплекснозначными коэффициентами. Доказана $G$--компактность этого класса. Рассмотрены вопросы усреднения для операторов с периодическими коэффициентами. Разработаны асимптотические методы усреднения уравнения Бельтрами. Изучена скорость сходимости к нулю разности точного решения и первого приближения.

%%%%%%%%%%%%%%%%%%%%%%%%%%%%%%%%%%%%%%%%%%%%%%%%%%%%%%%%%%%%%%%%%%%%%%%%%%%%%%%%%%

В последние годы возрос интерес к исследованию дифференциальных уравнений дробного порядка, в которых неизвестная функция содержится под знаком производной дробного порядка. Это обусловлено как развитием самой теории дробного интегрирования и дифференцирования, так и приложениями этой теории к различным областям науки.

Физики достаточно давно и плодотворно используют идеи дробного исчисления преимущественно во фрактальных средах. Дифференциальные уравнения дробного порядка встречаются при описании медленных и быстрых стохастических процессов, диффузии в средах с фрактальной геометрией, при изучении деформационно-прочностных свойств полимерных материалов. Полученные при этом результаты говорят о существовании мощного метода, каким является дробное исчисление при построении математических моделей в тех средах, где классическое дифференциальное исчисление не работает. Особый интерес к дробным производным проявляют гидрогеологи в связи с вопросами безопасности, хранения высокоактивных долгоживущих радиоизотопов в геологических формациях.

В последние годы возросла интенсивность изучения дифференциальные уравнения с дробными производными (см., например, \cite{bailu,wangwang,zhangS,qiubai,caballero,changnieto,shangSQ,beyb,beybShab,aleroev}). В частности, имеются публикации, посвященные существованию положительных решений краевых задач для нелинейных дифференциальных уравнений второго порядка с дробными производными (см., например, \cite{bailu,wangwang}, \cite{qiubai,caballero,changnieto}, \cite{beyb}, \cite{beybDavud}). При этом публикаций на тему построения численными методами положительных решений краевых задач для нелинейных дифференциальных уравнений сравнительно мало, хотя такие задачи возникают на практике.

Нами предложен численный метод построения положительное решение двухточечной краевой задачи для одного нелинейного дифференциального уравнения с дробными производными. Этот же метод позволяет доказать существование и единственность.
