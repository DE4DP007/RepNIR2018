

%Пусть $-1=\eta_{0}<\eta_{1}<\eta_{2}<\dots<\eta_{N-1}<\eta_{N}=1$, $\lambda_N = \max_{0\leq j \leq N-1} (\eta_{j+1}-\eta_{j})$.
%Работа посвящена исследованию свойств полиномов, образующих ортонормированную систему с весом Якоби $\kappa^{\alpha,\beta}(t) = (1-t)^\alpha (1+t)^\beta$ на произвольной (не обязательно равномерной) сетке $\Omega_N = \{ t_j \}_{j=0}^{N-1}$, такой что $\eta_{j}\leq t_{j} \leq \eta_{j+1}$.

Проведено исследование свойств полиномов, образующих ортонормированную систему с весом Якоби $\kappa^{\alpha,\beta}(t) = (1-t)^\alpha (1+t)^\beta$ на произвольной (не обязательно равномерной) сетке $\Omega_N = \{ t_j \}_{j=0}^{N-1}$, такой что $\eta_{j}\leq t_{j} \leq \eta_{j+1}$.

В случае целых $\alpha,\beta \geq 0$ для построенных таким образом дискретных ортонормированных полиномов $\hat{P}_{n,N}^{\alpha,\beta}(t)\quad (n=0,\ldots, N-1)$ при $n=O(\lambda_N^{-\frac13}) \quad (\lambda_N \rightarrow 0)$ получена асимптотическая формула вида $\hat{P}_{n,N}^{\alpha,\beta}(t) = \hat{P}_{n}^{\alpha,\beta}(t) + \upsilon_{n,N}^{\alpha,\beta}(t)$, в которой $\hat{P}_{n}^{\alpha,\beta}(t)$ -- классический полином Якоби, $\upsilon_{n,N}^{\alpha,\beta}(t)$ -- остаточный член. В качестве следствия асимптотической формулы получена весовая оценка полиномов $\hat{P}_{n,N}^{\alpha,\beta}(t)$ на отрезке $[-1, 1]$.

Для частного случая с единичным весом  $\kappa^{0,0}(t) = 1$ построены суммы Фурье по полиномам $\hat{P}_{n,N}(t) = \hat{P}_{n,N}^{0,0}(t)$, изучены их аппроксимативные свойства, получены оценки функции Лебега.


\subsection{Предварительные сведения}


Пусть
\begin{equation}
\label{SultM_eq1}
-1=\eta_{0}<\eta_{1}<\eta_{2}<\dots<\eta_{N-1}<\eta_{N}=1,
\end{equation}
\begin{equation*}
\vartriangle\eta_{j} = \eta_{j+1}-\eta_{j}\text{ } (0 \leq j \leq N-1), \quad \lambda_N = \max_{0\leq j \leq N-1} \vartriangle\eta_j.
\end{equation*}
Рассмотрим сетку $\Omega_N = \{ t_j \}_{j=0}^{N-1}$, в которой узлы $t_j$ удовлетворяют условию
\begin{equation}
\label{SultM_eq2}
\eta_{j}\leq t_{j} \leq \eta_{j+1} \quad (0\leq j \leq N-1),
\end{equation}
причем $t_i \neq t_j,$ если $i \neq j$. Для $\alpha,\beta \geq 0$ положим $\kappa^{\alpha,\beta}(t) = (1-t)^\alpha (1+t)^\beta,$ $\rho = \rho(t_j)=\kappa^{\alpha,\beta}(t_j)\vartriangle\eta_{j}$.
Рассмотрим пространство $l_{2,\rho}(\Omega_N)$ дискретных функций вида $f:\Omega_N \rightarrow R$, в котором скалярное произведение задано следующим образом:
\begin{equation}
\label{SultM_eq3}
<f,g> = \sum_{j=0}^{N-1} f(t_j)g(t_j)\rho(t_j) = \sum_{j=0}^{N-1} f(t_j)g(t_j) \kappa^{\alpha,\beta}(t_j)\vartriangle\eta_{j}.
\end{equation}
Через $\hat{P}_{n,N}^{\alpha,\beta}(t) \quad (0 \leq n \leq N-1)$ обозначим полиномы, образующие конечную ортонормированную систему относительно скалярного произведения \eqref{SultM_eq3}, т.е.
\begin{equation*}
<\hat{P}_{n,N}^{\alpha,\beta},\hat{P}_{m,N}^{\alpha,\beta}> = \sum_{j=0}^{N-1} \hat{P}_{n,N}^{\alpha,\beta}(t_j)\hat{P}_{m,N}^{\alpha,\beta}(t_j) \kappa^{\alpha,\beta}(t_j)\vartriangle\eta_{j} = \delta_{nm} =
\left\{
\begin{aligned}
0, \quad n \neq m,\\
1, \quad n = m.\\
\end{aligned}
\right.
\end{equation*}

Будем называть полиномы $\hat{P}_{n,N}^{\alpha,\beta}(t) \quad (0 \leq n \leq N-1)$ \textit{дискретными ортонормированными полиномами Якоби}. Нашей первой задачей является изучение асимптотических свойств $\hat{P}_{n,N}^{\alpha,\beta}(t)$ при $\lambda_N \rightarrow 0$, $n \rightarrow \infty$ в случае целых $\alpha,\beta$. При $n = O(\lambda_N^{-\frac{1}{3}})$ нами получена формула вида
\begin{equation*}
\hat{P}_{n,N}^{\alpha,\beta}(t) = \hat{P}_{n}^{\alpha,\beta}(t) + \upsilon_{n,N}^{\alpha,\beta}(t),
\end{equation*}
где $\hat{P}_{n}^{\alpha,\beta}(t)$ -- ортонормированный классический полином Якоби, $\upsilon_{n,N}^{\alpha,\beta}(t)$ -- остаточный член,  для которого установлена следующая оценка
\begin{equation*}
\left|\upsilon_{n,N}^{\alpha,\beta}(\cos{\theta})\right| \leq
c(\alpha,\beta,a) \left( \frac{3-\lambda_N \chi (2n+\alpha+\beta)^2}{1-\lambda_N^2 \chi^2 (2n+\alpha+\beta)^4} \right)^{\frac{1}{2}}
\left\{
\begin{aligned}
\theta^{-\alpha-\frac{1}{2}}n^{\frac{3}{2}}\sqrt{\lambda_N},\quad an^{-1}\leq\theta\leq \frac{\pi}{2},\\
n^{\alpha+2}\sqrt{\lambda_N},\quad 0\leq\theta\leq an^{-1},\\
\end{aligned}
\right.
\end{equation*}
где здесь и далее $c, c(\alpha), c(\alpha,\beta), c(\alpha,\beta,\ldots,\gamma)$ -- положительные числа, зависящие лишь от указанных параметров, и различные в разных местах,  $\chi$ -- наименьшая константа в интегральном неравенстве Маркова  об оценке интеграла от производной алгебраического полинома (см. \eqref{SultM_eq6}).  В качестве следствия асимптотической формулы, получены весовые оценки для полиномов $\hat{P}_{n,N}^{\alpha,\beta}(t)$.

Асимптотические свойства и весовые оценки полиномов, ортогональных на дискретных сетках, впервые были исследованы в работах Шарапудинова И.И. (см. \cite{idprmgreenBook} и цитированную там литературу). Им
был внесен существенный вклад в асимптотическую теорию дискретных ортогональных полиномов, в частности классических полиномов Чебышева, Мейкснера, Кравчука, ортогональных на равномерных сетках. В дальнейшем в работах Шарапудинова И.И. \cite{nushii2,nushii3,nushii4} и Нурмагомедова А.А. \cite{nunurik1, nunurik2} были проведены исследования асимптотических свойств полиномов, ортогональных на неравномерных сетках числовой оси. В частности, в работе \cite{nunurik2} рассмотрены асимптотические свойства дискретных полиномов Якоби $\hat{P}_{n,N}^{\alpha,\beta}(t)$ с целыми $\alpha,\beta$, ортогональных на неравномерных дискретных сетках $\Omega_N$ c
$t_{j} = \frac{\eta_{j}+\eta_{j+1}}{2} \quad (0\leq j \leq N-1)$.
Как уже отмечалось выше, в настоящей работе рассматривается более общий случай, когда $\eta_{j}\leq t_{j} \leq \eta_{j+1} \quad (0\leq j \leq N-1)$.
Другие аспекты асимптотической теории дискретных ортогональных полиномов отражены в \cite{mcbaik,ouwong,lopezsinus} и цитированной там литературе.

В дальнейшем нам понадобятся некоторые свойства классических полиномов Якоби, которые мы соберем в следующем параграфе.

\subsection{Некоторые сведения о полиномах Якоби и Лежандра}

Определим многочлены Якоби $P_{n}^{\alpha,\beta}(t) \quad (n=0,1,2,\dots)$ с помощью формулы Родрига
\begin{equation*}
P_{n}^{\alpha,\beta}(t) = \frac{(-1)^n}{2^{n}n!}\frac{1}{\kappa^{\alpha,\beta}(t)}\frac{d^n}{dt^{n}}\{ \kappa^{\alpha,\beta}(t))\sigma^n(t) \},
\end{equation*}
где $\alpha,\beta$ -- произвольные действительные числа, $\kappa^{\alpha,\beta}(t)=(1-t)^{\alpha} (1+t)^{\beta},$ $\sigma(t)=1-t^2.$

Если $\alpha,\beta > -1,$ то многочлены Якоби образуют ортогональную систему с весом $\kappa^{\alpha,\beta}(t),$ т.е.
\begin{equation*}
\int\limits_{-1}^{1} P_{n}^{\alpha,\beta}(t)P_{m}^{\alpha,\beta}(t)\kappa^{\alpha,\beta}(t)dt = h_n^{\alpha,\beta}\delta_{nm},
\end{equation*}
где
\begin{equation*}
h_n^{\alpha,\beta} = \frac{2^{\alpha+\beta+1}\Gamma(n+\alpha+1)\Gamma(n+\beta+1)}{n! (2n+\alpha+\beta+1)\Gamma(n+\alpha+\beta+1)},
\end{equation*}
и, следовательно, $h_n^{\alpha,\beta} \asymp n^{-1} \quad (n=1,2,\dots)$. Нам понадобится следующая весовая оценка $(-1 \leq t \leq 1)$ \cite{Sego}:
\begin{equation}
\label{SultM_eq4}
\sqrt{n}\left|P_{n}^{\alpha,\beta}(t)\right| \leq c(\alpha, \beta) \left( \sqrt{1-t} + \frac{1}{n} \right)^{-\alpha-\frac{1}{2}} \left( \sqrt{1+t} + \frac{1}{n} \right)^{-\beta-\frac{1}{2}},
\end{equation}
из которой, в частности, следует что
\begin{equation*}
\sqrt{n}\left|P_{n}^{\alpha,\beta}(t)\right| \leq c(\alpha, \beta) (1-t)^{-\frac{\alpha}{2}-\frac{1}{4}}, \quad (0 \leq t \leq 1 - n^{-2}),
\end{equation*}
\begin{equation*}
\sqrt{n}\left|P_{n}^{\alpha,\beta}(t)\right| \leq c(\alpha, \beta) n^{\alpha+\frac{1}{2}}, \quad (1 - n^{-2} \leq t \leq 1),
\end{equation*}
\begin{equation*}
\sqrt{n}\left|P_{n}^{\alpha,\beta}(t)\right| \leq c(\alpha, \beta) (1+t)^{-\frac{\beta}{2}-\frac{1}{4}}, \quad (- 1 + n^{-2} \leq t \leq 0),
\end{equation*}
\begin{equation*}
\sqrt{n}\left|P_{n}^{\alpha,\beta}(t)\right| \leq c(\alpha, \beta) n^{\beta+\frac{1}{2}}, \quad (-1 \leq t \leq - 1 + n^{-2}).
\end{equation*}


Соответствующие ортонормированные полиномы Якоби обозначим через $\hat{P}_{n}^{\alpha,\beta}(t)$, т.е.
$\hat{P}_{n}^{\alpha,\beta}(t) = (h_n^{\alpha,\beta})^{-\frac12} P_{n}^{\alpha,\beta}(t) \quad (n=0,1,2,\dots)$.

Важным частным случаем полиномов Якоби при $\alpha = \beta = 0$ являются полиномы Лежандра $P_{n}(t)$, ортогональные на отрезке $[-1, 1]$ с весом $\rho(t)\equiv 1$.
Обозначим $\hat{P}_{n}(t)= \sqrt{\frac{2n+1}{2}} P_{n}(t)$ $(n = 0, 1,2, \ldots)$ соответствующие ортонормированные полиномы Лежандра. Старший коэффициент полинома $\hat{P}_{n}(t)$ может быть записан следующим образом
\begin{equation}
\label{maincoeff}
k_n = \frac{(2n)!}{(n!)^2 2^n} \sqrt{\frac{2n+1}{2}}.
\end{equation}

\subsection{Вспомогательные результаты}

Ниже нам понадобятся следующие вспомогательные утверждения.

\begin{lemma}
\label{SMS1:lemm1}
    Пусть $\{ \eta_j \}_{j=0}^N$, $\{ t_j \}_{j=0}^{N-1}$ -- системы узлов, удовлетворяющие условиям \eqref{SultM_eq1} и \eqref{SultM_eq2} соответственно, $f(x)$ -- абсолютно непрерывная функция, заданная на $[ -1, 1 ]$.
    Тогда имеет место следующий аналог формулы Эйлера-Маклорена
    \begin{equation*}
    \int\limits_{-1}^{1} f(t)dt = \sum_{j=0}^{N-1} f(t_j)\vartriangle\eta_j + r_N(f),
    \end{equation*}
    где для остаточного члена справедлива оценка
    \begin{equation*}
    |r_N(f)| \leq \lambda_N \int\limits_{-1}^{1} |f'(t)|dt.
    \end{equation*}
\end{lemma}


\begin{corollary}
\label{SMS2:cor1}
Для абсолютно непрерывной на $[ -1, 1 ]$ монотонной неотрицательной функции $f(x)$ справедлива оценка
\begin{equation*}
\sum_{j=0}^{N-1} f(t_j)\vartriangle\eta_j \leq \int\limits_{-1}^{1} f(t)dt + \lambda |f(1) - f(-1)|.
\end{equation*}
\end{corollary}


\begin{remark}
\label{SMS2:remrk1} Лемма \ref{SMS1:lemm1} и ее следствие \ref{SMS2:cor1} могут быть легко распространены с отрезка $[-1,1]$ на произвольный отрезок $[a,b]$.
\end{remark}


Далее нам понадобится обобщение на интегральную метрику неравенства Маркова для оценки производной алгебраического многочлена (см. \cite{baritri, konyagin}), имеющее для $r=1$ следующий вид:
\begin{equation}
\label{SultM_eq6}
\int\limits_{-1}^{1}|q_m'(t)|dt \leq c(m)m^2\int\limits_{-1}^{1}|q_m(t)|dt,
\end{equation}
где $q_m(t)$ -- произвольный алгебраический полином степени $m$. Для каждого $m$ наименьшую из констант $c(m)$, удовлетворяющих неравенству \eqref{SultM_eq6}, обозначим через $\chi_m$, т.е.
\begin{equation*}
\chi_m = \sup_{q_m} \frac{\int\limits_{-1}^{1}|q_m'(t)|dt}{m^2\int\limits_{-1}^{1}|q_m(t)|dt},
\end{equation*}
где верхняя грань берется по всем полиномам $q_m(t)$ степени не выше $m$ и не равных нулю тождественно.
В работе Н.К. Бари \cite{baritri} было показано, что $\chi = \sup\limits_{m \geq 1} \chi_m <\infty$. С учетом этого факта из неравенства \eqref{SultM_eq6} мы выводим
\begin{equation}
\label{SultM_eq7}
\int\limits_{-1}^{1}|q_m'(t)|dt \leq \chi m^2\int\limits_{-1}^{1}|q_m(t)|dt.
\end{equation}

\begin{lemma}
\label{SMS1:lemm2}
    Пусть $\lambda_N\chi (2n+\alpha+\beta)^2 < 1$. Тогда для дискретных полиномов Якоби в случае целых  $\alpha,\beta \geq 0$ имеет место равенство
    \begin{equation}
    \label{SultM_eq8}
    \int\limits_{-1}^{1} \left( \hat{P}_{n, N}^{\alpha,\beta}(t)\right)^2 \kappa^{\alpha,\beta}(t) dt = 1 + R_{n,N},
    \end{equation}
    в котором для остаточного члена справедлива оценка
    \begin{equation*}
    |R_{n,N}| \leq \frac{\lambda_N\chi (2n+\alpha+\beta)^2}{1 - \lambda_N\chi (2n+\alpha+\beta)^2}.
    \end{equation*}
\end{lemma}

\begin{lemma}
\label{SMS1:lemm3}
    При целых  $\alpha,\beta \geq 0$  и $\lambda_N\chi (2n+\alpha+\beta)^2 <1$ имеет место следующее двойное неравенство
    \begin{equation*}
    1 - \frac{\lambda_N\chi(2n+\alpha+\beta)^2}{1+\lambda_N\chi(2n+\alpha+\beta)^2} \leq \frac{k_{n,N}^{\alpha,\beta}}{k_{n}^{\alpha,\beta}}  \leq \frac{1}{\left(1-\lambda_N\chi (2n+\alpha+\beta)^2\right)^{\frac12}},
    \end{equation*}
    где $k_{n}^{\alpha,\beta}, k_{n,N}^{\alpha,\beta}$ -- старшие коэффициенты полиномов $\hat{P}_{n}^{\alpha,\beta}$ и $\hat{P}_{n,N}^{\alpha,\beta}$ соответственно.
\end{lemma}


\noindent Ниже нам понадобится следующее

\begin{statement}
\label{SMS1:state1}
    (см., например, в \cite{idprmgreenBook}, $\S7.71$, Теорема 7.71.2) Пусть $\alpha,\beta>-1,$ $M_n(x)$ -- произвольный полином степени $n$, подчиненный условию:
    \begin{equation*}
    \int\limits_{-1}^{1} \kappa^{\alpha,\beta}(x) |M_n(x)|^2 dx = 1.
    \end{equation*}
    Тогда имеет место следующая оценка
    \begin{equation}
    \label{SultM_eq15}
    \left|M_n(\cos(\theta))\right| \leq c(\alpha,\beta,a)
    \left\{
    \begin{aligned}
    n^{\alpha+1},\quad 0\leq\theta\leq an^{-1}.\\
    \theta^{-\alpha-\frac{1}{2}}n^{\frac{1}{2}},\quad an^{-1}\leq\theta\leq \frac{\pi}{2},\\
    \end{aligned}
    \right.
    \end{equation}
    Аналогичные оценки справедливы на отрезке   $\theta \in \left[ \frac{\pi}{2}, \pi\right]$.
\end{statement}


\subsection{Асимптотическая формула для полиномов $\hat{P}_{n,N}^{\alpha,\beta}(t)$}

Далее для краткости все рассуждения мы будем проводить на отрезке $\theta \in \left[ 0,\frac{\pi}{2}\right]$, для отрезка $\theta \in \left[ \frac{\pi}{2}, \pi\right]$ они получаются абсолютно аналогично.

\begin{theorem}
\label{SMS1:th1}
    Пусть $0 \leq \alpha,\beta$ --- целые, $\lambda_N \chi(2n+\alpha+\beta)^2 < 1$. Тогда имеет место равенство:
    \begin{equation*}
    \hat{P}_{n,N}^{\alpha,\beta}(t) = \hat{P}_{n}^{\alpha,\beta}(t) + \upsilon_{n,N}^{\alpha,\beta}(t),
    \end{equation*}
    где для остаточного члена $\upsilon_{n,N}^{\alpha,\beta}$ имеет место оценка $($здесь $t=\cos\theta)$
    \begin{equation*}
    \left|\upsilon_{n,N}^{\alpha,\beta}(\cos{\theta})\right| \leq
    c(\alpha,\beta, a) \left( \frac{3-\lambda_N \chi (2n+\alpha+\beta)^2}{1-\lambda_N^2 \chi^2 (2n+\alpha+\beta)^4} \right)^{\frac{1}{2}}
    \left\{
    \begin{aligned}
    n^{\alpha+2}\sqrt{\lambda_N},\quad 0\leq\theta\leq an^{-1}\\
    \theta^{-\alpha-\frac{1}{2}}n^{\frac{3}{2}}\sqrt{\lambda_N},\quad an^{-1}\leq\theta\leq \frac{\pi}{2}\\
    \end{aligned}
    \right.
    \end{equation*}
\end{theorem}
\noindent Обозначим для краткости
\begin{equation*}
B = \left( \frac{3-\lambda_N \chi (2n+\alpha+\beta)^2}{1-\lambda_N^2 \chi^2 (2n+\alpha+\beta)^4} \right)^{\frac{1}{2}}.
\end{equation*}

\noindent Следствием теоремы \ref{SMS1:th1} является следующее утверждение.

\begin{theorem}
\label{SMS1:th2}
    Пусть $0 \leq \alpha,\beta$ --- целые и $\lambda_N \chi(2n+\alpha+\beta)^2 < 1$, тогда существует постоянная $c(\alpha,\beta,a)$, такая что:
    \begin{equation*}
    \left|\hat{P}_{n,N}^{\alpha,\beta}(\cos{\theta})\right| \leq
    c(\alpha,\beta,a) \left(1 + B\sqrt{n^{3}\lambda_N}\right)
    \left\{
    \begin{aligned}
    n^{\alpha+\frac12},& \quad 0\leq \theta \leq an^{-1},\\
    \theta^{-\alpha-\frac12},& \quad an^{-1}\leq \theta \leq \frac{\pi}{2}.\\
    \end{aligned}
    \right.
    \end{equation*}
\end{theorem}



