\section{Базисность системы Хаара в весовых пространствах Лебега с переменным показателем}\label{section-haar-basis}

\subsection{Введение}
Пусть $p(x)$ -- измеримая на $E$ функция, такая что $1 \le \underline{p}(E) \le \overline{p}(E) < \infty$.
Вопросы базисности классических ортогональных систем в пространствах Лебега с переменным показателем исследовались в статьях~\cite{shii-haar-basis,shii-approx-lpx-2007,izuki-lpx,izuki-lpx-pre}. Отметим, что наиболее важные результаты в этих пространствах в безвесовом случае связаны с условием Дини-Липшица
\begin{equation}\label{DiniLipschitzCond}
 \bigl|p(x)-p(y)\bigr|\ln\frac{1}{|x-y|} \le C,
\end{equation}
обнаруженным впервые в работах~\cite{shii-haar-basis,shii-conv}. В частности, в статье~\cite{shii-haar-basis} было показано, что система Хаара образует базис в пространстве Лебега $L^{p(x)}([0,1])$ с переменным показателем $p(x)$ тогда и только тогда, когда $p(x)$ удовлетворяет условию \eqref{DiniLipschitzCond}. Аналогичный результат для двумерного случая был доказан в статье~\cite{mmg-haar2d}. В упомянутых работах базисность системы Хаара рассматривалась относительно безвесовых пространств Лебега с переменным показателем. В данной работе исследуется этот вопрос для пространств Лебега $L^{p(x)}_w$ с весом $w(x)$, подчиняющимся определенным условиям (см. \eqref{wxOnB1}, \eqref{wxOnB}), одно из которых напоминает известное условие Макенхоупта~\cite{diening-book-2011,diening-muckenhoupt,cruz-maxop}
\begin{equation}\label{MuckenhouptCond}
  \sup\limits_{B\in \mathfrak{B}}
  \Bigl(
  \frac{1}{|B|}
  \int\limits_{B}w(x)dx
  \Bigr)
  \Bigl(\frac{1}{|B|} \int\limits_{B}w(t)^{-\frac{1}{p_B-1}}dt\Bigr)^{p_B-1} < \infty,
\end{equation}
где $\mathfrak{B}$ -- всевозможные интервалы, а $p_B = \Bigl(\frac{1}{|B|}\int\limits_B \frac{1}{p(x)}dx\Bigr)^{-1}$ -- среднее гармоническое $p$ над $B$.
Причина появления второго условия (см. \eqref{wxOnB1}) связана с тем, что в нашем случае переменный показатель может также принимать и значения, равные 1.


\subsection{Система Хаара в $L^{p(x)}_w$}\label{section-hep}
Функции Хаара $\{\chi_k(x)\}_{k=1}^\infty$ определяются на отрезке [0,1] следующим образом~\cite{kashin}:
\begin{equation*}
\chi_1(x)=1,\qquad
\chi_k(x)=\begin{cases}
0,&x\notin\overline{\Delta}_k,\\
2^{j/2},&x\in\Delta_k^+,\\
-2^{j/2},&x\in\Delta_k^-,
\end{cases}
\end{equation*}
где $k=2^j+i, j=0,1,\ldots, \,i=1,\ldots,2^j$, а $\Delta_k$ -- это двоичный интервал вида $\Delta_k=\Delta_j^i=\left(\dfrac{i-1}{2^j},\dfrac{i}{2^j}\right)$, $\overline{\Delta}_k$ -- замыкание интервала $\Delta_k$, а $\Delta_k^+, \Delta_k^-$ -- соответственно правая и левая половины интервала $\Delta_k$. Интервалы $\{\Delta_j^i\}_{i=1}^{2^j}$ называются интервалами $j$-й пачки.

Ясно, что $\chi_k \in L^{p(\cdot)}_w = L^{p(\cdot)}_w([0,1])$ для всякого суммируемого веса $w(x)$. Каждой функции $f\in L^{p(\cdot)}_w$ мы хотим поставить в соответствие ряд Фурье-Хаара:
\begin{equation}\label{FourierSeries}
  f \sim \sum\limits_{k=0}^\infty c_k\chi_k,
\end{equation}
где $c_k=\int\limits_0^1 f(x)\chi_k(x)dx$.
Однако не при всяком весе $w(x)$ для функции $f\in L^{p(x)}_w$ можно построить ряд Фурье.
Поэтому на вес $w(x)$ требуется наложить дополнительные условия, при которых станет возможным вычисление коэффициентов, т.е.
\begin{equation}\label{ck-fin}
  \int\limits_0^1 f(x)\chi_k(x)dx < \infty, \quad k=0,1,\ldots.
\end{equation}
При $k=0$ получаем требование
\begin{equation*}
  \int\limits_0^1 f(x)dx < \infty.
\end{equation*}
Другими словами, вес $w(x)$ должен быть таким, чтобы имело место вложение $L^{p(\cdot)}_w \subset L^1$. Очевидно, что в этом случае будут выполнены неравенства в \eqref{ck-fin} и при $k>0$.

Найдем условия на $w(x)$, достаточные для $f \in L^1([0,1])$. Для этого разобьем отрезок $E=[0,1]$ на множества $E_1=\{x \,|\, p(x)=1\}, E_2=E \setminus E_1$. Тогда
\begin{equation}\label{sumInt}
  \int\limits_E f(x)dx =   \int\limits_{E_1} f(x)dx +  \int\limits_{E_2} f(x)dx.
\end{equation}

На вопрос о существовании первого интеграла отвечает следующая лемма.
\begin{lemma}
Функция $f \in L^{p(\cdot)}_w(E)$ будет суммируемой на $E_1$ в том и только в том случае, если вес отграничен от нуля почти всюду на $E_1$:
\begin{equation}\label{wxE1}
  w(x) \ge C_1(w) > 0 \text{ для почти всех } x \in E_1.
\end{equation}
\end{lemma}



Легко показать, что при выполнении условия \eqref{wxE1} будет справедливо следующее неравенство:
\begin{equation}\label{intE1}
  \int\limits_{E_1} |f(x)|dx \le C(w)\|f\|_{p(\cdot),w}(E).
\end{equation}
Действительно, достаточно воспользоваться свойством \ref{normPropSets}:
\begin{equation*}
  \int\limits_{E_1} |f(x)|dx \le
  \frac{1}{C_1(w)}\int\limits_{E_1} |f(x)|w(x)dx =
  \frac{1}{C_1(w)} \|f\|_{p(\cdot),w}(E_1) \le
  \frac{1}{C_1(w)} \|f\|_{p(\cdot),w}(E).
\end{equation*}

Перейдем теперь к рассмотрению второго интеграла из \eqref{sumInt}:
\begin{equation}\label{intE2step1}
  \int\limits_{E_2} |f(x)|dx=
  \int\limits_{E_2} [w(x)]^{\frac{1}{p(x)}}|f(x)|[w(x)]^{-\frac{1}{p(x)}}dx.
\end{equation}

Применяя последовательно \ref{Holder}, \ref{normLpxwLpx} и \ref{normPropSets}, получим
\begin{multline}\label{intE2step2}
  \int\limits_{E_2} [w(x)]^{\frac{1}{p(x)}}|f(x)|[w(x)]^{-\frac{1}{p(x)}}dx \le
  C(p) \cdot \|w^{\frac{1}{p(\cdot)}}f\|_{p(\cdot)}(E_2) \|w^{-\frac{1}{p(\cdot)}}\|_{p'(\cdot)}(E_2)= \\
  C(p) \cdot \|f\|_{p(\cdot),w}(E_2) \|w^{-\frac{1}{p(\cdot)}}\|_{p'(\cdot)}(E_2) \le
  C(p) \cdot \|f\|_{p(\cdot),w}(E) \|w^{-\frac{1}{p(\cdot)}}\|_{p'(\cdot)}(E_2).
%  \quad (C(p)\le \frac{1}{\underline{p}}+\frac{1}{\underline{p}'}).
\end{multline}
Из \eqref{intE2step1} и \eqref{intE2step2} приходим к неравенству:
\begin{equation}\label{intE2}
  \int\limits_{E_2} |f(x)|dx \le
  C(p) \cdot
  \|f\|_{p(\cdot),w}(E) \cdot
  \|w^{-\frac{1}{p(\cdot)}}\|_{p'(\cdot)}(E_2).
\end{equation}
Из полученного неравенства видно, что для суммируемости $f \in L^{p(x)}_w(E)$ на $E_2$ достаточно, чтобы
\begin{equation*}
  \|w^{-\frac{1}{p(\cdot)}}\|_{p'(\cdot)}(E_2) < \infty.
\end{equation*}

Таким образом, получаем два условия на вес, которые обеспечивают существование рядов Фурье-Хаара в пространстве $L^{p(x)}_w(E)$:
\begin{eqnarray*}
&1) &w(x) \ge C_1(w) > 0, \quad x \in E_1 (\text{п.в.}),\\
&2) &\|w^{-\frac{1}{p(\cdot)}}\|_{p'(\cdot)}(E_2) < \infty.
\end{eqnarray*}

%Множество функций $w$, удовлетворяющих условиям 1) и 2), обозначим через $\mathcal{W}$.
Множество весовых функций $w(x)$, удовлетворяющих условиям 1) и 2), будем обозначать через $\mathcal{H}(E,p)$.

Отметим, что при выполнении условий 1) и 2) из \eqref{intE1} и \eqref{intE2} имеем
\begin{equation}\label{fL1Finite}
  \int\limits_E |f(x)|dx \le
  C(p,w) \cdot \|f\|_{p(\cdot),w}.
\end{equation}



\subsection{Основной результат}
В предыдущем пункте было показано, что при условиях 1), 2) каждому элементу $f \in L^{p(x)}_w(E)$ можно поставить в соответствие ряд Фурье \eqref{FourierSeries}. Возникает вопрос о том, в каких случаях указанный ряд будет сходиться к самой функции $f$.
%Ответ на этот вопрос дает теорема, приведенная ниже.
Ответ на это дают приведенные в этом пункте теоремы.

Введем некоторые обозначения.
Через $\mathfrak{B}_\nu$ обозначим множество всех двоичных интервалов из пачек с номерами $j \ge \nu$, а через $\mathfrak{B}_\nu^{1,p}$ -- такие двоичные интервалы $\Delta_k \in \mathfrak{B}_\nu$, что $\underline{p}(\Delta_k)=1$:
\begin{equation*}
    \mathfrak{B}_\nu = \{\Delta_j^i: j \ge \nu, i=1,\ldots,2^j\}, \quad
    \mathfrak{B}_\nu^{1,p} = \{\Delta_k \in \mathfrak{B}_\nu: \underline{p}(\Delta_k)=1\}.
\end{equation*}
Множество измеримых на $E$ функций $p(x) \ge 1$, удовлетворяющих условию \eqref{DiniLipschitzCond}, будем обозначать символом $\mathcal{P}^{log}(E)$.

\begin{theorem}
Пусть $p(x) \in \mathcal{P}^{log}(E)$, $w(x) \in \mathcal{H}(E,p)$. Тогда система Хаара будет базисом пространства $L^{p(x)}_w(E)$, если для некоторого $\nu \ge 0$ выполняются следующие условия
\begin{gather}
\label{wxOnB1}
\sup\limits_{B \in \mathfrak{B}_\nu^{1,p}} \frac{1}{|B|}\int\limits_B w(x)dx < C(w),\\
\label{wxOnB}
\sup\limits_{B \in \mathfrak{B}_\nu \setminus \mathfrak{B}_\nu^{1,p}}
\Bigl(\frac{1}{|B|}\int\limits_B w(x)dx\Bigr) \Bigl(\frac{1}{|B|}\int\limits_B w(x)^{-\frac{1}{\underline{p}(B)-1}} \Bigr)^{\underline{p}(B)-1} < C(p,w).
\end{gather}
\end{theorem}

Доказательство утверждений, приведенных в данном параграфе, можно найти в работе~\cite{mmg-haar-basis}.
