%Дата последнего изменения файла 12-06-2014
\nonstopmode
\documentclass[12pt]{book}
\include{format-izv}
\newcommand{\const}{\mathrm{const}}
\newcommand{\Span}{\mathrm{Span}\,}
\renewcommand{\Re}{\,\mathrm{Re}\,}
\renewcommand{\Im}{\,\mathrm{Im}\,}
\newcommand{\sgn}{\mathrm{sgn}\,}
\newcommand{\diag}{\mathrm{diag}\,}
%\numberwithin{equation}{section}%Двойная нумерация формул
%Если Вы подключаете новый пакет, то обязательно сообщите об этом в комментариях к тексту статьи.
\begin{document}
%%%%%%%%%%%%%%%%%%%%%%%%%%%%%%%%%%%%%%%%%%%%%%%%%%%%%%%%%%%%%%%%%%%%
\setcounter{section}{0}\setcounter{equation}{0}\setcounter{footnote}{0}
%Для колонтитула. Заполнять обязательно!!!
%\renewcommand{\izscopyrights}{\copyright Султанахмедов~М.~С., 2015}%
%\markboth{Математика}{М.~С.~Султанахмедов. Специальные вейвлеты на основе полиномов Чебышева второго рода
%}%

\UDC{517.51}

\Rtitle{СПЕЦИАЛЬНЫЕ ВЕЙВЛЕТЫ НА ОСНОВЕ ПОЛИНОМОВ ЧЕБЫШЕВА ВТОРОГО РОДА}%
%\thispagestyle{izsc}%не убирать!!!
\Rauthor{М.~С.~Султанахмедов$^1$}

%Фамилия Имя отчество, ученая степень, должность (с указанием кафедры, отдела), место работы (полное
%официальное название учреждения), e-mail;
\Raffil{$^1$Султанахмедов Мурад Салихович, научный сотрудник, Дагестанский научный центр РАН, г. Махачкала,
sultanakhmedov@gmail.com}


\Rabstract{В работе рассмотрена ортогональная система вейвлетов и скалярных функций, основанных на полиномах Чебышева второго рода и их нулях. На их базе построена полная ортонормированная  система функций. Показан недостаток в аппроксимативных свойствах частичных сумм соответствующего вейвлет-ряда, связанный со свойствами самих полиномов Чебышева и заключающийся в существенном ухудшении скорости их сходимости к исходной функции на концах отрезка ортогональности. В качестве альтернативы предлагается модифицировать вейвлет-ряд Чебышева второго рода по аналогии со специальными рядами по ортогональным полиномам со свойством <<прилипания>> на концах отрезка ортогональности. На примере лакунарных частичных сумм доказано, что такой новый специальный вейвлет-ряд лишен указанного недостатка, а, следовательно, обладает более привлекательными аппроксимативными свойствами.}

\Rkeywords{полиномиальные вейвлеты; специальный вейвлет-ряд; полиномы Чебышева второго рода; аппроксимация функций}

%%%%%%%%%%%%%%%%%%%%%%%%%%%%%%%%%%%%%%%%%%%%%%%
%Текст статьи
%%%%%%%%%%%%%%%%%%%%%%%%%%%%%%%%%%%%%%%%%%%%%%%

\section*{1. ВВЕДЕНИЕ}


На сегодняшний день вейвлеты зарекомендовали себя как достаточно эффективный инструмент
в задачах приближения функций, обработки и сжатия цифровых сигналов (временных рядов, звука, изображений и т.д.) (см., напр.,
[1--3]).
В последние годы многими
авторами активно проводятся исследования теории полиномиальных вейвлетов.
Так Chui C.K. и Mhaskar H.N. в работе [4]
впервые рассмотрели вейвлеты на основе тригонометрических полиномов.
Позднее, Kilgore T. и Prestin J. в [5]
заменили тригонометрические полиномы алгебраическими и доказали ортогональность полученной системы функций в смысле чебышевского веса первого рода.
Далее, Fischer B. и Prestin J. в [6],
разработали обобщенную теорию конструирования полиномиальных вейвлетов.
В дальнейшем техника разложения функций в ряды по полиномиальным вейвлетам получила развитие в ряде работ
([7--9] и др.).



Mohd~F. и Mohd~I. в [10]
представили новый, отличный от описанных ранее, способ построения вейвлетов и масштабирующих функций на основе полиномов Чебышева первого рода и их нулей. Используя аналогичную технику, в [11]
автором построена ортогональная система вейвлетов на основе полиномов Чебышева второго рода и исследованы аппроксимативные свойства лакунарных частичных сумм $\mathcal{V}_n(f,x)$ соответствующего вейвлет-ряда в случае $n = 2^m$.
В настоящей статье показан недостаток в свойствах сходимости частичных сумм $\mathcal{V}_n(f,x)$ к исходной функции $f(x)$, связанный со свойствами самих полиномов Чебышева второго рода. Предлагается модифицировать вейвлет-ряд по аналогии со специальными рядами по ортогональным полиномам со свойством <<прилипания>>, введенным в недавних работах Шарапудинова И.И.
([12--13]). Доказано, что такой специальный вейвлет-ряд обладает значительно более привлекательными аппроксимативными свойствами на концах отрезка $[-1, 1]$.



\section*{2. ПРЕДВАРИТЕЛЬНЫЕ СВЕДЕНИЯ}

Пусть $w(x) = \sqrt{1-x^2}$. Обозначим тогда через $L_{2, w}([-1; 1])$ евклидово пространство интегрируемых функций $f(x)$, таких что
$\int\limits_{-1}^{1} f^2(x)w(x)dx < \infty$.
Определим скалярное произведение в нем с помощью равенства
\begin{equation}
\label{scal}
<f, g> = \int\limits_{-1}^{1} f(x) g(x) w(x) dx.
\end{equation}
Хорошо известно, что полиномы Чебышева второго рода
 \begin{equation*}
\label{u2direct}
U_n(x) = \frac{\sin((n+1)\arccos{x})}{\sqrt{1-x^2}}, \quad n = 0,1,2, \ldots .
\end{equation*}
образуют ортогональный базис в $L_{2, w}([-1; 1])$, а именно
\begin{equation}
\label{sms1ortho}
<U_n, U_m> = \frac{\pi}{2}\delta_{nm} =
\left\{
\begin{aligned}
\frac{\pi}{2}, \quad n=m,\\
0, \quad  n \neq m,
\end{aligned}
\right.
\end{equation}
где $\delta_{nm}$ --- символ Кронекера.
Нули $n$-го полинома $U_n(x)$, очевидно, могут быть определены равенством
\begin{equation*}
\label{sms1zeros}
\xi_{k}^{(n)} = \cos{\theta_{k}^{(n)}} =  \cos{\frac{\pi (k+1)}{n+1}}, \quad (k = 0,...,n-1).
\end{equation*}



\section*{3. ВЕЙВЛЕТЫ ЧЕБЫШЕВА ВТОРОГО РОДА}

\noindent\textbf{Определение 1. }\textit{
Масштабирующей функцией Чебышева второго рода назовем полином вида
\begin{equation*}
\label{scaling}
\phi_{n,k}(x) = \sum\limits_{j=0}^{n}U_{j}(x)U_{j}(\xi_{k}^{(n+1)}),
\end{equation*}
где
$n=1,2,\ldots$ и $k=0,1,\ldots,n$.
}


\noindent\textbf{Определение 2. }\textit{
Назовем вейвлет--функцией Чебышева второго рода полином
\begin{equation*}
\label{sms1wavelet}
\psi_{n,k}(x) = \sum\limits_{j=n+1}^{2n}U_{j}(x)U_{j}(\xi_{k}^{(n)}),
\end{equation*}
для любых
$n=1,2,\ldots$ и $k=0,1,\ldots,n-1$.
}



В работе [11]
нами доказано, что системы масштабирующих функций  $\left\{ \phi_{n,k}(x)\right\}_{k=0}^{n}$ и
вейвлет--функций $\left\{ \psi_{n,k}(x)\right\}_{k=0}^{n-1}$ являются ортогональными в $L_{2, w}([-1; 1])$, а именно
\begin{equation*}
<\phi_{n,k}, \phi_{n,l}> =
\frac{\pi(n+2)}{4\sin^2{\frac{\pi(k+1)}{n+2}}} \delta_{kl}, \qquad <\psi_{n,k}, \psi_{n,l}> =
\frac{\pi(n+1)}{4\sin^2{\frac{\pi(k+1)}{n+1}}} \delta_{kl}.
\end{equation*}


Положим тогда
\begin{equation*}
\hat{\phi}_{m,k}(x) = \frac{\phi_{2^m,k}(x)}{\sqrt{<\phi_{2^m,k}, \phi_{2^m,k}>}} = \phi_{2^m,k}(x)   \frac{2\left|\sin{\frac{\pi(k+1)}{2^m+2}}\right|}{\sqrt{\pi(2^m+2)}},
\end{equation*}
\begin{equation*}
\hat{\psi}_{m,k}(x) = \frac{\psi_{2^m,k}(x)}{\sqrt{<\psi_{2^m,k}, \psi_{2^m,k}>}} = \psi_{2^m,k}(x)   \frac{2\left| \sin{\frac{\pi(k+1)}{2^m+1}}\right|}{\sqrt{\pi(2^m+1)}}
\end{equation*}
и введем обозначения

$\Phi_{0} = \left\{\hat{\phi}_{0,0}(x), \hat{\phi}_{0,1}(x) \right\}$,
$\Psi_{1} =  \left\{\hat{\psi}_{0,0}(x) \right\}$,
$\Psi_{2} =  \left\{\hat{\psi}_{1,0}(x), \hat{\psi}_{1,1}(x) \right\}$, \ldots ,

$\Psi_{m} =  \left\{\hat{\psi}_{m-1,0}(x), \hat{\psi}_{m-1,1}(x), \ldots, \hat{\psi}_{m-1, 2^{m-1}-1}(x)  \right\}$,

$\Psi_{m+1} =  \left\{\hat{\psi}_{m,0}(x), \hat{\psi}_{m,1}(x), \ldots, \hat{\psi}_{m, 2^{m}-1}(x)  \right\}$, \ldots,

$\mathcal{P}_m = \left\{\Phi_{0}, \Psi_1, \Psi_2, \ldots, \Psi_m \right\} =
\left\{\hat{\phi}_{0,0}(x), \hat{\phi}_{0,1}(x), \hat{\psi}_{0,0}(x), \hat{\psi}_{1,0}(x), \hat{\psi}_{1,1}(x), \ldots, \right.$

$\left. \hat{\psi}_{m-1,0}(x), \hat{\psi}_{m-1,1}(x), \ldots, \hat{\psi}_{m-1, 2^{m-1}-1}(x) \right\}$.

Далее, пусть $H_{2^{m}, w}([-1; 1])$ --- подпространство в $L_{2, w}([-1; 1])$, состоящее из алгебраических полиномов степени не выше $2^{m}$. В [11]
доказано, что система функций $\mathcal{P}_m$ образует ортонормированный базис в $H_{2^{m}, w}([-1; 1])$, т.е. любой полином $P_{n}(x) \in H_{2^{m}, w}([-1; 1])$ степени $n \leq 2^{m}$,
представим в виде линейной комбинации
\begin{equation}
\label{sms1polydistr}
P_{n}(x) = a_{0}\hat{\phi}_{0,0}(x) + a_{1}\hat{\phi}_{0,1}(x) + \sum\limits_{j=0}^{m-1} \sum\limits_{k=0}^{2^j-1} b_{j,k}\hat{\psi}_{j,k}(x).
\end{equation}

\noindent Кроме того, если рассмотреть теперь бесконечную систему функций

$\mathcal{P} = \left\{\Phi_{0}, \Psi_1, \Psi_2, \ldots, \Psi_m, \ldots \right\} =
\left\{\hat{\phi}_{0,0}(x), \hat{\phi}_{0,1}(x), \hat{\psi}_{0,0}(x), \hat{\psi}_{1,0}(x), \hat{\psi}_{1,1}(x), \ldots,\right.$

$\left. \hat{\psi}_{m-1,0}(x), \hat{\psi}_{m-1,1}(x), \ldots, \hat{\psi}_{m-1, 2^{m-1}-1}(x), \ldots \right\},$

\noindent то она образует ортонормированный базис в $L_{2, w}([-1; 1])$.


Из  последнего утверждения вытекает, что произвольная функция $f(x) \in L_{2, w}([-1; 1])$, может быть представлена в виде сходящегося в $L_{2, w}([-1; 1])$ ряда
\begin{equation}
\label{sms1fdistr}
f(x) = \hat{a}_{0}\hat{\phi}_{0,0}(x) + \hat{a}_{1}\hat{\phi}_{0,1}(x) + \sum\limits_{j=0}^{\infty} \sum\limits_{k=0}^{2^j-1} \hat{b}_{j,k}\hat{\psi}_{j,k}(x),
\end{equation}
\begin{equation*}
\label{sms1fcoeffA}
\text{где}\quad \hat{a}_{0} = \int\limits_{-1}^{1} f(t)\hat{\phi}_{0,0}(t)w(t)dt, \quad \hat{a}_{1} = \int\limits_{-1}^{1} f(t)\hat{\phi}_{0,1}(t)w(t)dt,
\end{equation*}
\begin{equation*}
\label{sms1fcoeffB}
\hat{b}_{j,k} = \int\limits_{-1}^{1} f(t)\hat{\psi}_{j,k}(t)w(t)dt, \quad (j=0,1, \ldots, m; k = 0, 1, \ldots, 2^j-1).
\end{equation*}
Через $\mathcal{V}_{2^m}(f,x)$ обозначим частичную сумму ряда \eqref{sms1fdistr} следующего вида
\begin{equation}
\label{sms1wavepartsum}
\mathcal{V}_{2^m}(f,x) = \hat{a}_{0}\hat{\phi}_{0,0}(x) + \hat{a}_{1}\hat{\phi}_{0,1}(x) + \sum\limits_{j=0}^{m-1} \sum\limits_{k=0}^{2^j-1} \hat{b}_{j,k}\hat{\psi}_{j,k}(x).
\end{equation}

\noindent\textbf{Замечание 1. }\textit{
В силу равенства \eqref{sms1polydistr}, $\mathcal{V}_{2^m}(f, x)$ представляет собой линейный оператор, проектирующий пространство $L_{2, w}([-1; 1])$ на $H_{2^m, w}([-1; 1])$.
}



\section*{4. АППРОКСИМАТИВНЫЕ СВОЙСТВА ЧАСТИЧНЫХ СУММ $\mathcal{V}_{2^m}(f,x)$}

В работе [11]
нами было показано, что частичные суммы \eqref{sms1wavepartsum} ряда \eqref{sms1fdistr} обладают теми же аппроксимативными свойствами, что и частичные суммы $S_{2^{m}}(f,x) = \sum_{k=0}^{n} \hat{f}_k U_k(x)$ ряда Фурье по полиномам Чебышева второго рода:
\begin{equation}
\label{sms1approxprops}
f(x) - \mathcal{V}_{2^m}(f,x) = f(x) - S_{2^{m}}(f,x).
\end{equation}

Опираясь на этот факт и используя оценки из [14-15]
для каждой внутренней точки отрезка $x \in (-1, 1)$ получаем
\begin{equation}
\label{sms1v2mforC}
\left| f(x) - \mathcal{V}_{2^m}(f, x) \right| \leq
E_{2^{m}}(f) \left(\frac{4 \ln{2}}{\pi^2}m + O(1)\right),  \quad f \in C[-1,1],
\end{equation}
где $E_{2^m}(f)$ -- наилучшее приближение функции $f(x)$ алгебраическими полиномами $p_n\in H_{2^{m}, w}([-1; 1])$. Кроме того, при $m \rightarrow \infty$ имеет  место равенство %%%pro lipschitza i (-1,1)
\begin{equation*}
\sup_{f \in \operatorname{Lip}{\alpha}, \atop 0 < \alpha < 1} \left| f(x) - \mathcal{V}_{2^m}(f,x) \right| =
\end{equation*}
\begin{equation}
\label{sms1lipshitz}
2^{-\alpha m} \left[
\frac{2^{\alpha+1}\ln{2}}{\pi} \left( 1-x^2 \right)^{\frac{\alpha}{2}}m
\int_{0}^{\frac{\pi}{2}} t^{\alpha} \sin{t} dt +
O \left( \frac{\sin{2^{m} \arccos {x}}}{\sqrt{1-x^2}} + 1\right)
\right].
\end{equation}

\noindent\textbf{Замечание 2. }\textit{
В случае $\alpha = 1$ последнее равенство выполняется равномерно на всем отрезке $[-1, 1]$.
}


%%%%%%%%%%%%%%%%%%%%%%%%%%%%%%%%%%%%%%%%%%%%%%%%%%%%%%%%%%%%%%%%%%%%%%%%%%%%%%%


В то же время можно привести простой пример аналитической функции $f(x)$, для которой имеет место неравенство
\begin{equation}
\label{sms15.2.16}
\frac{|f(\pm1)-\mathcal{V}_{2^m}(f, \pm1)|}{ E_{2^m}(f)}\ge
c_1 2^m.
\end{equation}
Для этого обратимся к обобщению полиномов Чебышева второго рода --- ультрасферическим полиномам $P_n^{\alpha, \alpha}(x)$ ($n =0,1,2,\ldots$; $\alpha > -1$), образующим ортогональную систему в пространстве $L_{2, \rho}([-1,1])$, где $\rho(x)=(1-x^2)^{\alpha}$.
Рассмотрим известное разложение (см. [16, стр. 94])
производящей функции для них
\begin{equation}
\label{sms15.2.17}
\sum_{n=0}^\infty\frac{\Gamma(\alpha)}{\Gamma(2\alpha+1)}\frac{
     \Gamma(n+2\alpha+1)}{\Gamma(n+\alpha)}P_n^{\alpha,\alpha}(x)
     z^n=(1-2xz+z^2)^{-\alpha-\frac12}.
\end{equation}
При $0<z<1$ функция
\begin{equation}
\label{sms15.2.18}
f(x)=f_z(x)=(1-2xz+z^2)^{-\alpha-\frac12}=(2z)^{-\alpha-\frac12}
     \left({\frac{1+z^2}{2z}}-x\right)^{-\alpha-\frac12}
\end{equation}
аналитична  на всей плоскости,  разрезанной вдоль положительной полуоси с началом в точке $x=\frac{1+z^2}{2z}=a>1$. Известно (см. [17, стр. 476]),
что
\begin{equation}
 E_n[(a-x)^{-\alpha-\frac12}]\asymp \frac{n^{\alpha+\frac12}}{
\Gamma\left(\alpha+\frac12\right)(\sqrt{a^2-1})^{\alpha+\frac32}(a+\sqrt{a^2-1})^n}
     \asymp n^{\alpha+\frac12}z^n.
\end{equation}
Отсюда находим
\begin{equation}
\label{sms15.2.19}
E_n(f)\asymp n^{\alpha+\frac12}z^n.
\end{equation}
С другой стороны, как показано в работе [18],
имеет место
\begin{equation}
\label{sms15.2.20}
|f(\pm1)- S_n^{\alpha,\alpha}(f,\pm1)| \asymp n^{2\alpha+1}z^n.
\end{equation}
Сопоставляя \eqref{sms15.2.19} и \eqref{sms15.2.20}, получаем
\begin{equation*}
\label{sms15.2.21}
{|f(1)-S_n^{\alpha,\alpha}(f, 1)|\over E_n(f)}\ge
c(\alpha)n^{\alpha+\frac12}, \quad (\alpha>-1/2).
\end{equation*}
В интересующем нас частном случае, при $\alpha=\frac12$ и $n = 2^m$, это неравенство принимает вид
\begin{equation}
\label{sms15.2.16.1}
{|f(\pm1)-S_{2^m}(f, \pm1)|\over E_{2^m}(f)} \ge c_1 2^m.
\end{equation}
Из \eqref{sms1approxprops} и \eqref{sms15.2.16.1} приходим к справедливости \eqref{sms15.2.16}.


Приведенный пример показывает, что частичные суммы $\mathcal{V}_{2^m}(f, x)$ на концах отрезка $[-1,1]$ плохо приближают не только непрерывные функции $f \in C[-1,1]$, но также аналитические функции (за исключением алгебраических полиномов).
Может случится, что $\mathcal{V}_{2^m}(f, x)$ приближает $f(x)$ по порядку в $2^m$ раз хуже, чем полином наилучшего приближения $P^{*}_{2^m}(f,x)$.
Этот отрицательный факт является следствием того, что функция Лебега для сумм Фурье-Чебышева второго рода
\begin{equation*}
L_{n}(x) = \int\limits_{-1}^{1}  \left|\sum\limits_{k=0}^{n}\hat U_{k}(x)\hat U_{k}(t)\right| w(t)dt
\end{equation*}
в точках $x = \pm1$ имеет порядок роста, равный $n$ $(n\to\infty)$ (см., напр., [14]).





\section*{5. СПЕЦИАЛЬНЫЙ ВЕЙВЛЕТ-РЯД СО СВОЙСТВОМ <<ПРИЛИПАНИЯ>>}

Для того чтобы устранить указанный негативный эффект, предлагается модифицировать вейвлет-ряд \eqref{sms1fdistr} по схеме, схожей с построением введенных в недавних работах [12--13]
предельных и специальных рядов по ультрасферическим полиномам, обладающих свойством <<прилипания>> на концах отрезка ортогональности. Следуя [13],
введем в рассмотрение функцию

\begin{equation}
\label{sms1Ffunk}
F(x) = \frac{f(x)-c(f,x)}{1-x^2} = \frac{g(f,x)}{1-x^2},
\end{equation}
\begin{equation*}
\label{sms1af}
\text{где}\quad c(f,x) = \frac{f(-1) + f(1)}{2} - \frac{f(-1) - f(1)}{2} x.
\end{equation*}
Если $f(x) \in L_{2, w}([-1; 1])$ то, очевидно, также и $F(x) \in L_{2, w}([-1; 1])$. Тогда она может быть представлена в виде ряда
\begin{equation}
\label{sms1eFdistr}
F(x) = \tilde{a}_{0}\hat{\phi}_{0,0}(x) + \tilde{a}_{1}\hat{\phi}_{0,1}(x) + \sum\limits_{j=0}^{\infty} \sum\limits_{k=0}^{2^j-1} \tilde{b}_{j,k}\hat{\psi}_{j,k}(x),
\end{equation}
\begin{equation*}
\label{sms1efcoeffA0}
\text{где}\quad \tilde{a}_{0} = \int\limits_{-1}^{1} F(t)\hat{\phi}_{0,0}(t) w(t)dt,
\end{equation*}
\begin{equation*}
\label{sms1efcoeffA1}
\tilde{a}_{1} = \int\limits_{-1}^{1} F(t)\hat{\phi}_{0,1}(t) w(t) dt,
\end{equation*}
\begin{equation*}
\label{sms1efcoeffB}
\tilde{b}_{j,k} = \int\limits_{-1}^{1} F(t)\hat{\psi}_{j,k}(t) w(t) dt, \quad (j=0,1, \ldots, m; k = 0, 1, \ldots, 2^j-1).
\end{equation*}
Выразим теперь из \eqref{sms1Ffunk} и \eqref{sms1eFdistr} исходную функцию
\begin{equation*}
\label{sms1fFunk}
f(x)=
c(f,x)+(1-x^2) \left[ \tilde{a}_{0}\hat{\phi}_{0,0}(x) + \tilde{a}_{1}\hat{\phi}_{0,1}(x) + \sum\limits_{j=0}^{\infty} \sum\limits_{k=0}^{2^j-1} \tilde{b}_{j,k}\hat{\psi}_{j,k}(x) \right].
\end{equation*}
Будем называть такой модифицированный ряд \textit{специальным вейвлет-рядом Чебышева второго рода}. Обозначим его частичную сумму
\begin{equation}
\label{sms1fFunkSum}
\tilde{\mathcal{V}}_{2^m}(f,x)  = c(f,x)+(1-x^2) \left[ \tilde{a}_{0}\hat{\phi}_{0,0}(x) + \tilde{a}_{1}\hat{\phi}_{0,1}(x) + \sum\limits_{j=0}^{m-1} \sum\limits_{k=0}^{2^j-1} \tilde{b}_{j,k}\hat{\psi}_{j,k}(x) \right].
\end{equation}

\noindent Рассмотрим некоторые свойства $\tilde{\mathcal{V}}_{2^m}(f,x)$. Из явного вида \eqref{sms1fFunkSum} легко следует

\noindent\textbf{Теорема 1. }\textit{
Частичная сумма $\tilde{\mathcal{V}}_{2^m}(f,x)$ на концах отрезка $[-1,1]$ совпадает с исходной функцией $f(x)$, т.е. $\tilde{\mathcal{V}}_{2^m}(f,\pm1) = f(\pm1)$.
}

\noindent Как отмечено в [13],
это свойство имеет важное значение в задачах, связанных с обработкой временных рядов и изображений.

\noindent\textbf{Теорема 2. }\textit{
Частичная сумма $\tilde{\mathcal{V}}_{2^m}(f,x)$ представляет собой линейный оператор, проектирующий пространство $L_{2, w}([-1; 1])$ на $H_{2^m+2, w}([-1; 1])$, т.е. для любого полинома $P_n(x)$ степени не выше $n\le 2^{m}+2$ справедливо равенство $\tilde{\mathcal{V}}_{2^m}(P_n,x) \equiv P_n(x)$.
}

% \noindent\textbf{Доказательство. }
% Для произвольного полинома $P_n(x)$ $(n\le 2^{m}+2)$ рассмотрим выражение вида \eqref{sms1Ffunk}:
% \begin{equation}
% \label{sms1Pfunk1}
% \frac{P_n(x)-c(P_n,x)}{1-x^2}=\frac{g(P_n,x)}{1-x^2}.
% \end{equation}
% Очевидно, $g(P_n,x)$ представляет из себя полином степени $n$, причем $g(P_n,\pm1)=0$. Вследствии теоремы Безу, он представим в виде $g(P_n,x)=(1-x^2)M_{n-2}(x)$, где $M_{n-2}(x)$ -- полином степени $n-2 \le 2^{m}$. Таким образом, из \eqref{sms1Pfunk1} имеем
% \begin{equation*}
% \label{sms1Pfunk2}
% \frac{P_n(x)-c(P_n,x)}{1-x^2}=M_{n-2}(x).
% \end{equation*}
% Воспользуемся теперь замечанием 1p, тогда $\mathcal{V}_{2^m}(M_{n-2},x) = M_{n-2}(x)$, откуда
% $$\mathcal{V}_{2^m}(M_{n-2},x) = M_{n-2}(x)= \frac{P_n(x)-c(P_n,x)}{1-x^2}.$$
% Выражая отсюда $P_n(x)$, получаем окончательно
% $$P_n(x)= c(P_n,x) + (1-x^2)\mathcal{V}_{2^m}(M_{n-2},x) = \tilde{\mathcal{V}}_{2^m}(P_n,x).$$
% \hfill$\Box$



\noindent\textbf{Теорема 3. }\textit{
Для любой функции $f\in C[-1,1]$ и любого $x \in (-1, 1)$ имеет место оценка
\begin{equation*}
\label{sms1thrm3eq}
|f(x)-\tilde{\mathcal{V}}_m(f,x)|\le c E_{2^{m}+2}(f)(1+\ln(1+(2^{m}+2)\sqrt{1-x^2})),
\end{equation*}
где $c >0$ -- константа.
}

% \noindent\textbf{Доказательство. }
% Как отмечалось выше, в работе [13]
% сконструированы специальные ряды по ультрасферическим полиномам, обладающие свойством <<прилипания>> в точках $x=\pm1$. Частичные суммы такого ряда в случае $\alpha = \frac12$ могут быть записаны в виде
% \begin{equation*}
% \label{sms1sigma}
% \sigma_{n}(f,x) = \sigma^{\frac12}_{n}(f,x) = c(f,x)+(1-x^2) S_{n-2}(F, x).
% \end{equation*}
% Рассмотрим отклонение этих частичных сумм от аппроксимируемой функции
% \begin{equation*}
% |f(x)-\sigma_n(f,x)| = |f(x)-c(f,x)-(1-x^2) S_{n-2}(F, x)| =
% \end{equation*}
% \begin{equation}
% \label{sms1sigmaV1}
% |g(f,x)-(1-x^2) S_{n-2}(F, x)|=(1-x^2) |F(x)-S_{n-2}(F, x)|.
% \end{equation}
% С другой стороны, преобразовав аналогичным образом погрешность приближения функции $f(x)$ частичной суммой $\tilde{\mathcal{V}}_m(f,x)$ специального вейвлет-ряда, получим
% \begin{equation*}
% |f(x)-\tilde{\mathcal{V}}_m(f,x)| = (1-x^2) |F(x)- \mathcal{V}_m(F, x)|.
% \end{equation*}
% Отсюда и из \eqref{sms1approxprops} тогда имеем
% \begin{equation}
% \label{sms1sigmaV2}
% |f(x)-\tilde{\mathcal{V}}_m(f,x)| = (1-x^2) \left|F(x) - S_{2^{m}}(F,x) \right|.
% \end{equation}
% Сопоставляя теперь \eqref{sms1sigmaV1} с \eqref{sigmaV2}, получаем
% \begin{equation}
% \label{sms1sigmaV3}
% |f(x)-\tilde{\mathcal{V}}_m(f,x)| = |f(x)-\sigma_{2^{m}+2}(f,x)|.
% \end{equation}
% В [13]
% доказана следующая оценка
% \begin{equation}
% \label{sms1sigmaestim}
% |f(x)-\sigma_n(f,x)|\le c E_n(f)(1+\ln(1+n\sqrt{1-x^2})).
% \end{equation}
% Из \eqref{sigmaV3}--\eqref{sms1sigmaestim} следует справедливость утверждения теоремы.
% \hfill$\Box$





Рассмотрим теперь верхнюю грань отклонения частичных сумм $\tilde{\mathcal{V}}_{2^m}(f,x)$ от функций из класса Липшица $\operatorname{Lip} \alpha$ $(0< \alpha <1)$.

\noindent\textbf{Теорема 4. }\textit{
Для каждой внутренней точки отрезка $x \in (-1, 1)$ при $m \rightarrow \infty$ справедливо асимптотическое равенство
\begin{equation*}
\sup_{f \in \operatorname{Lip}{\alpha}, \atop 0 < \alpha < 1} \left| f(x) - \tilde{\mathcal{V}}_{2^m}(f,x) \right| =
\end{equation*}
\begin{equation*}
\label{sms1thrm4eq}
2^{-\alpha m} \left[
\frac{2^{\alpha+1}\ln{2}}{\pi} \left( 1-x^2 \right)^{\frac{\alpha}{2}+1}m
\int_{0}^{\frac{\pi}{2}} t^{\alpha} \sin{t} dt +
O \left( \frac{\sin{2^{m} \arccos {x}}}{\sqrt{1-x^2}} + 1\right)
\right].
\end{equation*}
}


% \noindent\textbf{Доказательство. }
% Из \eqref{sms1fFunkSum} имеем
% \begin{equation}
% \tilde{\mathcal{V}}_{2^m}(f,x)  = c(f,x)+(1-x^2) \mathcal{V}_{2^m}(F,x).
% \end{equation}
% Отклонение частичных сумм специального вейвлет-ряда тогда может быть записано в следующем виде
% \begin{equation*}
% \left|f(x)-\tilde{\mathcal{V}}_{2^m}(f,x)\right| = \left|f(x)-c(f,x)-(1-x^2) \mathcal{V}_{2^m}(F,x)\right| =
% \end{equation*}
% \begin{equation}
% \label{sms1sigmaeq}
% \left|g(f,x)-(1-x^2) \mathcal{V}_{2^m}(F,x)\right| =
% \left|f(x)-\tilde{\mathcal{V}}_{2^m}(f,x)\right| =(1-x^2) \left|F(x)-\mathcal{V}_{2^m}(F,x)\right|.
% \end{equation}
% Воспользовавшись \eqref{sms1lipshitz}, имеем
% \begin{equation*}
% \sup_{f \in \operatorname{Lip}{\alpha}, \atop 0 < \alpha < 1} \left| f(x) - \tilde{\mathcal{V}}_{2^m}(f,x) \right| =
% \sup_{f \in \operatorname{Lip}{\alpha}, \atop 0 < \alpha < 1}
% \left|1-x^2\right| \left|F(x)-\mathcal{V}_{2^m}(F,x)\right| =
% \end{equation*}
% \begin{equation*}
% 2^{-\alpha m} \left[
% \frac{2^{\alpha+1}\ln{2}}{\pi} \left( \sqrt{1-x^2} \right)^{\alpha+2}m
% \int_{0}^{\frac{\pi}{2}} t^{\alpha} \sin{t} dt +
% O \left( \frac{\sin{2^{m} \arccos {x}}}{\sqrt{1-x^2}} + 1\right)
% \right].
% \end{equation*}

% \hfill$\Box$

\noindent\textbf{Замечание 2. }\textit{
Из замечания 2 следует, что в случае $\alpha = 1$ утверждение теоремы 4 справедливо на всем отрезке $[-1, 1]$.
}


Теоремы 1--4 показывают, что частичные суммы $\tilde{\mathcal{V}}_{2^m}(f,x)$ как аппарат приближения обладают весьма привлекательными аппроксимативными свойствами.




\vspace*{10mm}

\noindent \textit{Автор благодарит д.ф.-м.н. И.И. Шарапудинова за постановку задачи и ценные советы при ее решении.}




%%%%%%%%%%%%%%%%%%%%%%%%%%%%%%%%%%%%%%%%%%%%%%%%%%%%%%%%%%%%%%%%%%%%%%
%Библиографический список
%%%%%%%%%%%%%%%%%%%%%%%%%%%%%%%%%%%%%%%%%%%%%%%%%%%%%%%%%%%%%%%%%%%%%%
\vspace*{3mm}


\begin{Rtwocolbib}%Не убирать!!!

1. \textit{Meyer, Y.} Ondelettes et Operateurs. Paris: Hermann, 1990, Vol.~I–III.

2. \textit{Daubechies, L.} Ten Lectures on Wavelets. CBMS-NSF Regional Conference Series in Applied Mathematics Proceedings,
Vol. 61. Philadelphia, PA: SIAM, 1992, 357~p. ISBN 0-89871-274-2. DOI: 10.1137/1.9781611970104.

3. \textit{Chui, C.K.} An Introduction to  Wavelets. Boston: Academic Press., 1992, 271~p. ISBN 0-12-174584-8.

4. \textit{Chui, C.K., Mhaskar, H.N.} On Trigonometric Wavelets // Constructive Approximation, 1993. Vol.~9, Issue~2--3. pp.~167--190. DOI: 10.1007/BF01198002.

5. \textit{Kilgore, T., Prestin, J.} Polynomial wavelets on an interval // Constructive Approximation, 1996. Vol.~12, Issue~1. pp.~95--110. DOI: 10.1007/BF02432856.

6. \textit{Fischer, B., Prestin, J.} Wavelet based on orthogonal polynomials // Mathematics of computation, 1997. Vol.~66, Num.~220. pp.~1593--1618. DOI: 10.1090/S0025-5718-97-00876-4.

7. \textit{Fischer, B., Themistoclakis, W.} Orthogonal polynomial wavelets // Numerical Algorithms, 2002. Vol.~30, Issue~1. pp.~37--58. DOI: 10.1023/A:1015689418605.

8. \textit{Capobiancho, M.R., Themistoclakis, W.} Interpolating polynomial wavelet on $[-1,1]$ // Advanced in Computational Mathematics, 2005. Vol.~23, Issue~4. pp.~353–-374. DOI: 10.1007/s10444-004-1828-2.

9. \textit{Dao-Qing Dai, Wei Lin} Orthonormal polynomial wavelets on the interval // Proceedings of the American Mathematical Society, 2005. Vol.~134, Issue~5. pp.~1383–1390. DOI: 10.1090/S0002-9939-05-08088-3.

10. \textit{Mohd, F., Mohd, I.} Orthogonal Functions Based on Chebyshev Polynomials // Matematika, 2011. Vol.~27, Num.~1, pp.~97-–107.

11. \textit{Султанахмедов М.С.} Аппроксимативные свойства вейвлетов, построенных на основе полиномов Чебышева второго рода // Владикавказский математический журнал, 2015, Вып. , С.~.

12. \textit{Шарапудинов И.И.} Предельные ультрасферические ряды и их аппроксимативные свойства // Математические заметки, 2013. Т.~94, Вып.~2. С.~295–-309. DOI: 10.4213/mzm10292.

13. \textit{Шарапудинов И.И.} Некоторые специальные ряды по ультрасферическим полиномам и их аппроксимативные свойства // Известия РАН. Серия математическая. Т.~78,	Вып.~5,	C.~201–-224.  DOI: 10.4213/im8117.

14. \textit{Яхнин Б.М.} О функциях Лебега разложений в ряды по полиномам Якоби для случаев $\alpha=\beta=\frac12, \alpha=\beta=-\frac12, \alpha=\frac12, \beta=-\frac12$ // Успехи математических наук, 1958. Т.~13, Вып.~6(84), C.~207-–211.

15. \textit{Яхнин Б.М.} Приближение функций класса $Lip_\alpha$ частными суммами ряда Фурье по многочленам Чебышева второго рода // Изв. вузов. Матем., 1963. Вып.~1, C.~172-–178.

16. \textit{Сеге Г.} Ортогональные многочлены.  Москва: Физматлит, 1962. 500~с.

17. \textit{Тиман А.Ф.} Теория приближения функций действительного переменного. Москва: Физматгиз, 1960. 626~с.

18. \textit{Шарапудинов И.И.} О наилучшем приближении и суммах Фурье–Якоби // Математические заметки, 1983, Т.~34, Вып.~5, С.~651-–661. DOI: 10.1007/BF01157445.

\end{Rtwocolbib}%Не убирать!!!

\Etitle{Special Wavelets Based On Chebyshev Polynomials of the Second Kind and Their Approximative Properties}

\Eauthor{M.~S.~Sultanakhmedov$^1$}%

%На англ. языке: Фамилия Имя Отчество, официальное название организации, почтовый адрес
%организации, e-mail автора
\Eaffil{$^1$Sultanakhmedov Murad Salikhovich, scientific worker of Department of Mathematics and Computer Science, Daghestan Scientific Center of RAS, Gadzhiev st., 45, 367000, Makhachkala, Daghestan, Russia, sultanakhmedov@gmail.com}%

\Eabstract{
System of wavelets and scalar functions based on Chebyshev polynomials of the second kind and their zeros is considered.
With the help of them we construct a complete orthonormal system of functions.
It is shown %/ submitted
certain disadvantage in approximation properties of partial sums of the corresponding wavelet series,
related to the properties of Chebyshev polynomials themselves
and consisting in a significant decrease of the rate of their convergence to the original function at the endpoints of orthogonality segment.
%is shown.
As an alternative, we propose modification of Chebyshev wavelet series of the second kind by analogy to the special polynomial series with the property of adhesion. %at the endpoints of orthogonality segment.
These new special wavelet series is proved to be deprived of the mentioned disadvantage and to have better approximative properties.
}

\Ekeywords{polynomial wavelets; special wavelet series; Chebyshev polynomials of the second kind; function approximation}




%%%%%%%%%%%%%%%%%%%%%%%%%%%%%%%%%%%%%%%%%%%%%%%%%%%%%%%%%%%%%%%
%Ссылка на источник финансирования (если есть)
%%%%%%%%%%%%%%%%%%%%%%%%%%%%%%%%%%%%%%%%%%%%%%%%%%%%%%%%%%%%%%%

\vspace*{2mm}%\newpage

%%%%%%%%%%%%%%%%%%%%%%%%%%%%%%%%%%%%%%%%%%%%%%%%%%%%%%%%%%%%%%%%%%%%%%
%Библиографический список на англ. языке
%%%%%%%%%%%%%%%%%%%%%%%%%%%%%%%%%%%%%%%%%%%%%%%%%%%%%%%%%%%%%%%%%%%%%%
\begin{Etwocolbib}%Не убирать!!!

1. Meyer, Y. \textit{Ondelettes et Operateurs}. Paris: Hermann, 1990, Vol.~I–III.

2. Daubechies, L. \textit{Ten Lectures on Wavelets}. CBMS-NSF Regional Conference Series in Applied Mathematics Proceedings, Vol. 61. Philadelphia, PA: SIAM, 1992, 357~p. ISBN 0-89871-274-2. DOI: 10.1137/1.9781611970104.

3. Chui, C.K. \textit{An Introduction to  Wavelets. Wavelet Analysis and its Applications, Vol.~1.} Boston: Academic Press., 1992, 264~p. ISBN 0-12-174584-8.

4. Chui, C.K., Mhaskar, H.N. On Trigonometric Wavelets. \textit{Constructive Approximation}, 1993. Vol.~9, Issue~2--3. pp.~167--190. DOI: 10.1007/BF01198002.

5. Kilgore, T., Prestin, J. Polynomial wavelets on an interval. \textit{Constructive Approximation}, 1996. Vol.~12, Issue~1. pp.~95--110. DOI: 10.1007/BF02432856.

6. Fischer, B., Prestin, J. Wavelet based on orthogonal polynomials. \textit{Mathematics of computation}, 1997. Vol.~66, Num.~220. pp.~1593--1618. DOI: 10.1090/S0025-5718-97-00876-4.

7. Fischer, B., Themistoclakis, W. Orthogonal polynomial wavelets. \textit{Numerical Algorithms}, 2002. Vol.~30, Issue~1. pp.~37--58. DOI: 10.1023/A:1015689418605.

8. Capobiancho, M.R. and Themistoclakis, W. Interpolating polynomial wavelet on $[-1,1]$. \textit{Advanced in Computational Mathematics}, 2005. Vol.~23, Issue~4. pp.~353–-374. DOI: 10.1007/s10444-004-1828-2.

9.Dao-Qing Dai, Wei Lin. Orthonormal polynomial wavelets on the interval. \textit{Proceedings of the American Mathematical Society}, 2005. Vol.~134, Issue~5. pp.~1383–1390. DOI: 10.1090/S0002-9939-05-08088-3.

10. Mohd, F., Mohd, I. Orthogonal Functions Based on Chebyshev Polynomials. \textit{Matematika}, 2011. Vol.~27, Num.~1, pp.~97-–107.

11. Sultanakhmedov M.S. Approximative properties of the Chebyshev wavelet series of the second kind. \textit{Vladikavkaz. Mat. Zh} [Vladikavkazskij matematicheskij zhurnal], 2015, Vol. , pp.~.

12. Sharapudinov~I.I. Limit ultraspherical series and their approximation properties. \textit{Mathematical Notes} [Matematicheskie zametki], 2013. Vol.~94, Issue~1-2, pp.~281-293. DOI: 10.1134/S0001434613070274.

13. Sharapudinov~I.I. Some special series in ultraspherical polynomials and their approximation properties. \textit{Izvestiya: Mathematics} [Izvestija RAN. Serija matematicheskaja], 2014. Vol.~78, Num.~5, 1036, pp.~201–-224. DOI: 10.1070/IM2014v078n05ABEH002718.

14. Yakhnin~B.M. O funkcijah Lebega razlozhenij v rjady po polinomam Jakobi dlja sluchaev $\alpha=\beta=\frac12, \alpha=\beta=-\frac12, \alpha=\frac12, \beta=-\frac12$. [Lebesgue functions for expansions in series of Jacobi polynomials for the cases $\alpha=\beta=\frac12, \alpha=\beta=-\frac12, \alpha=\frac12, \beta=-\frac12$] \textit{Usp. Mat. Nauk} [Uspehi matematicheskih nauk], 1958. Vol.~13, No.~6(84), pp.~207-–211. (in Russian).

15. Yakhnin~B.M. Priblizhenie funkcij klassa $Lip_\alpha$ chastnymi summami rjada Fur'e po mnogochlenam Chebysheva 2-go roda. [Approximation of functions of class Lip $\alpha $ by partial sums of a Fourier series in Chebyshev polynomials of second kind] \textit{Russian Math.} [Izvestiya VUZ. Matematika], 1963. No.~1 (32), pp.~172-–178.  (in Russian).

16. Szeg$\ddot{o}$ G. \textit{Orthogonal Polynomials}. Colloquium Publications - American Mathematical Society (Vol.~23), 1939. 432~p. ISBN 0821889524, 9780821889527.

17. Timan~A.~F. \textit{Teorija priblizhenija funkcij dejstvitel'nogo peremennogo}
[Theory of approximation of functions of a real variable]. Moscow, Fizmatgiz, 1960. 626~p.  (in Russian).

18. Sharapudinov~I.I. Best approximation and the Fourier-Jacobi sums. \textit{Mathematical Notes} [Matematicheskie zametki], 1983, Vol.~34, Num.~5, Issue~5, pp.~816--821. DOI: 10.1007/BF01157445.


\end{Etwocolbib}%Не убирать!!!

\vspace*{5mm}


\newpage
\newpage

\noindent\begin{minipage}{160mm}
%%%%%%%%%%%%%%%%%%%%%%%%%%%%%%%%%%%%%%%%%%%%%%%%%%%%%%%%%%%%%%%%%%%%%
\noindent\textsf{УДК~517.51}

\titler{ПРИБЛИЖЕНИЕ ДИСКРЕТНЫХ ФУНКЦИЙ СРЕДНИМИ ВАЛЛЕ -- ПУССЕНА ЧАСТИЧНЫХ СУММ
КОНЕЧНОГО ПРЕДЕЛЬНОГО РЯДА ПО ПОЛИНОМАМ ЧЕБЫШ\"ЕВА, ОРТОГОНАЛЬНЫМ НА РАВНОМЕРНОЙ СЕТКЕ}

\textbf{М.\,С.~Султанахмедов}\vspace*{3mm}

Дагестанский Научный Центр РАН,\\
Отдел Математики и Информатики\\
E-mail: sultanakhmedov@gmail.com \vspace*{3mm}

Исследованы аппроксимативные свойства средних Валле -- Пуссена частичных сумм конечного предельного ряда по дискретным полиномам Чебыш\"ева $\left\{ T_n^{\alpha, \beta}(x, N) \right\}_{n=0}^{N-1}$, ортонормированным на равномерной сетке $\Omega_N=\left\{ 0, 1, \ldots, N-1\right\}$.

\vspace*{3mm}

\textbf{Ключевые слова:} средние Валле -- Пуссена; полиномы Чебыш\"ева; конечный предельный ряд; полиномы, ортогональные на сетках; равномерная сетка;  аппроксимативные свойства.\vspace*{3mm}

\textbf{Approximation of Discrete Functions by de la Vallee - Poussin means of Partial Sums of the Limit Series by Chebyshev Polynomials Orthogonal on a Uniform Grid}\vspace*{4pt}

\textbf{M.~S.~Sultanakhmedov}\vspace*{4pt}


Daghestan Scientific Center of RAS,\\
Department of Mathematics and Computer Science\\
E-mail: sultanakhmedov@gmail.com \vspace*{4pt}

Approximative properties of de la Vallee - Poussin means over the discrete partial sums of Limit Series by Chebyshev polynomials  $\left\{ T_n^{\alpha, \beta}(x, N) \right\}_{n=0}^{N-1}$, orthogonal on unform net $\Omega_N=\left\{ 0, 1, \ldots, N-1\right\}$ are investigated.

\textbf{Keywords:} de la Vallee - Poussin means; Chebychev polynomials; polynomials orthogonal on nets; limit series; uniform net; robust calculation; approximative properties.
\end{minipage}
\vspace*{3mm}




%%%%%%%%%%%%%%%%%%%%%%%%%%%%%%%%%%%%
%%%%%%%%%%%%%%%%%%%%%%%%%%%%%%%%%%%%
%%%%%%%%%%%%%%%%%%%%%%%%%%%%%%%%%%%%

\section*{ВВЕДЕНИЕ}

В статье рассмотрена задача о приближении дискретной действительнозначной функции $f$, заданной на равномерной сетке $\Omega_N = \left\{ 0, 1, \ldots, N-1 \right\}$, средними типа Валле -- Пуссена частичных сумм так называемого \textit{конечного предельного ряда} по классическим ортогональным полиномам Чебыш\"ева, ортогональным на $\Omega_N$. Основной результат настоящей статьи касается оценки функции Лебега для рассматриваемых средних типа Валле -- Пуссена. Для того, чтобы сформулировать его более точно, нам понадобятся некоторые обозначения и факты.


Для всех $\alpha,\beta>-1$ через $T_{n}^{\alpha,\beta}(x,N), (0\le n\le N-1)$ обозначим полиномы Чебыш\"ева, ортогональные на равномерной сетке $\Omega_N$ с весом
\begin{equation}
\label{sms2eq4}
\mu(x) = \mu(x; \alpha,\beta, N)={\Gamma(N)2^{\alpha+\beta+1}\over \Gamma(N+\alpha+\beta+1)}
{\Gamma(x+\beta+1)\Gamma(N-x+\alpha)\over \Gamma(x+1)\Gamma(N-x)}
\end{equation}
и нормированные условием $T_{n}^{\alpha,\beta}(N-1,N)={n+\alpha\choose n}$. Более точно, имеет место равенство
\begin{equation*}
\sum_{x=0}^{N-1}T_{n}^{\alpha,\beta}(x) T_{m}^{\alpha,\beta}(x)
\mu(x)= \delta_{n,m}h_{n,N}^{\alpha,\beta},\quad 0\le n,m\le N-1,
\end{equation*}
где
\begin{equation}
\label{sms2hn}
h_{n,N}^{\alpha,\beta}={(N+n+\alpha+\beta)^{[n]}\over
(N-1)^{[n]}}{\Gamma(n+\alpha+1)\Gamma(n+\beta+1)
2^{\alpha+\beta+1}\over
n!\Gamma(n+\alpha+\beta+1)(2n+\alpha+\beta+1)}.
\end{equation}
Построим тогда на основе полиномов $T_{n}^{\alpha,\beta}(x)$ ортонормированную на
$\Omega_N$ с весом $\mu(x)$ систему полиномов:
\begin{equation}
\label{sms2eq5}
\tau_{n}^{\alpha,\beta}(x) = \tau_{n}^{\alpha,\beta}(x,N)=
\left[h_{n,N}^{\alpha,\beta}\right]^{-1/2}
T_{n}^{\alpha,\beta}(x,N),
\end{equation}
\begin{equation}
\label{sms2eq6}
\sum_{x=0}^{N-1}\tau_{n}^{\alpha,\beta}(x)
\tau_{m}^{\alpha,\beta}(x)\mu(x)=\delta_{n,m}, \quad (0\le n\le N-1).
\end{equation}



В таком случае для заданной на сетке $\Omega_N$ дискретной функции $f(x)$ мы можем определить дискретный ряд Фурье-Чебыш\"ева
\begin{equation}
\label{sms2eq1}
f(x)=\sum\limits_{k=0}^{N-1}\hat{f}^{\alpha,\beta}_k\tau_{k}^{\alpha,\beta}(x),\quad (x \in \Omega_N),
\end{equation}
где коэффициенты Фурье-Чебыш\"ева задаются формулой
\begin{equation}
\label{sms2eq2}
\hat{f}_k = \sum\limits_{x=0}^{N-1}f(x)\tau_{k}^{\alpha,\beta}(x)\mu(x),\quad (0\le n\le N-1).
\end{equation}
Соответственно частичная сумма $n$-го порядка дискретной суммы Фурье -- Чебыш\"ева может быть записана в виде
\begin{equation}
\label{sms2eq5}
S_{n,N}^{\alpha,\beta}(f,x) = \sum\limits_{k=0}^{n}\hat{f}^{\alpha,\beta}_k \tau_{k}^{\alpha,\beta}(x),\quad (0\le x\le N-1).
\end{equation}

%Далее для краткости мы примем следующие обозначения: $\tau_{n,N}^{\alpha}(x) = \tau_{n,N}^{\alpha,\alpha}(x)$, $S_{n,N}^{\alpha}(f,x) = S_{n,N}^{\alpha,\alpha}(f,x)$.
Далее в случаях, когда $\alpha = \beta$, для краткости мы будем записывать в качестве индекса лишь один параметр, например: $\tau_{n}^{\alpha}(x) = \tau_{n}^{\alpha,\alpha}(x)$, $S_{n,N}^{\alpha}(f,x) = S_{n,N}^{\alpha,\alpha}(f,x)$ и т.д.




%%%%%%%%%%%%%%%%%%%%%%%%%%%%%%%%%%%%
%%%%%%%%%%%%%%%%%%%%%%%%%%%%%%%%%%%%
%%%%%%%%%%%%%%%%%%%%%%%%%%%%%%%%%%%%

\section*{НЕКОТОРЫЕ СВОЙСТВА ДИСКРЕТНЫХ ПОЛИНОМОВ ЧЕБЫШ\"ЕВА}

В дальнейшем изложении мы будем опираться на ряд свойств дискретных полиномов Чебыш\"ева, которые мы соберем в данном пункте (см., например, [1, \S\,3.2, ШИИ, Спецвыпуск в ДЭМИ]).

Имеет место следующее явное представление дискретных полиномов Чебыш\"ева:
\begin{equation}
\label{sms2eq3}
T_{n}^{\alpha,\beta}(x) = T_{n}^{\alpha,\beta}(x,N)=(-1)^n{\Gamma(n+\beta+1)\over n!}\sum_{k=0}^n (-1)^k
{n^{[k]}(n+\alpha+\beta+1)_kx^{[k]}\over \Gamma(k+\beta+1)
k!(N-1)^{[k]}}.
\end{equation}
Здесь $a^{[0]}=1, a^{[n]}=a(a-1)\ldots (a-n+1)$; $(a)_0=1, (a)_n=a(a+1)\ldots(a+n-1)$.


Для них справедливы следующие рекуррентные формулы:
\begin{equation}\label{recur}
  A^{\alpha,\beta}_{n,N} T^{\alpha,\beta}_{n+1}(x,N) = (x-B^{\alpha,\beta}_{n,N}) T^{\alpha,\beta}_{n}(x,N) - C^{\alpha,\beta}_{n,N} T^{\alpha,\beta}_{n-1}(x,N),
\end{equation}
где $T^{\alpha,\beta}_{0}(x,N)=1$, $T^{\alpha,\beta}_{1}(x,N)=-1-\beta+x(2+\alpha+\beta)/(N-1)$,
\begin{equation*}
  A^{\alpha,\beta}_{n,N} = \frac{(n+1)(n+\alpha+\beta+1)(N-n-1)}{(2n+\alpha+\beta+1)(2n+\alpha+\beta+2)},
\end{equation*}
\begin{equation*}
  C^{\alpha,\beta}_{n,N} = \frac{(n+\alpha)(n+\beta)(N+n+\alpha+\beta)}{(2n+\alpha+\beta)(2n+\alpha+\beta+1)},
\end{equation*}
\begin{equation*}
  B^{\alpha,\beta}_{n,N} = A^{\alpha,\beta}_{n,N}\,\frac{n+\beta+1}{n+1}+C^{\alpha,\beta}_{n,N}\,\frac{n}{n+\beta},
\end{equation*}
а также свойство симметрии:
\begin{equation}\label{simmetry}
  T^{\alpha,\beta}_{n}(x) = (-1)^{n} T^{\beta, \alpha}_{n}(N-1-x).
\end{equation}
Кроме того, из явного вида и свойств конечной разности вытекает следующее равенство
\begin{equation}\label{recur1}
  (n+1) T^{\alpha,\beta}_{n+1}(x,N) + (n+\beta+1)T^{\alpha,\beta}_{n}(x,N) = \frac{2n+\alpha+\beta+2}{N-1}\,x\,T^{\alpha,\beta+1}_{n}(x-1,N-1),
\end{equation}
из которого, в свою очередь, используя свойство симметрии \eqref{simmetry} получаем
\begin{equation}\label{recur2}
  (n+\alpha+1) T^{\alpha,\beta}_{n}(x,N) - (n+1)T^{\alpha,\beta}_{n+1}(x,N) = \frac{2n+\alpha+\beta+2}{N-1}\,(N-1-x)\,T^{\alpha+1,\beta}_{n}(x,N-1).
\end{equation}





Для этих полиномов справедлива формула Кристоффеля -- Дарбу:
\begin{equation*}
  D^{\alpha,\beta}_{n} (x,y) = D^{\alpha,\beta}_{n,N} (x,y) = \sum_{k=0}^{n} \frac{T^{\alpha,\beta}_{k}(x) T^{\alpha,\beta}_{k}(y)}{h^{\alpha,\beta}_{n,N}} =
\end{equation*}
\begin{equation*}
  \frac{(N-1)^{[n+1]}}{(N+n+\alpha+\beta)^{[n]}} \frac{2^{-\alpha-\beta-1}}{2n+\alpha+\beta+2}
  \frac{\Gamma(n+2)\Gamma(n+\alpha+\beta+2)}{\Gamma(n+\alpha+1)\Gamma(n+\beta+1)} \cdot
\end{equation*}
\begin{equation}\label{kristT}
  \frac{T_{n+1}^{\alpha,\beta}(x) T_{n}^{\alpha,\beta}(y) - T_{n+1}^{\alpha,\beta}(y) T_{n}^{\alpha,\beta}(x)}{x-y}.
\end{equation}

В работе [2, ШИИ, Асимпт.св-ва и вес.оценки Чеб.-Хана] показано, что для любых целых $\alpha,\beta \geq 0$ имеет место асимптотическая формула
\begin{equation}\label{asymptf}
 T_{n}^{\alpha,\beta}\left( \frac{N-1}{2}(1+t) \right) = P_{n}^{\alpha,\beta}(t)+r_{n,N}^{\alpha,\beta}(t),
\end{equation}
в которой $P_{n}^{\alpha,\beta}(t)$ -- классический полином Якоби, а для остаточного члена при $1 \leq n \leq a\sqrt{N}, (a>0)$, справедливы оценки
%\begin{equation}\label{asymptRest1}
%  \max\limits_{0\leq t \leq 1} |r_{n,N}^{\alpha,\beta}(t)| \leq c(\alpha,\beta,a)\,n^{\alpha+3/2}\,N^{-1/2},
%\end{equation}
%\begin{equation}\label{asymptRest1}
%  |r_{n,N}^{\alpha,\beta}(\cos{\theta})| \leq c(\alpha,\beta,a,c)\,n\,N^{-1/2}\,\theta^{-\alpha-1/2}, \quad \left( cn^{-1} \leq \theta \leq \pi/2\right).
%\end{equation}
\begin{equation}\label{asymptRest1}
  \max\limits_{0\leq t \leq 1} |r_{n,N}^{\alpha,\beta}(t)| \leq c(\alpha,\beta,a)\,n^{\alpha+1/2},
\end{equation}
\begin{equation}\label{asymptRest1}
  |r_{n,N}^{\alpha,\beta}(\cos{\theta})| \leq c(\alpha,\beta,a,c)\,\theta^{-\alpha-1/2}, \quad \left( cn^{-1} \leq \theta \leq \pi/2\right).
\end{equation}
В качестве следствия этой формулы, при тех же ограничениях на $n$ и $N$, получена весовая оценка
\begin{equation}\label{weightEst}
  \left| T_{n}^{\alpha,\beta}\left(\frac{N-1}{2}(1+t)\right)\right| \leq
    c(\alpha,\beta,a)\,n^{-1/2}\,
    \left[  \sqrt{1-t} + \frac1n \right]^{-\alpha-1/2}
    \left[  \sqrt{1+t} + \frac1n \right]^{-\beta-1/2},
\end{equation}
%\begin{equation}\label{weightEst}
%  \left| T_{n}^{\alpha,\beta}\left( \frac{N-1}{2}(1+\cos{\theta}) \right)\right| \leq
%  c(\alpha,\beta,a,c)\,
%  \left\{
%    \begin{aligned}
%    n^{\alpha+1/2}, \quad (0 \leq \theta \leq cn^{-1}),\\
%    \theta^{-\alpha-1/2}, \quad (cn^{-1} \leq \theta \leq \pi/2), \\
%    \end{aligned}
%  \right.
%\end{equation}
или иначе
%\begin{equation}\label{weightEst1}
%  \left| T_{n}^{\alpha,\beta}\left(\frac{N-1}{2}(1+t)\right)\right| \leq
%  c(\alpha,\beta,a,c)\,
%  \left\{
%    \begin{aligned}
%    n^{\beta+1/2}, \quad (-1 \leq \theta \leq -1+cn^{-2}),\\
%    (1+t)^{-\beta/2-1/4}, \quad (-1+cn^{-2} \leq \theta \leq 0), \\
%    (1-t)^{-\alpha/2-1/4}, \quad (0 \leq \theta \leq 1-cn^{-2}), \\
%    n^{\alpha+1/2}, \quad (1-cn^{-2} \leq \theta \leq 1).\\
%    \end{aligned}
%  \right.
%\end{equation}
\begin{equation}\label{weightEst1}
  \left| T_{n}^{\alpha,\beta}\left(\frac{N-1}{2}(1+t)\right)\right| \leq
  c(\alpha,\beta,a,c)\,
  \left\{
    \begin{aligned}
    n^{\beta}, \quad (-1 \leq \theta \leq -1+cn^{-2}),\\
    \left(n(1+t)^{\beta+1/2}\right)^{-1/2}, \quad (-1+cn^{-2} \leq \theta \leq 0), \\
    \left(n(1-t)^{\alpha+1/2}\right)^{-1/2}, \quad (0 \leq \theta \leq 1-cn^{-2}), \\
    n^{\alpha}, \quad (1-cn^{-2} \leq \theta \leq 1).\\
    \end{aligned}
  \right.
\end{equation}
%%%%%%%%%%%%%%%%%%%%%%%%%%%%%%%%%%%%%%%%%%
В работе [3, ШИИ, Об ограниченности...] показано, что оценки \eqref{weightEst} -- \eqref{weightEst1} сохраняются также для $T_{n}^{\alpha,\beta}\left(\frac{N-1}{2}(1+t)-j_1, N-j_2\right)$, где $j_1,j_2$ --- фиксированные целые числа.






\section*{ПРЕДЕЛЬНЫЕ РЯДЫ ПО ДИСКРЕТНЫМ ПОЛИНОМАМ ЧЕБЫШ\"ЕВА}

Как было отмечено выше, все перечисленные свойства справедливы при $\alpha, \beta > -1$. Если же взять $\alpha = \beta = -1$, то вес $\eqref{sms2eq4}$ обращается в бесконечность в точках $x=0$ и $x=N-1$, и суммы $\sum\limits_{x=0}^{N-1} f(x)\tau_{k}^{-1}(x)\mu^{-1}(x)$ теряют смысл.
%, а получаемая система полиномов $\left\{ \tau_n^{-1}(x) \right\}_{n=0}^{N-1}$ перестает быть ортогональной.
Однако, как показано в работе [4, ШТИ], можно рассмотреть дискретный ряд, получаемый в результате почленного предельного перехода в дискретном ряде \eqref{sms2eq1} при $\alpha,\beta \rightarrow -1$, т.е. $f(x) = \sum\limits_{k=0}^{N-1}\hat{f}^{-1}_k\tau_{k}^{-1}(x),\quad (x \in \Omega_N)$. %, где $\hat{f}^{-1}_k\tau_k^{-1}(x) = \lim_{\alpha \rightarrow -1} \hat{f}^{\alpha}_k\tau_k^{\alpha}(x)$.
%Не смотря на то, что суммы $\sum\limits_{x=0}^{N-1} f(x)\tau_k^{-1}(x)\mu^{-1}(x,N)$ невозможно вычислить (ввиду того что $\mu^{-1}(0)=\mu^{-1}(N-1)=\infty$), рассматривая выражения $\hat{f}^{-1}_k\tau_k^{-1}(x) = \lim_{\alpha \rightarrow -1} \hat{f}^{\alpha}_k\tau_k^{\alpha}(x)$ в целом, удается показать, что
Не смотря на то, что сами коэффициенты $f_k^{-1}$ не удается вычислить (ввиду того что $\mu^{-1}(0)=\mu^{-1}(N-1)=\infty$), рассматривая выражения $\hat{f}^{-1}_k\tau_{k}^{-1}(x) = \lim_{\alpha \rightarrow -1} \hat{f}^{\alpha}_k\tau_k^{\alpha}(x)$ в целом, удается показать, что
\begin{equation}
\label{limitR}
 f(x) = \sum\limits_{k=0}^{N-1}\hat{f}^{-1}_k  \tau_{k}^{-1}(x) =
a_N(f,x) + \frac{8x(N-x-1)}{N(N-1)} \sum\limits_{k=0}^{N-3} \hat{g}_k \tau_{k}^{1}(x-1,N-2),
\end{equation}
где
\begin{equation}
\label{af}
a_N(f,x) = \frac{f(N-1)+f(0)}{2} + \frac{f(N-1)-f(0)}{2}\left( \frac{2x}{N-1}-1\right),
\end{equation}
%\begin{equation}
%\label{bnN}
%b_{n,N} = \frac{8x(N-x-1)}{N(N-1)},
%\end{equation}
а коэффициенты
\begin{equation}
\label{gk}
\hat{g}_k = \hat{g}_k(N) = \frac{1}{N-2} \sum\limits_{j=1}^{N-2} g(j) \tau_{k}^{1}(j-1,N-2) ,
\end{equation}
получены для функции $g(x) = f(x) - a_N(f,x)$.

Ряд \eqref{limitR} называется \textit{предельным рядом } по полиномам Чебыш\"ева, ортогональным на равномерной сетке. Соответствующий предельный случай частичных сумм Фурье -- Чебыш\"ева тогда может быть записано
\begin{equation}
\label{limitS}
 S_{n,N}^{-1}(f,x) = \lim_{\alpha \rightarrow -1} S_{n,N}^{\alpha}(f,x) =
 a_N(f,x) + \frac{8x(N-x-1)}{N(N-1)} \sum\limits_{k=0}^{n-2} \hat{g}_k \tau_{k}^{1}(x-1,N-2).
\end{equation}







В работе [4, ШТИ] были исследованы аппроксимативные свойства частичных сумм предельного ряда и показано, что они обладают следующими тремя свойствами:
\begin{enumerate}
  \item $S_{n,N}^{-1}(f,x)$ совпадает с $f(x)$ в точках $x=0$ и $x=N-1$:%, т.е. $S_{n,N}^{-1}(f,0) = f(0)$, $S_{n,N}^{-1}(f,N-1)=f(N-1)$;
    \begin{equation}\label{limprop1}
        S_{n,N}^{-1}(f,0) = f(0), \quad S_{n,N}^{-1}(f,N-1)=f(N-1);
    \end{equation}
  \item $S_{n,N}^{-1}(f,x)$ представляет собой проектор на пространство $H_n$ всех алгебраических полиномов $P_n(x)$ степени, не выше $n$:%, т.е.   $S_{n,N}^{-1}(P_n,x) = P_n(x)$;
    \begin{equation}\label{limprop2}
        S_{n,N}^{-1}(P_n,x) = P_n(x);
    \end{equation}
  \item Подставляя в \eqref{limitS} выражение для \eqref{gk}, получим следующее представление для частичных сумм предельного ряда:
    %\begin{equation*}
%     S_{n,N}^{-1}(f,x) =
%     a_N(f,x) + \frac{8x(N-x-1)}{N(N-1)(N-2)} \sum\limits_{j=1}^{N-2} g(j) \sum\limits_{k=0}^{n-2} \tau_{k}^{1}(j-1,N-2) \tau_{k}^{1}(x-1,N-2).
%    \end{equation*}
    \begin{equation}
    \label{Sn_jay}
     S_{n,N}^{-1}(f,x) =
     a_N(f,x) + \frac{8x(N-x-1)}{N(N-1)(N-2)} \sum\limits_{j=1}^{N-2} g(j) J_{n-2,N-2}(x-1,j-1) ,
    \end{equation}
    где
    \begin{equation}\label{Jay}
      J_{l,N}(x,j) = \sum_{r=0}^{l} \tau^{1}_{r}(x) \tau^{1}_{r}(j).
    \end{equation}

  \item Если $0\le x\le N-3$, $n\le a\sqrt{N}$, то имеет место следующая оценка
    \begin{equation}\label{applimitR}
        |f(x)-S_{n,N}^{-1}(f,x)|\le c(a)E_{n,N}^{*}(f)\left[1+\ln\left(2+\frac{n}{N-1}\sqrt{x(N-3-x)}\right)\right],
    \end{equation}
    где
    $$E_{n,N}^{*}(f) = \inf_{p_n \in \hat{H}^n} \sup_{0\le x\le N-3} |f(x)-p_n(x)|$$
    -- наилучшее приближение функции $f(x)$ алгебраическими полиномами, которые в точках $0$ и $N-1$ совпадают со значениями самой функции $f(x)$.

    Здесь и далее через $c, c(a), c(a,b)$ и т.д. -- мы обозначаем положительные числа, зависящее только от указанных в скобках параметров и, вообще говоря, различные в разных местах.
\end{enumerate}












\section*{ВСПОМОГАТЕЛЬНЫЕ УТВЕРЖДЕНИЯ}

Далее нам понадобится модифицированный вариант полиномов Чебыш\"ева:
\begin{equation}\label{Qnn}
 Q_{n}^{\alpha,\beta}(t) = Q_{n}^{\alpha,\beta}(t,N) = T_{n}^{\alpha,\beta}\left( \frac{N-1}{2}(1+t) - 1, N-1 \right).
\end{equation}
для которого в работе [3, ШИИ, Об ограниченности...] показана справедливость оценок \eqref{weightEst} -- \eqref{weightEst1}, а также следующая рекуррентная формула
\begin{equation}\label{recurQ}
  \hat{A}^{\alpha,\beta}_{n,N} Q^{\alpha,\beta}_{n+1}(t) = \frac{N-1}{2}\,(t+\lambda^{\alpha,\beta}_{n,N}) Q^{\alpha,\beta}_{n}(t) - \hat{C}^{\alpha,\beta}_{n,N} Q^{\alpha,\beta}_{n-1}(t),
\end{equation}
где $\hat{A}^{\alpha,\beta}_{n,N}= A^{\alpha,\beta}_{n,N-1}$, $\hat{C}^{\alpha,\beta}_{n,N} = C^{\alpha,\beta}_{n,N-1}$,
\begin{equation*}
  \hat{A}^{\alpha,\beta}_{n,N} = \frac{(n+1)(n+\alpha+\beta+1)(N-n-2)}{(2n+\alpha+\beta+1)(2n+\alpha+\beta+2)},
\end{equation*}
\begin{equation*}
  \hat{C}^{\alpha,\beta}_{n,N} = \frac{(n+\alpha)(n+\beta)(N+n+\alpha+\beta-1)}{(2n+\alpha+\beta)(2n+\alpha+\beta+1)},
\end{equation*}
\begin{equation*}
  \lambda^{\alpha,\beta}_{n,N} = \frac{\alpha^2-\beta^2}{(2n+\alpha+\beta)(2n+\alpha+\beta+2)}+\frac{q(n,\alpha,\beta)}{N-1}, \quad \text{где } |q(n,\alpha,\beta)| \leq c(\alpha,\beta).
\end{equation*}
Рассмотрим отдельно два частных случая:

%1. $\alpha=\beta=1$:
%\begin{equation*}
%  \hat{A}^{1,1}_{n,N} Q^{1,1}_{n+1}(t) = \frac{N-1}{2}\,(t+\lambda^{1,1}_{n,N}) Q^{1,1}_{n}(t) - \hat{C}^{1,1}_{n,N} Q^{1,1}_{n-1}(t),
%\end{equation*}
%\begin{equation*}
%  \hat{A}^{1,1}_{n,N} = \frac{(n+1)(n+3)(N-n)}{(2n+3)(2n+4)},
%\end{equation*}
%\begin{equation*}
%  \hat{C}^{1,1}_{n,N} = \frac{(n+1)(N+n+1)}{2(2n+3)},
%\end{equation*}
%\begin{equation*}
%  \lambda^{1,1}_{n,N} = \frac{q(n,1,1)}{N-1}, \quad \text{где } |q(n,1,1)| \leq c,
%\end{equation*}
%
%%\begin{equation*}
%%   Q^{1,1}_{n+1}(t) = \frac{(2n+3)(2n+4)}{(n+1)(n+3)(N-n)} \frac{N-1}{2}\,\left(t-\frac{q(n,1,1)}{N-1}\right) Q^{1,1}_{n}(t) -
%%   \frac{(2n+3)(2n+4)}{(n+1)(n+3)(N-n)} \frac{(n+1)(N+n+1)}{2(2n+3)} Q^{1,1}_{n-1}(t),
%%\end{equation*}
%
%\begin{equation*}
%   Q^{1,1}_{n+1}(t) = \frac{(2n+3)(n+2)}{(n+1)(n+3)} \frac{N-1}{N-n}\,\left(t+\frac{q(n,1,1)}{N-1}\right) Q^{1,1}_{n}(t) -
%   \frac{n+2}{n+3} \frac{N+n+1}{N-n} Q^{1,1}_{n-1}(t),.
%\end{equation*}


1. $\alpha=1, \beta=2$:
%\begin{equation*}
%  \hat{A}^{1,2}_{n,N} Q^{1,2}_{n+1}(t) = \frac{N-1}{2}\,(t+\lambda^{1,2}_{n,N}) Q^{1,2}_{n}(t) - \hat{C}^{1,2}_{n,N} Q^{1,2}_{n-1}(t),
%\end{equation*}
%\begin{equation*}
%  \hat{A}^{1,2}_{n,N} = \frac{(n+1)(n+4)(N-n-2)}{2(n+2)(2n+5)},
%\end{equation*}
%\begin{equation*}
%  \hat{C}^{1,2}_{n,N} = \frac{(n+1)(N+n+2)}{2(2n+3)},
%\end{equation*}
%\begin{equation*}
%  \lambda^{1,2}_{n,N} = \frac{-3}{(2n+3)(2n+5)}+\frac{q(n,1,2)}{N-1}, \quad \text{где } |q(n,1,2)| \leq c,
%\end{equation*}

%\begin{equation*}
%   \frac{(n+1)(n+4)(N-n)}{2(n+2)(2n+5)} Q^{1,2}_{n+1}(t) = \frac{N-1}{2}\,(t+\frac{3}{(2n+3)(2n+5)}+\frac{q(n,1,2)}{N-1}) Q^{1,2}_{n}(t) - \frac{(n+1)(n+2)(N+n+2)}{2(2n+3)(n+2)} Q^{1,2}_{n-1}(t),
%\end{equation*}

%\begin{equation*}
%    \frac{(n+1)(n+4)(N-n)}{2(n+2)(2n+5)} \, Q^{1,2}_{n+1}(t) =
%    \frac{N-1}{2}\,
%    \left(t-
%    \frac{3}{(2n+3)(2n+5)}+\frac{q(n,1,2)}{N-1}
%    \right)\,
%    Q^{1,2}_{n}(t) -
%\end{equation*}
%\begin{equation*}
%    \frac{(n+1)(N+n+2)}{2(2n+3)}\,
%    Q^{1,2}_{n-1}(t),
%\end{equation*}

\begin{equation*}
    \frac{(n+1)(n+4)(N-n-2)}{(n+2)(2n+5)(N-1)} \, Q^{1,2}_{n+1}(t) =
    \left(t-
    \frac{3}{(2n+3)(2n+5)}+\frac{q(n,1,2)}{N-1}
    \right)\,
    Q^{1,2}_{n}(t) -
\end{equation*}
\begin{equation}\label{34}
    \frac{(n+1)(N+n+2)}{(2n+3)(N-1)}\,
    Q^{1,2}_{n-1}(t).
\end{equation}


%\begin{equation*}
%    Q^{1,2}_{n+1}(t) = \frac{(n+2)(2n+5)}{(n+1)(n+4)}\frac{N-1}{N-n}\,
%    \left(t+\frac{3}{(2n+3)(2n+5)}-\frac{q(n,1,2)}{N-1}\right)
%    Q^{1,2}_{n}(t) -
%\end{equation*}
%\begin{equation*}
%    \frac{(n+2)(2n+5)}{(n+4)(2n+3)}
%    \frac{N+n+2}{N-n}
%    Q^{1,2}_{n-1}(t),
%\end{equation*}


2. $\alpha=2, \beta=1$:
%\begin{equation*}
%  \hat{A}^{2,1}_{n,N} Q^{2,1}_{n+1}(t) = \frac{N-1}{2}\,(t+\lambda^{2,1}_{n,N}) Q^{2,1}_{n}(t) - \hat{C}^{2,1}_{n,n} Q^{2,1}_{n-1}(t),
%\end{equation*}
%\begin{equation*}
%  \hat{A}^{2,1}_{n,N} = \frac{(n+1)(n+4)(N-n-2)}{2(n+2)(2n+5)},
%\end{equation*}
%\begin{equation*}
%  \hat{C}^{2,1}_{n,N} = \frac{(n+1)(N+n+2)}{2(2n+3)},
%\end{equation*}
%\begin{equation*}
%  \lambda^{2,1}_{n,N} = \frac{3}{(2n+3)(2n+5)}+\frac{q(n,2,1)}{N-1}, \quad \text{где } |q(n,2,1)| \leq c,
%\end{equation*}

%\begin{equation*}
%  Q^{2,1}_{n+1}(t) = \frac{2(n+2)(2n+5)}{(n+1)(n+4)(N-n)} \frac{N-1}{2}\,
%  \left(t-\frac{3}{(2n+3)(2n+5)}+\frac{q(n,2,1)}{N-1}\right) Q^{2,1}_{n}(t) -
%  \frac{2(n+2)(2n+5)}{(n+1)(n+4)(N-n)} \frac{(n+1)(N+n+2)}{2(2n+3)} Q^{2,1}_{n-1}(t),
%\end{equation*}
%\begin{equation*}
%  Q^{2,1}_{n+1}(t) = \frac{(n+2)(2n+5)}{(n+1)(n+4)} \frac{N-1}{N-n}\,
%  \left(t-\frac{3}{(2n+3)(2n+5)}-\frac{q(n,2,1)}{N-1}\right) Q^{2,1}_{n}(t) -
%\end{equation*}
%\begin{equation*}
%  \frac{(n+2)(2n+5)}{(n+4)(2n+3)}
%  \frac{N+n+2}{N-n}
%  Q^{2,1}_{n-1}(t),
%\end{equation*}

\begin{equation*}
  \frac{(n+1)(n+4)(N-n-2)}{(n+2)(2n+5)(N-1)} Q^{2,1}_{n+1}(t) =
  \left(t+\frac{3}{(2n+3)(2n+5)}+\frac{q(n,2,1)}{N-1}\right) Q^{2,1}_{n}(t)
\end{equation*}
\begin{equation}\label{35}
  - \frac{(n+1)(N+n+2)}{(2n+3)(N-1)} \, Q^{2,1}_{n-1}(t).
\end{equation}





%%%%%%%%%%%%%%%%%%%%%%%%%%%%%%%%%%%%%%%%%%%%%%%%%%%%%
%%%%%%%%%%%%%%%%%%%%%%%%%%%%%%%%%%%%%%%%%%%%%%%%%%%%%
\begin{lemma}\label{lemma0}
Для всех $t,v \in [-1,1]$ имеет место равенство
\begin{equation*}
  T_{l+1}^1(x) T_{l}^1(j) - T_{l+1}^1(j) T_{l}^1(x) =
    \frac{(l+2)}{2(l+1)} \,
  \left[
    (1+t)(1-v)\,Q^{1,2}_{l}(t)\,Q^{2,1}_{l}\left(v+\frac{2}{N-1}\right)
  \right.
\end{equation*}
\begin{equation}\label{lem0eq}
  \left.
  - (1+v)(1-t)\,Q^{1,2}_{l}(v)\,Q^{2,1}_{l}\left(t+\frac{2}{N-1}\right)\right],
  \quad (l=0, \ldots, N-2).
\end{equation}

\end{lemma}

% \begin{proof}
% Из рекуррентных формул \eqref{recur1}-\eqref{recur2} имеем:
% %====================
% %\begin{equation*}
% %  (n+1)T^{\alpha,\beta}_{n+1}(x,N) + (n+\beta+1)T^{\alpha,\beta}_{n}(x,N) = \frac{2n+\alpha+\beta+2}{N-1}\,x\,T^{\alpha,\beta+1}_{n}(x-1,N-1)
% %\end{equation*}
% %\begin{equation*}
% %  (n+\alpha+1)T^{\alpha,\beta}_{n}(x,N) - (n+1)T^{\alpha,\beta}_{n+1}(x,N) = \frac{2n+\alpha+\beta+2}{N-1}\,(N-1-x)\,T^{\alpha+1,\beta}_{n}(x,N-1)
% %\end{equation*}
% %====================
% \begin{equation*}
  % (l+1)T^{1,1}_{l+1}(x,N) + (l+2)T^{1,1}_{l}(x,N) = \frac{2l+4}{N-1}\,x\,T^{1,2}_{l}(x-1,N-1),
% \end{equation*}
% \begin{equation*}
% (l+2)T^{1,1}_{l}(x,N) - (l+1)T^{1,1}_{l+1}(x,N) = \frac{2l+4}{N-1}\,(N-1-x)\,T^{2,1}_{l}(x,N-1).
% \end{equation*}
% Складывая эти равенства получаем:
% %\begin{equation*}
% %  (n+1)T^{1,1}_{n+1}(x,N) + (n+2)T^{1,1}_{n}(x,N) + (n+2)T^{1,1}_{n}(x,N) - (n+1)T^{1,1}_{n+1}(x,N)
% %\end{equation*}
% %\begin{equation*}
% %   = \frac{2n+4}{N-1}\,x\,T^{1,2}_{n}(x-1,N-1) + \frac{2n+4}{N-1}\,(N-1-x)\,T^{2,1}_{n}(x,N-1),
% %\end{equation*}
% %\begin{equation*}
% %  2(n+2)T^{1,1}_{n}(x,N) =
% %   \frac{2n+4}{N-1}\, \left[ x\,T^{1,2}_{n}(x-1,N-1) + (N-1-x)\,T^{2,1}_{n}(x,N-1) \right],
% %\end{equation*}
% %\begin{equation*}
% %  (N-1)T^{1,1}_{n}(x,N) =
% %   x\,T^{1,2}_{n}(x-1,N-1) + (N-1-x)\,T^{2,1}_{n}(x,N-1),
% %\end{equation*}
% \begin{equation*}
  % T^{1,1}_{l}(x,N) = \frac{1}{N-1}
   % \left[ x\,T^{1,2}_{l}(x-1,N-1) + (N-1-x)\,T^{2,1}_{l}(x,N-1) \right],
% \end{equation*}
% а вычитая из первого второе:
% %\begin{equation*}
% %  (n+1)T^{1,1}_{n+1}(x,N) + (n+2)T^{1,1}_{n}(x,N) - (n+2)T^{1,1}_{n}(x,N) + (n+1)T^{1,1}_{n+1}(x,N) =
% %\end{equation*}
% %\begin{equation*}
% %   \frac{2n+4}{N-1}\,x\,T^{1,2}_{n}(x-1,N-1) - \frac{2n+4}{N-1}\,(N-1-x)\,T^{2,1}_{n}(x,N-1),
% %\end{equation*}
% %\begin{equation*}
% %  (n+1)T^{1,1}_{n+1}(x,N) + (n+1)T^{1,1}_{n+1}(x,N) =
% %\end{equation*}
% %\begin{equation*}
% %   \frac{2n+4}{N-1}\,\left[ x\,T^{1,2}_{n}(x-1,N-1) - (N-1-x)\,T^{2,1}_{n}(x,N-1) \right],
% %\end{equation*}
% %\begin{equation*}
% %  \frac{n+1}{n+2} (N-1) T^{1,1}_{n+1}(x,N)=
% %   x\,T^{1,2}_{n}(x-1,N-1) - (N-1-x)\,T^{2,1}_{n}(x,N-1),
% %\end{equation*}
% \begin{equation*}
  % T^{1,1}_{l+1}(x,N)=
  % \frac{l+2}{(l+1)(N-1)}
   % \left[ x\,T^{1,2}_{l}(x-1,N-1) - (N-1-x)\,T^{2,1}_{l}(x,N-1) \right],
% \end{equation*}
% Тогда можем записать
% %\begin{equation*}
% %  T_{l+1}^1(x) T_{l}^1(j) - T_{l+1}^1(j) T_{l}^1(x) =
% %   \frac{l+2}{(l+1)(N-1)^2}
% %\end{equation*}
% %\begin{equation*}
% %    \left[
% %        \left( x\,T^{1,2}_{l}(x-1,N-1) - (N-1-x)\,T^{2,1}_{l}(x,N-1) \right)
% %        \left( j\,T^{1,2}_{l}(j-1,N-1) + (N-1-j)\,T^{2,1}_{l}(j,N-1) \right)
% %        -
% %    \right.
% %\end{equation*}
% %\begin{equation*}
% %\left.
% %  \left( j\,T^{1,2}_{l}(j-1,N-1) - (N-1-j)\,T^{2,1}_{l}(j,N-1) \right)
% %  \left( x\,T^{1,2}_{l}(x-1,N-1) + (N-1-x)\,T^{2,1}_{l}(x,N-1) \right)
% %\right] =
% %\end{equation*}
% \begin{equation*}
  % T_{l+1}^1(x) T_{l}^1(j) - T_{l+1}^1(j) T_{l}^1(x) =
   % \frac{l+2}{(l+1)(N-1)^2}
   % \left[
        % \left( x\,T^{1,2}_{l}(x-1,N-1) -
    % \right.\right.
% \end{equation*}
% \begin{equation*}
    % \left.
        % \left. (N-1-x)\,T^{2,1}_{l}(x,N-1) \right)
        % \left( j\,T^{1,2}_{l}(j-1,N-1) + (N-1-j)\,T^{2,1}_{l}(j,N-1) \right)
        % -
    % \right.
% \end{equation*}
% \begin{equation*}
% \left.
  % \left( j\,T^{1,2}_{l}(j-1,N-1) - (N-1-j)\,T^{2,1}_{l}(j,N-1) \right)
  % \left( x\,T^{1,2}_{l}(x-1,N-1) +  \right.
% \right.
% \end{equation*}
% \begin{equation*}
% \left.\left.
% + (N-1-x)\,T^{2,1}_{l}(x,N-1) \right)
% \right] =
% \end{equation*}
% %%%%%%%%%%%%%%%%%%
% \begin{equation*}
% \frac{2(l+2)}{(l+1)(N-1)^2} \left[
    % x(N-1-j)\,T^{1,2}_{l}(x-1,N-1)\,T^{2,1}_{l}(j,N-1)-
 % \right.
% \end{equation*}
% \begin{equation*}
% \left.
    % j(N-1-x)\,T^{1,2}_{l}(j-1,N-1)\,T^{2,1}_{l}(x,N-1)
% \right].
% \end{equation*}
% %\begin{equation*}
% %  xj\,T^{1,2}_{l}(x-1,N-1)\,T^{1,2}_{l}(j-1,N-1)
% %- j(N-1-x)\,T^{2,1}_{l}(x,N-1) \,T^{1,2}_{l}(j-1,N-1)
% %\end{equation*}
% %\begin{equation*}
% %+ x(N-1-j)\,T^{1,2}_{l}(x-1,N-1)\,T^{2,1}_{l}(j,N-1)
% %- (N-1-x)(N-1-j)\,T^{2,1}_{l}(x,N-1) \,T^{2,1}_{l}(j,N-1)
% %\end{equation*}
% %\begin{equation*}
% %  - xj\,T^{1,2}_{l}(j-1,N-1)\,T^{1,2}_{l}(x-1,N-1)
% %  + x(N-1-j)\,T^{2,1}_{l}(j,N-1) \,T^{1,2}_{l}(x-1,N-1)
% %  \end{equation*}
% %\begin{equation*}
% %  - j(N-1-x)\,T^{1,2}_{l}(j-1,N-1)\,T^{2,1}_{l}(x,N-1)
% %  + (N-1-j)(N-1-x)\,T^{2,1}_{l}(j,N-1)\,T^{2,1}_{l}(x,N-1)
% %\end{equation*}
% %
% %=============
% %
% %\begin{equation*}
% % 2x(N-1-j)\,T^{1,2}_{l}(x-1,N-1)\,T^{2,1}_{l}(j,N-1)
% %\end{equation*}
% %\begin{equation*}
% %  - 2j(N-1-x)\,T^{1,2}_{l}(j-1,N-1)\,T^{2,1}_{l}(x,N-1)
% %\end{equation*}




% Проведя замену переменных $\{x = \frac{N-1}{2}\,(1+t), j = \frac{N-1}{2}\,(1+v)\}$ приходим к справедливости утверждения Леммы.
% \end{proof}
















































\begin{lemma}\label{lemma1}
Пусть $k=0, \ldots, N-2$, тогда для любых $t,v \in [-1,1]$, тогда имеет место
\begin{equation*}
    \left(
    t-v
    \right)\,
   \sum_{l=0}^{k}
    \frac{(N-2)^{[l]}}{(N+l+2)^{[l]}} \, \frac{(l+2)(l+3)}{(l+1)}\,
    Q^{1,2}_{l}(t) \,Q^{2,1}_{l}\left(v +\frac{2}{N-1} \right)
    =
\end{equation*}
\begin{equation*}
   \frac{(N-2)^{[k+1]}}{(N+k+2)^{[k]}}  \,
   \frac{(k+3)(k+4)}{(2k+5)(N-1)}
   \left[
        Q^{1,2}_{k+1}(t) \,Q^{2,1}_{k}\left(v +\frac{2}{N-1} \right) -  Q^{1,2}_{k}(t) \, Q^{2,1}_{k+1} \left(v +\frac{2}{N-1} \right)
   \right]
\end{equation*}
\begin{equation}\label{lem1eq}
   +
   \sum_{l=0}^{k} \lambda_{l,N} \,
    \frac{(N-2)^{[l]}}{(N+l+2)^{[l]}} \, \frac{(l+2)(l+3)}{(l+1)}\,
    Q^{1,2}_{l}(t) \,Q^{2,1}_{l}\left(v +\frac{2}{N-1} \right),
\end{equation}
где $\lambda_{l,N} = \frac{6}{(2l+3)(2l+5)}+\frac{q(l)}{N-1}$, $|q(l)|<c$.
\end{lemma}

% \begin{proof}
% Умножив \eqref{34} на $Q^{2,1}_{n}(y)$ вычтем из него \eqref{35}, умноженное на $Q^{1,2}_{n}(t)$, получим

% %\begin{equation*}
% %    \frac{(n+1)(n+4)(N-n-2)}{(n+2)(2n+5)(N-1)} \, Q^{1,2}_{n+1}(t) \,Q^{2,1}_{n}(v)
% %    =
% %    \left(t-
% %    \frac{3}{(2n+3)(2n+5)}+\frac{q(n,1,2)}{N-1}
% %    \right)\,
% %    Q^{1,2}_{n}(t) \,Q^{2,1}_{n}(v) -
% %\end{equation*}
% %\begin{equation}\label{34}
% %    \frac{(n+1)(N+n+2)}{(2n+3)(N-1)}\,
% %    Q^{1,2}_{n-1}(t) \,Q^{2,1}_{n}(v).
% %\end{equation}
% %
% %------------
% %
% %\begin{equation*}
% %  \frac{(n+1)(n+4)(N-n-2)}{(n+2)(2n+5)(N-1)} Q^{2,1}_{n+1}(v)\,Q^{1,2}_{n}(t) =
% %  \left(v+\frac{3}{(2n+3)(2n+5)}+\frac{q(n,2,1)}{N-1}\right) Q^{2,1}_{n}(v)\,Q^{1,2}_{n}(t)
% %\end{equation*}
% %\begin{equation}\label{35}
% %  - \frac{(n+1)(N+n+2)}{(2n+3)(N-1)} \, Q^{2,1}_{n-1}(v)\,Q^{1,2}_{n}(t).
% %\end{equation}
% %
% %=============

% %
% %\begin{equation*}
% %    \frac{(n+1)(n+4)(N-n-2)}{(n+2)(2n+5)(N-1)} \left[ \, Q^{1,2}_{n+1}(t) \,Q^{2,1}_{n}(v)
% %    -  Q^{1,2}_{n}(t) \, Q^{2,1}_{n+1}(v) \right]
% %    =
% %\end{equation*}
% %\begin{equation*}
% %    \left(
% %    t-v - \frac{6}{(2n+3)(2n+5)}+\frac{q(n,1,2)- q(n,2,1)}{N-1}
% %    \right)\,
% %    Q^{1,2}_{n}(t) \,Q^{2,1}_{n}(v)
% %\end{equation*}
% %\begin{equation*}
% %    + \frac{(n+1)(N+n+2)}{(2n+3)(N-1)}\,\left[
% %    Q^{1,2}_{n}(t)\, Q^{2,1}_{n-1}(v) - Q^{1,2}_{n-1}(t) \,Q^{2,1}_{n}(v)
% %    \right].
% %\end{equation*}
% %Обозначим $\lambda = - \frac{6}{(2n+3)(2n+5)}+\frac{q(n,1,2)- q(n,2,1)}{N-1}$, тогда
% \begin{equation*}
    % \frac{(n+1)(n+4)(N-n-2)}{(n+2)(2n+5)(N-1)} \left[ \, Q^{1,2}_{n+1}(t) \,Q^{2,1}_{n}(y)
    % -  Q^{1,2}_{n}(t) \, Q^{2,1}_{n+1}(y) \right]
    % =
% \end{equation*}
% \begin{equation*}
    % \left(
    % t-y + \lambda_{n,N}^{1,2}-\lambda_{n,N}^{2,1}
    % \right)\,
    % Q^{1,2}_{n}(t) \,Q^{2,1}_{n}(y)
% \end{equation*}
% \begin{equation*}
    % + \frac{(n+1)(N+n+2)}{(2n+3)(N-1)}\,\left[
    % Q^{1,2}_{n}(t)\, Q^{2,1}_{n-1}(y) - Q^{1,2}_{n-1}(t) \,Q^{2,1}_{n}(y)
    % \right].
% \end{equation*}
% Умножим обе части на $\frac{(N-2)^{[n]}}{(N+n+2)^{[n]}} \, \frac{(n+2)(n+3)}{(n+1)}$, получаем:

% %\begin{equation*}
% %   \frac{(N-2)^{[n]}}{(N+n+2)^{[n]}}  \,
% %   \frac{(n+3)(n+4)(N-n-2)}{(2n+5)(N-1)}
% %   \left[
% %        Q^{1,2}_{n+1}(t) \,Q^{2,1}_{n}(v) -  Q^{1,2}_{n}(t) \, Q^{2,1}_{n+1}(v)
% %   \right] =
% %\end{equation*}
% %\begin{equation*}
% %    \frac{(N-2)^{[n]}}{(N+n+2)^{[n]}} \, \frac{(n+2)(n+3)}{(n+1)}\,
% %    \left(
% %    t-v + \lambda
% %    \right)\,
% %    Q^{1,2}_{n}(t) \,Q^{2,1}_{n}(v)
% %\end{equation*}
% %\begin{equation*}
% %    +
% %    \frac{(N-2)^{[n]}}{(N+n+2)^{[n]}} \,
% %    \frac{(n+2)(n+3)(N+n+2)}{(2n+3)(N-1)}\,\left[
% %    Q^{1,2}_{n}(t)\, Q^{2,1}_{n-1}(v) - Q^{1,2}_{n-1}(t) \,Q^{2,1}_{n}(v)
% %    \right],
% %\end{equation*}



% \begin{equation*}
   % \frac{(N-2)^{[n+1]}}{(N+n+2)^{[n]}}  \,
   % \frac{(n+3)(n+4)}{(2n+5)(N-1)}
   % \left[
        % Q^{1,2}_{n+1}(t) \,Q^{2,1}_{n}(y) -  Q^{1,2}_{n}(t) \, Q^{2,1}_{n+1}(y)
   % \right] =
% \end{equation*}
% \begin{equation*}
    % \frac{(N-2)^{[n]}}{(N+n+2)^{[n]}} \, \frac{(n+2)(n+3)}{(n+1)}\,
    % \left(
    % t-y + \lambda_{n,N}^{1,2}-\lambda_{n,N}^{2,1}
    % \right)\,
    % Q^{1,2}_{n}(t) \,Q^{2,1}_{n}(y)
% \end{equation*}
% \begin{equation}\label{qsumm}
    % +
    % \frac{(N-2)^{[n]}}{(N+n+1)^{[n-1]}} \,
    % \frac{(n+2)(n+3)}{(2n+3)(N-1)}\,\left[
    % Q^{1,2}_{n}(t)\, Q^{2,1}_{n-1}(y) - Q^{1,2}_{n-1}(t) \,Q^{2,1}_{n}(y)
    % \right].
% \end{equation}
% Если принять $a^{[-1]}=1$, $Q^{\alpha,\beta}_{-1}(t)=0$, то эта формула сохранят свою справедливость и при $n=0$.
% Просуммируем все равенства вида \eqref{qsumm} для $n=l, (l = 0,\ldots,k),$ и выведем:

% \begin{equation*}
   % \sum_{l=0}^{k}
   % \frac{(N-2)^{[l+1]}}{(N+l+2)^{[l]}}  \,
   % \frac{(l+3)(l+4)}{(2l+5)(N-1)}
   % \left[
        % Q^{1,2}_{l+1}(t) \,Q^{2,1}_{l}(y) -  Q^{1,2}_{l}(t) \, Q^{2,1}_{l+1}(y)
   % \right] =
% \end{equation*}
% \begin{equation*}
   % \sum_{l=0}^{k}
    % \frac{(N-2)^{[l]}}{(N+l+2)^{[l]}} \, \frac{(l+2)(l+3)}{(l+1)}\,
    % \left(
    % t-y + \lambda_{l,N}^{1,2}-\lambda_{l,N}^{2,1}
    % \right)\,
    % Q^{1,2}_{l}(t) \,Q^{2,1}_{l}(y) +
% \end{equation*}
% \begin{equation*}
    % \sum_{l=0}^{k}
    % \frac{(N-2)^{[l]}}{(N+l+1)^{[l-1]}} \,
    % \frac{(l+2)(l+3)}{(2l+3)(N-1)}\,\left[
    % Q^{1,2}_{l}(t)\, Q^{2,1}_{l-1}(y) - Q^{1,2}_{l-1}(t) \,Q^{2,1}_{l}(y)
    % \right].
% \end{equation*}
% Сгруппировав первую и последнюю суммы в равенстве, а также поменяв порядок суммирования получаем:
% %\begin{equation*}
% %   \sum_{l=1}^{k+1}
% %   \frac{(N-2)^{[l]}}{(N+l+1)^{[l-1]}}  \,
% %   \frac{(l+2)(l+3)}{(2l+3)(N-1)}
% %   \left[
% %        Q^{1,2}_{l}(t) \,Q^{2,1}_{l-1}(v) -  Q^{1,2}_{l-1}(t) \, Q^{2,1}_{l}(v)
% %   \right] =
% %\end{equation*}
% %\begin{equation*}
% %   \sum_{l=0}^{k}
% %    \frac{(N-2)^{[l]}}{(N+l+2)^{[l]}} \, \frac{(l+2)(l+3)}{(l+1)}\,
% %    \left(
% %    t-v + \lambda
% %    \right)\,
% %    Q^{1,2}_{l}(t) \,Q^{2,1}_{l}(v)
% %\end{equation*}
% %\begin{equation*}
% %+
% %    \sum_{l=0}^{k}
% %    \frac{(N-2)^{[l]}}{(N+l+1)^{[l-1]}} \,
% %    \frac{(l+2)(l+3)}{(2l+3)(N-1)}\,\left[
% %    Q^{1,2}_{l}(t)\, Q^{2,1}_{l-1}(v) - Q^{1,2}_{l-1}(t) \,Q^{2,1}_{l}(v)
% %    \right].
% %\end{equation*}
% \begin{equation*}
   % \sum_{l=1}^{k+1}
   % \frac{(N-2)^{[l]}}{(N+l+1)^{[l-1]}}  \,
   % \frac{(l+2)(l+3)}{(2l+3)(N-1)}
   % \left[
        % Q^{1,2}_{l}(t) \,Q^{2,1}_{l-1}(y) -  Q^{1,2}_{l-1}(t) \, Q^{2,1}_{l}(y)
   % \right] -
% \end{equation*}
% \begin{equation*}
    % \sum_{l=0}^{k}
    % \frac{(N-2)^{[l]}}{(N+l+1)^{[l-1]}} \,
    % \frac{(l+2)(l+3)}{(2l+3)(N-1)}\,\left[
    % Q^{1,2}_{l}(t)\, Q^{2,1}_{l-1}(y) - Q^{1,2}_{l-1}(t) \,Q^{2,1}_{l}(y)
    % \right]
    % =
% \end{equation*}
% \begin{equation*}
   % \sum_{l=0}^{k}
    % \frac{(N-2)^{[l]}}{(N+l+2)^{[l]}} \, \frac{(l+2)(l+3)}{(l+1)}\,
    % \left(
    % t-y + \lambda_{l,N}^{1,2}-\lambda_{l,N}^{2,1}
    % \right)\,
    % Q^{1,2}_{l}(t) \,Q^{2,1}_{l}(y),
% \end{equation*}
% откуда
% %\begin{equation*}
% %   \frac{(N-2)^{[k+1]}}{(N+k+2)^{[k]}}  \,
% %   \frac{(k+3)(k+4)}{(2k+5)(N-1)}
% %   \left[
% %        Q^{1,2}_{k+1}(t) \,Q^{2,1}_{k}(v) -  Q^{1,2}_{k}(t) \, Q^{2,1}_{k+1}(v)
% %   \right]
% %\end{equation*}
% %%\begin{equation*}
% %%-
% %%    \frac{(N-2)^{[0]}}{(N+1)^{[-1]}} \,
% %%    \frac{(2)(3)}{(3)(N-1)}\,\left[
% %%    Q^{1,2}_{0}(t)\, Q^{2,1}_{-1}(v) - Q^{1,2}_{-1}(t) \,Q^{2,1}_{0}(v)
% %%    \right]
% %%\end{equation*}
% %\begin{equation*}
% %=
% %   \sum_{l=0}^{k}
% %    \frac{(N-2)^{[l]}}{(N+l+2)^{[l]}} \, \frac{(l+2)(l+3)}{(l+1)}\,
% %    \left(
% %    t-v + \lambda
% %    \right)\,
% %    Q^{1,2}_{l}(t) \,Q^{2,1}_{l}(v),
% %\end{equation*}
% %и далее
% \begin{equation*}
    % \left(
    % t-y
    % \right)\,
   % \sum_{l=0}^{k}
    % \frac{(N-2)^{[l]}}{(N+l+2)^{[l]}} \, \frac{(l+2)(l+3)}{(l+1)}\,
    % Q^{1,2}_{l}(t) \,Q^{2,1}_{l}(y)
    % =
% \end{equation*}
% \begin{equation*}
   % \frac{(N-2)^{[k+1]}}{(N+k+2)^{[k]}}  \,
   % \frac{(k+3)(k+4)}{(2k+5)(N-1)}
   % \left[
        % Q^{1,2}_{k+1}(t) \,Q^{2,1}_{k}(y) -  Q^{1,2}_{k}(t) \, Q^{2,1}_{k+1}(y)
   % \right]
% \end{equation*}
% \begin{equation*}
   % +
   % \sum_{l=0}^{k} \left( \lambda_{l,N}^{2,1} - \lambda_{l,N}^{1,2} \right)\,
    % \frac{(N-2)^{[l]}}{(N+l+2)^{[l]}} \, \frac{(l+2)(l+3)}{(l+1)}\,
    % Q^{1,2}_{l}(t) \,Q^{2,1}_{l}(y).
% \end{equation*}
% Подставив теперь в это равенство $y = v + \frac{2}{N-1}$, сразу же приходим к справедливости утверждения Леммы.
% \end{proof}



\begin{lemma}\label{lemma2}
Для любых $t \in [0, 1]$ и $v \in [-1,1]$ при $k=0, \ldots, N-2$ справедлива оценка
\begin{equation*}
\frac{(N-2)^{[k+1]}}{(N+k+2)^{[k]}}  \,
   \frac{(k+3)(k+4)}{(2k+5)(N-1)}
   \left|
        Q^{1,2}_{k+1}(t) \,Q^{2,1}_{k}\left(v +\frac{2}{N-1} \right) -
        \right.
\end{equation*}
\begin{equation*}
\left.
        Q^{1,2}_{k}(t) \, Q^{2,1}_{k+1} \left(v +\frac{2}{N-1} \right)
   \right|
   \leq
\end{equation*}
\begin{equation}\label{lem2eq}
 c(a) \, (1-t)^{-1/4} (1-v)^{-3/4}(1+v)^{-3/4} \left[
    (1-t)^{-1/2} +
    (1-v)^{-1/2}
    \right].
\end{equation}
\end{lemma}
% \begin{proof}
% Из \eqref{recur2} и \eqref{Qnn} можно получить рекуррентное соотношение:
% \begin{equation*}
   % Q^{\alpha,\beta}_{n+1}(t) =
   % \frac{n+\alpha+1}{n+1} Q^{\alpha,\beta}_{n}(t)
   % -
% \end{equation*}
% \begin{equation*}
   % \frac{(2n+\alpha+\beta+2)(N-1)}{2(n+1)(N-2)}\,(1-t)\,T^{\alpha+1,\beta}_{n}\left(\frac{N-1}{2}(1+t)-1,N-2\right).
% \end{equation*}
% %\begin{equation*}
% %   Q^{1,2}_{n+1}(t) =
% %   \frac{n+2}{n+1} Q^{1,2}_{n}(t)
% %   -
% %   \frac{(2n+5)(N-1)}{2(n+1)(N-2)}\,(1-t)\,T^{2,2}_{n}\left(\frac{N-1}{2}(1+t)-1,N-2\right),
% %\end{equation*}
% %
% %\begin{equation*}
% %   Q^{2,1}_{n+1}(t) =
% %   \frac{n+3}{n+1} Q^{2,1}_{n}(t)-
% %   \frac{(2n+5)(N-1)}{2(n+1)(N-2)}\,(1-t)\,T^{3,1}_{n}\left(\frac{N-1}{2}(1+t)-1,N-2\right),
% %\end{equation*}
% Используя случаи $\alpha=1,\beta=2$ и $\alpha=2,\beta=1$ выведем
% \begin{equation*}
   % Q^{1,2}_{n+1}(t)Q^{2,1}_{n}\left(v + \frac{2}{N-1}\right) -
   % Q^{1,2}_{n}(t)Q^{2,1}_{n+1}\left(v + \frac{2}{N-1}\right)
   % =
% \end{equation*}
% %\begin{equation*}
% %   \left[
% %        \frac{n+2}{n+1} Q^{1,2}_{n}(t) -
% %        \frac{(2n+5)(N-1)}{2(n+1)(N-2)}\,(1-t)\,T^{2,2}_{n}\left(\frac{N-1}{2}(1+t)-1,N-2\right)
% %   \right]
% %   \, Q^{2,1}_{n}\left(v + \frac{2}{N-1}\right)
% %   -
% %\end{equation*}
% %\begin{equation*}
% %   \left[
% %        \frac{n+3}{n+1} Q^{2,1}_{n}\left(v + \frac{2}{N-1}\right)-
% %        \frac{(2n+5)(N-1)}{2(n+1)(N-2)}\,(1-t)\,T^{3,1}_{n}\left(\frac{N-1}{2}(1+v),N-2\right)
% %   \right]
% %   Q^{1,2}_{n}(t) =
% %\end{equation*}
% %\begin{equation*}
% %        \frac{n+2}{n+1} Q^{1,2}_{n}(t)
% %        \, Q^{2,1}_{n}\left(v + \frac{2}{N-1}\right)
% %        -
% %        \frac{(2n+5)(N-1)}{2(n+1)(N-2)}\,(1-t)\,T^{2,2}_{n}\left(\frac{N-1}{2}(1+t)-1,N-2\right)
% %        \, Q^{2,1}_{n}\left(v + \frac{2}{N-1}\right)
% %\end{equation*}
% %\begin{equation*}
% %    -
% %    \frac{n+3}{n+1} Q^{2,1}_{n}\left(v + \frac{2}{N-1}\right) \, Q^{1,2}_{n}(t)
% %    +
% %    \frac{(2n+5)(N-1)}{2(n+1)(N-2)}\,\left(1-v+\frac{2}{N-1}\right)\,T^{3,1}_{n}\left(\frac{N-1}{2}(1+v),N-2\right)
% %    \,Q^{1,2}_{n}(t) =
% %\end{equation*}
% %\begin{equation*}
% %        \frac{n+2}{n+1} Q^{1,2}_{n}(t)
% %        \, Q^{2,1}_{n}\left(v + \frac{2}{N-1}\right)
% %        -
% %        \frac{(2n+5)(N-1)}{2(n+1)(N-2)}\,(1-t)\,T^{2,2}_{n}\left(\frac{N-1}{2}(1+t)-1,N-2\right)
% %        \, Q^{2,1}_{n}\left(v + \frac{2}{N-1}\right)
% %\end{equation*}
% %\begin{equation*}
% %    -
% %    \frac{n+3}{n+1} Q^{2,1}_{n}\left(v + \frac{2}{N-1}\right) \, Q^{1,2}_{n}(t)
% %    +
% %    \frac{(2n+5)(N-1)}{2(n+1)(N-2)}\,\left(1-v+\frac{2}{N-1}\right)\,T^{3,1}_{n}\left(\frac{N-1}{2}(1+v),N-2\right)
% %    \,Q^{1,2}_{n}(t) =
% %\end{equation*}
% \begin{equation*}
        % \frac{(2n+5)(N-1)}{2(n+1)(N-2)}\,
        % \left[
            % \left(1-v+\frac{2}{N-1}\right)\,T^{3,1}_{n}\left(\frac{N-1}{2}(1+v),N-2\right)\,Q^{1,2}_{n}(t)
        % \right.
% \end{equation*}
% \begin{equation*}
        % \left.
            % -
            % (1-t)\,T^{2,2}_{n}\left(\frac{N-1}{2}(1+t)-1,N-2\right)\, Q^{2,1}_{n}\left(v + \frac{2}{N-1}\right)
        % \right]
% \end{equation*}
% \begin{equation}\label{q12q21}
        % -\frac{1}{n+1} Q^{1,2}_{n}(t)
        % \, Q^{2,1}_{n}\left(v + \frac{2}{N-1}\right).
% \end{equation}
% Тогда
% \begin{equation*}
% \frac{(N-2)^{[k+1]}}{(N+k+2)^{[k]}}  \,
   % \frac{(k+3)(k+4)}{(2k+5)(N-1)}
   % \left[
        % Q^{1,2}_{k+1}(t) \,Q^{2,1}_{k}\left(v +\frac{2}{N-1} \right) -  Q^{1,2}_{k}(t) \, Q^{2,1}_{k+1} \left(v +\frac{2}{N-1} \right)
   % \right]
   % =
% \end{equation*}
% \begin{equation*}
% \frac{(N-2)^{[k+1]}}{(N+k+2)^{[k]}}  \,
   % \frac{(k+3)(k+4)}{(2k+5)(N-1)}
    % \left\{
        % \frac{(2k+5)(N-1)}{2(k+1)(N-2)}\,
        % \left[
            % \left(1-v+\frac{2}{N-1}\right)\,T^{3,1}_{k}\left(\frac{N-1}{2}(1+v),N-2\right)\,Q^{1,2}_{k}(t)
            % -
    % \right.
    % \right.
% \end{equation*}
% \begin{equation*}
% \left.
    % \left.
        % (1-t)\,T^{2,2}_{k}\left(\frac{N-1}{2}(1+t)-1,N-2\right)\, Q^{2,1}_{k}\left(v + \frac{2}{N-1}\right)
    % \right]
        % -\frac{1}{k+1} Q^{1,2}_{k}(t)
        % \, Q^{2,1}_{k}\left(v + \frac{2}{N-1}\right)
   % \right\}=
% \end{equation*}

% \begin{equation*}
% \frac{(N-3)^{[k]}}{(N+k+2)^{[k]}}  \,
   % \frac{(k+3)(k+4)}{2(k+1)}
        % \left[
            % \left(1-v+\frac{2}{N-1}\right)\,T^{3,1}_{k}\left(\frac{N-1}{2}(1+v),N-2\right)\,Q^{1,2}_{k}(t)
            % -
    % \right.
% \end{equation*}
% \begin{equation*}
    % \left.
        % (1-t)\,T^{2,2}_{k}\left(\frac{N-1}{2}(1+t)-1,N-2\right)\, Q^{2,1}_{k}\left(v + \frac{2}{N-1}\right)
    % \right] -
% \end{equation*}
% \begin{equation*}
        % \frac{(N-2)^{[k+1]}}{(N-1)(N+k+2)^{[k]}}  \,\frac{(k+3)(k+4)}{(k+1)(2k+5)}
        % Q^{1,2}_{k}(t)
        % \, Q^{2,1}_{k}\left(v + \frac{2}{N-1}\right).
% \end{equation*}
% Используя весовые оценки \eqref{weightEst} для $t \in [0,1]$, можем записать:
% %\begin{equation*}
% %  \left| T_{n}^{\alpha,\beta}\left(\frac{N-1}{2}(1+t)\right)\right| \leq
% %    c(\alpha,\beta,a)\,n^{-1/2}\,
% %    \left[  \sqrt{1-t} + \frac1n \right]^{-\alpha-1/2}
% %    \left[  \sqrt{1+t} + \frac1n \right]^{-\beta-1/2},
% %\end{equation*}
% %\begin{equation*}
% %  \left| T_{n}^{2,2}\left(\frac{N-1}{2}(1+t)\right)\right| \leq
% %    c(2,2,a)\,n^{-1/2}\,
% %    \left[  \sqrt{1-t} + \frac1n \right]^{-2-1/2}
% %    \left[  \sqrt{1+t} + \frac1n \right]^{-2-1/2},
% %\end{equation*}
% %\begin{equation*}
% %  \left| T_{n}^{1,2}\left(\frac{N-1}{2}(1+t)\right)\right| \leq
% %    c(1,2,a)\,n^{-1/2}\,
% %    \left[  \sqrt{1-t} + \frac1n \right]^{-1-1/2}
% %    \left[  \sqrt{1+t} + \frac1n \right]^{-2-1/2},
% %\end{equation*}
% %\begin{equation*}
% %  \left| T_{n}^{2,1}\left(\frac{N-1}{2}(1+t)\right)\right| \leq
% %    c(2,1,a)\,n^{-1/2}\,
% %    \left[  \sqrt{1-t} + \frac1n \right]^{-2-1/2}
% %    \left[  \sqrt{1+t} + \frac1n \right]^{-1-1/2},
% %\end{equation*}
% %\begin{equation*}
% %  \left| T_{n}^{3,1}\left(\frac{N-1}{2}(1+t)\right)\right| \leq
% %    c(3,1,a)\,n^{-1/2}\,
% %    \left[  \sqrt{1-t} + \frac1n \right]^{-3-1/2}
% %    \left[  \sqrt{1+t} + \frac1n \right]^{-1-1/2},
% %\end{equation*}
% %Первая часть:
% \begin{equation*}
% \frac{(N-3)^{[k]}}{(N+k+2)^{[k]}}  \,
   % \frac{(k+3)(k+4)}{2(k+1)}\left(1-v+\frac{2}{N-1}\right)\,
   % \left|
        % T^{3,1}_{k}\left(\frac{N-1}{2}(1+v),N-2\right)\,
    % \right|
    % \left|
        % Q^{1,2}_{k}(t)
   % \right| \leq
% \end{equation*}
% \begin{equation*}
   % c(a) \,k\,\left(1-v+\frac{2}{N-1}\right)\,k^{-1/2} \,
    % \left(  \sqrt{1-v} + \frac1k \right)^{-7/2}
    % \left(  \sqrt{1+v} + \frac1k \right)^{-3/2}
    % k^{-1/2} \,
    % \left(  \sqrt{1-t} + \frac1k \right)^{-3/2}
   % \leq
% \end{equation*}
% \begin{equation*}
   % c(a) \,\left(\sqrt{1-v}+\sqrt{\frac{2}{N-1}}\right)^2\,
    % \left(  \sqrt{1-v} + \frac1k \right)^{-7/2}
    % \left(  \sqrt{1+v} + \frac1k \right)^{-3/2}
    % \left(  \sqrt{1-t} + \frac1k \right)^{-3/2}
   % \leq
% \end{equation*}
% \begin{equation*}
   % c(a) \,
    % \left(  \sqrt{1-v} + \frac1k \right)^{-3/2}
    % \left(  \sqrt{1+v} + \frac1k \right)^{-3/2}
    % \left(  \sqrt{1-t} + \frac1k \right)^{-3/2}
   % \leq
% \end{equation*}
% \begin{equation*}
   % c(a) \, \left[(1-v)(1+v)(1-t)\right]^{-3/4};
% \end{equation*}
% %Вторая часть:
% \begin{equation*}
% \frac{(N-3)^{[k]}}{(N+k+2)^{[k]}}  \,
   % \frac{(k+3)(k+4)}{2(k+1)}
        % (1-t)\,
        % \left| T^{2,2}_{k}\left(\frac{N-1}{2}(1+t)-1,N-2\right)\right|\,\left| Q^{2,1}_{k}\left(v + \frac{2}{N-1}\right)\right| \leq
% \end{equation*}
% \begin{equation*}
% c(a)k\,(1-t)\,k^{-1/2}
    % \left[  \sqrt{1-t} + \frac1k \right]^{-5/2}
    % \,k^{-1/2}
    % \left[  \sqrt{1-v} + \frac1k \right]^{-5/2}
    % \left[  \sqrt{1+v} + \frac1k \right]^{-3/2}
    % \leq
% \end{equation*}
% \begin{equation*}
% c(a)\,
% \left(\sqrt{1-t}+\frac1k\right)^2\,
    % \left[  \sqrt{1-t} + \frac1k \right]^{-5/2}
    % \left[  \sqrt{1-v} + \frac1k \right]^{-5/2}
    % \left[  \sqrt{1+v} + \frac1k \right]^{-3/2}
    % \leq
% \end{equation*}
% \begin{equation*}
% c(a)\,
    % %\left(\sqrt{1-t}+\frac1k\right)^{-1/2}\,
    % (1-t)^{-1/4}
    % (1-v)^{-5/4}
    % (1+v)^{-3/4};
% \end{equation*}
% %Третья часть:
% \begin{equation*}
% \left|
    % \frac{(N-2)^{[k+1]}}{(N-1)(N+k+2)^{[k]}}
    % \,\frac{(k+3)(k+4)}{(k+1)(2k+5)}
    % \,Q^{1,2}_{k}(t)
    % \,Q^{2,1}_{k} \left( v + \frac{2}{N-1} \right)
% \right| \leq
% \end{equation*}
% \begin{equation*}
    % c\,
    % \left|Q^{1,2}_{k}(t)\right| \,
    % \left|Q^{2,1}_{k} \left( v + \frac{2}{N-1} \right)\right|
    % \leq
% \end{equation*}
% %\begin{equation*}
% %    c(a)\,\frac1k
% %    \left[  \sqrt{1-t} + \frac1k \right]^{-3/2}
% %    \left[  \sqrt{1+t} + \frac1k \right]^{-5/2}
% %    \,
% %    \left[  \sqrt{1-v - \frac{2}{N-1}} + \frac1k \right]^{-5/2}
% %    \left[  \sqrt{1+v + \frac{2}{N-1}} + \frac1k \right]^{-3/2}
% %    \leq
% %\end{equation*}
% \begin{equation*}
    % \frac{c(a)}{k}
    % \left(  \sqrt{1-t} + \frac1k \right)^{-3/2}
    % \,
    % \left(  \sqrt{1-v} + \frac1k \right)^{-5/2}
    % \left(  \sqrt{1+v} + \frac1k \right)^{-3/2}
    % \leq
% \end{equation*}
% \begin{equation*}
    % c(a)\,\frac{\left(  \sqrt{1-t} + \frac1k \right)^{-1}}{k}
    % \left(  \sqrt{1-t} + \frac1k \right)^{-1/2}
    % \left(  1-v \right)^{-5/4}
    % \left(  1+v \right)^{-3/4}
    % \leq
% \end{equation*}
% \begin{equation*}
    % c(a) \,
    % \left(  \sqrt{1-t} + \frac1k \right)^{-1/2}
    % \left(  1-v \right)^{-5/4}
    % \left(  1+v \right)^{-3/4} \leq
% \end{equation*}
% \begin{equation*}
    % c(a) \,
    % \left(  1-t \right)^{-1/4}
    % \left(  1-v \right)^{-5/4}
    % \left(  1+v \right)^{-3/4}.
% \end{equation*}
% Объединяя оценки для каждой части суммы, приходим к справедливости оценки \eqref{lem2eq}.
% %Тогда величину $X^{1}_{k,N}(t,v)$ можно оценить следующим образом
% %
% %\begin{equation*}
% %X^{1}_{k,N}(t,v) =
% %\frac{(N-3)^{[k]}}{(N+k+2)^{[k]}}  \,
% %   \frac{(k+3)(k+4)}{2(k+1)}
% %        \left[
% %            \left(1-v+\frac{2}{N-1}\right)\,T^{3,1}_{k}\left(\frac{N-1}{2}(1+v),N-2\right)\,Q^{1,2}_{k}(t)
% %            -
% %    \right.
% %\end{equation*}
% %\begin{equation*}
% %    \left.
% %        (1-t)\,T^{2,2}_{k}\left(\frac{N-1}{2}(1+t)-1,N-2\right)\, Q^{2,1}_{k}\left(v + \frac{2}{N-1}\right)
% %    \right]
% %\end{equation*}
% %\begin{equation*}
% %        -
% %        \frac{(N-2)^{[k+1]}}{(N-1)(N+k+2)^{[k]}}  \,\frac{(k+3)(k+4)}{(k+1)(2k+5)}
% %        Q^{1,2}_{k}(t)
% %        \, Q^{2,1}_{k}\left(v + \frac{2}{N-1}\right).
% %\end{equation*}
% %Отсюда приходим к справедливости
% %\begin{equation*}
% %X^{1}_{k,N}(t,v) \leq
% % c(a) \, (1-t)^{-1/4} (1-v)^{-3/4}(1+v)^{-3/4} \left[
% %    (1-t)^{-1/2} +
% %    (1-v)^{-1/2}
% %    \right].
% %\end{equation*}
% \end{proof}





\section*{СРЕДНИЕ ВАЛЛЕ -- ПУССЕНА ДЛЯ ПРЕДЕЛЬНОГО РЯДА}

Аналогично классическому случаю введем в рассмотрение усреднение частичных сумм предельного ряда \eqref{limitS} вида
\begin{equation}\label{limitVP}
  \mathcal{V}^{-1}_{m,n}(f,x) = \frac{S_{m,N}^{-1}(f,x) + S_{m+1,N}^{-1}(f,x) + \ldots + S_{m+n,N}^{-1}(f,x)}{n+1},
\end{equation}
которое назовем \textit{средними Валле -- Пуссена для частичных сумм предельного ряда}.

Используя представление \eqref{Sn_jay}, можем переписать
\begin{equation*}
  \mathcal{V}^{-1}_{m,n}(f,x) = a_N(f,x) + %\tilde{b}_{n,N}
  \frac{8x(N-x-1)}{N(N-1)(N-2)(n+1)}
   \sum_{j=1}^{N-2} g(j) \left[ J_{m-2,N-2}(x-1,j-1) + \right.
\end{equation*}
\begin{equation*}
  \left.\ldots + J_{m+n-2,N-2}(x-1,j-1) \right] = a_N(f,x) +
\end{equation*}
\begin{equation*}
   \frac{8x(N-x-1)}{N(N-1)(N-2)(n+1)}
   \sum_{j=1}^{N-2} g(j) \left[ I_{m+n-2,N-2}(x-1,j-1) -  \right.
\end{equation*}
\begin{equation} \label{limVPf}
 \left. I_{m+n-2,N-2}(x-1,j-1) - I_{m-2,N-2}(x-1,j-1) \right],
\end{equation}
%\begin{equation*}
%  \mathcal{V}^{-1}_{m,n}(f,x) = a_N(f,x) +
%  \frac{8x(N-x-1)}{N(N-1)(n+1)}
%  \left( \sum_{k=0}^{m-2} \hat{g}_k \tau^{1}_{k}(x-1,N-2) + \ldots \right.
%\end{equation*}
%\begin{equation*}
%  \left. + \sum_{k=0}^{m+n-2} \hat{g}_k \tau^{1}_{k}(x-1,N-2) \right) =
%\end{equation*}
%\begin{equation*}
%    a_N(f,x) +
%    \frac{8x(N-x-1)}{N(N-1)(N-2)(n+1)}
%        \sum_{j=1}^{N-2} g(j)
%        \left[
%            \sum_{k=0}^{m-2} + \ldots
%        \right.
%\end{equation*}
%\begin{equation}\label{limitVP1}
%        \left.
%            + \sum_{k=0}^{m+n-2}
%        \right] \tau^{1}_{k}(j-1,N-2) \tau^{1}_{k}(x-1,N-2).
%\end{equation}
где
\begin{equation}\label{I}
  I_{k,N}(x,j) = \sum_{l=0}^{k} J_{l,N}(x,j).
\end{equation}

%%%%%%%%%%%%%%%%%%%%%%%%%%%%%%%%%%%%%%%%%%%%%%%
%%%%%%%%%%%%%%%%%%%%%%%%%%%%%%%%%%%%%%%%%%%%%%%
%%%%%%%%%%%%%%%%%%%%%%%%%%%%%%%%%%%%%%%%%%%%%%%


Рассмотрим норму оператора $\mathcal{V}^{-1}_{m,n}$, действующего  в пространстве $C[0, N-1]$:
\begin{equation}\label{norm}
  \| \mathcal{V}^{-1}_{m,n} \| = \sup_{\| f \| \leq 1} \| \mathcal{V}^{-1}_{m,n}(f,x) \|.
\end{equation}
Заметим, что $\max_{x \in [0,N-1]} |g(x)| \leq 1$ при $\| f(x) \| \leq 1$, а также что $\| a_N(f,x) \| = 1$. Тогда
\begin{equation}\label{norm2}
  \| \mathcal{V}^{-1}_{m,n} \| = 1 + \| {V}^{-1}_{m,n}(x) \|,
\end{equation}
где
\begin{equation*}
   V^{-1}_{m,n}  = \frac{8x(N-x-1)}{N(N-1)(N-2)(n+1)}
   \sum_{j=1}^{N-2} \left| I_{m+n-2,N-2}(x-1,j-1) - \right.
\end{equation*}
\begin{equation}\label{norm2}
   \left. I_{m-2,N-2}(x-1,j-1) \right|.
\end{equation}

%%%%%%%%%%%%%%%%%%%%%%%%%%%%%%%%%%%%%%%%%%%%%%%
%%%%%%%%%%%%%%%%%%%%%%%%%%%%%%%%%%%%%%%%%%%%%%%
%%%%%%%%%%%%%%%%%%%%%%%%%%%%%%%%%%%%%%%%%%%%%%%

%Для дальнейшего нам понадобятся некоторые выражения для $J_{l,N}(x,j)$ и $I_{k,N}(x,j)$.
Для дальнейшего нам понадобится преобразовать некоторые выражения для $J_{l,N}(x,j)$ и $I_{k,N}(x,j)$.
Используя формулу Кристоффеля -- Дарбу \eqref{kristT}, выводим:

\begin{equation*}
  J_{l,N} (x,j) = D^{1}_{l} (x,j) =
  \sum_{r=0}^{l} \frac{T^{1}_{r}(x) T^{1}_{r}(j)}{h^{1}_{r,N}} =
\end{equation*}
\begin{equation}\label{krist}
  \frac{(N-1)^{[l+1]}}{(N+l+2)^{[l]}} \, \frac{l+3}{16}\,
  \frac{T_{l+1}^1(x) T_{l}^1(j) - T_{l+1}^1(j) T_{l}^1(x)}{x-j}.
\end{equation}
%\begin{equation*}
%  J_{n,N+2} (x,y) = \sum_{k=0}^{n} \tau^{1}_{k,N}(x) \tau^{1}_{k,N}(y) =
%  \sum_{k=0}^{n} \frac{T^{1}_{k,N}(x) T^{1}_{k,N}(y)}{h^{1}_{n,N}} =
%\end{equation*}
%\begin{equation}\label{krist}
%  \frac{(n+3)(N-1)^{[n+1]}}{16(N+n+2)^{[n]}}
%  \frac{T_{n+1,N}^1(x) T_{n,N}^1(y) - T_{n+1,N}^1(y) T_{n,N}^1(x)}{x-y}.
%\end{equation}
%\begin{equation*}
%  D_{n,N}^{1} (x,y) = \sum_{k=0}^{n} \frac{T^{1}_{k,N}(x) T^{1}_{k,N}(y)}{h^{1}_{n,N}} =
%  \frac{(n+3)(N-1)^{[n+1]}}{16(N+n+2)^{[n]}}
%  \frac{T_{n+1,N}^1(x) T_{n,N}^1(y) - T_{n+1,N}^1(y) T_{n,N}^1(x)}{x-y}.
%\end{equation}
Отсюда и из определения \eqref{I} записываем%, обозначив для краткости $B_{l,N} = \frac{(N-1)^{[l+1]}}{(N+l+2)^{[l]}}\,\frac{l+3}{16}$,
\begin{equation*}
  I_{k,N}(x,j) = %\sum_{l=0}^{k} J_{l,N}(x,j) =
  \sum_{l=0}^{k} \frac{(N-1)^{[l+1]}}{(N+l+2)^{[l]}}\,\frac{l+3}{16}\,
  \frac{T_{l+1}^1(x) T_{l}^1(j) - T_{l+1}^1(j) T_{l}^1(x)}{x-j}.
\end{equation*}
%\begin{equation*}
%\sum_{l=0}^{k} \frac{(N-1)^{[l+1]}}{(N+l+2)^{[l]}} \, \frac{l+3}{16}\,
%  \frac{T_{l+1}^1(x) T_{l}^1(j) - T_{l+1}^1(j) T_{l}^1(x)}{x-j}.
%\end{equation*}
%\begin{equation*}
%\sum_{l=0}^{k} \frac{(N-1)^{[l+1]}}{(N+l+2)^{[l]}} \, \frac{l+3}{16}\, \left(
%    \frac{\left[T_{l+1}^1(x) - T_{l}^1(x) \right]\left[ T_{l+1}^1(j) +  T_{l}^1(j) \right]}{x-j} +
%  \right.
%\end{equation*}
%\begin{equation*}
%\left.
%    \frac{T_{l}^1(x) T_{l}^1(j) - T_{l+1}^1(x) T_{l+1}^1(j)}{x-j}
%\right)= A^{1}_k(x,j) + A^{2}_k(x,j).
%\end{equation*}
Используя Лемму \eqref{lemma0} и обозначив для краткости $B_{l,N} = \frac{(l+2)(l+3)}{16(l+1)}\,\frac{(N-2)^{[l]}}{(N+l+2)^{[l]}} $
получим
%
%\begin{equation*}
%  I_{k,N}(x,j) = \frac{2}{N-1}\,
%  \sum_{l=0}^{k} B_{l,N}\,
%  \frac{T_{l+1}^1(x) T_{l}^1(j) - T_{l+1}^1(j) T_{l}^1(x)}{x-j} =
%\end{equation*}
%\begin{equation*}
%  \frac{2}{N-1}\,\sum_{l=0}^{k} B_{l,N}\, \left[
%    \frac{x(N-1-j)}{x-j}\,T^{1,2}_{l}(x-1,N-1)\,T^{2,1}_{l}(j,N-1)
%  \right.
%\end{equation*}
%\begin{equation*}
%\left.
%  - \frac{j(N-1-x)}{x-j}\,T^{1,2}_{l}(j-1,N-1)\,T^{2,1}_{l}(x,N-1)
%\right].
%\end{equation*}
%
%
%
%
%
%
%Проведя замену переменных $\{x = \frac{N-1}{2}\,(1+t), j = \frac{N-1}{2}\,(1+v)\}$, перепишем эти выражения в терминах полиномов $Q_n^{\alpha,\beta}(t)$:
%\begin{equation*}
%  I_{k,N}(t,v) =
%  \sum_{l=0}^{k} B_{l,N}\,
%  \left[
%    \frac{\frac{(N-1)}{2}\,(1+t)(N-1-\frac{(N-1)}{2}\,(1+v))}{\frac{(N-1)}{2}\,(1+t)-\frac{(N-1)}{2}\,(1+v)}\,T^{1,2}_{l}(\frac{(N-1)}{2}\,(1+t)-1,N-1)\,T^{2,1}_{l}(\frac{(N-1)}{2}\,(1+v),N-1)
%  \right.
%\end{equation*}
%\begin{equation*}
%\left.
%  - \frac{\frac{(N-1)}{2}\,(1+v)(N-1-\frac{(N-1)}{2}\,(1+t))}{\frac{(N-1)}{2}\,(1+t)-\frac{(N-1)}{2}\,(1+v)}\,T^{1,2}_{l}(\frac{(N-1)}{2}\,(1+v)-1,N-1)\,T^{2,1}_{l}(\frac{(N-1)}{2}\,(1+t),N-1)
%\right].
%\end{equation*}
%\begin{equation*}
%  I_{k,N}(t,v) =
%  \sum_{l=0}^{k} B_{l,N} \, \frac{N-1}{2}\,
%  \left[
%    \frac{(1+t)(1-v)}{t-v}\,T^{1,2}_{l}(\frac{(N-1)}{2}\,(1+t)-1,N-1)\,T^{2,1}_{l}(\frac{(N-1)}{2}\,(1+v),N-1)
%  \right.
%\end{equation*}
%\begin{equation*}
%\left.
%  - \frac{(1+v)(1-t)}{t-v}\,T^{1,2}_{l}(\frac{(N-1)}{2}\,(1+v)-1,N-1)\,T^{2,1}_{l}(\frac{(N-1)}{2}\,(1+t),N-1)
%\right].
%\end{equation*}
\begin{equation*}
  \hat{I}_{k,N}(t,v) = I_{k,N}\left( \frac{N-1}{2}(1+t) , \frac{N-1}{2}(1+v) \right)=
\end{equation*}
\begin{equation*}
  \sum_{l=0}^{k} B_{l,N} \,
  \left[
    \frac{(1+t)(1-v)}{t-v}\,Q^{1,2}_{l}(t)\,Q^{2,1}_{l}\left(v+\frac{2}{N-1}\right)
  \right.
\end{equation*}
\begin{equation}\label{Usum}
\left.
  - \frac{(1+v)(1-t)}{t-v}\,Q^{1,2}_{l}(v)\,Q^{2,1}_{l}\left(t+\frac{2}{N-1}\right)
\right] = U_{k,N}(t,v) + U_{k,N}(v,t),
\end{equation}
%\begin{equation*}
%  \frac{(1+t)(1-v)}{t-v}\,
%  \sum_{l=0}^{k} B_{l,N} \,
%    Q^{1,2}_{l}(t)\,Q^{2,1}_{l}\left(v+\frac{2}{N-1}\right)
%\end{equation*}
%\begin{equation*}
%- \frac{(1+v)(1-t)}{t-v}\,
%  \sum_{l=0}^{k} B_{l,N} \,
%    Q^{1,2}_{l}(v)\,Q^{2,1}_{l}\left(t+\frac{2}{N-1}\right),
%\end{equation*}
где
\begin{equation}\label{UkN}
  U_{k,N}(t,v)=
  \frac{(1+t)(1-v)}{t-v}\,
  \sum_{l=0}^{k} B_{l,N} \,
    Q^{1,2}_{l}(t)\,Q^{2,1}_{l}\left(v+\frac{2}{N-1}\right).
\end{equation}
С помощью Леммы \eqref{lemma1} мы можем преобразовать последнее выражение к виду
%\begin{equation*}
%  U_{k,N}(t,v)=
%  \frac{(1+t)(1-v)}{16 (t-v)}\,
%  \sum_{l=0}^{k}
%   \frac{(N-2)^{[l]}}{(N+l+2)^{[l]}}\,\frac{(l+2)(l+3)}{(l+1)}
%  \,
%    Q^{1,2}_{l}(t)\,Q^{2,1}_{l}\left(v+\frac{2}{N-1}\right).
%\end{equation*}
%
%\begin{equation*}
%    \left(
%    t-v
%    \right)\,
%   \sum_{l=0}^{k}
%    \frac{(N-2)^{[l]}}{(N+l+2)^{[l]}} \, \frac{(l+2)(l+3)}{(l+1)}\,
%    Q^{1,2}_{l}(t) \,Q^{2,1}_{l}\left(v +\frac{2}{N-1} \right)
%    =
%\end{equation*}
%\begin{equation*}
%   \frac{(N-2)^{[k+1]}}{(N+k+2)^{[k]}}  \,
%   \frac{(k+3)(k+4)}{(2k+5)(N-1)}
%   \left[
%        Q^{1,2}_{k+1}(t) \,Q^{2,1}_{k}\left(v +\frac{2}{N-1} \right) -  Q^{1,2}_{k}(t) \, Q^{2,1}_{k+1} \left(v +\frac{2}{N-1} \right)
%   \right]
%\end{equation*}
%\begin{equation}\label{lem1eq}
%   +
%   \sum_{l=0}^{k} \left( \lambda_{n,N} + \frac{2}{N-1}\right)\,
%    \frac{(N-2)^{[l]}}{(N+l+2)^{[l]}} \, \frac{(l+2)(l+3)}{(l+1)}\,
%    Q^{1,2}_{l}(t) \,Q^{2,1}_{l}\left(v +\frac{2}{N-1} \right),
%\end{equation}
%где $\lambda_{n,N} = \frac{6}{(2n+3)(2n+5)}+\frac{q(n)}{N-1}$, $|q(n)|<c$.
%
%===============

\begin{equation*}
U_{k,N}(t,v)=
\frac{(1+t)(1-v)}{16 (t-v)^2}\,
\left\{
   \frac{(N-2)^{[k+1]}}{(N+k+2)^{[k]}}  \,
   \frac{(k+3)(k+4)}{(2k+5)(N-1)}
\right.
\end{equation*}
\begin{equation*}
   \left[
        Q^{1,2}_{k+1}(t) \,Q^{2,1}_{k}\left(v +\frac{2}{N-1} \right) -  Q^{1,2}_{k}(t) \, Q^{2,1}_{k+1} \left(v +\frac{2}{N-1} \right)
   \right]
\end{equation*}
\begin{equation*}
\left.
   +
   \sum_{l=0}^{k} \lambda_{l,N}\,
    \frac{(N-2)^{[l]}}{(N+l+2)^{[l]}} \, \frac{(l+2)(l+3)}{(l+1)}\,
    Q^{1,2}_{l}(t) \,Q^{2,1}_{l}\left(v +\frac{2}{N-1} \right)
\right\}=
\end{equation*}
\begin{equation*}
\frac{(1+t)(1-v)}{16 (t-v)^2}\,\left\{
    X^{1}_{k,N}(t,v) + X^{2}_{k,N}(t,v)
\right\}.
\end{equation*}
Таким образом, приходим к следующей схеме оценивания выражения \eqref{limVPf}:

Заметим теперь, что для $X^{1}_{k,N}(t,v)$ справедлива оценка \eqref{lem2eq} из Леммы \eqref{lemma2}, из которой следует
\begin{equation*}
 \left|
 X^{1}_{m+n-2,N}(t,v) - X^{1}_{m-2,N}(t,v)
 \right| \leq
\end{equation*}
\begin{equation}\label{x1mn}
 c(a) \, (1-t)^{-1/4} (1-v)^{-3/4}(1+v)^{-3/4} \left[
    (1-t)^{-1/2} +
    (1-v)^{-1/2}
    \right].
\end{equation}
Оценим кроме того разность
\begin{equation*}
  \left| X^{2}_{m+n-2,N}(t,v) - X^{2}_{m-2,N}(t,v) \right| \leq
\end{equation*}
\begin{equation*}
   \sum_{l=m-2}^{m+n-2}  \lambda_{l,N}\,
    \frac{(N-2)^{[l]}}{(N+l+2)^{[l]}} \, \frac{(l+2)(l+3)}{(l+1)}\,
    \left| Q^{1,2}_{l}(t) \right|\,\left| Q^{2,1}_{l}\left(v +\frac{2}{N-1} \right)\right|
\leq
\end{equation*}
\begin{equation*}
   c(a)\sum_{l=m-2}^{m+n-2}  \lambda_{l,N}\,
    \frac{(N-2)^{[l]}}{(N+l+2)^{[l]}} \, \frac{(l+2)(l+3)}{l(l+1)}\,
    \left(  \sqrt{1-t} + \frac1l \right)^{-3/2}
    %\left(  \sqrt{1+t} + \frac1l \right)^{-5/2}
    \left(  \sqrt{1-v} + \frac1l \right)^{-5/2}
    \left(  \sqrt{1+v} + \frac1l \right)^{-3/2}
\leq
\end{equation*}
\begin{equation*}
\left(  \sqrt{1-t} + \frac{1}{n+m} \right)^{-3/2}
%\left( 1-t\right)^{-3/4}
    \left(1-v\right)^{-5/4}
    \left(1+v \right)^{-3/4}
\sum_{l=m-2}^{m+n-2}
    \left[ \frac{ c(a)}{(2l+3)(2l+5)}+\frac{q(l)}{N-1}\right]
\leq
\end{equation*}
\begin{equation*}
    c(a)
    \left(  \sqrt{1-t} + \frac{1}{n+m} \right)^{-3/2}
    %\left( 1-t\right)^{-3/4}
    \left(1-v\right)^{-5/4}
    \left(1+v \right)^{-3/4}
\sum_{l=m-2}^{m+n-2}
    \left[ \frac{ 1}{(2l+3)(2l+5)}+\frac{1}{N-1}\right]
   =
\end{equation*}
\begin{equation*}
    c(a)
    \left(  \sqrt{1-t} + \frac{1}{n+m} \right)^{-3/2}
    %\left( 1-t\right)^{-3/4}
    \left(1-v\right)^{-5/4}
    \left(1+v \right)^{-3/4}
    \left[ \frac{ n+1}{4m^2+4mn-2n-1}+\frac{n+1}{N-1}\right].
\end{equation*}
Пользуясь теперь тем что $dn \leq m \leq bn$ и $n \leq a\sqrt{N}$, выводим
\begin{equation*}
  \left| X^{2}_{m+n-2,N}(t,v) - X^{2}_{m-2,N}(t,v) \right| \leq
    \frac{c(a)}{n}
    \left(  \sqrt{1-t} + \frac{1}{n} \right)^{-1-1/2}
    %\left( 1-t\right)^{-3/4}
    \left(1-v\right)^{-5/4}
    \left(1+v \right)^{-3/4}
    \leq
\end{equation*}
\begin{equation}\label{x2mn}
    c(a)
    \left(  \sqrt{1-t} + \frac{1}{n} \right)^{-1/2}
    \left(1-v\right)^{-5/4}
    \left(1+v \right)^{-3/4}
    \leq
    c(a)
    \left( 1-t\right)^{-1/4}
    \left(1-v\right)^{-5/4}
    \left(1+v \right)^{-3/4}.
\end{equation}
Объединяя оценки \eqref{x1mn} -- \eqref{x2mn}, получим
\begin{equation*}
\left| U_{m+n-2,N}(t,v) - U_{m-2,N}(t,v) \right| \leq
\frac{(1+t)(1-v)}{16 (t-v)^2}\,
\left(
   \left| X^{1}_{m+n-2,N}(t,v) - X^{1}_{m-2,N}(t,v) \right| +
\right.
\end{equation*}
\begin{equation*}
\left.
   \left| X^{2}_{m+n-2,N}(t,v) - X^{2}_{m-2,N}(t,v) \right|
\right) \leq
\end{equation*}
\begin{equation*}
c(a)\frac{(1+t)(1-v)}{(t-v)^2}\,
\left\{
(1-t)^{-1/4} (1-v)^{-3/4}(1+v)^{-3/4} \left[
    (1-t)^{-1/2} +
    (1-v)^{-1/2}
    \right]
+ \right.
\end{equation*}
\begin{equation*}
\left.
    \left( 1-t\right)^{-1/4}
    \left(1-v\right)^{-5/4}
    \left(1+v \right)^{-3/4}
\right\} \leq
\end{equation*}
\begin{equation*}
\frac{c(a)(1+t)}{(t-v)^2}\,
\left\{
    (1-t)^{-1/4} (1-v)^{1/4}(1+v)^{-3/4} \left[
    (1-t)^{-1/2} +
    (1-v)^{-1/2}
    \right]
+ \right.
\end{equation*}
\begin{equation*}
\left.
    \left(1-t\right)^{-1/4}
    \left(1-v\right)^{-1/4}
    \left(1+v \right)^{-3/4}
\right\} \leq
\end{equation*}
\begin{equation*}
\frac{c(a)(1-t)^{-1/4}(1-v)^{1/4}(1+v)^{-3/4} }{(t-v)^2}\,
\left\{
    (1-t)^{-1/2} +    (1-v)^{-1/2}
\right\}.
\end{equation*}

Рассмотрим
\begin{equation*}
\frac{8x(N-x-1)}{N(N-1)(N-2)(n+1)} \left| U_{m+n-2,N}(t,v) - U_{m-2,N}(t,v) \right| \leq
\end{equation*}
\begin{equation*}
\frac{8x(N-x-1)}{N(N-1)(N-2)(n+1)}
\frac{c(a)(1-t)^{-1/4}(1-v)^{1/4}(1+v)^{-3/4} }{(t-v)^2}\,
\left\{
    (1-t)^{-1/2} +    (1-v)^{-1/2}
\right\}
\leq
\end{equation*}
\begin{equation*}
(1+t)(1-t)
\frac{c(a)(1-t)^{-1/4}(1-v)^{1/4}(1+v)^{-3/4} }{nN(t-v)^2}\,
\left\{
    (1-t)^{-1/2} +    (1-v)^{-1/2}
\right\}
\leq
\end{equation*}
\begin{equation*}
\frac{c(a)(1+t)(1-t)^{3/4}(1-v)^{1/4}(1+v)^{-3/4} }{nN(t-v)^2}\,
\left\{
    (1-t)^{-1/2} +    (1-v)^{-1/2}
\right\}
\leq
\end{equation*}
\begin{equation*}
\frac{c(a)(1+t)(1+v)^{-3/4} }{nN(t-v)^2}\,
\left\{
    (1-t)^{1/4}(1-v)^{1/4}+   (1-t)^{3/4}(1-v)^{-1/4}
\right\}
\leq
\end{equation*}





















\newpage
\newpage





\begin{equation*}
  I_{k,N}(t,v) =
  \sum_{l=0}^{k} B_{l,N} \,
  \left[
    \frac{(1+t)(1-v)}{t-v}\,Q^{1,2}_{l}(t)\,Q^{2,1}_{l}\left(v+\frac{2}{N-1}\right)
  \right.
\end{equation*}
\begin{equation*}
\left.
  - \frac{(1+v)(1-t)}{t-v}\,Q^{1,2}_{l}(v)\,Q^{2,1}_{l}\left(t+\frac{2}{N-1}\right)
\right],
\end{equation*}

========================

\begin{equation*}
(1+t)(1-v)\,
Q^{1,2}_{l}(t)
\,
Q^{2,1}_{l}\left(v+\frac{2}{N-1}\right)
-
\end{equation*}
\begin{equation*}
  (1+v)(1-t)\,
  Q^{1,2}_{l}(v)
  \,
  Q^{2,1}_{l}\left(t+\frac{2}{N-1}\right)
\end{equation*}

=========================



\begin{equation*}
(1+t)(1-v)\,
\end{equation*}
\begin{equation*}
\left[
\left(1+t-\frac{2}{N-1}\right)\,Q^{1,3}_{n}\left(t-\frac{2}{N-1}\right) + \left(1-t+\frac{2}{N-1}\right)\,Q^{2,2}_{n}\left(t\right)
\right]
\,
\end{equation*}
\begin{equation*}
\left[
 \frac{2n+3}{4n}\,\left[
  (1+v)\,Q^{2,2}_{n-1}\left(v\right)-
  (1-v)\,Q^{3,1}_{n-1}\left(v+\frac{2}{N-1}\right)
  \right] + \frac{1}{2n} Q^{2,1}_{n-1}\left(v+\frac{2}{N-1},N+1\right)
\right]
\end{equation*}
\begin{equation*}
  - (1+v)(1-t)\,
\end{equation*}
\begin{equation*}
  \left[
    \frac{2n+3}{4n}\,
  \left[
    \left(1+v-\frac{2}{N-1}\right)\,Q^{1,3}_{n-1}\left(v-\frac{2}{N-1}\right)-
  \left(1-v+\frac{2}{N-1}\right)\,Q^{2,2}_{n-1}\left(v\right)
  \right]
  - \frac{1}{2n}\, Q^{1,2}_{n-1}\left(v,N+1\right)
\right]
  \,
\end{equation*}
\begin{equation*}
  \frac12
\left[
\left(1+t-\frac{2}{N-1}\right)\,Q^{1,3}_{n}\left(t-\frac{2}{N-1}\right) + \left(1-t+\frac{2}{N-1}\right)\,Q^{2,2}_{n}\left(t\right)
\right]
\end{equation*}





%
%
%\begin{equation*}
%(1+t)(1-v)\,
%\end{equation*}
%\begin{equation*}
%Q^{1,2}_{l}(t)
%\,
%\end{equation*}
%\begin{equation*}
%Q^{2,1}_{l}\left(v+\frac{2}{N-1}\right)
%-
%\end{equation*}
%\begin{equation*}
%  (1+v)(1-t)\,
%\end{equation*}
%\begin{equation*}
%  Q^{1,2}_{l}(v)
%  \,
%\end{equation*}
%\begin{equation*}
%  Q^{2,1}_{l}\left(t+\frac{2}{N-1}\right)
%\end{equation*}







\newpage
\newpage



\begin{equation*}
  I_{k,N}(t,v) = \frac{(1+t)(1-v)}{t-v}\,
  \sum_{l=0}^{k} B_{l,N} \,
    Q^{1,2}_{l}(t)\,Q^{2,1}_{l}\left(v+\frac{2}{N-1}\right) +
\end{equation*}
\begin{equation*}
\frac{(1+v)(1-t)}{v-t}\,
\sum_{l=0}^{k} B_{l,N} \,
  Q^{1,2}_{l}(v)\,Q^{2,1}_{l}\left(t+\frac{2}{N-1}\right) =
U_{k,N}(t,v) + U_{k,N}(v,t).
\end{equation*}


=========================


\begin{equation*}
  (n+1) T^{\alpha,\beta}_{n+1}(x,N) + (n+\beta+1)T^{\alpha,\beta}_{n}(x,N) = \frac{2n+\alpha+\beta+2}{N-1}\,x\,T^{\alpha,\beta+1}_{n}(x-1,N-1),
\end{equation*}
\begin{equation*}
  (n+\alpha+1) T^{\alpha,\beta}_{n}(x,N) - (n+1)T^{\alpha,\beta}_{n+1}(x,N) = \frac{2n+\alpha+\beta+2}{N-1}\,(N-1-x)\,T^{\alpha+1,\beta}_{n}(x,N-1).
\end{equation*}

Сложим:
\begin{equation*}
  (N-1)T^{\alpha,\beta}_{n}(x,N) = x\,T^{\alpha,\beta+1}_{n}(x-1,N-1) + (N-1-x)\,T^{\alpha+1,\beta}_{n}(x,N-1).
\end{equation*}

В терминах $Q^{\alpha,\beta}_{n,N}$:
\begin{equation*}
  2\, Q^{\alpha,\beta}_{n}\left(t+\frac{2}{N-1},N+1\right) =
  (1+t)\,Q^{\alpha,\beta+1}_{n}(t) + (1-t)\,Q^{\alpha+1,\beta}_{n}\left(t+\frac{2}{N-1}\right).
\end{equation*}

===================


\begin{equation*}
  2\, Q^{1,2}_{n}\left(t+\frac{2}{N-1},N+1\right) =
  (1+t)\,Q^{1,3}_{n}(t) + (1-t)\,Q^{2,2}_{n}\left(t+\frac{2}{N-1}\right).
\end{equation*}

\begin{equation*}
  2\, Q^{1,2}_{n}\left(t,N+1\right) =
  \left(1+t-\frac{2}{N-1}\right)\,Q^{1,3}_{n}\left(t-\frac{2}{N-1}\right) + \left(1-t+\frac{2}{N-1}\right)\,Q^{2,2}_{n}\left(t\right).
\end{equation*}

\begin{equation*}
  2\, Q^{2,1}_{n}\left(t+\frac{2}{N-1},N+1\right) =
  (1+t)\,Q^{2,2}_{n}(t) + (1-t)\,Q^{3,1}_{n}\left(t+\frac{2}{N-1}\right).
\end{equation*}

\begin{equation*}
  2\, Q^{2,1}_{n}\left(t,N+1\right) =
  \left(1+t-\frac{2}{N-1}\right)\,Q^{2,2}_{n}\left(t-\frac{2}{N-1}\right) + \left(1-t+\frac{2}{N-1}\right)\,Q^{3,1}_{n}(t).
\end{equation*}



==========================








\newpage
\newpage



==========================

Вычтем из первого второе:
\begin{equation*}
  2(n+1) T^{\alpha,\beta}_{n+1}(x,N) + (\beta-\alpha)T^{\alpha,\beta}_{n}(x,N)
  = \frac{2n+\alpha+\beta+2}{N-1}\,\left[
  x\,T^{\alpha,\beta+1}_{n}(x-1,N-1)-
  \right.
\end{equation*}
\begin{equation*}
\left.
  (N-1-x)\,T^{\alpha+1,\beta}_{n}(x,N-1)
  \right].
\end{equation*}


========================


\begin{equation*}
  2(n+1) T^{1,2}_{n+1}(x,N) + T^{1,2}_{n}(x,N)
  = \frac{2n+5}{N-1}\,\left[
  x\,T^{1,3}_{n}(x-1,N-1)-
  \right.
\end{equation*}
\begin{equation*}
\left.
  (N-1-x)\,T^{2,2}_{n}(x,N-1)
  \right].
\end{equation*}

\begin{equation*}
  2(n+1) Q^{1,2}_{n+1}\left(t+\frac{2}{N-1},N+1\right)
   =
  \frac{2n+5}{2}\,
  \left[
    (1+t)\,Q^{1,3}_{n}(t)-
  \right.
\end{equation*}
\begin{equation*}
\left.
  (1-t)\,Q^{2,2}_{n}\left(t+\frac{2}{N-1}\right)
  \right]
  - Q^{1,2}_{n}\left(t+\frac{2}{N-1},N+1\right).
\end{equation*}


\begin{equation*}
  Q^{1,2}_{n}\left(t+\frac{2}{N-1},N+1\right)
   =
  \frac{2n+3}{4n}\,
  \left[
    (1+t)\,Q^{1,3}_{n-1}(t)-
  \right.
\end{equation*}
\begin{equation*}
\left.
  (1-t)\,Q^{2,2}_{n-1}\left(t+\frac{2}{N-1}\right)
  \right]
  - \frac{1}{2n}\, Q^{1,2}_{n-1}\left(t+\frac{2}{N-1},N+1\right).
\end{equation*}

\begin{equation*}
  Q^{1,2}_{n}\left(t,N+1\right)
   =
  \frac{2n+3}{4n}\,
  \left[
    \left(1+t-\frac{2}{N-1}\right)\,Q^{1,3}_{n-1}\left(t-\frac{2}{N-1}\right)-
  \right.
\end{equation*}
\begin{equation*}
\left.
  \left(1-t+\frac{2}{N-1}\right)\,Q^{2,2}_{n-1}\left(t\right)
  \right]
  - \frac{1}{2n}\, Q^{1,2}_{n-1}\left(t,N+1\right).
\end{equation*}




+++++++++++++++++++++++++++++




\begin{equation*}
  2(n+1) T^{2,1}_{n+1}(x,N) - T^{2,1}_{n}(x,N)
  = \frac{2n+5}{N-1}\,\left[
  x\,T^{2,2}_{n}(x-1,N-1)-
  \right.
\end{equation*}
\begin{equation*}
\left.
  (N-1-x)\,T^{3,1}_{n}(x,N-1)
  \right].
\end{equation*}

\begin{equation*}
  2(n+1) Q^{2,1}_{n+1}\left(t+\frac{2}{N-1},N+1\right)
  = \frac{2n+5}{2}\,\left[
  (1+t)\,Q^{2,2}_{n}\left(t\right)-
  \right.
\end{equation*}
\begin{equation*}
\left.
  (1-t)\,Q^{3,1}_{n}\left(t+\frac{2}{N-1}\right)
  \right] + Q^{2,1}_{n}\left(t+\frac{2}{N-1},N+1\right).
\end{equation*}

\begin{equation*}
  Q^{2,1}_{n}\left(t+\frac{2}{N-1},N+1\right)
  = \frac{2n+3}{4n}\,\left[
  (1+t)\,Q^{2,2}_{n-1}\left(t\right)-
  \right.
\end{equation*}
\begin{equation*}
\left.
  (1-t)\,Q^{3,1}_{n-1}\left(t+\frac{2}{N-1}\right)
  \right] + \frac{1}{2n} Q^{2,1}_{n-1}\left(t+\frac{2}{N-1},N+1\right).
\end{equation*}

\begin{equation*}
  Q^{2,1}_{n}\left(t,N+1\right)
  = \frac{2n+3}{4n}\,\left[
  \left(1+t-\frac{2}{N-1}\right)\,Q^{2,2}_{n-1}\left(t-\frac{2}{N-1}\right)-
  \right.
\end{equation*}
\begin{equation*}
\left.
  \left(1-t+\frac{2}{N-1}\right)\,Q^{3,1}_{n-1}\left(t\right)
  \right] + \frac{1}{2n} Q^{2,1}_{n-1}\left(t,N+1\right).
\end{equation*}







\newpage
\newpage

Рассмотрим тогда выражение
\begin{equation*}
  I_{m+n-2,N-2}(x,j) - I_{m-2,N-2}(x,j) =
\end{equation*}
\begin{equation*}
  \frac{2}{N-3}\,
  \sum_{l=m-1}^{m+n-2} B_{l,N-2}\,
  \frac{T_{l+1}^1(x,N-2) T_{l}^1(j,N-2) - T_{l+1}^1(j,N-2) T_{l}^1(x,N-2)}{x-j} =
\end{equation*}
\begin{equation*}
  \frac{2}{N-3}\,\sum_{l=m-1}^{m+n-2} B_{l,N-2}\, \left[
    \frac{x(N-1-j)}{x-j}\,T^{1,2}_{l}(x-1,N-3)\,T^{2,1}_{l}(j,N-3)
  \right.
\end{equation*}
\begin{equation*}
\left.
  - \frac{j(N-1-x)}{x-j}\,T^{1,2}_{l}(j-1,N-3)\,T^{2,1}_{l}(x,N-3)
\right].
\end{equation*}

Проведя замену переменных $\{x = \frac{N-3}{2}\,(1+t), j = \frac{N-3}{2}\,(1+v)\}$, перепишем эти выражения в терминах полиномов $Q_n^{\alpha,\beta}(t)$:
\begin{equation*}
  I_{m+n-2,N-2}\left(t,v \right) - I_{m-2,N-2}\left(t,v \right) =
\end{equation*}
%\begin{equation*}
%  \sum_{l=m-1}^{m+n-2} B_{l,N-2}\,
%  \left[
%    \frac{\left(1+t\right) \left(1-v\right)}
%    {t-v}
%    \,T^{1,2}_{l}\left(\frac{N-3}{2}\,(1+t)-1,N-3\right)
%    \,T^{2,1}_{l}\left(\frac{N-3}{2}\,(1+v),N-3\right)
%  \right.
%\end{equation*}
%\begin{equation*}
%\left.
%  - \frac{(1+v)(1-t)}
%    {t-v}
%  \,T^{1,2}_{l}\left(\frac{N-3}{2}\,(1+v)-1,N-3\right)
%  \,T^{2,1}_{l}\left(\frac{N-3}{2}\,(1+t),N-3\right)
%\right]=
%\end{equation*}
\begin{equation*}
  \sum_{l=m-1}^{m+n-2} B_{l,N-2}\,
  \left[
    \frac{(1+t)(1-v)}
    {t-v}
    \,Q^{1,2}_{l,N-2}\left(t\right)
    \,Q^{2,1}_{l,N-2}\left(v+\frac{2}{N-3}\right)
  \right.
\end{equation*}
\begin{equation*}
\left.
  - \frac{(1+v)(1-t)}
    {t-v}
    \,Q^{1,2}_{l,N-2}\left(v\right)
    \,Q^{2,1}_{l,N-2}\left(t+\frac{2}{N-3}\right)
\right].
\end{equation*}

Отсюда, возвращаясь к \eqref{norm2}
\begin{equation*}
   V^{-1}_{m,n}  = \frac{8x(N-x-1)}{N(N-1)(N-2)(n+1)}
   \sum_{j=1}^{N-2} \left| I_{m+n-2,N-2}(x-1,j-1) - I_{m-2,N-2}(x-1,j-1) \right|.
\end{equation*}
Получаем
\begin{equation*}
   \hat{V}^{-1}_{m,n}  =
   \frac{8x(N-x-1)}{N(N-1)(N-2)(n+1)}
   \sum_{j=1}^{N-2} \left| I_{m+n-2,N-2}(x,j) - I_{m-2,N-2}(x,j) \right| =
\end{equation*}
\begin{equation*}
 \sum_{v\in\Omega_{N-3}} \left|
  \frac{8\left( \frac{N-3}{2}\,(1+t)\right)\left(N- 1-\frac{N-3}{2}\,(1+t)\right)}{N(N-1)(N-2)(n+1)} \,
  \sum_{l=m-1}^{m+n-2} B_{l,N-2}\,
  \left[
    \frac{(1+t)(1-v)}
    {t-v}
    \,Q^{1,2}_{l,N-2}\left(t\right)
    \,Q^{2,1}_{l,N-2}\left(v+\frac{2}{N-3}\right)
  \right.
  \right.
\end{equation*}
\begin{equation*}
\left.
\left.
  - \frac{(1+v)(1-t)}
    {t-v}
    \,Q^{1,2}_{l,N-2}\left(v\right)
    \,Q^{2,1}_{l,N-2}\left(t+\frac{2}{N-3}\right)
\right]
\right| =
\end{equation*}
\begin{equation*}
 \sum_{v\in\Omega_{N-3}} \left|
  \frac{4(N-3)^2 \,\left( 1+t\right)\left(1-t+\frac{4}{N-3}\right)}{N(N-1)(N-2)(n+1)} \,
  \sum_{l=m-1}^{m+n-2}
  \frac{(l+2)(l+3)}{16(l+1)}\,\frac{(N-4)^{[l]}}{(N+l)^{[l]}}   \,
  \right.
\end{equation*}
\begin{equation*}
\left.
  \left[
    \frac{(1+t)(1-v)}
    {t-v}
    \,Q^{1,2}_{l,N-2}\left(t\right)
    \,Q^{2,1}_{l,N-2}\left(v+\frac{2}{N-3}\right)
  \right.
  \right.
\end{equation*}
\begin{equation*}
\left.
\left.
  - \frac{(1+v)(1-t)}
    {t-v}
    \,Q^{1,2}_{l,N-2}\left(v\right)
    \,Q^{2,1}_{l,N-2}\left(t+\frac{2}{N-3}\right)
\right]
\right| =
\end{equation*}
\begin{equation*}
\frac{1}{N}
 \sum_{v\in\Omega_{N-3}} \left|
 \sum_{l=m-1}^{m+n-2}
 \frac{(N-3)^2}{(N-1)(N-2)}\,
 \frac{(l+2)(l+3)}{4(n+1)(l+1)}\,
 \frac{(N-4)^{[l]}}{(N+l)^{[l]}}\,
 \left( 1+t\right)\left(1-t+\frac{4}{N-3}\right)
  \right.
\end{equation*}
\begin{equation*}
\left.
  \left[
    \frac{(1+t)(1-v)}
    {t-v}
    \,Q^{1,2}_{l,N-2}\left(t\right)
    \,Q^{2,1}_{l,N-2}\left(v+\frac{2}{N-3}\right)
  \right.
  \right.
\end{equation*}
\begin{equation*}
\left.
\left.
  - \frac{(1+v)(1-t)}
    {t-v}
    \,Q^{1,2}_{l,N-2}\left(v\right)
    \,Q^{2,1}_{l,N-2}\left(t+\frac{2}{N-3}\right)
\right]
\right| =
\end{equation*}
\begin{equation*}
\frac{1}{N}
 \sum_{v\in\Omega_{N-3}} \left|
 \sum_{l=m-1}^{m+n-2}
 \hat{B}_{l,N}
 \left( 1+t\right)\left(1-t+\frac{4}{N-3}\right)
  \right.
\end{equation*}
\begin{equation*}
\left.
  \left[
    \frac{(1+t)(1-v)}
    {t-v}
    \,Q^{1,2}_{l,N-2}\left(t\right)
    \,Q^{2,1}_{l,N-2}\left(v+\frac{2}{N-3}\right)
  \right.
  \right.
\end{equation*}
\begin{equation*}
\left.
\left.
  - \frac{(1+v)(1-t)}
    {t-v}
    \,Q^{1,2}_{l,N-2}\left(v\right)
    \,Q^{2,1}_{l,N-2}\left(t+\frac{2}{N-3}\right)
\right]
\right|,
\end{equation*}
где
$$\hat{B}_{l,N} = \frac{(N-3)^2}{(N-1)(N-2)}\,\frac{(l+2)(l+3)}{4(n+1)(l+1)}\,\frac{(N-4)^{[l]}}{(N+l)^{[l]}} \leq c.$$
Далее,
\begin{equation*}
   \hat{V}^{-1}_{m,n}  =
\frac{1}{N}
 \sum_{v\in\Omega_{N-3}} \left|
 \sum_{l=m-1}^{m+n-2}
 \hat{B}_{l,N}
 \left( 1+t\right)\left(1-t+\frac{4}{N-3}\right)
  \right.
\end{equation*}
\begin{equation*}
\left.
  \left[
    \frac{(1+t)(1-v)}
    {t-v}
    \,Q^{1,2}_{l,N-2}\left(t\right)
    \,Q^{2,1}_{l,N-2}\left(v+\frac{2}{N-3}\right)
  \right.
  \right.
\end{equation*}
\begin{equation*}
\left.
\left.
  - \frac{(1+v)(1-t)}
    {t-v}
    \,Q^{1,2}_{l,N-2}\left(v\right)
    \,Q^{2,1}_{l,N-2}\left(t+\frac{2}{N-3}\right)
\right]
\right|,
\end{equation*}

===================
\newpage
\newpage


\begin{equation*}
   \hat{V}^{-1}_{m,n}  \leq
\frac{1}{N}
 \sum_{l=m-1}^{m+n-2}
 \hat{B}_{l,N}
 \left( 1+t\right)\left(1-t+\frac{4}{N-3}\right)\, W(t,v),
\end{equation*}
где
\begin{equation*}
W(t,v) = W_{m,n,N}(t,v)=
\sum_{v\in \Omega_{N-3}}
  \left[
    \frac{(1+t)(1-v)}
    {t-v}\,
    \left| Q^{1,2}_{l,N-2}\left(t\right)\right|\,
    \left| Q^{2,1}_{l,N-2}\left(v+\frac{2}{N-3}\right)
  \right|
  \right.
\end{equation*}
\begin{equation*}
\left.
  + \frac{(1+v)(1-t)}
    {t-v}
    \,
    \left| Q^{1,2}_{l,N-2}\left(v\right) \right|\,
    \left|Q^{2,1}_{l,N-2}\left(t+\frac{2}{N-3}\right)
    \right|
\right],
\end{equation*}
а точки сетки $\Omega_{N-3} = \left\{ -1 + \frac{2p}{N-1}\right\}_{p=1}^{N-2}$ --- не подходят к $\pm 1$ ближе чем на $\frac{2}{N-1}$.
%Так как в случае $n \leq a N^{\frac12}$, $\frac{2}{N-1} \leq \frac{2}{(an)^2-1}$


1. Пусть $t \in [0, 1-2n^{-2}]$, тогда


\newpage
\newpage

Рассмотрим тогда выражение
\begin{equation*}
  I_{m+n-2,N-2}(x-1,j-1) - I_{m-2,N-2}(x-1,j-1) =
\end{equation*}
\begin{equation*}
  \frac{2}{N-3}\,
  \sum_{l=m-1}^{m+n-2} B_{l,N-2}\,
  \frac{T_{l+1}^1(x-1,N-2) T_{l}^1(j-1,N-2) - T_{l+1}^1(j-1,N-2) T_{l}^1(x-1,N-2)}{x-j} =
\end{equation*}
\begin{equation*}
  \frac{2}{N-3}\,\sum_{l=0}^{k} B_{l,N-2}\, \left[
    \frac{(x-1)(N-2-j)}{x-j}\,T^{1,2}_{l}(x-2,N-3)\,T^{2,1}_{l}(j-1,N-3)
  \right.
\end{equation*}
\begin{equation*}
\left.
  - \frac{(j-1)(N-2-x)}{x-j}\,T^{1,2}_{l}(j-2,N-3)\,T^{2,1}_{l}(x-1,N-3)
\right].
\end{equation*}

Проведя замену переменных $\{x = \frac{N-3}{2}\,(1+t), j = \frac{N-3}{2}\,(1+v)\}$, перепишем эти выражения в терминах полиномов $Q_n^{\alpha,\beta}(t)$:
\begin{equation*}
  I_{m+n-2,N-2}\left(t-\frac{2}{N-3},v-\frac{2}{N-3} \right) - I_{m-2,N-2}\left(t-\frac{2}{N-3},v-\frac{2}{N-3} \right) =
\end{equation*}
%\begin{equation*}
%  \frac{2}{N-3}\,\sum_{l=m-1}^{m+n-2} B_{l,N-2}\,
%  \left[
%    \frac{(\frac{N-3}{2}\,(1+t)-1)(N-2-\frac{N-3}{2}\,(1+v))}{\frac{N-3}{2}\,(1+t)-\frac{N-3}{2}\,(1+v)}
%    \,T^{1,2}_{l}\left(\frac{N-3}{2}\,(1+t)-2,N-3\right)
%    \,T^{2,1}_{l}\left(\frac{N-3}{2}\,(1+v)-1,N-3\right)
%  \right.
%\end{equation*}
%\begin{equation*}
%\left.
%  - \frac{(\frac{N-3}{2}\,(1+v)-1)(N-2-\frac{N-3}{2}\,(1+t))}{\frac{N-3}{2}\,(1+t)-\frac{N-3}{2}\,(1+v)}
%  \,T^{1,2}_{l}\left(\frac{N-3}{2}\,(1+v)-2,N-3\right)
%  \,T^{2,1}_{l}\left(\frac{N-3}{2}\,(1+t)-1,N-3\right)
%\right].
%\end{equation*}
\begin{equation*}
  \sum_{l=m-1}^{m+n-2} B_{l,N-2}\,
  \left[
    \frac{\left[1+t-\frac{2}{N-3}\right] \left[1-\left(v-\frac{2}{N-3}\right)\right]}
    {t-v}
    \,T^{1,2}_{l}\left(\frac{N-3}{2}\,(1+t)-2,N-3\right)
    \,T^{2,1}_{l}\left(\frac{N-3}{2}\,(1+v)-1,N-3\right)
  \right.
\end{equation*}
\begin{equation*}
\left.
  - \frac{\left[1+v-\frac{2}{N-3}\right] \left[1-\left(t-\frac{2}{N-3}\right)\right]}
    {t-v}
  \,T^{1,2}_{l}\left(\frac{N-3}{2}\,(1+v)-2,N-3\right)
  \,T^{2,1}_{l}\left(\frac{N-3}{2}\,(1+t)-1,N-3\right)
\right]=
\end{equation*}
\begin{equation*}
  \sum_{l=m-1}^{m+n-2} B_{l,N-2}\,
  \left[
    \frac{\left[1+t-\frac{2}{N-3}\right] \left[1-\left(v-\frac{2}{N-3}\right)\right]}
    {t-v}
    \,Q^{1,2}_{l,N-2}\left(t-\frac{2}{N-3}\right)
    \,Q^{2,1}_{l,N-2}\left(v\right)
  \right.
\end{equation*}
\begin{equation*}
\left.
  - \frac{\left[1+v-\frac{2}{N-3}\right] \left[1-\left(t-\frac{2}{N-3}\right)\right]}
    {t-v}
    \,Q^{1,2}_{l,N-2}\left(v-\frac{2}{N-3}\right)
    \,Q^{2,1}_{l,N-2}\left(t\right)
\right].
\end{equation*}




==============

\begin{equation*}
  I_{m+n-2,N-2}(x-1,j-1) - I_{m-2,N-2}(x-1,j-1) =
\end{equation*}
\begin{equation*}
\sum_{l=m-1}^{m+n-2}
  B_{l,N}\,
  \frac{(x-1)\,T^{1,2}_{l}(x-2,N-3) T_{l}^1(j-1,N-2) - (j-1)\,T^{1,2}_{l}(j-2,N-3) T_{l}^1(x-1,N-2)}{x-j},
\end{equation*}

==============

\begin{equation*}
  I_{k,N}(x,j) = \frac{2}{N-1}\,
  \sum_{l=0}^{k} B_{l,N}\,
  \frac{T_{l+1}^1(x) T_{l}^1(j) - T_{l+1}^1(j) T_{l}^1(x)}{x-j} =
\end{equation*}
\begin{equation*}
  \frac{2}{N-1}\,\sum_{l=0}^{k} B_{l,N}\, \left[
    \frac{x(N-1-j)}{x-j}\,T^{1,2}_{l}(x-1,N-1)\,T^{2,1}_{l}(j,N-1)
  \right.
\end{equation*}
\begin{equation*}
\left.
  - \frac{j(N-1-x)}{x-j}\,T^{1,2}_{l}(j-1,N-1)\,T^{2,1}_{l}(x,N-1)
\right].
\end{equation*}





==============

\begin{equation*}
  I_{k,N}(t,v) =
  \sum_{l=0}^{k} B_{l,N} \,
  \left[
    \frac{(1+t)(1-v)}{t-v}\,Q^{1,2}_{l}(t)\,Q^{2,1}_{l}\left(v+\frac{2}{N-1}\right)
  \right.
\end{equation*}
\begin{equation*}
\left.
  - \frac{(1+v)(1-t)}{t-v}\,Q^{1,2}_{l}(v)\,Q^{2,1}_{l}\left(t+\frac{2}{N-1}\right)
\right],
\end{equation*}

\begin{equation*}
  I_{m+n-2,N-2}(x-1,j-1) - I_{m-2,N-2}(x-1,j-1) =
\end{equation*}
\begin{equation*}
\sum_{l=m-1}^{m+n-2}
  \hat{B}_{l,N}\,
  \frac{(N-j)\,T^{2,1}_{l}(j-1,N-3) T_{l}^1(x-1,N-2)
    - (N-x)\,T^{2,1}_{l}(x-1,N-3) T_{l}^1(j-1,N-2)}{x-j}.
\end{equation*}


Отсюда, возвращаясь к \eqref{limVPf}, имеем
\begin{equation*}
  \mathcal{V}^{-1}_{m,n}(f,x) = a_N(f,x) +
   \frac{8x(N-x-1)}{(n+1)N(N-1)(N-2)}
   \sum_{j=1}^{N-2} g(j) \sum_{l=m-1}^{m+n-2}
  B_{l,N}\,\times
\end{equation*}
\begin{equation*}
  \frac{(x-1)\,T^{1,2}_{l}(x-2,N-3) T_{l}^1(j-1,N-2) - (j-1)\,T^{1,2}_{l}(j-2,N-3) T_{l}^1(x-1,N-2)}{x-j} =
\end{equation*}

\begin{equation*}
  a_N(f,x) + \frac{1}{n+1}
   \sum_{l=m-1}^{m+n-2} x(N-x-1) W_{l,N} \sum_{j=1}^{N-2} g(j)
  \,\times
\end{equation*}
\begin{equation*}
  \frac{(x-1)\,T^{1,2}_{l}(x-2,N-3) T_{l}^1(j-1,N-2) - (j-1)\,T^{1,2}_{l}(j-2,N-3) T_{l}^1(x-1,N-2)}{x-j},
\end{equation*}
где
\begin{equation*}
 W_{l,N} = \frac{8}{N(N-1)(N-2)} \, \frac{(l+2)(l+3)}{8(l+1)}\, \frac{(N-2)^{[l]}}{(N+l+2)^{[l]}}
=
 \frac{(l+2)(l+3)}{(l+1)}\, \frac{(N-2)^{[l]}}{(N+l+2)^{[l+3]}}.
\end{equation*}

Либо иначе

\begin{equation*}
  \mathcal{V}^{-1}_{m,n}(f,x) = a_N(f,x) +
   \frac{8x(N-x-1)}{(n+1)N(N-1)(N-2)}
   \sum_{j=1}^{N-2} g(j) \sum_{l=m-1}^{m+n-2}
  \hat{B}_{l,N}\,\times
\end{equation*}
\begin{equation*}
  \frac{(N-j)\,T^{2,1}_{l}(j-1,N-3) T_{l}^1(x-1,N-2) - (N-x)\,T^{2,1}_{l}(x-1,N-3) T_{l}^1(j-1,N-2)}{x-j} =
\end{equation*}

\begin{equation*}
  a_N(f,x) + \frac{1}{n+1}
   \sum_{l=m-1}^{m+n-2} x(N-x-1) \hat{W}_{l,N} \sum_{j=1}^{N-2} g(j)
  \,\times
\end{equation*}
\begin{equation*}
  \frac{(N-j)\,T^{2,1}_{l}(j-1,N-3) T_{l}^1(x-1,N-2) - (N-x)\,T^{2,1}_{l}(x-1,N-3) T_{l}^1(j-1,N-2)}{x-j},
\end{equation*}
где
\begin{equation*}
 \hat{W}_{l,N} = \frac{8}{N(N-1)(N-2)} \, \frac{l+3}{8} \,\frac{(N-2)^{[l]}}{(N+l+2)^{[l]}} =
 (l+3)\,\frac{(N-2)^{[l]}}{(N+l+2)^{[l+3]}}.
\end{equation*}







\newpage
\newpage

Обозначив для краткости $B_{l,N} = \frac{(l+2)(l+3)}{16(l+1)}\,\frac{(N-2)^{[l]}}{(N+l+2)^{[l]}} $
получим

\begin{equation*}
  I_{k,N}(x,j) = \frac{2}{N-1}\,
  \sum_{l=0}^{k} B_{l,N}\,
  \frac{T_{l+1}^1(x) T_{l}^1(j) - T_{l+1}^1(j) T_{l}^1(x)}{x-j} =
\end{equation*}
\begin{equation*}
  \frac{2}{N-1}\,\sum_{l=0}^{k} B_{l,N}\, \left[
    \frac{x(N-1-j)}{x-j}\,T^{1,2}_{l}(x-1,N-1)\,T^{2,1}_{l}(j,N-1)
  \right.
\end{equation*}
\begin{equation*}
\left.
  - \frac{j(N-1-x)}{x-j}\,T^{1,2}_{l}(j-1,N-1)\,T^{2,1}_{l}(x,N-1)
\right].
\end{equation*}

Проведя замену переменных $\{x = \frac{(N-1)}{2}\,(1+t), j = \frac{(N-1)}{2}\,(1+v)\}$, перепишем эти выражения в терминах полиномов $Q_n^{\alpha,\beta}(t)$:




Рассмотрим тогда выражение
\begin{equation*}
  I_{m+n-2,N-2}(x-1,j-1) - I_{m-2,N-2}(x-1,j-1) =
\end{equation*}
\begin{equation*}
\sum_{l=m-1}^{m+n-2}
  B_{l,N}\,
  \frac{(x-1)\,T^{1,2}_{l}(x-2,N-3) T_{l}^1(j-1,N-2) - (j-1)\,T^{1,2}_{l}(j-2,N-3) T_{l}^1(x-1,N-2)}{x-j},
\end{equation*}

\begin{equation*}
  I_{m+n-2,N-2}(x-1,j-1) - I_{m-2,N-2}(x-1,j-1) =
\end{equation*}
\begin{equation*}
\sum_{l=m-1}^{m+n-2}
  \hat{B}_{l,N}\,
  \frac{(N-j)\,T^{2,1}_{l}(j-1,N-3) T_{l}^1(x-1,N-2)
    - (N-x)\,T^{2,1}_{l}(x-1,N-3) T_{l}^1(j-1,N-2)}{x-j}.
\end{equation*}


Отсюда, возвращаясь к \eqref{limVPf}, имеем
\begin{equation*}
  \mathcal{V}^{-1}_{m,n}(f,x) = a_N(f,x) +
   \frac{8x(N-x-1)}{(n+1)N(N-1)(N-2)}
   \sum_{j=1}^{N-2} g(j) \sum_{l=m-1}^{m+n-2}
  B_{l,N}\,\times
\end{equation*}
\begin{equation*}
  \frac{(x-1)\,T^{1,2}_{l}(x-2,N-3) T_{l}^1(j-1,N-2) - (j-1)\,T^{1,2}_{l}(j-2,N-3) T_{l}^1(x-1,N-2)}{x-j} =
\end{equation*}

\begin{equation*}
  a_N(f,x) + \frac{1}{n+1}
   \sum_{l=m-1}^{m+n-2} x(N-x-1) W_{l,N} \sum_{j=1}^{N-2} g(j)
  \,\times
\end{equation*}
\begin{equation*}
  \frac{(x-1)\,T^{1,2}_{l}(x-2,N-3) T_{l}^1(j-1,N-2) - (j-1)\,T^{1,2}_{l}(j-2,N-3) T_{l}^1(x-1,N-2)}{x-j},
\end{equation*}
где
\begin{equation*}
 W_{l,N} = \frac{8}{N(N-1)(N-2)} \, \frac{(l+2)(l+3)}{8(l+1)}\, \frac{(N-2)^{[l]}}{(N+l+2)^{[l]}}
=
 \frac{(l+2)(l+3)}{(l+1)}\, \frac{(N-2)^{[l]}}{(N+l+2)^{[l+3]}}.
\end{equation*}

Либо иначе

\begin{equation*}
  \mathcal{V}^{-1}_{m,n}(f,x) = a_N(f,x) +
   \frac{8x(N-x-1)}{(n+1)N(N-1)(N-2)}
   \sum_{j=1}^{N-2} g(j) \sum_{l=m-1}^{m+n-2}
  \hat{B}_{l,N}\,\times
\end{equation*}
\begin{equation*}
  \frac{(N-j)\,T^{2,1}_{l}(j-1,N-3) T_{l}^1(x-1,N-2) - (N-x)\,T^{2,1}_{l}(x-1,N-3) T_{l}^1(j-1,N-2)}{x-j} =
\end{equation*}

\begin{equation*}
  a_N(f,x) + \frac{1}{n+1}
   \sum_{l=m-1}^{m+n-2} x(N-x-1) \hat{W}_{l,N} \sum_{j=1}^{N-2} g(j)
  \,\times
\end{equation*}
\begin{equation*}
  \frac{(N-j)\,T^{2,1}_{l}(j-1,N-3) T_{l}^1(x-1,N-2) - (N-x)\,T^{2,1}_{l}(x-1,N-3) T_{l}^1(j-1,N-2)}{x-j},
\end{equation*}
где
\begin{equation*}
 \hat{W}_{l,N} = \frac{8}{N(N-1)(N-2)} \, \frac{l+3}{8} \,\frac{(N-2)^{[l]}}{(N+l+2)^{[l]}} =
 (l+3)\,\frac{(N-2)^{[l]}}{(N+l+2)^{[l+3]}}.
\end{equation*}








%%%%%%%%%%%%%%%%%%%%%%%%%%%%%%%%%%%%%%%%%%%%%%%%%%%%%%%%%%%%%%%%%%%
%%%%%%%%%%%%%%%%%%%%%%%%%%%%%%%%%%%%%%%%%%%%%%%%%%%%%%%%%%%%%%%%%%%
%%%%%%%%%%%%%%%%%%%%%%%%%%%%%%%%%%%%%%%%%%%%%%%%%%%%%%%%%%%%%%%%%%%
\newpage
\newpage


Справедливы следующие рекуррентные формулы:

\begin{equation*}
  A^{1,1}_{n,N} T^{1,1}_{n+1}(x,N) = (x-B^{1,1}_{n,N}) T^{1,1}_{n}(x,N) - C^{1,1}_{n,N} T^{1,1}_{n-1}(x,N),
\end{equation*}
где $T^{1,1}_{0}(x,N)=1$, $T^{1,1}_{1}(x,N)=-2+4x/(N-1)$,
\begin{equation*}
  A^{1,1}_{n,N} = \frac{(n+1)(n+3)(N-n-1)}{(2n+3)(2n+5)},
\quad
  C^{1,1}_{n,N} = \frac{(n+1)(N+n+2)}{2(2n+3)},
\end{equation*}
\begin{equation*}
  B^{1,1}_{n,N} = A^{1,1}_{n,N}\,\frac{n+2}{n+1}+C^{1,1}_{n,N}\,\frac{n}{n+1} =
\end{equation*}
\begin{equation*}
  \frac{(n+2)(n+3)(N-n-1)}{(2n+3)(2n+5)}+ \frac{n(N+n+2)}{2(2n+3)},
\end{equation*}

\begin{equation*}
  A^{1,2}_{n,N} T^{1,2}_{n+1}(x,N) = (x-B^{1,2}_{n,N}) T^{1,2}_{n}(x,N) - C^{1,2}_{n,N} T^{1,2}_{n-1}(x,N),
\end{equation*}
где $T^{1,2}_{0}(x,N)=1$, $T^{1,2}_{1}(x,N)=-3+5x/(N-1)$,
\begin{equation*}
  A^{1,2}_{n,N} = \frac{(n+1)(n+4)(N-n-1)}{2(n+2)(2n+5)},
\quad
  C^{1,2}_{n,N} = \frac{(n+1)(N+n+3)}{2(2n+3)},
\end{equation*}
\begin{equation*}
  B^{1,2}_{n,N} = A^{1,2}_{n,N}\,\frac{n+3}{n+1}+C^{1,2}_{n,N}\,\frac{n}{n+2}=
\end{equation*}
\begin{equation*}
  \frac{(n+3)(n+4)(N-n-1)}{2(n+2)(2n+5)}+\frac{(n+1)(N+n+3)}{2(2n+3)}\,\frac{n}{n+2},
\end{equation*}


\begin{equation*}
  A^{2,1}_{n,N} T^{2,1}_{n+1}(x,N) = (x-B^{2,1}_{n,N}) T^{2,1}_{n}(x,N) - C^{2,1}_{n,N} T^{2,1}_{n-1}(x,N),
\end{equation*}
где $T^{2,1}_{0}(x,N)=1$, $T^{2,1}_{1}(x,N)=-2+5x/(N-1)$,
\begin{equation*}
  A^{2,1}_{n,N} = \frac{(n+1)(n+4)(N-n-1)}{2(n+2)(2n+5)},
\quad
  C^{2,1}_{n,N} = \frac{(n+1)(N+n+3)}{2(2n+3)},
\end{equation*}
\begin{equation*}
  B^{2,1}_{n,N} = A^{2,1}_{n,N}\,\frac{n+2}{n+1}+C^{2,1}_{n,N}\,\frac{n}{n+1}=
\end{equation*}
\begin{equation*}
  \frac{(n+4)(N-n-1)}{2(2n+5)}+\frac{n(N+n+3)}{2(2n+3)},
\end{equation*}


Справедливы следующие рекуррентные формулы:
\begin{equation*}
  (n+1) T^{1,1}_{n+1}(x,N) + (n+2)T^{1,1}_{n}(x,N) = \frac{2n+4}{N-1}\,x\,T^{1,2}_{n}(x-1,N-1),
\end{equation*}
\begin{equation*}
  (n+2) T^{1,1}_{n}(x,N) - (n+1)T^{1,1}_{n+1}(x,N) = \frac{2n+2}{N-1}\,(N-1-x)\,T^{2,1}_{n}(x,N-1).
\end{equation*}

\begin{equation*}
  (n+1) T^{1,2}_{n+1}(x,N) + (n+3)T^{1,2}_{n}(x,N) = \frac{2n+5}{N-1}\,x\,T^{1,3}_{n}(x-1,N-1),
\end{equation*}
\begin{equation*}
  (n+2) T^{1,2}_{n}(x,N) - (n+1)T^{1,2}_{n+1}(x,N) = \frac{2n+3}{N-1}\,(N-1-x)\,T^{2,2}_{n}(x,N-1).
\end{equation*}

\begin{equation*}
  (n+1) T^{2,1}_{n+1}(x,N) + (n+3)T^{2,1}_{n}(x,N) = \frac{2n+5}{N-1}\,x\,T^{2,2}_{n}(x-1,N-1),
\end{equation*}
\begin{equation*}
  (n+3) T^{2,1}_{n}(x,N) - (n+1)T^{2,1}_{n+1}(x,N) = \frac{2n+3}{N-1}\,(N-1-x)\,T^{3,1}_{n}(x,N-1).
\end{equation*}

Соотношения из желтой книги:
\begin{equation*}
  (n+1)T^{\alpha,\beta}_{n+1}(x,N) + (n+\beta+1)T^{\alpha,\beta}_{n}(x,N) = \frac{2n+\alpha+\beta+2}{N-1}\,x\,T^{\alpha,\beta+1}_{n}(x-1,N-1)
\end{equation*}
%\begin{equation*}
%  nT^{\alpha,\beta}_{n}(x,N) + (n+\beta)T^{\alpha,\beta}_{n-1}(x,N) = \frac{2n+\alpha+\beta}{N-1}\,x\,T^{\alpha,\beta+1}_{n-1}(x-1,N-1)
%\end{equation*}
\begin{equation*}
  (n+\alpha+1)T^{\alpha,\beta}_{n}(x,N) - (n+1)T^{\alpha,\beta}_{n+1}(x,N) = \frac{2n+\alpha+\beta+2}{N-1}\,(N-1-x)\,T^{\alpha+1,\beta}_{n}(x,N-1)
\end{equation*}
%\begin{equation*}
%  (n+\alpha)T^{\alpha,\beta}_{n-1}(x,N) - nT^{\alpha,\beta}_{n}(x,N) = \frac{2n+\alpha+\beta}{N-1}\,(N-1-x)\,T^{\alpha+1,\beta}_{n-1}(x,N-1)
%\end{equation*}
Складывая эти два равенства, получаем
%\begin{equation*}
%  (n+\beta+1)T^{\alpha,\beta}_{n}(x,N) + (n+\alpha+1)T^{\alpha,\beta}_{n}(x,N)
%  = \frac{2n+\alpha+\beta+2}{N-1} \, \left[
%    x\,T^{\alpha,\beta+1}_{n}(x-1,N-1) + (N-1-x)\,T^{\alpha+1,\beta}_{n}(x,N-1)\right],
%\end{equation*}
\begin{equation*}
  T^{\alpha,\beta}_{n}(x,N)
  = \frac{x\,T^{\alpha,\beta+1}_{n}(x-1,N-1) + (N-1-x)\,T^{\alpha+1,\beta}_{n}(x,N-1)}{N-1}.
\end{equation*}
\begin{equation}\label{rec1221}
  T^{1,1}_{n}(x,N)
  = \frac{x\,T^{1,2}_{n}(x-1,N-1) + (N-1-x)\,T^{2,1}_{n}(x,N-1)}{N-1}.
\end{equation}
%%%%%%%%%%%%%%%%%%%%%%%%%%%%
Разностные свойства:
\begin{equation*}
 T^{\alpha,\beta}_{n}(x,N) =
  \frac{(\alpha+\beta+2)x - (\beta+1)(N-1)}{n(N-1)}\,T^{\alpha+1,\beta+1}_{n-1}(x-1,N-1)-
\end{equation*}
\begin{equation*}
  \frac{(\beta+1-x)(N-1-x)(n+\alpha+\beta+2)}{n(N-1)(N-2)}\,T^{\alpha+2,\beta+2}_{n-2}(x-1,N-2),
\end{equation*}

\begin{equation*}
 T^{1,1}_{n}(x,N) =
  2\,\frac{2x - N+1}{n(N-1)}\,T^{2,2}_{n-1}(x-1,N-1)-
  \frac{(2-x)(N-1-x)(n+4)}{n(N-1)(N-2)}\,T^{3,3}_{n-2}(x-1,N-2),
\end{equation*}


%%%%%%%%%%%%%%%%%%%%%%%%%%%%%%%%%%%%%%%%%%%%%%%%%%%%%%%%%%%%%%%%%%%
%%%%%%%%%%%%%%%%%%%%%%%%%%%%%%%%%%%%%%%%%%%%%%%%%%%%%%%%%%%%%%%%%%%
%%%%%%%%%%%%%%%%%%%%%%%%%%%%%%%%%%%%%%%%%%%%%%%%%%%%%%%%%%%%%%%%%%%
\newpage
\newpage




Рассмотрим разность коэффициентов
\begin{equation*}
  B_{l,N}-B_{l-1,N} = \frac{l+3}{16} \, \frac{(N-3)^{[l+1]}}{(N+l)^{[l]}}-
  \frac{l+2}{16} \, \frac{(N-3)^{[l]}}{(N+l-1)^{[l-1]}} =
\end{equation*}
\begin{equation*}
  \frac{(N-3)^{[l]}}{16(N+l)^{[l]}}\left(
    (N-l-3)(l+3) - (N+l)(l+2)
  \right)=
\end{equation*}
\begin{equation*}
  \frac{(N-3)^{[l]}}{(N+l)^{[l]}}\, \frac{N-2l(l+4)-9}{16}.
\end{equation*}

















\newpage
\newpage
























\newpage
\vspace*{3mm}
%%%%%%%%%%%%%%%%%%%%%%%%%%%%%%%%%%%%%%%%%%%%%%%%%%%%%%%%%%%%%%%
\textbf{Библиографический список}
%%%%%%%%%%%%%%%%%%%%%%%%%%%%%%%%%%%%%%%%%%%%%%%%%%%%%%%%%%%%%%%

1. \textit{И.\,И.~Шарапудинов.} Смешанные ряды по классическим ортогональным полиномам // Дагестанские Электронные Математические Известия, 2015. Т.~3 (Специальный выпуск), С.~1-254.

2. \textit{И.\,И.~Шарапудинов.} Асимптотические свойства и весовые оценки для ортогональных многочленов Чебыш\"ева–Хана // Математический сборник, 1991. Т.~182, №~3, С.~408–420.

3. \textit{И.\,И.~Шарапудинов.} Об ограниченности в $C[-1,1]$ средних Валле-Пуссена для дискретных сумм Фурье–Чебыш\"eва // Математический сборник, 1996. Т.~187, №~1, С.~143–160.

4. \textit{Т.\,И.~Шарапудинов.} Конечные предельные ряды по полиномам Чебыш\"ева, ортогональным на равномерных сетках // Изв. Сарат. ун-та. Нов. сер. Сер. Математика. Механика. Информатика, 2013. Т. 13, вып.~1, ч.~2, С.~104-108.



\newpage

\textsf{УДК~517.51}


Султанахмедов М. С. Аппроксимативные свойства средних Валле -- Пуссена сумм Фурье по полиномам Чебыш\"ева, ортогональным на равномерной сетке

Sultanakhmedov M. S. Approximative Properties of de la Vallee - Poussin means of Fourier sums by Chebyshev Polynomials Orthogonal on a Uniform Grid

\vspace*{5mm}
%%%%%%%%%%%%%%%%%%%%%%%%%%%%%%%%%%%%%%%%%%%%%%%%%%%%%%%%%%%%%%%%%%%%
%Полные сведения об авторах
%%%%%%%%%%%%%%%%%%%%%%%%%%%%%%%%%%%%%%%%%%%%%%%%%%%%%%%%%%%%%%%%%%%%

\noindent%
\textit{Султанахмедов Мурад Салихович}~--- научный сотрудник Отдела математики и информатики, Дагестанский научный центр РАН. E-mail: sultanakhmedov@gmail.com; 367000, г. Махачкала, ул. Гаджиева, 45. \vspace*{3pt}

\noindent%
\textit{Sultanakhmedov Murad Salikhovich}~--- scientific worker of Department of Mathematics and Computer Science, Daghestan Scientific Center of RAS. E-mail: sultanakhmedov@gmail.com; 367000, Makhachkala, Gadzhiev st., 45.



Ответственный за переписку Султанахмедов Мурад Салихович
%E-mail: sultanakhmedov@gmail.com
%сот.(или д.) тел. 8-988-2000-778
\end{document} 