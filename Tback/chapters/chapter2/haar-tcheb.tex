\chapter{Системы функций, ортогональные по Соболеву}
%\begin{abstract}
%Для заданной ортонормированной на  $(a,b)$  c весом   $\rho(x)$  системы функций $\left\{\varphi_k(x)\right\}$ и натурального   $r$ построена ассоциированная с ней новая система функций $\left\{\varphi_{r,k}(x)\right\}_{k=0}^\infty$, ортонормированная относительно скалярного произведения типа Соболева следующего вида
%\begin{equation*}
%<f,g>=\sum_{\nu=0}^{r-1}f^{(\nu)}(a)g^{(\nu)}(a)+\int_{a}^{b} f^{(r)}(t)g^{(r)}(t)\rho(t) dt.
%\end{equation*}
%Исследованы вопросы сходимости ряда Фурье по системе $\left\{\varphi_{r,k}(x)\right\}_{k=0}^\infty$. Рассмотрены важные частные случаи систем типа $\left\{\varphi_{r,k}(x)\right\}_{k=0}^\infty$, порожденныx функциями  Хаара и полиномами Чебышева $T_n(x)=\cos(n\arccos x)$. В этих случаях  для порожденных функций $\varphi_{r,k}(x)$ получены явные представления, которые могут быть использованы при исследовании асимптотических свойств функций $\varphi_{r,k}(x)$ при $k\to\infty$ и  аппроксимативных свойств сумм Фурье по системе $\left\{\varphi_{r,k}(x)\right\}_{k=0}^\infty$.  Основное внимание уделено исследованию аппроксимативных свойств рядов Фурье по системам типа $\left\{\varphi_{r,k}(x)\right\}_{k=0}^\infty$, порожденным функциями  Хаара и полиномами Чебышева.
%
%Библиография:  27 названий.
%
%\end{abstract}

\section{Смешанные ряды и ортогональность по Соболеву}

В ряде наших работ  \cite{Haar-Tcheb-Shar11} -- \cite{Haar-Tcheb-Shar18}  были введены так называемые \textit{смешанные} ряды по классическим ортогональным полиномам, частичные суммы которых  обладают свойством совпадения их значений в концах области ортогональности  со значениями исходной функции. Общая идея, которая лежит в основе построения смешанных рядов, заключается в следующем. Предположим, что система функций  $\left\{\varphi_k(x)\right\}$ ортонормирована  на $(a,b)$  c весом   $\rho(x)$, т.е.
 \begin{equation}\label{Haar-Tcheb-1.1}
\int\limits_a^b\varphi_k(x)\varphi_l(x)\rho(x)dx=\delta_{kl},
\end{equation}
где $\delta_{kl}$ -- символ Кронекера. Через $L^p_\rho(a,b)$ обозначим пространство  функций $f(x)$, измеримых  на  $(a,b)$, для которых
 \begin{equation}\label{Haar-Tcheb-1.2}
\int\limits_a^b|f(x)|^p\rho(x)dx<\infty.
\end{equation}
Если $\rho(x)\equiv1$, то будем писать $L^p_\rho(a,b)=L^p(a,b)$ и $L(a,b)=L^1(a,b)$. Нетрудно показать, что если весовая функция $\rho(x)$ удовлетворяет некоторым (естественным) условиям, то пространство $L^p_\rho(a,b)$ представляет собой банахово пространство с нормой $\|f\|_{p,\rho}=(\int_a^b|f(x)|^p\rho(x)dx)^\frac1p$. Для этого достаточно, например, считать, что $1/\rho(x)\in L(a,b)$. В этом случае из  неравенства Гельдера вытекает также, что $L^2_\rho(a,b)\subset L(a,b)$.

Из \eqref{Haar-Tcheb-1.1} следует, что $\varphi_k(x)\in L^2_\rho(a,b)$ $(k=0,1,\ldots)$. Мы добавим к этому условию еще одно, считая, что $\varphi_k(x)\in L(a,b)$ $(k=0,1,\ldots)$. Тогда мы можем определить следующие порожденные системой $\{\varphi_k(x)\}$ функции
 \begin{equation}\label{Haar-Tcheb-1.3}
\varphi_{r,r+k}(x) =\frac{1}{(r-1)!}\int\limits_a^x(x-t)^{r-1}\varphi_{k}(t)dt, \quad k=0,1,\ldots.
\end{equation}
 Кроме того, определим конечный набор функций
  \begin{equation}\label{Haar-Tcheb-1.4}
\varphi_{r,k}(x) =\frac{(x-a)^k}{k!}, \quad k=0,1,\ldots, r-1.
\end{equation}

 Из \eqref{Haar-Tcheb-1.3} и \eqref{Haar-Tcheb-1.4} следует, что
 \begin{equation}\label{Haar-Tcheb-1.5}
(\varphi_{r,k}(x))^{(\nu)} =\begin{cases}\varphi_{r-\nu,k-\nu}(x),&\text{если $0\le\nu\le r-1$, $r\le k$,}\\
\varphi_{k-r}(x)\quad\text{для п.в. $x\in (a,b)$},&\text{если  $\nu=r\le k$,}\\
\varphi_{r-\nu,k-\nu}(x),&\text{если $\nu\le k< r$,}\\
0,&\text{если $k< \nu\le r$}.
  \end{cases}
\end{equation}
При изучении свойств системы функций $\{\varphi_{r,k}(x)\}$ нам понадобится весовое пространство Соболева $W^r_{L^p_\rho(a,b)}$, состоящее из функций $f(x)$, непрерывно дифференцируемых на $[a,b]$ $r-1$ раз, причем $f^{(r-1)}(x)$ абсолютно непрерывна на $[a,b]$  и $f^{(r)}(x)\in L^p_\rho(a,b)$.
Скалярное произведение в пространстве $W^r_{L^2_\rho(a,b)}$ определим с помощью равенства
\begin{equation}\label{Haar-Tcheb-1.6}
<f,g>=\sum_{\nu=0}^{r-1}f^{(\nu)}(a)g^{(\nu)}(a)+\int_{a}^{b} f^{(r)}(t)g^{(r)}(t)\rho(t) dt.
\end{equation}
Для того, чтобы величина $<f,g>$, определяемая равенством \eqref{Haar-Tcheb-1.6}, удовлетворяла всем четырем аксиомам скалярного произведения достаточно, например, считать, что $1/\rho(x)\in L(a,b)$.  Тогда для $f\in W^r_{L^2_\rho(a,b)}$ мы можем определить норму $\|f\|_{W^r_{L^2_\rho(a,b)}}=\sqrt{<f,g>}$, которое превращает $W^r_{L^2_\rho(a,b)}$ в банахово пространство и, стало быть, $W^r_{L^2_\rho(a,b)}$ -- гильбертово пространство со скалярным произведением \eqref{Haar-Tcheb-1.6}.

 Пользуясь определением функций  $\varphi_{r,k}(x)$ (см. \eqref{Haar-Tcheb-1.3} и \eqref{Haar-Tcheb-1.4}) и равенством  \eqref{Haar-Tcheb-1.5} нетрудно увидеть(см. теорему 1),  что система $\{\varphi_{r,k}(x)\}_{k=0}^\infty$ является ортонормированной в пространстве $W^r_{L^2_\rho(a,b)}$.  Мы будем называть систему $\{\varphi_{r,k}(x)\}_{k=0}^\infty$ \textit{ортонормированной по Соболеву } относительно скалярного произведения \eqref{Haar-Tcheb-1.6} и  \textit{порожденной} ортонормированной системой $\{\varphi_{k}(x)\}_{k=0}^\infty$.

В цитированных выше   работах  \cite{Haar-Tcheb-Shar11} -- \cite{Haar-Tcheb-Shar18}, а также в \cite{Haar-Tcheb-Shar19}   были рассмотрены некоторые частные случаи  систем функций вида $\{\varphi_{r,k}(x)\}_{k=0}^\infty$, порожденных классическими ортонормированными системами Якоби, Лежандра, Чебышева, Лагерра и Хаара. С другой стороны, в последние годы интенсивное развитие получила (см.\cite{Haar-Tcheb-IserKoch}--\cite{Haar-Tcheb-MarcelXu} и цитированную там литературу) теория полиномов, ортогональных относительно различных скалярных произведений соболевского типа (полиномы, ортогональные по Соболеву). Скалярные произведения соболевского типа характеризуются тем, что они включают в себя слагаемые, которые <<контролируют>> поведение соответствующих ортогональных полиномов  в нескольких заданных точках числовой оси. Например, в некоторых случаях оказывается так, что полиномы, ортогональные по Соболеву на интервале $(a,b)$, могут иметь нули, совпадающие с одним или с обоими концами этого интервала. Это обстоятельство имеет важное значение для некоторых приложений, в которых требуется, чтобы значения  частичных сумм ряда Фурье функции $f(x)$ по рассматриваемой системе ортогональных полиномов совпали в концах интервала $(a,b)$ со значениями $f(a)$ и $f(b)$.  Заметим, что обычные ортогональные с положительным на  $(a,b)$ весом полиномы этим важным свойством не обладают. Скалярное произведение \eqref{Haar-Tcheb-1.6}, рассматриваемое в настоящей работе, имеет одну особую точку, а именно, точку $a$, в окрестности которой <<контролируется>> поведение функций $\varphi_{r,k}(x)$, ортогональных по Соболеву и порожденных исходной ортонормированной системой $\{\varphi_{k}(x)\}_{k=0}^\infty$ посредством равенства \eqref{Haar-Tcheb-1.3}.



В дальнейшем будет показано,  что ряд Фурье функции $f(x)\in W^r_{L^2_\rho(a,b)}$ по системе  $\{\varphi_{r,k}(x)\}_{k=0}^\infty$ имеет смешанный характер, а, более точно, имеет следующий вид
  \begin{equation}\label{Haar-Tcheb-1.7}
f(x)\sim \sum_{k=0}^{r-1} f^{(k)}(a)\frac{(x-a)^k}{k!}+ \sum_{k=r}^\infty f_{r,k}\varphi_{r,k}(x),
\end{equation}
где
  \begin{equation}\label{Haar-Tcheb-1.8}
 f_{r,k}=\int\limits_a^b f^{(r)}(t) \varphi^{(r)}_{r,k}(t)\rho(t)dt=\int\limits_a^b f^{(r)}(t) \varphi_{k-r}(t)\rho(t)dt,
\end{equation}
поэтому ряд Фурье вида \eqref{Haar-Tcheb-1.7} будем (следуя \cite{Haar-Tcheb-Shar11} -- \cite{Haar-Tcheb-Shar18})  называть \textit{смешанным рядом} по  системе $\{\varphi_{k}(x)\}_{k=0}^\infty$, считая это название условным и сокращенным обозначением полного названия: <<\textit{ряд Фурье по системе  $\{\varphi_{r,k}(x)\}_{k=0}^\infty$, ортонормированной по Соболеву, порожденной ортонормированной системой $\{\varphi_{k}(x)\}_{k=0}^\infty$}>>.

Отметим некоторые важные свойства смешанного ряда \eqref{Haar-Tcheb-1.7}, непосредственно вытекающие из \eqref{Haar-Tcheb-1.5}. Первое из них связано с дифференциальным свойством смешанного ряда, а именно, если $r>1$, то в результате почленного дифференцирования смешанного ряда \eqref{Haar-Tcheb-1.7} мы получим смешанный ряд для производной $f'(x)$, соответствующий случаю, когда вместо $r$ фигурирует $r-1$, другими словами
\begin{equation}\label{Haar-Tcheb-1.9}
f'(x)\sim  \sum_{k=1}^\infty f'_{r-1,k-1}\varphi_{r-1,k-1}(x)=\sum_{k=1}^\infty (f_{r,k}\varphi_{r,k}(x))'.
\end{equation}
Второе свойство связано c почленным интегрированием с переменным верхним пределом и имеет вид
\begin{equation}\label{Haar-Tcheb-1.10}
\int\limits_a^xf'(t)dt\sim \sum_{k=1}^\infty f'_{r-1,k-1}\int\limits_a^x\varphi_{r-1,k-1}(t)dt=\sum_{k=1}^\infty f_{r,k}\varphi_{r,k}(x).
\end{equation}
Важное значение имеет свойство  смешанного ряда \eqref{Haar-Tcheb-1.7}, которое заключается в том, что его частичная сумма вида
\begin{equation}\label{Haar-Tcheb-1.11}
Y_{r,N}(f,x)=\sum_{k=0}^{r-1} f^{(k)}(a)\frac{(x-a)^k}{k!}+ \sum_{k=r}^{N} f_{r,k}\varphi_{r,k}(x)
\end{equation}
 при   $N\ge r$, $r$-кратно совпадает с исходной функцией $f(x)$ в точке $x=a$, т.е.
\begin{equation}\label{Haar-Tcheb-1.12}
(Y_{r,N}(f,x))^{(\nu)}_{x=a}=f^{(\nu)}(a)\quad (0\le\nu\le r-1).
\end{equation}
Кроме того, из \eqref{Haar-Tcheb-1.5} и \eqref{Haar-Tcheb-1.11} следует, что $(0\le\nu\le r-1)$
\begin{equation}\label{Haar-Tcheb-1.13}
 Y_{r,N}^{(\nu)}(f,x)=\sum_{n=0}^{r-1-\nu} f^{(n+\nu)}(a)\frac{(x-a)^n}{n!}+ \sum_{n=r-\nu}^{N-\nu} f_{r-\nu,n}^{(\nu)}\varphi_{r-\nu,n}(x)=Y_{r-\nu,N-\nu}(f^{(\nu)},x),
 \end{equation}
отсюда, в свою очередь, выводим $(0\le\nu\le r-2)$
 $$
f^{(\nu)}(x)-Y_{r,N}^{(\nu)}(f,x)= \frac{1}{(r-\nu-2)!}\int_a^x (x-t)^{r-\nu-2}(f^{(r-1)}(t)-Y_{r,N}^{(r-1)}(f,t))dt=
$$
  \begin{equation}\label{Haar-Tcheb-1.14}
\frac{1}{(r-\nu-2)!}\int_a^x (x-t)^{r-\nu-2}(f^{(r-1)}(t)-Y_{1,N-r+1}(f^{(r-1)},t))dt.
 \end{equation}

\textbf{Замечание.} \textit{
Еще одно важное свойство систем функций  $\{\varphi_{r,k}(x)\}_{k=0}^\infty$ возникает в том случае, когда исходная система $\{\varphi_{k}(x)\}_{k=0}^\infty$  является ортонормированной с весом $\rho(x)$ системой алгебраических полиномов. Как хорошо известно \cite{Haar-Tcheb-Sege}, в этом случае между тремя полиномами $\varphi_{k}(x)$, $\varphi_{k-1}(x)$ и $\varphi_{k-2}(x)$ существует рекуррентная связь вида $\varphi_{k}(x)=(a_kx+b_k)\varphi_{k-1}(x)+c_k\varphi_{k-2}(x)$. С учетом этого равенства из \eqref{Haar-Tcheb-1.3} нетрудно получить рекуррентное соотношение следующего вида
 $$ra_k\varphi_{r+1,r+k}(x)=(a_kx+b_k)\varphi_{r,r+k-1}(x)-\varphi_{r,r+k}(x)
 +c_k\varphi_{r,r+k-2}(x).$$
 Рекуррентные соотношения других типов можно получить в различных частных случаях. Например, из  цитируемого ниже результата \eqref{Haar-Tcheb-1.21}, полученного в \cite{Haar-Tcheb-Shar11},  можно вывести рекуррентное для полиномов $p_{r,r+k}^{0,0}(x)$, ортогональных по Соболеву и порожденных полиномами Лежандра, исходя из рекуррентных формул для полиномов Якоби $P_n^{r,r}(x)$.
На подробном рассмотрении этого вопроса мы здесь не будем останавливаться, поскольку он является объектом исследования другой нашей работы.}

В \cite{Haar-Tcheb-Shar11} -- \cite{Haar-Tcheb-Shar18}  было показано, что частичные суммы смешанных рядов по классическим ортогональным полиномам, в отличие от сумм Фурье по этим же полиномам, успешно могут быть использованы в задачах, в которых требуется одновременно приближать дифференцируемую функцию и ее несколько производных. В качестве примера рассмотрим задачу Коши  для линейного дифференциального уравнения (ОДУ)
\begin{equation}\label{Haar-Tcheb-1.15}
 a_r(x)y^{(r)}(x)+a_{r-1}(x)y^{(r-1)}(x)+\cdots+a_0(x)y(x)=h(x)
 \end{equation}
с начальными условиями $y^{(k)}(-1)=y_k$, $k=0,1,\ldots,r-1$.  Наряду с различными сеточными методами для решения этой задачи часто применяют так называемые спектральные методы \cite{Haar-Tcheb-Tref1} -- \cite{Haar-Tcheb-MMG2016}. Напомним, что суть спектрального метода решения задачи Коши  для ОДУ  заключается в том, что в первую очередь искомое решение $y(x)$ представляется в виде ряда Фурье
\begin{equation}\label{Haar-Tcheb-1.16}
 y(x)=\sum_{k=0}^\infty \hat y_k\psi_k(x)
 \end{equation}
по подходящей ортонормированной системе $\{\psi_k(x)\}_{k=0}^\infty$ (чаще всего в качестве $\{\psi_k(x)\}_{k=0}^\infty$ используют одну из классических ортонормированных систем или систему  всплесков (вэйвлетов)). На втором этапе осуществляется подстановка вместо $y(x)$ ряда \eqref{Haar-Tcheb-1.16} в уравнение \eqref{Haar-Tcheb-1.15}. Это приводит к системе уравнений относительно неизвестных коэффициентов $\hat y_k$ ($k=0,1,\ldots$). На третьем этапе требуется решить эту систему с учетом начальных условий  $y^{(k)}(-1)=y_k$, $k=0,1,\ldots,r-1$, исходной задачи Коши.
Одна из основных трудностей, которые возникают на этом этапе, состоит в том, чтобы
выбрать такой ортонормированный базис $\{\psi_k(x)\}_{k=0}^\infty$, для которого искомое решение $y(x)$ уравнения \eqref{Haar-Tcheb-1.15}, представленное в виде ряда  \eqref{Haar-Tcheb-1.16}, удовлетворяло бы начальным условиям $y^{(k)}(-1)=y_k$, $k=0,1,\ldots,r-1$. Более того, поскольку в результате решения системы уравнений относительно неизвестных коэффициентов $\hat y_k$  будет найдено только конечное их число с $k=0,1,\ldots, n$, то весьма важно, чтобы частичная сумма ряда \eqref{Haar-Tcheb-1.16}  вида $y_n(x)=\sum_{k=0}^n\hat y_k\psi_k(x)$,
 будучи приближенным решением рассматриваемой задачи Коши, также удовлетворяла начальным условиям $y_n^{(k)}(-1)=y_k$, $k=0,1,\ldots,r-1$. Покажем, что  базис $\{\psi_k(x)=p_{r,k}^{\alpha,\beta}(x)\}_{k=0}^\infty \subset W^r_{L^2_\rho(-1,1)} $, состоящий из полиномов $p_{r,k}^{\alpha,\beta}(x)$, ортонормированных по Соболеву относительно скалярного произведения
\begin{equation}\label{Haar-Tcheb-1.17}
<f,g>=\sum_{\nu=0}^{r-1}f^{(\nu)}(-1)g^{(\nu)}(-1)+\int_{-1}^{1}f^{(r)}(t)g^{(r)}(t)\rho(x)dt
\end{equation}
 и порожденных ортонормированными с весом $\rho(x)=(1-x)^\alpha(1+x)^\beta$ ($-1<\alpha,\beta<1$) полиномами Якоби $p_{k}^{\alpha,\beta}(x)$  посредством равенств
   \begin{equation*}
p_{r,k}^{\alpha,\beta}(x) =\frac{(x+1)^k}{k!}, \quad k=0,1,\ldots, r-1,
\end{equation*}
  \begin{equation}\label{Haar-Tcheb-1.18}
p_{r,r+n}^{\alpha,\beta}(x) =\frac{1}{(r-1)!}\int\limits_{-1}^x(x-t)^{r-1}p_{n}^{\alpha,\beta}(t)dt, \quad n=0,1,\ldots,
\end{equation}
   обладает указанными свойствами. С этой целью заметим, что ряд Фурье \eqref{Haar-Tcheb-1.16} в случае, когда $\{\psi_k(x)=p_{r,k}^{\alpha,\beta}(x)\}_{k=0}^\infty$, приобретает в силу \eqref{Haar-Tcheb-1.7} следующий вид
   \begin{equation}\label{Haar-Tcheb-1.19}
y(x)= \sum_{k=0}^{r-1} y^{(k)}(-1)\frac{(x+1)^k}{k!}+ \sum_{k=r}^\infty \hat y_{r,k}p_{r,k}^{\alpha,\beta}(x),
\end{equation}
где
  \begin{equation*}
 \hat y_{r,k}=\int_{-1}^1 y^{(r)}(t)p_{k-r}^{\alpha,\beta}(t)\rho(t)dt.
\end{equation*}
С другой стороны, из свойства \eqref{Haar-Tcheb-1.12}, присущего частичным суммам смешанных рядов, и того, что в силу определения \eqref{Haar-Tcheb-1.18} при $k\ge r$ справедливы равенства $(p_{r,k}^{\alpha,\beta}(x))^{(\nu)}|_{x=-1}=0$ для всех $0\le\nu\le r-1$, следует, что  функция $y(x)$, представленная в виде ряда \eqref{Haar-Tcheb-1.19}, как и частичная  сумма этого ряда вида
 \begin{equation}\label{Haar-Tcheb-1.20}
\mathcal{Y}_N^{\alpha,\beta}(x)=\mathcal{Y}_N^{\alpha,\beta}(y,x)= \sum_{k=0}^{r-1} y^{(k)}(-1)\frac{(x+1)^k}{k!}+ \sum_{k=r}^N \hat y_{r,k}p_{r,k}^{\alpha,\beta}(x),
\end{equation}
удовлетворяют начальным условиям задачи Коши для уравнения \eqref{Haar-Tcheb-1.15}.

Таким образом, полиномы, ортогональные по Соболеву относительно скалярного произведения \eqref{Haar-Tcheb-1.17}, тесно связаны с задачей Коши для уравнения \eqref{Haar-Tcheb-1.15}.  Разумеется, не является исключением и один из основных объектов исследования настоящей работы -- полиномы $T_{r,n}(x)$, порожденные полиномами Чебышева первого рода $T_n(x)=\cos(n\arccos x)$,   представляющие собой  частный случай полиномов $p_{r,k}^{\alpha,\beta}(x)$, соответствующий выбору  $\alpha=\beta=-1/2$.  Отметим еще, что полиномы $p_{r,k}^{0,0}(x)$, порожденные полиномами Лежандра, могут служить удобным и эффективным методом приближенного решения не только задачи Коши, но  также   двухточечной краевой задачи для уравнений типа \eqref{Haar-Tcheb-1.15}. В связи с этим возникает задача об аппроксимативных свойствах частичных сумм рядов Фурье по полиномам, ортогональным по Соболеву. В наших работах  \cite{Haar-Tcheb-Shar11} -- \cite{Haar-Tcheb-Shar18},
используя несколько иные  обозначения,  были исследованы апроксимативные свойства частичных сумм вида \eqref{Haar-Tcheb-1.20}. Наиболее детально были исследованы аппроксимативные свойства частичных сумм смешанных рядов по полиномам Лежандра (случай $\alpha=\beta=0$). Дело в том, что, как было обнаружено еще в работе \cite{Haar-Tcheb-Shar11} и многократно использовалось в \cite{Haar-Tcheb-Shar11} -- \cite{Haar-Tcheb-Shar18}, полиномы $p_{r,k}^{\alpha,\beta}(x)$, определенные равенством \eqref{Haar-Tcheb-1.18}, в случае $\alpha=\beta=0$ допускают следующее представление
\begin{equation}\label{Haar-Tcheb-1.21}
p_{r,r+k}^{0,0}(x) =
\frac{(-1)^r}{2^rk^{[r]}\sqrt{ h_k^{0,0}}}(1-x^2)^rP_{k-r}^{r,r}(x) \quad (k=r,r+1,\ldots),
\end{equation}
где $P_{n}^{\alpha,\beta}(x)$ -- классический полином Якоби, $h_n^{0,0}$ -- квадрат нормы полинома Лежандра $P_{n}^{0,0}(x)$ (см.\S5), которое играет ключевую роль при исследовании аппроксимативных свойств частичных сумм $\mathcal{Y}_N^{0,0}(y,x)$, определенных равенством \eqref{Haar-Tcheb-1.20}. В частности, в работе \cite{Haar-Tcheb-Shar15}, существенно используя представление \eqref{Haar-Tcheb-1.21},  была доказана следующая неулучшаемая по порядку (при $N\to\infty$) оценка
\begin{equation}\label{Haar-Tcheb-1.22}
\sup_{f\in W^r}\max_{-1\le x\le 1}{\left|f^{(\nu)}(x)-\left(\mathcal{Y}_N^{0,0}(f,x)\right)^{(\nu)}\right|\over(1-x^2)^{(r-\nu)-1/4}}
\le c(r)\frac{\ln N}{N^{r-\nu}},\,\, 0\le \nu\le r,
\end{equation}
где  $W^r$ -- класс функций, непрерывно дифференцируемых $r$-раз, для которых $\max_{-1\le x\le 1}|f^{(r)}(x)|\le1$.

Что касается случая дробных $\alpha$ и $\beta$, то представления типа \eqref{Haar-Tcheb-1.21} для
$p_{r,k}^{\alpha,\beta}(x)$ получены не были, поэтому вопрос о доказательстве оценок, аналогичных \eqref{Haar-Tcheb-1.22}, для $\mathcal{Y}_N^{\alpha,\beta}(f,x)$ при
$-1<\alpha,\beta<1$ остается открытым.



В настоящей научно-исследовательской работе вводятся и исследуются системы функций $\{\varphi_{r,k}(x)\}_{k=0}^\infty$, ортонормированных по Соболеву относительно скалярного произведения \eqref{Haar-Tcheb-1.6} и порожденных системой $\{\varphi_{k}(x)\}_{k=0}^\infty$. Рассмотрена задача о представлении функций $\varphi_{r,k}(x)$, определенных равенством \eqref{Haar-Tcheb-1.3}, в виде, более удобном для исследования их асимптотических свойств при $k\to\infty$. Эта задача решена для функций, ортогональных по Соболеву, порожденных различными классическими ортогональными системами. Отдельное внимание уделено исследованию условий равномерной сходимости смешанных рядов по рассматриваемым классическим системам.

\section{ О сходимости смешанных рядов  по общим ортонормированным системам }

Рассмотрим сначала задачу о полноте в $W^r_{L^2_\rho(a,b)}$ системы $\{\varphi_{r,k}(x)\}_{k=0}^\infty$, состоящей из функций, определенных равенствами   \eqref{Haar-Tcheb-1.3} и \eqref{Haar-Tcheb-1.4}.
\begin{theorem}\label{completeness}
Предположим, что    функции $\varphi_k(x)$ $(k=0,1,\ldots)$ образуют полную в $L^2_\rho(a,b)$ ортонормированную   c весом   $\rho(x)$ систему на отрезке $[a,b]$. Тогда система $\{\varphi_{r,k}(x)\}_{k=0}^\infty$, порожденная системой $\{\varphi_{k}(x)\}_{k=0}^\infty$ посредством равенств \eqref{Haar-Tcheb-1.3} и \eqref{Haar-Tcheb-1.4}, полна  в $W^r_{L^2_\rho(a,b)}$ и ортонормирована относительно скалярного произведения \eqref{Haar-Tcheb-1.6}.
\end{theorem}


\begin{theorem}\label{uni-conv}
Предположим, что  $ \frac{1}{\rho(x)}\in L(a,b) $, а  функции $\varphi_k(x)$ $(k=0,1,\ldots)$  образуют полную в $L^2_\rho(a,b)$ ортонормированную   c весом   $\rho(x)$ систему на отрезке $[a,b]$, $\{\varphi_{r,k}(x)\}_{k=0}^\infty$ -- система, ортонормированная в $W^r_{L^2_\rho(a,b)}$ относительно скалярного произведения \eqref{Haar-Tcheb-1.6},  порожденная системой $\{\varphi_{k}(x)\}_{k=0}^\infty$ посредством равенств \eqref{Haar-Tcheb-1.3} и \eqref{Haar-Tcheb-1.4}.
Тогда, если $f(x)\in W^r_{L^2_\rho(a,b)}$, то ряд Фурье (смешанный ряд) \eqref{Haar-Tcheb-1.7} сходится к функции $f(x)$ равномерно относительно $x\in[a,b]$.
\end{theorem}

Ниже мы остановимся на  важных частных случаях систем функций, ортогональных по Соболеву относительного скалярного произведения вида  \eqref{Haar-Tcheb-1.6}, порожденных  классическими ортогональными системами такими, как система  Хаара и система  полиномов Чебышева первого рода и рассмотрим их особенности.

\chapter{Система полиномов, ортогональных по Соболеву и порожденных функциями Хаара}
\section{Некоторые сведения о системе  Хаара}
При исследовании системы функций $\mathcal{X}_{r,n}(x)$ $(n=0,1,\ldots)$, ортогональных по Соболеву, порожденных функциями  Хаара $\varphi_k(x)=\chi_{k+1}(x)$ $(k=0,1,\ldots)$ нам понадобятся следующие обозначения \cite{Haar-Tcheb-KashSaak}. Пусть $n=2^k+i$, $i=1,2,\ldots, 2^k$,  $k=0,1,\ldots$,
$\Delta_n=\Delta_k^i=(\frac{i-1}{2^k},\frac{i}{2^k})$, $\bar\Delta_n=[\frac{i-1}{2^k},\frac{i}{2^k}]$, ($n\ge2$), $\Delta_1=[0,1]$,
$$
 \Delta_n^+=(\Delta_k^i)^+=(\frac{i-1}{2^k},\frac{2i-1}{2^{k+1}})=\Delta_{k+1}^{2i-1}, \quad \Delta_n^-=(\Delta_k^i)^-=(\frac{2i-1}{2^{k+1}},\frac{i}{2^{k}})=\Delta_{k+1}^{2i}.
$$
Система Хаара - это система функций, $\chi=\{\chi_n\}_{n=1}(x)^\infty$, $x\in[0,1]$, в которой $\chi_1(x)=1$, a функция $\chi_n(x)$
с $2^k<n\le 2^{k+1}$, $k=0,1,\ldots$ определяется так
\begin{equation}\label{Haar-Tcheb-3.1}
\chi_n(x)=\begin{cases} 0,&\text{если $x\notin \bar\Delta_n$,}\\
2^{k/2},& \text{если $x\in \Delta_n^+$,}\\
-2^{k/2},& \text{если $x\in \Delta_n^-$.}
\end{cases}
\end{equation}
Значения в точках разрыва функции $\chi_n(x)$ выбирается так, чтобы выполнялись равенства:
$$
\chi_n(x)=\lim_{\delta\to0}\frac12[\chi_n(x+\delta)+\chi_n(x-\delta)],\quad x\in (0,1),
$$
$$
\chi_n(0)=\lim_{\delta\to0} \chi_n(\delta),\quad \chi_n(1)=\lim_{\delta\to0} \chi_n(1-\delta).
$$
Непосредственно из \eqref{Haar-Tcheb-3.1} выводим следующее свойство ортогональности функций Хаара:
\begin{equation}\label{Haar-Tcheb-eq3.2}
\int_0^1\chi_n(x)\chi_m(x)dx=\delta_{nm}.
\end{equation}
Если функция $f=f(x)$ интегрируема на $[0,1]$, то мы можем определить коэффициенты Фурье-Хаара
\begin{equation}\label{Haar-Tcheb-eq3.3}
f_{0,k}=\int_0^1f(t)\chi_k(t)dt
\end{equation}
 и ряд Фурье- Хаара
\begin{equation}\label{Haar-Tcheb-eq3.4}
f(x)=\sum_{k=1}^\infty f_{0,k}\chi_k(x).
\end{equation}
Через $Q_n(f,x)$ обозначим частичную сумму  Фурье-Хаара вида
\begin{equation}\label{Haar-Tcheb-eq3.5}
Q_n(f)=Q_n(f,x)=\sum_{k=1}^n f_{0,k}\chi_k(x).
\end{equation}
Хорошо известно \cite{Haar-Tcheb-KashSaak}, что если $p\ge1$ и  $f\in L^p(0,1)$, то
\begin{equation}\label{Haar-Tcheb-eq3.6}
\int\limits_0^1|f(x)-Q_n(f,x)|^pdx\to 0 \quad(n\to\infty).
\end{equation}
Далее рассмотрим весовую функцию $\rho=\rho(x)=(x(1-x))^{-\frac12}$, заданную на $(0,1)$.  Имеет место
\begin{lemma}\label{haar-series-conv}
Пусть $f\in L^1_\rho(0,1)$. Тогда $Q_n(f)$ сходится к $f$ в метрике пространства $L^1_\rho(0,1)$.  Другими словами, если $$\int_0^1{|f(x)|\over\sqrt{x(1-x)}}dx<\infty,$$
то
\begin{equation}\label{Haar-Tcheb-eq3.7}
I_n=\int_0^1{|f(x)-Q_n(f,x)|\over\sqrt{x(1-x)}}dx\to 0 \quad(n\to\infty).
\end{equation}
\end{lemma}

\section{Ортонормированная по Соболеву система функций, порожденная функциями   Хаара}
Положим $\varphi_k(x)=\chi_{k+1}(x)$ $(k=0,1,\ldots)$, и для каждого натурального $r$ мы определим систему функций $\{\mathcal{X}_{r,n}\}_{n=r}^\infty$ следующим образом:
\begin{equation}\label{Haar-Tcheb-4.1}
\mathcal{X}_{r,k+r}(x) =\frac{1}{(r-1)!}\int\limits_0^x(x-t)^{r-1}\varphi_k(x)dt, \quad k=0,1, 2, \ldots.
\end{equation}
 Кроме того, определим конечный набор функций
  \begin{equation}\label{Haar-Tcheb-4.2}
\mathcal{X}_{r,n}(x) =\frac{x^n}{n!}, \quad n=0,1,\ldots, r-1.
\end{equation}

   Пусть $W^r_{L^2(0,1)}$ --  пространство Соболева с  $L^2(0,1)=L^2_\rho(0,1)$, а $\rho=\rho(x) \equiv 1$. Из теоремы \ref{completeness} непосредственно вытекает
 \begin{corollary}
   Система $\{\mathcal{X}_{r,n}(x)\}_{n=0}^\infty$, порожденная системой Хаара $\{\chi_{n}(x)\}_{n=1}^\infty$ посредством равенств \eqref{Haar-Tcheb-4.1} и \eqref{Haar-Tcheb-4.2}, полна  в $W^r_{L^2(0,1)}$ и ортонормирована относительно скалярного произведения
 \begin{equation}\label{Haar-Tcheb-4.3}
<f,g>=\sum_{\nu=0}^{r-1}f^{(\nu)}(0)g^{(\nu)}(0)+\int_{0}^{1} f^{(r)}(t)g^{(r)}(t) dt.
\end{equation}
 \end{corollary}

Ряд Фурье по системе $\{\mathcal{X}_{r,n}(x)\}_{n=0}^\infty$ имеет следующий вид (см. \eqref{Haar-Tcheb-1.7})
  \begin{equation}\label{Haar-Tcheb-4.4}
f(x)\sim \sum_{n=0}^{r-1} f^{(n)}(0)\frac{x^n}{n!}+ \sum_{n=r}^\infty f_{r,n}\mathcal{X}_{r,n}(x),
\end{equation}
где
  $$
 f_{r,n}=\int_0^1 f^{(r)}(t)\varphi_{n-r}(t)dt=\int_0^1 f^{(r)}(t) \chi_{n-r+1}(t)dt.
$$
Используя обозначения, отличные от принятых в настоящей работе, в \cite{Haar-Tcheb-Shar19}  получено  следующее представление для функций $\mathcal{X}_{r,n}(x)$, определенных равенством \eqref{Haar-Tcheb-4.1}
 ($n=2^k+i$, $i=1,2,\ldots,2^k $, $k=0,1,\ldots$ )
  \begin{equation}\label{Haar-Tcheb-4.5}
 \mathcal{X}_{r,n+r-1}(x)=\frac{2^{\frac{k}{2}}}{r!}
 \begin{cases} 0,&\text{$0\le x\le\frac{i-1}{2^k}$;}\\
 (x-\frac{i-1}{2^k})^r,&\text{$\frac{i-1}{2^k}\le x\le \frac{2i-1}{2^{k+1}}$;}\\
 (x-\frac{i-1}{2^k})^r-2(x-\frac{2i-1}{2^{k+1}})^r,&\text{$\frac{2i-1}{2^{k+1}}\le x\le \frac{i}{2^{k}}$;}\\
  (x-\frac{i-1}{2^k})^r-2(x-\frac{2i-1}{2^{k+1}})^r+(x-\frac{i}{2^{k}})^r, &\text{$\frac{i}{2^{k}}\le x\le1$}
   \end{cases}
  \end{equation}
 и, как следствие, доказано, что
   \begin{equation}\label{Haar-Tcheb-4.6}
 \mathcal{X}_{r,n+r-1}(x)=\frac{2^{k/2}}{r!}\Delta^2_\frac{1}{2^{k+1}}(x-t)^r_+
 \big|_{t=\frac{i-1}{2^k}},
\end{equation}
где $\Delta^2_h g(t)$ -- конечная разность второго порядка с шагом $h$,
$$
a_+^r=\begin{cases}a^r,&\text{если $a\ge0$,}\\0, &\text{если $a<0$.}   \end{cases}
$$
Используя свойства \eqref{Haar-Tcheb-4.5} и \eqref{Haar-Tcheb-4.6}, в цитированной выше работе  \cite{Haar-Tcheb-Shar19} исследованы аппроксимативные свойства частичных сумм  ряда \eqref{Haar-Tcheb-4.4} следующего вида
\begin{equation}\label{Haar-Tcheb-4.7}
 Y_{r,N}(f,x)=\sum_{n=0}^{r-1} f^{(n)}(0)\frac{x^n}{n!}+ \sum_{n=r}^{N} f_{r,n}\mathcal{X}_{r,n}(x)
 \end{equation}
 для функций $f$ из  классов  $W^r_{L^p(0,1)}$ с $p=1$ и $p=2$ . В частности, для $f\in W^r_{L(0,1)}$ получены следующие оценки
  \begin{equation}\label{Haar-Tcheb-4.8}
 |f^{(\nu)}(x)-Y_{r,N}^{(\nu)}(f,x)|\le \frac{x^{r-1-\nu}}{(r-1-\nu)!}\omega_2(f^{(r-1)},\frac{1}{N-r+1}), \quad 0\le x\le1,
 \end{equation}
  \begin{equation}\label{Haar-Tcheb-4.9}
 |f^{(\nu)}(x)-Y_{r,N}^{(\nu)}(f,x)|\le \frac{x^{r-1-\nu}}{(r-1-\nu)!}\eta(f^{(r)},\frac{1}{N-r+1}), \quad 0\le x\le1,
 \end{equation}
 где $0\le \nu\le r-1$,
 $$
\omega_2(g,\delta)=\sup_{0\le h\le\delta,\atop h\le x\le 1-h}|g(x+h)+g(x-h)-2g(x)|,
$$
$$
\eta(g,\delta)=\sup_{0\le h\le\delta,\atop h\le x\le 1-h}\left|\int_0^h[g(x+t)-g(x+t-h)]dt\right|.
$$

В качестве следствия неравенства \eqref{Haar-Tcheb-4.9} в \cite{Haar-Tcheb-Shar19} показано, что если $f^{(r)}(x)$ удовлетворяет условию Липшица
$
|f^{(r)}(x)-f^{(r)}(y)|\le M|x-y|^\alpha, x,y\in[0,1], 0< \alpha\le1,
$
то имеет место неравенство ($0\le\nu\le r-1$)
\begin{equation}\label{Haar-Tcheb-4.10}
 |f^{(\nu)}(x)-Y_{r,N}^{(\nu)}(f,x)|\le \frac{Mx^{r-1-\nu}}{(r-1-\nu)!}(N-r+1)^{-1-\alpha}, \quad 0\le x\le1. \end{equation}
Для $\nu=r-1$ неравенство \eqref{Haar-Tcheb-4.8}  приобретает следующий вид
\begin{equation}\label{Haar-Tcheb-4.11}
 |f^{(r-1)}(x)-Y_{r,N}^{(r-1)}(f,x)|\le \omega_2(f^{(r-1)},\frac{1}{N-r+1}).
 \end{equation}

Заметим, что из \eqref{Haar-Tcheb-1.12} вытекают следующие равенства
$(Y_{r,N}(f,x))^{(\nu)}_{x=0}=f^{(\nu)}(0)$ $ (0\le\nu\le r-1)$, однако оценка \eqref{Haar-Tcheb-4.11} не учитывает этого обстоятельства в том смысле, что величина $\omega_2(f^{(r-1)},\frac{1}{N-r+1})$ не зависит от переменной $x$ (и не стремится к нулю при $x\to0$). В связи с этим возникает задача о получении такой оценки величины $|f^{(\nu)}(x)-Y_{r,N}^{(\nu)}(f,x)|$, которая учитывает тот факт, что $$\lim_{x\to0}|f^{(\nu)}(x)-Y_{r,N}^{(\nu)}(f,x)|=0.$$
Рассмотрим предварительно случай $\nu=r-1$.
Пусть $f\in W^r_{L_\rho(0,1)}$, $g(x)=f^{(r-1)}(x)$. Тогда, очевидно, $g\in W^1_{L_\rho(0,1)}$, поэтому в силу равенства
 \begin{equation}\label{Haar-Tcheb-4.12}
 (Y_{1,N-r+1}(f,x))'=Q_{N-r+1}(f',x),
 \end{equation}
которое в силу \eqref{Haar-Tcheb-1.5}, \eqref{Haar-Tcheb-4.7} и \eqref{Haar-Tcheb-eq3.5} справедливо для $f\in W^1_{L(0,1)}$  при почти всех $x\in (0,1)$,    мы можем записать
\begin{equation}\label{Haar-Tcheb-eq4.13}
g(x)-Y_{1,N-r+1}(g,x)=\int_0^x [g'(t)- Q_{N-r+1}(g',t)]dt.
\end{equation}
Пусть $n=N-r+1=2^k+i$, $1\le i\le 2^k$. Тогда мы можем воспользоваться следующим равенством \cite{Haar-Tcheb-KashSaak}
\begin{equation}\label{Haar-Tcheb-eq4.14}
Q_n(f,x)=\begin{cases} Q_{2^{k+1}}(f,x), &\text{$x\in[0,\frac{i}{2^k})$,}\\
 Q_{2^{k}}(f,x), &\text{$x\in(\frac{i}{2^k},1]$.}
\end{cases}
\end{equation}
Рассмотрим три случая: 1) $x\in[0,\frac{i}{2^k})$;
2) $x\in(\frac{i}{2^k},1)$; 3) $x=\frac{i}{2^k}$ или $x=1$. Если $x\in[0,\frac{i}{2^k})$, то через $s_x$ обозначим такое натуральное число, для которого $x\in [\frac{s_x}{2^{k+1}},\frac{s_x+1}{2^{k+1}}) $.
Тогда из \eqref{Haar-Tcheb-eq4.13} имеем
$$
g(x)-Y_{1,N-r+1}(g,x)=\sum_{s=1}^{s_x} \int_{\Delta_{k+1}^s}\left[g'(t)-2^{k+1}\int_{\Delta_{k+1}^s}g'(\tau)d\tau \right]dt+
$$
$$
\int_{\frac{s_x}{2^{k+1}}}^x \left[g'(t)-2^{k+1}\int_{\Delta_{k+1}^{s_x+1}}g'(\tau)d\tau \right]dt=
$$
$$
\sum_{s=1}^{s_x}\left[\int_{\Delta_{k+1}^s}g'(t)-2^{k+1}\int_{\Delta_{k+1}^s}
\int_{\Delta_{k+1}^s}g'(\tau)d\tau \right]dt+
$$
$$
g(x)-g(\frac{s_x}{2^{k+1}})-2^{k+1}(x-\frac{s_x}{2^{k+1}})\left[g(\frac{s_x+1}{2^{k+1}})-g(\frac{s_x}{2^{k+1}}) \right].
$$
Отсюда для $x\in[0,\frac{i}{2^k})$ находим
$$
g(x)-Y_{1,N-r+1}(g,x)=
$$
\begin{equation}\label{Haar-Tcheb-eq4.15}
g(x)-g(\frac{s_x}{2^{k+1}})-2^{k+1}
(x-\frac{s_x}{2^{k+1}})\left[g(\frac{s_x+1}{2^{k+1}})-g(\frac{s_x}{2^{k+1}}) \right]=\Gamma_n^1(g,x).
\end{equation}
Перейдем к случаю $x\in(\frac{i}{2^k},1)$ и обозначим через $j_x$  такое натуральное число, для которого $x\in [\frac{j_x}{2^{k+1}},\frac{j_x+1}{2^{k+1}}) $. В силу \eqref{Haar-Tcheb-eq4.13} имеем
\begin{equation}\label{Haar-Tcheb-eq4.16}
g(x)-Y_{1,N-r+1}(g,x)=U_1+U_2,
\end{equation}
где
$$
U_1=\int_0^\frac{i}{2^k} [g'(t)- Q_{n}(g',t)]dt=\int_0^\frac{2i}{2^{k+1}} [g'(t)dt- Q_{n}(g',t)]dt=
$$
\begin{equation}\label{Haar-Tcheb-eq4.17}
\sum_{s=1}^{2i}\left[\int_{\Delta_{k+1}^s}g'(t)dt- \int_{\Delta_{k+1}^s}2^{k+1}\int_{\Delta_{k+1}^s}g'(\tau)d\tau dt\right]=0,
\end{equation}
$$
U_2=\int_\frac{i}{2^k}^x [g'(t)- Q_{n}(g',t)]dt=\int_\frac{i}{2^k}^\frac{j_x}{2^k} [g'(t)dt- Q_{n}(g',t)]dt+
$$
$$
\int_\frac{j_x}{2^k}^x [g'(t)- Q_{n}(g',t)]dt=\int_\frac{j_x}{2^k}^x g'(t)dt-\int_\frac{j_x}{2^k}^x  Q_{2^k}(g',t)dt=
$$
\begin{equation}\label{Haar-Tcheb-eq4.18}
 g(x)-g(\frac{j_x}{2^k})-2^k(x-\frac{j_x}{2^k})[g(\frac{j_x+1}{2^k})-g(\frac{j_x}{2^k})].
\end{equation}

В третьем случае, когда $x=\frac{i}{2^k}$ или $x=1$, нетрудно увидеть, что
\begin{equation}\label{Haar-Tcheb-eq4.19}
 g(x)-Y_{1,N-r+1}(g,x)=0.
 \end{equation}
Из \eqref{Haar-Tcheb-eq4.16}- \eqref{Haar-Tcheb-eq4.19} находим для $x\in[\frac{i}{2^k},1]$
\begin{equation}\label{Haar-Tcheb-eq4.20}
 g(x)-Y_{1,N-r+1}(g,x)=
 g(x)-g(\frac{j_x}{2^k})-2^k(x-\frac{j_x}{2^k})[g(\frac{j_x+1}{2^k})-g(\frac{j_x}{2^k})]=
 \Gamma_n^2(g,x).
 \end{equation}

 Для $n=N-r+1=2^k+i$ введем в рассмотрение следующую функцию
 \begin{equation}\label{Haar-Tcheb-eq4.21}
 \Gamma_n(g,x)=\begin{cases}
 \Gamma_n^1(g,x), &\text{$x\in[0,\frac{i}{2^k})$},\\
 \Gamma_n^2(g,x), &\text{$x\in[\frac{i}{2^k},1]$},
  \end{cases}
 \end{equation}
где функции $\Gamma_n^1(g,x)$ и $\Gamma_n^2(g,x)$ определены с помощью равенств \eqref{Haar-Tcheb-eq4.15} и \eqref{Haar-Tcheb-eq4.20}.
Тогда из \eqref{Haar-Tcheb-eq4.15}, \eqref{Haar-Tcheb-eq4.20} -- \eqref{Haar-Tcheb-eq4.21} нетрудно увидеть, что
 \begin{equation}\label{Haar-Tcheb-eq4.22}
 |\Gamma_n(g,x)|\le \begin{cases}\omega\left(g,x-\frac{s_x}{2^{k+1}}\right)+2^{k+1}
(x-\frac{s_x}{2^{k+1}})\omega\left(g,2^{-k-1}\right),&\text{$x\in[0,\frac{i}{2^k})$,}\\
\omega\left(g,x-\frac{j_x}{2^{k}}\right)+2^{k}
(x-\frac{j_x}{2^{k}})\omega\left(g,2^{-k}\right),&\text{$x\in[\frac{i}{2^k},1]$},
\end{cases}
 \end{equation}
где
$$
\omega(g,\delta)=\max_{x,y\in[0,1],|x-y|\le\delta}|g(x)-g(y)|
$$
-- модуль непрерывности функции $g(x)$.
Таким образом, в случае $\nu=r-1$ задача об оценке отклонения $|g(x)-Y_{1,N-r+1}(g,x)|$, учитывающей его поведение при приближении переменной $x$ к $0$, полностью решена с помощью точных равенств  \eqref{Haar-Tcheb-eq4.15} и \eqref{Haar-Tcheb-eq4.20} и неравенства \eqref{Haar-Tcheb-eq4.22}.
При исследовании этой задачи при $0\le\nu\le r-2$ нам понадобится следующая величина
 \begin{equation}\label{Haar-Tcheb-eq4.23}
 \gamma_n(g,x)=\max_{0\le t\le x}|\Gamma_n(g,t)|,
 \end{equation}
 поведение которой при $n\to\infty$  мы рассмотрим позже. Но уже сейчас заметим, что, как это непосредственно вытекает из \eqref{Haar-Tcheb-eq4.22} и  \eqref{Haar-Tcheb-eq4.23}, имеет место предельное соотношение $\lim_{x\to0}\gamma_n(g,x)=0$.

    Переходя к задаче об оценке разности $f^{(\nu)}(x)-Y_{r,N}^{(\nu)}(f,x)$, мы отметим предварительно  дальнейшие свойства    частичных сумм $Y_{r,N}(f,x)$. Из \eqref{Haar-Tcheb-1.5} и  \eqref{Haar-Tcheb-4.7} следует, что
\begin{equation}\label{Haar-Tcheb-4.24}
 Y_{r,N}^{(\nu)}(f,x)=\sum_{k=0}^{r-1-\nu} f^{(k)}(0)\frac{x^k}{k!}+ \sum_{k=r-\nu}^{N-\nu} f_{r-\nu,k}^{(\nu)}\mathcal{X}_{r-\nu,k}(x)=Y_{r-\nu,N-\nu}(f^{(\nu)},x).
 \end{equation}
 Кроме того в силу \eqref{Haar-Tcheb-1.12} $(Y_{r-\nu,N-\nu}(f^{(\nu)},x))^{(j)}_{x=0}-f^{(j)}(0)=0$ при  $0\le j\le r-\nu-2$,
 поэтому, с учетом \eqref{Haar-Tcheb-4.24},
 $$
f^{(\nu)}(x)-Y_{r,N}^{(\nu)}(f,x)= \frac{1}{(r-\nu-2)!}\int_0^x (x-t)^{r-\nu-2}(f^{(r-1)}(t)-Y_{r,N}^{(r-1)}(f,t))dt=
$$
  \begin{equation}\label{Haar-Tcheb-4.25}
\frac{1}{(r-\nu-2)!}\int_0^x (x-t)^{r-\nu-2}(f^{(r-1)}(t)-Y_{1,N-r+1}(f^{(r-1)},t))dt.
 \end{equation}
Подставим в \eqref{Haar-Tcheb-4.25} вместо $f^{(r-1)}(t)-Y_{1,N-r+1}(f^{(r-1)},t)$  функцию $\Gamma_n(g,x)$, определенную равенством \eqref{Haar-Tcheb-eq4.21}, в котором $n=N-r+1=2^k+i$.  Это дает
 \begin{equation}\label{Haar-Tcheb-4.26}
f^{(\nu)}(x)-Y_{r,N}^{(\nu)}(f,x)=\frac{1}{(r-\nu-2)!}\int_0^x (x-t)^{r-\nu-2}\Gamma_n(f^{(r-1)},t)dt.
 \end{equation}
Из \eqref{Haar-Tcheb-eq4.23} и \eqref{Haar-Tcheb-4.26} выводим следующее неравенство $(n=N-r+1=2^k+i)$
$$
|f^{(\nu)}(x)-Y_{r,N}^{(\nu)}(f,x)|\le \frac{1}{(r-\nu-2)!}\int_0^x (x-t)^{r-\nu-2}|\Gamma_n(f^{(r-1)},t)|dt
$$
\begin{equation}\label{Haar-Tcheb-4.27}
\le\frac{x^{r-\nu-1}}{(r-\nu-1)!}\gamma_n(f^{(r-1)},x),
 \end{equation}
в котором  величина $\gamma_n(f^{(r-1)},x)$,   как уже отмечалось, стремится к нулю вместе с переменной  $x$. Нам остается рассмотреть поведение этой величины при $n\to\infty$.
С этой целью заметим, что для произвольного интервала $(a,b)\subset(0,1)$ и произвольной непрерывной на $[0,1]$ функции $g(x)$ имеет место (см. \cite{Haar-Tcheb-KashSaak}, стр. 208)  неравенство
\begin{equation}\label{Haar-Tcheb-4.28}
J=\max_{x\in[a,b]}|g(x)-l(x)|\le\omega_2\left(g,\frac{b-a}{2}\right),
\end{equation}
где $l(x)=g(a)+\frac{g(b)-g(a)}{b-a}(x-a)$, $x\in[0,1]$. Положим $n=N-r+1=2^k+i$ и применим неравенство \eqref{Haar-Tcheb-4.28}
к интервалу $\Delta_k^{j_x+1}$ и разности $g(x)-l(x)$, где
$$
l(x)=g(\frac{j_x}{2^k})+2^k(x-\frac{j_x}{2^k})[g(\frac{j_x+1}{2^k})-g(\frac{j_x}{2^k})].
$$
Это дает
\begin{equation}\label{Haar-Tcheb-4.29}
|\Gamma_n^2(g,x)|=|g(x)-l(x)|\le \omega_2\left(g,\frac{1}{2^{k+1}}\right), \quad x\in[\frac{i}{2^k},1].
\end{equation}
Совершенно аналогично находим
\begin{equation}\label{Haar-Tcheb-4.30}
|\Gamma_n^1(g,x)|\le \omega_2\left(g,\frac{1}{2^{k+2}}\right), \quad x\in[0,\frac{i}{2^k}].
\end{equation}
Сопоставляя  \eqref{Haar-Tcheb-4.29} и \eqref{Haar-Tcheb-4.30} с \eqref{Haar-Tcheb-eq4.21}, мы получаем для $g=f^{(r-1)}$ оценку
\begin{equation}\label{Haar-Tcheb-4.31}
|\Gamma_n(f^{(r-1)},x)|\le \omega_2\left(f^{(r-1)},\frac{1}{n}\right), \quad x\in[0,1].
\end{equation}
Из \eqref{Haar-Tcheb-eq4.23} и \eqref{Haar-Tcheb-4.31}, в свою очередь, выводим
\begin{equation}\label{Haar-Tcheb-4.32}
|\gamma_n(f^{(r-1)},x)|\le \omega_2\left(f^{(r-1)},\frac{1}{n}\right), \quad x\in[0,1].
\end{equation}

В случае, когда  функция $f^{(r-1)}(x)$ является абсолютно непрерывной на $[0,1]$, величины $\omega_2\left(f^{(r-1)},\delta\right)$  и $\eta\left(f^{(r)},\delta\right)$ (см.\eqref{Haar-Tcheb-4.8} и \eqref{Haar-Tcheb-4.9}) совпадают, т.е. $$\omega_2\left(f^{(r-1)},\delta\right)=\eta\left(f^{(r)},\delta\right),$$
 поэтому вместо \eqref{Haar-Tcheb-4.31} и \eqref{Haar-Tcheb-4.32} можно использовать также неравенства
\begin{equation}\label{Haar-Tcheb-4.33}
|\Gamma_n(f^{(r-1)},x)|\le \eta\left(f^{(r)},\frac{1}{n}\right),\quad |\gamma_n(f^{(r)},x)|\le \eta\left(f^{(r)},\frac{1}{n}\right).
\end{equation}




Подводя итоги, мы можем сформулировать следующий результат.

\begin{theorem}\label{mixed-haar-est}
Пусть $r\ge1$ $f\in W^r_{L(0,1)}$, $n=N-r+1=2^k+i$.  Тогда имеют место следующие оценки
$$
|f^{(r-1)}(x)-Y_{r,N}^{(r-1)}(f,x)| \le |\Gamma_n(f^{(r-1)},x)|,
$$
$$
|f^{(\nu)}(x)-Y_{r,N}^{(\nu)}(f,x)| \le\frac{x^{r-\nu-1}}{(r-\nu-1)!}\gamma_n(f^{(r-1)},x),\quad  0\le\nu\le r-2,
$$
в которых для функции $\Gamma_n(f^{(r-1},x)$  справедливы неравенства \eqref{Haar-Tcheb-eq4.22} и \eqref{Haar-Tcheb-4.31}, а для функции $\gamma_n(f^{(r-1)},x)$ верно неравенство   \eqref{Haar-Tcheb-4.32}.
\end{theorem}







Из леммы \ref{haar-series-conv}  и равенства  \eqref{Haar-Tcheb-4.12} вытекает справедливость следующего  утверждения.

\begin{lemma}\label{haar-series-deriv-conv}
Пусть $\rho=\rho(x)=(x(1-x))^{-\frac12}$,   $f\in W^1_{L_\rho(0,1)}$.
Тогда
\begin{equation*}
\int_0^1{|f'(x)-(Y_{1,N}(f,x))'|\over\sqrt{x(1-x)}}dx\to 0 \quad(N\to\infty).
\end{equation*}
\end{lemma}
Из леммы \ref{haar-series-deriv-conv} и теоремы \ref{mixed-haar-est}, в свою очередь, выводим следующий результат, который нам понадобится при исследовании задачи о равномерной сходимости ряда Фурье по полиномам, ортогональным по Соболеву, порожденным полиномами Чебышева первого рода.


\begin{proposition}\label{mixed-metric-conv}%Предложение 4.1.
Пусть $\rho=\rho(x)=(x(1-x))^{-\frac12}$,   $f\in W^1_{L_\rho(0,1)}$. Тогда
$$
\max_{0\le x\le1}|f(x)-Y_{1,N}(f,x)|+
\int_0^1{|f'(x)-(Y_{1,N}(f,x))'|\over\sqrt{x(1-x)}}dx\to 0 \quad(N\to\infty).
$$
\end{proposition}

\chapter{Система полиномов, ортогональных по Соболеву и порожденных полиномами Якоби}
%\section{Введение}
При исследовании вопросов сходимости рассматриваемых рядов непосредственно возникает, как это всегда и бывает в теории ортогональных полиномов, задача об изучении поведения самих полиномов $p_{r,n}(x)$ при $n\to\infty$. С вычислительной точки зрения важно также знание таких свойств этих полиномов, как рекуррентные соотношения, которыми связаны эти полиномы между собой,  явное их представление посредством элементарных или классических специальных функций. В данном параграфе предпринята попытка получить ответы на эти вопросы. Касаясь особенностей рядов Фурье по системе полиномов $p_{r,k}(x)$ $(k=0,1,\ldots)$, прежде всего следует выделить одно важное свойство их частичных сумм вида
\begin{equation}\label{sob-leg-1.2}
\mathcal{Y}_{r,n}(f,x)=\sum_{k=0}^{n}<f,p_{r,k}>p_{r,k}(x),
\end{equation}
заключающееся в том, что если исходная функция $f(x)$ непрерывно дифференцируема $r-1$-раз
и $f^{(r-1)}(x)$ абсолютно непрерывна  на $[-1,1]$, то $\mathcal{Y}_{r,n}(f,x)$ $r$-кратно совпадает с $f(x)$ в концах $-1$ и $1$, т. е.
$(\mathcal{Y}_{r,n}(f,x))^{(\nu)}(\pm1)=f^{(\nu)}(\pm1)$ при $\nu=0,\ldots,r-1$. Это очень важное свойство, которое в сочетании с хорошими аппроксимативными свойствами сумм Фурье
$\mathcal{Y}_{r,n}(f,x)$ делает их весьма эффективным инструментом приближенного решения краевых задач для обыкновенных дифференциальных уравнений спектральными методами.   Заметим при этом, что суммы Фурье по классическим ортогональным полиномам этим важным свойством не обладают.

На протяжении всей работы существенно будет использован аппарат теории общих полиномов Якоби (не только ортогональных). Поэтому мы соберем для удобства ссылок в следующем параграфе необходимые сведения об этих полиномах.

Основной целью настоящего параграфа является получение различных выражений для полиномов $p_{r,k}(x)$, ортогональных по Соболеву относительно скалярного произведения \eqref{Haar-Tcheb-1.17} и порожденных полиномами Якоби $p^{\alpha,\beta}_{k}(x)$ посредством  равенства \eqref{Haar-Tcheb-1.18}. Эти результаты используются  при изучении асимптотических свойств полиномов $p_{r,k}(x)$ при $x\in[-1,1]$ и $n\to\infty$ и аппроксимативных свойств рядов Фурье по ним.

\section{Некоторые сведения о полиномах Якоби}
При изучении свойств полиномов, ортогональных по Соболеву, порожденных полиномами Чебышева первого рода $T_n(x)=\cos(n\arccos x)$, нам понадобится ряд свойств классических полиномов Якоби.  Для произвольных действительных $\alpha$ и $\beta$ полиномы Якоби  $P_n^{\alpha,\beta}(x)$ можно определить \cite{Haar-Tcheb-Sege}  с помощью формулы Родрига
 \begin{equation}\label{Haar-Tcheb-eq5.1}
P_n^{\alpha,\beta}(x) = {(-1)^n\over2^nn!}{1\over\rho(x)}{d^n\over
dx^n} \left\{\rho(x)\sigma^n(x)\right\},
\end{equation}
где  $\rho(x)=\rho(x;\alpha,\beta) =
(1-x)^\alpha(1+x)^\beta,\,\,\sigma(x)=1-x^2$. Если
$\alpha,\beta>-1$, то полиномы Якоби образуют ортогональную
систему с весом $\rho(x)$, т.е.
\begin{equation}\label{Haar-Tcheb-eq5.2}
\int_{-1}^1P_n^{\alpha,\beta}(x)P_m^{\alpha,\beta}(x)\rho(x)dx =
h_n^{\alpha,\beta}\delta_{nm},
\end{equation}
где
\begin{equation}\label{Haar-Tcheb-eq5.3}
h_n^{\alpha,\beta} =
{\Gamma(n+\alpha+1)\Gamma(n+\beta+1)2^{\alpha+\beta+1} \over
n!\Gamma(n+\alpha+\beta+1)(2n+\alpha+\beta+1)}.
\end{equation}
Нам понадобятся еще следующие свойства полиномов Якоби~\cite{Haar-Tcheb-Sege}, \cite{Haar-Tcheb-Gasper}:


\begin{equation}\label{Haar-Tcheb-eq5.4}
{d\over dx}P_n^{\alpha,\beta}(x) =
{1\over2}(n+\alpha+\beta+1)P_{n-1}^{\alpha+1,\beta+1}(x),
\end{equation}
\begin{equation}\label{Haar-Tcheb-eq5.5}
{d^\nu\over dx^\nu}P_n^{\alpha,\beta}(x) =
{(n+\alpha+\beta+1)_\nu\over2^\nu} P_{n-\nu}^{\alpha+\nu,\beta+\nu}(x),
\end{equation}
где $(a)_0=1$, $(a)_\nu=a(a+1)\dots(a+\nu-1)$, $a^{[0]}=1$,

 \begin{equation}\label{Haar-Tcheb-eq5.6}
 {n\choose l}P_n^{\alpha,-l}(x)= {n+\alpha\choose
l}\left({x+1\over2}\right)^lP_{n-l}^{\alpha,l}(x),
     \quad 1\le l \le n,
\end{equation}
\begin{equation}\label{Haar-Tcheb-eq5.7}
P_n^{\alpha,\beta}(t) ={n+\alpha\choose n}
\sum_{k=0}^n{(-n)_k(n+\alpha+\beta+1)_k\over k!(\alpha+1)_k}
\left({1-t\over 2}\right)^k,
\end{equation}
\begin{equation}\label{Haar-Tcheb-eq5.8}
(1-x)^\alpha(1+x)^\beta P_n^{\alpha,\beta}(x)
={(-1)^m\over2^mn^{[m]}}{d^m\over dx^m}
\left\{(1-x)^{m+\alpha}(1+x)^{m+\beta} P_{n-m}^{m+\alpha,m+\beta}(x)
\right\},
\end{equation}
где $k^{[0]}=1$, $k^{[r]}=k(k-1)\dots(k-r+1)$,
\begin{equation}\label{Haar-Tcheb-eq5.9}
P_n^{\alpha,\beta}(-1)=(-1)^n {n+\beta\choose n},\quad P_n^{\alpha,\beta}(1)= {n+\alpha\choose n},
\end{equation}

$$
{P_n^{a,a}(x)\over P_n^{a,a}(1)}=\sum_{j=0}^{[n/2]}
   { n!(\alpha+1)_{n-2j}(n+2a+1)_{n-2j}(1/2)_j(a-\alpha)_j
    \over (n-2j)!(2j)!(a+1)_{n-2j}(n-2j+2\alpha+1)_{n-2j}}
     $$
\begin{equation}\label{Haar-Tcheb-eq5.10}
    \times{1\over (n-2j+a+1)_j(n-2j+\alpha+3/2)_j}
     {P_{n-2j}^{\alpha,\alpha}(x)\over
P_{n-2j}^{\alpha,\alpha}(1)},
\end{equation}
где $[b]$ -- целая часть числа $b$,

\textit{весовая оценка} $(1\le x\le1)$
\begin{equation}\label{sob-jac-discrete-eq2.11}
\sqrt{n}\left| P_n^{\alpha,\beta}(x)\right|\le c(\alpha,\beta)
\left(\sqrt{1-x}+{1\over n}\right)^{-\alpha-{1\over2}}
\left(\sqrt{1+x}+{1\over n}\right)^{-\beta-{1\over2}},
\end{equation}
где здесь и всюду в дальнейшем
$c(\alpha),c(\alpha,\beta),c(\alpha,\beta,\dots,\gamma)$ означают
положительные числа, зависящие лишь от указанных параметров, вообще
говоря, различные в разных местах;

\textit{асимптотическая формула}
\begin{equation}\label{sob-jac-discrete-eq2.12}
P_n^{\alpha,\beta}(\cos\theta)=n^{-1/2}
\kappa(\theta)\left\{\cos(\lambda\theta+\gamma)+ {r_n(\theta)\over
n\sin\theta}\right\},
\end{equation}
где $\alpha$, $\beta$ -- произвольные действительные числа,
$$
\kappa(\theta) =\kappa^{\alpha,\beta}(\theta)=
\pi^{-1/2}\left(\sin{\theta\over2}\right)^{-\alpha-1/2}
\left(\cos{\theta\over2}\right)^{-\beta-1/2},
$$
$$
\lambda =\lambda^{\alpha,\beta}_n=
n+{\alpha+\beta+1\over2},\quad\gamma = \gamma_\alpha= -\left(
\alpha+{1\over2}\right){\pi\over2},
$$
а для остаточного члена $r_n(\theta)=r_n^{\alpha,\beta}(\theta)$
имеет место оценка
$$
|r_n(\theta)|\le c(\alpha,\beta,\delta)\quad \left(0< {\delta\over
n}\le\theta\le\pi-{\delta\over n}\right);
$$
\textit{формула Кристоффеля-Дарбу }
$$
 K_n^{\alpha,\beta}(x,y)=
\sum_{k=0}^n{P_k^{\alpha,\beta}(x)P_k^{\alpha,\beta}(y)\over
h_k^{\alpha,\beta}}=
 $$
\begin{equation}\label{sob-jac-discrete-eq2.13}
 {2^{-\alpha-\beta}\over
2n+\alpha+\beta+2} {\Gamma(n+2)\Gamma(n+\alpha+\beta+2)\over
\Gamma(n+\alpha+1)\Gamma(n+\beta+1)}
 {P_{n+1}^{\alpha,\beta}(x)P_n^{\alpha,\beta}(y)-
P_n^{\alpha,\beta}(x)P_{n+1}^{\alpha,\beta}(y)\over x-y}.
\end{equation}




\begin{lemma} Пусть $\alpha>-1$, $k,r$ -- целые, $r\ge1$,
     $k\ge r+1$. Тогда
     $$
P_{k+r}^{\alpha-r,\alpha-r}(x)=\sum_{j=0}^r\lambda_j^\alpha
P_{k+r-2j}^{\alpha,\alpha}(x),
     $$
где
     $$
 \lambda_j^\alpha=\lambda_j^\alpha(r,k)=
{(-1)^j(k-r+2\alpha+1)_{k+r-2j}(1/2)_jr^{[j]}
     (\alpha+k)^{[j]}
\over(k+r-2j+2\alpha+1)_{k+r-2j} (k+r-2j+\alpha+3/2)_j(2j)!}.
     $$
\end{lemma}

\begin{lemma} Пусть  $k,r$ -- целые, $r\ge1$,
     $k\ge r+1$. Тогда
     $$
P_{k+r}^{-\frac12-r,-\frac12-r}(x)=\sum_{j=0}^r\lambda_j^{-\frac12}(k,r)
P_{k+r-2j}^{-\frac12,-\frac12}(x),
     $$
где
$$
 \lambda_j^{-\frac12}(k,r)=
{(-1)^j(k-r)_{k+r-2j}(1/2)_jr^{[j]}
     (k-1/2)^{[j]}
\over(k+r-2j)_{k+r-2j} (k+r-2j+1)_j(2j)!}=
$$
\begin{equation}\label{Haar-Tcheb-eq5.11}
(-1)^j{((k+r-2j)!)^22^{2k+2r-4j}\over(2(k+r-2j))!}
{(2k)!\over(k!)^22^{2k+2r}}{r^{[j]}\over j!}
{k^{[r+1]}\over(k+r-j)^{[r+1]}}.
\end{equation}
\end{lemma}

Пусть $T_n(x)=\cos(n\arccos x)$ -- полином Чебышева первого рода. Тогда \cite{Haar-Tcheb-Sege}
$$P_n^{-\frac{1}{2},-\frac{1}{2}}(x)=\frac{1\cdot3\cdot\ldots\cdot(2n-1)}
{2\cdot4\cdot\ldots\cdot2n}T_n(x)=\frac{(2n)!}{2^{2n}{n!}^2}T_n(x)=$$
\begin{equation}\label{Haar-Tcheb-eq5.12}
=\frac{1}{\sqrt{\pi
n}}(1+\sigma_n)T_n(x)\quad\left(\sigma_n=O\left(1/n\right)\right);
\end{equation}
Имеет место следующая

\begin{lemma}\label{jacobi-repr}
Пусть  $k,r$ -- целые, $r\ge1$,
     $k\ge r+1$. Тогда
 $$
P_{k+r}^{-\frac12-r,-\frac12-r}(x)={(2k)!\over(k!)^22^{2k+2r}}\sum_{j=0}^r{(-1)^j\over j!}
{r^{[j]}k^{[r+1]}\over(k+r-j)^{[r+1]}}T_{k+r-2j}(x).
     $$
\end{lemma}

Пусть $A,B\in \mathbb{R} $,  $p\ge1$. Обозначим через $ L^p_{A,B}$ пространство измеримых функций $f=f(x)$,
определенных на $[-1,1]$, для которых
     $$
\|f\|_{L^p_{A,B}}= \left(\int\limits_{-1}^1
(1-x)^A(1+x)^B|f(x)|^pdx\right)^{1/p}.
     $$
Если $\alpha,\beta>-1$, $f\in  L^p_{\alpha,\beta}$, то мы можем определить коэффициенты
     Фурье-Якоби
          $$
          f^{\alpha,\beta}_k={1\over h_k^{\alpha,\beta}}
       \int\limits_{-1}^1(1-x)^\alpha(1+x)^\beta
f(x)P_k^{\alpha,\beta}(x)dx
          $$
 и рассмотреть сумму Фурье-Якоби   по полиномам Якоби $P_k^{\alpha,\beta}(x)$:

     $$
S_n^{\alpha,\beta}(f)=S_n^{\alpha,\beta}(f,x)= \sum_{k=0}^n
f_k^{\alpha,\beta} P_k^{\alpha,\beta}(x),
     $$
которая при $\alpha=\beta=-\frac12$ представляет собой  \cite{Haar-Tcheb-Sege} сумму Фурье по полиномам Чебышева $T_k(x)=\cos(k\arccos x)$.  В работе \cite{Haar-Tcheb-Muckenhoupt} доказана следующая

 \textbf{Теорема M (Mackenhoupt).}

Пусть $\alpha,\beta>-1$, $A,B\in\mathbb{R}$,  $p>1$ таковы, что
$$
\left|\frac{A+1}{p}-\frac{\alpha+1}{2}\right|<
\min\left\{\frac{1}{4},\frac{\alpha+1}{2}\right\},
$$
$$
\left|\frac{B+1}{p}-\frac{\beta+1}{2}\right|<\min\left\{\frac{1}{4},\frac{\beta+1}{2}\right\}.
$$
Тогда, если $f\in L^p_{A,B}$, то имеет место соотношение
$$
\lim_{n\to\infty}\|f-S_n^{\alpha,\beta}(f)\|_{L^p_{A,B}}=0.
$$


\section{Полиномы, порожденные многочленами Чебышева первого рода}
    Полиномы Чебышева
\begin{equation}\label{Haar-Tcheb-6.1}
\frac{1}{\sqrt{2}},\quad T_k(x)=\cos(k\arccos x), \quad k=1,2,\ldots
\end{equation}
образуют ортонормированную  в $L_\rho^2(-1,1)$ с весом  $\rho(x)=\kappa(x)=\frac2\pi(1-x^2)^{-\frac12}$ систему. Как хорошо известно \cite{Haar-Tcheb-Sege}, система полиномов Чебышева \eqref{Haar-Tcheb-6.1} полна в $L_\kappa^2(-1,1)$.   Эта система порождает на $[-1,1]$ систему полиномов $T_{r,k}(x)$ $(k=0,1,\ldots)$, определенных равенствами

  \begin{equation}\label{Haar-Tcheb-6.2}
T_{r,k}(x) =\frac{(x+1)^k}{k!}, \quad k=0,1,\ldots, r-1,
\end{equation}

  \begin{equation}\label{Haar-Tcheb-6.3}
 T_{r,r}(x) =\frac{(x+1)^r}{\sqrt{2}r!},\quad T_{r,r+k}(x) =\frac{1}{(r-1)!}\int\limits_{-1}^x(x-t)^{r-1}T_k(t)dt, \, k=1,2\ldots.
\end{equation}
Из теоремы \ref{completeness} непосредственно вытекает
\begin{corollary}
  Система полиномов $\{T_{r,k}(x)\}_{k=0}^\infty$, порожденная системой ортонормированных полиномов Чебышева \eqref{Haar-Tcheb-6.1} посредством равенств \eqref{Haar-Tcheb-6.2} и \eqref{Haar-Tcheb-6.3}, полна  в $W^r_{L^2_\kappa(-1,1)}$ и ортонормирована относительно скалярного произведения
\begin{equation}\label{Haar-Tcheb-6.4}
<f,g>=\sum_{\nu=0}^{r-1}f^{(\nu)}(-1)g^{(\nu)}(-1)+\int_{-1}^{1} f^{(r)}(t)g^{(r)}(t)\kappa(t) dt.
\end{equation}
\end{corollary}

Ряд Фурье \eqref{Haar-Tcheb-1.7} для системы   $\{T_{r,k}(x)\}_{k=0}^\infty$ приобретает вид
\begin{equation}\label{Haar-Tcheb-6.5}
f(x)\sim \sum_{k=0}^{r-1} f^{(k)}(-1)\frac{(x+1)^k}{k!}+ \sum_{k=r}^\infty f_{r,k}T_{r,k}(x),
\end{equation}
где
  \begin{equation}\label{Haar-Tcheb-6.6}
f_{r,r}=\frac{1}{\sqrt{2}}\int_{-1}^1 f^{(r)}(t)\kappa(t)dt,\quad f_{r,r+j}=\int_{-1}^1 f^{(r)}(t)T_{j}(t)\kappa(t)dt\quad(j\ge1).
\end{equation}

\begin{corollary}\label{mixed-tcheb-uni-conv}
 Если $f(x)\in W^r_{L^2_\kappa(-1,1)}$, то ряд Фурье (смешанный ряд) \eqref{Haar-Tcheb-6.5} сходится к функции $f(x)$ равномерно относительно $x\in[-1,1]$.
\end{corollary}

В дальнейшем мы значительно усилим утверждение следствия \ref{mixed-tcheb-uni-conv}, распространив его на более широкие, чем $W^r_{L^2_\kappa(-1,1)}$ классы Соболева $W^r_{L^{p}_\kappa(-1,1)}$ с показателем $p$. Но для этого нам нужны дальнейшие свойства полиномов $T_{r,k}(x)$, определенных равенствами \eqref{Haar-Tcheb-6.2} и \eqref{Haar-Tcheb-6.3}.

Пусть $\alpha=\beta=-\frac{1}{2}$. Тогда $\lambda=\alpha+\beta=-1$ и если $(k-1)^{[r]}\neq0$, то мы можем воспользоваться равенством \eqref{Haar-Tcheb-eq5.5} и записать
\begin{equation}\label{Haar-Tcheb-6.7}
P^{-\frac{1}{2},-\frac{1}{2}}_k(t)={2^r \over (k-1 )^{[r]}}\frac{d^r}{dt^r}P_{k+r}^{-\frac{1}{2}-r,-\frac{1}{2}-r}(t).
\end{equation}
 Если  $k\ge r+1$, то, очевидно, $(k-1)^{[r]}\neq0$ и для таких $k$ мы
можем  воспользоваться равенством \eqref{Haar-Tcheb-6.7}. Итак, пусть $(k-1)^{[r]}\neq0$. Тогда в силу  \eqref{Haar-Tcheb-6.7}
$$
\frac{1}{(r-1)!}\int\limits^x_{-1}(x-t)^{r-1}P_k^{-\frac{1}{2},-\frac{1}{2}}(t)\,dt=
$$
$$
\frac{2^r}{(k-1)^{[r]}}\frac{1}{(r-1)!}\int\limits^x_{-1}(x-t)^{r-1}
\frac{d^r}{dt^r}P_{k+r}^{-\frac{1}{2}-r,-\frac{1}{2}-r}(t)\,dt=
$$
\begin{equation}\label{Haar-Tcheb-6.8}
\frac{2^r}{(k-1)^{[r]}}\left[P_{k+r}^{-\frac{1}{2}-r,-\frac{1}{2}-r}(x)-\sum^{r-1}_{\nu=0}
\frac{(1+x)^\nu}{\nu!}\left\{P_{k+r}^{-\frac{1}{2}-r,-\frac{1}{2}-r}(t)
\right\}_{t=-1}^{(\nu)}\right].
\end{equation}
 Далее, в силу \eqref{Haar-Tcheb-eq5.5}
 \begin{equation}\label{Haar-Tcheb-6.9}
\left\{P_{k+r}^{-\frac{1}{2}-r,-\frac{1}{2}-r}(t)\right\}^{(\nu)}=
\frac{(k-r)_\nu}{2^\nu}P_{k+r-\nu}^{-\frac{1}{2}+\nu-r,-\frac{1}{2}+\nu-r}(t),
\end{equation}
а из \eqref{Haar-Tcheb-eq5.9} имеем
$$P_{k+r-\nu}^{-\frac{1}{2}+\nu-r,-\frac{1}{2}+\nu-r}(-1)=(-1)^{k+r-\nu}{k-\frac{1}{2}\choose k+r-\nu}=$$
\begin{equation}\label{Haar-Tcheb-6.10}
\frac{(-1)^{k+r-\nu}\Gamma(k+\frac{1}{2})}{\Gamma(\nu-r+\frac{1}{2})(k+r-\nu)!}.
\end{equation}
Из \eqref{Haar-Tcheb-6.9}  и \eqref{Haar-Tcheb-6.10} находим
\begin{equation}\label{Haar-Tcheb-6.11}
\left\{P_{k+r}^{-\frac{1}{2}-r,-\frac{1}{2}-r}(t)\right\}_{t=-1}^{(\nu)}=
\frac{(-1)^{k+r-\nu}\Gamma(k+\frac{1}{2})(k-r)_{\nu}}
{\Gamma(\nu-r+\frac{1}{2})(k+r-\nu)!2^\nu}=A_{\nu,k,r}.
\end{equation}
Сопоставляя \eqref{Haar-Tcheb-6.8} и \eqref{Haar-Tcheb-6.11} мы можем записать
$$\frac{1}{(r-1)!}\int\limits^x_{-1}(x-t)^{r-1}P_k^{-\frac{1}{2},-\frac{1}{2}}(t)\,dt=$$
\begin{equation}\label{Haar-Tcheb-6.12}
\frac{2^r}{(k-1)^{[r]}}\left[P_{k+r}^{-\frac{1}{2}-r,-\frac{1}{2}-r}(x)-\sum^{r-1}_{\nu=0}
\frac{A_{\nu,k,r}}{\nu!}(1+x)^{\nu}\right].
\end{equation}



Из  \eqref{Haar-Tcheb-6.12},  \eqref{Haar-Tcheb-eq5.12} и \eqref{Haar-Tcheb-6.3}  имеем
$$
T_{r,r+k}(x)=\frac{1}{(r-1)!}\int\limits^x_{-1}(x-t)^{r-1}T_k(t)\,dt=
$$
$$
\frac{2^{2k}k!^2}{(2k)!(r-1)!}\int\limits^x_{-1}(x-t)^{r-1}P_k^{-\frac12,-\frac12}(t)\,dt=
$$
\begin{equation}\label{Haar-Tcheb-6.13}
\frac{k!^2}{(2k)!}
\frac{2^{r+2k}}{(k-1)^{[r]}}\left[P_{k+r}^{-\frac12-r,-\frac12-r}(x)-\sum^{r-1}_{\nu=0}
\frac{A_{\nu,k,r}}{\nu!}(1+x)^{\nu}\right],
\end{equation}
где в силу \eqref{Haar-Tcheb-6.11} и равенств
$$
\Gamma(z)\Gamma(1-z)=\frac{\pi}{\sin(\pi z)},\quad \Gamma(z+1/2)=\frac{\sqrt{\pi}\Gamma(2z)}{\Gamma(z)2^{2z-1}}
$$
 для $k\ge r+1$ находим
$$
A_{\nu,k,r}=
\frac{(-1)^{k+r-\nu}\Gamma(k+1/2)(k-r)_{\nu}}{\Gamma(\nu-r+1/2)(k+r-\nu)!2^\nu}
$$
$$
=\frac{(-1)^{k}\Gamma(k+1/2)(k-r)_{\nu}\Gamma(r-\nu+1/2)}{\pi (k+r-\nu)!2^\nu}=
$$
\begin{equation}\label{Haar-Tcheb-6.14}
\frac{(-1)^{k}(2k-1)!(2(r-\nu)-1)!(k-r)_{\nu}}{(k-1)!(r-\nu-1)! (k+r-\nu)!2^{2(k+r-1)-\nu}}.
\end{equation}
Таким образом, при $k\ge r+1$ мы получаем следующее представление
\begin{equation}\label{Haar-Tcheb-6.15}
T_{r,r+k}(x)=\frac{k!^2}{(2k)!}
\frac{2^{r+2k}}{(k-1)^{[r]}}\left[P_{k+r}^{-\frac12-r,-\frac12-r}(x)-
\sum^{r-1}_{\nu=0}\frac{A_{\nu,k,r}}{\nu!}(1+x)^{\nu}\right].
\end{equation}
Теперь обратимся к лемме \ref{jacobi-repr}, из которой выводим
\begin{equation}\label{Haar-Tcheb-6.16}
\frac{k!^2}{(2k)!}
\frac{2^{r+2k}}{(k-1)^{[r]}}P_{k+r}^{-\frac12-r,-\frac12-r}(x)=
\sum_{j=0}^r(-1)^j{r\choose j}
{kT_{k+r-2j}(x)\over2^r(k+r-j)^{[r+1]}}.
\end{equation}
Сопоставляя \eqref{Haar-Tcheb-6.15} и  \eqref{Haar-Tcheb-6.16}, мы приходим к следующему результату.
\begin{theorem}
Если  $k\ge r+1$, то
\begin{equation}\label{Haar-Tcheb-6.17}
T_{r,r+k}(x)=\sum_{j=0}^r{r\choose j}
{(-1)^jkT_{k+r-2j}(x)\over2^r(k+r-j)^{[r+1]}}
-\frac{k!^22^{r+2k}}{(2k)!(k-1)^{[r]}}
\sum^{r-1}_{\nu=0}\frac{A_{\nu,k,r}}{\nu!}(1+x)^{\nu}.
\end{equation}
\end{theorem}

Рассмотрим два важных частных случая, соответствующие  значениям   $r=1$  и $r=2$.


1). Пусть $r=1$. Тогда из \eqref{Haar-Tcheb-6.14} имеем
\begin{equation}\label{Haar-Tcheb-6.18}
A_{0,k,1}=\frac{(-1)^{k}(2k-1)!}{(k-1)!(k+1)!2^{2k}}, \quad k=2,3,\ldots.
\end{equation}

Из \eqref{Haar-Tcheb-6.17} и \eqref{Haar-Tcheb-6.18}  для $k\ge2$ находим
$$
T_{1,k+1}(x)=\sum_{j=0}^1(-1)^j
{kT_{k+1-2j}(x)\over2(k+1-j)^{[2]}}-\frac{k!^2}{(2k)!}
\frac{2^{2k+1}}{(k-1)}\frac{(-1)^{k}(2k-1)!}{(k-1)!(k+1)!2^{2k}}
$$
$$
=\sum_{j=0}^1(-1)^j
{kT_{k+1-2j}(x)\over2(k+1-j)^{[2]}}-\frac{(-1)^k}{k^2-1}.
$$
Отсюда и из \eqref{Haar-Tcheb-6.2} и \eqref{Haar-Tcheb-6.3} мы водим
\begin{corollary} Имеют место равенства
\begin{equation}\label{Haar-Tcheb-6.19}
T_{1,k+1}(x)={T_{k+1}(x)\over2(k+1)}- {T_{k-1}(x)\over2(k-1)} -\frac{(-1)^k}{k^2-1}\quad (k\ge 2),
\end{equation}
\begin{equation}\label{Haar-Tcheb-6.20}
T_{1,0}(x)=1, \quad T_{1,1}(x)=\frac{1+x}{\sqrt{2}}, \quad T_{1,2}(x)=x^2-1.
\end{equation}
\end{corollary}
2). Для $r=2$ и $k\ge3$ из \eqref{Haar-Tcheb-6.14} и \eqref{Haar-Tcheb-6.17} имеем
$$
A_{0,k,2}=\frac{6(-1)^{k}(2k-1)!}{(k-1)! (k+2)!2^{2(k+1)}},\quad A_{1,k,2}=\frac{(-1)^{k}(2k-1)!(k-2)}{(k-1)! (k+1)!2^{2k+1}},
$$
$$
T_{2,k+2}(x)=\sum_{j=0}^2{r\choose j}
{(-1)^jkT_{k+2-2j}(x)\over2^2(k+2-j)^{[3]}}
-\frac{k!^2}{(2k)!}
\frac{2^{2+2k}}{(k-1)^{[2]}}\sum^1_{\nu=0}\frac{A_{\nu,k,2}}{\nu!}(1+x)^{\nu},
$$
поэтому при $k\ge3$
$$
T_{2,k+2}(x)=\sum_{j=0}^2{2\choose j}
{(-1)^jkT_{k+2-2j}(x)\over4(k+2-j)^{[3]}}
-(-1)^k\left[\frac{1+x}{k^2-1}+\frac{3}{(k^2-1)(k^2-4)}\right].
$$
Отсюда и из \eqref{Haar-Tcheb-6.2} и \eqref{Haar-Tcheb-6.3} мы выводим
\begin{corollary} Имеют место равенства
$$
T_{2,k+2}(x)={T_{k+2}(x)\over4(k+2)(k+1)}-{T_{k}(x)\over2(k^2-1)}+
{T_{k-2}(x)\over4(k-1)(k-2)}-
$$
\begin{equation}\label{Haar-Tcheb-6.21}
(-1)^k\left[\frac{1+x}{k^2-1}+\frac{3}{(k^2-1)(k^2-4)}\right]\quad(k\ge3),
\end{equation}
\begin{equation}\label{Haar-Tcheb-6.22}
T_{2,0}(x)=1, \quad T_{2,1}(x)=1+x, \quad T_{2,2}(x)=\frac{(1+x)^2}{2\sqrt{2}},
\end{equation}
\begin{equation}\label{Haar-Tcheb-6.23}
T_{2,3}(x)=\frac16(x-2)(x+1)^2, \quad T_{2,4}(x)=\frac16x(x-2)(x+1)^2.
\end{equation}
\end{corollary}

\section{Условия равномерной сходимости ряда Фурье по полиномам, ортогональным по Соболеву, порожденным полиномами Чебышева первого рода}
Вернемся к вопросу об условиях равномерной сходимости ряда Фурье (смешанного ряда) \eqref{Haar-Tcheb-6.5}. Имеет место следующая

\begin{theorem}
 Пусть $ A,B\in\mathbb{R},\,p>1$ таковы, что
\begin{equation}\label{Haar-Tcheb-eq7.1}
\left|\frac{A+1}{p}-\frac{1}{4}\right|<\frac{1}{4},\quad
\left|\frac{B+1}{p}-\frac{1}{4}\right|<\frac{1}{4},
\end{equation}
$\rho(x)=(1-x)^A(1+x)^B$. Тогда, если $f\in
W^r_{L_\rho^p}(-1,1)$, то ряд \eqref{Haar-Tcheb-6.5}  равномерно на $[-1,1]$ сходятся к $f(x)$.
\end{theorem}

Из теоремы 5 непосредственно вытекает

\begin{corollary}
 Пусть $\rho(x)=(1-x^2)^{-\frac12}$, $p>1$. Тогда, если $f\in
W^r_{L_\rho^p(-1,1)}$, то ряд \eqref{Haar-Tcheb-6.5}  равномерно на $[-1,1]$ сходятся к $f(x)$.
\end{corollary}

Заметим, что для построения смешанного ряда \eqref{Haar-Tcheb-6.5}(и тем более для его равномерной сходимости) необходимо выполнение условия $f\in W^r_{L_\rho^1(-1,1)}$. Возникает вопрос о том, не является ли это условие также и достаточным для равномерной сходимости ряда \eqref{Haar-Tcheb-6.5}. Другими словами, нельзя ли в следствии 6 заменить   условие $p>1$ на $p=1$. Утвердительный ответ на этот вопрос дает следующая

\begin{theorem}
Пусть $\rho(x)=(1-x^2)^{-\frac12}$, $r\ge1$. Тогда, если $f\in
W^r_{L_\rho^1(-1,1)}$, то ряд \eqref{Haar-Tcheb-6.5}  равномерно на $[-1,1]$ сходятся к $f(x)$.
\end{theorem} 