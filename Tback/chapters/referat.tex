\Referat %Реферат отчёта, не более 1 страницы


Отчет содержит 190~с., 13~иллюстраций,  5~таблиц,  197~источников.

 \bigskip
 \textbf{ Ключевые
  слова:}
  ФУНКЦИОНАЛЬНЫЕ ПРОСТРАНСТВА ЛЕБЕГА И СОБОЛЕВА С ПЕРЕМЕННЫМ ПОКАЗАТЕЛЕМ; ВЕСОВЫЕ ПРОСТРАНСТВА; ТЕОРИЯ ПРИБЛИЖЕНИЙ; ОРТОГОНАЛЬНЫЕ ПОЛИНОМЫ; ПРЕДЕЛЬНЫЕ И СПЕЦИАЛЬНЫЕ РЯДЫ; СПЕЦИАЛЬНЫЕ ДИСКРЕТНЫЕ РЯДЫ; НАИЛУЧШЕЕ ПРИБЛИЖЕНИЕ И СКОРОСТЬ СХОДИМОСТИ; ФУНКЦИЯ ЛЕБЕГА; НЕРАВНОМЕРНЫЕ СЕТКИ; ЭЛЛИПТИЧЕСКИЕ ОПЕРАТОРЫ; СРЕДНИЕ ВАЛЛЕ--ПУССЕНА; УРАВНЕНИЯ ИТО; УРАВНЕНИЯ БЕЛЬТРАМИ; ТЕОРИЯ РАСПИСАНИЙ; ЛУЧЕВОЕ ПРЕОБРАЗОВАНИЕ РАДОНА.
% %%%%%%%%%%%%%%%%%%%%%%%%%%%%%%%%%%%%%%%%%
% базисные последовательности,
% дополняемые подпространства,
% пространства К{\"e}те,
% ква\-зи\-эк\-ви\-ва\-лен\-тность базисов,
% базисы Шаудера, монтелевские прос\-т\-ран\-ства,
% пространства бесконечномерной голоморфности, янгиан, супералгебра Ли, скрученный янгиан,
% янгианный модуль, универсальная $R$-матрица, тeплицева матрица,
% теорема Сег{\"e},\linebreak асимптотика спектра, гиперболический оператор, оценки
% роста собственных функций, интерполяционная тройка, ко\-нус,  оператор
% свертки, гиперсингулярные операторы, операторы типа потенциала,
% пространства Харди, пространства пе\-ре\-мен\-но\-го порядка, свертка,
% осциллирующий символ, обобщенная \linebreak функция, мультипликатор.}

 \bigskip


%\textbf{Отчет состоит из} титульного листа, Списка исполнителей, Реферата, Содержания, Обозначений и сокращений, Введения, Основной части, Заключения, Списка использованных источников. Основная часть содержит 6 глав, разбитых на параграфы и пункты.

Настоящий отчёт содержит итоги работы за 2014 год Отдела математики и информатики ДНЦ РАН по 6 темам
%осуществлению фундаментальных научных исследований в соответствии с
из Программы фундаментальных научных исследований государственных академий наук на 2013–2020 годы.

%тема 1

\textit{Тема 1}.
\input chapters/refs/ref1.tex


%тема 2
\textit{Тема 2}.
\input chapters/refs/ref2.tex


%тема 3
\textit{Тема 3}.
\input chapters/refs/ref3.tex


%тема 4
\textit{Тема 4}.
\input chapters/refs/ref4.tex

%тема 5
\textit{Тема 5}.
\input chapters/refs/ref5.tex


%тема 6
\textit{Тема 6}.
\input chapters/refs/ref6.tex





%\textbf{Отчет содержит:}  190 страниц,  13 иллюстраций,  5 таблиц,  197 источников.
%
%%\textbf{Библиография} содержит ссылки на  источников.
%
%\textbf{Ключевые слова:} функциональные пространства Лебега и Соболева с переменным показателем; весовые пространства; теория приближений; ортогональные полиномы; предельные и специальные ряды; специальные дискретные ряды; наилучшее приближение и скорость сходимости; функция Лебега; неравномерные сетки; эллиптические операторы; средние Валле--Пуссена; уравнения Ито; уравнения Бельтрами; теория расписаний; лучевое преобразование Радона.


