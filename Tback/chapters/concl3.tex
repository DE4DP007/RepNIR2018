%СММ->
В отчетном году получены оценки усреднения и операторные оценки усреднения для обобщенных уравнений Бельтрами. Эти уравнения являются недивергентными, поэтому полученные операторные оценки усреднения  отличаются от оценок для дивергентных операторов. Применены асимптотические методы.

%КРИ->>
%За отчетный период б
Были исследованы вопросы глобальной экспоненциональной $p$-устой\-чи\-вос\-ти $(2 \le p < \infty)$ систем линейных дифференциальных уравнений Ито с запаздываниями
специального вида, используя теорию положительно обратимых матриц.
Кроме того, были исследованы вопросы асимптотической
$p$-устойчивости ($2 \le p < \infty $) тривиального решения
относительно начальных данных для линейной однородной импульсной
системы дифференциальных уравнений Ито с линейными запаздываниями
методом вспомогательных или модельных уравнений. Получены достаточные условия устойчивости в терминах параметров исследуемых систем. Результаты исследований опубликованы в работах \cite{kad11,kad12,kad13,kad14}.

%АЭИ->>
Изучены вопросы единственности положительного решения задачи Дирихле
для уравнения \eqref{EI1} при $n\geq 2.$  Результаты
относительно единственности положительного радиально-симметричного
решения, полученные в \cite{LitEI7, LitEI8, LitEI9} в случае
$a(|x|)=|x|^m, m\geq 0 $, обобщены на более общий случай $ a(|x|).$

Получены достаточные условия существования и единственности
 положительного решения двухточечной краевой задачи для нелинейного
 дифференциального уравнения с дробными производными и разработан
 численный метод его построения.
 
%МЗГ->
Были введены в рассмотрение два двухпараметрических семейства ломаных на плоскости. Решена задача восстановления функции по ее интегралам вдоль этих ломаных, когда весовая функция --- квазимногочлен. Для частных случаев весовых функций получены формулы обращения. В общем случае доказана единственность решения поставленной задачи. Результаты применены к доказательству единственности задачи интегральной геометрии с возмущением.

Доказана единственность восстановления функции, суммируемой в полосе на плоскости, заданной своими интегралами вдоль дуг двухпараметрических кривых второго порядка с весом, аналитическим по части переменных.

Доказаны формулы для определения неизвестного векторного поля на плоскости, заданного своим поперечным лучевым преобразованием в ограниченном угловом диапазоне. В первой формуле используется интегральная формула интерполяции функции с ограниченным спектром. Восстанавливается преобразование Фурье дивергенции неизвестного поля. Во второй формуле интерполяция производится по дискретным значениям функции. Восстанавливаются координатные функции потенциальной части искомого поля. Векторное поле ищется в классе вектор-функций, сосредоточенных в некоторой полосе, достаточно быстро убывающих на бесконечности и име-ющих непрерывные вторые производные.





%%%%%%%%%%%%%%%%%%%%%%%%%%%%%
%%%%%%%%% ПРОШЛЫЙ ГОД %%%%%%%
%%%%%%%%%%%%%%%%%%%%%%%%%%%%%



%
%В отчетном году продолжены исследования эллиптических уравнений второго порядка. Для недивергентных эллиптических уравнений второго порядка, коэффициенты которых локально периодичны (с малым периодом) по одной из переменных выведены усредненные уравнения.
%
%В задаче Римана -- Гильберта для системы уравнений Бельтрами доказано свойство гельдеровости решения, а для обобщенных уравнений Бельтрами получены оценки усреднения решения.
%
%В случае уравнения Бельтрами с периодическим коэффициентом, зависящим от малого параметра, построено асимптотическое разложение решения задачи Римана – Гильберта и оценена невязка.
%
%
%
%
%Для задачи Дирихле для нелинейного дифференциального уравнения с $p(x)$-лапласиа\-ном конструируются верхнее и нижнее решения путем склеивания на двух участках.
%Построенные верхнее и нижнее решения позволяют не только обосновать существование слабого решения, но и оценить решение сверху и  снизу.
%
