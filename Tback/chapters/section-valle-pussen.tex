\section{Аппроксимативные свойства классических средних Валле-Пуссена для кусочно-гладких функций}

Пусть $f(x)$ -- некоторая суммируемая $2\pi$-периодическая функция. Рядом Фурье для такой функции называют ряд
\begin{equation*}
\frac{a_0}{2}+\sum\limits_{k=1}^\infty a_k \cos kx + b_k \sin kx,
\end{equation*}
где $a_k=\dfrac{1}{\pi}\int\limits_{0}^{2\pi} f(t)\cos kt dt$, $b_k=\dfrac{1}{\pi}\int\limits_{0}^{2\pi} f(t)\sin kt dt$.

Частичные суммы ряда Фурье порядка $n$ определяются следующим образом:
\begin{equation*}
S_n(f,x) = \frac{a_0}{2}+\sum\limits_{k=1}^{n} a_k \cos kx + b_k \sin kx.
\end{equation*}

Средние Валле-Пуссена вводятся как средние арифметические частичных сумм Фурье:
\begin{equation*}
V_n(f,x) = \dfrac{1}{n}\sum\limits_{k=n}^{2n-1} S_k(f,x) = \dfrac{S_n(x)+S_{n+1}(x)+\ldots+S_{2n-1}(x)}{n}.
\end{equation*}

Целью данной параграфа является исследование скорости сходимости средних Валле-Пуссена для кусочно гладких функций $f(x)$ из пространства $W_\infty^{3,\mathds{A}}$ (определение приведено ниже), т.е. получение оценки скорости стремления к нулю величины $|f(x)-V_n(f,x)|$ при $n\rightarrow\infty$.

\noindent Основное утверждение доказывается с помощью следующей леммы.
\begin{lemma}\label{SumSinCosLemma}
Для функций
\begin{equation*}
f(x)=\sum\limits_{k=1}^\infty \dfrac{\sin kx}{k}, \quad f(x)=\sum\limits_{k=1}^\infty \dfrac{\cos kx}{k}
\end{equation*}
имеет место оценка $(\varepsilon > 0, n \ge 1)$
$$
|f(x) - V_n(f,x)| \leq \dfrac{1}{\sin^2\frac{\varepsilon}{2}} \cdot \dfrac{1}{n(n+1)}, \quad x\in M=[\varepsilon,2\pi-\varepsilon].
$$
\end{lemma}
Доказательство данной леммы основано на использовании преобразования Абеля. Напомним, что преобразованием Абеля называется формула
\begin{equation*}
\sum\limits_{k=1}^n \alpha_k \beta_k = \sum\limits_{k=1}^{n-1} (\alpha_k-\alpha_{k+1})B_k + \alpha_n B_n, \quad B_k=\sum\limits_{j=1}^k \beta_j.
\end{equation*}
Если $\alpha_nB_n \to 0,\, n \to \infty,$ то преобразование Абеля обобщается на ряды:
\begin{equation*}
\sum\limits_{k=1}^\infty \alpha_k \beta_k = \sum\limits_{k=1}^{\infty} (\alpha_k-\alpha_{k+1})B_k.
\end{equation*}
Полное доказательство см. в \cite{mmg-classic-vallee-pussen}

Для формулировки основного результата нам потребуются некоторые обозначения.
Пространством Соболева $W_p^r([a,b])$ называется множество $r-1$ раз непрерывно дифференцируемых на $[a,b]$ функций $f(x)$, для которых $f^{(r-1)}(x)$ абсолютно непрерывна на $[a,b]$, а $f^{(r)}(x)\in L^p([a,b])$.
Через $\|f\|_\infty$ будем обозначать $\operatorname*{ess\,sup}\limits_{[0,2\pi]}|f(x)|$.
%имеющих $r-1$ непрерывную производную на $[a,b]$, причем $f^{(r-1)}$ абсолютно непрерывна на $[a,b]$.

\noindent Далее, пусть дано конечное разбиение отрезка $[0,2\pi]$
$$
    \mathds{A} = \{0=\theta_0<\theta_1<\ldots<\theta_q=2\pi\}.
$$
Тогда через $W_\infty^{3,\mathds{A}}$ обозначим класс $2\pi$-периодических функций, которые на каждом отрезке $[\theta_i,\theta_{i+1}]$ можно превратить в функцию из $W_\infty^3([\theta_i,\theta_{i+1}])$ путём переопределения её на концах. Другими словами, если $f(x) \in W_\infty^{3,\mathds{A}}$, то, во-первых, $\operatorname*{ess\,sup}_{[0,2\pi]}|f'''(x)| < \infty$ и, во-вторых, на каждом отрезке $[\theta_i,\theta_{i+1}]$ функции $f(x)$, $f'(x)$ и $f''(x)$ можно сделать абсолютно непрерывными, изменив $f(x)$ разве что лишь в двух точках -- концах отрезка.

\noindent Имеет место теорема (доказательство см. в \cite{mmg-classic-vallee-pussen}).
\begin{theorem}
Для функций $f(x)$ из класса $W_\infty^{3,\mathds{A}}$ справедлива следующая оценка остатка при приближении суммами Валле-Пуссена ($n \ge 1$):
\vspace{-3mm}
$$
%|\mathfrak{R}_n(f,x)|=
|f(x)-V_n(f,x)| \leq
\Bigl[
\frac{6q}{\pi \sin^2\frac{\varepsilon}{2}}+2
\Bigr] \cdot
\frac{M_f}{n(n+1)}, \quad
x \in D = \bigcup\limits_{i=1}^{q}[\theta_{i-1}+\varepsilon,\theta_{i}-\varepsilon]
$$
где $M_f=\max\{\|f\|_\infty,\|f'\|_\infty,\|f''\|_\infty\,\|f'''\|_\infty\}$, $\varepsilon$ -- любое положительное число.
\end{theorem}
