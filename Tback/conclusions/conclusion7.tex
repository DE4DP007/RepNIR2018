\Conclusion

В 2016 году в Отделе математики и информатики Дагестанского научного центра РАН проведены научно-исследовательские работы по теме
<<Разработка алгоритма для численно-аналитического решения задачи Коши для обыкновенного дифференциального уравнения на основе полиномов, ортогональных по Соболеву, порожденных полиномами Чебышева первого рода>>.



В отчетный период сотрудниками ОМИ были сконструированы новые системы функций, ортогональные в смысле Соболева, порожденные классическими ортогональными системами. Установлена связь между рядами Фурье по соболевским полиномам и смешанными рядами, введенными Шарапудиновым И.И.
Исследован ряд аппроксимативных свойств рядов Фурье по вновь построенным системам. Разработан алгоритм для численно-аналитического решения задачи Коши для обыкновенных дифференциальных уравнений (ОДУ) на основе полиномов, ортогональных по Соболеву, порожденных полиномами Чебышева, ортогональными на сетке, и системой функций Хаара.
Была создана программа, которая позволяет найти решение задачи Коши для линейных ОДУ численным методом, основанным на разложении функции и ее производных в ряд Фурье по полиномам, ортогональным по Соболеву и порожденным упомянутыми системами функций. Основное преимущество данного подхода заключается в том, что в разложении  функции и всех ее производных участвуют одни и те же коэффициенты.
При этом оказываются учтенными начальные условия задачи Коши. Программа может быть использована в ряде прикладных направлений (мат. физика, мат. моделирование и др.). Данная программа зарегистрирована (Свидетельство № 2016617831 о государственной регистрации программы для ЭВМ <<MixedHaarDeqSolver>>).




