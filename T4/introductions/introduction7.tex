\Introduction

Согласно плану научно-исследовательской работы за 2014--2016 годы исследования, проводимые в Отделе математики и информатики Дагестанского научного центра РАН, включают в себя работы по теме
<<Разработка алгоритма для численно-аналитического решения задачи Коши для обыкновенного дифференциального уравнения на основе полиномов, ортогональных по Соболеву, порожденных полиномами Чебышева первого рода>>.


Для численного решения задачи Коши для обыкновенного дифференциального уравнения наряду с различными сеточными методами часто применяют так называемые спектральные методы. Напомним, что суть спектрального метода решения задачи Коши  для ОДУ  заключается в том, что в первую очередь искомое решение $y(x)$ представляется в виде ряда Фурье по подходящей ортонормированной системе (чаще всего в качестве таковой используют одну из классических ортонормированных систем или систему  всплесков (вэйвлетов)). Затем полученный ряд подставляется вместо $y(x)$ в исходное дифференциальное уравнение. Это приводит к системе уравнений относительно неизвестных коэффициентов Фурье $\hat y_k$, $k=0,1,\ldots$. Далее, требуется решить эту систему с учетом начальных условий  $y^{(k)}(0)=y_k$, $k=0,1,\ldots,r-1$, задачи Коши. Одна из основных трудностей, которые возникают на этом этапе, состоит в том, чтобы
выбрать такой ортонормированный базис, для которого искомое решение $y(x)$ уравнения, представленное в виде ряда, удовлетворяло бы начальным условиям. Более того, поскольку в результате решения системы уравнений относительно неизвестных коэффициентов $\hat y_k$  будет найдено только конечное их число с $k=0,1,\ldots, n$, то весьма важно, чтобы частичная сумма указанного ряда, будучи приближенным решением рассматриваемой задачи Коши, также удовлетворяла бы начальным условиям. Как оказалось, базис, состоящий из функций, ортонормированных по Соболеву относительно специально заданного скалярного произведения типа Соболева, обладает указанными свойствами. В отчетном году были сконструированы системы функций, ортогональные в смысле Соболева, порожденные классическими ортогональными системами. Установлена связь между рядами Фурье по соболевским полиномам и смешанными рядами, введенными в работах Шарапудинова И.И.
Изучены аппроксимативные свойства рядов Фурье по построенным системам функций. Разработан алгоритм для численно-аналитического решения задачи Коши для обыкновенных дифференциальных уравнений (ОДУ) на основе полиномов, ортогональных по Соболеву, порожденных дискретными полиномами Чебышева и системой функций Хаара.
Была разработана программа, которая позволяет найти решение задачи Коши для линейных ОДУ численным методом, основанным на разложении функции и ее производных в ряд Фурье по полиномам, ортогональным по Соболеву и порожденным упомянутыми системами функций. Преимущество такого подхода заключается в том, что в разложении  функции и всех ее производных участвуют одни и те же коэффициенты. При этом оказываются учтенными начальные условия задачи Коши. Программа может быть использована в ряде прикладных направлений (мат. физика, мат. моделирование и др.) Данная программа зарегистрирована (Свидетельство № 2016617831 о государственной регистрации программы для ЭВМ <<MixedHaarDeqSolver>>).


%В отчетный период сотрудниками ОМИ были сконструированы новые системы функций, ортогональные в смысле Соболева, порожденные классическими ортогональными системами. Установлена связь между рядами Фурье по соболевским полиномам и смешанными рядами, введенными в работах Шарапудинова И.И. Изучены аппроксимативные свойства рядов Фурье по построенным системам функций. Разработан алгоритм для численно-аналитического решения задачи Коши для обыкновенных дифференциальных уравнений (ОДУ) на основе полиномов, ортогональных по Соболеву, порожденных дискретными полиномами Чебышева и системой функций Хаара. Была составлена программа, которая позволяет найти решение задачи Коши для линейных ОДУ численным методом, основанным
%на разложении функции и ее производных в ряд Фурье по полиномам, ортогональным по Соболеву и порожденным упомянутыми системами функций. Основное преимущество данного подхода заключается в том, что в разложении  функции и всех ее производных участвуют одни и те же коэффициенты. При этом
%оказываются учтенными начальные условия задачи Коши. Программа может быть использована в ряде прикладных направлений (мат. физика, мат. моделирование и др.) Данная программа зарегистрирована (Свидетельство № 2016617831 о государственной регистрации программы для ЭВМ <<MixedHaarDeqSolver>>).




