Фундаментальные научные исследования в соответствии с Программой фундаментальных научных исследований государственных академий наук на 2013–2020 годы.
Пункт программы ФНИ государственных академий наук на 2013–2020 годы и наименование направления исследований в части:

1. Теоретическая математика. Исследования по теории аппроксимации и интерполяции в вещественной и комплексной области. Развитие теории функциональных пространств и многомерного гармонического анализа. исследования по функциональному анализу, анализу Фурье и теории сингулярных интегральных операторов, спектральной теории операторов в гильбертовом пространстве. Исследования по теории ортогональных рядов и ее применениям и геометрическим проблемам одномерного и многомерного комплексного анализа; применение результатов этих исследований в математической и теоретической физике. развитие теории устойчивости и исследование качественных свойств решений дифференциально-разностных и интегро-дифференциальных уравнений и их приложений. Развитие общей теории дифференциальных уравнений в частных производных и ее приложений к задачам математической физики, в частности, исследование системы уравнений Навье-Стокса. Изучение связанных с физикой краевых задач для эллиптических, параболических и гиперболических уравнений; их применение в технических разработках. Развитие общей теории эллиптических дифференциальных уравнений в частных производных и ее приложений к задачам математической физики. Некоторые вопросы теории приближений в функциональных пространствах с переменным показателем суммируемости и их приложения (№ гос. регистрации: 01201266508). Развитие теории устойчивости и исследование качественных свойств решений дифференциально-разностных и интегро-дифференциальных уравнений и их приложение.


2. Вычислительная математика. Разработка вычислительных алгоритмов по массопереносу в пористых и трещинных средах.
Разработка и исследование новых многомасштабных алгоритмов, матричных методов, методов решения систем линейных и нелинейных уравнений, задач на собственные значения, численно-аналитических методов конформных отображений, методов многомерной высокоточной монотонной интерполяции и дискретизации, методов быстрого автоматического дифференцирования, методов вычислений характеристик моделей с минимальной сложностью, построение квадратурных формул высокой точности для сингулярных интегралов и интегралов типа Коши.


3. Математическое моделирование. Транспортные задачи; задачи составления расписаний; приближенные вероятностные алгоритмы решения транспортных задач на графах.


5. Теоретическая информатика и дискретная математика. Разработка алгоритмов для распознавания изображений на мобильных платформах, в распределенных информационных системах, применение технологии «облачных вычислений» для анализа изображений. Разработка алгоритмов и создание наукоемкого программного обеспечения для моделирования сложных систем, возникающих в задачах обработки сигналов и изображений. Разработка математических методов и алгоритмов распознавания образов и восстановление зависимостей по некомплектным и зашумленным данным. Разработка алгоритмов восстановления векторных и тензорных полей по их лучевым преобразованиям. Исследования по теории расписаний. 