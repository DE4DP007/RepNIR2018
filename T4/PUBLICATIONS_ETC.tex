% Generated by GrindEQ Word-to-LaTeX 2010
% ========== UNREGISTERED! ========== Please register! ==========
% LaTeX/AMS-LaTeX

\documentclass{article}

%%% remove comment delimiter ('%') and specify encoding parameter if required,
%%% see TeX documentation for additional info (cp1252-Western,cp1251-Cyrillic)
%\usepackage[cp1252]{inputenc}

%%% remove comment delimiter ('%') and select language if required
%\usepackage[english,spanish]{babel}

\usepackage{amsthm,amsfonts,amsmath,amssymb,amscd} % Математические дополнения от AMS
\usepackage[T2A]{fontenc}
\usepackage[utf8]{inputenc} %% ваша любимая кодировка здесь
\usepackage[english,russian]{babel} %% это необходимо для включения переносов
\usepackage{float}
\usepackage[dvips]{graphicx}
\usepackage[all,cmtip]{xy}
%%% remove comment delimiter ('%') and specify parameters if required
%\usepackage[dvips]{graphics}

\begin{document}

%%% remove comment delimiter ('%') and select language if required
%\selectlanguage{spanish}

\noindent
\[1\]









\textbf{2. ПУБЛИКАЦИИ В 2014 Г.}

\noindent

\begin{enumerate}
\item  \textbf{Sharapudinov I.I. }On Direct And Inverse Theorems Of Approximation Theory In Variable Lebesgue And Sobolev Spaces // Azerbaijan Journal of Mathematics. 2014. Vol. 4, №1. 55-72.

\item  \textbf{R.Kadiev, A.Ponosov. }Partial stability of stochastic functional differential equations and the W-transform. // The International Journal of Dynamics of Continuous, Discrete \& Impulsive Systems (DCDIS) Series A: Mathematical Analysis, Vol. 21, No. 1 (2014), p.1-35.

\item  \textbf{Магомедов А.М. }Сокращение перебора двудольных графов // Международный научно-исследовательский журнал, N 3 \eqref{GrindEQ__22_}, 2014. стр. 11-12. ISSN 2227 -- 6017.

\item  Шарапудинов И.И. Приближение функций из пространств Лебега и Соболева с переменным показателем суммами Фурье-Хаара // Мат. сборник. 2014. Том 205, №2. С. 145-160.

\item  Шарапудинов И.И.\textbf{ }Некоторые специальные ряды по ультрасферическим полиномам и их аппроксимативные свойства // Известия РАН: Серия математическая. 2014. Том 78, вып. 5. 201--224.

\item  Шарапудинов И.И.\textbf{ }Пространство Соболева с переменным показателем и приближение алгебро-тригонометрическими полиномами // Вестник Дагестанского научного центра РАН. 2014. № 53. С. 5-21.

\item  Шарапудинов И.И.\textbf{ }Некоторые специальные двумерные ряды по системе  и их аппроксимативные свойства // Изв. Сарат. ун-та. Нов. сер. Сер. Математика. Механика. Информатика. 2014. Т. 14, вып. 4, часть 1. С. 407-412.

\item  Сиражудинов М.М.,\textbf{ }Джамалудинова С.П.\textbf{ }О G-компактности одного класса эллиптических операторов второго порядка с комплекснозначными коэффициентами// Вестник ДГУ, 2014, Вып.1, С.77-80.

\item  Сиражудинов М.М., Джамалудинова С.П.\textbf{ }Усреднение одного эллиптического уравнения второго порядка с комплекснозначными периодическими коэффициентами// Вестник ДГУ, 2014, Вып. 6.

\item  Абдулаев Ш.-С.О., Черкашин В.И. Новая концепция устойчивости эколого-экономического развития горных экосистем (на примере РД) // Аридные экосистемы, №4\eqref{GrindEQ__61_}, Том 20, С. 5-10, 2014 г.

\item  Абдурагимов Э.И. «Существование положительного решения двухточечной краевой задачи для одного нелинейного ОДУ четвертого порядка». Вестник СамГУ, 2014, № 10\eqref{GrindEQ__121_}, с. 9-16.

\item  Абдурагимов Э.И., Омарова Р.А. «Численный метод построения положительного решения двухточечной краевой задачи для одного дифференциального уравнения второго порядка с дробной производной». Вестник ДГУ, 2014, № 6, с.40-46.

\item  Абдурагимов Э.И., Гаджиева Т.Ю. «Численный метод нахождения радиально-симметричного положительного решения задачи Дирихле для одного нелинейного дифференциального уравнения второго порядка». Вестник ДГУ, 2014, № 6, с.47-52.

\item  Шарапудинов И.И., Акниев Г.Г. Дискретные преобразования со свойством прилипания на основе системы  и системы полиномов Чебышева второго рода // Изв. Сарат. ун-та. Нов. сер. Сер. Математика. Механика. Информатика. 2014. Т. 14, вып. 4, часть 1. С. 413-422.

\item  Кадиев Р.И. Устойчивость решений линейных разностных уравнений Ито с последействием. // Дифференц. уравнения. Минск. Т.50, №, 2014. С. (принята к печати 29 сентября 2014 г.)

\item  Кадиев Р.И., Шахбанова З.И. Устойчивость решений скалярных линейных разностных уравнений Ито с последействием // Вестник Дагестанского госуниверситета 2014 , № 1, стр. 97-103.

\item  Кадиев Р.И., Поносов А.В. W-метод Н.В. Азбелева в теории линейных стохастических функционально-дифференциальных уравнений // Вестник Пермского университета. Математика. Механика. Информатика. 2014, выпуск 2\eqref{GrindEQ__25_}, стр. 10-14.

\item  Меджидов З.Г. Обращение лучевого преобразования симметричного тензорного поля с источниками на кривой. Вестник ДГУ. Вып 6. 2013. С. 107-113. (не вошел в прошлогодний отчет)

\item  Магомед-Касумов М.Г. Базисность системы Хаара в весовых пространствах Лебега с переменным показателем // Владикавказский математический журнал. 2014. Том 16, вып. 3. С. 38-46.

\item  Магомед-Касумов М.Г. Приближение функций суммами Хаара в весовых пространствах Лебега и Соболева с переменным показателем // Изв. Сарат. ун-та. Нов. сер. Сер. Математика. Механика. Информатика. 2014. Т. 14, вып. 3. С. 295-304.

\item  Магомед-Касумов М.Г. Аппроксимативные свойства классических средних Валле-Пуссена для кусочно гладких функций // Вестник ДНЦ. 2014. Вып. 54. С. 1-12.

\item  Магомедов А.М. К вопросу об интервальной раскраске двудольных графов // Автоматика и телемеханика / в печати (указанный редакцией срок -- декабрь 2014).

\item  Лавренченко C.A, Магомедов A.M. К гипотезе Грюнбаума о раскраске ребер графа // Вестник Дагестанского государственного университета. 2014. Вып. 6. C. 27-31.

\item  Рамазанов А-Р.К. Одна модификация построения поля действительных чисел// Вестник ДГУ -- Махачкала, 2014. -- Вып. 1. -- С. 63 -- 67.

\item  Султанахмедов М.С. Асимптотические свойства и весовые оценки полиномов, ортогональных на неравномерной сетке с весом Якоби // Известия Саратовского Университета. Новая Серия. Серия: Математика. Механика. Информатика. 2014. Том 14, вып. 1. С. 38-47.

\item  Гасанов Г.Н., Султанахмедов М.С. и др. Гидротермические условия формирования видового состава и продуктивности фитоценозов Терско-Кумской низменности // Аридные Экосистемы. 2014. Т. 20, №4\eqref{GrindEQ__61_}. С. 93-98.

\item  Гасанов Г.Н., Асварова Т.А., Гаджиев К.М., Ахмедова З.Н., Абдулаева А.С., Баширов Р.Р., Султанахмедов М.С. Теоретически возможная и практически реализуемая по условиям влагообеспеченности и засоленности продуктивность светло-каштановой почвы Северо-Западного Прикаспия (на примере Кочубейской биосферной станции ПИБР ДНЦ РАН) // ЮГ РОССИИ: ЭКОЛОГИЯ, РАЗВИТИЕ. 2014. Т. 31, №2\eqref{GrindEQ__31_}. С. 130-138.

\item  Шах-Эмиров Т.Н. О равномерной ограниченности некоторых семейств интегральных операторов свертки в весовых пространствах Лебега спеременным показателем. // Изв. Сарат. ун-та. Нов. сер. Сер. Математика.Механика. Информатика. 2014. Т. 14, вып. 4, ч. 1. С. 422-428

\item  Шах-Эмиров Т.Н. О равномерной ограниченности семейства операторов Стеклова в весовых пространствах Лебега с переменным показателем // Вестн. ДНЦ РАН. 2014. Вып. 54. С. 12--17.

\item  Абдулаев Ш.-С.О., Идармачев Ш.Г. и др. Анализ изменения газового состава изливающихся скважин сети пунктов наблюдения в сейсмоактивной области Дагестана // Труды Института геологии ДНЦ РАН, №62, 2013 г. (0,9 печ.листа)

\item  Абдулаев Ш.-С.О., Черкашин В.И. и др. Анализ сейсмического материала в целях среднемасштабного сейсмического районирования территории Дагестана // Труды Института геологии ДНЦ РАН, №62, 2013 г. (0,8 печ.листа)

\item  Абдулаев Ш.-С.О., Садыкова А.М. Специализация промышленного производства регионов в рамках стратегического планирования России (на примере РД) // VIII Международная НПК «Россия: тенденции и перспективы развития». Вып. 9, часть 2, М.2014. (0,5/25)

\item  Абдулаев Ш.-С.О., Садыкова А.М., Деневизюк Д.А. Модернизация и инновации в промышленности для достижения стратегических целей // Региональные проблемы преобразования экономики №7, 2014. (0,5/03)

\item  Магомедов А.М. Двудольные (6,3)-бирегулярные графы, не допускающие интервальной раскраски // Дагестанские электронные математические издания. --  ~Том 1. -- 2013/14. -- С. 70-81.

\item  Магомедов А.М., Магомедов М.А. Веб-камеры в проектах Delphi. Учебное пособие. -- Махачкала: Изд-во ООО "Радуга-1", 2014. -- 38 с.

\item  Магомедов А.М. Учебное пособие по дискретной математике (макетирование и просмотр). Учебное пособие -- Махачкала: Изд-во ООО "Радуга-1", 2014. - 32 с.

\item  Магомедов А.М. Основы программирования для математиков. Часть 1. Курс лекций. -- Махачкала: Изд-во ООО "Радуга-1", 2014. -- 58 с.
\end{enumerate}

\noindent Абдулаев Ш.-С.О., Алиев Э.А., Гаджиагаев В.А., Магомедов Д.А. Математические и компьютерные методы в медицине, биологии и экологии. // Монография под научн.ред. В.И. Левина. Вып. 2, г. Пенза, г. Москва: Приволжский дом знаний; МИЭМП, 2013 г., 112 с. (вышла в 2014 г.) \eject

\begin{enumerate}
\item  Меджидов З.Г. Восстановление симметричного тензорного поля по его интегралам вдоль прямых, пересекающих бесконечно удаленную кривую. Материалы межд. конф. «Мухтаровские чтения -- Современные проблемы математики и смежные вопросы», 17-18 апреля 2014г.

\item  Сиражудинов М.М. «Метод асимптотических разложений для системы уравнений Бельтрами»// Международная конференция по дифференциальным уравнениям  и динамическим системам, г. Суздаль, 2014.

\item  Сиражудинов М.М. «О концевом символе А.П. Солдатова», г. Махачкала, Материалы межд.науч.конф. «Мухтаровские чтения», С.100-101.

\item  Абдулаев Ш.-С.О. Политика и экономика инновационного развития субъектов РФ для формирования современных производственных сил // Материалы конференции «IV Всероссийская научно-практическая конференция ~«Региональные проблемы преобразования экономики: социально-демографические приоритеты субъектов СКФО РФ» (6-7 ноября 2013 г.)

\item  Абдулаев Ш.-С.О. Преобразования гражданского общества для повышения эффективности региональной инновационной экономики // Материалы всероссийской научно-практической конференции «Общее и особенное в формировании гражданского общества на Северном Кавказе », (30 мая 2014 г.). Изд. ДГУ, 2014, С. 21-23.

\item  Шарапудинов И.И., Акниев Г.Г. Специальные дискретные преобразования посредством полиномов Чебышева второго рода, со свойством прилипания // ~Современные проблемы теории функций и их приложения: Материалы 17-й Сарат. Зимней школы. -- Саратов: ООО «Издательство «Научная книга», 2014. -- с. 299-302. (Махачкала, ДНЦ РАН)

\item  Алишаев М.Г. «Неизотермическая фильтрация почвенного воздуха с фазовыми переходами вода-пар» // Материалы научной сессии ДНЦ РАН, посвящённой Дню российской науки. Возобновляемая энергетика: проблемы и перспективы // Выпуск 3. Махачкала, АЛЕФ, 2014. С. 100-110.

\item  R. Kadiev, A. Ponosov. ~Stability of linear impulsive Ito equations with delay ~(тезисы доклада на заочной конференции в Тбилиси, декабрь 2014)

\item  Магомед-Касумов М.Г. Приближение функций суммами Хаара в весовых пространствах Лебега с переменным показателем // ~Современные проблемы теории функций и их приложения: Материалы 17-й Сарат. Зимней школы. -- Саратов: ООО «Издательство «Научная книга», 2014. -- с. 173-176.

\item  Магомед-Касумов М.Г. ~Приближение кусочно гладких функций суммами Валле-Пуссена // Тезисы докладов Международной научной конференции "Теория операторов, комплексный анализ и математическое моделирование" (пос. Дивноморское, 7-13 сентября 2014 года). С. 52-53.

\item  Магомедов А.М. (6,3)-бирегулярные графы, нераскрашиваемые интервально 6 цветами // Материалы XVII международной конференции «Проблемы теоретической кибернетики» (Казань, 16-20 июня 2014 г.). Под общей редакцией Ю.И. Журавлева. -- Казань: Отечество, 2014. -- 307 с. (с. 182-183).

\item  Магомедов А.М., Магомедов Т.А. Паросочетания специального вида в задачах расписаний / Информационные технологии в профессиональной деятельности и научной работе: сборник материалов Всероссийской научно-практической конференции с международным участием. Поволжский технологический университет, 2014. -- 336 с. (с. 25-29).

\item  Рамазанов А-Р.К. Оценка скорости полиномиального приближения ограниченной функции относительно ограниченного знакочувствительного веса // Сборник «Актуальные проблемы математики и смежные вопросы». ~Матер. Междунар. конф. «Мухтаровские чтения». -- Махачкала, 2014. С. 86-88.

\item  Султанахмедов М.С. Некоторые специальные ряды по полиномам, ортогональным на неравномерных сетках, и их аппроксимативные свойства // ~Современные проблемы теории функций и их приложения: Материалы 17-й Сарат. Зимней школы. -- Саратов: ООО «Издательство «Научная книга», 2014. -- с.

\item  Шарапудинов И.И. Двумерные специальные ряды по системе  и их аппроксимативные свойства // ~Современные проблемы теории функций и их приложения: Материалы 17-й Сарат. Зимней школы. -- Саратов: ООО «Издательство «Научная книга», 2014. -- с. 297-299.

\item  Шарапудинов И.И., Акниев Г.Г. Специальные дискретные преобразования посредством полиномов Чебышева второго рода, со свойством прилипания // ~Современные проблемы теории функций и их приложения: Материалы 17-й Сарат. Зимней школы. -- Саратов: ООО «Издательство «Научная книга», 2014. -- с. 299-302.
\end{enumerate}

\noindent \eject

\noindent \textbf{4. СВЯЗЬ С ВУЗОВСКОЙ НАУКОЙ}

\begin{enumerate}
\item  10 сотрудников Отдела математики и информатики преподают в ВУЗах республики Дагестан (ДГУ, ДГПУ, ДГТУ, ДГИНХ).

\item  Меджидов З.Г. ДГУ, кафедра Теории функций и функционального анализа. Читаемые курсы: 1) Прикладной функциональный анализ; 2) Обобщенные функции; 3) Дифференциальные уравнения; 4) Анализ и обработка изображений. Связь с органами народного образования: член жюри республиканской олимпиады по математике (Мин обр РД)

\item  Сиражудинов М.М. Профессор, зав. кафедрой Теории функций и функционального анализа ДГУ. Читает лекции по уравнениям в частных производных, теории функций комплексного переменного, дополнительным главам уравнений в частных производных. Тесно сотрудничает с кафедрой математического анализа Владимирского государственного университета, с кафедрой дифференциальных и интегральных уравнений мехмата МГУ. Председатель жюри республиканского этапа всероссийской математической олимпиады школьников.

\item  Абдурагимов Э.И. Работает на кафедре прикладной математики на 0.5 ставке доцента. Читает курсы: «Решение больших СЛАУ»  и «Численные методы решения уравнений матфизики» для магистров 1-го года обучения, «Численные методы решения прикладных граничных задач» для студентов 3-го курса специальности прикладная математика и информатика.

\item  Алишаев М.Г. читавшиеся в 2014 году курсы: Компьютерное моделирование в механике жидкостей и газов; Mathcad 14. руководство дипломными работами (количество и вуз): 2, ДГУ издание учебно-методических пособий: пока не издано, есть на  флешке «Лекции по методам компьютерного моделирования» (спецкурс, 20 лекций, 160 с).

\item  Кадиев Р.И. Дагестанский государственный университет, профессор кафедры прикладной математики, Численные методы уравнений математической физики, Спецкурс, Цифровая обработка информации.

\item  Магомедов А.М. ДГУ, кафедра дискретной математики и информатики. Читаемые курсы: 1) Математические основы компьютерной безопасности, 2) Языки программирования, 3) Дискретная математика, 4) Компьютерная графика, 5) Технологии мультимедиа.

\item  Рамазанов А-Р.К. Читаемые курсы: математический анализ и спецкурсы (факультет математики и компьютерных наук  Даггосуниверситета). Руководит одной дипломной работой и одной магистерской диссертацией в Даггосуниверситете.

\item
\end{enumerate}

\noindent \textbf{6. ГРАНТЫ И ПРОЕКТЫ}

\noindent

\noindent 1.  Сиражудинов М.М. Руководитель гранта РФФИ (\textbf{грант №12-01-96501-р\_юг\_а} «Математическое моделирование и разработка методов фазовой диагностической томографии среды», 2012-2014 гг.) (грант в ДГУ) Настоящий проект посвящен решению математической проблемы восстановления внутренней структуры объектов, исследованию реконструктивных возможностей фазовой (фазоконтрастной и рефрактивной) томографии как метода компьютерной томографии структуры объекта. В проекте предполагается решить следующие задачи математического моделирования: - Разработка и создание математических моделей описывающих структуру объекта,а также математических моделей, связующих параметры структуры с результатом измерений фазовых томографических проекций: - Изучение особенностей формирования фазовых проекций как проекций Радона (преобразований Радона) при различных методах не инвазивной диагностики неоднородной среды для получения ее объемного распределения; - Создание методов оптимального оценивания параметров внутренней структуры объекта, инвариантных относительно изменений неизвестных параметров модели измерения. допускающих интеллектуальный диалог с исследователем на основе подходов нечеткой математики; - Создание методов проверки адекватности построенных моделей и адекватности полученных оценок; - Проведение вычислительных экспериментов с использованием параллельных вычислений на Супер\_ЭВМ; - Выработка рекомендаций и предложений по результатам ММ и ВЭ для формирования основ фазовой томографии и функционирования режимов работы для различных схем диагностики. Разработка и создание математических методов для решения реконструктивных задач по фазовой томографической диагностике объектов будут использованы новые и эффективные алгоритмы на базе Фурье-, Хартли-, Вейвлет-преобразований.

\noindent 5.  Магомедов А.М. \textbf{Грант РФФИ 12-07-96500-р-юг\_а} «Непрерывный мониторинг Республики Дагестан по компьютерной карте: поиск, связь, навигация» (2012-2014). Планируется создание эффективной геоинформационной системы для Республики Дагестан. Проект включает решение комплекса взаимосвязанных фундаментальных проблем из области, граничной для информационных технологий, дискретной математики и математической кибернетики. Выполнение проекта начинается со сшивки компьютерной карты Республики Дагестан (РД) из фрагментов подробной отсканированной карты РД и создания компьютерной программы (с использованием средств графической библиотеки OPENGL) просмотра карты. Достичь высокую скорость просмотра карты значительных размеров предполагается имитацией вывода всей карты путем прорисовки в окне просмотра лишь фрагмента небольших размеров (512x512), актуального в текущий момент времени. Предусмотрены несколько способов указания в интерактивном режиме конкретного населенного пункта (н.п.): из локального списка названий н.п. текущего района, из глобального списка, вводом названия в редактируемое поле, визуальным указанием на карте. Будет решена программистская проблема автоматизации подключения к одной из глобальных компьютерных карт России с целью обеспечить просмотр окрестности данного н.п. в одном из пяти режимов детализации. Запланировано создание подробного трехмерного рельефа части районов РД; соответствующая работа для одного из муниципиальных районов (Гумбетовского) уже проводится. По любому выбранному в интерактивном режиме населенному пункту (н.п.) планируется вывод фотографий данного н.п. (до 10 фотографий для отдельного н.п.), краткой истории соответствующего района РД, списка населенных пунктов данного района, автомобильных маршрутов из столицы РД до районного центра,скоростной поиск и показ н.п. на карте с автоматическим выполнением прокрутки.

\noindent \textbf{Заявки на гранты, поданные в отчетном году}

\noindent 1.  Сиражудинов М.М. Подана одна заявка на грант HAAB.

\noindent 2.  Кадиев Р.И. Подана заявка на грант  РФНИ Название проекта "Уравнения математический физики и их приложения" - не поддержан.

\noindent 3.  Магомедов А.М. Заявка на конкурс РФФИ №15-37-20060\_мол\_а\_вед "Разработка алгоритмического и программного обеспечения семантической сегментации трехмерных изображений сцены".

\noindent 4.  Магомедов А.М. Заявка № 2014-14-579-0177-9276 «Исследование и разработка технологии унифицированного аппаратно-программного комплекса на отечественной платформе для построения, совершенствования и развития перспективных интегрированных информационно-телекоммуникационных систем» на участие конкурсе проектов на проведение прикладных научных исследований и экспериментальных разработок  по приоритетному направлению «Информационно-телекоммуникационные системы» в рамках реализации федеральной целевой программы «Исследования и разработки по приоритетным направлениям развития научно-технологического комплекса России на 2014 - 2020 годы», Мероприятие 1.3, 22 очередь.

\noindent 5.  Магомедов А.М. Заявка на проект №2014/33 «Интервальная раскраска графов и устойчивость решений» на выполнение государственных работ в сфере научной деятельности в рамках базовой части государственного задания Минобрнауки России (2014-2016).

\noindent


\end{document}

% == UNREGISTERED! == GrindEQ Word-to-LaTeX 2010 ==

