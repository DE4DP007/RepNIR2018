\documentclass[utf8,usehyperref,12pt]{G7-32}
\usepackage{amsthm,amsfonts,amsmath,amssymb,amscd}
\usepackage[T2A]{fontenc}
\usepackage[utf8]{inputenc}
\usepackage[english,russian]{babel}
\usepackage{float}
\usepackage{graphicx}
\usepackage{cmap}
\usepackage{color}
\usepackage[all,cmtip]{xy}
\graphicspath{{pictures/}}


\usepackage{cite}

\usepackage{dsfont}
\usepackage{mathrsfs}
%\newtheorem{theoremA}{Теорема}
%\renewcommand*{\thetheoremA}{\Alph{theoremA}}

\TableInChaper
\PicInChaper
\setlength\GostItemGap{2mm}


\NirOrgLongName{\textsc{ФЕДЕРАЛЬНОЕ АГЕНТСТВО НАУЧНЫХ ОРГАНИЗАЦИЙ\\
ФЕДЕРАЛЬНОЕ ГОСУДАРСТВЕННОЕ БЮДЖЕТНОЕ УЧРЕЖДЕНИЕ НАУКИ \\ ДАГЕСТАНСКИЙ НАУЧНЫЙ ЦЕНТР РОССИЙСКОЙ АКАДЕМИИ НАУК}}

\NirBoss{Врио председателя ДНЦ РАН}{Муртазаев А.К.} %% Заказчик, утверждающий НИР
\NirManager{Зав. Отделом математики и информатики ДНЦ РАН, доктор физ.-мат. наук}{Шарапудинов И.И.}

\NirYear{2018}
\NirTown{г. Махачкала,}




\NirUdk{УДК \No }
\NirGosNo{Регистрационный \No }

\NirStage{
}{итоговый отчет за 2017 г.}{
}

\bibliographystyle{unsrt}

\input commands.tex
\usepackage[shortlabels]{enumitem}
%%%%%%%<------------- НАЧАЛО ДОКУМЕНТА
\begin{document}
\usefont{T2A}{ftm}{m}{} %%% Использование шрифтов Т2 для возможности скопировать текст из PDF-файлов.

\frontmatter %%% <-- это выключает нумерацию ВСЕГО; здесь начинаются ненумерованные главы типа Исполнители, Обозначения и прочее

\NirTitle{%\textbf{<<Торсионные наногенераторы плазменных стволовых клеток с протонной накачкой>>}

%\begin{figure}[h]
%\center{\includegraphics[natwidth=2cm, natheight=6cm, width=2cm, height=6cm]{black.bmp}}
%\caption{Зависимость сигнала от шума для данных.}
%\label{ris:image}
%\end{figure}

\input annots/annot2.tex


} %%% Название НИР и генерация титульного листа


\Executors %% Список исполнителей здесь
%% это рисует линию размера 3мм и толщиной 0.1 пункт
\begin{longtable}{p{0.35\linewidth}p{0.2\linewidth}p{0.35\linewidth}}
\input executors/executors2.tex

Нормоконтролер, н.с. ОМИ,  &		&	\\
Султанахмедов М.С. & \rule{1\linewidth}{0.1pt}& \\
\end{longtable}

\input referats/referat2.tex

\setcounter{tocdepth}{2} %hide subsections

\tableofcontents

%\NormRefs % Нормативные ссылки
%\Defines % Необходимые определения


\Abbreviations %% Список обозначений и сокращений в тексте
\begin{abbreviation}
\item[ДНЦ] Дагестанский научный центр
\item[ОМИ] Отдел математики и информатики
\item[РАН] Российская академия наук
\end{abbreviation}


\input introductions/introduction2.tex

\mainmatter %% это включает нумерацию глав и секций в документе ниже


\input chapters/chapter2.tex




\backmatter %% Здесь заканчивается нумерованная часть документа и начинаются заключение и ссылки

\input conclusions/conclusion2.tex% заключение к отчёту

\input biblios/biblio2.tex

% \bibliography{biblio/filosofy} %% вместо вставки библиографии можно использовать базы BiBTeX - просто раскомментируйте эту строку.
\end{document}
