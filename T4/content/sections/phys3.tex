
=======================


\section{Магомедов М.А.}








\section{Аннотация}

Модель Поттса на треугольной решетке с учетом взаимодействия как первых, так и вторых ближайших соседей, исследована алгоритмом Ванга-Ландау метода Монте-Карло. Вычислена плотность состояний системы и рассчитаны температурные зависимости энтропии S. Показано, что в зависимости от соотношений обменных взаимодействий между первыми и вторыми бли-жайшими соседями, основное состояние может быть как сильно вырожденным, что свидетель-ствует о наличии фрустрации в системе, так и слабо вырожденным.
Определены структуры основного состояния и показано, что в данной системе реализуется Страйповое, триплетное или смешанное страйпово-триплетное состояние для фазы 1, неупорядо-ченное сильно вырожденное состояние или многослойное слабо вырожденное состояние для фрустрированной фазы и упорядоченное феррромагнитное состояние для ферромагнитной фазы.
Показано, что фазовый переход из ферромагнитной и страйповой-триплетной фаз в пара-магнитную является фазовым переходом первого рода, в то время как переход из фрустрирован-ной области в парамагнитную является переходом второго рода. Рассчитаны температурные зави-симости различных термодинамических параметров. Определены температуры фазовых перехо-дов и построена фазовая диаграмма системы.




\section{Введение}


В последние годы  значительное внимание уделяется экспериментальному и теоретиче-скому исследованию различных низкоразмерных, квази-одномерных или двумерных систем, в том числе наноструктур. Такие системы обладают рядом интересных свойств, перспективных в плане практического применения в различных электронных устройствах Изучение различных свойства этих объектов открывает широкие перспективы для экспериментальных приложений. В ближайшие годы нанотехнологии позволят совершить поистине гигантский технологический скачок в самых различных областях науки и техники [1].

Одними из таких материалов являются делафосситы, названные в честь французского кри-сталлографа XIX века Ж. Делафосса  (Delafosse) [2-3]. делафосситы имеют общую химическую формулу A+B3+X2, где А и В – катионы, Х – кислород. Одним из ярких представителей семейства делафосситов, является CuFeO2, структура которого приведена на рисунке 1.



Рис. 1. Структура делафоссита CuFeO2. Справа приведено схематическое и цветовое изоб-ражение трех возможных направлений ориентации спинов в трех-вершинной  модели Поттса.





Таким образом, CuFeO2  имеет явно выраженную плоскостную структуру, взаимодействием между слоями можно пренебречь.

Схематическое и цветовое изображение трех возможных направлений ориентации спинов  Fe  в материале CuFeO2 приведено на вставке рисунке 1. Таким образом, имеется три возможных направления спинов железа. Данная система хорошо описывается моделью Поттса с числом со-стояний q = 3.

Гамильтониан модели Поттса с числом состояний q = 3 может быть представлен в следую-щем виде [4]:


где J1 и J2 – параметры обменного взаимодействия для ближайших и вторых ближайших соседей. i,j, i,k – углы между взаимодействующими спинами Si - Sj и Si - Sk соответственно.
Численные расчеты, проведенные в работе [4], показали, что при учете только первых бли-жайших соседей с величиной J1<0, эта модель демонстрирует поведение характерное для ФП первого рода. При учете первых и вторых ближайших соседей с величинами J1<0 и J2<0 соответ-ственно в рассматриваемой модели возможны фрустрации. На данном этапе исследований нами проведены исследования модели Поттса при J1 = 0 и различных значениях J2.




Структура основного состояния и фазовая диаграмма трех-вершинной модели Потт-са на треугольной решетке.

Далее нами приводятся результаты исследования модели Поттса на треугольной решетке методом Ванга-Ландау [4-7].

На рисунке 2 приведена зависимость энергии основного состояния от величины второго обменного взаимодействия J2. Как наглядно видно из рисунка, в системе реализуются один из трех вариантов упорядочения спинов, энергия которых приведены разными цветами. В зависимо-сти от величины J2 приоритетным оказывается один из этих сценариев.



Рисунок 2. Зависимость энергии основного состояния от величины второго обменного вза-имодействия J2.


На рисунке 3 приведена зависимость плотности состояний, рассчитанная методом Ванга-Ландау, от величины второго обменного взаимодействия J2. Плотность состояний имеет куполо-образную форму с максимумом при нуле. При некоторых значениях J2 основное состояние вы-рождено.


Рисунок 3. Зависимость плотности состояний от величины второго обменного взаимодей-ствия J2.


Для определения температуры фазового перехода и его типа использовался метод произ-водной от плотности состояний [7]. Пример определения точки фазового перехода данным мето-дом приведен на рисунке 4.



Рисунок 4. Производная от плотности состояний при  J1 = 1 и  J2 = -1.5.

Зная плотность состояний системы можно рассчитать температурную зависимость любого интересующего нас термодинамического параметра. На рисунке 5 приведена температурная за-висимость энтропии системы, рассчитанная из плотности состояний, при различных значениях обменного взаимодействия. Как видно из рисунка, при высоких температурах энтропия стремит-ся к теоретическому значению ln3. С понижением температуры в зависимости от величины J2 энтро-пия может как обнуляться, так и стремиться к некоторому неннулевому значению.

Рисунок 5. Температурная зависимость энтропии системы при  J1 = 1 и различных J2.

В результате анализа основного состояния системы были определены структуры основного состояния, приведенные на рисунке 6.

Рисунок 6. Структуры основного состояния, реализуемые в системе при   J1 = 1 и различных значениях  J2.


Фазовая диаграмма системы приведена на рисунке 7. На рисунке также в фигурных скоб-ках приведены соответствующие данной фазе структуры с рисунка 6.

Рисунок 7. Фазовая диаграмма модели Поттса.

Как видно из рисунка, в зависимости от величин обменных взаимодействий, с понижением температуры системе возможны три сценария упорядочения:

•	Страйповое (рисунок 6.а), триплетное (рисунок 6.b) или смешанное страйпово-триплетное состояние (рисунок 6.c);

•	Фрустрированное неупорядоченное(рисунок 6.d) или многослойное состояние (ри-сунок 6.e, 6.f);

•	Упорядочееное ферромагнитное состояние (рисунок 6.g).





\section{Заключение}

Методом Ванга-Ландау исследована модель Поттса с числом состояний q=3 на треугольной решетке с учетом обменного взаимодействия между первыми и вторыми ближайшими сосе-дями. Вычислена плотность состояний системы и рассчитаны температурные зависимости энтропии S. Показано, что в зависимости от соотношений обменных взаимодействий между первыми и вторыми ближайшими соседями, основное состояние системы может быть сильно вырожденным.

Определены структуры основного состояния и показано, что в данной системе реализуется Страйповое, триплетное или смешанное страйпово-триплетное состояние для фазы 1, неупорядо-ченное сильно вырожденное состояние или многослойное слабо вырожденное состояние для фрустрированной фазы и упорядоченное феррромагнитное состояние для ферромагнитной фазы.

Показано, что фазовый переход из ферромагнитной и страйповой-триплетной фаз в пара-магнитную является фазовым переходом первого рода, в то время как переход из фрустрирован-ной области в парамагнитную является переходом второго рода. Определены температуры фазовых переходов и построена фазовая диаграмма системы.





