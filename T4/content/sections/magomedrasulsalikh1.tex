\section{Быстрое вычисление линейных комбинаций соболевских функций, порожденных функциями Хаара}

В данном подразделе рассмотрены некоторые свойства ортогональных по Соболеву функций $\chi_{1,n}(x)$, порожденных функциями Хаара. В частности, получены рекуррентные формулы для функций $\chi_{1,n}(x)$. Разработан алгоритм вычисления линейных комбинаций $N$ функций $\chi_{1,n}(x)$ за $O(\log N)$ операций.

\subsection{Введение}

Система Хаара состоит из ортогональных на отрезке $[0,1]$ кусочно постоянных функций $\chi_n(x)$, определяемых следующим образом \cite[глава 3, с. 70]{mmgmsr1-Kashin}:
\begin{equation}\label{mmgmsr1-haar-def}
\chi_1(x)=1, \quad
\chi_n(x)=\begin{cases} 0,&\text{если $x\notin \bar\Delta_n$,}\\
2^{k/2},& \text{если $x\in \Delta_n^+$,}\\
-2^{k/2},& \text{если $x\in \Delta_n^-$,}
\end{cases}
\end{equation}
где $n=2^k+i$, $1 \le i \le 2^k$, $\Delta_n=\Delta_k^i=(\frac{i-1}{2^k},\frac{i}{2^k})$ --- двоичный интервал, а $\bar\Delta_n$ --- замыкание двоичного интервала $\Delta_n$.

В работах Шарапудинова И.\,И. (см., например, \cite{mmgmsr1-Sha18, mmgmsr1-Shii-Shti-izvvuzov2017, mmgmsr1-Shii-matzam2017, mmgmsr1-ShaGadGad16, mmgmsr1-ShaGad16, mmgmsr1-Shii-lag-demi2015, mmgmsr1-SHII-MMG-Demi2015}) разработаны теория и методы построения систем функций $\Phi_r=\{\varphi_{r,n}\}_{n=0}^\infty$, $r \ge 1$, ортонормированных относительно скалярного произведения типа Соболева
\begin{equation}\label{mmgmsr1-inner-prod-gen}
\langle f,g \rangle =
\sum_{\nu=0}^{r-1}f^{(\nu)}(a)g^{(\nu)}(a)+\int_{a}^{b}f^{(r)}(x)g^{(r)}(x)\mu(x)dx.
\end{equation}
Для построения системы $\Phi_r$ выбирается одна из классических систем $\Phi =\{\varphi_n\}_{n=0}^\infty$, ортонормированных относительно обычного скалярного произведения
\begin{equation}\label{mmgmsr1-classic-mul}
\langle f,g \rangle =\int_{a}^{b}f(t)g(t)\mu(t)dt,
\end{equation}
и на её основе строятся функции $\varphi_{r,n}(x)$ посредством равенств
\begin{gather}
\label{mmgmsr1-phirk-1}
\varphi_{r,n}(x) =\frac{(x-a)^n}{n!}, \quad n=0,1,\ldots, r-1,\\
\label{mmgmsr1-phirk-2}
\varphi_{r,n}(x) =\frac{1}{(r-1)!}\int\limits_{a}^x(x-t)^{r-1}\varphi_{n-r}(t)dt, \quad n=r,r+1,\ldots.
\end{gather}
Система функций $\Phi_r=\{\varphi_{r,n}\}_{n=0}^\infty$, определённая формулами \eqref{mmgmsr1-phirk-1}, \eqref{mmgmsr1-phirk-2}, как было показано в упомянутых выше работах, будет ортонормирована относительно скалярного произведения \eqref{mmgmsr1-inner-prod-gen}.
Ряд Фурье функции $y(x)$ по системе функций $\Phi_r$ имеет следующий вид:
\begin{equation}\label{mmgmsr1-fourier-series}
y(x) \sim \sum_{n=0}^{r-1} y^{(n)}(a)\frac{(x-a)^n}{n!}+ \sum_{n=r}^\infty c_{r,n}(y) \varphi_{r,n}(x),
\end{equation}
где
\begin{equation}\label{mmgmsr1-crk}
c_{r,n}(y)=\int_{-1}^1 y^{(r)}(t)\varphi_{n-r}(t)\mu(t)dt.
\end{equation}

Ряды Фурье по функциям, ортогональным в смысле Соболева, могут быть эффективно использованы при решении задачи Коши для систем обыкновенных дифференциальных уравнений. В частности, в \cite{mmgmsr1-ShaOdeDemi2017,mmgmsr1-ShaMagOdeCos2017,mmgmsr1-SMS-SHTI-Demi2017,mmgmsr1-SHII-MSR-Demi2017} предложен основанный на применении рядов Фурье по системам $\Phi_1=\{\varphi_{1,n}\}$ итерационный метод решения задачи Коши для нелинейных систем обыкновенных дифференциальных уравнений первого порядка. Метод основан на представлении неизвестного решения $y(x)$ рассматриваемой задачи Коши в виде равномерно сходящегося ряда Фурье по системе вида $\Phi_1$:
\begin{equation*}
y(x) = y(a)+\sum\limits_{n=1}^{\infty}
c_{1,n}(y)\varphi_{1,n}(x),
\end{equation*}
и поиске посредством некоторых итерационных процедур первых $N$ неизвестных коэффициентов $c_{1,n}(y)$, $k=1,\ldots,N$. В качестве приближенного решения берется частичная сумма указанного ряда:
\begin{equation}\label{mmgmsr1-y-part-sum}
y(x) \approx y(a)+\sum\limits_{n=0}^{N-1}
c_{1,n+1}(y)\varphi_{1,n+1}(x).
\end{equation}
Упомянутый выше итерационный метод устроен так, что при его выполнении приходится неоднократно вычислять в узлах некоторой сетки линейные комбинации вида
%При этом для вычисления приближенных значений решения в конкретных точках требуется, как это видно из \eqref{mmgmsr1-y-part-sum}, вычислять линейные комбинации вида
\begin{equation}\label{mmgmsr1-linear-comb}
\sum\limits_{n=0}^{N-1}\alpha_n\varphi_{1,n+1}(x),
\end{equation}
где $\alpha_n$ --- известные вещественные числа. В данном подразделе предложен быстрый алгоритм вычисления линейных комбинаций \eqref{mmgmsr1-linear-comb} для соболевских функций $\varphi_{1,n}(x)=\chi_{1,n}(x)$, порожденных функциями Хаара \eqref{mmgmsr1-haar-def} по формулам \eqref{mmgmsr1-phirk-1}, \eqref{mmgmsr1-phirk-2}, в которых $\varphi_n(x) = \chi_{n+1}(x)$, $n \ge 0$.

\subsection{Некоторые свойства функций $\chi_{1,n}$}

В статье \cite{mmgmsr1-SharapudinovMuratova} получен явный вид функций $\chi_{1,n}(x)$, $n=2^k+i$, $1 \le i \le 2^k$, $k \ge 0$:
\begin{equation}\label{mmgmsr1-chi-1k}
\chi_{1,1}(x)=x, \quad
\chi_{1,n}(x)=2^{\frac{k}{2}}
\begin{cases} 0,&\text{$0\le x\le\frac{i-1}{2^k}$;}\\
x-\frac{i-1}{2^k},&\text{$\frac{i-1}{2^k}\le x\le \frac{2i-1}{2^{k+1}}$;}\\
\frac{i}{2^{k}}-x,&\text{$\frac{2i-1}{2^{k+1}}\le x\le \frac{i}{2^{k}}$;}\\
0, &\text{$\frac{i}{2^{k}}\le x\le1$}.
\end{cases}
\end{equation}
При фиксированном $k \ge 0$ совокупность функций $\chi_{1,n}$ с номерами $n=2^k+i$, $1 \le i \le 2^k$, называется $k$-й пачкой ($\chi_{1,1}$ не попадает ни в какую пачку). Чтобы подчеркнуть принадлежность функции $\chi_{1,n}$ к той или иной пачке, мы будем использовать обозначение $\chi_{1,k,i}(x)=\chi_{1,n}(x)$, $n=2^k+i$.

Заметим, что для заданной точки $x \in [0,1]$ в $k$-й пачке найдется не более одной функции, отличной от нуля в точке $x$. Именно этот факт позволит нам значительно ускорить процесс вычисления линейных комбинаций вида \eqref{mmgmsr1-linear-comb}. Более того, если $x$ является двоично-рациональной, то, начиная с некоторой пачки, все функции $\chi_{1,n}$ будут равны нулю в точке $x$. Остановимся более подробно на этом вопросе.

Ниже мы будем считать функции $\chi_{1,n}(x)$, $n \ge 2$, продолженными по непрерывности на всю ось, т. е. мы полагаем $\chi_{1,n}(x)=0$, $x \notin [0,1]$, $n \ge 2$. В дальнейшем особую роль будет играть функция $\chi_{1,2}(x)$, поэтому мы выпишем тут ее явную формулу для наглядности:
\begin{equation}\label{mmgmsr1-chi12}
\chi_{1,2}(x)=
\begin{cases}
x, &x \in [0,\frac{1}{2}],\\
1-x, &x \in [\frac{1}{2},1],\\
0, &x \notin [0,1].
\end{cases}
\end{equation}

\begin{statement}
	Для функций $\chi_{1,n}(x)$ имеют место следующие соотношения:
	\begin{align}
	\label{mmgmsr1-chi-k-m}
	&\chi_{1,k+m,i}(x)=2^{-\frac{m}{2}}\chi_{1,k,i}(2^m x), \quad 1 \le i \le 2^k, \; k \ge 0, \; m \ge 0;\\
	\label{mmgmsr1-chi-i-1}
	&\chi_{1,k,i}(x)=\chi_{1,k,1}\bigl(x - \frac{i-1}{2^k}\bigr), \quad 1 \le i \le 2^k, \; k \ge 0;\\
	\label{mmgmsr1-chi-ki-12}
	&\chi_{1,k,i}(x)=2^{-\frac{k}{2}}\chi_{1,2}(2^kx-i+1), \quad 1 \le i \le 2^k, \; k \ge 0.
	\end{align}
\end{statement}
\begin{proof}
	Первые две формулы получаются непосредственно из определения \eqref{mmgmsr1-chi-1k}, а последняя формула легко выводится с помощью первых двух:
	\begin{equation*}
	\chi_{1,k,i}(x)=\chi_{1,k,1}(x - \frac{i-1}{2^k})=2^{-\frac{k}{2}}\chi_{1,0,1}(2^kx-i+1).
	\end{equation*}
\end{proof}

\begin{statement}\label{mmgmsr1-st-zero-for-bin}
	Для любой двоично-рациональной точки $x$ все функции последовательности $\chi_{1,n}$, начиная с некоторого номера, будут равны нулю в этой точке. Более точно, если $x=\frac{j}{2^m}$, $0 \le j \le 2^m$, то
	\begin{equation*}
	\chi_{1,k,i}(x)=0, \quad k \ge m, \; 1 \le i \le 2^k.
	\end{equation*}
\end{statement}
\begin{proof}
	Из соотношения \eqref{mmgmsr1-chi-ki-12} вытекает, что
	\begin{equation*}
	\chi_{1,k,i}(\frac{j}{2^m})=2^{-\frac{k}{2}}\chi_{1,2}(2^{k-m}j-i+1),
	\end{equation*}
	причем аргумент $2^{k-m}j-i+1$ при условиях утверждения всегда является целым числом. Остается заметить, что, как это видно из \eqref{mmgmsr1-chi12}, функция $\chi_{1,2}$ обращается в ноль в любой целочисленной точке.
\end{proof}

\begin{statement}\label{mmgmsr1-st-one-in-pack}
	В каждой пачке для любой фиксированной точки $x \in [0,1]$ отличной от нуля в этой точке может быть лишь функция с номером $\nu(k,x)=[2^kx]+1$, где $k$ --- номер пачки, а $[t]$ --- целая часть $t$, т. е.
	\begin{equation}\label{mmgmsr1-chi-1ki-zero}
	\chi_{1,k,i}(x)=0, \quad i \ne \nu(k,x), \; k \ge 0.
	\end{equation}
	Кроме того,
	\begin{equation}\label{mmgmsr1-chi-1knu}
	\chi_{1,k,\nu(k,x)}(x)=2^{-\frac{k}{2}}\chi_{1,2}(2^kx-[2^kx]).
	\end{equation}
	%причем $\chi_{1,k,\nu(k,x)}(x) \ne 0$ для любого $k \ge 0$, если $x$ является двоично-иррациональным, и $\chi_{1,k,\nu(k,x)}(x) \ne 0$
	%вида $\frac{j}{2^m}$, то
\end{statement}
\begin{proof}
	Поскольку функция $\chi_{1,2}(x)$ отлична от нуля только при $0 < x < 1$, то из формулы \eqref{mmgmsr1-chi-ki-12} следует, что при заданных $k$ и $x$ неравенство $\chi_{1,k,i}(x) \ne 0$ имеет место только для тех $i$, которые удовлетворяют условию $0 < 2^kx-i+1 < 1$. Следовательно, для $i$ имеем  $2^kx < i < 2^kx+1$. Так как $i$ --- целое, то последнее соотношение может быть выполнено только при $i=[2^kx]+1$. Отсюда следует справедливость соотношения \eqref{mmgmsr1-chi-1ki-zero}. Оставшаяся часть утверждения получается непосредственной подстановкой выражения для $\nu(k,x)$ в формулу \eqref{mmgmsr1-chi-ki-12}.
\end{proof}

\subsection{Математические основы алгоритма быстрого вычисления линейных комбинаций функций $\chi_{1,n}$}

Пусть для заданных $x\in[0,1]$ и $\alpha_n$, $0 \le n \le N-1$, требуется вычислить линейную комбинацию
\begin{equation}\label{mmgmsr1-linear-comb-chi}
S_{1,N}(x)=S_{1,N}(x; \{\alpha_n\})=
\sum\limits_{n=0}^{N-1}
\alpha_n \chi_{1,n+1}(x).
\end{equation}
Нетрудно подсчитать, что для вычисления $S_{1,N}(x)$ по данной формуле понадобится $O(N)$ операций.
%1) $cN$ операций для вычисления значений функций $\chi_{1,n}(x)$, $1 \le n \le N$, которые включают в себя операции по определению $k$ и $i$ по заданному $n$, по проверке условий из \eqref{mmgmsr1-chi-1k}, по вычислению значений $2^{k/2}$ и др.;
%
%2) $N$ умножений и $N-1$ сложений для подсчета самой суммы.

Для оптимизации процесса вычисления линейной комбинации $S_{1,N}(x)$, $N=2^K+I$, $1 \le I \le 2^K$, представим ее в следующем виде:
\begin{equation*}
S_{1,N}(x)=\alpha_0\chi_{1,1}(x) +
\sum\limits_{k=0}^{K-1}
\sum\limits_{n=2^k+1}^{2^{k+1}}\alpha_{n-1}\chi_{1,n}(x)+
\sum\limits_{n=2^K+1}^{N}\alpha_{n-1}\chi_{1,n}(x).
\end{equation*}
Отсюда, используя утверждение \eqref{mmgmsr1-st-one-in-pack}, выводим:
\begin{equation}\label{mmgmsr1-s-fast-algorithm}
S_{1,N}(x)=
\alpha_0x+
\sum\limits_{k=0}^{\tilde{K}}\alpha_{2^k+\nu(k,x)-1}2^{-\frac{k}{2}}\chi_{1,2}(2^kx-[2^kx]),
\tilde{K}=
\begin{cases}
K, &\nu(K,x) \le I,\\
K-1, &\nu(K,x) > I.
\end{cases}
\end{equation}
Для вычисления $S_{1,N}(x)$ по этой формуле требуется $O(K)=O(\log N)$ операций, что на порядок быстрее, чем при использовании формулы \eqref{mmgmsr1-linear-comb-chi}.

\subsection{Компьютерная реализация}
%
%\lstset{language=[Sharp]C, basicstyle=\ttfamily\small, stringstyle=\ttfamily, tabsize=2,
%	showstringspaces=false,
%	breaklines=true,
%	breakatwhitespace=false,
%	mathescape=true,
%	texcl = true}
%\lstMakeShortInline[columns=fixed]!
%
В данном пункте приводится описание численного алгоритма вычисления линейной комбинации $S_{1,N}(x)$,
основанного на использовании формулы \eqref{mmgmsr1-s-fast-algorithm} и реализованного в методе \textit{FastCalc} класса \textit{SobolevHaarLinearCombination} %(см. листинг~\ref{mmgmsr1-lst-SobolevHaar}).

На вход метод \textit{FastCalc} принимает массив коэффициентов \textit{alpha} длины $N$ и точку $x \in [0,1]$.
Основные вычисления производятся в цикле по $k$, который соответствует сумме в формуле \eqref{mmgmsr1-s-fast-algorithm}. На каждой итерации требуется вычислять величины $2^{-k/2}$, $2^kx$,  $\nu(k, x) = [2^kx] + 1$ и $n = 2^k + \nu(k, x)$. В целях оптимизации вычисления указанных величин вводятся дополнительные переменные \textit{pow2k}, \textit{pow2kx}, \textit{pow2k2}, которые на $k$-й итерации содержат значения $2^k$, $2^kx$, $2^{-k/2}$ соответственно. Эти переменные инициализируются значениями \textit{1}, \textit{x}, \textit{1} и на каждой итерации обновляются согласно формулам: $2^{k+1} = 2^k * 2$, $2^{k+1}x = 2^kx * 2$, и $2^{-\frac{k+1}{2}} = \frac{2^{-\frac{k}{2}}}{\sqrt{2}}$. Затем введенные переменные используются для вычисления очередного $k$-го члена суммы из формулы \eqref{mmgmsr1-s-fast-algorithm}.

Заметим, что выполнение цикла досрочно завершается на $k$-ой итерации, если число $x$ является двоично-рациональным вида $\frac{j}{2^k}$, $j = 0, 1,\ldots,2^k$, поскольку согласно утверждению \ref{mmgmsr1-st-zero-for-bin} функции $\chi_{1, l, i}(\frac{j}{2^k}) = 0$ для всех $l \geq k$ и $1 \leq i \leq 2^l$. В коде условие досрочного прерывания цикла имеет вид \textit{pow2kx+1==nu}, где \textit{nu=(int)pow2kx+1}, так как выполнение этого условия означает равенство $2^kx=[2^kx]$, откуда и следует, что $x$ двоично-рациональное число вида $\frac{j}{2^k}$.

%\begin{listing}[H]
%	\caption{Класс \textsl{SobolevHaarLinearCombination}}
%	\label{mmgmsr1-lst-SobolevHaar}
%	\begin{minted}[gobble=2 ,tabsize=4, mathescape=true]{csharp}
%	class SobolevHaarLinearCombination
%	{
%	private static double sqrt2 = Math.Sqrt(2);
%	
%	//метод для вычисления линейной комбинации $S_{1,N}(x)$
%	public static double FastCalc(double[] alpha, double x)
%	{
%	int N = alpha.Length;
%	// инициализируем значения для нулевой пачки
%	int pow2k = 1; //$2^k$
%	double pow2kx = x; //$2^kx$
%	double pow2k2 = 1; //$2^{-\frac{k}{2}}$
%	int nu = (int)x + 1; //$\nu(k, x) = [2^kx] + 1$
%	
%	//$n = 2^k + \nu(k, x), 1 \leq \nu(k, x) \leq 2^k$,
%	//указывает на номер ненулевой функции в каждой пачке
%	int n = 1 + nu;
%	
%	double result = alpha[0] * x;
%	
%	//вместо нахождения $\tilde{K}$ мы просто продолжаем цикл,
%	//пока $n = 2^k + \nu(k, x)$ не превосходит $N$
%	for (int k = 0; n <= N; k++)
%	{
%	//проверка, является ли $x$ двоично-рациональным
%	//числом вида $x = \frac{j}{2^k}, j = 1,\ldots,2^k$
%	if (pow2kx + 1 == nu)
%	break;
%	
%	//нахождение дробной части $2^kx$
%	double xd = pow2kx - (int)pow2kx;
%	//вычисление k-го элемента суммы
%	result += alpha[n - 1]
%	* pow2k2
%	* (xd <= 0.5 ? xd : 1 - xd);
%	
%	//обновление значений переменных
%	//для следующей k+1-й пачки
%	pow2k *= 2;
%	pow2kx *= 2;
%	pow2k2 /= sqrt2;
%	nu = (int)pow2kx + 1;
%	n = pow2k + nu;
%	}
%	
%	return result;
%	}
%	}
%	\end{minted}
%\end{listing}


%\subsection{Численный эксперимент}
%Проведем сравнительный анализ скорости вычисления суммы $S_{1, N}(x)$ методом !FastCalc! и методом !Calc!, вычисляющим линейную комбинацию по формуле \eqref{mmgmsr1-linear-comb-chi}. Методика сравнения состоит в следующем. Для заданного $N$ проводится $E$ экспериментов. В каждом $i$-м эксперименте переменной !x! и массиву !alpha! задаются случайные значения, для которых производится вычисление $S_{1,N}(x)$ указанными методами $K$ раз с фиксацией времени выполнения $t_{i, Calc}$, $t_{i, FastCalc}$ всех $K$ итераций для каждого метода и получением среднего значения времени выполнения $\delta_{i, Calc} = \frac{t_{i, Calc}}{K}$ и $\delta_{i, FastCalc} = \frac{t_{i, FastCalc}}{K}$ в рамках $i$-го эксперимента. Полученные во всех экспериментах значения усредняются: $\Delta_{Calc} = \frac{\sum_{i = 1}^{E}\delta_{i, Calc}}{E}$, $\Delta_{FastCalc} = \frac{\sum_{i = 1}^{E}\delta_{i, FastCalc}}{E}$.
%Результаты экспериментов для $E = 100$ и $K = 100$ приведены в таблице \ref{mmgmsr1-tbl-timetest}.
%
%\begin{table}[h]
%	\begin{tabularx}{\textwidth}{|l|X|X|}
%		\hline
%		$N$ & $\Delta_{Calc}$, с. & $\Delta_{FastCalc}$, с. \\ \hline
%		10 &  $3,505 * 10^{-6}$ & $9,876 * 10^{-8}$ \\ \hline
%		100 &  $3,4557 * 10^{-5}$ & $1,2082 * 10^{-7}$ \\ \hline
%		1000 &  $3,2373 * 10^{-4}$ & $1,7115 * 10^{-7}$ \\ \hline
%		10000 &  $3,4374 * 10^{-3}$ & $2,1418 * 10^{-7}$ \\ \hline
%		100000 &  $3,5052 * 10^{-2}$ & $2,6795 * 10^{-7}$ \\ \hline
%		%1000000 & $3,4906 * 10^{-1}$ & $3,3438 * 10^{-6}$ \\ \hline
%	\end{tabularx}
%	\caption{Сравнение скорости работы $\Delta_{Calc}$ и $\Delta_{FastCalc}$ при разных значениях $N$, где $E = 100, K = 100$}
%	\label{mmgmsr1-tbl-timetest}
%\end{table}
%
%Заметим, что если $x$ является двоично-рациональным числом вида $\frac{j}{2^m}$, то, начиная с $N=2^m$, время выполнения метода !FastCalc! перестает расти вместе с $N$, что продемонстрировано в таблице \ref{mmgmsr1-tbl-bin-rac}, в которой приведены результаты, полученные с помощью описанной выше методики, но при $E=1$ и некотором фиксированном $x$, заданным вручную.
%
%\begin{table}[h]
%	\begin{tabularx}{\textwidth}{|l|X|}
%		\hline
%		$N$ & $\Delta_{FastCalc}$ в точке $x = \frac{1}{2^{10}}$, с. \\ \hline
%		10 &  $7,4221 * 10^{-8}$ \\ \hline
%		100 &  $1,1189 * 10^{-7}$ \\ \hline
%		1000 &  $1,4978 * 10^{-7}$ \\ \hline
%		10000 &  $1,5148 * 10^{-7}$ \\ \hline
%		100000 &  $1,5155 * 10^{-7}$ \\ \hline
%		1000000 & $1,503 * 10^{-7}$ \\ \hline
%	\end{tabularx}
%	\caption{Время выполнения метода FastCalc при $x = \frac{1}{2^{10}}, K = 10^7$}
%	\label{mmgmsr1-tbl-bin-rac}
%\end{table}
%
%Приведенные выше компьютерные эксперименты выполнены на персональном компьютере с CPU Intel Core i5 6200U, 2.8ГГц, вычисления производились в однопоточном режиме.

%\subsection{Заключение}
%
%Разработан алгоритм вычисления линейных комбинаций $N$ функций $\chi_{1,n}(x)$ за $O(\log N)$ операций.
%Рассмотрены некоторые свойства ортогональных по Соболеву функций $\chi_{1,n}(x)$, порожденных функциями Хаара. 