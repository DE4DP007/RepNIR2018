\section{Быстрый алгоритм решения задачи Коши для ОДУ с помощью ортогональных по Соболеву полиномов, порожденных полиномами Чебышева первого рода}



\subsection{Введение}

Рассмотрим задачу Коши
%Исследуется задача Коши вида
\begin{equation}\label{sms-stn-2-CauchyProbl}
\begin{aligned}
&y'=f(x,y),\ x\in[0,h],\\
&y(0)=y_0,
\end{aligned}
\end{equation}
где $f(x,y)$ полагаем непрерывной в некоторой области $G$, причем на $[0,h]\times R\subset G$ функция $f(x,y)$ удовлетворяет условию Липшица по второй переменной
\begin{equation}\label{sms-stn-2-LipSecVar}
|f(x,q')-f(x,q'')|\le\lambda|q'-q''|,\ x\in[0,h].
\end{equation}
%В настоящей работе описывается быстрый алгоритм реализации разработанного в \cite{sms-stn-2-PolOrtPorSobChebUrav} итерационного метода на основе ортогональных по Соболеву полиномов, порожденных полиномами Чебышева первого рода. 

В данном подразделе рассмотрена задача о численной реализации итерационного процесса для решения задачи Коши для ОДУ \eqref{sms-stn-2-CauchyProbl} с использованием ортогональных по Соболеву полиномов, порожденных полиномами Чебышева первого рода $T_0=1/\sqrt{2}$, $T_k(x)=\cos k\arccos x$ ($k\ge1$). Составлен алгоритм, реализующий указанный итерационный процесс с применением быстрых косинус-преобразова\-ний. Разработана программа и проведен ряд численных экспериментов, которые показывают, что ряды Фурье по порожденным полиномам являются удобным инструментом для решения дифференциальных уравнений.

\subsection{Итерационный метод}

%Идея построения приближенного решения для задачи \eqref{sms-stn-2-CauchyProbl} заключается в следующем.
Положим $x=h(t+1)/2$, $t\in[-1,1]$ и задачу \eqref{sms-stn-2-CauchyProbl} сведем к следующей
\begin{equation}\label{sms-stn-2-CauchyProblEta}
\begin{aligned}
&\eta'=\frac h2f\left(\frac h2(t+1),\eta\right),\ t\in[-1,1],\\
&\eta(-1)=y_0,
\end{aligned}
\end{equation}
где $\eta(t)= y(h(t+1)/2)$. Легко заметить, что $\eta\in W^1_{L^2_\mu(-1,1)}$.

%Следуя \cite{sms-stn-2-PolOrtPorSobChebUrav}, введем в рассмотрение гильбертово пространство $l_2$, состоящее из последовательностей $C=(c_0,c_1,\ldots,)$ с нормой $\|C\|=\left(\sum_{j=0}^\infty c_j^2\right)^\frac12$ и

Нам понадобятся некоторые соотношения. Разложение функции $\eta(t)\in W_{L_\mu^2}^1$ в ряд Фурье по системе $\{T_{1,k}\}$
\begin{equation}\label{sms-stn-2-SolutionSeries}
\eta(t)=\eta(-1)+\sum\limits_{k=1}^{\infty}\hat\eta_{1,k} T_{1,k}(t),\text{ где } \hat\eta_{1,k}=\int\limits_{-1}^1\eta'(t)T_{k-1}(t)\mu(t)dt.
\end{equation}
Ряды Фурье по системе $\{T_k\}$ функций $\eta'(t)\in L^2_\mu$ и $\phi(t)\in L^2_\mu$ соответственно имеют вид 
\begin{equation}\label{sms-stn-2-DerivativeChebSeries}
\eta'(t)=\sum\limits_{k=0}^{\infty}c_k(\eta) T_k(t),
\end{equation}
\begin{equation}\label{sms-stn-2-DerivativeChebSeries2}
\phi(t)=f\left(\frac h2(t+1),\eta(t)\right)=\sum\limits_{k=0}^{\infty}\hat\phi_k T_k(t),
\end{equation}
где
\begin{equation}\label{sms-stn-2-ForierCoeffs}
c_k(\eta)=\hat\eta_{1,k+1},\quad\hat\phi_k=\int\limits_{-1}^1\phi(t)T_k(t)\mu(t)dt.
\end{equation}
Здесь ряд \eqref{sms-stn-2-SolutionSeries} сходится равномерно на $[-1,1]$ (см. \cite{sms-stn-2-SharIzVuz}), а ряды \eqref{sms-stn-2-DerivativeChebSeries} и \eqref{sms-stn-2-DerivativeChebSeries2} -- в метрике $L^2_\mu$. В силу \eqref{sms-stn-2-CauchyProblEta}, \eqref{sms-stn-2-SolutionSeries} и \eqref{sms-stn-2-ForierCoeffs} коэффициенты $c_k$ принимают следующий вид
\begin{equation}\label{sms-stn-2-operatorA}
c_k(\eta)=\frac h2\hat\phi_k=\frac h2\int\limits_{-1}^1f\left[\frac h2(t+1),\eta(-1)+\sum\limits_{l=0}^{\infty}c_l T_{1,l+1}(t)\right]T_k(t)\mu(t)dt.
\end{equation}
Выражение в правой части \eqref{sms-stn-2-operatorA} представляет собой оператор $A$, действующий в гильбертовом пространстве $l_2$, которое состоит из вещественнозначных последовательностей $C=(c_0,c_1,\ldots)$, для которых конечна величина $\|C\|=\bigl(\sum_{j=0}^\infty c_j^2)^{1/2}$, задающая норму в $l_2$. Из \eqref{sms-stn-2-operatorA} следует, что $C(\eta) = (c_0(\eta),c_1(\eta),\dots)$ является неподвижной точкой для оператора $A$. В \cite{sms-stn-2-PolOrtPorSobChebUrav} было показано, что оператор $A$ является сжимающим в метрике пространства $l_2$ при условии
\begin{equation}\label{sms-stn-2-ContractionCond}
h\kappa\lambda<1,
\end{equation}
где
\begin{equation*}
\kappa=\left(\int\limits_{-1}^1\sum_{k=1}^{\infty}(T_{1,k}(x))^2\mu(x)dx\right)^{1/2},
\end{equation*}
а $\lambda$ берется из условия \eqref{sms-stn-2-LipSecVar}. Таким образом, выбором величины $h$ можно добиться выполнения условия \eqref{sms-stn-2-ContractionCond}.
Для практических приложений важно рассмотреть конечномерный аналог $A$ -- оператор $A_N : \mathbb{R}^N\rightarrow\mathbb{R}^N$, ставящий в соответствие точке $C_N = (c_0,\ldots,c_{N-1})$ точку $C_N' = (c_0',\ldots,c_{N-1}')$ по правилу
\begin{equation}\label{sms-stn-2-operatorA_N}
c_k'=\frac h2\int\limits_{-1}^1f\bigl[\frac h2(t+1),\eta(-1)+\sum\limits_{l=0}^{N-1}c_l T_{1,l+1}(t)\bigr]T_k(t)\mu(t)dt,\ k=1,\ldots,N-1.
\end{equation}
Этот оператор также является сжимающим \cite{sms-stn-2-PolOrtPorSobChebUrav} и, следовательно, итерационный процесс $C_N^{j+1}=A_N(C_N^j)$ сходится к точке $C_N(\eta)=(c_0(\eta),\dots,c_{N-1}(\eta))$. В ходе выполнения НИР был разработан быстрый алгоритм приближенного вычисления коэффициентов $c_k(\eta)$, $k=0,\ldots,N-1$. Путем подстановки этих коэффициентов в частичную сумму ряда \eqref{sms-stn-2-SolutionSeries}
\begin{equation}\label{sms-stn-2-FurSobPartialSum}
S_{N}(t)=\eta(-1)+\sum\limits_{k=0}^{N-1}c_k(\eta) T_{1,k+1}(t)
\end{equation}
получаем приближенное решение задачи \eqref{sms-stn-2-CauchyProblEta}. В предыдущем подразделе для вычисления суммы из правой части \eqref{sms-stn-2-FurSobPartialSum} предложен и реализован в виде компьютерной программы быстрый алгоритм. В  дальнейшем нам потребуется преобразовать выражение \eqref{sms-stn-2-operatorA_N} посредством замены $t=\cos\tau$ к следующему
\begin{equation*}
c_k'=  \frac h\pi\int\limits_0^{\pi}f\left[\frac h2(\cos\tau+1),\eta(-1)+\sum\limits_{l=0}^{N-1}c_l T_{1,l+1}(\cos\tau)\right]T_k(\cos\tau)d\tau=
\end{equation*}
\begin{equation}\label{sms-stn-2-trigOperatorA_N}
\frac h{2\pi}\int\limits_0^{2\pi}f\left[\frac h2(\cos\tau+1),\eta(-1)+\sum\limits_{l=0}^{N-1}c_l T_{1,l+1}(\cos\tau)\right]T_k(\cos\tau)d\tau.
\end{equation}
Для вычисления интегралов \eqref{sms-stn-2-trigOperatorA_N} мы будем использовать следующую квадратурную формулу
\begin{equation}\label{sms-stn-2-quadratureA_N}
c_k'\approx  \frac h\pi\sum\limits_{j=0}^{N}f\left[\frac h2(\cos\tau_j+1),\eta(-1)+\sum\limits_{l=0}^{N-1}c_l T_{1,l+1}(\cos\tau_j)\right]T_k(\cos\tau_j)d\tau,
\end{equation}
где $\tau_j=\frac{2\pi j}{N+1}$, $j=0,\ldots,N$. Отметим, что в \eqref{sms-stn-2-quadratureA_N} частичные суммы
\begin{equation}\label{sms-stn-2-PartialSolSum}
\eta_j=S_N(\cos\tau_j)=\eta(-1)+\sum\limits_{l=0}^{N-1}c_l T_{1,l+1}(\cos\tau_j),\ j=0,\ldots,N
\end{equation}
не зависят от $k$, что позволяет ограничиться однократным вычислением $\eta_j$ и существенно сэкономить время вычисления $c_k$. Отметим, что в предыдущем подразделе быстрое вычисление сумм $\sum_{l=0}^{N-2}c_l T_{1,l+1}(\cos\theta_l)$, $\theta_l=\frac{(2j+1)\pi}{2N}$, производилось путем сведения их к сумме
\begin{equation}\label{sms-stn-2-DCT-II}
Q_N(\theta)=\sum_{l=0}^{N-1}d_l T_{l}(\cos\theta_l)
\end{equation}
и ее вычислением с помощью быстрого дискретного косинусного преобразования DCT-II.
%Для вычисления $\eta_j$ мы воспользуемся разработанным в \cite{sms-stn-2-demiSMS_ShTN}  быстрым алгоритмом вычисления сумм
%\begin{equation}\label{sms-stn-2-PartialSolSum}
%  s_{N-1}(\cos\tau_j)=\sum\limits_{l=0}^{N-1}c_l T_{1,l+1}(\cos\tau_j),\ j=0,\ldots,N,
%\end{equation}
%основанном на представлении $s_N(t)$  через полиномы  $T_k(t)$
%\begin{equation}\label{sms-stn-2-ChebPartialSum}
%  s_{N-1}(t)=\sum_{k=0}^{N}d_kT_k(t)
%\end{equation}

\subsection{Описание алгоритма}
В настоящем пункте будет описан быстрый алгоритм решения задачи Коши \eqref{sms-stn-2-CauchyProblEta}. Входными данными являются
{
	\renewcommand{\labelitemi}{$\circ$}
	\begin{itemize}
		\item[-] $y_0$ -- начальное значение задачи Коши в точке 0;
		\item[-] $f(x,y)$ -- правая часть задачи Коши \eqref{sms-stn-2-CauchyProbl};
		\item[-] $N$  -- порядок частичной суммы \eqref{sms-stn-2-FurSobPartialSum};
		
		\item[-] $C^0=(c^0_0,\ldots,c^0_{N-2})$ -- начальное приближение;
		\item[-] $\varepsilon$ -- порог близости векторов $C^{i-1},C^i\in\mathbb{R}^{N-1}$ ($C^i=(c^i_0,\ldots,c^i_{N-2})$) в метрике $R^{N-1}$, где $C^i=A_{N-1}(C^{i-1})$, $i=0,1,\dots$ при достижении которого останавливается итерационный процесс;
		\item[-] $M$ -- количество итераций.
	\end{itemize}
}
Алгоритм состоит из следующих шагов:
\begin{enumerate}
	\item[1.] На вход подаются начальное условие $y_0$, правая часть $f(x,y)$ и порядок частичной суммы $N$, равный степени двух.
	\item[2.] Задается начальное приближение $C^0=(c_0^0,c_1^0,\ldots,c_{N-2}^0)$.
	\item[3.] Вычисляются и сохраняются в массив числа $x_j=h(\cos\frac{2\pi j}N+1)$, $j=0,1,\ldots,N-1$.
	\item[4.] Осуществляется переход от коэффициентов $c_l^i$ к $d_m^i$, $l=0,\ldots,N-2$, $m=0,\ldots,N-1$ (см. \cite{sms-stn-2-demiSMS_ShTN}, п. 2).
	\item[5.] Вычисляются $\eta_j$, $j=0,\ldots,N-1$, с использованием быстрого обратного косинусного преобразования массива $d_m^i$.
	\item[6.] Создается массив $F_j=hf(x_j,\eta_j)$, $j=0,\ldots,N-1$.
	\item[7.] С применением быстрого прямого косинусного преобразования к массиву $F_j$, вычисляются новые коэффициенты $c_k^{i+1}$.
	\item[8.] Повторение шагов 4 -- 8 пока не будет достигнут порог точности $\varepsilon$ или же проведено требуемое количество итераций, задаваемое числом $M$.
	%Если не достигнут порог точности $\varepsilon$ или число проведенных итераций меньше требуемого -- $M$, то переход к шагу 4 с новыми коэффициентами $c_k^{i+1}$.
\end{enumerate}

% TODO Нужно заключение.