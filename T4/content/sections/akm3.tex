\chapter{Задачи и тесты для дистанционной олимпиады по программированию среди вузов СКФО}
Аннотация 


Сотрудники ОМИ ДНЦ РАН и кафедры Дискретной математики ДГУ провели 1.11.2018 межрегиональную дистанционную олимпиаду по программированию среди вузов СКФО. Выполнена трудоемкая работа по разработке заданий, тестов и программного обеспечения для олимпиады.


\section{Формулировки олимпиадных заданий}
1. Ошибка компиляции – 10 баллов.\\
Компиляторы ряда языков (например, PascalABC.NET, C\#) диагностируют ошибку пере-полнения при вычислении произведения
1000000000*10*0,
при этом переполнение не имеет места при вычислении выражения 10000000000*10*0.
\\
Чем это можно объяснить? Текст объяснений наберите в текстовом редакторе Блокнот и сохраните в файле out1.txt.
\par\smallskip
2. Знакомое число --- 10 баллов. Время счета --- 3 мин.\\
Пусть n - целое положительное число и вычислены n случайных точек (x,y), такие, что 0<=x<1, 0<=y<1. Обозначим через m количество тех точек, которые попали в единичный круг (круг с центром в точке (0;0) и с радиусом 1). При достаточно большом n значение 4*m/n приближенно равно некоторому хорошо известному в математике числу, которое обозначим через P.
Программа должна найти значение 4*m/n, содержащее пять верных знаков числа P, и вы-вести n и соответствующее значение 4*m/n, а также время счета: а) в консольное окно; б) в файл out2.txt в следующем виде:\\
n= …         P=…         t=…    sec.
\par\smallskip
3. Треугольник --- 20 баллов. Время счета --- 1 сек.\\
Четное положительное число a и целое положительное число h расположены в первой строке файла in3.txt и разделены пробелом Для равнобедренного треугольника с вершинами в точках (-a/2;0), (a/2;0), (0;h) найти количество точек n с целочисленными коорди-натами, расположенных внутри треугольника (но не на его сторонах). Значения a, h, n и время счета вывести а) в консольное окно; б) в файл out3.txt в следующем виде:\\
a = … h= … n= … t= … sec.\\
Например, если a и h равны соответственно 8 и 5, искомое число n равно 16.\\
Ограничения: 1<a<=1000, 1<h<=1000.
\par\smallskip
4. Бинарный код --- 20 баллов. Время счета --- 5 сек.\\
Требуется сгенерировать последовательность всех подмножеств n-элементного множества так, что каждое следующее подмножество получается из предыдущего удалением или до-бавлением одного элемента.

Каждое подмножество задается последовательностью длины n, состоящей из нулей и единиц: если i-й элемент исходного множества включается в данное подмножество, то i-я по-зиция последовательности занята единицей, в противном случае – нулем.
Значение n задано в файле in4.txt, n<10. Вывод осуществляется в файл out4.txt. Например, при n=3 вывод может быть таким:\\
000\\
100\\
110\\
010\\
011\\
111\\
101\\
001\\
\par\smallskip
5. Разложение --- 40 баллов. Время счета --- 10 сек.\\
Найти число разложений F(N) заданного натурального N на натуральные слагаемые и вывести в консольное окно и в файл out5.txt. Значение N задано в файле in5.txt, 0<N<501.

Примечание. Само число N также считается разложением.\\
Пример вывода: F(4)=5; вот сами разложения (их вывод не требуется):\\
1+1+1+1\\
1+1+2\\
1+3\\
2+2\\
4\\
\section{Пример решения}
Приведем решение задачи 5.

Обозначим через $r(n,m)$ мощность множества $R(n,m)$ разложений $n$ на $m$ натуральных слагаемых.

Тогда $F(N) = r(N,1)+…+r(N,N)$.
Рассмотрим способ вычисления $r(n,m)$.

Если $n=m$, то разложение единственно и состоит из $n$ единиц, следовательно, $r(n,m)=1$;

если $m=1$, то разложение также единственно и состоит из $n$, т.е. $r(n,m)=1$;

если $n<m$, то разложение не существует, т.е. $r(n,m)=0$.

В противном случае, т.е. при $n>m>1$, то
$r(n,m)= r(n-m, m) + r(n-1, m-1)$. Почему?
Разобьем множество $R(n,m)$ на два:
$R'(n,m)$ -- множество разложений $n$ на $m$ слагаемых, превышающих 1;
$R''(n,m)$ -- множество разложений $n$ на $m$ слагаемых, хотя бы одно из которых равно 1.

Легко видеть, что $|R'(n,m)|= |R(n-m,m)|$, $|R''(n,m)|=|R(n-1,m-1)|$. Таким образом, $r(n,m)= r(n-m, m) + r(n-1, m-1)$.

Заполним верхний треугольник матрицы $r[1..N, 1..M]$ нулевыми значениями, диагональные элементы и элементы первого столбца --- единицами; затем\\
для каждого $n=3,…,N$\\
для каждого $m=2,…, n-1$ выполним:\\
\{\\
если $n-m<m$, то $a:=0$, иначе $a:=r(n-m,m)$;\\
$r(n,m):= a + r(n-1, m-1)$;\\
\}\\
\section{Заключение}
В олимпиаде приняли участие 34 команды вузов СКФО. Результаты позволяют утверждать, что степень трудности заданий была продумана методически грамотно: так, команда, занявшая 1-е место, набрала 75 баллов из 100 возможных, а команда, занявшая 2-е место --- 63 балла (из 100 возможных). 