\Referat %Реферат отчёта, не более 1 страницы

Отчет содержит X с., X~источников, X~таблицы.

 \bigskip
 \textbf{ Ключевые
  слова:}
  ТЕОРИЯ ПРИБЛИЖЕНИЙ; ОРТОГОНАЛЬНЫЕ ПОЛИНОМЫ; ПОЛИНОМЫ ЧЕБЫШЕВА; УЛЬТРАСФЕРИЧЕСКИЕ ПОЛИНОМЫ ЯКОБИ; ПРЕДЕЛЬНЫЕ И СПЕЦИАЛЬНЫЕ РЯДЫ; ДИСКРЕТНЫЕ СПЕЦИАЛЬНЫЕ РЯДЫ; ФУНКЦИЯ ЛЕБЕГА; НЕРАВНОМЕРНЫЕ СЕТКИ; СРЕДНИЕ ВАЛЛЕ ПУССЕНА; ТЕОРИЯ РАСПИСАНИЙ; ТРЕХМЕРНОЕ МОДЕЛИРОВАНИЕ; ГИПОТЕЗА ГРЮНБАУМА; ПАКЕТЫ ПРОГРАММ.

 \bigskip

%Настоящий отчёт содержит итоги работы за 2018 год Отдела математики и информатики ДНЦ РАН по теме
%<<Разработка алгоритмов и создание наукоемкого программного обеспечения для моделирования сложных систем. Некоторые вопросы цифровой обработки сигналов и изображений. Исследования по теории графов и теории оптимизации расписаний, компьютерное сопровождение вузовского учебного процесса>>
%%осуществлению фундаментальных научных исследований в соответствии с
%из Программы фундаментальных научных исследований государственных академий наук на 2013–2020 годы.
%

%%%%%%%%%%%%%%%%%%%%%%%%%%
%%%%%%%%%%%%%%%%%%%%%%%%%%
%%%%%%%%%%%%%%%%%%%%%%%%%%
















%Основные результаты, полученные в отчетном году по данной теме касаются разработки алгоритмов и программных пакетов для использования в решении важных прикладных задач, таких как обработка и сжатие временных рядов и изображений, численно-аналитическое решение систем линейных и нелинейных дифференциальных и разностных уравнений, интервальная реберная раскрашиваемость двудольных графов и оптимизация расписаний для мультипроцессорных систем,  программное обеспечение для компьютерного сопровождения процесса распределения учебной нагрузки.
%
%
%В частности, разработаны алгоритмы и компьютерные программы для обработки временных рядов и изображений методом перекрывающих преобразований, основанных на повторных средних типа Валле Пуссена для тригонометрических сумм Фурье, для численно-аналитического решения систем линейных и нелинейных дифференциальных и разностных уравнений спектральными методами, основанными на использовании систем функций, ортогональных по Соболеву и порожденных такими классическими системами как система Хаара, система полиномов Чебышева первого рода и система косинусов.
%





%%%%%%%%%%%%%%%%%%%%%%%%%%%%%%%
%%%%%%%%%%%%%%%%%%%%%%%%%%%%%%%
%%%%%%%%%%%%%%%%%%%%%%%%%%%%%%%

===============

В исследовании проблемы существования двудольных графов заданного порядка, не допускающих интервальной реберной раскраски, востребован алгоритм перечисления графов по принципу «один представитель из каждого класса изоморфных графов». Ввиду алгоритмических трудностей задачи это условие обычно заменяется более слабым условием: «малое число представителей из каждого класса изоморфных графов». В 2018 г. найдены некоторые способы усиления фильтрации изоморфных графов.


Все известные формулы для перечисления разбиений прямоугольника используют действия с плавающей запятой. С помощью авторского программного обеспечения для ряда случаев уточнены точные значения для количества разбиений с применением ранее полученных автором формул, использующих лишь операции над целыми числами.


Начатый ранее процесс создания программного обеспечения по компьютерному сопровождению деятельности вузовской кафедры продолжен в направлении построения сводных таблиц учебных нагрузок вузовской кафедры и автоматической генерации тестовых заданий по структурированным данным языков программирования.



Продолжена работа по созданию методической и алгоритмической базы проведения межрегиональных олимпиад по программированию среди вузов. На ее основе проведена дистанционная межрегиональная олимпиада по программированию среди вузов СКФО.


================


%Рамазанов М.К.

Получены все возможные магнитные структуры основного состояния в зависимости от соотношений обменных взаимодействий $r$ для антиферромагнитной модели Изинга на объемно-центрированной кубической решетке с учетом взаимодействий первых и вторых ближайших соседей. Показано, что в зависимости от величины $r$ в системе возможно $6$ различных упорядочений в основном состоянии. Построена фазовая диаграмма зависимости критической температуры от величины взаимодействия следующих ближайших соседей. Впервые на диаграмме обнаружена узкая область $(2/3<r \leq 3/4)$, где переход из антиферромагнитной фазы в парамагнитную является переходом первого рода. Обнаружено, что при значении $r=2/3$ конкуренция обменных взаимодействий не приводит к возникновению фрустрации и вырождению основного состояния. Показано, что в исследуемой модели при $r = 2/3$ наблюдается фазовый переход второго рода.


================

% Бабаев






















%Задача об интервальной реберной раскрашиваемости двудольных графов порядка $15$, $16$, $17$ и $18$, сформулированная в 1999 г. К.Giaro, является открытой проблемой дискретной математики, востребованной в вопросах оптимизации расписаний для мультипроцессорных систем.
%
%
%С привлечением компьютерных ресурсов и методов теории графов в 2017 г. было доказано, что все двудольные графы порядка $15$ интервально раскрашиваемы; интервально раскрашиваемы также и все двудольные графы $G=(X,Y,E)$ порядка $16$, за исключением, быть может, случаев $|X|=7$ и $|X|=8$.
%
%
%Создано программное обеспечение по проблеме интервальной раскраски, а также для сопровождения следующих аспектов вузовского учебного процесса: контроль посещаемости, верификация согласованности индивидуальных планов преподавателей кафедры, отображение графических структур.


	


%%%%%%%%%%%%%%%%%%%%%%%%%%%%%%%
%%%%%%%%%%%%%%%%%%%%%%%%%%%%%%%
%%%%%%%%%%%%%%%%%%%%%%%%%%%%%%%

%
%
%
%Задача об интервальной реберной раскрашиваемости двудольных графов порядка 15, 16, 17 и 18, сформулированная в 1999 г.
%К.Giaro, является открытой проблемой дискретной математики, востребованной в вопросах
%оптимизации расписаний для мультипроцессорных систем.
%%С привлечением компьютерных ресурсов и методов теории графов в 2017 г. с доказано, что все двудольные графы порядка 15 интервально раскрашиваемы; интервально раскрашиваемы также и все двудольные графы $G=(X,Y,E)$ порядка 16 при $min(|X|, |Y|)<7$.
%Для проверки существования интервальной реберной раскраски связного двудольного графа и ее построения в случае существования в отчетном году был сконструирован жадный алгоритм, который осуществляет перебор с возвратами последовательных ребер кусочно-непрерывного пути; последний представляет собой такой упорядоченный набор всех ребер графа, что подграф, порожденный любым подмножеством его ребер с номерами $1, 2, \ldots,$ является связным.
%	
%Построен алгоритм, позволяющий для множества $S$ всех двудольных графов заданного порядка $H$ сгенерировать множество $S_0$ двудольных графов существенно меньшей мощности, содержащее для каждого графа $G$ из $S$ граф, изоморфный  $G$. Применением жадного алгоритма к	каждому графу из $S_0$ с привлечением компьютерных ресурсов показано, что при малых значениях $H$ все графы из $S_0$ интервально раскрашиваемы.
%В частности, установлено, что все двудольные графы $G=(X,Y,E)$ порядка 16 при $|X|<7$ обладают интервальной раскраской; отсюда следует, что достаточно проверить интервальную раскрашиваемость двудольных графов порядка 16 при $|X|=7$ и $X=8$. Программное обеспечение, разработанное на основе упомянутых алгоритмов, обладает также рядом дополнительных свойств, способствующих графическому	отображению интервальной раскраски в удобной для восприятия форме.
%


%Задача об интервальной реберной раскрашиваемости двудольных графов порядка 15, 16, 17 и 18, сформулированная в 1999 г.
%К.Giaro, является открытой проблемой дискретной математики, востребованной в вопросах
%оптимизации расписаний для мультипроцессорных систем.
%%С привлечением компьютерных ресурсов и методов теории графов в 2017 г. с доказано, что все двудольные графы порядка 15 интервально раскрашиваемы; интервально раскрашиваемы также и все двудольные графы $G=(X,Y,E)$ порядка 16 при $min(|X|, |Y|)<7$.
%Разработан эффективный для практических целей новый алгоритм проверки интервальной раскрашиваемости графа, на его основе
%создано программное обеспечение, обладающее рядом возможностей по выбору удобного представления графа и по отображению
%его интервальной раскраски.  Разработан алгоритм малоизбыточного перечисления всех
%двудольных графов заданного порядка (не более 20) путем отсеивания значительного количества изоморфных графов, алгоритм
%воплощен в программное обеспечение.



%%%%%%%%%%%%%%%%%%%%%%%%%
%
%Разработан алгоритм для решения задачи автоматизации создания тестовых заданий по учебной дисциплине <<Язык программирования Delphi 7.0>>, создано программное обеспечение, отправлена статья в редакцию журнала из списка
%ВАК. Выполненная в данном направлении под руководством г.н.с. ОМИ ДНЦ Магомедова А.М. работа студента 2к ФМиКН ДГУ Гаджимирзаева Ш.М. представлена на Всероссийскую научную конференцию молодежи.
%
%Создано программное обеспечение для генерации сводных таблиц распределения учебной нагрузки вузовской кафедры, проверки
%согласованности индивидуальных планов преподавателей кафедры, а также автоматизации подготовки исходных данных к
%расписанию. Выполненная в данном направлении работа  под руководством г.н.с. ОМИ ДНЦ Магомедова А.М. магистранта ФМиКН ДГУ Ибрагимовой З.И. представлена на конкурс
%<<Умник>> (в настоящее время известно, что работа вышла в финал).
%
%Изданы в виде учебных пособий нестандартные упражнения по 3\foreignlanguage{english}{ds} \foreignlanguage{english}{Max} и олимпиадные задания по программированию с решениями. В направлении 3\foreignlanguage{english}{ds}
%\foreignlanguage{english}{Max} команда студентов ФМиКН ДГУ под руководством г.н.с. ОМИ ДНЦ Магомедова А.М. приняла участие во Всероссийском конкурсе, получены два
%диплома 1 степени --- в личном (Гаджиева М.) и командном зачетах. Под руководством Магомедова А.М. прошла региональная
%дистанционная олимпиада студентов вузов СКФО по программированию.
%
%Написан программный пакет для отображения графических структур, найдены способы экономного задания графов, соответствующая
%статья издана в журнале из списка ВАК, программа для ЭВМ зарегистрирована в реестре Роспатента.
%
%





















%%%%%%%%%%%%%%%%%%%%
%%%%%%%%%%%%%%%%%%%% ПРОШЛЫЙ ГОД
%%%%%%%%%%%%%%%%%%%%




 %Разработан численный алгоритм эффективного вычисления повторных средних Валле Пуссена. На основе данного алгоритма составлена компьютерная программа и проведены численные эксперименты. Данный алгоритм может быть использован в алгоритме сжатия изображений.
%
%Найдены достаточные условия существования беспростойных расписаний и указаны подходы к построению расписаний минимальной длительности. Исследованы теоретико-графовые модели расписаний, для решения задач интервальной реберной раскраски разработано алгоритмическое и программное обеспечение.
%
%  Разработаны алгоритмы построения однопроцессорного расписания с условиями частичного предшествования и мультипроцессорного расписания без простоев и без условий частичного предшествования. Основные результаты опубликованы в \cite{akm-DEMI},
%	 \cite{akm-artIRJ}, \cite{akm-artDMA}.
%
%Приведенное в разделе \ref{akm1} описание зарегистрированного в 2016 г. в Госпатенте программного обеспечения посвящено вопросам оптимизации расписания, сформулированным в терминологии теории графов.
%
%Получена явная формула обращения лучевого преобразования симметричного тензорного поля второго ранга, когда это преобразование задано на семействе прямых, направления которых определяется точками кусочно-гладкой кривой. Сама кривая удовлетворяет условию полноты, аналогично условию Кириллова -- Туя. В случае, когда веерное преобразование тензорного поля известно на семействе лучей, выходящих из точек незамкнутой кривой (не обязательно удовлетворяющей условию Кириллова -- Туя) соленоидальная часть искомого поля вычислена в точках прямолинейного отрезка, соединяющего концы кривой.
%
% Получены явные формулы вычисления значений финитных, скалярных и тензорных полей первого и второго ранга по заданным интегралам с весами от этих полей вдоль лучей и ломанных. В случае скалярных полей все лучи имеют одно направление, а весовая функция – квазимногочлен. В случае векторных полей для обращения веерного преобразования достаточно семейство лучей, имеющих два фиксированных направления, а в случае тензорных полей второго ранга – три направления. Полученные формулы обобщают известные в литературе формулы, используемые в линейных задачах интегральной геометрии с возмущениями.
%
%
%Разработан алгоритм решения задачи Коши на отрезке $[0,1]$ для обыкновенных дифференциальных уравнений с помощью рядов Фурье по функциям, ортогональным по Соболеву и порожденным классическими ортогональными функциями Хаара. На основе данного алгоритма составлена и зарегистрирована программа для ЭВМ MixedHaarDeqSolver (свидетельство № 2016617831 от 14.07.2016 г.). Компьютерные эксперименты, проведенные с помощью этой программы, показали высокую эффективность данного алгоритма для численного решения задачи Коши.
%
%Предложен новый метод приближенного решения задачи Коши для разностного уравнения, основанный на разложении искомого решения в конечные ряды Фурье по полиномам, ортогональным относительно дискретного скалярного произведения типа Соболева и ассоциированным с полиномами Чебышева. Для указанных полиномов получено явное представление через классические полиномы Чебышева. Рассмотрены некоторые разностные свойства частичных сумм Фурье по этим полиномам.
%
%Предложен алгоритм решения задачи Коши на полуоси для обыкновенного дифференциального уравнения посредством рядов Фурье по полиномам, ортогональным в смысле Соболева, ассоциированным с классическими полиномами Лагерра. Составлена программа, реализующая данный алгоритм, и проведены численные эксперименты.
%
%Разработаны два пакета прикладных программ для обработки временных рядов и дискретных функций: с помощью конечных предельных рядов и их усреднений типа Валле Пуссена, а также с помощью специальных вейвлет-рядов на основе полиномов Чебышева второго рода, обладающих так называемым свойством <<прилипания>>. Проведены эксперименты на дискретных функциях различной природы.

