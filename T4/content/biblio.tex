\begin{thebibliography}{1} %% здесь библиографический список



%LappedVallePussen
\bibitem{LapVPMalvar}
Malvar H.S. Signal processing with lapped transform. Boston{ $\cdot$} London. Artech House. 1992

\bibitem{LapVPZhuk} Жук В.В. Аппроксимация периодических функций. Ленинград. 1982.

\bibitem{LapVPNIK} Никольский С.М. О некоторых методах приближения тригонометрическими суммами // Изв. АН СССР. Сер. матем. 1940. Т. 4, № 6. С. 509--520


\bibitem{LapVPEFIM} Ефимов А.В. О приближении периодических функций суммами Валле-Пуссена // Изв. АН СССР. Сер. матем. 1959. Т. 23, № 5. С. 737--770.

\bibitem{LapVPTEl} Теляковский С.А. О приближении дифференцируемых функций линейными средними их рядов Фурье // Изв. АН СССР. Сер. матем. 1960. Т. 24, № 2. С. 213--242.

\bibitem{LapVPmmg} Магомед-Касумов М.Г. Аппроксимативные свойства классических средних Валле-Пуссена для кусочно гладких функций // Вестник Дагестанского научного центра РАН. 2014. Вып. 54. C. 5--12.

\bibitem{LapVPShar7} Шарапудинов И.И. Аппроксимативные свойства средних Валле-Пуссена на классах типа Соболева с переменным показателем // Вестник Дагестанского научного центра РАН. 2012. Вып. 45 С. 5--13.

\bibitem{LapVPShar8} Шарапудинов И.И. Приближение гладких функций в $L_{2\pi}^{p(x)}$ средними Валле-Пуссена // Известия Саратовского университета. Серия: Математика. Механика. Информатика. 2012. Т. 13, вып. 1, часть 1. С. 45--49.

\bibitem{LapVPShar6} Шарапудинов И.И. Приближение функций в $L^{p(x)}_{2\pi}$ тригонометрическими полиномами // Известия РАН: Серия математическая. 2013. Т. 77, № 2. С. 197--224.

%%%%%%%%%%%%%%%%%%%%%%%%%%%%%%%%%%



\bibitem{equ102-Shar2016}
{Шарапудинов И.И.}
Системы функций, ортогональные по Соболеву, порожденные ортогональными функциями // Материалы 18-й международной Саратовской зимней школы «Современные проблемы теории функций и их приложения». Саратов. ООО «Издательство «Научная книга». 2016. Стр. 329--332.

\bibitem{equ102-Shar20}
Шарапудинов И.И. Ортогональные  по Соболеву системы, порожденные ортогональными функциями // Изв. РАН. Сер. Математическая. 2018. Том. 82. (Принята к печати)

\bibitem{equ102-IserKoch}
{Iserles A., Koch P.E., Norsett S.P. Sanz-Serna J.M.}
On polynomials  orthogonal  with respect  to certain Sobolev inner products // J. Approx. Theory. 1991. Vol. 65. Pp. 151--175.

\bibitem{equ102-MarcelAlfaroRezola}
{Marcellan F., Alfaro M., Rezola M.L.}
Orthogonal polynomials on Sobolev spaces: old and new directions // Journal of Computational and Applied Mathematics. North-Holland. 1993. Vol. 48. Pp. 113--131.

\bibitem{equ102-Meijer}
{Meijer H.G.}
Laguerre polynimials generalized to a certain discrete Sobolev inner product space // J. Approx. Theory. 1993. Vol. 73. Pp. 1--16.

\bibitem{equ102-KwonLittl1}
{Kwon K.H., Littlejohn L.L.}
The orthogonality of the Laguerre polynomials $\{L_n^{(-k)}(x)\}$ for positive integers $k$ // Ann. Numer. Anal. 1995. Vol. 2. Pp. 289–-303.

\bibitem{equ102-Lopez1995}
{Lopez G. Marcellan F. Vanassche W.}
Relative Asymptotics for Polynomials Orthogonal with Respect to a Discrete Sobolev Inner-Product // Constr. Approx. 1995. Vol. 11:1. Pp. 107–-137.

\bibitem{equ102-KwonLittl2}
{Kwon K.H., Littlejohn L.L.}
Sobolev orthogonal polynomials and second-order differential equations // Rocky Mountain J. Math. 1998. Vol. 28. Pp. 547–-594.

\bibitem{equ102-MarcelXu}
{Marcellan F., Yuan Xu}
On Sobolev orthogonal polynomials // Expositiones Mathematicae. 2015. Vol. 33. Pp. 308--352.

\bibitem{equ102-Shar17}
{Шарапудинов И.И.}
Смешанные ряды по ультрасферическим полиномам и их аппроксимативные свойства // Математический сборник. 2003. Т. 194. Вып. 3. Стр. 115--148.

\bibitem{equ102-Shar13}
{Шарапудинов И.И.}
Смешанные ряды по ортогональным полиномам // Издательство Дагестанского научного центра. Махачкала. 2004. Стр. 1--176.

\bibitem{equ102-Tref1}
{Trefethen  L.N.}
Spectral methods in Matlab. Fhiladelphia. SIAM. 2000.

\bibitem{equ102-Tref2}
{Trefethen L.N.}
Finite difference and spectral methods for ordinary and partial differential equation. Cornell University. 1996.

\bibitem{equ102-SolDmEg}
{Солодовников В.В., Дмитриев А.Н., Егупов Н.Д.}
Спектральные методы расчета и проектирования систем управления. Москва. Машиностроение. 1986.

\bibitem{equ102-Pash}
{Пашковский С.}
Вычислительные применения многочленов и рядов Чебышева. Москва. Наука. 1983.

\bibitem{equ102-Arush2014}
{Арушанян О.Б., Волченскова Н.И., Залеткин С.Ф.}
Применение рядов Чебышева для интегрирования обыкновенных дифференциальных уравнений // Сиб. электрон. матем. изв. 2014. Вып. 11. Стр. 517--531.

\bibitem{equ102-Lukom2016}
{Лукомский Д.С., Терехин П.А.}
Применение системы Хаара к численному решению задачи Коши для линейного дифференциального уравнения первого порядка // Материалы 18-й международной Саратовской зимней школы «Современные проблемы теории функций и их приложения». Саратов. ООО «Издательство «Научная книга». 2016. Стр. 171--173.

\bibitem{equ102-MMG2016}
{Магомед-Касумов М.Г.}
Приближенное решение обыкновенных дифференциальных уравнений с использованием смешанных рядов по системе Хаара // Материалы 18-й международной Саратовской зимней школы «Современные проблемы теории функций и их приложения». Саратов. ООО «Издательство «Научная книга». 2016. Стр. 176--178.

\bibitem{equ102-DiffUr2017}
{Шарапудинов И.И., Магомед-Касумов М.Г.}
О представлении решения задачи Коши  рядом Фурье  по полиномам, ортогональным по  Соболеву, порожденным многочленами Лагерра. Дифференциальные уравнения. 2017 (принята к печати)

\bibitem{equ102-Shar18}
{Шарапудинов И.И., Шарапудинов Т.И.}
Смешанные ряды по полиномам Якоби и Чебышева и их дискретизация // Математические заметки. 2010. Т. 88. Вып. 1. Стр. 116--147.

\bibitem{equ102-KashSaak}
{Кашин Б.С., Саакян А.А.}
Ортогональные ряды. Москва. АФЦ 1999.

\bibitem{equ102-Shar19}
{Шарапудинов И.И., Муратова Г.Н.}
Некоторые свойства r-кратно интегрированных рядов по системе Хаара // Изв. Сарат. ун-та. Нов. сер. Сер. Математика. Механика. Информатика. 2009. Т. 9. Вып. 1. Стр. 68 -- 76

\bibitem{equ102-Faber}
{G. Faber}
Ober die Orthogonalfunktionen des Herrn Haar // Jahresber. Deutsch. Math. Verein. 1910. Vol. 19. Pp. 104--112.

\bibitem{equ102-Shar25}
{Шарапудинов И.И.}
Асимптотические свойства полиномов, ортогональных по Соболеву, порожденных полиномами Якоби // Дагестанские электронные математические известия. 2016. Вып. 6.	Стр. 1–-24.

\bibitem{equ102-Sege}
{Сеге Г.}
Ортогональные многочлены. Москва. Физматгиз. 1962.
%%
%% end lit. section2-equ102

\bibitem{equ130-Shar20}
{Шарапудинов И.И.}
Ортогональные по Соболеву системы, порожденные ортогональными функциями // Изв. РАН. Сер. Математическая. 2018. Т. 82. (принята к печати)





\bibitem{equ130-Shar2016}
{Шарапудинов И.И.}
Системы функций, ортогональные по Соболеву, порожденные ортогональными функциями // Материалы 18-й международной Саратовской зимней школы «Современные проблемы теории функций и их приложения». ООО «Издательство «Научная книга». Саратов. 2016. С. 329--332.




\bibitem{equ130-IserKoch}
{Iserles A., Koch P.E., Norsett S.P., Sanz-Serna J.M.}
On polynomials  orthogonal  with respect  to certain Sobolev inner products // J. Approx. Theory. 1991. Vol. 65. Pp. 151--175.

\bibitem{equ130-MarcelAlfaroRezola}
{Marcellan F., Alfaro M., Rezola M.L.}
Orthogonal polynomials on Sobolev spaces: old and new directions // Journal of Computational and Applied Mathematics. North-Holland. 1993. Vol. 48. Pp. 113--131.

\bibitem{equ130-Meijer}
{Meijer H.G.}
Laguerre polynimials generalized to a certain discrete Sobolev inner product space // J. Approx. Theory. 1993. Vol. 73. Pp. 1--16.


\bibitem{equ130-KwonLittl1}
{Kwon K.H., Littlejohn L.L.}
The orthogonality of the Laguerre polynomials $\{L_n^{(-k)}(x)\}$ for positive integers $k$ // Ann. Numer. Anal. 1995. Issue 2. Pp. 289--303.

\bibitem{equ130-Lopez1995}
{Lopez G. Marcellan F. Vanassche W.}
Relative Asymptotics for Polynomials Orthogonal with Respect to a Discrete Sobolev Inner-Product // Constr. Approx. 1995. Vol. 11:1. Pp. 107–-137.



\bibitem{equ130-KwonLittl2}
{Kwon K.H., Littlejohn L.L.}
Sobolev orthogonal polynomials and second-order differential equations // Rocky Mountain J. Math. 1998. Vol. 28. Pp. 547--594.


\bibitem{equ130-MarcelXu}
{Marcellan F., Yuan Xu.}
On Sobolev orthogonal polynomials // Expositiones Mathematicae. 2015. Vol. 33. Issue 3. Pp. 308--352.

\bibitem{equ130-Shar17}
{Шарапудинов И.И.}
Смешанные ряды по ультрасферическим полиномам и их аппроксимативные свойства // Математический сборник. 2003. Т. 194, вып. 3. С. 115--148.


\bibitem{equ130-Shar13}
{Шарапудинов И.И.}
Смешанные ряды по ортогональным полиномам. Издательство Дагестанского научного центра. Махачкала. 2004. С. 1--176.

\bibitem{equ130-Tref1}
{Trefethen L.N.}
Spectral methods in Matlab. SIAM. Philadelphia. 2000.

\bibitem{equ130-Tref2}
{Trefethen L.N.}
Finite difference and spectral methods for ordinary and partial differential equation. Cornell University. 1996.


\bibitem{equ130-SolDmEg}
{Солодовников В.В., Дмитриев А.Н., Егупов Н.Д.}
Спектральные методы расчета и проектирования систем управления. Москва. Машиностроение. Москва. 1986.

\bibitem{equ130-Pash}
{Пашковский С.}
Вычислительные применения многочленов и рядов Чебышева. Москва. Наука. 1983.

\bibitem{equ130-Arush2014}
{Арушанян О.Б., Волченскова Н.И., Залеткин С.Ф.}
Применение рядов Чебышева для интегрирования обыкновенных дифференциальных уравнений // Сиб. электрон. матем. изв. 2014. Вып. 11. С. 517--531.

\bibitem{equ130-Lukom2016}
{Лукомский Д.С., Терехин П.А.}
Применение системы Хаара к численному решению задачи Коши для линейного дифференциального уравнения первого порядка // Материалы 18-й международной Саратовской зимней школы «Современные проблемы теории функций и их приложения».  Саратов. ООО «Издательство «Научная книга». 2016. С. 171--173.


\bibitem{equ130-MMG2016}
{Магомед-Касумов М.Г.}
Приближенное решение обыкновенных дифференциальных уравнений с использованием смешанных рядов по системе Хаара // Материалы 18-й международной Саратовской зимней школы «Современные проблемы теории функций и их приложения». Саратов. ООО «Издательство «Научная книга». 2016. С. 176--178.

\bibitem{equ130-DiffUr2017}
{Шарапудинов И.И., Магомед-Касумов М.Г.}
О представлении решения задачи Коши  рядом Фурье  по полиномам, ортогональным по  Соболеву, порожденным многочленами Лагерра // Дифференциальные уравнения. 2017. (принята к печати)







\bibitem{equ130-Shar18}
{Шарапудинов И.И., Шарапудинов Т.И.}
Смешанные ряды по полиномам Якоби и Чебышева и их дискретизация // Математические заметки. 2010. Т. 88, вып. 1, С. 116--147.



\bibitem{equ130-KashSaak}
{Кашин Б.С., Саакян А.А.}
Ортогональные ряды. Москва. АФЦ. 1999.




\bibitem{equ130-Shar19}
{Шарапудинов И.И., Муратова Г.Н.}
Некоторые свойства r-кратно интегрированных рядов по системе Хаара // Изв. Сарат. ун-та. Нов. сер. Сер. Математика. Механика. Информатика. 2009. Т. 9, Вып. 1. С. 68--76.




\bibitem{equ130-Faber}
{Faber G.}
Ober die Orthogonalfunktionen des Herrn Haar // Jahresber. Deutsch. Math. Verein. 1910. Vol. 19. Pp. 104--112.

\bibitem{equ130-Shar25}
{Шарапудинов И.И.}
Асимптотические свойства полиномов, ортогональных по Соболеву, порожденных полиномами Якоби // Дагестанские электронные математические известия. 2016. Вып. 6. С. 1–-24

\bibitem{equ130-Sege}
Сеге Г. Ортогональные многочлены. Москва. Физматгиз. 1962.
%%
%% end lit. section2-equ130





%%%%%%%%%%%%%%%%%%%%%%%%%%%%%%%%%%








\bibitem{AKM_ch1_bib1}
Гэрри M., Джонсон Д. Вычислительные машины и труднорешаемые задачи: Пер. с англ. – М.: Мир, 1982. – 416 с.

\bibitem{AKM_ch1_bib2}
Norbert Blum, A Solution of the P versus NP Problem // Instit\"{u}t fur Informatik,
Universit\"{e}t Bonn / arXiv:1708.03486vl [cs.CC] 11 Aug 2017.

\bibitem{AKM_ch1_bib3}
Разборов А. А. Алгебраическая сложность. — М.:
\href{http://ru.wikipedia.org/wiki/МЦНМО}{http://ru.wikipedia.org/wiki/МЦНМО}, 2016. – 32 с.

\bibitem{AKM_ch1_bib4}
Асратян А. С., Камалян Р. Р. Интервальные раскраски ребер мультиграфа // Прикладная математика, вып. 5. Ереван:
Изд-во Ереванского ун-та, 1987. С. 25-34.

\bibitem{AKM_ch1_bib5}
Севастьянов С. В. Об интервальной раскрашиваемости ребер двудольного графа // Методы дискретного анализа, т. 50,
1990. – C.~61-72.

\bibitem{AKM_ch1_bib6}
Giaro K., Compact task scheduling on dedicated processors with no waiting period (in
Polish) // PhD thesis, Technical University of Gdansk, IETI Faculty, Gdansk. – 1999.

\bibitem{AKM_ch1_bib7}
Petrosyan P.A., Khachatrian H.H., Interval non-edge-colorable bipartite graphs
andmultigraphs, Journal of Graph Theory 76, Issue 3, 2014. – Р.~200-216.

\bibitem{AKM_ch1_bib8}
Giaro K., Kubale M. and Malafiejski M., On the deficiency of bipartite graphs, Discrete Appl.
Math. 94 (1999). – P.~193-203.

\bibitem{AKM_ch1_bib9}
Магомедов А.М. Элиминация перебора двудольных графов на 15 вершинах // ДЭМИ, Дагестанский научный центр РАН, 2016,
№5. – С.~20-24.

\bibitem{AKM_ch1_bib10}
Магомедов А.М. Цепочечные структуры в задачах о расписаниях // Прикладная дискретная математика. 2016, №3 (33). – С.~67-77.





%1. Магомедов А.М., Ибрагимова З.И. Анализ внутренней согласованности
%индивидуальных учебных нагрузок преподавателей. Свидетельство о государственной регистрации программы для ЭВМ <<Анализ
%внутренней согласованности индивидуальных учебных нагрузок преподавателей>> №2017617739. Заявка 2017614507, дата
%поступления 16 мая 2017 г. Дата государственной регистрации в Реестре программ для ЭВМ 11 июля 2017 г.
%
%2. Магомедов А.М. Свидетельство о государственной регистрации программы для ЭВМ <<Программа для
%вычислительного сопровождения распределения учебной нагрузки вузовской кафедры>> №
%2017661062. Заявка № 2017618227, дата поступления 15 августа 2017 г. Дата государственной
%регистрации в Реестре программ для ЭВМ 03 октября 2017 г.





\bibitem{AKM_ch3_bib1}
Магомедов А.М. Учебно-ознакомительный практикум по Autodesk 3ds MAX // Махачкала, изд-во ДГУ, 2017.
– 58 с.

\bibitem{AKM_ch3_bib2}
Магомедов А.М. Задания межрегиональной дистанционной олимпиады по программированию среди команд
вузов СКФО (2017 г.) // Махачкала, изд-во <<Радуга-1>>, 2017. – 18 с.


\bibitem{AKM_ch3_bib3}
Магомедов А.М. Вебкамеры в проектах Delphi [Электронный ресурс]. URL:
\url{http://eor.dgu.ru/lectures/list} (Дата обращения: 10.01.2018)



\bibitem{AKM_ch4_bib1}
Новиков Ф.А. Дискретная математика для программистов. – СПб.: Питер, 2000. – 304 с.

\bibitem{AKM_ch4_bib2}
Порев В.Н. Компьютерная графика. – СПб.: БХВ-Петербург, 2004. – 432~с.

\bibitem{AKM_ch4_bib3}
Гуков Д. Детали реализации двойной буферизации в Windows Forms: [Электронный ресурс].
URL: \href{https://habrahabr.ru/post/144294/}{https://habrahabr.ru/post/144294/} (дата обращения: 11.05.2017).

\bibitem{AKM_ch4_bib4}
Магомедов А.М. Воспроизведение и интерактивное редактирование графа в проекте Visual C\# //
Международный научно-исследовательский журнал (International research journal). № 06 (60) 2017 Часть~3. Июнь, Екатеринбург. - С. 152-155. ISSN 2303-9868 PRINT ISSN 2227-6017 ONLINE.


\bibitem{AKM_ch4_bib5}
Магомедов А.М. Свидетельство о государственной регистрации программы для ЭВМ «Построение и
визуальное редактирование графа с сокращенными списками смежности вершин» № 2017617670. Заявка № 2017614444, дата
поступления 12 мая 2017 г. Дата государственной регистрации в Реестре программ для ЭВМ 11 июля 2017 г.




\end{thebibliography}
