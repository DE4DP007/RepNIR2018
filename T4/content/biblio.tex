\begin{thebibliography}{111}




\bibitem{akm_2}
Kateleyn P.W. The statistic of dimers on a lattice I: The number of dimer arrangements on quadratic lattice.
Physica. --- 1961. --- V. 27. --- P. 1209---1225.

\bibitem{akm_3}
Temperley H.N.V. and Fisher M.E.
Dimer problem in statistical mechanics --- an exact result.
 Phil. Mag. --- 1961. --- V. 6. --- P. 1061---1063.
%
\bibitem{akm_6}
Ronald L. Graham, Donald E. Knuth, and Oren Patashnik.
Concrete Mathematics. --- Massachusetts: Addison-Wesley, 1994. --- 657 p.

\bibitem{akm_7}
Грэхем Р., Кнут Д., Паташник О.
Конкретная математика. Основание информатики. Пер. с англ. ---
М.: Мир, 1998. --- 704 с.
%

\bibitem{akm_10}
С.\,Г.~Волченков. Задача <<Паркет>> // Информатика и образование. --- 1994.
 --- № 3. --- C. 52---54.

\bibitem{akm_13}
Свами M.,Тхуласираман K. Графы, сети и алгоритмы. Пер. с англ. --- М.: Мир, 1984. --- 455 с.

\bibitem{akm_14}
Гэри М., Джонсон Д. Вычислительные машины и труднорешаемые задачи. Пер. с англ. --- М.: Мир, 1982. --- 416 с.

\bibitem{akm_15}
Lawler E.L. Combinatorial Optimization: Networks and Matroids. ---
University of California at Berkeley, Holt, Rinehart and Winston,  1976. --- 384 c.

\bibitem{akm_16}
Емеличев В.А.,Мельников О.И., Сарванов В.И., Тышкевич Р.И.
Лекции по теории графов. --- М.: Книжный дом <<Либроком>>, 2009. --- 392 с.

\bibitem{akm_17}
Магомедов А,М.,Магомедов Т.А.
Перечисление разбиений прямоугольника // Пенза, Проблемы теоретической кибернетики: XVIII международная конференция (19-23 июня 2017 г.). Материалы под редакцией Ю. И. Журавлева. --- М.: МАКС Пресс, 2017. --- C. 152---154.





%%%%%%%%%%%%%%%%%%%
%%%%%%%%%%%%%%%%%%%




\bibitem{akm_Lapt}
Лаптев В.Н. Методы разработки тестовых заданий в автоматизированной контролирующей системе <<Контроль>> В.Н. Лаптев, Е.В. Михайленко // Политематический сетевой электронный научный журнал Кубанского государственного аграрного университета (Научный журнал КубГАУ) [Электронный ресурс]. --- Краснодар: КубГАУ. --- 2015. --- № 09(113). С. 826---840. --- IDA [article ID]: 1131509061. --- URL: http://ej.kubagro.ru/.

\bibitem{akm_Mta}
Магомедов Т.А. Свидетельство № 2012616494 от 18.06.2012 о государственной регистрации программы для ЭВМ <<Генерация тестовых пунктов по дисциплине “Основы программирования”>>.

\bibitem{akm_Mam}
Магомедов А.М. Свидетельство № 2018612561 от 20.02.2018 о государственной регистрации программы для ЭВМ <<Автоматизация генерации тестовых пунктов по теме “Множества в языке Delphi”>>.





%%%%%%%%%%%%%%%%%%%
%%%%%%%%%%%%%%%%%%%
%%%%%%%%%%%%%%%%%%%%%%%%%%

\bibitem{RamShaMag}{Шарапудинов И.И., Магомед-Касумов М.Г.} Численный метод решения задачи Коши для систем обыкновенных дифференциальных уравнений с помощью ортогональной в смысле Соболева системы, порожденной системой косинусов // Дагестанские электронные математические известия. --- 2017. --- \No\ 8. --- С. 53---60.

\bibitem{RamSharDemr}{Шарапудинов И.И.} О приближении решения задачи Коши для нелинейных систем ОДУ посредством рядов Фурье по функциям, ортогональным по Соболеву // Дагестанские электронные математические известия. --- 2017. --- \No\ 7. --- С. 66---76.

\bibitem{RamShaIzv}{Шарапудинов И.И.} Системы функций, ортогональные по Соболеву, ассоциированные с ортогональной системой // Изв. РАН. Сер. матем. --- 2018. --- \No\  82:1. --- С. 212---244.

\bibitem{AGG_GRM}{Акниев Г.Г., Гаджимирзаев Р.М.} Алгоритм численной реализации полиномов по функциям, ортогональным по Соболеву и порожденным косинусами // Дагестанские электронные математические известия. --- 2018. --- \No\ 9. --- С. 1---6.









%%%%%%%%%%%%%%%%%%%
%%%%%%%%%%%%%%%%%%%
%Рамазанов М.К.

\bibitem{ph1_1}
Dotsenko V.S. Critical phenomena and quenched disorder. //
Phys Usp. --- 1995. --- \No\ 38(5). --- P. 457---496.

\bibitem{ph1_2}
Korshunov S.E. Phase transitions in two-dimensional systems with continuous de-generacy. //
Phys Usp. --- 2006. --- \No\ 49(3). --- P. 225–--262.

\bibitem{ph1_3}
Ramazanov M.K., Murtazaev A.K., Magomedov M.A.
Thermodynamic, critical prop-erties and phase transitions of the Ising model on a square lattice with competing in-teractions. //
Solid State Comm. --- 2016. --- \No\ 233. --- P. 35---40.

\bibitem{ph1_4}
Landau D.P., Binder K.
Monte Carlo simulations in statistical physics. --- Cambridge: Cambridge University Press, 2000.



%%%%%%%%%%%%%%%%%%%
%%%%%%%%%%%%%%%%%%%
%Бабаев

\bibitem{ph2_1}
Wiseman S. and Domany E. Self-averaging, distribution of pseudocritical temperatures, and finite size scaling in critical disordered systems // Phys. Rev. E. --- 1998. ---  V. 58. --- P. 2938.

\bibitem{ph2_2}
Wiseman S. and Domany E. Finite-Size Scaling and Lack of Self-Averaging in Critical Disordered Systems // Phys. Rev. Lett. --- 1998. --- V. 81. --- P. 22.

\bibitem{ph2_3}
Aharony A., Harris A.B. and Wiseman S. Critical Disordered Systems with Con-straints and the Inequality $\nu > 2/d$  // Phys. Rev. Lett. --- 1998. --- V. 81. --- P. 252.

\bibitem{ph2_4}
Pierre-Emmanuel Berche, Christophe Chatelain, Bertrand Berche, Wolfhard Janke. Bond dilution in the $3D$ Ising model: a Monte Carlo study // European Physical Journal B. --- 2004. --- V. 38. --- P. 463.

\bibitem{ph2_5}
Marqués M.I., Gonzalo J.A., Íñiguez J. Self-averaging of random and thermally disordered diluted Ising systems // Phys. Rev. E. --- 1999. --- V. 60. --- P. 2394.

\bibitem{ph2_6}
Marqués M.I., Gonzalo J.A., Íñiguez J. Universality class of thermally diluted Ising systems at criticality // Phys. Rev. E. --- 2000. --- V. 62. --- P. 191.

\bibitem{ph2_7}
Prudnikov V.V., Prudnikov P.V. and Fedorenko A.A. Field-theory approach to critical behaviour of systems with long-range correlated defects // Phys. Rev. B. --- 2000. --- V. 62. --- P. 8777.

\bibitem{ph2_8}
Прудников В.В., Прудников П.В., Вакилов А.Н., Криницин А.С. Компьютерное моделирование критического поведения трехмерной модели неупорядоченной модели Изинга // ЖЭТФ. --- 2007. --- Т. 132. --- С. 417.

\bibitem{ph2_9}
Доценко В.С. Критические явления в спиновых системах с беспорядком // УФН. --- 1995. --- Т. 165. --- С. 481.







%%%%%%%%%%%%%%%%%%%
%%%%%%%%%%%%%%%%%%%
%Магомедов М.А.

\bibitem{ph3_1}
Landau D.P., Binder K. Monte Carlo Simulations in Statistical Physics, Third Edition. --- Cambridge: Cambridge University Press, 2009. [https://doi.org/10.1017/cbo9780511994944]

\bibitem{ph3_2}
Shannon R.D., Rogers D.B. and Prewitt C.T. Chemistry of Noble Metal Oxides 1. Syntheses and Properties of ABO2 Delafossite Compounds. // Inorg. Chem. --- 1971. --- \No\ 10. --- P. 713--–718. [https://doi.org/10.1021/ic50098a011]

\bibitem{ph3_3}
Melanie J., Soraya Heuss-Aßbichler So-Hyun Park et all., Low-temperature synthesis of CuFeO2 (delafossite) at 70C: A new process solely by precipitation and ageing. // J. Solid State Chem. --- 2016. --- \No\ 233. --- 390---396. [https://doi.org/10.1016/j.jssc.2015.11.011]

\bibitem{ph3_4}
Murtazaev A.K., Babaev A.B., Magomedov M.A., Kassan-Ogly F.A., Proshkin A.I. Frustrations and phase transitions in the three vertex Potts model with next nearest neighbor interactions on a triangular lattice. // JETP letters. --- 2014. --- \No\ 100(4). --- P. 242---246. [https://doi.org/10.1134/s0021364014160115]

\bibitem{ph3_5}
Wang F., Landau D.P. Efficient, multiple-range random walk algorithm to calculate the density of states. // Phys. Rev. Lett. --- 2001. --- \No\ 864. --- P. 2050---2053. [https://doi.org/10.1103/physrevlett.86.2050]

\bibitem{ph3_6}
Landau D.P., Wang F. A new approach to Monte Carlo simulations in statistical physics. // Braz. J. Phys. --- 2004. --- \No\ 34(2A). 354---362. [https://doi.org/10.1590/s0103-97332004000300004]

\bibitem{ph3_7}
Komura Y., Okabe Y. Difference of energy density of states in the Wang-Landau algorithm. // Phys. Rev. E. --- 2012. --- \No\ 85. --- 010102(R). [https://doi.org/10.1103/physreve.85.010102]





%%%%%%%%%%%%%%%%%%%
%%%%%%%%%%%%%%%%%%%


%%%%%%%%%%%%%%%%%%%
%%%%%%%%%%%%%%%%%%%
%magomedrasulsalikh1

\bibitem{mmgmsr1-Kashin}
Кашин Б.С., Саакян А.А. Ортогональные ряды. --- М.: Изд-во АФЦ, 1999. --- 560 с.

\bibitem{mmgmsr1-Sha18}
Шарапудинов И.И. Системы функций, ортогональные по Соболеву, ассоциированные с ортогональной системой // Изв. РАН. Сер. матем. --- 2018. --- Т. 82. № 1. --- С. 225---258.

\bibitem{mmgmsr1-Shii-Shti-izvvuzov2017}
Шарапудинов И.И., Шарапудинов Т.И. Полиномы, ортогональные по Соболеву, порожденные многочленами Чебышева, ортогональными на сетке // Изв. вузов. Матем. --- 2017. --- № 8. --- С. 67---79.

\bibitem{mmgmsr1-Shii-matzam2017}
Шарапудинов И.И. Аппроксимативные свойства рядов Фурье по многочленам, ортогональным по Соболеву с весом Якоби и дискретными массами // Матем. заметки. --- 2017. --- Т. 101. --- № 4. --- С. 611---629. (Math. Notes. --- 2017. --- V. 101. --- № 4. --- P. 718---734.)

\bibitem{mmgmsr1-ShaGadGad16}
Шарапудинов И.И., Гаджиева З.Д., Гаджимирзаев Р.М. Системы функций, ортогональных относительно скалярных произведений типа Соболева с дискретными массами, порожденных классическими ортогональными системами // Дагестанские электронные математические известия. --- 2016. --- № 6. --- С. 31---60.

\bibitem{mmgmsr1-ShaGad16}
Шарапудинов И.И., Гаджиева З.Д. Полиномы, ортогональные по Соболеву, порожденные многочленами Мейкснера // Изв. Сарат. ун-та. Нов. сер. Сер. Математика. Механика. Информатика. --- 2016. --- Т. 16. --- № 3. --- С. 310---321.

\bibitem{mmgmsr1-Shii-lag-demi2015}
Шарапудинов И.И. Некоторые специальные ряды по общим полиномам Лагерра и ряды Фурье по полиномам Лагерра, ортогональным по Соболеву // Дагестанские электронные математические известия. --- 2015. --- № 4. --- С. 31---73.

\bibitem{mmgmsr1-SHII-MMG-Demi2015}
Шарапудинов И.И., Магомед-Касумов М.Г., Магомедов С.Р. Полиномы, ортогональные по Соболеву, ассоциированные с полиномами Чебышева первого рода // Дагестанские электронные математические известия. --- 2015. --- № 4. --- С. 1---14.

\bibitem{mmgmsr1-ShaOdeDemi2017}
Шарапудинов И.И. О приближении решения задачи Коши для нелинейных систем ОДУ посредством рядов Фурье по функциям, ортогональным по Соболеву // Дагестанские электронные математические известия. --- 2017. --- № 7. --- С. 66---76.

\bibitem{mmgmsr1-ShaMagOdeCos2017}
Шарапудинов И.И., Магомед-Касумов М.Г. Численный метод решения задачи Коши для систем обыкновенных дифференциальных уравнений с помощью ортогональной в смысле Соболева системы, порожденной системой косинусов // Дагестанские электронные математические известия. --- 2017. --- Вып. 8. --- С. 53---60.

\bibitem{mmgmsr1-SMS-SHTI-Demi2017}
Султанахмедов М.С., Шарапудинов Т.И. Приближенное решение задачи Коши для систем ОДУ с помощью ортогональной в смысле Соболева системы, порожденной полиномами Чебышева первого рода // Дагестанские Электронные Математические Известия. --- 2017. --- № 8. --- С. 100---109.

\bibitem{mmgmsr1-SHII-MSR-Demi2017}
Шарапудинов И.И., Магомедов С.Р. Системы функций, ортогональные по Соболеву, ассоциированные с функциями Хаара, и задача Коши для ОДУ // Дагестанские Электронные Математические Известия. --- 2017. --- № 7. --- C. 1---15.

\bibitem{mmgmsr1-ShaGadGad17}
Шарапудинов И.И., Гаджиева З.Д., Гаджимирзаев Р.М. Разностные уравнения и полиномы, ортогональные по Соболеву, порожденные многочленами Мейкснера // Владикавк. матем. журн. --- 2017. --- Т. 19. --- № 2. --- С. 58---72.

\bibitem{mmgmsr1-SharapudinovMuratova}
Шарапудинов И.И., Муратова Г.Н. Некоторые свойства r-кратно интегрированных рядов по системе Хаара // Изв. Сарат. ун-та. Нов. сер. Сер. Математика. Механика. Информатика. --- 2009. --- Т. 9. --- № 1. --- С. 68---76.

%%%%%%%%%%%%%%%%%%%
%%%%%%%%%%%%%%%%%%%
%magomedrasulsalikh2

\bibitem{mmgmsr2-SHII-Demi2017-ODESystems}
Шарапудинов И.И. О приближении решения задачи Коши для нелинейных систем ОДУ посредством рядов Фурье по функциям, ортогональным по Соболеву // Дагестанские Электронные Математические Известия. --- 2017. --- № 7. --- C. 66---76.

\bibitem{mmgmsr2-SharapudinovMuratova}
Шарапудинов И.И., Муратова Г.Н. Некоторые свойства r-кратно интегрированных рядов по системе Хаара // Изв. Сарат. ун-та. Нов. сер. Сер. Математика. Механика. Информатика. --- 2009. --- Т. 9. --- № 1. --- С. 68---76.

\bibitem{mmgmsr2-Shii-asymp-demi2016}
Шарапудинов И.И.  Асимптотические свойства полиномов, ортогональных по Соболеву, порожденных полиномами Якоби // Дагестанские электронные математические известия. --- 2016. --- № 6. --- С. 1---24.

\bibitem{mmgmsr2-Shii-Gadzhieva2016}
Шарапудинов И.И. Гаджиева З.Д. Полиномы, ортогональные по Соболеву, порожденные многочленами Мейкснера // Изв. Сарат. ун-та. Нов. сер. Сер. Математика. Механика. Информатика. --- 2016. --- Т. 16. --- № 3. --- С. 310---321.

\bibitem{mmgmsr2-Shii-matzam2017}
Шарапудинов И.И. Аппроксимативные свойства рядов Фурье по многочленам, ортогональным по Соболеву с весом Якоби и дискретными массами // Матем. заметки. --- 2017. --- Т. 101. --- № 4. --- С. 611---629.

\bibitem{mmgmsr2-Shii-Shti-izvvuzov2017}
Шарапудинов И.И., Шарапудинов Т.И. Полиномы, ортогональные по Соболеву, порожденные многочленами Чебышева, ортогональными на сетке // Изв. вузов. Матем. --- 2017. --- № 8. --- С. 67---79.

\bibitem{mmgmsr2-Shii-lag-demi2015}
Шарапудинов И.И. Некоторые специальные ряды по общим полиномам Лагерра и ряды Фурье по полиномам Лагерра, ортогональным по Соболеву // Дагестанские электронные математические известия. --- 2015. --- № 4. --- С. 31---73.

\bibitem{mmgmsr2-SHII-MMG-Demi2015}
Шарапудинов И.И., Магомед-Касумов М.Г., Магомедов С.Р. Полиномы, ортогональные по Соболеву, ассоциированные с полиномами Чебышева первого рода // Дагестанские электронные математические известия. --- 2015. --- № 4. --- С. 1---14.

\bibitem{mmgmsr2-SHII-MMG-Demi2017-CosOde}
Шарапудинов И.И., Магомед-Касумов М.Г. Численный метод решения задачи Коши для систем обыкновенных дифференциальных уравнений с помощью ортогональной в смысле Соболева системы, порожденной системой косинусов // Дагестанские Электронные Математические Известия. --- 2017. --- № 8. --- С. 53---60.

\bibitem{mmgmsr2-Kashin}
Кашин Б.С., Саакян А.А. Ортогональные ряды. --- Москва: АФЦ, 1999.

\bibitem{mmgmsr2-SHII-MSR-Demi2017}
Sharapudinov I.I., Magomedov S.R. Systems of functions orthogonal in the sense of Sobolev associated with Haar functions and the Cauchy problem for ODEs // Дагестанские Электронные Математические Известия. --- 2017. --- № 7. --- C. 1---15.

\bibitem{mmgmsr2-bib-haar-fast-calc}
Магомед-Касумов М.Г., Магомедов С.Р. Быстрое вычисление линейных комбинаций соболевских функций, порожденных функциями Хаара // Дагестанские Электронные Математические Известия. --- 2018. --- № 9. --- Стр. 7---14.

\bibitem{mmgmsr2-bib-farcov}
Фарков Ю.А. Ряды Фурье и основы вейвлет-анализа. --- Москва, 2007.

%%%%%%%%%%%%%%%%%%%
%%%%%%%%%%%%%%%%%%%
%sms-stn-1

\bibitem{sms-stn-1-MarcelXu} Marcellan F., Yuan Xu. On Sobolev orthogonal polynomials // Expositiones Mathematicae. --- 2015. --- V. 33. --- \No\ 3. --- P. 308---352.

\bibitem{sms-stn-1-SharIzVuz} Шарапудинов И.И. Системы функций, ортогональные по Соболеву, ассоциированные с ортогональной системой // Изв. РАН. Сер. матем. --- 2018. --- Т. 82. --- № 1. --- С. 225---258.

\bibitem{sms-stn-1-dctBook}
K. Rao, P. Yip. Discrete Cosine Transform. Algorithms, Advantages, Applications. 1st Edition. --- Academic Press, 2014. --- 512 p.

%%%%%%%%%%%%%%%%%%%
%%%%%%%%%%%%%%%%%%%
%sms-stn-1

\bibitem{sms-stn-2-PolOrtPorSobChebUrav} Шарапудинов И.И. Полиномы, ортогональные по Соболеву, ассоциированные с полиномами Чебышева первого рода, и задача Коши для обыкновенных дифференциальных уравнений // Дифференциальные уравнения. --- 2018. --- Т. 54. --- № 12. --- С. 1645---1662.

\bibitem{sms-stn-2-demiSMS_ShTN} Султанахмедов М.С., Шах-Эмиров Т.Н. Алгоритм быстрого дискретного преобразования для сумм Фурье по ортогональным по Соболеву полиномам, порожденным полиномами Чебышева первого рода // Дагестанские электронные математические известия. --- 2018. --- № 9. --- С. 52---61.

\bibitem{sms-stn-2-MarcelXu} Marcellan F., Yuan Xu. On Sobolev orthogonal polynomials // Expositiones Mathematicae. --- 2015. --- V. 33. № 3. --- P. 308---352.

\bibitem{sms-stn-2-SharIzVuz} Шарапудинов И.И. Системы функций, ортогональные по Соболеву, ассоциированные с ортогональной системой // Изв. РАН. Сер. матем. --- 2018. --- Том. 82. --- № 1. --- С. 225---258.

\bibitem{sms-stn-2-dctBook}
K. Rao, P. Yip. Discrete Cosine Transform. Algorithms, Advantages, Applications. 1st Edition. --- Academic Press, 2014. --- 512 p.

\end{thebibliography}
