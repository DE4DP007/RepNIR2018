\begin{thebibliography}{111}




\bibitem{2}
P.W.~Kateleyn. The statistic of dimers on a lattice I: The number of dimer arrangements on quadratic lattice.
Physica. Vol.\,27. 1961. PP. 1209--1225.

\bibitem{3}
H.\,N.\,V.~Temperley and M.\,E.~Fisher.
Dimer problem in statistical mechanics --- an exact result.
 Phil.\,Mag. Vol.\,6. 1961. PP. 1061--1063.
%
\bibitem{6}
Ronald~L.~Graham, Donald~E.~Knuth, and~Oren~Patashnik.
Concrete Mathematics. Massachusetts, Addison-Wesley. 1994. 657 p.

\bibitem{7}
Грэхем~Р., Кнут~Д., Паташник~О.
Конкретная математика. Основание информатики. Пер. с англ.
М.: Мир, 1998. 704 с.
%

\bibitem{10}
С.\,Г.~Волченков. Задача <<Паркет>> // Информатика и образование. 1994.
№ 3. C. 52--54.

\bibitem{13}
M.~Свами, K.~Тхуласираман. Графы, сети и алгоритмы. Пер. с англ. М.: Мир, 1984. 455 с.

\bibitem{14}
М.~Гэри, Д.~Джонсон. Вычислительные машины и труднорешаемые задачи. Пер. с англ.
М.: Мир, 1982. 416 с.

\bibitem{15}
E.\,L.~Lawler. Combinatorial Optimization: Networks and Matroids.
University of California at Berkeley, Holt, Rinehart and Winston. 1976. 384 c.

\bibitem{16}
В.\,А.~Емеличев , О.\,И.~Мельников, В.\,И.~Сарванов, Р.\,И.~Тышкевич.
Лекции по теории графов. М.: Книжный дом <<Либроком>>. 2009. 392 с.

\bibitem{17}
А.\,М.~Магомедов, Т.\,А.~Магомедов.
Перечисление разбиений прямоугольника // Пенза, Проблемы теоретической кибернетики: XVIII международная конференция (19-23 июня 2017 г.). Материалы под редакцией Ю. И. Журавлева. -- М.: МАКС Пресс, 2017. C. 152--154.





%%%%%%%%%%%%%%%%%%%
%%%%%%%%%%%%%%%%%%%




\bibitem{Lapt}
Лаптев В.\,Н. Методы разработки тестовых заданий в автоматизированной контролирующей системе <<Контроль>> / В.\,Н. Лаптев, Е.\,В. Михайленко // Политематический сетевой электронный научный журнал Кубанского государственного аграрного университета (Научный журнал КубГАУ) [Электронный ресурс]. -- Краснодар: КубГАУ. 2015. -- № 09(113). С. 826--840. -- IDA [article ID]: 1131509061. -- Режим доступа: http://ej.kubagro.ru/.

\bibitem{Mta}
Магомедов Т.\,А. Свидетельство № 2012616494 от 18.06.2012 о государственной регистрации программы для ЭВМ <<Генерация тестовых пунктов по дисциплине “Основы программирования”>>.

\bibitem{Mam} Магомедов А.\,М. Свидетельство № 2018612561 от 20.02.2018 о государственной регистрации программы для ЭВМ <<Автоматизация генерации тестовых пунктов по теме “Множества в языке Delphi”>>.







%%%%%%%%%%%%%%%%%%%
%%%%%%%%%%%%%%%%%%%
%Рамазанов М.К.

\bibitem{ph1_1}
Dotsenko V.\,S. Critical phenomena and quenched disorder.
Phys Usp. 1995; 38(5). Pp.~457--496.

\bibitem{ph1_2}
Korshunov S.\,E. Phase transitions in two-dimensional systems with continuous de-generacy.
Phys Usp. 2006; 49 (3). Pp.~225–262.

\bibitem{ph1_3}
Ramazanov M.\,K., Murtazaev A.\,K., Magomedov M.\,A.
Thermodynamic, critical prop-erties and phase transitions of the Ising model on a square lattice with competing in-teractions.
Solid State Comm. 2016; 233. Pp.~35--40.

\bibitem{ph1_4}
Landau D.\,P., Binder K.
Monte Carlo simulations in statistical physics. Cambridge: Cambridge University Press; 2000.



%%%%%%%%%%%%%%%%%%%
%%%%%%%%%%%%%%%%%%%
%Бабаев

\bibitem{ph2_1}
S. Wiseman and E. Domany. Self-averaging, distribution of pseudocritical temperatures, and finite size scaling in critical disordered systems // Phys. Rev. E, 1998, v. 58, p. 2938.

\bibitem{ph2_2}
S. Wiseman and E. Domany. Finite-Size Scaling and Lack of Self-Averaging in Critical Disordered Systems // Phys. Rev. Lett., 1998, v. 81, p. 22.

\bibitem{ph2_3}
A. Aharony, A.B. Harris, and S. Wiseman. Critical Disordered Systems with Con-straints and the Inequality $\nu > 2/d$  // Phys. Rev. Lett., 1998, v. 81,  p. 252.

\bibitem{ph2_4}
Pierre-Emmanuel Berche, Christophe Chatelain, Bertrand Berche, Wolfhard Janke. Bond dilution in the $3D$ Ising model: a Monte Carlo study // European Physical Jour-nal B, 2004, v. 38, p. 463.

\bibitem{ph2_5}
M.I. Marqués, J.A. Gonzalo, J. Íñiguez. Self-averaging of random and thermally disordered diluted Ising systems // Phys. Rev. E, 1999, v. 60, p. 2394.

\bibitem{ph2_6}
M.I. Marqués, J.A. Gonzalo, J. Íñiguez. Universality class of thermally diluted Ising systems at criticality // Phys. Rev. E, 2000, v. 62, p. 191.

\bibitem{ph2_7}
V.V. Prudnikov, P.V. Prudnikov, and A.A. Fedorenko. Field-theory approach to critical behaviour of systems with long-range correlated defects // Phys. Rev. B, 2000, v. 62, p. 8777.

\bibitem{ph2_8}
В.В. Прудников, П.В. Прудников, А.Н. Вакилов, А.С. Криницин. Компьютерное моделирование критического поведения трехмерной модели неупорядоченной модели Изинга // ЖЭТФ, 2007, т. 132, c. 417.

\bibitem{ph2_9}
В.С. Доценко. Критические явления в спиновых системах с беспорядком // УФН, 1995, т. 165, с. 481.







%%%%%%%%%%%%%%%%%%%
%%%%%%%%%%%%%%%%%%%
%Магомедов М.А.

\bibitem{ph3_1}
Landau D.P., Binder K., Monte Carlo Simulations in Statistical Physics, Third Edition, Cambridge University Press, Cambridge, 2009. [https://doi.org/10.1017/cbo9780511994944]

\bibitem{ph3_2}
Shannon R.D., Rogers D.B. and Prewitt C.T., Chemistry of Noble Metal Oxides 1. Syntheses and Properties of ABO2 Delafossite Compounds. Inorg. Chem. 1971, 10, 713–8. [https://doi.org/10.1021/ic50098a011]

\bibitem{ph3_3}
Melanie J., Soraya Heuss-Aßbichler, So-Hyun Park et all., Low-temperature synthesis of CuFeO2 (delafossite) at 70C: A new process solely by precipitation and ageing. J. Solid State Chem. 2016, 233, 390-396. [https://doi.org/10.1016/j.jssc.2015.11.011]

\bibitem{ph3_4}
Murtazaev A.K., Babaev A.B., Magomedov M.A., Kassan-Ogly F.A., Proshkin A.I., Frustrations and phase transitions in the three vertex Potts model with next nearest neighbor interactions on a triangular lattice. JETP letters, 2014, 100(4), 242-246. [https://doi.org/10.1134/s0021364014160115]

\bibitem{ph3_5}
Wang F., Landau D.P., Efficient, multiple-range random walk algorithm to calculate the density of states. Phys. Rev. Lett 2001, 864, 2050-2053. [https://doi.org/10.1103/physrevlett.86.2050]

\bibitem{ph3_6}
Landau D.P., Wang F., A new approach to Monte Carlo simulations in statistical physics. Braz. J. Phys. 2004, 34(2A), 354-362. [https://doi.org/10.1590/s0103-97332004000300004]

\bibitem{ph3_7}
Komura Y., Okabe Y., Difference of energy density of states in the Wang-Landau algorithm. Phys. Rev. E 2012, 85, 010102(R). [https://doi.org/10.1103/physreve.85.010102]





%%%%%%%%%%%%%%%%%%%
%%%%%%%%%%%%%%%%%%%






\end{thebibliography}
