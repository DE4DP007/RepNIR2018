\Conclusion

В 2018 году в Отделе математики и информатики Дагестанского научного центра РАН проведены научно-исследовательские работы по теме
<<Разработка алгоритмов и создание наукоемкого программного обеспечения для моделирования сложных систем. Некоторые вопросы цифровой обработки сигналов и изображений. Исследования по теории графов и теории оптимизации расписаний, компьютерное сопровождение вузовского учебного процесса>>.



%%%%%%%%%%%%%%%%%%%%%%%%%%%%
%%%%%%%%%%%%%%%%%%%%%%%%%%%%
%==========================


Все известные формулы для перечисления разбиений прямоугольника используют действия с плавающей запятой. С помощью авторского программного обеспечения для ряда случаев уточнены точные значения для количества разбиений с применением ранее полученных автором формул, использующих лишь операции над целыми числами.




Задача перечисления разбиений прямоугольника заданных целочисленных размеров $h\times w$ на прямоугольники $1\times 2$ рассматривалась рядом авторов в связи с вопросами термодинамики потоков жидкости и проблемой перечисления совершенных паросочетаний плоского графа за полиномиальное время.

Но все известные формулы решения задачи используют действия с плавающей запятой, что  сопряжено с проблемами округления; исключение составляет лишь полученная в известной монографии Д. Кнута и др. для случая $w=3$ система из двух взаимно-рекуррентных формул, использующая только операции сложения целых чисел.

В разделе 1 разработан алгоритм, компьютерное воплощение которого способно для искомого перечисления разбиений прямоугольника сгенерировать систему взаимно-рекуррентных формул, использующих лишь операции сложения целых чисел и, таким образом, свободных от проблем округления вещественных чисел.





%%%%%%%%%%%%%%%%%%%%%%%%%%%%
%%%%%%%%%%%%%%%%%%%%%%%%%%%%
%==========================


%Приведенный описательный алгоритм компьютерной генерации тестовой единицы 15,3 (тема <<Множества>>, форма 3) позволяет, как мы надеемся, получить вполне определенное представление о структуре и способе генерации тестовых единиц и по другим темам, независимо от выбора одной из пяти перечисленных выше форм.
%
%Алгоритм воплощен в компьютерные программы \cite{Mta}--\cite{Mam}, которые в течение нескольких лет используются для автоматизации разработки практически неограниченного количества разнообразных тестовых единиц по основным темам языка Delphi 7.0 и для достижения надежности тестового материала, а также для проведения независимого промежуточного контроля усвоения материала.
%

%%%%%%%%%%%%%%%%%%%%%%%%%%%%
%%%%%%%%%%%%%%%%%%%%%%%%%%%%
%==========================

В олимпиаде приняли участие 34 команды вузов СКФО. Результаты позволяют утверждать, что степень трудности заданий была продумана методически грамотно: так, команда, занявшая 1-е место, набрала 75 баллов из 100 возможных, а команда, занявшая 2-е место --- 63 балла (из 100 возможных).

%%%GasanRamis
Программно реализован итерационный метод нахождения приближенного решения задачи Коши для ОДУ, путем вычисления коэффициентов разложения этого решения в ряд Фурье по функциям, ортонормированным
относительно скалярного произведения Соболева и порожденным косинусами.
Вычисление неизвестных коэффициентов осуществляется посредством применения алгоритма быстрого преобразования Фурье. Разработанная программа зарегистрирована в Роспатенте под названием <<Программа нахождения приближенного решения задачи Коши для ОДУ>>.

%%%MRasulSalih1

Разработан алгоритм вычисления линейных комбинаций $N$ функций $\chi_{1,n}(x)$ за $O(\log N)$ операций.
Рассмотрены некоторые свойства ортогональных по Соболеву функций $\chi_{1,n}(x)$, порожденных функциями Хаара.

%%%MRasulSalih2

Рассмотрен итерационный метод решения задачи Коши для систем нелинейных дифференциальных уравнений, основанный на использовании ортогональной в смысле Соболева системы функций, порожденной функциями Хаара.


%%%%%%%%%%%%%%%%%%%%%%%%%%%%
%%%%%%%%%%%%%%%%%%%%%%%%%%%%
%==========================







%В отчетном году была разработана общая методика приближенного решения дифференциальных и разностных уравнений с помощью функций, ортогональных по Соболеву, а также конкретные методы, основанные на системах функций, ортогональных по Соболеву и порожденных такими классическими системами как система Хаара, система полиномов Чебышева первого рода и система косинусов. Эти методы были использованы для построения алгоритмов численно-аналитического решения дифференциальных и разностных уравнений. Разработаны прикладные компьютерные программы, реализующие эти алгоритмы, и с их помощью проведена серия численных экспериментов.
%
%
%
%%%%%%%%%%%%%%%%%%%%%%%%%%
%
%Далее, на основе тригонометрических сумм Фурье и классических средних Валле Пуссена учеными ОМИ были введены в рассмотрение повторные средние Валле Пуссена. На их основе сконструированы операторы, которые осуществляют так называемые перекрывающие преобразования (lapped transform). Эти операторы, в свою очередь, легли в основу разработанных алгоритмов и созданного программного обеспечения для обработки и сжатия временных рядов и изображений, которые могут быть использованы в решении важных прикладных задач.
%
%
%
%
%%%%%%%%%%%%%%%%%%%%%%%%%%
%
%
%
%
%В области теории оптимизации расписаний для мультипроцессорных систем исследована задача об интервальной реберной раскрашиваемости двудольных графов порядка 15, 16, 17 и 18. С привлечением компьютерных ресурсов и методов теории графов в 2017 г. с доказано, что все двудольные графы порядка 15 интервально раскрашиваемы; интервально раскрашиваемы также и все двудольные графы $G=(X,Y,E)$ порядка 16 при $min(|X|, |Y|)<7$. Разработан эффективный для практических целей новый алгоритм проверки интервальной раскрашиваемости графа, на его основе
%создано программное обеспечение, обладающее рядом возможностей по выбору удобного представления графа и по отображению
%его интервальной раскраски.  Разработан алгоритм малоизбыточного перечисления всех
%двудольных графов заданного порядка (не более 20) путем отсеивания значительного количества изоморфных графов, алгоритм
%воплощен в программное обеспечение.
%
%
%
%
%%%%%%%%%%%%%%%%%%%%%%%%%%
%
%
%Компьютерное тестирование занимает определенную нишу в решении проблемы независимого контроля знаний учащихся. В
%		свою очередь, создание надежного и массового тестового материала является столь трудоемким процессом, что его
%		автоматизация весьма востребована. Если в общем случае перспективы такой автоматизации неясны, то в случае конкретных
%		учебных дисциплин компьютерная генерация содержательного, массового и надежного тестового материала является вполне
%		достижимой целью. 	
%Нами был разработан алгоритм компьютерного формирования тестовых заданий по
%			основам программирования на языке Delphi 7.0.
%Алгоритм воплощен в компьютерную программу. По каждой теме учебной дисциплины рассмотрены пять
%			формализованных структур тестовых заданий и одна «нестандартная» форма, предусматривающая творческий анализ той или
%			иной нестандартной ситуации.
%
%
%
%%%%%%%%%%%%%%%%%%%%%%%%%%
%
%В качестве практического применения теории оптимизации расписаний в сопровождении вузовского учебного процесса рассмотрена задача об автоматизации процесса распределения учебной нагрузки вузовской кафедры.
%Разработано программное обеспечение для генерации сводных таблиц распределения учебной нагрузки, проверки
%согласованности индивидуальных планов преподавателей кафедры, а также автоматизации подготовки исходных данных к
%расписанию.
%
%
%
%%%%%%%%%%%%%%%%%%%%%%%%%%
%
%В связи с сопровождением вузовского учебного процесса в области компьютерных наук, проводилась работа по подготовке студентов к конкурсам и олимпиадам по программированию, а также соответствующая учебно-методическая работа (в частности, издание учебных пособий с решениями нестандартных задач). Сотрудниками ОМИ в 2017 г. были изданы учебное пособие, содержащее набор нестандартных упражнений по 3ds Max с решениями, учебно-методическое пособие по итогам студенческой олимпиады вузов СКФО по программированию, а также (в электронном виде) цикл лекций по программированию. Все три пособия \cite{AKM_ch4_bib1}-\cite{AKM_ch4_bib3}  размещены в сети Интернет для свободного скачивания студентами.
%
%%%%%%%%%%%%%%%%%%%%%%%%%%
%
%Были рассмотрены вопросы визуального представления графов. Необходимость в отображении графов возникает как в ходе учебных
%занятиях по дискретной математике и компьютерной графике, так и для создания полиграфических документов (статей, учебных пособий) с пояснительными рисунками по тематике теории графов. Предложены компактная структура представления исходных данных и способ повышения устойчивости к ошибкам набора данных, изложен способ визуального редактирования графа. Рассмотренная компактная структура исходного файла с информацией о графе допускает тривиальное обобщение на случай смешанного графа.





