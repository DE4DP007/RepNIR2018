\Introduction
Интерес к теории функций, ортогональных относительно скалярных произведений типа Соболева в последнее время интенсивно растет  (см., например, обзор \cite{sms-stn-1-MarcelXu} и цитированную там литературу). Неослабевающее внимание специалистов  к этому направлению теории ортогональных функций можно объяснить в том числе и тем обстоятельством, что ряды Фурье по функциям, ортогональным по Соболеву, оказались естественным и весьма удобным инструментом для представления решений  дифференциальных (разностных) уравнений. Это можно показать, в частности, на примере  задачи Коши для дифференциального уравнения
\begin{equation}\label{intro-1.1}
F(x,y,y',\ldots,y^{(r)})=0
 \end{equation}
с начальными условиями $y^{(k)}(a)=y_k$, $k=0,1,\ldots,r-1$.  Наряду с различными сеточными методами для решения этой задачи часто применяют так называемые спектральные методы. Напомним, что суть спектрального метода решения задачи Коши  для ОДУ \eqref{intro-1.1} заключается в том, что в первую очередь искомое решение $y(x)$ представляется в виде ряда Фурье
\begin{equation}\label{intro-1.2}
 y(x)=\sum_{k=0}^\infty \hat y_k\psi_k(x)
 \end{equation}
по подходящей ортонормированной системе $\{\psi_k(x)\}_{k=0}^\infty$ (чаще всего в качестве $\{\psi_k(x)\}_{k=0}^\infty$ используют    тригонометрическую систему, ортогональные полиномы, вэйвлеты, корневые функции того или иного дифференциального оператора  и некоторые другие). На втором этапе осуществляется подстановка вместо $y(x)$ ряда \eqref{intro-1.2} в уравнение \eqref{intro-1.1}. Это приводит к системе уравнений относительно неизвестных коэффициентов $\hat y_k$ ($k=0,1,\ldots$). На третьем этапе требуется решить эту систему с учетом начальных условий  $y^{(k)}(-1)=y_k$, $k=0,1,\ldots,r-1$ исходной задачи Коши.
Одна из основных трудностей, которая возникает на этом этапе, состоит в том, чтобы
выбрать такой ортонормированный базис $\{\psi_k(x)\}_{k=0}^\infty$, для которого искомое решение $y(x)$ уравнения \eqref{intro-1.1}, представленное в виде ряда  \eqref{intro-1.2}, удовлетворяло бы начальным условиям $y^{(k)}(a)=y_k$, $k=0,1,\ldots,r-1$. Более того, поскольку в результате решения системы уравнений относительно неизвестных коэффициентов $\hat y_k$  будет найдено только конечное их число с $k=0,1,\ldots, n$, то весьма важно, чтобы частичная сумма ряда \eqref{intro-1.2} вида $ y_n(x)=\sum_{k=0}^n\hat y_k\psi_k(x)$,
 будучи приближенным решением рассматриваемой задачи Коши, также удовлетворяла начальным условиям $y_n^{(k)}(a)=y_k$, $k=0,1,\ldots,r-1$.

В работах И.\,И. Шарапудинова (см., например, \cite{RamShaIzv, mmgmsr1-Shii-Shti-izvvuzov2017, mmgmsr1-Shii-matzam2017, mmgmsr1-ShaGadGad16, mmgmsr1-ShaGad16, mmgmsr1-Shii-lag-demi2015, mmgmsr1-SHII-MMG-Demi2015}) показано, что базис $\Phi_r=\{\varphi_{r,n}\}_{n=0}^\infty$, $r \ge 1$, состоящий из функций, ортонормированных относительно скалярного произведения типа Соболева
\begin{equation}\label{mmgmsr1-inner-prod-gen}
\langle f,g \rangle =
\sum_{\nu=0}^{r-1}f^{(\nu)}(a)g^{(\nu)}(a)+\int_{a}^{b}f^{(r)}(x)g^{(r)}(x)\mu(x)dx.
\end{equation}
обладает указанными свойствами.
В обозначенных выше работах разработаны также теория и методы построения систем функций $\Phi_r$. Для построения системы $\Phi_r$ выбирается одна из классических систем $\Phi =\{\varphi_n\}_{n=0}^\infty$, ортонормированных относительно обычного скалярного произведения
\begin{equation}\label{mmgmsr1-classic-mul}
\langle f,g \rangle =\int_{a}^{b}f(t)g(t)\mu(t)dt,
\end{equation}
и на её основе строятся функции $\varphi_{r,n}(x)$ посредством равенств
\begin{gather}
\label{mmgmsr1-phirk-1}
\varphi_{r,n}(x) =\frac{(x-a)^n}{n!}, \quad n=0,1,\ldots, r-1,\\
\label{mmgmsr1-phirk-2}
\varphi_{r,n}(x) =\frac{1}{(r-1)!}\int\limits_{a}^x(x-t)^{r-1}\varphi_{n-r}(t)dt, \quad n=r,r+1,\ldots.
\end{gather}
Система функций $\Phi_r=\{\varphi_{r,n}\}_{n=0}^\infty$, определённая формулами \eqref{mmgmsr1-phirk-1}, \eqref{mmgmsr1-phirk-2}, как было показано в упомянутых выше работах, будет ортонормирована относительно скалярного произведения \eqref{mmgmsr1-inner-prod-gen}.

Ряды Фурье по функциям, ортогональным в смысле Соболева, могут быть эффективно использованы при решении задачи Коши для систем ОДУ. В частности, в \cite{RamSharDemr, RamShaMag, mmgmsr1-SMS-SHTI-Demi2017, mmgmsr1-SHII-MSR-Demi2017} предложен основанный на применении рядов Фурье по системам $\Phi_1=\{\varphi_{1,n}\}$ итерационный метод решения задачи Коши для нелинейных систем ОДУ первого порядка. Метод основан на представлении неизвестного решения $y(x)$ рассматриваемой задачи Коши в виде равномерно сходящегося ряда Фурье по системе вида $\Phi_1$:
\begin{equation*}
y(x) = y(a)+\sum\limits_{n=1}^{\infty}
c_{1,n}(y)\varphi_{1,n}(x),
\end{equation*}
и поиске посредством некоторых итерационных процедур первых $N$ неизвестных коэффициентов $c_{1,n}(y)$, $k=1,\ldots,N$. В качестве приближенного решения берется частичная сумма указанного ряда:
\begin{equation}\label{mmgmsr1-y-part-sum}
y(x) \approx y(a)+\sum\limits_{n=0}^{N-1}
c_{1,n+1}(y)\varphi_{1,n+1}(x).
\end{equation}

В рамках НИР была поставлена задача численной реализации указанного метода для некоторых конкретных систем, ортогональных по Соболеву. При этом одно из основных требований заключалось в том, чтобы в расчетах использовались быстрые алгоритмы. Для некоторых конкретных систем данная задача была решена (см. раздел \ref{ch-ODE}).











%Согласно плану научно-исследовательской работы за 2018 год исследования, проводимые в Отделе математики и информатики Дагестанского научного центра РАН, включают в себя работы по теме
%<<Разработка алгоритмов и создание наукоемкого программного обеспечения для моделирования сложных систем. Некоторые вопросы цифровой обработки сигналов и изображений. Исследования по теории графов и теории оптимизации расписаний, компьютерное сопровождение вузовского учебного процесса>>.

%%%%%%%%%%%%%%%%%%%%%%%%
%%%%%%%%%%%%%%%%%%%%%%%%
%%%%%%%%%%%%%%%%%%%%%%%%

%===============================
%===============================

В связи с некоторыми прикладными задачами, такими как вопросы термодинамики потоков жидкости и проблемой перечисления совершенных паросочетаний плоского графа за полиномиальное время, рядом авторов рассматривалась ранее задача перечисления разбиений прямоугольника заданных целочисленных размеров $h\times w$ на прямоугольники $1\times 2$.
За исключением  случая $w=3$, для которого получена система из двух взаимно-рекуррентных формул (Д. Кнут и др.), все известные формулы решения этой задачи используют действия с плавающей запятой. Это сопряжено с проблемами округления.
%
В отчетном году предпринята попытка получения решения указанной задачи, используя лишь операции сложения целых чисел.
%В отчетном году в ОМИ разработан алгоритм, компьютерное воплощение которого способно для искомого перечисления разбиений прямоугольника сгенерировать систему взаимно-рекуррентных формул, использующих лишь операции сложения целых чисел, т.е. свободных от проблем округления вещественных чисел.

%%%%%%%%%%%%%%%%%%%%%%%%
%%%%%%%%%%%%%%%%%%%%%%%%

В исследовании проблемы существования двудольных графов заданного порядка, не допускающих интервальной реберной раскраски, востребован алгоритм перечисления графов по принципу «один представитель из каждого класса изоморфных графов». Ввиду алгоритмических трудностей задачи это условие обычно заменяется более слабым условием: «малое число представителей из каждого класса изоморфных графов». В 2018 г. найдены некоторые способы усиления фильтрации изоморфных графов.

%В разделе \ref{akm2-chap} предложен алгоритм компьютерного формирования тестовых заданий по основам программирования. По каждой теме учебной дисциплины рассмотрены пять формализованных структур тестовых заданий. Подробно изложен способ воплощения формализованных структур тестовых заданий в программное обеспечение.
%Результаты статьи находят применение для автоматической генерации практически неограниченного количества тестовых пунктов для компьютерного тестирования по основам языков программирования.


%Вопросы организации компьютерного тестирования рассмотрены в ряде работ, см., например, \cite{Lapt}. Раздел посвящен описанию алгоритма автоматизации создания тестовых единиц по языку программирования Delphi 7.0.
%
%
%Cформулирован алгоритм построения тестовой единицы в одной из нескольких общепринятых форм и приведены основные фрагменты воплощения алгоритма средствами Delphi 7.0. Подчеркнем, что нам не удалось отыскать в журнальной (и иной) литературе работы, посвященные вопросам автоматизации создания тестовых единиц.




%%%%%%%%%%%%%%%%%%%%%%%%
%%%%%%%%%%%%%%%%%%%%%%%%
%%%%%%%%%%%%%%%%%%%%%%%%



Начатый ранее процесс создания программного обеспечения по компьютерному сопровождению деятельности вузовской кафедры продолжен в направлении построения сводных таблиц учебных нагрузок вузовской кафедры и автоматической генерации тестовых заданий по структурированным данным языков программирования.
Продолжена работа по созданию методической и алгоритмической базы проведения межрегиональных олимпиад по программированию среди вузов. На ее основе проведена дистанционная межрегиональная олимпиада по программированию среди вузов СКФО.
%Кроме того, в отчетном году сотрудники ОМИ ДНЦ РАН и кафедры Дискретной математики ДГУ провели 01.11.2018 г. межрегиональную дистанционную олимпиаду по программированию среди вузов СКФО. Выполнена трудоемкая работа по разработке заданий, тестов и программного обеспечения для олимпиады, результаты которой частично освещены в разделе \ref{akm3-chap}.





%%%%%%%%%%%%%%%%%%%%%%%%
%%%%%%%%%%%%%%%%%%%%%%%%
%%%%%%%%%%%%%%%%%%%%%%%%

%===============================



%Рамазанов М.К.
В настоящее время продолжается активное исследование магнитных состояний, фазовых переходов (ФП), критических и термодинамических свойств в спиновых системах с конкурирующими обменными взаимодействиями. Конкуренция обменного взаимодействия может привести к возникновению в системе эффектов фрустрации. Наличие фрустраций в системе может привести к целому ряду изменений свойств фундаментального характера \cite{ph1_1, ph1_2, ph1_3}. Учет антиферромагнитных взаимодействий следующих ближайших соседей в классической трехмерной модели Изинга приводит к вырождению основного состояния, появлению различных фаз, ФП и аномалий критических и термодинамических свойств \cite{ph1_4}.

В 2018 году нами на основе метода Монте-Карло (МК) проведено исследование ФП, магнитной структуры основного состояния и термодинамических свойств антиферромагнитной модели Изинга на объемно-центрированной кубической решетке для соотношения величины обменного взаимодействия следующих и ближайших соседей $r=J_2/J_1=2/3$ ($J_1$ и $J_2$ --- константы обменного взаимодействия ближайших и следующих за ближайшими соседей, соответственно).

%Бабаев
На основе метода Монте-Карло рассчитаны относительные дисперсии намагниченности  $R_m$ и восприимчивости $R_\chi$  в трехмерной неупорядоченной спиновой решеточной модели Изинга в зависимости от концентрации спинов. Показано, что внесение беспорядка в виде немагнитных примесей в трехмерную модель Изинга приводит к отличным от нуля значениям для $R_m$ и $R_\chi$.


%Магомедов
Модель Поттса на треугольной решетке с учетом взаимодействия как первых, так и вторых ближайших соседей, исследована алгоритмом Ванга -- Ландау метода Монте-Карло. Вычислена плотность состояний системы и рассчитаны температурные зависимости энтропии $S$. Показано, что в зависимости от соотношений обменных взаимодействий между первыми и вторыми ближайшими соседями, основное состояние может быть как сильно вырожденным, что свидетельствует о наличии фрустрации в системе, так и слабо вырожденным.

Определены структуры основного состояния и показано, что в данной системе реализуется страйповое, триплетное или смешанное страйпово-триплетное состояние для фазы $1$, неупорядоченное сильно вырожденное состояние или многослойное слабо вырожденное состояние для фрустрированной фазы и упорядоченное ферромагнитное состояние для ферромагнитной фазы.

В отчетном году показано, что фазовый переход из ферромагнитной и страйповой-триплетной фаз в парамагнитную является фазовым переходом первого рода, в то время как переход из фрустрированной области в парамагнитную является переходом второго рода. Рассчитаны температурные зависимости различных термодинамических параметров. Определены температуры фазовых переходов и построена фазовая диаграмма системы.































%%%%%%%%%%%%%%%%%%%%%%%%
%%%%%%%%%%%%%%%%%%%%%%%%
%%%%%%%%%%%%%%%%%%%%%%%%
%===============================



%Основные результаты, полученные в отчетном году по данной теме касаются разработки алгоритмов и программных пакетов для использования в решении важных прикладных задач, таких как обработка и сжатие временных рядов и изображений, численно-аналитическое решение систем линейных и нелинейных дифференциальных и разностных уравнений, интервальная реберная раскрашиваемость двудольных графов и оптимизация расписаний для мультипроцессорных систем,  программное обеспечение для компьютерного сопровождения процесса распределения учебной нагрузки.
%
%
%В частности, разработаны алгоритмы и компьютерные программы для обработки временных рядов и изображений методом перекрывающих преобразований, основанных на повторных средних типа Валле Пуссена для тригонометрических сумм Фурье, для численно-аналитического решения систем линейных и нелинейных дифференциальных и разностных уравнений спектральными методами, основанными на использовании систем функций, ортогональных по Соболеву и порожденных такими классическими системами как система Хаара, система полиномов Чебышева первого рода и система косинусов.


%Первый результат по данной теме касается разработки алгоритмов и программных пакетов для обработки и сжатия временных рядов и изображений, которые могут быть использованы в решении важных прикладных задач. На основе тригонометрических сумм Фурье $S_n(f,x)$ и классических средних Валле Пуссена
%$
%_1V_{n,m}(f,x)= \frac1n\sum\nolimits_{l=m}^{m+n-1}S_l(f,x)
%$
%учеными ОМИ были введены в рассмотрение повторные средние Валле Пуссена следующим образом
% $
%_2V_{n,m}(f,x)= \frac1n\sum\nolimits_{k=m}^{m+n-1}{}_1V_{n,k}(f,x),
%$
%$
%{}_{l+1}V_{n,m}(f,x)= \frac1n\sum\nolimits_{k=m}^{m+n-1} {}_{l}V_{n,k}(f,x)\quad(l\ge1).
%$
%Сконструированы операторы, которые осуществляют на основе средних $_2V_{n,m}(f,x)$ так называемые перекрывающие преобразования (lapped transform), хорошо зарекомендовавшие себя на практике для обыкновенных сумм Фурье. Были исследованы аппроксимативные свойства этого нового вида преобразований в случае, когда исходный сигнал представляет собой непрерывную (вообще говоря, непериодическую) функцию. Эти операторы легли в основу разработанного алгоритма для приближения сигналов и временных рядов. Алгоритм был реализован в виде пакета прикладных программ, и с его помощью проведен ряд численных экспериментов.




%%%%%%%%%%%%%%%%%%%
%%%%%%%%%%%%%%%%%%%
%%%%%%%%%%%%%%%%%%%



%Еще одно прикладное направление исследований, проводимых в ОМИ в 2017 году, связано с численно-аналитическим решением систем линейных и нелинейных дифференциальных и разностных уравнений  спектральными методами, основанными на использовании систем функций, ортогональных по Соболеву и порожденных такими классическими системами как система Хаара, система полиномов Чебышева первого рода и система косинусов. Указанные методы легли в основу разработанных алгоритмов и пакетов реализующих их компьютерных программ.




%Была разработана общая методика приближенного решения дифференциальных и разностных уравнений с помощью функций, ортогональных по Соболеву, а также конкретные методы, основанные на системах функций, ортогональных по Соболеву и порожденных такими классическими системами как система Хаара, система полиномов Чебышева первого рода и система косинусов.







%%%%%%%%%%%%%%%%%%%
%%%%%%%%%%%%%%%%%%%
%%%%%%%%%%%%%%%%%%%






%%%%%%%%%%%%%%%%%%%
%%%%%%%%%%%%%%%%%%%
%%%%%%%%%%%%%%%%%%%

%Исследования в области теории расписаний приводят к появлению новых методов и даже целых
%направлений в теории графов. Например, знаменитая проблема четырех красок возникла в связи с задачами теории расписаний
%и разбиений \cite[c. 101]{AKM_ch1_bib1}. В свою очередь, с проблемой четырех красок связана задача о правильной вершинной раскраске
%графа в три цвета, т.е. задача о существовании такого отображения множества вершин графа в множество цветов \{1, 2,
%3\}, что концевым вершинам каждого ребра сопоставляются разные цвета. Задача раскраски графа в три цвета
%$\mathit{NP}${}-полна \cite[c. 101]{AKM_ch1_bib1}, следовательно, не может быть разрешима за полиномиальное время, если верна знаменитая
%гипотеза <<$NP\neq P$>> (отметим недавнее сообщение Норберта Блюма о доказательстве данной гипотезы \cite{AKM_ch1_bib2} и
%работу \cite{AKM_ch1_bib3}).
%
%В 2017 году для проверки существования интервальной реберной раскраски связного двудольного графа, а также ее построения в случае существования в ОМИ был сконструирован жадный алгоритм, осуществляющий перебор с возвратами последовательных ребер кусочно-непрерывного пути. Кусочно-непрерывным путем мы называем такой упорядоченный набор всех ребер графа, что подграф, порожденный любым подмножеством его ребер с номерами $1, 2, \ldots,$ является связным.
%Построен алгоритм, позволяющий для множества $S$ всех двудольных графов заданного порядка $H$ сгенерировать множество $S_0$ двудольных графов существенно меньшей мощности, содержащее для каждого графа $G \in S$ граф, изоморфный  $G$. Применением жадного алгоритма к	каждому графу из $S_0$ с привлечением компьютерных ресурсов показано, что при малых значениях $H$ все графы из $S_0$ интервально раскрашиваемы.
%	
%Как следствие установлено, что все двудольные графы $G=(X,Y,E)$ порядка 16 при $|X|<7$ обладают интервальной раскраской. Это, в свою очередь, означает, что достаточно проверить интервальную раскрашиваемость двудольных графов порядка 16 при $|X|=7$ и $X=8$.
%Разработанное на основе упомянутых алгоритмов программное обеспечение обладает также рядом дополнительных свойств, способствующих удобному графическому отображению интервальной раскраски.




%%%%%%%%%%%%%%%%%%%
%%%%%%%%%%%%%%%%%%%
%%%%%%%%%%%%%%%%%%%
%В направлении сопровождения вузовского учебного процесса был рассмотрен вопрос автоматической генерации тестовых заданий по учебным дисциплинам.
%Описан алгоритм компьютерного формирования тестовых заданий по
%			основам программирования на языке Delphi 7.0, алгоритм воплощен в компьютерную программу. По каждой теме учебной дисциплины рассмотрены пять
%			формализованных структур тестовых заданий и одна <<нестандартная>> форма, предусматривающая творческий анализ той или
%			иной нестандартной ситуации.
%%%%%%%%%%%%%%%%%%%
%%%%%%%%%%%%%%%%%%%
%%%%%%%%%%%%%%%%%%%

%В этом же направлении была поставлена задача о создании алгоритмического и программного обеспечения для компьютерного сопровождения процесса распределения учебной нагрузки вузовской кафедры.
%Распределение учебной нагрузки практически никогда не начинается <<с
%чистого листа>>. В качестве исходного, <<чернового>> выступает распределение, унаследованное от предыдущего учебного года.
%С другой стороны, <<черновое>> распределение никогда не может быть принято за <<беловик>> без дополнительной работы над
%ошибками (изменяется количество студентов в учебных группах, ежегодно изменяется состав актуальных учебных дисциплин
%категории <<по выбору>>, в рамках оптимизации учебных нагрузок в связи с финансовыми проблемами вуза могут быть
%скорректированы объемы учебных часов). Таким образом, автоматизация (хотя бы частичная) в подготовке итогового распределения учебных часов  является актуальной и востребованной задачей.

%%%%%%%%%%%%%%%%%%%
%%%%%%%%%%%%%%%%%%%
%%%%%%%%%%%%%%%%%%%

%При разработке тем, связанных с компьютерным сопровождением вузовского учебного процесса, естественным является сочетание научно-исследовательской деятельности с учебно-методической работой (в частности, издание учебных пособий с решениями нестандартных задач) и работой по подготовке студентов к конкурсам и олимпиадам по программированию. Нами был издан набор нестандартных упражнений по 3ds Max с решениями в виде учебного пособия. В направлении 3ds Max команда студентов из учебной группы ФМиКН ДГУ приняла участие во Всероссийском конкурсе, получены два диплома 1 степени – в личном и командном первенстве. Издано учебно-методическое пособие по итогам студенческой олимпиады вузов СКФО по программированию, а также издан (в электронном виде) цикл лекций по программированию.


%%%%%%%%%%%%%%%%%%%
%%%%%%%%%%%%%%%%%%%
%%%%%%%%%%%%%%%%%%%

%Предложен компактный формат исходных данных для построения графа. Предлагаемый способ
%воспроизведения и интерактивного редактирования ориентированных и неориентированных графов способствует достижению
%хорошего полиграфического качества рисунков графов, сопровождающих статьи и учебные пособия по теории графов.
%Этот метод включает визуальное изменение координат вершин на рисунке графа с применением стека изменений
%практически неограниченной глубины, масштабирование всего рисунка и отдельных его элементов (размеров изображения
%вершин, толщины линий и стрелок с сохранением координат вершин), ведение протокола изменений в rtf-формате.
%Результаты находят применение в создании полиграфических документов с рисунками графов.

%%%%%%%%%%%%%%%%%%%
%%%%%%%%%%%%%%%%%%%
%%%%%%%%%%%%%%%%%%%



%%%%%%%%%%%%%%%%%%%
%%%%%%%%%%%%%%%%%%%
%%%%%%%%%%%%%%%%%%%



%%%%%%%%%%%%%%%%%%%
%%%%%%%%%%%%%%%%%%%
%%%%%%%%%%%%%%%%%%%



%%%%%%%%%%%%%%%%%%%
%%%%%%%%%%%%%%%%%%%
%%%%%%%%%%%%%%%%%%%

