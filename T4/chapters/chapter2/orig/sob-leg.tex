%\subsection{Полиномы, порожденные полиномами Якоби и Лежандра}

%\begin{abstract}
%Для  произвольного натурального $r$ рассмотрены полиномы $p^{\alpha,\beta}_{r,k}(x)$ $(k=0,1,\ldots)$, ортонормированные относительно скалярного произведения типа Соболева следующего вида
%$$
%<f,g>=\sum_{\nu=0}^{r-1}f^{(\nu)}(-1)g^{(\nu)}(-1)+
%\int_{-1}^{1}f^{(r)}(t)g^{(r)}(t)(1-t)^\alpha(1+t)^\beta dt
%$$
%и изучены  их свойства. Введены в рассмотрение ряды Фурье по полиномам $p_{r,k}(x)=p^{0,0}_{r,k}(x)$ и некоторые их обобщения, частичные суммы которых  сохраняют некоторые важные  свойства частичных сумм ряда Фурье по полиномам $p_{r,k}(x)$, в том числе и свойство $r$-кратного совпадения (<<прилипания>>) частичных сумм ряда Фурье по полиномам $p_{r,k}(x)$  в  точках $-1$ и $1$ между собой и с исходной функцией $f(x)$.  Основное внимание уделено  исследованию вопросов приближения гладких и аналитических функций  частичными суммами упомянутых обобщений, представляющих собой   специальные ряды  по ультрасферическим полиномам Якоби со свойством <<прилипания>> их частичных сумм  точках $-1$ и $1$ .
%\end{abstract}

%\begin{keywords}
%Ряды Фурье по полиномам, ортогональным по Соболеву, полиномы Лежандра и Якоби, специальные (прилипающие) ряды по ультрасферическим полиномам, аппроксимативные свойства
%\end{keywords}

\section{Полиномы, порожденные  ортонормированными полиномами \\ Якоби $p_{k}^{\alpha,\beta}(x)$}
Из равенства \eqref{Haar-Tcheb-eq5.2} следует, что если $\alpha,\beta>-1$, то полиномы
\begin{equation}\label{sob-leg-3.32}
p_n^{\alpha,\beta}(x)={P_n^{\alpha,\beta}(x)\over\sqrt{ h_n^{\alpha,\beta}}}\quad(n=0,1,\ldots)
\end{equation}
образуют ортонормированную  в $L_\kappa^2(-1,1)$ с весом $\kappa(x)=(1-x)^\alpha(1+x)^\beta$ систему. В частности, если $\alpha=\beta=\frac12$, $x=\cos\theta$, то
$$
 p_n^{\frac12,\frac12}(x)=\sqrt{2/\pi}\frac{\sin(n+1)\theta}{\sin\theta}\quad (n=0,1,\ldots)
 $$
 -- ортонормированные полиномы Чебышева второго рода, а если $\alpha=\beta=-\frac12$, то
$$
p_0^{-\frac12,-\frac12}(x)=\sqrt{1/\pi},\quad  p_n^{-\frac12,-\frac12}(x)=\sqrt{2/\pi}\cos n\theta\quad (n=1,\ldots)
 $$
 -- ортонормированные полиномы Чебышева первого рода.



 Как хорошо известно \cite{sob-leg-Sege}, система полиномов Якоби \eqref{sob-leg-3.32} полна в $L_\kappa^2(-1,1)$.   Она порождает на $[-1,1]$ систему полиномов  $p_{r,k}^{\alpha,\beta}(x)$ $(k=0,1,\ldots)$, определенных равенствами
   \begin{equation}\label{sob-leg-3.33}
p_{r,k}^{\alpha,\beta}(x) =\frac{(x+1)^k}{k!}, \quad k=0,1,\ldots, r-1,
\end{equation}
  \begin{equation}\label{sob-leg-3.34}
p_{r,r+n}^{\alpha,\beta}(x) =\frac{1}{(r-1)!}\int\limits_{-1}^x(x-t)^{r-1}p_n^{\alpha,\beta}(t)dt, \quad n=0,1,\ldots.
\end{equation}
Если мы определим скалярное произведение
\begin{equation}\label{sob-leg-3.35}
<f,g>=\sum_{\nu=0}^{r-1}f^{(\nu)}(-1)g^{(\nu)}(-1)+
\int_{-1}^{1}f^{(r)}(t)g^{(r)}(t)\kappa(t)dt,
\end{equation}
то из теоремы \ref{completeness} непосредственно вытекает
\begin{corollary}\label{completeness-jacobi}
Пусть $\alpha,\beta>-1$. Тогда система полиномов $\{p_{r,k}^{\alpha,\beta}(x)\}_{k=0}^\infty$, порожденная системой ортонормированных полиномов Якоби \eqref{sob-leg-3.32} посредством равенств \eqref{sob-leg-3.33} и \eqref{sob-leg-3.32}, полна  в $W^r_{L^2_\kappa(-1,1)}$ и ортонормирована относительно скалярного произведения \eqref{sob-leg-3.35}.
\end{corollary}
Ряд Фурье для системы   $\{p_{r,k}^{\alpha,\beta}(x)\}_{k=0}^\infty$ приобретает вид
\begin{equation}\label{sob-leg-3.36}
f(x)\sim \sum_{k=0}^{r-1} f^{(k)}(-1)\frac{(x+1)^k}{k!}+ \sum_{k=r}^\infty \hat f_{r,k}p_{r,k}^{\alpha,\beta}(x),
\end{equation}
где
  \begin{equation}\label{sob-leg-3.37}
 \hat f_{r,k}=\int_{-1}^1 f^{(r)}(t)p_{k-r}^{\alpha,\beta}(t)\kappa(t)dt.
\end{equation}

\begin{corollary}
Пусть $-1<\alpha,\beta<1$. Тогда если $f(x)\in W^r_{L^2_\kappa(-1,1)}$, то ряд Фурье (смешанный ряд) \eqref{sob-leg-3.36} сходится к функции $f(x)$ равномерно относительно $x\in[-1,1]$.
\end{corollary}

При исследовании дальнейших (более глубоких) аппроксимативных свойств частичных сумм смешанного ряда \eqref{sob-leg-3.36} возникает задача об асимптотических свойствах полиномов
$p_{r,k}^{\alpha,\beta}(x)$, которая, в свою очередь, приводит к вопросу о получении различных представлений для этих полиномов, отличных от  \eqref{sob-leg-3.34} и не содержащих знаков интеграла с переменным пределом. Прежде всего мы найдем явный вид полиномов $p_{r,k}^{\alpha,\beta}(x)$.

\begin{theorem} Для произвольных $\alpha, \beta>-1$ и $n\ge0$
имеет место следующее равенство
\begin{equation}\label{sob-leg-3.38}%теорема 3
p_{r,n+r}^{\alpha,\beta}(x)=\frac{(-1)^n2^{r}}{\sqrt{ h_n^{\alpha,\beta}}}
{n+\beta\choose n}
\sum_{k=0}^n{(-n)_k(n+\alpha+\beta+1)_k\over (\beta+1)_k(k+r)!}
\left({1+x\over 2}\right)^{k+r}.
\end{equation}
\end{theorem}

Равенство \eqref{sob-leg-3.38}, установленное в данной теореме, может быть использовано при исследовании асимптотических свойств полиномов $p_{r,n+r}^{\alpha,\beta}(x)$ в окрестности точки $x=-1$. Если же $-1+\varepsilon\le x$, то формула  \eqref{sob-leg-3.38} становится непригодной для изучения асимптотического  поведения полиномов  $p_{r,n+r}^{\alpha,\beta}(x)$ при $n\to\infty$, поэтому возникает задача найти иные представления для этих полиномов, которые могли бы быть использованы для исследования их поведения при $n\to\infty$ в том случае, когда точка $x$ не находится в непосредственной близости от $-1$. Мы перейдем теперь к рассмотрению этого вопроса.
\begin{theorem} Пусть $\alpha, \beta>-1$, $\lambda=\alpha+\beta$. Тогда  при условии $(k+\lambda)^{[r]}\neq0$ имеет место следующее равенство
\begin{equation}\label{sob-leg-3.47}
p_{r,r+k}^{\alpha,\beta}(x) ={1\over\sqrt{ h_k^{\alpha,\beta}}}
\frac{2^r}{(k+\lambda)^{[r]}}\left[P_{k+r}^{\alpha-r,\beta-r}(x)
-\sum^{r-1}_{\nu=0}\frac{A_{\nu,k,r}^{\alpha,\beta}}{\nu!}(1+x)^{\nu}\right] ,
\end{equation}
где
\begin{equation}\label{sob-leg-3.45}
A_{\nu,k,r}^{\alpha,\beta}=\left\{P_{k+r}^{\alpha-r,\beta-r}(t)\right\}_{t=-1}^{(\nu)}=
\frac{(-1)^{k+r-\nu}\Gamma(k+\beta+1)(k+\lambda-r+1)_{\nu}}
{\Gamma(\nu-r+\beta+1)(k+r-\nu)!2^\nu}.
\end{equation}
\end{theorem}


 \textit{Замечание 1.} Выражения, аналогичные тем, которые фигурируют в правой части равенства  \eqref{sob-leg-3.47}, впервые появились в   работах \cite{sob-leg-Shar13}, \cite{sob-leg-Shar17}, \cite{sob-leg-Shar18} в связи исследованием задачи об аппроксимативных свойств смешанных рядов \eqref{sob-leg-3.36} по общим полиномам Якоби $P_{k}^{\alpha,\beta}(x)$.

  Рассмотрим некоторые частные случаи.

\subsection{Полиномы, порожденные полиномами Якоби $p_{k}^{\alpha,0}(x)$}


 Пусть $\beta=0$, $\alpha$ -- дробное. Тогда, во-первых $(k+\lambda)^{[r]}\neq0$ для всех $k\ge0$, во-вторых,  из \eqref{sob-leg-3.45} следует, что $A_{\nu,k,r}^{\alpha,0}=0$ при всех  $\nu=0,1,\dots, r-1$, поэтому равенство
\eqref{sob-leg-3.47} можно переписать так
 \begin{equation}\label{sob-leg-3.48}
p_{r,r+k}^{\alpha,0}(x) ={1\over\sqrt{ h_k^{\alpha,0}}}
\frac{2^r}{(k+\alpha)^{[r]}}P_{k+r}^{\alpha-r,-r}(x) \quad (k=0,1,\ldots).
\end{equation}
С учетом свойства \eqref{Haar-Tcheb-eq5.6} этому равенству можно придать также следующий вид
\begin{equation}\label{sob-leg-3.49}
p_{r,r+k}^{\alpha,0}(x) =
\frac{(1+x)^rP_{k}^{\alpha-r,r}(x)}{(k+r)^{[r]}\sqrt{ h_k^{\alpha,0}}},
 \quad (k=0,1,\ldots).
\end{equation}
Соответствующий ряд Фурье (смешанный ряд) \eqref{sob-leg-3.36} принимает в рассматриваемом случае следующий вид
\begin{equation}\label{sob-leg-3.50}
f(x)\sim\sum_{k=0}^{r-1} f^{(k)}(0)\frac{(1+x)^k}{k!}+ \sum_{k=r}^\infty \frac{2^r \hat f_{r,k}P_{k}^{\alpha-r,-r}(x)}{\sqrt{h_{k-r}^{\alpha,0}}(k+\alpha-r)^{[r]}}
\end{equation}
или, что то же,
\begin{equation}\label{sob-leg-3.51}
f(x)\sim\sum_{k=0}^{r-1} f^{(k)}(0)\frac{(1+x)^k}{k!}+ \sum_{k=r}^\infty \frac{ \hat f_{r,k}(1+x)^rP_{k-r}^{\alpha-r,r}(x)}{k^{[r]}\sqrt{h_{k-r}^{\alpha,0}}},
\end{equation}
где
  \begin{equation}\label{sob-leg-3.52}
 \hat f_{r,k}=\frac{1}{\sqrt{ h_{k-r}^{\alpha,0}}}\int\limits_{-1}^1 f^{(r)}(t)P_{k-r}^{\alpha,0}(t)(1-t)^\alpha dt.
\end{equation}


\subsection{ Полиномы, порожденные полиномами Лежандра $p_{n}^{0,0}(x)$}

 Рассмотрим ортогональные по Соболеву полиномы $p_{r,n}(x)=p_{r,n}^{0,0}(x)$, порожденные полиномами \textit{Лежандра}. С этой целью положим $\beta=\alpha=0$. Тогда  $(k+\lambda)^{[r]}\neq0$ для всех $k\ge r$, а из \eqref{sob-leg-3.45} следует, что $A_{\nu,k,r}^{0,0}=0$ при всех  $\nu=0,1,\dots, r-1$. Поэтому
\begin{equation}\label{sob-leg-3.53}
p_{r,r+k}(x) =\sqrt{k+1/2}
\frac{2^r}{k^{[r]}}P_{k+r}^{-r,-r}(x) \quad (k=r,r+1,\ldots).
\end{equation}
Если мы обратимся к равенству \eqref{Haar-Tcheb-eq5.6}, то можем записать
\begin{equation}\label{sob-leg-3.54}
P_{k+r}^{-r,-r}(x)= \frac{(-1)^r(1-x^2)^r}{2^{2r}}P_{k-r}^{r,r}(x) \quad (k=r,r+1,\ldots).
\end{equation}
Из \eqref{sob-leg-3.53} и \eqref{sob-leg-3.54} имеем
\begin{equation}\label{sob-leg-3.55}
p_{r,r+k}(x) =
\frac{(-1)^r}{2^rk^{[r]}}\sqrt{k+1/2}(1-x^2)^rP_{k-r}^{r,r}(x) \quad (k=r,r+1,\ldots).
\end{equation}
Соответствующий этому случаю ряд Фурье \eqref{sob-leg-3.36} по полиномам $p_{r,k}(x)=p_{r,k}^{0,0}(x)$, ортогональным по Соболеву (или, что то же, \textit{смешанный ряд по полиномам  Лежандра}), приобретает вид
$$
f(x)\sim \sum_{k=0}^{r-1} f^{(k)}(-1)\frac{(x+1)^k}{k!}+\sum_{k=r}^\infty \hat f_{r,k}p_{r,k}(x)=
$$
$$
 \sum_{k=0}^{r-1} f^{(k)}(-1)\frac{(x+1)^k}{k!}+\sum_{k=r}^{2r-1} \hat f_{r,k}p_{r,k}(x)
$$
$$
+\frac{(-1)^r(1-x^2)^r}{2^r}\sum_{k=2r}^\infty \frac{\hat f_{r,k}\sqrt{k-r+1/2}P_{k-2r}^{r,r}(x)}{(k-r)^{[r]}}=
$$
$$
 \sum_{k=0}^{r-1} f^{(k)}(-1)\frac{(x+1)^k}{k!}+\sum_{k=r}^{2r-1} \hat f_{r,k}p_{r,k}(x)+
$$
\begin{equation}\label{sob-leg-3.56}
 (1-x^2)^r\sum_{k=0}^\infty\frac{(-1)^r\hat f_{r,k+2r}}{2^r} \frac{\sqrt{(k+r+1/2)h_k^{r,r}}}{ (k+r)^{[r]}}p_{k}^{r,r}(x)
\end{equation}
В связи с представлением \eqref{sob-leg-3.56} отметим, что  $p_{r,k}(x)$ в силу теоремы 3 могут быть  выражены формулой
\begin{equation}\label{sob-leg-5.57}
p_{r,n+r}(x)=(-1)^n2^{r}\sqrt{n+1/2}
\sum_{k=0}^n{(-n)_k(n+1)_k\over k!(k+r)!}
\left({1+x\over 2}\right)^{k+r}.
\end{equation}
В частности, для  $r=1$ и $r=2$ имеем:
\begin{equation}\label{sob-leg-3.58}
 p_{1,1}(x)=\frac{1}{\sqrt{2}}(1+x)\quad (r=1);
\end{equation}
\begin{equation}\label{sob-leg-3.59}
 p_{2,2}(x)=\frac{1}{2\sqrt{2}}(1+x)^2,\quad p_{2,3}(x)=\frac{1}{2\sqrt{6}}(1+x)^2(x-2)\quad(r=2).
\end{equation}
Аппроксимативные свойства частичных сумм ряда \eqref{sob-leg-3.56} вида
$$
\mathcal{Y}_{r,N}(f,x)=\sum_{k=0}^{r-1} f^{(k)}(-1)\frac{(x+1)^k}{k!}+\sum_{k=r}^N \hat f_{r,k}p_{r,k}(x)=
$$
\begin{equation}\label{sob-leg-3.60}
\mathcal{Y}_{r,2r-1}(f,x)+(1-x^2)^r\sum_{k=0}^{N-2r}\frac{(-1)^r\hat f_{r,k+2r}}{2^r} \frac{\sqrt{(k+r+1/2)h_k^{r,r}}}{ (k+r)^{[r]}}p_{k}^{r,r}(x)
\end{equation}
 были весьма подробно исследованы в работах \cite{sob-leg-Shar11}, \cite{sob-leg-Shar12} -- \cite{sob-leg-Shar18}, \cite{sob-leg-SHII}. Мы напомним здесь некоторые из них. Прежде всего отметим, что оператор  $f\to \mathcal{Y}_{r,n}(f)$ представляет собой проектор на подпространство алгебраических полиномов $p_n$ степени не выше $n$, т.е. $\mathcal{Y}_{r,n}(p_n)=p_n$. С другой стороны, если $f(x)\in W^r_{L^2(-1,1)}$, то в силу теоремы 2 (см. также следствие 2) имеет место равенство
$$
f(x)=\sum_{k=0}^{r-1} f^{(k)}(-1)\frac{(x+1)^k}{k!}+\sum_{k=r}^\infty \hat f_{r,k}p_{r,k}(x)=
$$
$$
 \sum_{k=0}^{r-1} f^{(k)}(-1)\frac{(x+1)^k}{k!}+\sum_{k=r}^{2r-1} \hat f_{r,k}p_{r,k}(x)+
$$
\begin{equation}\label{sob-leg-3.61}
 (1-x^2)^r\sum_{k=0}^\infty\frac{(-1)^r\hat f_{r,k+2r}}{2^r} \frac{\sqrt{(k+r+1/2)h_k^{r,r}}}{ (k+r)^{[r]}}p_{k}^{r,r}(x),
\end{equation}
 в котором ряд, фигурирующий в правой части, сходится равномерно на $[-1,1]$. Отсюда, в свою очередь, следует, что   $\mathcal{Y}_{r,n}(f,x)$ при $n\ge2r-1 $ совпадает с функцией $f(x)\in W^r_{L^2_\kappa(-1,1)}$ $r$-кратно в точках $-1$ и $1$, т.е. $f^{(\nu)}(\pm1)=\mathcal{Y}_{r,n}^{(\nu)}(f,\pm1),\quad \nu=0,1,\ldots, r-1$. Стало быть, в частном случае, когда $n=2r-1$,
   \begin{equation}\label{sob-leg-3.62}
D_{2r-1}(x)=\mathcal{Y}_{r,2r-1}(f,x)=\sum_{k=0}^{r-1} f^{(k)}(-1)\frac{(x+1)^k}{k!}+\sum_{k=r}^{2r-1} \hat f_{r,k}p_{r,k}(x)
\end{equation}
представляет собой \cite{sob-leg-Shar17} интерполяционный полином Эрмита степени $2r-1$.
В работе \cite{sob-leg-Shar15}  была доказана следующая неулучшаемая по порядку (при $N\to\infty$) оценка
\begin{equation}\label{sob-leg-3.63}
\sup_{f\in W^r}\max_{-1\le x\le 1}{\left|f^{(\nu)}(x)-\left(\mathcal{Y}_{r,N}(f,x)\right)^{(\nu)}\right|\over(1-x^2)^{(r-\nu)/2-1/4}}
\le c(r)\frac{\ln N}{N^{r-\nu}},\,\, 0\le \nu\le r-1,
\end{equation}
 а в работе \cite{sob-leg-Shar16}, используя  обозначения, отличные от принятых в настоящей  статье,   для средних типа Валле Пуссена вида
\begin{equation}\label{sob-leg-3.64}
\mathcal{V }_{r,n+2r,m}(f,x)=\frac{1}{m+1}\sum_{k=n}^{n+m}\mathcal{Y}_{r,k+2r}(f,x)
\end{equation}
 при $0<a\le m/n\le b$ доказано, что если $f\in W^{r}$, $N=n+2r+m$, то
  \begin{equation}\label{sob-leg-3.65}
   {\left|f^{(\nu)}(x)-\left(\mathcal{V }_{r,n+2r,m}(f,x)\right)^{(\nu)}\right|\over(1-x^2)^{(r-\nu)/2}}
\le\frac{ c(r,a,b)}{N^{r-\nu}}\omega(f^{(r)},\frac{1}{N}),\,\, 0\le \nu\le r-1,
\end{equation}
где  $W^r$ -- класс функций, непрерывно дифференцируемых $r$-раз, для которых $\max_{-1\le x\le 1}|f^{(r)}(x)|\le1$, $\omega(g,\delta)$ -- модуль непрерывности функции $g\in C[-1,1]$.

Доказательство оценок \eqref{sob-leg-3.55}  и \eqref{sob-leg-3.56} основано \cite{sob-leg-Shar15}, \cite{sob-leg-Shar17}
на неравенствах типа Лебега для $\mathcal{Y}_{r,n+2r}(f,x)$  и  $\mathcal{V }_{r,n+2r,m}(f,x)$, которые имеют следующий вид
$$
{\left|f^{(\nu)}(x)-\left(\mathcal{Y}_{r,n+2r}(f,x)\right)^{(\nu)}\right|
\over(1-x^2)^{\frac{r-\nu}{2}-\frac14}}\le
$$
  \begin{equation}\label{sob-leg-3.66}
    ((1-x^2)^\frac14+(1-x^2)^{\frac{r-\nu}{2}+\frac14}l^{r-\nu}_{n+\nu}(x))
E^{r-\nu}_{n+2r-\nu}(f^{(\nu)}),
\end{equation}


  \begin{equation}\label{sob-leg-3.67}
   {\left|f^{(\nu)}(x)-\left(\mathcal{V }_{r,n+2r,m}(f,x)\right)^{(\nu)}\right|\over(1-x^2)^\frac{r-\nu}{2}}
\le  (1+(1-x^2)^\frac{r-\nu}{2}J^{r-\nu}_{n+\nu,m}(x))
E^{r-\nu}_{n+2r-\nu}(f^{(\nu)}),
\end{equation}
где
\begin{equation}\label{sob-leg-3.68}
I^{d}_{l}(x)= \int_{-1}^{1}|K^{d,d}_{l}(x,t)|(1-t^2)^{\frac{d}{2}}dt,
\end{equation}
\begin{equation}\label{sob-leg-3.69}
J^{d}_{l,m}(x)= \int_{-1}^{1}|V_{d,l,m}(x,t)|(1-t^2)^{\frac{d}{2}}dt,
\end{equation}
\begin{equation}\label{sob-leg-3.70}
V_{d,l,m}(x,t)=\frac{1}{m+1}\sum_{i=l}^{l+m}K^{d,d}_i(x,t),
\end{equation}
а величина
\begin{equation}\label{sob-leg-3.71}
E_s^d(f)=\inf_{p_s}\sup_{-1<x<1}{|f(x)-p_s(x)|\over (1-x^2)^\frac{d}{2}}
\end{equation}
представляет собой наилучшее (весовое) приближение функции $f\in W^d$ алгебраическими полиномами $p_s(x)$ степени
$s$, обладающими свойством $p_s^{(\nu)}(\pm1)=f^{(\nu)}(\pm1)$ $(\nu=0,\ldots, d-1)$. Для величин, определенных равенствами \eqref{sob-leg-3.68} -- \eqref{sob-leg-3.71}, в \cite{sob-leg-Shar15}, \cite{sob-leg-Shar17} получены следующие оценки
\begin{equation}\label{sob-leg-3.72}
I^d_n(x)\le c(d)(1-x^2)^{-\frac{d}{2}}\left[\ln(n\sqrt{1-x^2}+1)+(1-x^2)^{-\frac14}\right]\quad(-1<x<1),
\end{equation}
\begin{equation}\label{sob-leg-3.73}
J^d_{n,m}(x)\le c(d,a,b)(1-x^2)^{-\frac{d}{2}}\quad (-1<x<1,\, 0< a<n/m\le b),
\end{equation}
а из известной теоремы Теляковского -- Гопенгауза \cite{sob-leg-TEL},\cite{sob-leg-GOP} следует, что
\begin{equation}\label{sob-leg-3.74}
E_N^d(f)\le c(d)N^{-d}\omega(f^{(d)},\frac{1}{N})\quad (f\in W^d).
\end{equation}

Отдельно рассмотрим частный случай неравенства \eqref{sob-leg-3.66}, соответствующий выбору $\nu=0$. В этом случае  учитывая оценку \eqref{sob-leg-3.72} и полагая $N=n+2r$, мы получаем следующее неравенство типа Лебега
\begin{equation}\label{sob-leg-3.75}
{\left|f(x)-\mathcal{Y}_{r,N}(f,x)\right|
\over(1-x^2)^{\frac{r}{2}-\frac14}}\le c(r)\left[(1-x^2)^{\frac14}\ln(N\sqrt{1-x^2}+1)+1\right]E^{r}_{N}(f).
  \end{equation}

