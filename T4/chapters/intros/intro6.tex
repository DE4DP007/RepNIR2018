В вопросах обработки и сжатия сигналов (функций) широко применяются ряды Фурье по различным ортогональным системам функций. При проведении компьютерных экспериментов нами было обнаружено, что для приближения и сжатия сигналов, имеющих кусочно гладкую природу, более эффективным является использование так называемых средних Валле Пуссена (напомним, средние Валле Пуссена -- это один из методов суммирования рядов Фурье). Данный вывод был подтвержден теоретически авторами настоящего отчета: было доказано, что средние Валле Пуссена для классов кусочно гладких функций дают на порядок более высокую скорость приближения к исходной функции, чем сами ряды Фурье. Естественным развитием этой идеи является переход к повторным средним Валле Пуссена. В отчетном году были получены теоретические результаты, в которых показано, что применение повторного усреднения увеличивает скорость сходимости для определенного класса кусочно гладких функций. Это свойство делает повторные средние Валле Пуссена весьма привлекательным инструментом решения важных прикладных задач таких, например, как конструирование цифровых фильтров, обработка и сжатие речи и т.д. В отчетном году разработан также численный алгоритм эффективного вычисления повторных средних Валле Пуссена. На основе данного алгоритма составлена компьютерная программа и проведены численные эксперименты.


В практике организации вузовского учебного процесса возникают прикладные проблемы, решение которых восходит к классическим задачам дискретной математики и кибернетики, в частности – к задачам теории графов и теории алгоритмов. Так, задачи существования и оптимизации расписаний суть задачи раскраски графов и поиска подклассов NP-полной задачи интервальной реберной раскраски, разрешимых за полиномиальное время. В свою очередь, результаты, полученные в области раскраски графов, находят применение не только в теории и практике расписаний, но и в задачах распределения регистров при компиляции, в задачах распределения частот и др.
\par\smallskip
В разделе \ref{akm1} исследована задача составления <<безоконного>> расписания минимальной длительности, получены достаточные условия разбиения множества ребер графа, задающего исходные данные к расписанию, на паросочетания, образующие расписание требуемого вида.

В разделе \ref{akm2} разработано алгоритмическое обеспечение для построения однопроцессорного расписания с условиями частичного предшествования и мультипроцессорного расписания без простоев и без условий частичного предшествования.

Раздел \ref{akm3} посвящен воплощению в компьютерные программы основных алгоритмов, разработанных в 2016 г. сотрудниками ОМИ для задач, рассмотренных в первых разделах.


Получены явные формулы, восстанавливающие гладкое симметричное тензорное поле в трехмерном евклидовом пространстве в случае, когда заданы линейные интегралы от поля вдоль прямых, направление которых определяется точками гладкой кривой, удовлетворяющей некоторому условию полноты. При более слабых требованиях на кривую, получена формула, восстанавливающая значения поля в точках прямолинейного отрезка, соединяющего концы кривой. В случае, когда источники могут лежать в области восстановления, получены формулы обращения веерных преобразований скалярного и тензорного полей с весовыми функциями по данным на многообразии лучей с конечным числом направлений.

Разработаны два пакета прикладных программ для обработки временных рядов и дискретных функций: с помощью конечных предельных рядов и их усреднений типа Валле Пуссена, а также с помощью специальных вейвлет-рядов на основе полиномов Чебышева второго рода, обладающих так называемым свойством <<прилипания>>. Проведены эксперименты на дискретных функциях различной природы.

Рассмотрены вопросы численного-аналитического решения задачи Коши для линейного обыкновенного дифференциального уравнения
\begin{equation}\label{intro-ode}
 a_r(x)y^{(r)}(x)+a_{r-1}(x)y^{(r-1)}(x)+\cdots+a_0(x)y(x)=h(x)
 \end{equation}
с начальными условиями $y^{(k)}(0)=y_k$, $k=0,1,\ldots,r-1$.  Как известно, наряду с различными сеточными методами для решения этой задачи часто применяют так называемые спектральные методы. Спектральный метод основан на разложении искомого решения $y(x)$ в ряд Фурье по некоторой ортонормированной системе. Получающиеся при этом неизвестные коэффициенты Фурье $\hat{y}_k$ функции $y(x)$ находят с помощью подстановки упомянутого разложения в дифференциальное уравнение \eqref{intro-ode} и решения получившегося уравнения относительно неизвестных коэффициентов $\hat{y}_k$. Основная сложность при таком подходе заключается в том, чтобы учесть начальные условия: найденное разложение должно удовлетворять начальным условиям задачи Коши. В отчетном году были разработаны эффективные алгоритмы решения задачи Коши, основанные на применении рядов Фурье по специально сконструированным системам функций, ортогональным относительно скалярного произведения типа Соболева. Специальная конструкция рядов Фурье позволяет автоматически учитывать начальные условия, причем эти условия оказываются выполненными не только для целого ряда, но и для всех его частичных сумм. В частных случаях с помощью разработанного подхода решение задачи Коши удается свести к решению ленточной системы алгебраических уравнений.

