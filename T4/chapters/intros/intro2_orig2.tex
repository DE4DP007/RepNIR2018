В последние годы интенсивное развитие получила (см.\cite{Haar-Tcheb-IserKoch}--\cite{Haar-Tcheb-MarcelXu} и цитированную там литературу) теория полиномов, ортогональных относительно различных скалярных произведений соболевского типа (полиномы, ортогональные по Соболеву).
Были исследованы различные свойства полиномов, ортогональных по Соболеву, включая некоторые вопросы, связанные с асимптотикой таких полиномов. В частности, в работе \cite{Haar-Tcheb-Lopez1995}, существенно используя технику, предложенную А.А.~Гончаром \cite{Haar-Tcheb-Gonchar1975}, была рассмотрена задача сравнительной асимптотики полиномов, ортогональных по Соболеву.
Скалярные произведения соболевского типа характеризуются тем, что они включают в себя слагаемые, которые <<контролируют>> поведение соответствующих ортогональных полиномов на границе области ортогональности. Например, в некоторых случаях оказывается так, что полиномы, ортогональные по Соболеву на отрезке $[a,b]$, могут иметь нули, совпадающие с одним или с обоими концами этого отрезка. Это обстоятельство имеет важное значение для некоторых приложений, в которых требуется, чтобы значения  частичных сумм ряда Фурье функции $f(x)$ по рассматриваемой системе ортогональных полиномов совпали в концах отрезка $[a,b]$ со значениями $f(a)$ и $f(b)$. Заметим, что обычные ортогональные с положительным на  $[a,b]$ весом полиномы этим важным свойством не обладают. \cite{Haar-Tcheb-Lopez1995, Haar-Tcheb-Gonchar1975}.

С другой стороны, отметим, что в работах  \cite{Haar-Tcheb-Shar11} -- \cite{Haar-Tcheb-Shar18}  были введены так называемые смешанные ряды по классическим ортогональным полиномам, частичные суммы которых также обладают свойством совпадения их значений в концах области ортогональности со значениями исходной функции.  В \cite{Haar-Tcheb-Shar11} -- \cite{Haar-Tcheb-Shar18} были подробно исследованы аппроксимативные свойства смешанных рядов для функций из различных функциональных пространств и классов. В частности, было показано, что частичные суммы смешанных рядов по классическим ортогональным полиномам, в отличие от сумм Фурье по этим же полиномам, успешно могут быть использованы в задачах, в которых требуется одновременно приближать дифференцируемую функцию и ее несколько производных. Кроме того, отметим, что в тех случаях, когда для целого $r\ge1$ классические полиномы Якоби $P_n^{\alpha-r,\beta-r}(x)$ и Лагерра $L_n^{\alpha-r}(x)$ образуют ортогональные системы в смысле Соболева, ряды Фурье по этим системам являются частным случаем смешанных рядов по соответствующим полиномам Якоби и Лагерра. Это, в свою очередь,  позволяет  применить к исследованию аппроксимативных свойств рядов Фурье по полиномам Якоби -- Соболева методы и подходы, разработанные нами ранее для решения аналогичной задачи для смешанных рядов по классическим ортогональным полиномам. При этом отметим, что в  работах
\cite{Haar-Tcheb-Shar11} -- \cite{sob-jac-discrete-Shar17} основное внимание уделялось исследованию аппроксимативных свойств смешанных рядов по ультрасферическим полиномам Якоби  $P_n^{\alpha,\alpha}(x)$, тогда как в работе \cite{Haar-Tcheb-Shar18} были найдены условия на параметры $\alpha$ и $\beta$, которые обеспечивают равномерную сходимость смешанных рядов по общим полиномам Якоби $P_n^{\alpha,\beta}(x)$.

В настоящей работе  исследованы  аппроксимативные свойства смешанных рядов по различным классическим непрерывным и дискретным ортогональным системам, и поскольку, как это уже отмечалось, ряды Фурье по полиномам, ортогональным по Соболеву, являются частным случаем смешанных рядов, то попутно также были исследованы  аппроксимативные свойства их частичных сумм.

При исследовании вопросов сходимости рассматриваемых рядов непосредственно возникают, как это всегда и бывает в теории ортогональных полиномов, вопросы о поведении самих полиномов. С вычислительной точки зрения важно также знание таких свойств этих полиномов, как рекуррентные соотношения, которыми связаны эти полиномы между собой, явное их представление посредством элементарных или классических специальных функций. В отчетном году были получены ответы на эти вопросы.

В \cite{Haar-Tcheb-Shar11} -- \cite{Haar-Tcheb-Shar18} было показано, что частичные суммы смешанных рядов по классическим ортогональным полиномам, в отличие от сумм Фурье по этим же полиномам, успешно могут быть использованы в задачах, в которых требуется одновременно приближать дифференцируемую функцию и ее несколько производных. В качестве примера в отчетном году была рассмотрена задача Коши  для линейного дифференциального уравнения (ОДУ)
\begin{equation}\label{intro-ode}
 a_r(x)y^{(r)}(x)+a_{r-1}(x)y^{(r-1)}(x)+\cdots+a_0(x)y(x)=h(x)
 \end{equation}
с начальными условиями $y^{(k)}(-1)=y_k$, $k=0,1,\ldots,r-1$, и для ее решения предложен спектральный метод, основанный на применении рядов Фурье по полиномам, ортогональным в смысле Соболева. Аналогичная задача рассмотрена и в дискретном случае, где вместо дифференциального уравнения \eqref{intro-ode} фигурирует разностное уравнение.

Для непрерывных функций построены рациональные интерполянты и на их базе -- интерполяционные рациональные сплайны. Последовательности сплайнов по двухточечным и трехточечным интерполянтам для любой последовательности сеток с диаметром, стремящимся к нулю, равномерно сходятся к самой функции. В случае трехточечных интерполянтов этим свойством безусловной сходимости обладают первые производные сплайнов, а в случае четырехточечных – первые и вторые производные сплайнов. Получены также оценки скорости сходимости сплайнов и, соответственно, производных сплайнов.

Получена оценка скорости сходимости интерполяционных сплайнов на базе
трехточечных рациональных интерполянтов для произвольной непрерывной на
отрезке функции через модуль непрерывности индуцированной функции.

Построены двухточечные, трехточечные и четырехточечные рациональные интерполянты и на их базе --
интерполяционные сплайны.
Даны оценки скорости сходимости таких сплайнов и их производных.

Показано, что для произвольной непрерывной на данном отрезке функции  скорость равномерной
сходимости к ней    рациональных сплайнов на базе трехточечных интерполянтов можно оценить
через модуль непрерывности индуцированной функции.



%Вопросы сходимости полиномиальных сплайнов и их производных для классических
%функциональных пространств исследованы в достаточно полной форме (см., например, \cite{ark-1}--\cite{ark-6} и цитированную в них литературу).


%Изучены также некоторые аппроксимативные свойства рациональных сплайнов специальных видов при
%дополнительных ограничениях типа монотонности, выпуклости и др. (см., например, \cite{ark-7}--\cite{ark-10} и цитированные в них источники).
%
%Нами построены двухточечные, трехточечные и четырехточечные рациональные интерполянты и на их основе
%построены интерполяционные рациональные сплайны. Как показано (\cite{ark-11}--\cite{ark-13}), аппроксимативные свойства сплайнов по
%двухточечным интерполянтам аналогичны свойствам полиномиальных сплайнов первой степени, а аппроксимативные свойства
%по четырехточечным интерполянтам аналогичны свойствам кубических сплайнов. Что же касается сплайнов по трехточечным
%рациональным интерполянтам, они обладают свойством, которое не наблюдается в случае гладких полиномиальных сплайнов,
%а именно, последовательности таких рациональных сплайнов и их производных для любой последовательности сеток
%узлов с диаметром, стремящимся к нулю, равномерно сходятся соответственно к самой функции в случае произвольной непрерывной
%на данном отрезке функции и к производной функции в случае произвольной непрерывно дифференцируемой функции.
%
%Даны оценки скорости сходимости рациональных сплайнов всех трех рассматриваемых видов к данной функции,
% а производных сплайнов -- к соответствующим производным
%функции через модули непрерывности функции и соответствующей производной функции.
%
%В случае рациональных сплайнов по трехточечным интерполянтам получена также оценка скорости их сходимости
%к данной непрерывной функции через модуль непрерывности индуцированной функции.


