
%Основные результаты, полученные в отчетном году по данной теме касаются разработки алгоритмов и программных пакетов для использования в решении важных прикладных задач, таких как обработка и сжатие временных рядов и изображений, численно-аналитическое решение систем линейных и нелинейных дифференциальных и разностных уравнений, интервальная реберная раскрашиваемость двудольных графов и оптимизация расписаний для мультипроцессорных систем,  программное обеспечение для компьютерного сопровождения процесса распределения учебной нагрузки.
%
%
%В частности, разработаны алгоритмы и компьютерные программы для обработки временных рядов и изображений методом перекрывающих преобразований, основанных на повторных средних типа Валле Пуссена для тригонометрических сумм Фурье, для численно-аналитического решения систем линейных и нелинейных дифференциальных и разностных уравнений спектральными методами, основанными на использовании систем функций, ортогональных по Соболеву и порожденных такими классическими системами как система Хаара, система полиномов Чебышева первого рода и система косинусов.


Первый результат по данной теме касается разработки алгоритмов и программных пакетов для обработки и сжатия временных рядов и изображений, которые могут быть использованы в решении важных прикладных задач. На основе тригонометрических сумм Фурье $S_n(f,x)$ и классических средних Валле Пуссена
$
_1V_{n,m}(f,x)= \frac1n\sum\nolimits_{l=m}^{m+n-1}S_l(f,x)
$
учеными ОМИ были введены в рассмотрение повторные средние Валле Пуссена следующим образом
 $
_2V_{n,m}(f,x)= \frac1n\sum\nolimits_{k=m}^{m+n-1}{}_1V_{n,k}(f,x),
$
$
{}_{l+1}V_{n,m}(f,x)= \frac1n\sum\nolimits_{k=m}^{m+n-1} {}_{l}V_{n,k}(f,x)\quad(l\ge1).
$
Сконструированы операторы, которые осуществляют на основе средних $_2V_{n,m}(f,x)$ так называемые перекрывающие преобразования (lapped transform), хорошо зарекомендовавшие себя на практике для обыкновенных сумм Фурье. Были исследованы аппроксимативные свойства этого нового вида преобразований в случае, когда исходный сигнал представляет собой непрерывную (вообще говоря, непериодическую) функцию. Эти операторы легли в основу разработанного алгоритма для приближения сигналов и временных рядов. Алгоритм был реализован в виде пакета прикладных программ, и с его помощью проведен ряд численных экспериментов.




%%%%%%%%%%%%%%%%%%%
%%%%%%%%%%%%%%%%%%%
%%%%%%%%%%%%%%%%%%%



Еще одно прикладное направление исследований, проводимых в ОМИ в 2017 году, связано с численно-аналитическим решением систем линейных и нелинейных дифференциальных и разностных уравнений  спектральными методами, основанными на использовании систем функций, ортогональных по Соболеву и порожденных такими классическими системами как система Хаара, система полиномов Чебышева первого рода и система косинусов. Указанные методы легли в основу разработанных алгоритмов и пакетов реализующих их компьютерных программ.




%Была разработана общая методика приближенного решения дифференциальных и разностных уравнений с помощью функций, ортогональных по Соболеву, а также конкретные методы, основанные на системах функций, ортогональных по Соболеву и порожденных такими классическими системами как система Хаара, система полиномов Чебышева первого рода и система косинусов. 







%%%%%%%%%%%%%%%%%%%
%%%%%%%%%%%%%%%%%%%
%%%%%%%%%%%%%%%%%%%






%%%%%%%%%%%%%%%%%%%
%%%%%%%%%%%%%%%%%%%
%%%%%%%%%%%%%%%%%%%

Исследования в области теории расписаний приводят к появлению новых методов и даже целых
направлений в теории графов. Например, знаменитая проблема четырех красок возникла в связи с задачами теории расписаний
и разбиений \cite[c. 101]{AKM_ch1_bib1}. В свою очередь, с проблемой четырех красок связана задача о правильной вершинной раскраске
графа в три цвета, т.е. задача о существовании такого отображения множества вершин графа в множество цветов \{1, 2,
3\}, что концевым вершинам каждого ребра сопоставляются разные цвета. Задача раскраски графа в три цвета
$\mathit{NP}${}-полна \cite[c. 101]{AKM_ch1_bib1}, следовательно, не может быть разрешима за полиномиальное время, если верна знаменитая
гипотеза <<$NP\neq P$>> (отметим недавнее сообщение Норберта Блюма о доказательстве данной гипотезы \cite{AKM_ch1_bib2} и
работу \cite{AKM_ch1_bib3}).

В 2017 году для проверки существования интервальной реберной раскраски связного двудольного графа, а также ее построения в случае существования в ОМИ был сконструирован жадный алгоритм, осуществляющий перебор с возвратами последовательных ребер кусочно-непрерывного пути. Кусочно-непрерывным путем мы называем такой упорядоченный набор всех ребер графа, что подграф, порожденный любым подмножеством его ребер с номерами $1, 2, \ldots,$ является связным.
Построен алгоритм, позволяющий для множества $S$ всех двудольных графов заданного порядка $H$ сгенерировать множество $S_0$ двудольных графов существенно меньшей мощности, содержащее для каждого графа $G \in S$ граф, изоморфный  $G$. Применением жадного алгоритма к	каждому графу из $S_0$ с привлечением компьютерных ресурсов показано, что при малых значениях $H$ все графы из $S_0$ интервально раскрашиваемы.
	
Как следствие установлено, что все двудольные графы $G=(X,Y,E)$ порядка 16 при $|X|<7$ обладают интервальной раскраской. Это, в свою очередь, означает, что достаточно проверить интервальную раскрашиваемость двудольных графов порядка 16 при $|X|=7$ и $X=8$.
Разработанное на основе упомянутых алгоритмов программное обеспечение обладает также рядом дополнительных свойств, способствующих удобному графическому отображению интервальной раскраски.




%%%%%%%%%%%%%%%%%%%
%%%%%%%%%%%%%%%%%%%
%%%%%%%%%%%%%%%%%%%
В направлении сопровождения вузовского учебного процесса был рассмотрен вопрос автоматической генерации тестовых заданий по учебным дисциплинам.
Описан алгоритм компьютерного формирования тестовых заданий по
			основам программирования на языке Delphi 7.0, алгоритм воплощен в компьютерную программу. По каждой теме учебной дисциплины рассмотрены пять
			формализованных структур тестовых заданий и одна <<нестандартная>> форма, предусматривающая творческий анализ той или
			иной нестандартной ситуации.
%%%%%%%%%%%%%%%%%%%
%%%%%%%%%%%%%%%%%%%
%%%%%%%%%%%%%%%%%%%

В этом же направлении была поставлена задача о создании алгоритмического и программного обеспечения для компьютерного сопровождения процесса распределения учебной нагрузки вузовской кафедры. 
Распределение учебной нагрузки практически никогда не начинается <<с
чистого листа>>. В качестве исходного, <<чернового>> выступает распределение, унаследованное от предыдущего учебного года.
С другой стороны, <<черновое>> распределение никогда не может быть принято за <<беловик>> без дополнительной работы над
ошибками (изменяется количество студентов в учебных группах, ежегодно изменяется состав актуальных учебных дисциплин
категории <<по выбору>>, в рамках оптимизации учебных нагрузок в связи с финансовыми проблемами вуза могут быть
скорректированы объемы учебных часов). Таким образом, автоматизация (хотя бы частичная) в подготовке итогового распределения учебных часов  является актуальной и востребованной задачей.

%%%%%%%%%%%%%%%%%%%
%%%%%%%%%%%%%%%%%%%
%%%%%%%%%%%%%%%%%%%

При разработке тем, связанных с компьютерным сопровождением вузовского учебного процесса, естественным является сочетание научно-исследовательской деятельности с учебно-методической работой (в частности, издание учебных пособий с решениями нестандартных задач) и работой по подготовке студентов к конкурсам и олимпиадам по программированию. Нами был издан набор нестандартных упражнений по 3ds Max с решениями в виде учебного пособия. В направлении 3ds Max команда студентов из учебной группы ФМиКН ДГУ приняла участие во Всероссийском конкурсе, получены два диплома 1 степени – в личном и командном первенстве. Издано учебно-методическое пособие по итогам студенческой олимпиады вузов СКФО по программированию, а также издан (в электронном виде) цикл лекций по программированию.


%%%%%%%%%%%%%%%%%%%
%%%%%%%%%%%%%%%%%%%
%%%%%%%%%%%%%%%%%%%

Предложен компактный формат исходных данных для построения графа. Предлагаемый способ
воспроизведения и интерактивного редактирования ориентированных и неориентированных графов способствует достижению
хорошего полиграфического качества рисунков графов, сопровождающих статьи и учебные пособия по теории графов.
Этот метод включает визуальное изменение координат вершин на рисунке графа с применением стека изменений
практически неограниченной глубины, масштабирование всего рисунка и отдельных его элементов (размеров изображения
вершин, толщины линий и стрелок с сохранением координат вершин), ведение протокола изменений в rtf-формате.
Результаты находят применение в создании полиграфических документов с рисунками графов.

%%%%%%%%%%%%%%%%%%%
%%%%%%%%%%%%%%%%%%%
%%%%%%%%%%%%%%%%%%%



%%%%%%%%%%%%%%%%%%%
%%%%%%%%%%%%%%%%%%%
%%%%%%%%%%%%%%%%%%%



%%%%%%%%%%%%%%%%%%%
%%%%%%%%%%%%%%%%%%%
%%%%%%%%%%%%%%%%%%%



%%%%%%%%%%%%%%%%%%%
%%%%%%%%%%%%%%%%%%%
%%%%%%%%%%%%%%%%%%%

