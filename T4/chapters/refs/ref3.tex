

%%%%%%%%%%%%%%% SMM





Метод усреднения дифференциальных операторов, основанный на асимптотических разложениях по малому параметру, широко используется  в математической и физической литературе. Этот метод позволяет помимо теоремы усреднения получить оценки разности точного решения и его приближений. Нами впервые к усреднению обобщенного уравнения Бельтрами привлечены асимптотические методы, что позволило получить теорему усреднения и оценки порядка $O(\sqrt{\varepsilon})$ разности точного решения и его приближений в нормах пространств Лебега и Соболева.


Эти оценки получены асимптотическими методами при минимальных предположениях гладкости на данные задачи:

\begin{itemize}
  \item
  коэффициенты --- измеримые ограниченные $\varepsilon$-периодические функции;
  \item
  граница области из класса $C^2$;
  \item
  правая часть из пространства Соболева $W_2^1$.
\end{itemize}

Получены операторные оценки усреднения обобщенных уравнений Бельтрами. Изучены вопросы гельдеровости решений задачи Римана-Гильберта для обобщенных систем уравнений Бельтрами при минимальных условиях на данные задачи.



%%%%%%%%%%%%%%% KRI


Применяя метод модельных уравнений, исследованы вопросы моментной устойчивости решений по части переменных относительно начальных данных для линейных импульсных систем  дифференциальных уравнений Ито с последействием.
%Получены достаточные условия устойчивости в терминах параметров исследуемых систем.
Получены достаточные условия устойчивости решений стохастических дифференциальных уравнений в терминах параметров этих уравнений.  Эти результаты могут быть применены при исследованию на  устойчивость развитие различных процессов биологии, физики, химии, экономики, подверженных случайным воздействиям.


Изучены вопросы  глобальной экспоненциональной $p$-устойчивости $(2 \le p < \infty)$ систем линейных дифференциальных уравнений Ито с запаздываниями специального вида, используя теорию положительно обратимых матриц. Для этого применяется идеи и методы, разработанная Н.В. Азбелевым и его учениками для исследования вопросов устойчивости для детерминированных функционально--дифференциальных уравнений. Получены достаточные условия глобальной экспоненциональной $p$-устойчивости $(2 \le p < \infty)$ систем нелинейных  дифференциальных уравнений Ито с запаздываниями в терминах положительной обратимости матрицы, построенной по исходной системе. Проверена  выполнимость этих условий для конкретных уравнений.


Исследованы вопросы моментной устойчивости решений  относительно начальных данных и допустимости пар пространств для линейных дифференциально--разностных уравнений Ито. Исследование проведено методом вспомогательных или модельных уравнений.







%%%%%%%%%%%%%%% AEI

Исследованы вопросы существования, единственности, построения решения численными методами решения задачи Дирихле для одного нелинейного дифференциального уравнения второго порядка с $p$-лапласианом.


Доказано существование и единственность решения двухточечной краевой задачи для одного семейства нелинейных обыкновенных дифференциальных уравнений четвертого порядка.


Исследованы вопросы существования, единственности, построения решения численными методами решения одной нелинейной двухточечной краевой задачи для одного нелинейного обыкновенного дифференциального уравнения с дробными производными.




%%%%%%%%%%%%%%% MZG

Введены в рассмотрение два двухпараметрических семейства ломаных на плоскости. Решена задача восстановления функции по ее интегралам вдоль этих ломаных, когда весовая функция – квазимногочлен. Для частных случаев весовых функций получены формулы обращения. В общем случае доказана единственность решения поставленной задачи. Результаты применены к доказательству единственности задачи интегральной геометрии с возмущением.


Доказана единственность восстановления функции, суммируемой в полосе на плоскости, заданной своими интегралами вдоль дуг двухпараметрических кривых второго порядка с весом, аналитическим по части переменных.

Доказаны формулы для определения неизвестного векторного поля на плоскости, заданного своим поперечным лучевым преобразованием в ограниченном угловом диапазоне. В первой формуле используется интегральная формула интерполяции функции с ограниченным спектром. Восстанавливается преобразование Фурье дивергенции неизвестного поля. Во второй формуле интерполяция производится по дискретным значениям функции. Восстанавливаются координатные функции потенциальной части искомого поля. Векторное поле ищется в классе вектор-функций, сосредоточенных в некоторой полосе, достаточно быстро убывающих на бесконечности и имеющих непрерывные вторые производные.








%В отчетном году удалось получить асимптотическое разложение и оценку невязки для решения задачи Римана – Гильберта для уравнения Бельтрами с периодическим коэффициентом, зависящим от малого параметра.
%
%Кроме того, выведены усредненные уравнения для недивергентных эллиптических уравнений второго порядка, коэффициенты которых локально периодичны (с малым периодом) по одной из переменных.
%
%Доказано свойство гельдеровости решения задачи Римана -- Гильберта для системы уравнений Бельтрами, а также получены оценки усреднения решения задачи Римана -- Гильберта для обобщенных уравнений Бельтрами.
%
%
%%Для задачи Дирихле для нелинейного дифференциального уравнения с $p(x)$-лапласиа\-ном конструируются верхнее и нижнее решения путем склеивания на двух участках. Построенные верхнее и нижнее решения позволяют не только обосновать существование слабого решения, но и оценить решение сверху и  снизу.
%
%В последние годы  интенсивно развивается теория дифференциальных
%уравнений с $p(x)$\-лапласианом в главной части. В настоящее время
%имеется много публикаций, посвященных задаче Дирихле для
%дифференциальных уравнений с $p(x)$-лапласианом (см., например,
%\cite{Abd13,Abd14,Abd15,Abd16,Abd17,Abd18} и др.). Поскольку $p(x)$-лапласиан не является однородным
%оператором, возникают трудности при доказательстве существования и
%единственности решений и иследовании этих решений. Эти трудности
%разные авторы преодолевают с применением методов
%функционального анализа. Для этой цели мы воспользовались известным
%методом нижних и верхних решений, который позволяет не только обосновать существование слабого решения, но и оценить решение сверху и снизу. 