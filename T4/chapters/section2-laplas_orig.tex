\chapter{Обращение преобразования Лапласа посредсством обобщенных специальных рядов по  полиномами Лагерра}

%Рассмотрена задача об обращении преобразования Лапласа
% посредством специального ряда по полиномам Лагерра, который в  частном случае совпадают с рядом Фурье по   полиномам $l_{r,k}^{\gamma}(x)$ $(r\in \mathbb{N}, k=0,1,\ldots)$, ортогональным относительно скалярного произведения типа Соболева следующего вида
%\begin{equation*}
%<f,g>=\sum\nolimits_{\nu=0}^{r-1}f^{(\nu)}(0)g^{(\nu)}(0)+\int_0^\infty f^{(r)}(t)g^{(r)}(t)t^\gamma e^{-t}dt, \gamma>-1.
%\end{equation*}
%  Даны оценки приближения функций частичными суммами специального ряда по полиномам Лагерра.












\section{Введение}
В ряде работ автора \cite{Shar13} -- \cite{Shar16}  были введены, так называемые смешанные ряды по классическим ортогональным полиномам, которые, как выяснилось позже \cite{Shar11}, \cite{Shar12}, представляют собой не что иное, как ряды Фурье по полиномам, ортогональным по Соболеву, ассоциированным с соответствующими классическими ортогональными  полиномами.
Интерес к теории систем   функций (особенно, полиномов), ортогональных по Соболеву в последние годы заметно усилился (см., например, \cite{KwonLittl1} -- \cite{MarcelXu} и цитированную там литературу). Были исследованы некоторые общие свойства полиномов, ортогональных по Соболеву, в том числе и связанные со сравнительной асимптотикой \cite{MarcelVanash}. Число работ по данной теме  неуклонно растет  \cite{MarcelXu}. В работах \cite{Shar13} -- \cite{Shar12} были подробно исследованы аппроксимативные свойства смешанных рядов для функций из различных функциональных пространств и классов. В частности, было показано, что частичные суммы смешанных рядов по классическим ортогональным полиномами, в отличие от сумм Фурье по этим же полиномам, успешно могут быть использованы в задачах, в которых требуется одновременно приближать дифференцируемую функцию и ее несколько производных. Такие проблемы непременно возникают, например, при решении краевых задач для обыкновенных дифференциальных уравнений.

В настоящей статье эти вопросы рассмотрены для классических полиномов Лагерра $L_n^\alpha(x)$ в связи с задачей обращения преобразования Лапласа. Тот факт, что ряд Фурье по полиномам, ортогональным по Соболеву относительно скалярного произведения
\begin{equation}\label{laplas-1.1}
<f,g>=\sum_{\nu=0}^{r-1}f^{(\nu)}(0)g^{(\nu)}(0)+\int_0^\infty f^{(r)}(t)g^{(r)}(t)t^\alpha e^{-t}dt, \alpha>-1,
\end{equation}
ассоциированным с полиномами Лагерра представляют собой смешанные ряды по полиномам Лагерра, позволяет применить к исследованию их аппроксимативных свойств  методы и подходы, разработанные нами ранее \cite{Shar13} -- \cite{Shar16} для решения аналогичной задачи для смешанных рядов по классическим ортогональным полиномам. Именно на таком пути в работах  \cite{Shar11} и \cite{Shar12}, в частности, были исследованы апрроксимативные свойства частичных сумм ряда Фурье по полиномам, ортогональным по Соболеву, порожденным классическими полиномами Лагерра. С другой стороны в \cite{Shar11} и \cite{Shar12} были введены некоторые специальные ряды по полиномам Лагерра, которые также тесно связаны со смешанными рядами по полиномами Лагерра и в отдельных важных случаях совпадают с рядами Фурье по полиномам, ортогональным по Соболеву и порожденным полиномами Лагерра. В \cite{Shar11} и \cite{Shar12} были весьма подробно исследованы аппроксимативные свойства частичных сумм указанных специальных рядов. В настоящей статье в связи задачей об обращении преобразования Лапласа мы введем \textit{ обобщенные специальные ряды} по полиномам Лагерра,  для которых специальные ряды, введенные в \cite{Shar11} и \cite{Shar12} являются частным случаем.

 Отметим, что если нам задано преобразование Лапласа
$\bar f(p)=\int_0^\infty e^{-pt}f(t)dt$ и для оригинала существуют производные $f^{(\nu)}(0)$ с $\nu=0,1,\ldots, r-1$, то мы можем считать заданным и преобразование Лапласа $F(p)=\int_0^\infty e^{-pt}[f(t)-P_{r-1}(t)]dt$, где $P_{r-1}(t)=P_{r-1}(x)=\sum\nolimits_{\nu=0}^{r-1}f^{(\nu)}(0)\frac{x^\nu}{\nu!}$ -- полином Тейлора для функции $f(t)$ в точке $t=0$. В настоящей статье мы будем рассматривать задачи об обращении преобразования Лапласа второго из указанных типов. Именно на этом пути возникают специальные ряды по общим полиномам Лагерра $L_n^{\alpha}(t)$, обладающие на конечных отрезках вида $[0,A]$ лучшими, чем у классических рядов Фурье-Лагерра аппроксимативными свойствами для дифференцируемых функций   (см. п.\ref{laplas6}).



\section{Некоторые сведения о полиномах Лагерра}
В дальнейшем нам понадобится ряд свойств свойств классических полиномов Лагерра $L_n^{\alpha}(t)$, которых для удобства ссылок мы соберем в этом параграфе. Пусть $\alpha$ -- произвольное действительное число. Тогда  имеют место \cite{Sege}:

\textit{Формула Родрига}
\begin{equation}\label{laplas-2.1}
L_n^{\alpha}(t) = \frac{1}{n!}t^{-\alpha}e^{t} \left\{ t^{n+\alpha} e^{-t} \right\}^{(n)};
\end{equation}

\textit{Явный вид}
\begin{equation}\label{laplas-2.2}
L_n^\alpha(t) =
\sum\limits_{\nu=0}^{n}
\binom{n+\alpha}{n-\nu}
\frac{(-x)^\nu}{\nu!};
\end{equation}

\textit{Соотношение ортогональности}

\begin{equation}
\label{laplas-2.3}
\int_0^{\infty} t^{\alpha} e^{-t} L^{\alpha}_{n}(t) L^{\alpha}_{m}(t) dt = \delta_{nm} h^{\alpha}_n \quad (\alpha > -1),
\end{equation}
где $\delta_{nm}$ --- символ Кронекера,
\begin{equation}\label{laplas-2.4}
h^{\alpha}_n = \left( n+\alpha \atop n \right) \Gamma(\alpha +1);
\end{equation}
\textit{Формула Кристоффеля -- Дарбу}
\begin{equation}\label{laplas-2.5}
\mathcal{K}_n^\alpha(t,\tau)=
\sum\limits_{k=0}^{n}\frac{L_\nu^\alpha(t)L_\nu^\alpha(\tau)}{h_\nu^\alpha}=
\frac{n+1}{h_n^\alpha}
\frac{L_n^\alpha(t)L_{n+1}^\alpha(\tau) - L_n^\alpha(\tau)L_{n+1}^\alpha(t)}{t-\tau};
\end{equation}

\textit{Свертка}
\begin{equation}
\label{laplas-2.6}
\int_0^{t} L_{n}(t-\tau) L_{m}(\tau) d\tau = L_{n+m}(t) - L_{n+m+1}(t).
\end{equation}

Далее отметим следующие равенства
\begin{equation}\label{laplas-2.7}
\frac{d}{dt} L_n^{\alpha}(t) = -L_{n-1}^{\alpha+1}(t),
\end{equation}

\begin{equation} \label{laplas-2.8}
\frac{d^r}{dt^r} L_{k+r}^{\alpha-r}(t) = (-1)^{r} L_{k}^{\alpha}(t),
\end{equation}
\begin{equation}\label{laplas-2.9}
L_{k}^{-r}(t) = \frac{(-t)^{r}}{k^{[r]}} L_{k-r}^{r}(t),
\end{equation}
где $k^{[r]} = k(k-1)\ldots(k-r+1)$,
\begin{equation}\label{laplas-2.10}
L_n^{\alpha+1}(t)-L_{n-1}^{\alpha+1}(t)=L_n^\alpha(t),
     \end{equation}
 \begin{equation}\label{laplas-2.11}
(n+\alpha)L_n^{\alpha-1}(t)=\alpha L_n^\alpha(t)-
xL_{n-1}^{\alpha+1}(t),
\end{equation}







\textit{весовая оценка} \cite{AskeyWaiger}
\begin{equation}\label{laplas-2.12}
e^{-\frac{t}{2}}|L_n^\alpha(t)| \le c(\alpha) B_n^\alpha(t), \quad \alpha>-1,
\end{equation}
где здесь и далее $c,c(\alpha),c(\alpha,\ldots,\beta)$ -- положительные числа, зависящие лишь от указанных параметров,
\begin{equation*}
B_n^\alpha(t)=
\begin{cases}
\theta^\alpha, &0 \le t \le \frac{1}{\theta},\\
\theta^{\frac{\alpha}{2} - \frac{1}{4}}\,t^{-\frac{\alpha}{2} - \frac{1}{4}}, & \frac{1}{\theta} < t \le \frac{\theta}{2},\\
\Bigl[
\theta(\theta^{\frac{1}{3}}+|t-\theta|)
\Bigr]^{-\frac{1}{4}}, & \frac{\theta}{2} < t \le \frac{3\theta}{2},\\
e^{-\frac{t}{4}}, &\frac{3\theta}{2}< t,
\end{cases}
\end{equation*}
где $\theta=\theta_n=\theta_n(\alpha)=4n+2\alpha+2$.

Для нормированных полиномов Лагерра
\begin{equation}\label{laplas-2.13}
l_n^\alpha(t)=
\Bigl\{h_n^\alpha \Bigr\}^{-\frac{1}{2}} L_n^\alpha(t)
\end{equation}
имеет место оценка \cite{AskeyWaiger}
\begin{equation}\label{laplas-2.14}
e^{-\frac{t}{2}}
\Bigl|
l_{n+1}^\alpha(t)-
l_{n-1}^\alpha(t)
\Bigr|\le
\begin{cases}
\theta^{\frac{\alpha}{2}-1}, &0 \le t \le \frac{1}{\theta},\\
\theta^{-\frac{3}{4}}\,t^{-\frac{\alpha}{2} + \frac{1}{4}}, & \frac{1}{\theta} < t \le \frac{\theta}{2},\\
t^{-\frac{\alpha}{2}}\,
\theta^{-\frac{3}{4}}
\Bigl[
\theta^{\frac{1}{3}}+|t-\theta|
\Bigr]^{\frac{1}{4}}, & \frac{\theta}{2} < t \le \frac{3\theta}{2},\\
e^{-\frac{t}{4}}, &\frac{3\theta}{2}< t.
\end{cases}
\end{equation}
Поскольку $h_n^\alpha=\frac{\Gamma(n+\alpha+1)}{n!} \asymp n^\alpha$, то из \eqref{laplas-2.12} и \eqref{laplas-2.13} следует, что
\begin{equation}\label{laplas-2.15}
e^{-\frac{t}{2}}
|l_n^\alpha(t)|\le
c(\alpha)\theta_n^{-\frac{\alpha}{2}}B_n^\alpha(t), \quad t \ge 0.
\end{equation}


\section{Обращение преобразования Лапласа посредством специальных рядов по полиномами Лагерра}
Пусть для функции $f(x)$, определенной на полуоси $[0,\infty)$, существуют производные $f^{(\nu)}(0)$ с $\nu=0,1,\ldots, r-1$. Тогда мы можем определить на $(0,\infty)$  новую функцию ($\alpha\in\mathbb{R}$)
\begin{equation}\label{laplas-3.1}
g(x)=\frac{1}{x^\alpha}(f(x)-P_{r-1}(x)),\quad \text{где} \quad P_{r-1}(x)=\sum\nolimits_{\nu=0}^{r-1}f^{(\nu)}(0)\frac{x^\nu}{\nu!},
\end{equation}
причем будем считать, что если $r=0$, то $P_{r-1}(x)\equiv0$. Предположим, что $g(x)\in L_{2,\rho}$, где $\rho(x)=e^{-x}x^\alpha$, $\alpha>-1$.
Тогда  $g(x)$ можно представитьв виде ряда Фурье по  полиномам Лагерра $L_k^\alpha(x)$:
\begin{equation}\label{laplas-3.2}
g(x)=\sum_{k=0}^{\infty} g_k^\alpha L_k^\alpha(x),
\end{equation}
где
\begin{equation}\label{laplas-3.3}
 g_k^\alpha=\frac{1}{h_k^\alpha} \int_0^\infty g(t) L_k^\alpha(t)t^\alpha e^{-t}dt.
\end{equation}
Кроме того представим в виде ряда Фурье по полиномам Лагерра $L_k^\alpha(x)$  функцию $\eta(x)=e^{-(p-1)x}$:
\begin{equation}\label{laplas-3.4}
\eta(x)=\sum_{k=0}^{\infty} \eta_k^\alpha L_k^\alpha(x),
\end{equation}
\begin{equation}\label{laplas-3.5}
 \eta_k^\alpha=\frac{1}{h_k^\alpha} \int_0^\infty \eta(t) L_k^\alpha(t)t^\alpha e^{-t}dt=\frac{1}{h_k^\alpha} \int_0^\infty e^{-pt}L_k^\alpha(t)t^\alpha dt.
\end{equation}
Чтобы найти значение последнего интеграла воспользуемся тем фактом, что он представляет собой преобразование Лапласа функции
$L_k^\alpha(t)t^\alpha$ и, как хорошо известно \cite{DitPrud}, \cite{KrylovSkob}, имеет место равенство
\begin{equation}\label{laplas-3.6}
  \int_0^\infty e^{-pt}L_k^\alpha(t)t^\alpha dt=\frac{1}{p^{\alpha+1}}\left(1-\frac1p\right)^kh_k^\alpha.
\end{equation}
Из \eqref{laplas-3.5} и \eqref{laplas-3.6} имеем
\begin{equation}\label{laplas-3.7}
 \eta_k^\alpha=\frac{1}{p^{\alpha+1}}\left(1-\frac1p\right)^k.
\end{equation}
Пусть задано преобразование Лапласа
\begin{equation}\label{laplas-3.8}
F(p)=\int_0^\infty e^{-pt}\left[f(t)-P_{r-1}(t)\right]dt,
\end{equation}
которое с учетом \eqref{laplas-3.1} можем переписать так
\begin{equation}\label{laplas-3.9}
F(p)=\int_0^\infty e^{-(p-1)t} g(t)t^\alpha e^{-t}dt.
\end{equation}
Если $p>\frac12$, то к последнему интегралу мы можем применить обобщенное равенство Парсеваля, что с учетом \eqref{laplas-3.3} и \eqref{laplas-3.8} дает
\begin{equation}\label{laplas-3.10}
p^{\alpha+1}F(p)=p^{\alpha+1}\sum_{k=0}^\infty h_k^\alpha g_k^\alpha\eta_k=\sum_{k=0}^\infty g_k^\alpha h_k^\alpha\left(1-\frac1p\right)^k.
\end{equation}
Полагая $z=1-1/p$, мы можем переписать это равенство еще так:
\begin{equation}\label{laplas-3.11}
G(z)=\frac{1}{(1-z)^{\alpha+1}}F\left(\frac{1}{1-z}\right)=\sum_{k=0}^\infty g_k^\alpha h_k^\alpha z^k.
\end{equation}
Поскольку  $G(z)$ -- заданная функция, аналитическая в круге $|z|<1$,  то неизвестные коэффициенты $g_k^\alpha$ из \eqref{laplas-3.10} можно найти из равенств
\begin{equation}\label{laplas-3.12}
g_k^\alpha=\frac{1}{R^{k} h_k^\alpha}\frac{1}{2\pi}\int_0^{2\pi}G(Re^{i\varphi})e^{-ik\varphi}d\varphi, \quad k=0,1,\ldots,
\end{equation}
где $0<R<1$. При этом заметим, что приближенные значения интегралов из \eqref{laplas-3.12} c $0\le k\le N $ для произвольного натурального $N$ (особенно для $N=2^m$) можно найти путем применения быстрого дискретного преобразования Фурье.

Равенство \eqref{laplas-3.2} с учетом \eqref{laplas-3.1} можно переписать  так
\begin{equation}\label{laplas-3.13}
f(x)=\sum\nolimits_{\nu=0}^{r-1}f^{(\nu)}(0)\frac{x^\nu}{\nu!}+x^\alpha\sum_{k=0}^{\infty} g_k^\alpha L_k^\alpha(x),
\end{equation}
в, частности, если  $\alpha=r$, то
\begin{equation}\label{laplas-3.14}
f(x)=\sum\nolimits_{\nu=0}^{r-1}f^{(\nu)}(0)\frac{x^\nu}{\nu!}+x^r\sum_{k=0}^{\infty} g_k^r L_k^r(x).
\end{equation}

 Отметим, что рассмотренный нами выше подход к обращению преобразования  Лапласа \eqref{laplas-3.8} приводит к задаче об исследованиии вопросов сходимости  специальных рядов по полиномам Лагерра вида \eqref{laplas-3.13} и \eqref{laplas-3.14}. Специальные ряды \eqref{laplas-3.14} являлись одним из объектов исследования работ  \cite{Shar11} и \cite{Shar12}. Ниже мы покажем, что ряд \eqref{laplas-3.13} представляет собой частный случай некоторых обобщенных специальных рядов по полиномам Лагерра, а ряд \eqref{laplas-3.14} представляет собой, не что иное, как ряд Фурье по полиномам, ортогональным по Соболеву, порожденным полиномами Лагерра $L_k^0(x)$.














\section{Обобщенные специальные ряды по полиномам Лагерра}

Пусть $1\le r$ -- целое, $\beta\in \mathbb{R}$, $f(t)$ -- $r-1$ раз дифференцируемая в точке $t=0$,
\begin{equation}\label{laplas-4.1}
  P_{r-1}(f)=P_{r-1}(f)(t)=\sum\limits_{i=0}^{r-1}\frac{f^{(i)}(0)}{i!}t^i,
\end{equation}
\begin{equation}\label{laplas-4.2}
  f_\beta(t)=\frac1{t^\beta}[f(t)-P_{r-1}(f)(t)].
\end{equation}

Предположим, что для функции $f_\beta(t)$, определенной равенством \eqref{laplas-4.2} существуют коэффициенты Фурье-Лагерра
\begin{equation*}
  \hat{f}_{\beta,k}^\gamma=\frac1{h_k^\gamma}\int_0^\infty f_\beta(\tau)t^\gamma e^{-t}L_k^\gamma(t)dt=
\end{equation*}
\begin{equation}\label{laplas-4.3}
  \frac1{h_k^\gamma}\int_0^\infty [f(t)-P_{r-1}(f)(t)]t^{\gamma-\beta}e^{-t}L_k^\gamma(t)dt,
\end{equation}
где $h_n^\gamma=\Gamma(n+\gamma+1)/n!$.
Тогда мы можем рассмотреть ряд Фурье-Лагерра функции $f_\beta(t)$:
\begin{equation}\label{laplas-4.4}
  f_\beta(t)\sim\sum\nolimits_{k=0}^\infty\hat{f}_{\beta,k}^\gamma L_k^\gamma(t).
\end{equation}
Если ряд \eqref{laplas-4.4} сходится к $f_\beta(t)$, то с учетом \eqref{laplas-4.2} мы можем записать
\begin{equation}\label{laplas-4.5}
  f(t)=P_{r-1}(f)(t)+t^\beta\sum\nolimits_{k=0}^\infty\hat{f}_{\beta,k}^\gamma L_k^\gamma(t).
\end{equation}
 Это и есть \textit{ обобщенный специальный ряд по полиномам Лагерра}. Из равенств \eqref{laplas-3.1} и \eqref{laplas-4.2}  непосредственно следует, что  если $\gamma=\beta=\alpha$, то $g_(x)=f_\beta(x)$ и ряд \eqref{laplas-4.5} совпадает с рядом \eqref{laplas-3.13}. В следующем параграфе мы покажем, в частности, что ряд \eqref{laplas-3.14} представляет собой ряд Фурье по полиномам, ортогональным по Соболеву, порожденным полиномами Лагерра $L_n^0(x)$.





\section{Ортогональные по Соболеву полиномы, порожденные полиномами Лагерра}
Пусть $-1<\gamma$,  $\rho=\rho(x)=x^\gamma e^{-x}$, $1\le p<\infty $,  $\mathcal{ L}_{\rho}^p$ -- пространство измеримых функций $f(x)$, определенных на полуоси $[0,\infty)$ и таких, что
     $$
\|f\|_{\mathcal{ L}_{\rho}^p}=
\left(\int_0^\infty|f(x)|^p\rho(x)dx\right)^{1/p}<\infty.
    $$
Из равенства \eqref{laplas-2.3} следует, что если $\gamma>-1$, то полиномы $l_n^{\gamma}(x),\quad(n=0,1,\ldots)$ (см.\eqref{laplas-2.13})
образуют ортонормированную  в $\mathcal{ L}_\rho^2$  систему. Как хорошо известно \cite{Sege}, система полиномов Лагерра  \eqref{laplas-2.13} полна в $\mathcal{ L}_\rho^2$.   Эта система порождает на $[0,\infty)$ систему полиномов $l_{r,k}^{\gamma}(x)$ $(k=0,1,\ldots)$, определенных равенствами

  \begin{equation}\label{laplas-5.1}
l_{r,k}^{\gamma}(x) =\frac{x^k}{k!}, \quad k=0,1,\ldots, r-1,
\end{equation}
  \begin{equation}\label{laplas-5.2}
l_{r,r+k}^{\gamma}(x) =\frac{1}{(r-1)!}\int_{0}^x(x-t)^{r-1}l_{k}^{\gamma}(t)dt, \quad k=0,1,\ldots.
\end{equation}
 Через $W_{\mathcal{ L}_{\rho}^p}^r$ обозначим  подкласс функций $f=f(x)$ из $\mathcal{ L}_{\rho}^p$,
непрерывно дифференцируемых $r-1$ раз, для которых $f^{(r-1)}(x)$
абсолютно непрерывна на произвольном сегменте $[a,b]\subset[0,\infty)$,
а $f^{(r)}\in \mathcal{ L}_{\rho}^p$. В $W_{\mathcal{ L}_{\rho}^2}^r$ мы введем скалярное произведение \eqref{laplas-1.1}, которое превращает $W_{\mathcal{ L}_{\rho}^2}^r$ в гильбертово пространство.
В работах  \cite{Shar11} и \cite{Shar12} была доказана следующая теорема.

\begin{theoremA}\label{laplastheo1}
Пусть $\gamma>-1$. Тогда система полиномов $\{l_{r,k}^{\gamma}(x)\}_{k=0}^\infty$, порожденная системой ортонормированных полиномов Лагерра \eqref{laplas-2.13} посредством равенств \eqref{laplas-5.1} и \eqref{laplas-5.2}, полна  в $W^r_{\mathcal{ L}^2_\rho}$ и ортонормирована относительно скалярного произведения \eqref{laplas-1.1}.
\end{theoremA}

%\begin{proof}
%Из \eqref{laplas-5.1} и \eqref{laplas-5.2} следует, что
% \begin{equation}\label{laplas-5.3}
%(l^\gamma_{r,k}(x))^{(\nu)} =\begin{cases}l^\gamma_{r-\nu,k-\nu}(x),&\text{если $0\le\nu\le r-1$, $r\le k$,}\\
%l^\gamma_{k-r}(x),&\text{если  $\nu=r\le k$,}\\
%l^\gamma_{r-\nu,k-\nu}(x),&\text{если $\nu\le k< r$,}\\
%0,&\text{если $k< \nu\le r$}.
%  \end{cases}
%\end{equation}
%В силу первого из равенств  \eqref{laplas-5.3} следует, что если $r\le k$ и $0\le\nu\le r-1$, то  $(l^\gamma_{r,k}(x))^{(\nu)}_{x=0}=0$, поэтому
%в силу второго равенства из  \eqref{laplas-5.3},  имеем
%$$
%<l^\gamma_{r,k},l^\gamma_{r,l}>= \int_{0}^\infty(l^\gamma_{r,k}(x))^{(r)}(l^\gamma_{r,l}(x))^{(r)}\rho(x) dx=
%$$
%\begin{equation}\label{laplas-5.4}
%    \int_{0}^\infty l^\gamma_{k-r}(x)l^\gamma_{l-r}(x)\rho(x) dx=\delta_{kl},
%    \quad k,l\ge r,
%  \end{equation}
% а в силу третьего  и четвертого из равенств  \eqref{laplas-5.3} получаем
%\begin{equation}\label{laplas-5.5}
%  <l^\gamma_{r,k},l^\gamma_{r,l}>=
%  \sum_{\nu=0}^{r-1}(l^\gamma_{r,k}(x))^{(\nu)}|_{x=0}
%  (l^\gamma_{r,l}(x))^{(\nu)}|_{x=0}=\delta_{kl},\quad k,l< r.
%  \end{equation}
%  Очевидно также, что
%  \begin{equation}\label{laplas-5.6}
%  <l^\gamma_{r,k},l^\gamma_{r,l}>=0,\quad \text{если}\quad k< r\le l\quad \text{или} \quad l< r\le k.
%  \end{equation}
% Равенства \eqref{laplas-5.4} -- \eqref{laplas-5.6}  означают, что функции  $l^\gamma_{r,k}(x)\, (k=0,1,\ldots) $ образуют   в $W^r_{\mathcal{ L}^2_\rho}$ ортонормированную  систему относительно скалярного произведения \eqref{laplas-1.1}.  Остается убедиться в ее полноте в $W^r_{\mathcal{ L}^2_\rho}$. С этой целью покажем, что если для некоторой функции $f=f(x)\in W^r_{\mathcal{ L}^2_\rho}$ и для  всех $k=0,1,\ldots$ справедливы равенства $<f,l^\gamma_k>=0$, то $f(x)\equiv0$. В самом деле, если $k\le r-1$, то  $<f,l^\gamma_{r,k}>=f^{(k)}(0)$, поэтому с учетом того, что $<f,l^\gamma_{r,k}>=0$,  для нашей функции  $f(x)$ формула Тейлора
% \begin{equation*}
%f(x)=\sum_{k=0}^{r-1} f^{(k)}(0)\frac{x^k}{k!}+{1\over (r-1)!}\int\limits_{0}^x(x-t)^{r-1} f^{(r)}(t)dt
%     \end{equation*}
% приобретает вид
%\begin{equation}\label{laplas-5.7}
%f(x)={1\over (r-1)!}\int\limits_{0}^x(x-t)^{r-1} f^{(r)}(t)dt.
%     \end{equation}
%С другой стороны, для всех $k\ge r$ имеем
%$$
% 0= <f,l^\gamma_{r,k}>=\int_{0}^\infty f^{(r)}(x) (l^\gamma_{r,k}(x))^{(r)}\rho(x) dx=
%  \int_{0}^\infty f^{(r)}(x)l^\gamma_{k-r}(x) \rho(x) dx .
%$$
%Отсюда и из того, что $l^\gamma_m(x)$ ($m=0,1,\ldots$)  образуют в $\mathcal{ L}^2_{\rho}$ полную ортонормированную систему имеем $f^{(r)}(x)=0$ почти всюду на $[0,\infty)$. Поэтому из \eqref{laplas-5.7} следует, что   $f(x)\equiv0$. Теорема I доказана.
%
%\end{proof}

Ряд Фурье функции $f\in W^r_{\mathcal{ L}^2_\rho}$ по системе $\{l_{r,k}^{\gamma}(x)\}_{k=0}^\infty$
мы можем записать в виде
\begin{equation}\label{laplas-5.8}
f(x)\sim  \sum_{k=0}^\infty <f,l_{r,k}^\gamma>  l_{r,k}^\gamma(x),
     \end{equation}
где
\begin{equation}\label{laplas-5.9}
<f,l_{r,k}^\gamma>=f^{(k)}(0),\quad k=0,\ldots, r-1,
     \end{equation}
\begin{equation}\label{laplas-5.10}
<f,l_{r,k}^\gamma>=\int\limits_0^\infty f^{(r)}(t) l_{k-r}^\gamma(t)e^{-t}t^\gamma dt=f_{r,k}^\gamma,\quad k=r,r+1,\ldots.
     \end{equation}
В силу \eqref{laplas-5.9}  и \eqref{laplas-5.10} мы можем \eqref{laplas-5.8} переписать еще так
\begin{equation}\label{laplas-5.11}
f(x)\sim \sum_{k=0}^{r-1} f^{(k)}(0)\frac{x^k}{k!}+ \sum_{k=r}^\infty f_{r,k}^\gamma l_{r,k}^\gamma(x).
\end{equation}

Ряд, фигурирующий в правой части соотношения \eqref{laplas-5.11} впервые был исследован в работе \cite{Shar13}, где он был назван \textit{ смешанным рядом по полиномам Лагерра $L_{k}^\gamma(x)$ }. Из теоремы \ref{laplastheo1} следует, что если $f\in W^r_{\mathcal{ L}^2_\rho}$, то ряд \eqref{laplas-5.11}, будучи  рядом Фурье  по системе $\{l_{r,k}^{\gamma}(x)\}_{k=0}^\infty$, сходится к $f$ в метрике гильбертова пространства $W^r_{\mathcal{ L}^2_\rho}$ со скалярным произведением \eqref{laplas-1.1}, другими словами, имеет место предельное соотношение
 \begin{equation*}
 \lim_{n\to\infty}\sum_{k=n}^\infty (f_{r,k}^\gamma)^2= 0 .
\end{equation*}

Перейдем к получению некоторых  представлений для полиномов
$l_{r,r+k}^{\gamma}(x)$ при $k\ge0$.Важное представление для полиномов $l_{r,n+r}^{\gamma}(x)$ можно получить если мы обратимся к равенствам \eqref{laplas-2.2} и \eqref{laplas-2.13} и запишем
\begin{equation*}
l_n^\gamma(x) =\frac{1}{(h_n^\gamma)^{1/2}}
\sum\limits_{\nu=0}^{n}
\binom{n+\gamma}{n-\nu}
\frac{(-x)^\nu}{\nu!}.
\end{equation*}
Из этого равенства, с учетом того, что
\begin{equation*}
{1\over (r-1)!}\int\limits_{0}^x(x-t)^{r-1}t^\nu dt=\frac{x^{\nu+r}}{(\nu+r)^{[r]}},
\end{equation*}
убеждаемся в справедливости следующего утверждения.
\begin{theorem}
Имеют место равенства
\begin{equation*}
l_{r,n+r}^{\gamma}(x)=
\frac{1}{(h_n^\gamma)^{1/2}}
\sum\limits_{\nu=0}^{n}(-1)^\nu \binom{n+\gamma}{n-\nu}
\frac{x^{\nu+r}}{\nu!(\nu+r)^{[r]}}\quad (n=0,1,\ldots).
\end{equation*}
\end{theorem}
Для получения дальнейших представлений полиномов $l_{r,k}^{\gamma}(x)$ обратимся  к свойству \eqref{laplas-2.8} и запишем
 $$
 {1\over
(r-1)!}\int\limits_{0}^x(x-t)^{r-1}
     L^\gamma_k(t)dt=
{(-1)^r\over (r-1)!}\int\limits_{0}^x(x-t)^{r-1}
{d^r\over dt^r}L_{k+r}^{\gamma-r}(t)dt
     $$
 \begin{equation}\label{laplas-5.12}
 =(-1)^rL_{k+r}^{\gamma-r}(x)-(-1)^r\sum_{\nu=0}^{r-1}
{x^\nu\over\nu!}\{L_{k+r}^{\gamma-r}(t)\}_{t=0}^{(\nu)}.
 \end{equation}
Далее
\begin{equation}\label{laplas-5.13}
 \{L_{k+r}^{\gamma-r}(t)\}^{(\nu)}=(-1)^\nu
L_{k+r-\nu}^{\gamma-r+\nu}(t),
  \end{equation}
 а в силу \eqref{laplas-2.2}
 \begin{equation}\label{laplas-5.14}
L_{k+r-\nu}^{\gamma-r+\nu}(0)= {k+\gamma\choose
k+r-\nu}={\Gamma(k+\gamma+1)\over\Gamma(\nu-r+
\gamma+1)(k+r-\nu)!}.
\end{equation}
Сопоставляя \eqref{laplas-5.13} и \eqref{laplas-5.14}, имеем
\begin{equation}\label{laplas-5.15}
B_{k,\nu}^\gamma=\{L_{k+r}^{\gamma-r}(t)\}^{(\nu)}_{t=0}=
{(-1)^\nu\Gamma(k+\gamma+1)\over\Gamma(\nu-r+ \gamma+1)(k+r-\nu)!}.
\end{equation}
Из \eqref{laplas-5.12} и \eqref{laplas-5.15} находим
\begin{equation}\label{laplas-5.16}
{1\over (r-1)!}\int\limits_{0}^x(x-t)^{r-1}
 L^\gamma_k(t)dt=
(-1)^rL_{k+r}^{\gamma-r}(x)-
     (-1)^r\sum_{\nu=0}^{r-1}
{B_{k,\nu}^\gamma x^\nu\over\nu!}.
\end{equation}
С другой стороны, в силу определения \eqref{laplas-5.2} и равенства \eqref{laplas-2.13}    имеем
 \begin{equation}\label{laplas-5.17}
l_{r,r+k}^{\gamma}(x) =\frac{1}{\sqrt{h_k^\gamma}(r-1)!}\int\limits_{0}^x(x-t)^{r-1}L_{k}^{\gamma}(t)dt, \quad k=0,1,\ldots.
\end{equation}
 Сопоставляя \eqref{laplas-5.16} с \eqref{laplas-5.17} мы приходим к следующему результату.
\begin{theorem}
Пусть $\gamma>-1$, $k\ge0$ . Тогда имеет место равенство
\begin{equation}\label{laplas-5.18}
l_{r,r+k}^{\gamma}(x)=\frac{(-1)^r}{\sqrt{h_k^\gamma}}\left[L_{k+r}^{\gamma-r}(x)-
     \sum_{\nu=0}^{r-1}
{B_{k,\nu}^\gamma x^\nu\over\nu!}\right],
\end{equation}
в котором
$$
B_{k,\nu}^\gamma={(-1)^\nu\Gamma(k+\gamma+1)\over\Gamma(\nu-r+ \gamma+1)(k+r-\nu)!}.
$$
\end{theorem}

\begin{corollary}\label{laplascor1}
Пусть  $k\ge0$ . Тогда
$$
l_{r,r+k}^{0}(x)=(-1)^rL_{k+r}^{-r}(x)=\frac{x^{r}L_{k}^{r}(x)}{(k+r)^{[r]}}.
$$
\end{corollary}
\begin{proof}
Из \eqref{laplas-5.16} следует, что если $\gamma=0$, то $B_{k,\nu}^\gamma=0$ для всех $\nu=0,1,\ldots, r-1$ и, как следствие, в этом случае равенство \eqref{laplas-5.18} принимает вид
\begin{equation}\label{laplas-5.19}
l_{r,r+k}^{0}(x)=(-1)^rL_{k+r}^{-r}(x),\quad k=0,1,\ldots
\end{equation}
Поскольку в силу равенства \eqref{laplas-2.7}
$$
L_{k+r}^{-r}(x) = \frac{(-x)^{r}}{(k+r)^{[r]}} L_{k}^{r}(x),
$$
то  утверждения следствия вытекает  из \eqref{laplas-5.19}.
\end{proof}

В качестве одного из приложений следствия \ref{laplascor1} покажем, что
 смешанный ряд \eqref{laplas-5.11} в случае $\gamma=0$ совпадет со специальным рядом \eqref{laplas-3.14}.  В силу \eqref{laplas-5.19} ряд \eqref{laplas-5.11} при $\gamma=0$ приобретает следующий вид
\begin{equation}\label{laplas-5.20}
f(x)\sim \sum_{k=0}^{r-1} f^{(k)}(0)\frac{x^k}{k!}+ x^{r}\sum_{k=r}^\infty  \frac{f_{r,k}^0}{(k+r)^{[r]}} L_{k}^{r}(x).
\end{equation}
Далее
\begin{equation*}
  {f}_{r,k}^0=\int\limits_0^\infty f^{(r)}(\tau)e^{-\tau}L_k(\tau)d\tau=\frac1{k!}\int\limits_0^\infty(f(\tau)-P_{r-1}(f)(\tau))^{(r)}(e^{-\tau}\tau^k)^{(k)}d\tau=
\end{equation*}
\begin{equation*}
  \frac{(-1)^r}{k!}\int\limits_0^\infty(f(\tau)-P_{r-1}(f)(\tau))(e^{-\tau}\tau^k)^{(k+r)}d\tau=
\end{equation*}
\begin{equation*}
  \frac{(-1)^r}{k!}\int\limits_0^\infty(f(\tau)-P_{r-1}(f)(\tau))\tau^{-r}e^{-\tau}L_{k+r}^{-r}(\tau)(k+r)!d\tau=
\end{equation*}
\begin{equation*}
  \frac{(k+r)!}{k!}(-1)^r\int\limits_0^\infty\frac{(f(\tau)-P_{r-1}(f)(\tau))}{\tau^r}e^{-\tau}\frac{(-\tau)^r}{(k+r)^{[r]}}L_k^r(\tau)d\tau=
\end{equation*}
\begin{equation}\label{laplas-4.6}
  \int_0^\infty\frac{f(t)-P_{r-1}(f)(t)}{t^r}e^{-\tau}t^rL_k^r(\tau)d\tau=h_k^r\hat{f}_{r,k}^r.
\end{equation}
В силу \eqref{laplas-4.6} и того, что  $h_k^r=(k+1)_r$, ряд \eqref{laplas-5.20} мы можем переписать так
\begin{equation*}
  f(t)=P_{r-1}(f)(t)+t^r\sum\nolimits_{k=0}^\infty\frac{h_k^r\hat{f}_{r,k}^rL_k^r(t)}{(k+1)_r}=
  P_{r-1}(f)(t)+t^r\sum\nolimits_{k=0}^\infty\hat{f}_{r,k}^rL_k^r(t).
\end{equation*}
 С другой стороны, из равенств \eqref{laplas-3.3} и \eqref{laplas-4.3} следует, что $\hat{f}_{r,k}^r=g_k^r$. Таким образом, в случае $\gamma=0$ ряд  \eqref{laplas-5.11}, который представляет собой ряд Фурье по полиномам $l_{r,k}^{0}(x)$, ортогональным по Соболеву, порожденным полиномами Лагерра $L_k^0(x)$,   совпадает со специальным рядом \eqref{laplas-3.14}.




\section{Неравенство Лебега для частичных сумм специального ряда  по полиномам Лагерра}\label{laplas6}
Вернемся к вопросу об обращении преобразования Лапласа \eqref{laplas-3.8} посредством специального ряда \eqref{laplas-3.13}, коэффициенты $g_k^\alpha$ которого могут быть найдены с помощью равенства  \eqref{laplas-3.12}. При этом следует отметить, что мы можем найти только конечное число  коэффициентов $g_k^\alpha$ с $k=0,1,\ldots, N$, поэтому вместо искомого оригинала $f(t)$ ( или, что то же, $f(t)-\sum\nolimits_{\nu=0}^{r-1}f^{(\nu)}(0)\frac{t^\nu}{\nu!}$ ) мы получим его приближение
\begin{equation*}
Y_N(t)=\sum\nolimits_{\nu=0}^{r-1}f^{(\nu)}(0)\frac{t^\nu}{\nu!}+t^\alpha\sum\nolimits_{k=0}^{N} g_k^\alpha L_k^\alpha(t).
\end{equation*}
 Отсюда возникает задача об исследовании величины $|f(t)-Y_N(t)|$. Более подробно эту задачу мы рассмотрим для частичных сумм более общего специального ряда, получающегося из обобщенного специального ряда \eqref{laplas-4.5}  в случае $\beta=r$. Через $\mathcal{L}_n^\gamma(f)=\mathcal{L}_n^\gamma(f)(t)$ обозначим частичную сумму этого ряда  вида
\begin{equation*}
  \mathcal{L}_n^\gamma(f)(t)=P_{r-1}(f)+t^r\sum\limits_{k=0}^{n-r}\hat{f}_{r,k}^\gamma L_k^\gamma(t).
\end{equation*}
 Заметим, что если $f(t)=q_n(t)$ представляет собой алгебраический полином степени $n$, то
$\mathcal{L}_n^\gamma(q_n)(t)\equiv q_n(t)$ при $\gamma>-1$, другими словами, оператор $\mathcal{L}_n^\gamma(f)$ является проектором на подпространство $H^n$, состоящем из алгебраических полиномов степени $n$.
Это свойство частичных сумм  $\mathcal{L}_n^\gamma(f)(t)$ играет важную роль при решении задачи об оценке отклонения $\mathcal{L}_n^\gamma(f)(t)$ от исходной функции $f=f(t)$. Пусть $f(t)$ -- непрерывная функция, заданная на полуоси $[0,\infty)$ и такая, что в точке $t=0$ существуют производные $f^{(\nu)}(0)$ $(\nu=0,1,\dots,r-1)$. Кроме того будем считать, что для всех $k=0,1,\ldots$ существуют коэффициенты $\hat{f}_{r,k}^\gamma$, определяемые равенством \eqref{laplas-4.3} с $\beta=r$. Тогда мы можем определить специальный ряд \eqref{laplas-4.5} и его частичную сумму $\mathcal{L}_n^\gamma(f)(t)$. Рассмотрим задачу об оценке величины
\begin{equation}\label{laplas-6.1}
  R_{n,r}^\gamma(f)(t)=|f(t)-\mathcal{L}_n^\gamma(f)(t)|t^{-\frac r2+\frac14}e^{-\frac t2}.
\end{equation}
Весовой множитель $t^{-\frac r2+\frac14}$, фигурирующий в правой части равенства \eqref{laplas-6.1}, связан с тем обстоятельством, что разность
$|f(t)-\mathcal{L}_n^\gamma(f)(t)|$ стремится к нулю вместе с $t$ со скоростью, не меньшей, чем  $t^{\frac r2-\frac14}$.
Обозначим через $q_n(t)$ - алгебраический полином степени $n$, для которого
\begin{equation}\label{laplas-6.2}
  f^{(\nu)}(0)=q_n^{(\nu)}(0)\text{ }(\nu=0,1,\ldots,r-1).
\end{equation}
Тогда
\begin{equation*}
  f(t)-\mathcal{L}_n^\gamma(f)(t)=f(t)-q_n(t)+q_n(t)-\mathcal{L}_n^\gamma(f)(t)=
\end{equation*}
\begin{equation}\label{laplas-6.3}
  f(t)-q_n(t)+\mathcal{L}_n^\gamma(q_n-f)(t),
\end{equation}
поэтому в силу \eqref{laplas-6.1} и \eqref{laplas-6.3}
\begin{equation}\label{laplas-6.4}
  |R_{n,r}(f)(t)|\le|f(t)-q_n(t)|t^{-\frac r2+\frac14}e^{-\frac t2}+|\mathcal{L}_n^\gamma(q_n-f)(t)|t^{-\frac r2+\frac14}e^{-\frac t2}.
\end{equation}
С другой стороны, в силу \eqref{laplas-6.2} $P_{r-1}(q_n-f)\equiv0$, поэтому  имеем
\begin{equation*}
  \mathcal{L}_n^\gamma(q_n-f)(t)=t^r\sum\limits_{k=0}^{n-r}(\widehat{q_n-f})_{r,k}L_k^\gamma(t)=
\end{equation*}
\begin{equation*}
  t^r\sum\limits_{k=0}^{n-r}\frac1{h_k^\gamma}\int\limits_0^\infty(q_n(\tau)-f(\tau))\tau^{\gamma-r}e^{-\tau}L_k^\gamma(\tau)L_k^\gamma(t)d\tau.
\end{equation*}
Отсюда
$$
e^{-\frac t2}t^{-\frac r2+\frac14}\mathcal{L}_n^\gamma(q_n-f)(t)=
$$
\begin{equation}\label{laplas-6.5}
  e^{-\frac t2}t^{\frac r2+\frac14}\int\limits_0^\infty(q_n(\tau)-f(\tau))e^{-\tau}\tau^{\gamma-r}\sum\limits_{k=0}^{n-r}\frac{L_k^\gamma(t)L_k^\gamma(\tau)}{h_k^\gamma}d\tau.
\end{equation}
Положим
\begin{equation}\label{laplas-6.6}
  E_n^r(f)=\inf\limits_{q_n}\sup\limits_{t>0}|q_n(t)-f(t)|e^{-\frac t2}t^{-\frac r2+\frac14},
\end{equation}
где нижняя грань берется по всем алгебраическим полиномам $q_n(t)$ степени $n$ для которых $f^{(\nu)}(0)=q_n^{(\nu)}(0)$ $(\nu=0,\ldots,r-1)$. Тогда из \eqref{laplas-6.5} находим
\begin{equation}\label{laplas-6.7}
 e^{-\frac{t}{2}} t^{-\frac r2+\frac14}|\mathcal{L}_n^\gamma(q_n-f)(t)|\le E_n^r(f)\lambda_{r,n}^\gamma(t),
\end{equation}
где
\begin{equation}\label{laplas-6.8}
  \lambda_{r,n}^\gamma(t)=t^{\frac r2+\frac14}\int\limits_0^\infty e^{-\frac{\tau+t}2}\tau^{\gamma-\frac r2-\frac14}|\mathcal{K}_{n-r}^\gamma(t,\tau)|d\tau,
\end{equation}
а ядро $\mathcal{K}_{n-r}^\gamma(t,\tau)$ определяется равенством \eqref{laplas-2.5}.
Из \eqref{laplas-6.4}, \eqref{laplas-6.6} -- \eqref{laplas-6.8} выводим следующее неравенство типа Лебега
\begin{equation}\label{laplas-6.9}
  |R_{n,r}^\gamma(f)(t)|\le E_n^r(f)(1+\lambda_{r,n}^\gamma(t)).
\end{equation}
В связи с неравенством \eqref{laplas-6.9} возникает задача об оценке функции Лебега $\lambda_{r,n}^\gamma(t)$, определяемой равенством \eqref{laplas-6.8}. С этой целью мы введем следующие обозначения: $G_1=[0,\frac3{\theta_n}]$, $G_2=[\frac3{\theta_n},\frac{\theta_n}2]$, $G_3=[\frac{\theta_n}2,\frac{3\theta_n}2]$, $G_4=[\frac{3\theta_n}2,\infty]$. В работах \cite{Shar11} и \cite{Shar12}, существенно используя весовыые оценки \eqref{laplas-2.12}, \eqref{laplas-2.14} и \eqref{laplas-2.15},  были получены оценки для $\lambda_{r,n}^\gamma(t)$ при $t\in G_s$ $(s=1,2,3,4)$. А менно, доказанна

\begin{theorem}
 Пусть $1\le r$ -- целое, $r-\frac12<\gamma< r+\frac12$, $\theta_n=4n+2\gamma+2$. Тогда имеют место следующие оценки:

1) если $t \in G_1=[0,\frac3{\theta_n}]$,  то
\begin{equation}\label{laplas-6.10}
\lambda^\gamma_{r,n}(t) \leq c(\gamma,r)[\ln(n+1)+n^{\gamma-r}];
\end{equation}

2) если $t \in G_2=[\frac3{\theta_n},\frac{\theta_n}2]$, то
\begin{equation}
\lambda_{r,n}^\gamma(t) \leq c(\gamma,r)\left[\ln(n+1)+\left({n\over t}\right)^{\gamma-r\over2}\right];
\label{laplas-6.11}
\end{equation}

3) если $t \in G_3=[\frac{\theta_n}2,\frac{3\theta_n}2]$, то
\begin{equation}
\lambda_{r,n}^\gamma(t) \leq c(\gamma,r)\left[\ln(n+1)+\left({t\over \theta_n^{1/3}+|t-\theta_n| }\right)^{1/4}\right];
\label{laplas-6.12}
\end{equation}

4) если $t \in G_4=[\frac{3\theta_n}2,\infty)$, то
\begin{equation}
\lambda_{r,n}^\gamma(t) \leq c(\gamma,r)n^{-\frac{r}{2}+\frac54}t^{\frac r2+\frac14}e^{-\frac{t}{4}}.
\label{laplas-6.13}
\end{equation}

\end{theorem}



\begin{thebibliography}{999}
\bibitem{Shar13} {Шарапудинов И.И.}
Смешанные ряды по ортогональным полиномам // Издательство Дагестанского научного центра. Махачкала. 2004. Стр. 1--176.

\bibitem{Shar14}{Шарапудинов И.И.}
Смешанные ряды по полиномам Чебышева, ортогональным на равномерной сетке // Математические заметки. 2005. Т. 78. Вып. 3. Стр. 442–-465.

\bibitem{Shar15} {Шарапудинов И.И.}
Аппроксимативные свойства смешанных рядов по полиномам Лежандра на классах $W^r$ // Математический сборник. 2006. Т. 197. Вып. 3. Стр. 135–-154.

\bibitem{Shar16} {Шарапудинов И.И.}
Аппроксимативные свойства средних типа Валле-Пуссена частичных сумм смешанных рядов по полиномам Лежандра // Математические заметки. 2008. Т. 84. Вып. 3. Стр. 452--471.

\bibitem{Shar11} {Шарапудинов И.И.}
Специальные ряды по полиномам Лагерра и их аппроксимативные свойства // Сибирский математический журнал. 2017. Т. 58. Вып. 2. Стр. 440--467.

\bibitem{Shar12} {Шарапудинов И.И.}
Некоторые специальные ряды по общим полиномам Лагерра и ряды Фурье по полиномам Лагерра, ортогональным по Соболеву //  Дагестанские электронные математические известия. 2015. Вып. 4. Стр. 31–-73.

\bibitem{KwonLittl1} {Kwon K.H., Littlejohn L.L.}
The orthogonality of the Laguerre polynomials $\{L_n^{(-k)}(x)\}$ for positive integers $k$ // Ann. Numer. Anal. 1995. Vol. 2. Pp. 289–-303.

\bibitem{ KwonLittl2} {Kwon K.H., Littlejohn L.L.}
Sobolev orthogonal polynomials and second-order differential equations // Rocky Mountain J. Math. 1998. Vol. 28. Pp. 547–-594.

\bibitem{MarcelAlfaroRezola } {Marcellan F., Alfaro M., Rezola M.L.}
Orthogonal polynomials on Sobolev spaces: old and new directions // Journal of Computational and Applied Mathematics. North-Holland. 1993. Vol. 48. Pp. 113--131.

\bibitem{IserKoch } {Iserles A., Koch P.E., Norsett S.P., Sanz-Serna J.M.}
On polynomials  orthogonal  with respect  to certain Sobolev inner products // J. Approx. Theory. 1991. Vol. 65. Pp. 151--175.

\bibitem{Meijer} {Meijer H.G.}
Laguerre polynimials generalized to a certain discrete Sobolev inner product space // J. Approx. Theory. 1993. Vol. 73. Pp. 1--16.

\bibitem{MarcelXu} {Marcellan F., Yuan Xu}
ON SOBOLEV ORTHOGONAL POLYNOMIALS. arXiv: 6249v1 [math.C.A] 25 Mar 2014. Pp. 1--40

\bibitem{MarcelVanash} {Lopez G., Marcellan F., Van Assche W.}
Relative asymptotics for polynomials orthogonal with respect to a discrete Sobolev inner product // Constr. Approx. 1995. Vol. 11. Issue 1. Pp. 107--137.

\bibitem{Sege} {Сеге Г.}
Ортогональные многочлены. Москва. Физматгиз. 1962.

\bibitem{AskeyWaiger} {Askey R., Wainger S.}
Mean convergence of expansions in Laguerre and Hermite series // Amer. J. Mathem. 1965. Vol. 87. Pp. 698--708.

\bibitem{DitPrud} {Диткин В.А., Прудников А.П.}
Операционное исчисление. Москва. Высшая школа. 1975.

\bibitem{KrylovSkob} {Крылов В.И., Скобля Н.С.}
Методы приближенного преобразования Фурье и обращения преобразования Лапласа. Москва. Наука. 1974.
\end{thebibliography}