\chapter{Аппроксимативные свойства повторных средних Валле Пуссена для кусочно-гладких функций}
%\begin{abstract}
%На основе тригонометрических сумм Фурье $S_n(f,x)$ и классических средних Валле Пуссена
%$$
%_1V_{n,m}(f,x)= \frac1n\sum_{l=m}^{m+n-1}S_l(f,x)
%$$
% в настоящей статье вводятся  повторные средние Валле Пуссена следующим образом
% $$
%_2V_{n,m}(f,x)= \frac1n\sum_{k=m}^{m+n-1}{}_1V_{n,k}(f,x),
%$$
%$$
%{}_{l+1}V_{n,m}(f,x)= \frac1n\sum_{k=m}^{m+n-1} {}_{l}V_{n,k}(f,x)\quad(l\ge1).
%$$
%Исследованы аппроксимативные свойства повторных средних $_2V_{n,m}(f,x)$ для кусочно-гладких функций $f(x)$.
%\end{abstract}

%\begin{keywords}
%Повторные средние Валле Пусссена, кусочно-гладкие функции, локальные аппроксимативные свойства
%\end{keywords}

\section{Введение}\label{valle-pussen-2-s1}

Пусть $f$ -- $2\pi$-периодическая функция, интегрируемая на периоде,
\begin{equation*}
    a_k=a_k(f)=\frac1\pi\int_{-\pi}^\pi f(t)\cos ktdt,\quad b_k=b_k(f)=\frac1\pi\int_{-\pi}^\pi f(t)\sin ktdt
\end{equation*}
-- коэффициенты Фурье,
\begin{equation}\label{valle-pussen-2-1.1}
    f(x) \sim \frac{a_0}{2}+ \sum_{k=1}^\infty a_k\cos kx+b_k\sin kx
\end{equation}
-- ряд Фурье функции $f$. Далее, пусть $A_k(f)=A_k(f,x)=a_k\cos kx+b_k\sin kx$,

\begin{equation} \label{valle-pussen-2-1.2}
 S_n(f)=   S_n(f,x)=\frac{a_0}{2}+ \sum_{k=1}^n A_k(f,x)
\end{equation}
-- сумма Фурье,
\begin{equation}\label{valle-pussen-2-1.3}
 _1V_{n,m}(f)= _1V_{n,m}(f,x)=\frac1{n}[S_m(f,x)+\cdots+S_{m+n-1}(f,x)]
\end{equation}
-- средние Валле Пуссена. Повторные средние Валле Пуссена (средние Валле Пуссена второго порядка) определим с помощью равенства
\begin{equation}\label{valle-pussen-2-1.4}
 _2V_{n,m}(f)= _2V_{n,m}(f,x)=\frac1{n}[ _1V_{n,m}(f,x)+\cdots+ _1V_{n,m+n-1}(f,x)].
\end{equation}
Непосредственно из равенств \eqref{valle-pussen-2-1.3} и \eqref{valle-pussen-2-1.4} следует, что если $T_m=T_m(x)$ -- произвольный тригонометрический полином порядка $m$, то
\begin{equation}\label{valle-pussen-2-1.5}
 _1V_{n,m}(T_m)=T_m, \quad _2V_{n,m}(T_m)=T_m.
\end{equation}
Другими словами, оператор $_2V_{n,m}=_2V_{n,m}(f)$ также, как и оператор $_1V_{n,m}=$ $_1V_{n,m}(f)$, является проектором на на пространство тригонометрических полиномов $T_m$ порядка $m$. Отметим также, что в силу \eqref{valle-pussen-2-1.2} -- \eqref{valle-pussen-2-1.4} $ _2V_{n,m}(f)$ допускает следующее представление
$$
 _2V_{n,m}(f,x)=\frac{a_0}{2}+ \sum_{k=1}^m A_k(f,x)+
$$
$$
 \sum_{k=1}^{n-1}{2n^2-k(k+1)\over2n^2}A_{m+k}(f,x)+
$$
$$
 \sum_{j=0}^{n-2}{(n-j)(n-j-1)\over2n^2}A_{m+n+j}(f,x)
$$
или, что то же,
$$
 _2V_{n,m}(f,x)=\frac{a_0}{2}+ \sum_{k=1}^m A_k(f,x)+
$$
$$
 \sum_{k=m+1}^{m+n-1}{2n^2-(k-m)(k-m+1)\over2n^2}A_{k}(f,x)+
$$
\begin{equation}\label{valle-pussen-2-1.6}
 \sum_{k=m+n}^{m+2n-2}{(m+2n-k)(m+2n-k-1)\over2n^2}A_{k}(f,x).
\end{equation}
Из \eqref{valle-pussen-2-1.3} и \eqref{valle-pussen-2-1.4} вытекают следующие интегральные представления для операторов
$_1V_{n,m}(f)$ и $_2V_{n,m}(f)$:
\begin{equation}\label{valle-pussen-2-1.7}
 _1V_{n,m}(f,x)=\frac{1}{\pi}\int_{-\pi}^\pi f(x-t)_1v_{n,m}(t)dt,
 \end{equation}

\begin{equation}\label{valle-pussen-2-1.8}
 _2V_{n,m}(f,x)=\frac{1}{\pi}\int_{-\pi}^\pi f(x-t)_2v_{n,m}(t)dt,
 \end{equation}
где
\begin{equation}\label{valle-pussen-2-1.9}
 _1v_{n,m}(u)=\frac{1}{n}[D_m(u)+\ldots+D_{m+n-1}(u)],
 \end{equation}
\begin{equation}\label{valle-pussen-2-1.10}
 _2v_{n,m}(u)=\frac{1}{n}[_1v_{n,m}(u)+\ldots+_1v_{n,m+n-1}(u)],
 \end{equation}
 а
\begin{equation}\label{valle-pussen-2-1.11}
D_k(u) =\frac{1}{2}+\cos u +\ldots+\cos ku =\frac{\sin(k+\frac12)u}{2\sin\frac{u}{2}}
 \end{equation}
-- ядро Дирихле. Из \eqref{valle-pussen-2-1.9} и \eqref{valle-pussen-2-1.11} легко выводится следующие хорошо известные (см., например, \cite{valle-pussen-2-Zhuk}) равенства
\begin{equation}\label{valle-pussen-2-1.12}
_1v_{n,m}(u) =\frac{\cos mu-\cos (m+n)u}{4n\sin^2\frac{u}{2}}=
\frac{\sin\frac{nu}{2}\sin(2m+n)\frac{u}{2}}{2n\sin^2\frac{u}{2}}.
 \end{equation}
Из \eqref{valle-pussen-2-1.10} и \eqref{valle-pussen-2-1.12} имеем
\begin{equation*}
 _2v_{n,m}(u)=\frac{1}{4n^2\sin^2\frac{u}{2}}\left(\sum_{k=m}^{m+n-1}\cos ku-
 \sum_{k=m}^{m+n-1}\cos(k+n)u\right)=\frac{1}{8n^2\sin^3\frac{u}{2}}\times
 \end{equation*}
\begin{equation*}
 \left(\sin(m+n-\frac12)u-\sin(m-\frac12)u
 -\sin(m+2n-\frac12)u+\sin(m+n-\frac12)u\right)=
 \end{equation*}
\begin{equation*}
 \frac{\sin\frac{nu}{2}}{4n\sin^3\frac{u}{2}}
 \left(\cos(2m+n-1)\frac{u}{2}-\cos(2m+3n-1)\frac{u}{2}\right),
  \end{equation*}
поэтому
\begin{equation}\label{valle-pussen-2-1.13}
 _2v_{n,m}(u)=\frac{\sin^2\frac{nu}{2}\sin(m+n-\frac12)u}{2n^2\sin^3\frac{u}{2}}.
  \end{equation}
Интегральные представления \eqref{valle-pussen-2-1.7} и \eqref{valle-pussen-2-1.8} с учётом  \eqref{valle-pussen-2-1.12} и \eqref{valle-pussen-2-1.13}  принимают следующий вид
\begin{equation}\label{valle-pussen-2-1.14}
 _1V_{n,m}(f,x)=\frac{1}{2\pi n}\int_{-\pi}^\pi f(x-t) \frac{\sin\frac{nt}{2}\sin(2m+n)\frac{t}{2}}{\sin^2\frac{t}{2}}dt,
 \end{equation}
\begin{equation}\label{valle-pussen-2-1.15}
 _2V_{n,m}(f,x)=\frac{1}{2\pi n^2}\int_{-\pi}^\pi f(x-t) \frac{\sin^2\frac{nt}{2}\sin(2m+2n-1)\frac{t}{2}}{\sin^3\frac{t}{2}}dt.
 \end{equation}
Равенство \eqref{valle-pussen-2-1.14} хорошо известно и широко использовалось \cite{valle-pussen-2-Zhuk} при исследовании аппроксимативных свойств операторов $_1V_{n,m}(f)$ в различных функциональных пространствах.  Что же касается представления \eqref{valle-pussen-2-1.15}, то оно, как и сами операторы  $ _2V_{n,m}(f)$,
насколько известно автору, является новым.

 Отметим, что повторные средние Валле Пуссена $_kV_{n,m}(f,x)$, введенные выше для $k=2$, допускают дальнейшее обобщение на $k\ge2$    методом индукции следующим образом
\begin{equation}\label{valle-pussen-2-1.16}
  {}_{k}V_{n,m}(f,x)=\frac1{n}[ _{k-1}V_{n,m}(f,x)+\cdots+ _{k-1}V_{n,m+n-1}(f,x)],
\end{equation}
например, при $k=3$ мы можем записать
\begin{equation}\label{valle-pussen-2-1.17}
  _3V_{n,m}(f,x)=\frac1{n}[ _2V_{n,m}(f,x)+\cdots+ _2V_{n,m+n-1}(f,x)].
\end{equation}
Нетрудно также получить для $_kV_{n,m}(f,x)$ интегральное представление, аналогичное \eqref{valle-pussen-2-1.8} или \eqref{valle-pussen-2-1.15}. К примеру, для $k=3$ из \eqref{valle-pussen-2-1.15} и \eqref{valle-pussen-2-1.17} имеем
$$
 _3V_{n,m}(f,x)=\frac{1}{2\pi n^3}\int_{-\pi}^\pi f(x-t) \frac{\sin^2\frac{nt}{2}}{\sin^3\frac{t}{2}}\sum_{l=m}^{m+n-1}\sin(l+n-\frac12)tdt
$$
  \begin{equation}\label{valle-pussen-2-1.18}
= \frac{1}{2\pi n^3}\int_{-\pi}^\pi f(x-t) \frac{\sin^3\frac{nt}{2}\sin(2m+3n-2)\frac{t}{2}}{\sin^4\frac{t}{2}}dt,
  \end{equation}
и, вообще, если $k\ge2$, то
 \begin{equation}\label{valle-pussen-2-1.19}
_kV_{n,m}(f,x)= \frac{1}{2\pi n^k}\int\limits_{-\pi}^\pi f(x-t) \frac{\sin^k\frac{nt}{2}\sin(2m+k(n-1)+1)\frac{t}{2}}{\sin^{k+1}\frac{t}{2}}dt.
  \end{equation}






Обозначим через $C_{2\pi}$ пространство непрерывных $2\pi$-периодических функций $f$  с нормой $f=\max_x|f(x)|$ и рассмотрим $_1V_{n,m}=_1V_{n,m}(f)$ и $_2V_{n,m}={}_2V_{n,m}(f)$
как операторы, действующие в нормированном пространстве $C_{2\pi}$. Из интегральных представлений \eqref{valle-pussen-2-1.14} и \eqref{valle-pussen-2-1.15} непосредственно следует, что нормы этих операторов равны:
\begin{equation}\label{valle-pussen-2-1.20}
\|_1V_{n,m}\|=\frac{1}{2\pi n}\int_{-\pi}^\pi \frac{|\sin\frac{nt}{2}\sin(2m+n)\frac{t}{2}|}{\sin^2\frac{t}{2}}dt,
 \end{equation}
\begin{equation}\label{valle-pussen-2-1.21}
\|_2V_{n,m}\|=\frac{1}{2\pi n^2}\int_{-\pi}^\pi  \frac{\sin^2\frac{nt}{2}|\sin(m+n-\frac12)t|}{|\sin^3\frac{t}{2}|}dt.
 \end{equation}
Из \eqref{valle-pussen-2-1.5}, \eqref{valle-pussen-2-1.14}  и \eqref{valle-pussen-2-1.15} непосредственно вытекают следующие неравенства
\begin{equation}\label{valle-pussen-2-1.22}
\|f-_1V_{n,m}\|\le E_m(f)(1+\|_1V_{n,m}\|),
 \end{equation}
\begin{equation}\label{valle-pussen-2-1.23}
\|f-_2V_{n,m}\|\le E_m(f)(1+\|_2V_{n,m}\|),
 \end{equation}
где $E_m(f)$ -- наилучшее приближение функции $f\in C_{2\pi}$ тригонометрическими полиномами $T_m$ порядка $m$. Величина $\|_1V_{n,m}\|$ достаточно хорошо изучена \cite{valle-pussen-2-NIK} -- \cite{valle-pussen-2-Zhuk}. Например, в \cite{valle-pussen-2-Zhuk} установлена следующая оценка
\begin{equation}\label{valle-pussen-2-1.24}
\|_1V_{n,m}\|\le \frac{4}{\pi^2}\ln\left(1+\frac{m+n}{n}\right)+1,7.
 \end{equation}
С другой стороны, из \eqref{valle-pussen-2-1.8} и \eqref{valle-pussen-2-1.10} следует, что
\begin{equation}\label{valle-pussen-2-1.25}
\|_2V_{n,m}\|\le \frac1n\sum_{k=m}^{m+n-1}\|_1V_{n,k}\|.
 \end{equation}
Из \eqref{valle-pussen-2-1.24} и \eqref{valle-pussen-2-1.25} выводим
\begin{equation}\label{valle-pussen-2-1.22}
\|_2V_{n,m}\|\le \frac{4}{\pi^2}\ln\left(3+\frac{m-1}{n}\right)+1,7.
 \end{equation}
 Оценка \eqref{valle-pussen-2-1.22}, скорее всего, не является окончательной, но этот вопрос в настоящей статье рассматриваться не будет. Мы сосредоточим внимание на исследовании локальных аппроксимативных свойств повторных средних Валле Пуссена $_2V_{n,m}(f,x)$ для кусочно гладких функций, которые, как показано в \ref{valle-pussen-ss3}, существенно отличаются от соответствующих свойств  сумм Фурье $S_k(f,x)$ и классических средних Валле Пуссена $_1V_{n,m}(f,x)$. Если, к примеру, мы рассмотрим кусочно постоянную функцию $f(x)=sign\sin x$, то из результатов, полученных в \ref{valle-pussen-ss3} вытекает оценка
\begin{equation}\label{valle-pussen-2-1.27}
|f(x)-_2V_{n,n}(f,x)|\le \frac{c(\varepsilon)}{n^3} \quad(|x-k\pi|>\varepsilon, k\in \mathbb{Z}),
 \end{equation}
 тогда как для сумм Фурье $S_n(f,x)$  и классических средних Валле Пуссена $_1V_{n,n}(f,x)$
имеют место соотношения
\begin{equation}\label{valle-pussen-2-1.28}
\max_{x\atop|x-k\pi|>\varepsilon, k\in \mathbb{Z} }|f(x)-S_n(f,x)|\asymp n^{-1},
 \end{equation}
\begin{equation}\label{valle-pussen-2-1.29}
\max_{x\atop|x-k\pi|>\varepsilon, k\in \mathbb{Z} }|f(x)- _1V_{n,n}(f,x)|\asymp n^{-2}.
 \end{equation}
 В оценке \eqref{valle-pussen-2-1.27} и всюду в дальнейшем через $c$, $c(a),\ c(a,b),\ c_k(a,b),\ \ldots$ обозначаются положительные постоянные, зависящие лишь от указанных параметров, вообще говоря различные в  разных местах.
В \ref{valle-pussen-ss3} показано, что совершенно аналогичная ситуация имеет место для любой кусочно-гладкой $2\pi$-периодической функции $f$ с конечным числом точек разрыва первого рода на периоде. Это свойство делает $_2V_{n,n}(f,x)$ весьма привлекательным инструментом решения важных прикладных задач таких, например, как конструирование цифровых фильтров, обработка и сжатие речи и т.д.




\section{Некоторые вспомогательные результаты}\label{valle-pussen-ss2}
В дальнейшем нам понадобятся некоторые классы кусочно-гладких $2\pi$ - периодических функций,
которых мы определим в настоящем разделе. Пусть $-\pi=x_0<x_1<\cdots<x_s<x_{s+1}=\pi$,
$f(x)$ -- $2\pi$-периодическая функция такая, что на каждом из отрезков $[x_j,x_{j+1}]$ $(j=0,1,\ldots,s)$ её можно превратить в абсолютно непрерывную функцию путем переопределения на концах $x_j$ и $x_{j+1}$. Множество всех таких функций мы обозначим  через $\mathcal{ P}_\Omega$, где $\Omega=\{x_0,x_1,\ldots x_s,x_{s+1}\}$.  Через $W_1^r(a,b)$ обозначим пространство Соболева, состоящее из функций $f$, $r-1$-раз непрерывно дифференцируемых на $[a,b]$, для которых $f^{(r-1)}$ абсолютно непрерывна, а $f^{(r)}$ интегрируема на $[a,b]$. Если $f\in \mathcal{ P}_\Omega$ и ее сужение на $[x_j,x_{j+1}]$ можно переопределить в точках
$x_j$ и $x_{j+1}$ так, чтобы было $f\in W_1^r(x_j,x_{j+1})$, то мы будем говорить, что $f\in _\Omega W_1^r$. Если функция $f\in_\Omega W_1^r$, то будем называть её кусочно-гладкой (порядка $r$). Исследование локальных аппроксимативных свойств повторных средних Валле Пуссена $_2V_{n,n}(f,x)$ для функций $f\in_\Omega W_1^r$ является основной задачей настоящей работы. Для этого нам потребуется ряд вспомогательных утверждений.

Если $f\in \mathcal{ P}_\Omega$, то ее ряд Фурье сходится к ней в каждой её точке непрерывности и допускает представление
\begin{equation}\label{valle-pussen-2-2.1}
    f(x) = \frac{a_0}{2}+ \sum_{k=1}^\infty a_k\cos kx+b_k\sin kx,
\end{equation}
в котором для коэффициенты Фурье можно записать следующие равенства
\begin{equation}\label{valle-pussen-2-2.2}
a_k(f)=\frac1\pi\int_{-\pi}^\pi f(t)\cos kt=\sum_{j=0}^s a_k^j(f),
   \end{equation}
\begin{equation}\label{valle-pussen-2-2.3}
b_k(f)=\frac1\pi\int_{-\pi}^\pi f(t)\sin kt=\sum_{j=0}^s b_k^j(f),
   \end{equation}
\begin{equation}\label{valle-pussen-2-2.4}
a_k^j(f)=\frac1\pi\int_{x_j}^{x_{j+1}} f(t)\cos kt,
   \end{equation}
\begin{equation}\label{valle-pussen-2-2.5}
b_k^j(f)=\frac1\pi\int_{x_j}^{x_{j+1}} f(t)\sin kt.
   \end{equation}
 Положим
\begin{equation}\label{valle-pussen-2-2.6}
R_m(f,x)=f(x)-S_m(f,x)=\sum_{k=m+1}^\infty a_k\cos kx+b_k\sin kx,
   \end{equation}
тогда из \eqref{valle-pussen-2-2.2} -- \eqref{valle-pussen-2-2.6} имеем
\begin{equation}\label{valle-pussen-2-2.7}
R_m(f,x)=\sum_{j=0}^s\sum_{k=m+1}^\infty a_k^j\cos kx+b_k^j\sin kx.
   \end{equation}
Если $f\in _\Omega W_1^r$, то, $r$-кратно применяя метод интегрирования по частям, имеем
$$
a_k^j\cos kx+b_k^j\sin kx=\frac1\pi\int_{x_j}^{x_{j+1}}f(t)\cos k(t-x)dt=
$$
$$
\frac{1}{\pi k}(f(x_j)\sin k(x-x_j)-f(x_{j+1})\sin k(x-x_{j+1}))-\frac1{\pi k}\int\limits_{x_j}^{x_{j+1}}f'(t)\sin k(t-x)dt=
$$
$$
\frac{1}{\pi}
\sum_{\nu=1}^r\frac{1}{k^\nu}f^{(\nu-1)}(x_j)\cos\left(k(x-x_j)-\frac{\nu\pi}{2}\right)-
$$
$$
\frac{1}{\pi}
\sum_{\nu=1}^r\frac{1}{k^\nu} f^{(\nu-1)}(x_{j+1})\cos\left(k(x-x_{j+1})-\frac{\nu\pi}{2}\right)+
$$
\begin{equation}\label{valle-pussen-2-2.8}
\frac1{\pi k^r}\int\limits_{x_j}^{x_{j+1}}f^{(r)}(t)\cos\left(k(t-x)+\frac{\pi r}{2}\right)dt.
   \end{equation}
Из \eqref{valle-pussen-2-2.7} и \eqref{valle-pussen-2-2.8} мы заключаем, что справедлива
\begin{lemma}\label{valle-pussen-l2.1}
Если $f\in _\Omega W_1^r$, то имеет место равенство
\begin{equation}\label{valle-pussen-2-2.9}
R_l(f,x)= \hat R_l(f,x)+\tilde R_l(f,x),
   \end{equation}
в котором
$$
 \hat R_l(f,x)=\frac{1}{\pi}\sum_{j=0}^s
\sum_{\nu=1}^rf^{(\nu-1)}(x_j)\sum_{k=l+1}^\infty
{\cos\left(k(x-x_j)-\frac{\nu\pi}{2}\right)\over k^r}
$$
\begin{equation}\label{valle-pussen-2-2.10}
-\frac{1}{\pi}\sum_{j=0}^s
\sum_{\nu=1}^r f^{(\nu-1)}(x_{j+1})\sum_{k=l+1}^\infty
{\cos\left(k(x-x_{j+1})-\frac{\nu\pi}{2}\right)\over k^r},
\end{equation}
\begin{equation}\label{valle-pussen-2-2.11}
\tilde R_l(f,x)=\frac1{\pi}\int\limits_{-\pi}^{\pi}f^{(r)}(t)\sum_{k=l+1}^\infty
{\cos\left(k(t-x)+\frac{\pi r}{2}\right)\over k^r}dt.
\end{equation}
\end{lemma}
Положим
\begin{equation}\label{valle-pussen-2-2.12}
_1R_{n,m}(f,x)=f(x)-_1V_{n,m}(f,x),
\end{equation}
\begin{equation}\label{valle-pussen-2-2.13}
_2R_{n,m}(f,x)=f(x)-_2V_{n,m}(f,x)
\end{equation}
и заметим, что из \eqref{valle-pussen-2-1.3}, \eqref{valle-pussen-2-1.4} и \eqref{valle-pussen-2-2.6}, \eqref{valle-pussen-2-1.12} и \eqref{valle-pussen-2-2.13} вытекают следующие равенства
$$
_1R_{n,m}(f,x)=\frac1n\sum_{k=m}^{m+n-1}R_k(f,x),
$$
$$
_2R_{n,m}(f,x)=\frac1n\sum_{k=m}^{m+n-1}
{_1}R_{n,k}(f,x)=\frac1{n^2}\sum_{k=m}^{m+n-1}
\sum_{l=k}^{k+n-1}R_l(f,x).
$$
Если теперь обратимся к лемме \ref{valle-pussen-l2.1}, то эти равенства можно переписать так
\begin{equation}\label{valle-pussen-2-2.14}
_1R_{n,m}(f,x)=_1\hat R_{n,m}(f,x)+_1\tilde R_{n,m}(f,x),
\end{equation}
\begin{equation}\label{valle-pussen-2-2.15}
_2R_{n,m}(f,x)=_2\hat R_{n,m}(f,x)+_2\tilde R_{n,m}(f,x),
\end{equation}
где
\begin{equation}\label{valle-pussen-2-2.16}
_1\hat R_{n,m}(f,x)=\frac1{n}
\sum_{l=m}^{m+n-1}\hat R_l(f,x),\quad_1\tilde R_{n,m}(f,x)=\frac1{n}
\sum_{l=m}^{m+n-1}\tilde R_l(f,x),
\end{equation}
\begin{equation}\label{valle-pussen-2-2.17}
_2\hat R_{n,m}(f,x)=\frac1{n^2}\sum_{k=m}^{m+n-1}
\sum_{l=k}^{k+n-1}\hat R_l(f,x),
\end{equation}
\begin{equation}\label{valle-pussen-2-2.18}
_2\tilde R_{n,m}(f,x)=\frac1{n^2}\sum_{k=m}^{m+n-1}
\sum_{l=k}^{k+n-1}\tilde R_l(f,x),
\end{equation}
 а величины $\hat R_l(f,x)$  и $\tilde R_l(f,x)$ определены равенствами \eqref{valle-pussen-2-2.10} и \eqref{valle-pussen-2-2.11}. Займемся  вопросом об оценках для этих двух величин, остановившись сначала на  $\tilde R_l(f,x)$.
\begin{lemma}\label{valle-pussen-l2.2}
Пусть $r\ge2$, $f\in _\Omega W^r_1$. Тогда имеет место оценка
\begin{equation}\label{valle-pussen-2-2.19}
|\tilde R_l(f,x)|\le \frac{c(f,r)}{l^{r-1}}.
\end{equation}
\end{lemma}

\begin{lemma}\label{valle-pussen-l2.3}
Пусть $r\ge4$, $f\in _\Omega W^r_1$. Тогда имеет место оценка
\begin{equation*}
|_2\tilde R_l(f,x)|\le \frac{c(r)I_r(f)}{(m+n)^2m^{r-3}}, \quad |_1\tilde R_l(f,x)|\le \frac{c(r)I_r(f)}{(m+n)m^{r-2}}
\end{equation*}
где $I_r(f)=\int_{-\pi}^\pi|f^{(r)}(t)|dt$.
\end{lemma}

Прежде, чем перейти к вопросу об оценках для величин $|\hat R_l(f,x)|$, $|_1\hat R_l(f,x)|$ и $|_2\hat R_l(f,x)|$, мы предварительно докажем некоторые вспомогательные утверждения. Положим
\begin{equation}\label{valle-pussen-2-2.20}
\mathcal{ K}_l^\nu(u)= \sum_{z=l+1}^{\infty}{\cos(zu+\frac{\nu\pi}{2})\over z^\nu},
\end{equation}
\begin{equation}\label{valle-pussen-2-2.21}
_1\mathcal{ K}_{n,m}^\nu(u)=\frac1n \sum_{l=m}^{m+n-1}\mathcal{ K}_l^\nu(u),
\end{equation}
\begin{equation}\label{valle-pussen-2-2.22}
_2\mathcal{ K}_{n,m}^\nu(u)=\frac1n \sum_{k=m}^{m+n-1} {_1}\mathcal{ K}_{n,k}^\nu(u)=
\frac1{n^2} \sum_{k=m}^{m+n-1}\sum_{l=k}^{k+n-1}\mathcal{ K}_l^\nu(u).
\end{equation}


\begin{lemma}\label{valle-pussen-l2.4} Имеют место следующие равенства
 $$
 _1\mathcal{ K}_{n,k}^{2\mu}(u)=
 $$
 $$
 {(-1)^\mu\over n}\sum\limits_{\lambda=1}^{n-1}
\sum\limits_{\kappa=1}^{\infty}\frac{\sin\frac{\lambda}{2}u\sin\frac{\kappa}{2}u
\cos(k+\frac{\kappa+\lambda}{2})u}{\sin^2\frac u2}\Delta^2g_\mu(k+\kappa+\lambda-1)
$$
 \begin{equation}\label{valle-pussen-2-2.23}
    +{(-1)^{\mu-1}\over n}\sum\limits_{\kappa=1}^{\infty}
\frac{\sin\frac {n}{2}u\sin\frac{\kappa}{2}u\cos(k+\frac{\kappa+n}{2})u}{\sin^2\frac u2}
\Delta g_\mu(k+n+\kappa-1),
 \end{equation}
 $$
 _1\mathcal{ K}_{n,k}^{2\mu-1}(u)=
 $$
 $$
 {(-1)^\mu\over n}\sum\limits_{\lambda=1}^{n-1}
\sum\limits_{\kappa=1}^{\infty}\frac{\sin\frac{\lambda}{2}u\sin\frac{\kappa}{2}u
\sin(k+\frac{\kappa+\lambda}{2})u}{\sin^2\frac u2}\Delta^2q_\mu(k+\kappa+\lambda-1)
$$
 \begin{equation}\label{valle-pussen-2-2.24}
    +{(-1)^{\mu-1}\over n}\sum\limits_{\kappa=1}^{\infty}
\frac{\sin\frac {n}{2}u\sin\frac{\kappa}{2}u\sin(k+\frac{\kappa+n}{2})u}{\sin^2\frac u2}
\Delta q_\mu(k+n+\kappa-1),
 \end{equation}



 где $g_\mu(t)=t^{-2\mu}$, $q_\mu(t)=t^{-2\mu+1}$, $\Delta\varphi(t)=\varphi(t+1)-\varphi(t)$,
 $\Delta^2\varphi(t)=\varphi(t+2)-2\varphi(t+1)+\varphi(t)$.
\end{lemma}

Вернемся к равенству \eqref{valle-pussen-2-2.22} и рассмотрим случай $\nu=2\mu$ -- четное.
Тогда в силу леммы 2.4 мы можем записать
$$
 _2\mathcal{ K}_{n,m}^{2\mu}(u)=
 $$
 $$
 {(-1)^\mu\over n^2}\sum\limits_{\lambda=1}^{n-1}
\sum\limits_{\kappa=1}^{\infty}\frac{\sin\frac{\lambda}{2}u\sin\frac{\kappa}{2}u}
{\sin^2\frac u2}
\sum_{k=m}^{m+n-1}\cos(k+\frac{\kappa+\lambda}{2})u\Delta^2g_\mu(k+\kappa+\lambda-1)
$$
 \begin{equation}\label{valle-pussen-2-2.35}
    +{(-1)^{\mu-1}\over n^2}\sum\limits_{\kappa=1}^{\infty}
\frac{\sin\frac {n}{2}u\sin\frac{\kappa}{2}u}{\sin^2\frac u2}
\sum_{k=m}^{m+n-1}\cos(k+\frac{\kappa+n}{2})u\Delta g_\mu(k+n+\kappa-1).
 \end{equation}
 К внутренним суммам вида $\sum_{k=m}^{m+n-1}$, фигурирующим в правой части равенства \eqref{valle-pussen-2-2.35}, применим преобразование Абеля, что дает
 $$
 \sum_{k=m}^{m+n-1}\cos(k+\frac{\kappa+\lambda}{2})u\Delta^2g_\mu(k+\kappa+\lambda-1)=
 $$
 \begin{equation}\label{valle-pussen-2-2.36}
-\sum_{k=m}^{m+n-2}\Delta^3g_\mu(k+\kappa+\lambda-1)X_k^{\lambda,\kappa}
+\Delta^2g_\mu(m+n+\kappa+\lambda-2)X_{m+n-1}^{\lambda,\kappa},
\end{equation}
$$
\sum_{k=m}^{m+n-1}\cos(k+\frac{\kappa+n}{2})u\Delta g_\mu(k+n+\kappa-1)=
$$
 \begin{equation}\label{valle-pussen-2-2.37}
-\sum_{k=m}^{m+n-2}\Delta^2g_\mu(k+n+\kappa-1)X_k^{n,\kappa}
+\Delta g_\mu(m+2n+\kappa-2)X_{m+n-1}^{n,\kappa},
    \end{equation}
где
$$
X_k^{\lambda,\kappa}=\sum_{j=m}^k \cos(ju+\frac{(\kappa+\lambda)u}{2})=
$$
$$
\frac{\cos(k+\kappa+\lambda)\frac{u}{2}\sin(k+1)\frac{u}{2}-
\cos(m+\kappa+\lambda-1)\frac{u}{2}\sin\frac{mu}{2}}{\sin\frac{u}{2}}=
$$
$$
=\frac{\sin(2k+\kappa+\lambda+1)\frac{u}{2}-\sin(2m+\kappa+\lambda-1)\frac{u}{2}}{2\sin\frac{u}{2}}
$$
 \begin{equation}\label{valle-pussen-2-2.38}
=\frac{\sin(k-m+1)\frac{u}{2}\cos(m+k+\kappa+\lambda)\frac{u}{2}}{\sin\frac{u}{2}}.
    \end{equation}
Из равенств \eqref{valle-pussen-2-2.35} -- \eqref{valle-pussen-2-2.38} мы выводим следующий результат.

\begin{lemma}\label{valle-pussen-l2.5}
Имеет место равенство
$$
_2\mathcal{ K}_{n,m}^{2\mu}(u)={(-1)^{\mu-1}\over n^2\sin^3\frac{u}{2}}\left(\sum\limits_{\lambda=1}^{n-1}\sum_{k=1}^{n-1}
\sum\limits_{\kappa=1}^{\infty}\Delta^3g_\mu(m+k+\kappa+\lambda-2)\times\right.
$$
$$
\sin\frac{\lambda u}{2}\sin\frac{\kappa u}{2}\sin\frac{ku}{2}\cos(2m+k+\kappa+\lambda-1)\frac{u}{2}
$$
$$
 -\sum\limits_{\lambda=1}^{n-1}
\sum\limits_{\kappa=1}^{\infty}\Delta^2g_\mu(m+n+\kappa+\lambda-2)\times
$$
$$
\sin\frac{\lambda u}{2}\sin\frac{\kappa u}{2}\sin\frac{nu}{2}\cos(2m+n+\kappa+\lambda-1)\frac{u}{2}
$$
$$
 -\sum_{k=1}^{n-1}
\sum\limits_{\kappa=1}^{\infty}\Delta^2g_\mu(m+n+k+\kappa-2)\times
$$
$$
\sin\frac{nu}{2}\sin\frac{\kappa u}{2}\sin\frac{ku}{2}\cos(2m+n+k+\kappa-1)\frac{u}{2}
$$
$$
 \left.+\sum\limits_{\kappa=1}^{\infty}\Delta g_\mu(m+2n+\kappa-2)
\sin^2\frac{nu}{2}\sin\frac{\kappa u}{2}\cos(2m+2n+\kappa-1)\frac{u}{2}\right).
$$
\end{lemma}
Рассмотрим \eqref{valle-pussen-2-2.22} в случае $\nu=2\mu-1$ --  нечетно. Из леммы \ref{valle-pussen-l2.4} (равенство \eqref{valle-pussen-2-2.24}) имеем
$$
 _2\mathcal{ K}_{n,m}^{2\mu-1}(u)=
 $$
 $$
 {(-1)^\mu\over n^2}\sum\limits_{\lambda=1}^{n-1}
\sum\limits_{\kappa=1}^{\infty}\frac{\sin\frac{\lambda}{2}u\sin\frac{\kappa}{2}u}
{\sin^2\frac u2}
\sum_{k=m}^{m+n-1}\Delta^2g_\mu(k+\kappa+\lambda-1)\sin(ku+(\kappa+\lambda)\frac{u}{2})
$$
 \begin{equation}\label{valle-pussen-2-2.39}
    -{(-1)^{\mu}\over n^2}\sum\limits_{\kappa=1}^{\infty}
\frac{\sin\frac {n}{2}u\sin\frac{\kappa}{2}u}{\sin^2\frac u2}
\sum_{k=m}^{m+n-1}\Delta g_\mu(k+n+\kappa-1)\sin(ku+(n+\kappa)\frac{u}{2}).
 \end{equation}
К внутренним суммам вида $\sum_{k=m}^{m+n-1}$, фигурирующим в правой части равенства \eqref{valle-pussen-2-2.39}, применим преобразование Абеля, что дает
 $$
 \sum_{k=m}^{m+n-1}\Delta^2g_\mu(k+\kappa+\lambda-1)\sin(ku+(\kappa+\lambda)\frac{u}{2})=
 $$
 \begin{equation}\label{valle-pussen-2-2.40}
-\sum_{k=m}^{m+n-2}\Delta^3q_\mu(k+\kappa+\lambda-1)Z_k^{\lambda,\kappa}
+\Delta^2q_\mu(m+n+\kappa+\lambda-2)Z_{m+n-1}^{\lambda,\kappa},
\end{equation}
$$
\sum_{k=m}^{m+n-1}\sin(k+\frac{\kappa+n}{2})u\Delta q_\mu(k+n+\kappa-1)=
$$
 \begin{equation}\label{valle-pussen-2-2.41}
-\sum_{k=m}^{m+n-2}\Delta^2q_\mu(k+n+\kappa-1)Z_k^{n,\kappa}
+\Delta q_\mu(m+2n+\kappa-2)Z_{m+n-1}^{n,\kappa},
    \end{equation}
где
$$
Z_k^{\lambda,\kappa}=\sum_{j=m}^k \sin(ju+\frac{(\kappa+\lambda)u}{2})=
$$
$$
\frac{\sin(k+\kappa+\lambda)\frac{u}{2}\sin(k+1)\frac{u}{2}-
\sin(m+\kappa+\lambda-1)\frac{u}{2}\sin\frac{mu}{2}}{\sin\frac{u}{2}}
$$
$$
=\frac{\cos(2k+\kappa+\lambda+1)\frac{u}{2}-\cos(\kappa+\lambda-1)\frac{u}{2}}{2\sin\frac{u}{2}}
$$
 \begin{equation}\label{valle-pussen-2-2.42}
=\frac{\sin(k-m+1)\frac{u}{2}\sin(m+k+\kappa+\lambda)\frac{u}{2}}{\sin\frac{u}{2}}.
    \end{equation}
Из равенств \eqref{valle-pussen-2-2.39} -- \eqref{valle-pussen-2-2.42} мы выводим следующий результат.

\begin{lemma}\label{valle-pussen-l2.6}
Имеет место равенство
$$
_2\mathcal{ K}_{n,m}^{2\mu-1}(u)={(-1)^{\mu-1}\over n^2\sin^3\frac{u}{2}}\left(\sum\limits_{\lambda=1}^{n-1}\sum_{k=1}^{n-1}
\sum\limits_{\kappa=1}^{\infty}\Delta^3q_\mu(m+k+\kappa+\lambda-2)\times\right.
$$
$$
\sin\frac{\lambda u}{2}\sin\frac{\kappa u}{2}\sin\frac{ku}{2}\sin(2m+k+\kappa+\lambda-1)\frac{u}{2}
$$
$$
 -\sum\limits_{\lambda=1}^{n-1}
\sum\limits_{\kappa=1}^{\infty}\Delta^2q_\mu(m+n+\kappa+\lambda-2)\times
$$
$$
\sin\frac{\lambda u}{2}\sin\frac{\kappa u}{2}\sin\frac{nu}{2}\sin(2m+n+\kappa+\lambda-1)\frac{u}{2}
$$
$$
 -\sum_{k=1}^{n-1}
\sum\limits_{\kappa=1}^{\infty}\Delta^2q_\mu(m+n+k+\kappa-2)\times
$$
$$
\sin\frac{nu}{2}\sin\frac{\kappa u}{2}\sin\frac{ku}{2}\sin(2m+n+k+\kappa-1)\frac{u}{2}
$$
$$
 \left.+\sum\limits_{\kappa=1}^{\infty}\Delta q_\mu(m+2n+\kappa-2)
\sin^2\frac{nu}{2}\sin\frac{\kappa u}{2}\sin(2m+2n+\kappa-1)\frac{u}{2}\right).
$$
\end{lemma}

\begin{lemma}\label{valle-pussen-l2.7} Имеют место равенства
$$
\mathcal{ K}_l^{2\mu}(u)=
(-1)^{\mu-1}\sum_{\kappa=1}^{\infty}
\Delta g_\mu(\kappa+l){\sin\frac{\kappa u}{2}\cos(2l+\kappa+1)\frac{u}{2}\over\sin\frac{u}{2}},
$$
$$
\mathcal{ K}_l^{2\mu-1}(u)=
(-1)^{\mu-1}\sum_{\kappa=1}^{\infty}
\Delta q_\mu(\kappa+l){\sin\frac{\kappa u}{2}\sin(2l+\kappa+1)\frac{u}{2}\over\sin\frac{u}{2}}.
$$

где $g_\mu(t)=t^{-2\mu}$, $q_\mu(t)=t^{-2\mu-1}$.
\end{lemma}

\section{Приближение кусочно-гладких функций}\label{valle-pussen-ss3}


Вернемся теперь к вопросу об оценке величин  $ |R_n(f,x)|$, $| _1R_{n,m}(f,x)|$ и
$|_2R_{n,m}(f,x)|$ \linebreak(см.\eqref{valle-pussen-2-2.6}, \eqref{valle-pussen-2-2.12},  \eqref{valle-pussen-2-2.13}). Поскольку для этих величин
имеют место представления \eqref{valle-pussen-2-2.9}, \eqref{valle-pussen-2-2.14} и \eqref{valle-pussen-2-2.15}, а величины
$|\tilde R_n(f,x)|$, $|_1\tilde R_{n,m}(f,x)|$ и  $|_2\tilde R_{n,m}(f,x)|$ уже оценены (см. леммы \ref{valle-pussen-l2.2} и \ref{valle-pussen-l2.3}), то нам остаётся рассмотреть
 $|\hat R_n(f,x)|$, $|_1\hat R_{n,m}(f,x)|$ и  $|_2\hat R_{n,m}(f,x)|$ (см. \eqref{valle-pussen-2-2.10}, \eqref{valle-pussen-2-2.16},   \eqref{valle-pussen-2-2.17}). С этой целью введем следующие обозначения. Если $f\in _\Omega W_1^r$, то положим
\begin{equation}\label{valle-pussen-2-3.1}
J_{\Omega,r}(f)=\max\{|f^{(\nu)}(x_j)|: x_j\in\Omega,0\le\nu\le r-1\},
\end{equation}
\begin{equation}\label{valle-pussen-2-3.2}
Q_{\Omega,\varepsilon}=\bigcup_{j=0}^s[x_j+\varepsilon,x_{j+1}+\varepsilon],
\end{equation}
где $\quad 0<\varepsilon<\frac12\min\{x_{j+1}-x_{j}:0\le j\le s\}$ и
заметим, что в силу \eqref{valle-pussen-2-2.10} и \eqref{valle-pussen-2-2.20} мы можем записать
$$
 \hat R_l(f,x)=\frac{1}{\pi}\sum_{j=0}^s
\sum_{\nu=1}^rf^{(\nu-1)}(x_j)\mathcal{ K}_l^\nu(x-x_j)
$$
\begin{equation}\label{valle-pussen-2-3.3}
-\frac{1}{\pi}\sum_{j=0}^s
\sum_{\nu=1}^r f^{(\nu-1)}(x_{j+1})\mathcal{ K}_l^\nu(x-x_{j+1}),
\end{equation}
поэтому из \eqref{valle-pussen-2-2.16}, \eqref{valle-pussen-2-2.17}, \eqref{valle-pussen-2-2.21} и \eqref{valle-pussen-2-2.22} находим
$$
 _1\hat R_{n,m}(f,x)=\frac{1}{\pi}\sum_{j=0}^s
\sum_{\nu=1}^rf^{(\nu-1)}(x_j)_1\mathcal{ K}_{n,m}^\nu(x-x_j)
$$
\begin{equation}\label{valle-pussen-2-3.4}
-\frac{1}{\pi}\sum_{j=0}^s
\sum_{\nu=1}^r f^{(\nu-1)}(x_{j+1})_1\mathcal{ K}_{n,m}^\nu(x-x_{j+1}),
\end{equation}
$$
 _2\hat R_{n,m}(f,x)=\frac{1}{\pi}\sum_{j=0}^s
\sum_{\nu=1}^rf^{(\nu-1)}(x_j)_2\mathcal{ K}_{n,m}^\nu(x-x_j)
$$
\begin{equation}\label{valle-pussen-2-3.5}
-\frac{1}{\pi}\sum_{j=0}^s
\sum_{\nu=1}^r f^{(\nu-1)}(x_{j+1})_2\mathcal{ K}_{n,m}^\nu(x-x_{j+1}).
\end{equation}
Чтобы оценить величины $|_2\mathcal{ K}_{n,m}^\nu(x-x_{j}|$ обратимся к леммам \ref{valle-pussen-l2.5} и \ref{valle-pussen-l2.6}. Если $x\in Q_{\Omega,\varepsilon}$, то $|\sin(x-x_j)/2|>\sin(\varepsilon/2)$, поэтому в силу указанных лемм мы можем записать
$$
|_2\mathcal{ K}_{n,m}^{\nu}(x-x_j)|\le{c(\nu,\varepsilon)\over n^2}
\left(\sum\limits_{\lambda=1}^{n-1}\sum_{k=1}^{n-1}
\sum\limits_{\kappa=1}^{\infty}|\Delta^3F(m+k+\kappa+\lambda-2)|+\right.
$$
$$
 \sum\limits_{\lambda=1}^{n-1}
\sum\limits_{\kappa=1}^{\infty}|\Delta^2F(m+n+\kappa+\lambda-2)|
 +\sum_{k=1}^{n-1}
\sum\limits_{\kappa=1}^{\infty}\Delta^2|F(m+n+k+\kappa-2)|
$$
$$
 \left.+\sum\limits_{\kappa=1}^{\infty}|\Delta F(m+2n+\kappa-2)|\right),
$$
где $F(t)=t^\nu $, $\nu\ge1$. Отсюда следует, что если $x\in Q_{\Omega,\varepsilon}$, то
\begin{equation}\label{valle-pussen-2-3.6}
|_2\mathcal{ K}_{n,m}^{\nu}(x-x_j)|\le{c(\nu,\varepsilon)\over n^2(m+1)}.
\end{equation}
Аналогично, из леммы \ref{valle-pussen-l2.4}  выводим следующую оценку
\begin{equation}\label{valle-pussen-2-3.7}
|_1\mathcal{ K}_{n,m}^{\nu}(x-x_j)|\le{c(\nu,\varepsilon)\over n(m+1)}, \quad x\in Q_{\Omega,\varepsilon}, 0\le j\le s+1,
\end{equation}
а из леммы \ref{valle-pussen-l2.7}  вытекает оценка
\begin{equation}\label{valle-pussen-2-3.8}
|\mathcal{ K}_{m}^{\nu}(x-x_j)|\le{c(\nu,\varepsilon)\over m+1}, \quad x\in Q_{\Omega,\varepsilon}, 0\le j\le s+1.
\end{equation}





Сопоставляя \eqref{valle-pussen-2-3.6} -- \eqref{valle-pussen-2-3.8} c \eqref{valle-pussen-2-3.3} -- \eqref{valle-pussen-2-3.5}, мы можем сформулировать следующее утверждение.
\begin{lemma}\label{valle-pussen-l3.1}
Пусть $r \ge4$, $f\in _\Omega W_1^r$. Тогда если $x\in Q_{\Omega,\varepsilon}$, то
имеют место оценки
$$
|\hat R_{m}(f,x)|\le{c(r,\varepsilon)J_{\Omega,r}(f)\over m+1},
$$
$$
|_1\hat R_{n,m}(f,x)|\le{c(r,\varepsilon)J_{\Omega,r}(f)\over n(m+1)},
$$
$$
|_2\hat R_{n,m}(f,x)|\le{c(r,\varepsilon)J_{\Omega,r}(f)\over n^2(m+1)}.
$$
\end{lemma}

Из лемм \ref{valle-pussen-l2.2}, \ref{valle-pussen-l2.3}, \ref{valle-pussen-l3.1} и равенств  \eqref{valle-pussen-2-2.9}, \eqref{valle-pussen-2-2.14},  \eqref{valle-pussen-2-2.15} выводим следующий основной результат настоящего пункта.
\begin{theorem}\label{valle-pussen-t1}
 Пусть $-\pi=x_0<x_1<\cdots<x_s<x_{s+1}=\pi$, $\Omega=\{x_0,x_1,\ldots x_s,x_{s+1}\}$,  $\quad 0<\varepsilon<\frac12\min\{x_{j+1}-x_{j}:0\le j\le s\}$, множество $Q_{\Omega,\varepsilon}$ определено равенством \eqref{valle-pussen-2-3.2},
$f\in _\Omega W_1^r$. Тогда если $r\ge4$, $x\in Q_{\Omega,\varepsilon}$, то
имеют место оценки
$$
|R_{m}(f,x)|\le{c(r,\varepsilon)J_{\Omega,r}(f)\over m+1}+{c(r)I_r(f)\over (m+1)^{r-1}},
$$
$$
|_1R_{n,m}(f,x)|\le{c(r,\varepsilon)J_{\Omega,r}(f)\over n(m+1)}+{c(r)I_r(f)\over (n+m)(m+1)^{r-2}},
$$
$$
|_2R_{n,m}(f,x)|\le{c(r,\varepsilon)J_{\Omega,r}(f)\over n^2(m+1)}+{c(r)I_r(f)\over (n+m)^2(m+1)^{r-3}},
$$
где величина $J_{\Omega,r}(f)$ определена равенством \eqref{valle-pussen-2-3.1},
 $I_r(f)=\int_{-\pi}^{\pi}|f^{(r)}(t)|dt$.
\end{theorem}

Покажем, что оценки, установленные в теореме \ref{valle-pussen-t1}, носят окончательный характер (по порядку). С этой целью обозначим через ${}_\Omega W^r_1(M,N)$ класс функций $f\in_\Omega W^r_1$, для которых $J_{\Omega,r}(f)\le M$, $I_r(f)\le N$. Положим
 $$
 R_m^\varepsilon(f)=\max_{x\in Q_{\Omega,\varepsilon}}|R_{m}(f,x)|,
 $$
 $$
 _1 R_{n,m}^\varepsilon(f)=\max_{x\in Q_{\Omega,\varepsilon}}|_1R_{n,m}(f,x)|,
 $$
 $$
 _2 R_{n,m}^\varepsilon(f)=\max_{x\in Q_{\Omega,\varepsilon}}|_2R_{n,m}(f,x)|.
 $$
 Справедлива следующая
\begin{theorem}
 Пусть $-\pi=x_0<x_1<\cdots<x_s<x_{s+1}=\pi$, $\Omega=\{x_0,x_1,\ldots x_s,x_{s+1}\}$,  $\quad 0<\varepsilon<\frac12\min\{x_{j+1}-x_{j}:0\le j\le s\}$, множество $Q_{\Omega,\varepsilon}$ определено равенством \eqref{valle-pussen-2-3.2}, $r\ge4$. Тогда имеет место соотношения:
$$
\sup_{f\in{}_\Omega W^r_1(M,N) }R_m^\varepsilon(f)\asymp{1\over m+1},
$$
$$
\sup_{f\in {}_\Omega W^r_1(M,N)} {}_1R_{n,m}^\varepsilon(f)\asymp{1\over n(m+1)},
$$
$$
\sup_{f\in{}_\Omega W^r_1(M,N) } {}_2R_{n,m}^\varepsilon(f)\asymp{1\over n^2(m+1)}.
$$
Другими словами, имеют место следующие неравенства
\begin{equation}\label{valle-pussen-2-3.9}
{c_1\over m+1}\le\sup_{f\in{}_\Omega W^r_1(M,N) }R_m^\varepsilon(f)\le{c_2\over m+1},
\end{equation}
\begin{equation}\label{valle-pussen-2-3.10}
{c_1\over n(m+1)}\le\sup_{f\in {}_\Omega W^r_1(M,N)} {}_1R_{n,m}^\varepsilon(f)\le{c_2\over n(m+1)},
\end{equation}
\begin{equation}\label{valle-pussen-2-3.11}
{c_1\over n^2(m+1)}\le\sup_{f\in{}_\Omega W^r_1(M,N) } {}_2R_{n,m}^\varepsilon(f){c_2\over n^2(m+1)},
\end{equation}
где $c_i=c_i(r,\varepsilon,M,N)$.
\end{theorem} 