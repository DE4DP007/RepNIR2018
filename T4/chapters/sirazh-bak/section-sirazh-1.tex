\subsection{О G--компактности одного класса эллиптических операторов второго порядка с комплекснозначными коэффициентами}

\textit{
    Рассмотрены вопросы G--сходимости одного класса эллиптических операторов второго порядка с комплекснозначными коэффициентами. Доказана G--компактность этого класса.
}



Многие задачи математической физики приводят к изучению вопросов G--сходимости дифференциальных операторов. Такие вопросы возникают в теории упругости, электродинамике и других разделах физики и механики.

Вопросам G--сходимости  дифференциальных операторов посвящено много работ (см. монографию В. В. Жикова и др. \cite{JikovKozlov} и имеющуюся там литературу). Теория  G--сходимости  дивергентных эллиптических операторов второго порядка в общих чертах завершена.



G--сходимость дифференциальных операторов -- это, иначе, слабая сходимость соответствующих обратных операторов. Поэтому по понятным причинам в задачах G--сходимости кроме корректной разрешимости краевых задач требуются также оценки решений, равномерные относительно любого оператора. Для недивергентных эллиптических операторов и систем (к которым относятся и рассматриваемые в работе операторы) такого рода оценки мало изучены, поэтому G--сходимость недивергентных операторов также изучена не столь детально, как для дивергентных операторов.

Вопросам G--сходимости и усреднению недивергентных эллиптических операторов посвящены работы \cite{JikovSirazh1, JikovSirazh2, Sirazh1, Sirazh2, Sirazh3}.



Пусть $W(Q)$ -- подпространство пространства Соболева $W_{2}^{2} (Q;C)$ комплекснозначных функций над полем $\mathrm{R}$, определенное равенством:

\[W(Q)=\left\{\begin{array}{l} {} \\ {} \end{array}\right. u\in W_{2}^{2} (Q;C)\, \, \, |\; \, Re\, u=0,\; Re\, \, \partial _{z} u=0\quad \mathrm{ =0}\; \partial {\kern 1pt} Q,.    \]

\begin{equation} \label{sirM1.1_} \left. \int _{\mathrm{ Q}}\mathrm{ Im}\, \, u\, \mathrm{ dx}=\mathrm{ 0,}\; \int _{\mathrm{ Q}}\mathrm{ Im}\, \, \partial _{z} u\, \mathrm{ dx}=\mathrm{ 0}  \right\} \end{equation}

Рассмотрим следующую краевую задачу Пуанкаре

\begin{equation} \label{sirM1.2_} Au\, \equiv \partial _{z\bar{z}}^{2} u+\mu \, \partial _{zz}^{2} u+\nu \; \partial _{\bar{z}\bar{z}}^{2} \bar{u}=f\in L_{2} (Q;C),\quad \; \quad \quad u\in W(Q), \end{equation}
где  $\mu =\mu (x)$ и  $\nu =\nu (x)$ --  измеримые в области  $Q$,  комплекснозначные функции удовлетворяющие условию:
\begin{equation} \label{sirM1.3_} \mathop{\mathrm{ vrai}\, \sup }\limits_{\, \, x\in Q} \; \, \left(|\mu (x)|+|\nu (x)|\right)\le k_{0} <1 \end{equation}
$k_{0} $ -- положительная постоянная; $Q$ -- ограниченная односвязная область плоскости с границей класса $C^{2+\alpha } $;
\begin{equation*}
\partial _{\bar{z}} =2^{-1} \, \left(\frac{\partial }{\partial x_{1} } +i\frac{\partial }{\partial x_{2} } \right),      \partial _{z} =2^{-1} \, \left(\frac{\partial }{\partial x_{1} } -i\frac{\partial }{\partial x_{2} } \right);
\end{equation*}
символ $C$ в обозначении пространства (здесь и далее) означает, что это пространство комплекснозначных функций над полем $R$ действительных чисел.

В дальнейшем класс операторов вида \eqref{sirM1.2_}, \eqref{sirM1.3_} будем обозначать $A(k_{0} ;Q)$.

Справедлива следующая \cite{Jamaludinova}

\textbf{Теорема. }\textit{Краевая задача Пуанкаре }\eqref{sirM1.2_} \textit{однозначно разрешима для любой правой части $f\in L_{2} (Q;C)$. Более того, имеют место априорные оценки}
\begin{equation}
\label{sirM1.4_}
(1-k_{0} )\left\| \, \partial _{z\bar{z}}^{2} u\, \right\| _{L_{2} (Q;C)} \le \left\| \, Au\, \right\| _{L_{2} (Q;C)} \le (1+k_{0} )\left\| \, \partial _{z\bar{z}}^{2} u\, \right\| _{L_{2} (Q;C)},
\end{equation}
\begin{equation}\label{sirM1.5_}
(1-k_{0} )\left\| \, \partial _{z\bar{z}}^{2} u\, \right\| ^{2} _{L_{2} (Q;C)} \le Re\int _{Q}Au\, \, \overline{\partial _{z\bar{z}}^{2} u} \, dx,\quad u\in W(Q).
\end{equation}

Заметим, что выражение $\left\| u\right\| _{W(Q)} =\left\| \partial _{z\bar{z}}^{2} u\right\| _{L_{2} (Q;C)} $, $u\in W(Q)$, задает в подпространстве $W(Q)$ пространства $W_{2}^{2} (Q;C)$ норму эквивалентную норме пространства $W_{2}^{2} (Q;C)$ (см. \cite{Jamaludinova}). 
Дадим понятие $G$--сходимости.

\textbf{Определение.} \textit{Скажем, что последовательность операторов $\{ A_{k} \} \subset A{\kern 1pt} \, \left(k_{0} ;Q\right)$  }
\textit{G--сходится  в области $Q$ к $A\in A\, (k_{0} ;Q)$}(\textit{и будем писать $G{\kern 1pt} -\lim A_{k} =A$})\textit{, если $\left\{A_{k}^{-1} \right\}$ слабо сходится к $A^{-1} $,  где $A_{k} $ и $A$ операторы краевых задач Пуанкаре:    $A_{k} u_{k} =f\in L_{2} (Q;C)$,$\; u_{k} \in W(Q)$,$\quad Au=f\in L_{2} (Q;C),\quad u\in W(Q)$.}

Иначе говоря, G--сходимость означает слабую сходимость решений $u_{k} \to u$ в $W(Q)$ для любой правой части $f\in L_{2} (Q;C)$. Справедлива следующая

\textbf{Теорема 1.}  \textit{Класс  $A\, (k_{0} ;Q)$ G--компактен}, \textit{то есть из любой последовательности операторов из $A\, (k_{0} ;Q)$ можно выделить G--сходящуюся подпоследовательность.}

\textbf{\textit{Доказательство. }} Пусть $\{ A_{k} \} $ -- последовательность из класса $A\, (k_{0} ;Q)$. Рассмотрим задачу Пуанкаре:
\begin{equation}
\label{sirM1.6_} \left\{\begin{array}{l} {A_{k} u_{k} =f\in L_{2} (Q;C),} \\ {u_{k} \in W(Q).} \end{array}\right.  \end{equation}
Она согласно предыдущей теореме, однозначно разрешима. Пусть $v_{k} =\partial _{z} u_{k} $, согласно \eqref{sirM1.1_} имеем:
\begin{equation*}
v_{k} \in W_{o} (Q;C)=\left\{v\in W_{2}^{1} (Q;C)\; \quad |\quad Rev=0\; \quad \mathrm{ =0}\quad \partial Q,\quad \int _{Q}Imv\,  dx=0\right\},
\end{equation*}
и, согласно \eqref{sirM1.6_},  $v_{k} $ есть решение задачи  Римана--Гильберта (Р--Г)
\begin{equation} \label{sirM1.7_} \left\{\begin{array}{l} {D_{k} v_{k} \equiv \partial _{\bar{z}} v_{k} +\mu _{k} \partial _{z} v_{k} +v_{k} \partial _{\bar{z}} \bar{v}_{k} =f\in L_{2} (Q;C),\quad } \\ {v_{k} \in W_{0} (Q;C)} \end{array}\right.  \end{equation}

Из оценок \eqref{sirM1.4_}, \eqref{sirM1.5_} следует, что оператор  $D_{k} $ принадлежит классу $B(k_{0} ;Q)$ из
\cite{Sirazh3}. Так как этот класс   $G$--компактен  (см. \cite{Sirazh3}), то найдется подпоследовательность $\{ D_{k'} \} \subset \{ D_{k} \} $, которая G--сходится к оператору $D\in B(k_{0} ;Q)$. Значит, решение $v_{k'} $ задачи Р--Г \eqref{sirM1.7_} с $k=k'$ слабо в $W_{0} (Q;C)$ сходится к решению $G$--предельной задачи:
\begin{equation}\label{sirM1.8_}
\left\{\begin{array}{l} {Dv\equiv \partial _{\bar{z}} v+\mu \partial _{z} v+v\partial _{\bar{z}} \bar{v}=f\in L_{2} (Q;C),} \\ {v\in W(Q;C).} \end{array}\right.
\end{equation}
Из левой части оценок \eqref{sirM1.4_} для $A=A_{k'} $ получим, что
\begin{equation*}
\left\| \, \partial _{\bar{z}z}^{2} u_{k'} \right\| _{L_{2} (Q;C)} \le (1-k_{0} )^{-1} \left\| \, f\right\| _{L_{2} (Q;C)}.
\end{equation*}

   Следовательно, $\{ u_{k'} \} $-- ограничена в $W(Q)$, значит, последовательность $\{ v_{k'} \} =\{ \partial _{z} u_{k'} \} $ -- ограничена в $W_{0} (Q)$. С другой стороны, так как \textit{$G{\kern 1pt} -\lim D_{k'} =D$} в $Q$, то $v_{k} $, слабо сходится к $v$ в $W_{0} (Q)$, где $v$ решение задачи \eqref{sirM1.8_}. Отсюда следует, что $\{ u_{k'} \} $слабо сходится в $W(Q)$ к  $u\in W(Q),\quad \partial _{z} u=v$. Подставив $v$ в \eqref{sirM1.8_}, получим, что последовательность $A_{k'} $ $G$--сходится к оператору $A$, определенному соотношением $Au=D\, \partial _{z} u$:
\begin{equation*}
A\, u\equiv \partial _{z\bar{z}}^{2} u+\mu \, \partial _{zz}^{2} u+\nu \, \partial _{z\bar{z}}^{2} \bar{u}=f,\quad u\in W(Q),
\end{equation*}
коэффициенты $\mu $ и $\nu $ удовлетворяют оценке \eqref{sirM1.3_}, ввиду G--компактности класса $B(k_{0} ;Q)$   (см. \cite{Sirazh3}).



\textbf{\textit{Следствие.}} \textit{Пусть $G{\kern 1pt} -\lim A_{k} =A$ в области $Q$; $A_{k} $, $A\in A\, (k_{0} ;Q)$; $\mu _{k} $, $\nu _{k} $, $\mu $, $\nu $ -- коэффициенты $A_{k} $, $A$. Тогда:}

\begin{enumerate}
\item \textit{ если $\nu _{k} =e^{i\alpha } \mu _{k} ,\quad k=1,2,....$, где $\alpha \in [-\pi ,\pi )$ -- фиксированное число, то $\nu =e^{i\alpha } \mu $;}

\item \textit{ если $\nu _{k} =e^{i\alpha } \overline{\mu _{k} },\quad k=1,2,....$, то $\nu =e^{i\alpha } \bar{\mu }$.}
\end{enumerate}

\textit{В частности, при $\nu _{k} =\mu _{k} $ имеем $\nu =\mu $, при $\nu _{k} =\overline{\mu _{k} }$ имеем $\nu =\bar{\mu }$. Черта над функцией означает переход к комплексно сопряженной функции.}

\textbf{\textit{Доказательство.}}  Согласно \cite{Sirazh3} в случае 1) коэффициенты \textit{$G$--}предельного оператора \textit{$D$ }из \eqref{sirM1.8_} связаны соотношением \textit{$\nu =e^{i\alpha } \mu $}, следовательно, коэффициенты \textit{$G$}--предела \textit{$A$} связаны этим же соотношением. Аналогично доказываются и остальные утверждения следствия.
