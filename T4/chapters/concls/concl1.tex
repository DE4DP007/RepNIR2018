
%%%%%%%%%%%%%%%%%%%%%%%
%%%%%%%%%%%%%%%%%%%%%%%
%%%%%%%%%%%%%%%%%%%%%%%

Найдены необходимые и достаточные условия на переменный показатель $p(x)>1$, гарантирующие равномерную ограниченность последовательности  сумм Фурье $S_n^{\alpha,\alpha}(f)$ ($n=0,1,\ldots$) по ультрасферическим полиномам Якоби $P_k^{\alpha,\alpha}(x)$
в весовом пространстве Лебега \linebreak$L_\mu^{p(x)}([-1,1])$ с весом $\mu=\mu(x)=(1-x^2)^\alpha$, где $\alpha>-1/2$. В случае $\alpha=-1/2$ показано, что для равномерной ограниченности  последовательности  сумм Фурье-Чебышева $S_n^{-\frac12,-\frac12}(f)$ ($n=0,1,\ldots$) в пространстве $L_\mu^{p(x)}([-1,1])$ достаточно и, в определенном смысле,  необходимо, чтобы переменный показатель $p$ подчинялся условию Дини -- Липшица  вида $|p(x)-p(y)|\le \frac{d}{-\ln|x-y|}$, где $|x-y|\le\frac12$, $x,y\in[-1,1]$, $d>0$, а также условию $p(x)>1$ для всех $x\in[-1,1]$.
Аналогичная задача решена для общих полиномов Якоби $P_k^{\alpha,\beta}(x)$ при $\alpha,\beta>-1/2$.

Исследованы вопросы, связанные с прямыми и обратными теоремами теории приближений функций в весовых пространствах Лебега  с переменным показателем. В частности, показано, что семейство сдвигов функции Стеклова вида
$$s_{\lambda,\tau}(f)(x) = \lambda \int_{x-\frac{1}{2\lambda}+\tau}^{x+\frac{1}{2\lambda}+\tau} f(t)dt.$$
равномерно ограничено в весовых пространствах Лебега с переменным показателем $L_{2\pi, w}^{p(x)}$, где $w = w(x)$ --- весовая функция, удовлетворяющая аналогу известного условия Макенхоупта.
Этот результат может быть использован при конструировании модуля гладкости в весовом пространстве Лебега с переменным показателем и доказательстве прямых теорем в этих пространствах.

На основе перекрывающих преобразований и повторных средних Валле Пуссена вида
 $$
_2V_{n,m}(f,x)= \frac1n\sum\nolimits_{k=m}^{m+n-1}{}_1V_{n,k}(f,x),
$$
были сконструированы операторы, осуществляющие приближения непрерывных функций и исследованы их аппроксимативные свойства. Кроме того показано, что локальные аппроксимативные свойства повторных средних ${}_2V_{n,m}(f,x)$ на порядок лучше чем у классических средних Валле Пуссена.


Изучены аппроксимативные свойства частичных сумм ряда Фурье по модифицированным полиномам Мейкснера $M_{n,N}^\alpha(x)=M_n^\alpha(Nx)$ $(n=0, 1, \dots)$.


Были доказаны оценки приближения полиномами $L_{n,N}(f,x)$ $2\pi$-периодических функций $f_1 = |x|$ и $f_2=\mbox{sign } x$ отрезке $[-\pi, \pi]$.
Для первой из этих функций показано, что вместо оценки $\left|f_{1}(x)-L_{n,N}(f_{1},x)\right| \leq c\ln n/n$,
вытекающей из известного неравенства Лебега, для полиномов $L_{n,N}(f,x)$ установлена точная по порядку оценка
$\left|f_{1}(x)-L_{n,N}(f_{1},x)\right| \leq c/n$ ($x \in \mathbb{R}$), которая имеет место равномерно относительно $1 \leq n \leq \lfloor N/2\rfloor$.
Кроме того, получена локальная оценка  $\left|f_{1}(x)-L_{n,N}(f_{1},x)\right| \leq c(\varepsilon)/n^2$ ($\left|x - \pi k\right| \geq \varepsilon$), которая также имеет место равномерно относительно $1 \leq n \leq \lfloor N/2\rfloor$.
Что касается второй из указанных функций $f_2(x)$, то для нее равномерно относительно $1 \leq n \leq \lfloor N/2\rfloor$ была получена оценка $\left|f_{2}(x)-L_{n,N}(f_{2},x)\right| \leq c(\varepsilon)/n$ ($\left|x - \pi k\right| \geq \varepsilon$).



%%%%%%%%%%%%%%%%%%%%%%%
%%%%%%%%%%%%%%%%%%%%%%%
%%%%%%%%%%%%%%%%%%%%%%%
Построены сплайн-функции по рациональным интерполянтам с автономными полюсами
и даны оценки скорости сходимости как самих сплайнов, так и их производных первого и
 второго порядков.	
 
Даны точные по порядку оценки скорости сходимости рациональных сплайн-функций
для непрерывных и непрерывно дифференцируемых функций.


%%%%%%%%%%%%%%%%%%%%%%%
%%%%%%%%%%%%%%%%%%%%%%%
%%%%%%%%%%%%%%%%%%%%%%%













%%%%%%%%%%%%%%%%%%%%%%%%%%%%%%%%%%%%%%%%%%
%%%%%%% ПРОШЛЫЙ ГОД %%%%%%%%%%%%%%%%%%%%%%
%%%%%%%%%%%%%%%%%%%%%%%%%%%%%%%%%%%%%%%%%%







%
%Исследованы аппроксимативные свойства частичных сумм специальных рядов для функций из пространств Соболева с переменным показателем. %Полученные результаты показывают, что применение пространств с переменным показателем позволяет учитывать существенно переменное поведение производных аппроксимируемой функции при оценке точности приближения заданной гладкой функции частичными суммами специального ряда.
%
%Получена оценка скорости приближения функций из пространств Лебега и Соболева с переменным показателем средними Валле Пуссена тригонометрических сумм Фурье в метрике этих пространств.
%
%%из реферата
%%Получены условия на переменный показатель, при которых последовательность операторов Бернштейна -- Канторовича равномерно ограничена в пространствах Лебега с переменным показателем.
%
%%Найден класс(непривычно звучит - Расул) показателей, для которого последовательность операторов Бернштейна -- Канторовича равномерно ограничена в пространствах Лебега с переменным показателем.
%
%%второй вариант
%Получены условия на показатель, при которых последовательность операторов Бернштейна -- Канторовича  равномерно ограничена в пространствах Лебега с переменным показателем и показана сходимость этой последовательности в метрике этих пространств.
%
%%реферат
%%Исследован вопрос о равномерной ограниченности семейства сдвигов функций Стеклова в весовых пространствах Лебега с переменным показателем. Получены условия на вес, при которых будет иметь место равномерная ограниченность упомянутого семейства.
%%Исследован вопрос о равномерной ограниченности семейства сдвигов функций Стеклова в весовых пространствах Лебега с переменным показателем. Найден класс (непривычно звучит - Расул) весов, для которого имеет место равномерная ограниченность упомянутого семейства.
%
%%второй вариант
%Получены условия на вес, обеспечивающие равномерную ограниченность семейства сдвигов функций Стеклова в весовых пространствах Лебега с переменным показателем.
%
