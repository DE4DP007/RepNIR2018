Разработан численный алгоритм эффективного вычисления повторных средних Валле Пуссена. На основе данного алгоритма составлена компьютерная программа и проведены численные эксперименты. Данный алгоритм может быть использован в алгоритме сжатия изображений.

Найдены достаточные условия существования беспростойных расписаний и указаны подходы к построению расписаний минимальной длительности. Исследованы теоретико-графовые модели расписаний, для решения задач интервальной реберной раскраски разработано алгоритмическое и программное обеспечение.

  Разработаны алгоритмы построения однопроцессорного расписания с условиями частичного предшествования и мультипроцессорного расписания без простоев и без условий частичного предшествования. Основные результаты опубликованы в \cite{akm-DEMI},
	 \cite{akm-artIRJ}, \cite{akm-artDMA}.

  Приведенное в разделе \ref{akm1} описание зарегистрированного в 2016 г. в Госпатенте программного обеспечения посвящено вопросам оптимизации расписания, сформулированным в терминологии теории графов.

  Получена явная формула обращения лучевого преобразования симметричного тензорного поля второго ранга, когда это преобразование задано на семействе прямых, направления которых определяется точками кусочно-гладкой кривой. Сама кривая удовлетворяет условию полноты, аналогично условию Кириллова -- Туя. В случае, когда веерное преобразование тензорного поля известно на семействе лучей, выходящих из точек незамкнутой кривой (не обязательно удовлетворяющей условию Кириллова -- Туя) соленоидальная часть искомого поля вычислена в точках прямолинейного отрезка, соединяющего концы кривой.

 Получены явные формулы вычисления значений финитных, скалярных и тензорных полей первого и второго ранга по заданным интегралам с весами от этих полей вдоль лучей и ломанных. В случае скалярных полей все лучи имеют одно направление, а весовая функция – квазимногочлен. В случае векторных полей для обращения веерного преобразования достаточно семейство лучей, имеющих два фиксированных направления, а в случае тензорных полей второго ранга – три направления. Полученные формулы обобщают известные в литературе формулы, используемые в линейных задачах интегральной геометрии с возмущениями.


Разработан алгоритм решения задачи Коши на отрезке [0,1] для обыкновенных дифференциальных уравнений с помощью рядов Фурье по функциям, ортогональным по Соболеву и порожденным классическими ортогональными функциями Хаара. На основе данного алгоритма составлена и зарегистрирована программа для ЭВМ MixedHaarDeqSolver (свидетельство № 2016617831 от 14.07.2016 г.). Компьютерные эксперименты, проведенные с помощью этой программы, показали высокую эффективность данного алгоритма для численного решения задачи Коши.

Предложен новый метод приближенного решения задачи Коши для разностного уравнения, основанный на разложении искомого решения в конечные ряды Фурье по полиномам, ортогональным относительно дискретного скалярного произведения типа Соболева и ассоциированным с полиномами Чебышева. Для указанных полиномов получено явное представление через классические полиномы Чебышева. Рассмотрены некоторые разностные свойства частичных сумм Фурье по этим полиномам.

Предложен алгоритм решения задачи Коши на полуоси для обыкновенного дифференциального уравнения посредством рядов Фурье по полиномам, ортогональным в смысле Соболева, ассоциированным с классическими полиномами Лагерра. Составлена программа, реализующая данный алгоритм, и проведены численные эксперименты.

 %Разработан пакет прикладных программ для обработки временных рядов и дискретных функций с помощью конечных предельных рядов и их усреднений типа Валле Пуссена. Проведены эксперименты на дискретных функциях различной природы. 

Разработаны два пакета прикладных программ для обработки временных рядов и дискретных функций: с помощью конечных предельных рядов и их усреднений типа Валле Пуссена, а также с помощью специальных вейвлет-рядов на основе полиномов Чебышева второго рода, обладающих так называемым свойством <<прилипания>>. Проведены эксперименты на дискретных функциях различной природы. 