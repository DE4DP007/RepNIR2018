\chapter{О гельдеровости решений задачи Римана-Гильберта для обобщенной
системы уравнений Бельтрами}

\section{Обобщенная система Бельтрами}

Изучаются вопросы гельдеровости решений следующей системы
уравнений
\begin{equation}\label{vfd 1}
\partial_{\bar z}w+\mu_1(x)\partial_{z}w+\mu_2(x) \partial_{\bar z}\overline w
+c_1(x)w+c_2(x)\overline{w} =f, \quad x\in Q,
\end{equation}
где $\mu_1$, $\mu_2$, $c_1$, $c_2$ --- квадратные матрицы порядка $n$,
$n\geqslant 1$, элементы которых %
принадлежат $L_{\infty}(Q;{\mathbb C})$; $w$, $f$ --- вектор--столбцы.

Как известно, уравнением Бельтрами называется скалярное уравнение
\eqref{vfd 1} без младших членов и $\mu_2=0$, причем почти всюду в
области $Q$ имеет место неравенство
$|\mu_1(x)|\leqslant k_0<1$, где $k_0>0$ --- постоянная. Отсюда следует,
что естественно назвать \textit {системой Бельтрами} систему
\begin{equation}\label{fvd:2}
\partial_{\bar z}w+\mu(x)\partial_{z}w =f, \quad x\in Q,
\end{equation}
спектральный радиус $\rho(\mu(x))$ матрицы
$\mu(x)$ которой  удовлетворяет условию:
\begin{equation}\label{fvd:3}
\rho(\mu(x))\leqslant k_0<1\qquad \text{п.\,в.\ \ $x\in Q.$}
\end{equation}
(Здесь и далее $k_0>0$ --- постоянная.)

Условие \eqref{fvd:3} означает, что корни характеристического уравнения
$$
\Delta(\lambda)\equiv \det(\lambda E+\mu(x))=0,
$$
где $E$ --- единичная матрица, расположены внутри круга $|\lambda|\leqslant k_0<1$,
следовательно, условие \eqref{fvd:3} --- условие эллиптичности системы
Бельтрами \eqref{fvd:2}. Отметим, что без нарушения эллиптичности нельзя заменить $k_0$ на единицу. Например, уравнение $\partial_{\overline z}u+\partial_zu=f$
не является эллиптическим.

Пусть мы имеем систему $n$ уравнений $\mathbf{a}\mathscr D_1w+\mathbf{b}\mathscr
D_2w=f$, где $w$, $f$ --- $n$-компонент\-ные вектор-функции,  $\mathbf{a}$, $\mathbf{b}$ ---  квадратные матрицы порядка $n$ с
измеримыми комплекснозначными ограниченными элементами и пусть все
 собственные значения матрицы $\mathbf{a}^{-1}\mathbf{b}$ для почти
 всех $x\in Q$
расположены в  полуплоскости $\{\text{Im\,}\lambda\geqslant \alpha_0>0\}$, где $\alpha_0$ ---
постоянная. Тогда эту систему можно
представить в виде системы Бельтрами $\partial_{\bar z}w+\mu\,\partial_{z}w=\mathbf{a}^{-1}f$,
 где $\mu=(\mathbf{a}^{-1}\mathbf{b}+iE)^{-1}(
\mathbf{a}^{-1}\mathbf{b}-iE)$, при этом в \eqref{fvd:3} $k_0=(1+\alpha_0)^{-1}$.


Систему \eqref{vfd 1}, где $\mu_2\ne0$, будем называть \textit{обобщенной системой
Бельтрами\,}.
Эллиптичность системы \eqref{vfd 1}
означает выполнение почти всюду в области $Q$ одного из двух следующих эквивалентных условий:
\begin{itemize}
\item
Для любого $\xi\in\mathbb C\setminus\{0\}$ имеем: $\det D(\xi,\overline\xi,\overline\xi,\xi)\ne0$,
где
\begin{equation}\label{f:d1}
D(\xi,\eta,\delta,\gamma)=\left(\begin{smallmatrix}
E\xi+\mu_1(x)\eta& \mu_2(x)\,\xi\\[1mm]
\overline{\mu_2(x)}\,\delta&E\delta+\overline{\mu_1(x)}\,\gamma\end{smallmatrix}\right),
\end{equation}
$E$ --- единичная матрица.
\item
Характеристическое уравнение $\Delta(\lambda)\equiv\det D(\lambda,1,\overline\lambda,1)=0$
не имеет корней на единичной
окружности $|\lambda|=1$.
\end{itemize}
Так как система \eqref{vfd 1} является обобщением системы Бельтрами, естественно
считать, что
\textit{ корни характеристического уравнения $\Delta(\lambda)=0$ расположены
внутри единичного круга $|\lambda|<1$}.
\textit{В дальнейшем под условием эллиптичности системы  \eqref{vfd 1} будем
понимать именно это условие. Кроме того (если не оговорено противное) будем предполагать, что система \eqref{vfd 1} удовлетворяет
условию}
\begin{equation}\label{vf3}
\mathop{vrai\,sup\,}_{x\in Q}(\mathbf{\|}\mu_1(x)\mathbf{\|}+\mathbf{\|}\mu_2(x)\mathbf{\|})\leqslant k_0<1,
\end{equation}
$k_0$ --- положительная постоянная (\textit {постоянная эллиптичности}),  $ \|\mu_j\|$
--- норма матрицы $\mu_j$, рассматриваемой как оператор умножения в пространстве %
$ {\mathbb C}^n$.

Из условия \eqref{vf3} следует, что корни характеристического уравнения $\Delta(\lambda)=0$
расположены внутри единичного круга $|\lambda|<1$. Действительно, из структуры
матрицы $D(\lambda,1,\overline\lambda,1)$ (см. \eqref{f:d1}), где $\lambda$ --- корень характеристического уравнения, следует, что однородная система
$Dg=0$ линейных алгебраических уравнений имеет решения вида $g=\left(\begin{smallmatrix}h\\[0.5mm]
\overline h
\end{smallmatrix}\right)$, $h\in\mathbb{C}^n$, $h\ne0$. Следовательно, имеем
 равенство $\lambda\left(Eh+\mu_2\overline{h}\right)=-\mu_1h$. Отсюда, согласно
 \eqref{vf3}, получим:
 $$
 |\lambda|\left(1-\|\mu_2\|\right)\leqslant\|\mu_1\|\quad \Longrightarrow\quad |\lambda|\leqslant\frac{\|\mu_1\|}
 {1-\|\mu_2\|}<1.
 $$

Как известно, любую равномерно эллиптическую систему двух уравнений первого
порядка с действительными непрерывными коэффициентами можно представить в виде скалярного уравнения \eqref{vfd 1} ($n=1$), причем выполнено условие \eqref{vf3}.

Отметим, что характеристическое уравнение скалярного уравнения
\eqref{vfd 1} есть уравнение окружности: $|\lambda-\mu_1|=|\mu_2|$. Поэтому
условие эллиптичности в этом случае совпадает с условием \eqref{vf3}. В общем
случае $n>1$ это не так.
 Условие \eqref{vf3} налагает
 дополнительное ограничение на расположение
 корней характеристического уравнения $\Delta(\lambda)=0$,
 аналогично условию Кордеса для
 недивергентных эллиптических уравнений
 второго порядка. Действительно,  рассмотрим, например,
 систему Бельтрами. Условием эллиптичности
 является условие \eqref{fvd:3}: для почти каждого $x\in Q$ спектральный радиус $\rho(\mu_1(x))$ матрицы
 $\mu_1(x)$ меньше единицы. Как известно,
 спектральный радиус
 меньше или равен норме матрицы, поэтому условие
 \eqref{vf3} сужает класс рассматриваемых
 обобщенных систем Бельтрами.

Пусть  $\mu_1$, $\mu_2$ --- нормальные матрицы, т.\,е. они перестановочны со своими сопряженными (например, самосопряженные). Тогда, как известно, спектральные радиусы
матриц совпадают с их нормами: $\rho(\mu_j)=\|\mu_j\|$, $j=1,2$.
Такое же совпадение имеет место и для матриц представимых в виде ортогональной суммы $\mu_j=\nu_j\oplus r_j$, где $\nu_j$ --- нормальная матрица, а $r_j$ удовлетворяет оценкам: $\rho(r_j)\leqslant \rho(\nu_j)$, $\|r_j\|\leqslant\|\nu_j\|$.
Следовательно, для таких матриц нормы в \eqref{vf3} можно заменить на спектральные радиусы. Отсюда следует, что для системы Бельтрами с нормальной матрицей коэффициентов
(или с матрицей коэффициентов вида $\mu_1=\nu_1\oplus r_1$) условие  \eqref{vf3} не требуется, оно есть следствие условия эллиптичности \eqref{fvd:3}.


\section{Об одной краевой задаче Римана -- Гильберта  в односвязной области}\label{P1}

Сначала мы изучим вопросы гельдеровости решений одной специальной
краевой задачи Римана -- Гильберта в односвязной области  для обобщенной системы
уравнений Бельтрами без младших членов %
\begin{align}\label{fd 1}
 &A_0w =f\in L_2(Q;\,\mathbb C),\qquad
w\in W^1_2(Q;\,\mathbb C), \\\label{fd 2}
 &\text{Re}\,w=0 \quad \text{на} \quad {\partial Q}_0,
\end{align}
где $Q$ --- односвязная область плоскости с гладкой (класса $C^{1+\alpha}$,
$0<\alpha<1$) границей;
\begin{equation*}
\nonumber  A_0w =\partial_{\bar z}w+\mu_1\partial_{z}w+\mu_2 \partial_{\bar z}\overline w,
\end{equation*}
Здесь $\mu_1$, $\mu_2$ --- квадратные матрицы порядка $n$, $n\geqslant 1$, с элементами из  $L_{\infty}(Q;{\mathbb C})$, для которых
имеет место неравенство \eqref{vf3}.

Имеет место
\begin{theorem}\label{Td1} Задача Римана -- Гильберта \eqref{fd 1}, \eqref{fd 2}
при дополнительном
предположении
\begin{equation}\label{fd 3}
\int_{\partial Q}{\mathrm {Im}}\,w\,ds=0.
\end{equation}
однозначно разрешима для любой правой части $f\in L_2(Q;\mathbb C)$.
При этом имеет место априорная оценка
\begin{equation}\label{fd 0}
c_0\|w\|_{W^1_2(Q;\mathbb C)}\leqslant \|A_0w\|_{L_2(Q;\mathbb C)},
\end{equation}
где $c_0>0$ --- постоянная, зависящая только от $k_0$, $n$ и $Q$.

Общее решение задачи \eqref{fd 1}, \eqref{fd 2} дается формулой: $w=w_0+ic$, $c\in
{\mathbb R}^n$, где $w_0$ решение задачи \eqref{fd 1}, \eqref{fd 2},
\eqref{fd 3}; $\widetilde w=ic$, $c\in\mathbb{R}^n$, --- общее решение однородной
задачи \eqref{fd 1}, \eqref{fd 2}.

Для любого $w\in W^1_2(Q;\mathbb C)
$, $\mathop{\mathrm{Re}}w|_{\partial Q}=0$ имеет место априорная оценка
\begin{equation}\label{fd -1}
\frac{1-k_0}4\sum_{j=1}^2\left\|\mathscr{D}_jw\right\|_{L_2(Q;
\mathbb C)}\leqslant \|A_0w\|_{L_2(Q;\mathbb C)}.
\end{equation}
\end{theorem}

Отметим, что имеет место и $L_p$-аналог этого утверждения  ($p>2$). Сформулируем его.
Рассмотрим задачу Римана -- Гильберта
\begin{align}\label{f1}
 &A_0w =f\in L_p(Q;\,\mathbb C),\qquad
w\in W^1_p(Q;\,\mathbb C), \\\label{fd 4}
 &\text{Re}\,w=0 \qquad \text{на} \qquad {\partial Q}_0.
\end{align}
Справедлива следующая

\begin{theorem}\label{T1}
Существует показатель повышенной суммируемости $p_0>2$, $ p_0=p_0(k_0,n)$, зависящее
только от $k_0$ и $n$, такой, что задача \eqref{f1}, \eqref{fd 4},
\eqref{fd 3} однозначно разрешима для любой правой части $f\in L_p(Q;\mathbb C)$, $2<p\leqslant p_0$. Причем имеет место априорная оценка
\begin{equation}\label{fd 00}
c_0\|w\|_{W^1_p(Q;\mathbb C)}\leqslant \|A_0w\|_{L_p(Q;\mathbb C)},
\quad w\in \Big\{u\in  W^1_p(Q;\,\mathbb C)\ \Big|\ \mathrm{Re}\,u|_{\partial Q}=0,\
\int_{\partial Q}\mathrm{Im}\,u\,ds=0\Big\},
\end{equation}
где $c_0>0$ --- постоянная зависящая только от $k_0$, $n$ и $Q$.
 Общее решение задачи \eqref{f1}, \eqref{fd 4} дается формулой: $w=w_0+ic$, $c\in
{\mathbb R}^n$, где $w_0$ решение задачи \eqref{f1}, \eqref{fd 4},
\eqref{fd 3}; $\widetilde w=ic$, $c\in\mathbb{R}^n$, --- общее решение однородной
задачи \eqref{f1}, \eqref{fd 4}.
Любое решение задачи \eqref{f1}, \eqref{fd 4} удовлетворяет оценкам: %
\begin{align}\label{fd4.2}
& \sum_{j=1}^2\left\|\mathscr{D}_jw\right\|_{L_p(Q;
\mathbb C)}\leqslant
{\alpha}_0\|\,f\,\|_{L_p(Q;\mathbb C)},\\[2mm]
|w(x)-w(y)| & \leqslant \alpha_1\|\,f\,\|_{ L_p(Q;\mathbb C)} \cdot
|x-y|^{(p-2)/p} \qquad (\forall x,y\in {\overline Q}_0),\label{f4.2}
\end{align}
где ${\alpha}_0,\,{\alpha}_1>0$ --- константы, зависящие только от $k_0$, $n$ и $Q$.%
\end{theorem}

В дальнейшем, если нет необходимости, опускаем индекс $0$ в показателе повышенной суммируемости.

Из теорем \ref{Td1} и \ref{T1}, очевидно, вытекает следующее свойство
гельдеровости решения задачи Римана -- Гильберта \eqref{fd 1}, \eqref{fd 2}.
\begin{corollary}\label{Sl1}
Пусть $w\in W_2^1(Q;\mathbb C)$ --- решение задачи Римана -- Гильберта \eqref{fd 1},
\eqref{fd 2}, где $f\in L_q(Q;\mathbb C)$, $q>2$ и пусть $2<p\leqslant \min\{q,p_0\}$, где
$p_0$ -- показатель повышенной суммируемости из теоремы \ref{T1}.
Тогда $w\in W_p^1(Q;\mathbb C)$ и имеют место оценки \eqref{fd4.2}, \eqref{f4.2}. Кроме того, если $w$ удовлетворяет еще и граничному условию \eqref{fd 3}, то
имеет место оценка \eqref{fd 00}.
\end{corollary}

Следующий пример показывает достаточность условия \eqref{vf3} для
гельдеровости решения.
\begin{example}\label{Ex1}
В единичном круге $Q=\{|z|<1\}$ рассмотрим следующую задачу Римана -- Гильберта для системы
двух уравнений
\begin{equation}\label{fd:1.14}
\left\{\begin{aligned}
&\partial_{\overline z}w+\mu\,\partial_zw=f\in L_q(Q; \mathbb C), \quad w\in
 W^1_2(Q;\mathbb C),\\[1mm]
 &\mathrm{Re}\,w=0\quad\text{на\quad $\partial Q$},\qquad\int_{\partial Q}
 \mathrm{Im}\,w\,ds=0,
 \end{aligned}\right.
 \end{equation}
где $q>2$, $\mu=\Big(\begin{matrix}{\,}0&1\\[-4pt]
{\,}0&a
\end{matrix} \Big)$,
 $a$ -- измеримая комплекснозначная ограниченная функция,  $|a(z)|\leqslant c_0<1$,
 $z\in Q$, $c_0>0$ -- постоянная; $w=\left(\begin{smallmatrix}w_1\\
w_2\end{smallmatrix}\right)$, $f=\left(\begin{smallmatrix}f_1\\
f_2\end{smallmatrix}\right)$  ---
 вектор-столбцы. Покажем, что решение задачи \eqref{fd:1.14} гельдерово.
\end{example}

Собственные значения матрицы $\mu$ --- $0$ и $a$, следовательно, спектральный радиус
$\rho(\mu)\leqslant c_0$. Норма матрицы $\mu$, как оператора умножения в евклидовом пространстве, связана
со спектральным радиусом матрицы $\mu\mu^\ast$ равенством $\|\mu\|=\sqrt{\rho(\mu\mu^\ast)}$. Легко видеть, что $\lambda_1=0$, $\lambda_2=1+|a|^2$ --- собственные значения матрицы $\mu\mu^\ast$.
 Следовательно,
$\|\mu(z)\|=\sqrt{1+|a(z)|^2}\geqslant1$, $z\in Q.$
Значит, система \eqref{fd:1.14} не удовлетворяет
условию \eqref{vf3}. Тем не менее для задачи
\eqref{fd:1.14} имеет место гельдеровость решения.
Действительно, ввиду треугольности матрицы $\mu$,
второе уравнение следующего вида $\partial_{\bar
z}w_2+a\,\partial_zw_2=f_2\in L_q(Q; \mathbb C)$.
Отсюда, согласно граничным условиям
\eqref{fd:1.14} и следствию
\ref{Sl1} получим, что $w_2\in W^1_p(Q; \mathbb C)$, где
$p=p(c_0)>2$ --- показатель повышенной суммируемости.
С учетом этого из первого уравнения получим
$w_1\in W^1_p(Q; \mathbb C)$. Следовательно, в
силу вложения
$W^1_p(Q; \mathbb C)\subset C^\alpha(Q)$,
$\alpha=(p-2)/p$, получим гельдеровость решения.

\section{Треугольные системы}
Пусть матрицы $\mu_1=\{\mu_{lj}^1\}$, $\mu_2=\{\mu_{lj}^2\}$
коэффициентов системы \eqref{vfd 1} нижние треугольные (или
верхние треугольные) матрицы и пусть выполнены условия:
\begin{equation}\label{f:1ts}
\left|\mu_{ll}^1(x)\right|+\left|\mu_{ll}^2(x)\right|\leqslant c_0<1,\quad l=1, \dots, n, \quad\text{п.\,в. $x\in Q$},
\end{equation}
где $c_0>0$ --- постоянная.
Тогда перестановкой строк и
столбцов детерминанта \eqref{f:d1} получим, что
 $\det D(\xi,\overline\xi,\overline\xi,\xi)$  представляется в
виде определителя квазитреугольной матрицы. При этом диагональными
блоками будут следующие квадратные матрицы второго порядка
\begin{equation*}
\left(\begin{matrix}
\xi+\mu_{ll}^1\overline{\xi}&\mu_{ll}^2\xi\\
\overline{\mu_{ll}^2\xi}&\overline{\xi}+\overline{\mu_{ll}^1}\xi
\end{matrix}
\right),\quad l=1,\dots,n.
\end{equation*}
Следовательно, характеристическое уравнение
$\Delta(\lambda)=0$ имеет следующий вид
\begin{equation*}
\prod_{l=1}^n\left(\left|\lambda+\mu_{ll}^1\right|^2-\left|\mu_{ll}^2\right|^2\right)=0.
\end{equation*}
Ввиду \eqref{f:1ts}, каждая из окружностей $\left|\lambda+\mu_{ll}^1\right|=\left|\mu_{ll}^2\right|$,
$l=1, \dots, n$, расположена внутри единичного круга $|\lambda|<1$.
 Значит, эта система эллиптическая.

Аналогично примеру \ref{Ex1} получим гельдеровость решений системы с треугольной главной частью. А, именно, имеет место
\begin{theorem}
Пусть $w\in W_2^1(Q;\mathbb{C})$ --- решение
задачи Римана -- Гильберта \eqref{fd 1},
\eqref{fd 2}, \eqref{fd 3}, где $f\in L_q(Q;\mathbb{C})$, $q>2$; $\mu_1$, $\mu_2$ --- верхние
(нижние) треугольные матрицы, удовлетворяющие
\eqref{f:1ts}. Тогда найдется показатель $p>2$, $p\leqslant q$, зависящий только от $c_0$ из
\eqref{f:1ts} такой, что $w$ удовлетворяет условию гельдера \eqref{f4.2}, где $\alpha_1>0$ --- постоянная зависящая только от
$c_0$ и $Q$.
\end{theorem}

\section{О гельдеровости решений общей задачи Римана -- Гильберта}\label{1.2}

Пусть $Q$ --- ограниченная связная область плоскости (класса $C^{1+\alpha}$, $0<\alpha<1$), граница которой %
состоит из объединения непересекающихся контуров $L_0,\dots, L_m$, $ m\geqslant0$, гомеоморфных окружности. %
Причем $ \Gamma_0$ содержит внутри себя остальные. Иначе говоря, $Q$ есть ($m+1$)--%
связная область плоскости. Рассмотрим следующую задачу Римана -- Гильберта
\begin{equation}\label{f2.5}
\begin{cases}
Aw \equiv \partial _{\bar z} w+\mu_1 \partial _z w+\mu _2 \partial _{\bar z}
\overline w+ c_1 w+ c_2 \overline w=f\in L_2(Q;\,\mathbb C), \\
w\in W^1_2(Q;\,\mathbb C), \\
 \text{Re}\,(\bar \lambda w)=0 \qquad \text{на} \qquad \partial Q,
\end{cases}
\end{equation}
где $c_1$ и $c_2$ --- квадратные матрицы порядка $n$, элементы которых принадлежат
$L_\infty(Q;\,\mathbb C)$; $\lambda $ --- гладкая (класса %
$C^1(\partial Q)$) унитарная матрица-функция, определенная на $\partial Q$,
матрицы $\mu_1$ и $\mu_2$ такие же как выше.
Известно (см. \cite{9}), что краевая задача Римана -- Гильберта \eqref{f2.5} \textit {нетерова, причем индекс краевой задачи дается формулой:
$\mathrm{ind}\,A=2\kappa-n(m-1)$, где $\kappa$ --- индекс функции $\mathrm{det}\,\lambda$, т.\,е. деленное на $2\pi$ приращение аргумента функции при одном полном обходе  границы области в положительную сторону.}
Имеет место следующая
\begin{theorem}\label{T2}
Пусть $w\in W^1_2(Q;\,\mathbb C)$ --- решение
задачи Римана -- Гильберта \eqref{f2.5}, где $f\in { L_p}(Q;\,\mathbb C)$, $p>2$ --- показатель
повышенной суммируемости из теоремы \ref{T1}. Тогда
$w\in W^1_p(Q;\,\mathbb C)$ и имеет место оценка
\begin{equation}\label{fd.13}
\alpha_0\|w\|_{W^1_p(Q;\,\mathbb C)}\leqslant \alpha_1
\|w\|_{L_p(Q;\,\mathbb C)}+\|f\|_{L_p(Q;\,\mathbb C)},
\end{equation}
где $\alpha_0>0$, $\alpha_1>0$ -- постоянные, зависящие
только от  $k_0$, $n$, $Q$, $L_{\infty}$\!--норм коэффициентов при
младших членах и $C^1$\!-норм
элементов матрицы $\lambda$.
\end{theorem}

Из теоремы \ref{T2}, ввиду вложения $W^1_p(Q;\,
\mathbb{C})\subset C^\alpha(\overline{Q})$, $\alpha=(p-2)/p$, получим справедливость
следующего утверждения.
\begin{theorem} Пусть $w\in W^1_2(Q;\,\mathbb C)$ --- решение %
задачи Римана -- Гильберта \eqref{f2.5}, где $f\in { L_p}(Q;\,\mathbb C)$,  $p>2$ --- показатель
повышенной суммируемости. Тогда %
$w$ удовлетворяет в замыкании области $Q$ условию гельдера с показателем
$\alpha=(p-2)/p$ и
постоянной, зависящей только от $k_0$, $n$, $Q$, $ L_\infty$\!-норм коэффициентов при младших членах и $C^1$\!-норм %
элементов матрицы $\lambda$,
а также от $L_p$\!-норм $w$ и $f$.
\end{theorem}

\section{О нетеpовости задачи Римана -- Гильберта в $W^1_p(Q;\,\mathbb C)$}

Пусть $p=p(k_0,n)>2$ --- показатель повышенной суммируемости из теоремы \ref{T1}. Рассмотрим  задачу %
Римана -- Гильберта: %
\begin{equation}\label{5.6}
\begin{aligned}
Aw &=f \in L_p(Q;\,\mathbb C),\qquad w\in  W^1_p(Q;\,\mathbb C),\\
\text{Re} &\,({\bar \lambda}w)=0 \qquad \text{на} \qquad \partial Q,
\end{aligned}
\end{equation}
где оператор $A$ тот же, что и в  \eqref{f2.5}.
Если $f=0$, то согласно теореме \ref{T2}  элементы ядра оператора
$A$ из %
$ W^1_2(Q;\,\mathbb C)$ принадлежат ${ W_p^1}(Q;\,%
\mathbb C)$. Следовательно, ввиду вложения $ W^1_p(Q) \subset
W^1_2(Q)$, ядро оператора $A$ в $W^1_2$ совпадает с
ядром в %
$W^1_p$ , т.\,е.
$
\text{Ker\,}(A,W^1_2)=\text{Ker\,}(A, W^1_p)
$.
Согласно оценке \eqref{fd.13} из теоремы \ref{T2}, получим: задача \eqref{5.6} нормально %
pазpешима (см. [10, \S\,7] ). Далее, очевидно, что %
$\text{Ker\,}(A^{\ast}; L_2)\subset \text{Ker\,}(A^{\ast},L_{p^{\prime}})$, %
где $1/p+1/{p^{\prime}}=1$. На самом деле здесь имеет место совпадение ядер. %
Докажем это от противного. Пусть существует вектор $f\in  L_p(Q;\,\mathbb C)$, такой, что имеем %
                                $$
J=\text{Re}\int \limits _Q (f, \varphi)_{{\mathbb C}^n}\,dx=0 \qquad \text{для любого}
\qquad \varphi   \in \text{Ker\,}(A^{\ast}; L_2)
                                $$
и $J\ne 0$ хотя бы для одного $\varphi \in \text{Ker\,}(A^{\ast};
L_{p^{\prime}})$. Следовательно, задача \eqref{5.6} при таком $f$ не имеет
решений. С другой стороны, как было отмечено в предыдущем пункте,
оператор $A:W^1_2(Q;\,\mathbb C)\to L_2(Q;\,\mathbb C)$ -- нетеpовый. Поэтому, в силу ортогональности $f$ ядру
$\text{Ker\,}(A^{\ast}; L_2)$, задача \eqref{5.6} pазpешима в
$W^1_2(Q;\,\mathbb C)$. Тогда согласно теореме \ref{T2}
получим, что это решение принадлежит $W_p^1(Q;\,\mathbb C)$.
Полученное противоречие  доказывает равенство %
$\text{Ker\,}(A^{\ast};L_2)= \text{Ker\,}(A^{\ast};L_{p^{\prime}})$.
Отсюда и теоремы \ref{T2}, следует справедливость следующего утверждения.
\begin{theorem} Пусть $p=p(k_0,n)>2$ --- показатель повышенной
суммируемости. Тогда задача P--Г  \eqref{5.6} нетерова и ее индекс равен
$\kappa=2\mathrm{ind}(\mathrm{det}\lambda)-n(m-1)$.

Далее, если $w$ --- решение  \eqref{5.6}, то для него в $Q$ имеет место оценка:
                                $$
\|w\|_{ W^1_p(Q;\,\mathbb C)}\leqslant c(\|w\|_{L_p(Q;\,\mathbb C)}+\|f\|_{L_p(Q;\,\mathbb C)}),
                                $$
где $c>0$ -- константа, зависящая только от $k_0$, $n$, $Q$, $L_\infty$\!--норм коэффициентов при младших членах и $\sup$\!-норм
элементов матрицы $\lambda$ и ее производных первого порядка.
\end{theorem}

\chapter{Операторные оценки усреднения  обобщенных уравнений  Бельтрами}

\section{Формулировка результатов}


\subsection{Задача Римана -- Гильберта}\label{par1.1}


Рассмотрим следующую задачу Римана~--~Гильберта:

\begin{equation}\label{f:1.1}
   \left\{\begin{array}{l}
A u\equiv\partial_{\bar{z}}u+\mu\,\partial_z u+\nu\,\partial_{\bar z}\overline{u}=f\in L_2(Q;\mathbb{C}), \\[3mm]
   u\in W_0(Q),
\end{array}\right.
\end{equation}
где коэффициенты $\mu=\mu(x)$, $\nu=\nu(x)$ --- измеримые ограниченные комплекснозначные функции, удовлетворяющие условию
\begin{equation}\label{f:1.2}
   \mathop{\mathrm {vrai\,sup}}_{x\in Q}\left(|\mu(x)|+|\nu(x)|\right)\leqslant k_0 <1,
\end{equation}
 $k_0>0$ --- постоянная (\textit{константа эллиптичности}); $Q$ --- ограниченная
  односвязная область плоскости с кусочно-гладкой границей. Как известно,
\textit{
задача Римана--Гильберта \eqref{f:1.1} однозначно разрешима для любой правой части из $L_2(Q;\mathbb{C})$,
причем имеют место априорные оценки}:
\begin{gather}\label{f:1.3}
   (1-k_0)\left\|\partial_{\bar{z}}u\right\|^2_{L_2(Q;\mathbb{C})}\leqslant \text{Re\,} \int\limits_{Q} Au\cdot \overline{\partial_{\bar{z}}u}\,dx,\\
\label{f:1.4}
  (1-k_0)\left\|\partial_{\bar{z}}u\right\|_{L_2(Q;\mathbb{C})}\leqslant \left\| Au\right\|_{L_2(Q;\mathbb{C})} \leqslant
   (1+k_0)\left\|\partial_{\bar{z}}u\right\|_{L_2(Q;\mathbb{C})}.
\end{gather}


Оценки \eqref{f:1.3}, \eqref{f:1.4} справедливы для любого элемента $u\in W_2^1(Q;\mathbb{C})$, удовлетворяющего условию
$\text{Re\,}u|_{\partial Q}=0$, условие на мнимую часть не требуется.
Более того они верны и для многосвязных областей.

Выражение
$$
\left\| u\right\|_{W_0(Q)}=\left\| \partial_{\bar{z}}u\right\|_{L_2(Q;\mathbb{C})}, \quad u\in W_0(Q)
$$
задает в подпространстве $W_0(Q)$ норму, эквивалентную норме исходного пространства $W_2^1(Q;\mathbb{C})$ (см. \cite{1}),
поэтому имеют место оценки:
\begin{equation}\label{f:1.5}
c_1\|u\|_{W^1_2(Q;\mathbb{C})}\leqslant\|Au\|_{L_2(Q;\mathbb{C})}\leqslant c_2 \|u\|_{W^1_2(Q;\mathbb{C})},
\end{equation}
где $c_1$, $c_2>0$ --- положительные постоянные, зависящие только от постоянной эллиптичности $k_0$.

\subsection{$G$-сходимость}\label{shpr}
\noindent Обозначим через $\mathscr{A}(k_0;\,Q)$--множество обобщенных  операторов Бельтрами \eqref{f:1.1}, $\mathscr{A}_0(k_0;\,Q)$-- подмножество $\mathscr{A}(k_0;\,Q)$  операторов Бельтрами \eqref{f:1.1} с $\nu=0$.

\begin{definition}
Скажем, что последовательность операторов $\left\{ \textit {A}_k\right\}$ из класса $\mathscr{A}(k_0;\,Q)$
$G$-сходится в области $Q$ к оператору $A\in \mathscr{A}(k_0;\,Q)$ (и будем писать $A_k\overset{G}{\longrightarrow} A$),
если для любого $f\in L_2(Q;\mathbb{C})$ последовательность $u_k$ решений задачи Р--Г: $A_ku_k=f$, $u_k\in W_0(Q)$,
сходится в $L_2(Q;\mathbb{C})$ к решению задачи Р--Г: $Au=f$, $u\in W_0(Q)$.
\end{definition}

$G$-предел определен единственным образом.
Как известно, см. \cite{1,6}
%\begin{theorem}
\textit{ классы $\mathscr{A}(k_0;\,Q)$, $\mathscr{A}_0(k_0;\,Q)$ обобщенных
операторов Бельтрами, операторов Бельтрами $G$-компактны}.

\subsection{ Усреднение}\label{Usred}

Рассмотрим задачу Римана -- Гильберта:
\begin{equation}\label{f:1.6}
   \left\{\begin{array}{l}
A_\varepsilon u\equiv\partial_{\bar{z}}u_\varepsilon+\mu^{\varepsilon}\,\partial_z u_\varepsilon
+\nu^\varepsilon\,\partial_{\bar z}\overline u_\varepsilon=f\in L_2(Q;\mathbb{C}), \\[3mm]
   u_\varepsilon\in W_0(Q),
\end{array}\right.
\end{equation}
где $0<\varepsilon<1$, $\mu^\varepsilon=\mu(\varepsilon^{-1}x)$,
$\nu^\varepsilon=\nu(\varepsilon^{-1}x)$; $\mu(x)=\mu(x_1,x_2)$,  $\nu(x)=\nu(x_1,x_2)$ --- измеримые ограниченные комплекснозначные периодические (периода $T$ по каждой переменной)
функции, удовлетворяющие условию эллиптичности \eqref{f:1.2} на всей плоскости
\begin{equation*}
   \mathop{\mathrm {vrai\,sup}}_{x\in \mathbb{R}^2}\left(|\mu(x)|+|\nu(x)|\right)\leqslant k_0 <1.
\end{equation*}
Очевидно, что оператор $A_\varepsilon$ принадлежит классу $\mathscr{A}(k_0;\,Q)$.
\begin{definition}
Скажем, что семейство $\left\{ A_\varepsilon \right\}$ допускает усреднение, если $A_\varepsilon \overset{G}{\longrightarrow}A\in \mathscr{A}(k_0;\,Q) $ при
$\varepsilon\to 0$.
\end{definition}

В вопросах усреднения важную роль играет ядро оператора $\mathscr{A}^*$, сопряженного оператору периодической задачи:
\begin{equation}\label{f:1.7}
 \mathscr{A}u\equiv \partial_{\bar{z}}u +\mu(x) \partial_{z}u +\nu(x)\,
 \partial_{\bar z}\overline u=f\in L_2(\square;\mathbb{C}),\quad
 u\in W_2^1(\square;\mathbb{C}).
\end{equation}
Сформулируем результаты по периодической задаче, необходимые в дальнейшем.
\begin{theorem}\label{T_1}
Справедливы следующие утверждения:
\begin{itemize}
\item Периодическая задача \eqref{f:1.7} фредгольмова.
\item Ядро $\mathrm{Ker}\mathscr{A}^{\ast}$ сопряженного
оператора  $\mathscr{A}^\ast:L_2(\square;\mathbb{C})\to
W_2^{-1}(\square;\mathbb{C})$   --- двумерное подпространство
 $L_2(\square;\mathbb{C})$ напомним, что
наши пространства --- пространства над полем $\mathbb{R}$,
причем один из базисов $\left\{p_1,p_2\right\}$ ядра
$\mathrm{Ker}\mathscr{A}^\ast$ обладает свойствами
\begin{equation*}%\label{f:1.10}
\left< p_1\right>=1,\quad \left< p_2\right>=i.
\end{equation*}
\item
В случае оператора Бельтрами $(\nu=0)$ имеем: $p_2=ip_1$.
\end{itemize}
\end{theorem}


Сформулируем теорему об усреднении.
\begin{theorem}[(Об усреднении)]\label{T_2}
Для семейства операторов \eqref{f:1.6}:
\begin{equation*}
   A_\varepsilon=\partial_{\bar{z}}+\mu^\varepsilon\partial_z+\nu^\varepsilon\overline
   \partial_{z},
\end{equation*}
где $\mu^\varepsilon=\mu(\varepsilon^{-1}x)$, $\nu^\varepsilon=\nu(\varepsilon^{-1}x)$,
имеет место усреднение, причем коэффициенты усредненного оператора
$$
A_0=\partial_{\bar{z}}+\mu^0\,\partial_z +\nu^0\,\overline{\partial}_{z}
$$
постоянные, определяемые  равенствами
\begin{equation}\label{f:1.10}
\mu^0=\left<\,\mu\, \mathscr Q+\overline{\nu}\,\mathscr{P}\,\right<, \quad
\nu^0=\left<\,\overline{\mu}\,\mathscr{P}+\nu\, \mathscr Q\,\right<,
\end{equation}
где
$$
\mathscr{P}=2^{-1}(p_1+ip_2),
\quad \mathscr Q=2^{-1}(\overline{p_1}+i\,\overline{p_2}),
$$
$p_1$, $p_2$ ---  базисные векторы ядра $\mathrm{Ker}\mathscr{A}^\ast$ из теоремы \ref{T_1}, $\overline{p_1}$, $\overline{p_2}$ --- комплексно сопряженные $p_1$, $p_2$ функции;  $\overline\partial_{z}$ --- дифференциальный оператор, определенный равенством
$\overline\partial_{z} u=\partial_{\bar z}\overline{u}$.

В случае уравнения Бельтрами имеем: $\mu^0=\left<\mu\,\overline{p_1}\right>$,
$\nu^0=0$.
\end{theorem}

\subsection{Задача на ячейке}

Рассмотрим следующую периодическую задачу
\begin{equation}\label{f:1.11}
\left\{\begin{aligned}
&\mathscr{A}N_j\equiv \partial_{\bar z}N_j+\mu\,\partial_z N_j+\nu\,
\partial_{\bar z}\overline{N}_j=\chi_j,\\
& N_j\in W_2^{1}(\square;\mathbb{C}),\quad j=1;2,
\end{aligned}\right.
\end{equation}
где
\begin{equation}\label{f:1.12}
\begin{aligned}
\chi_1=\frac12\left(\mu^0+\nu^0-\mu(x)-\nu(x)\right),\quad
\chi_2=\frac{i}2\left(\mu^0-\nu^0-\mu(x)+\nu(x)\right).
\end{aligned}
\end{equation}
Здесь $\mu^0$, $\nu^0$ --- коэффициенты \eqref{f:1.10} усредненного уравнения.
Справедлива следующая
\begin{theorem}[(см.\,\cite{2})]\label{T_3}
Периодическая задача \eqref{f:1.11} разрешима и решения представляются в виде
\begin{equation*}
N_j+c_j,\quad \quad j=1;2,
\end{equation*}
где $c_j\in\mathbb{C}$ --- произвольная постоянная,
$N_j\in W_2^1(\square;
\mathbb{C})$ --- решение \eqref{f:1.11} со средним равным нулю $\left<N_j\right<=0$.

В случае уравнения Бельтрами $\nu=0$ имеем
$N_2=iN_1$.
\end{theorem}

\subsection{Первое приближение к решению задачи  \eqref{f:1.6}}
Пусть правая часть задачи \eqref{f:1.6} $f$ принадлежит пространству $W_2^1(Q;\mathbb{C})$
и пусть $u_\varepsilon$ --- решение задачи Римана--Гильберта \eqref{f:1.6}.
Первым приближением к решению $u_\varepsilon$  задачи  \eqref{f:1.6} является функция
\begin{equation}\label{f:1.15}
u_1^\varepsilon(x)=u^0(x)+\varepsilon\left(\left(N_1(y)-iN_2(y)\right)\partial_zu^0(x)+\left(N_1(y)+iN_2(y)\right)\partial_{\bar{z}}\overline{u^0(x)}\right),
\end{equation}
где $N_1$ и $N_2$ --- периодические решения задачи на ячейке (см. теорему \ref{T_3}),
$y=\varepsilon^{-1}x$; $u^0$ --- решение усредненной задачи: $A_0u^0=f$, $u^0\in W_0(Q;\mathbb{C})$. Оператор $A_0$ --- эллиптический оператор с постоянными коэффициентами,
поэтому $u^0$ принадлежит $W_2^2(Q;\mathbb{C})\cap W_0(Q)$, ввиду свойств регулярности решений эллиптических уравнений.
При этом справедливо соотношение

\begin{equation}\label{f:1.17}
   A_{\varepsilon}u_1^{\varepsilon}=f+\varepsilon r_\varepsilon,
\end{equation}
где
\begin{multline*}
r_\varepsilon=\left(N_1(y)-iN_2(y)\right)
\left(\partial^2_{\bar{z}z}u^0(x)+\mu(y)\,\partial^2_{zz}u^0(x)+\nu(y)\,\partial^2_{\bar{z}\bar{z}}\overline{u^0(x)
}\right)+\\
+\left(N_1(y)+iN_2(y)\right)
\left(\partial^2_{\bar{z}\bar{z}}\overline{u^0(x)}+\mu(y)\,\partial^2_{z\bar{z}}\overline{u^0(x)}+
\nu(y)\,\partial^2_{\bar{z}z}u^0(x)
\right),\quad y=\varepsilon^{-1}x,
\end{multline*}
причем невязка $r_\varepsilon$ принадлежит пространству $L_2(Q;\mathbb{C})$.

Отметим, что первое приближение $u_1^\varepsilon$ принадлежит $W^1_2(Q;\mathbb{C})$.

Главным результатом работ \cite{1,2} является  следующая
\begin{theorem}\label{T_4}
Пусть правая часть $f$ задачи Римана -- Гильберта \eqref{f:1.6} принадлежит пространству $W^1_2(Q;\mathbb{C})$,  $Q$ --- односвязная область с гладкой класса $C^2$ границей, тогда имеют место оценки
\begin{equation}\label{f:1.16}
\|u_\varepsilon-u_1^\varepsilon\|_{W^1_2(Q;\mathbb{C})}\leqslant c\sqrt{\varepsilon}\,\|f\|_{W^1_2(Q;\mathbb{C})},\quad
\|u_\varepsilon-u^0\|_{L_2(Q;\mathbb{C})}\leqslant c\sqrt{\varepsilon}\,\|f\|_{W^1_2(Q;\mathbb{C})},
\end{equation}
где $c>0$ --- постоянная, независящая от $\varepsilon$ и $f$.
\end{theorem}

Первая из оценок \eqref{f:1.16} --- оценка разности точного решения задачи \eqref{f:1.6}  и первого приближения, а вторая есть оценка скорости сходимости точного решения к решению усредненной задачи.

\subsection{Операторные оценки  усреднения}\label{OPOC}
 Имеет место следующая

\begin{theorem}\label{T_5}
Справедливы следующие операторные оценки усреднения задачи Римана--Гильберта \eqref{f:1.6}
\begin{equation}\label{1.15}
\left\|\left(A_\varepsilon^{-1}-A_0^{-1}\right)\partial_{\overline z}^{-1}\right\|_{L_2(Q;\,\mathbb{C})\to L_2(Q;\,\mathbb{C})}
\leq c\,\sqrt\varepsilon,
\end{equation}
\begin{equation}\label{1.16}
\left\|A_\varepsilon^{-1}-A_0^{-1}\right\|_{W_0(Q)\to L_2(Q;\,\mathbb{C})}
\leq c\,\sqrt\varepsilon,
\end{equation}
\begin{equation}\label{1.17}
\left\|\left(A_\varepsilon^{-1}-A_0^{-1}\right)\partial_{\overline z}^{-1}-\varepsilon\Big(N
\partial_zA_0^{-1}
\partial_{\overline z}^{-1}\right.+\\
+\left.M\partial_{\overline z}\overline{A_0^{-1}\partial_{\overline z}^{-1}}\Big)
\right\|_{L_2(Q;\,\mathbb{C})\to W^1_2(Q;\,\mathbb{C})}
\leq c\,\sqrt\varepsilon,
\end{equation}
\vspace{-7mm}
\begin{equation}\label{1.18}
\left\|A_\varepsilon^{-1}-A_0^{-1}-\varepsilon\left(N\partial_zA_0^{-1}
+M\partial_{\overline z}\overline{A_0^{-1}}\right)
\right\|_{W^1_2(Q;\,\mathbb{C})\to W^1_2(Q;\,\mathbb{C})}
\leq c\,\sqrt\varepsilon,
\end{equation}
где $c>0$ --- постоянная; $\partial_{\overline z}^{-1}$ --- оператор обратный к
оператору краевой задачи
\eqref{f:1.6} для уравнения Коши -- Римана;
$\overline{A_0^{-1}\partial_{\overline z}^{-1}}$ --- оператор,
определенный равенством $\overline{A_0^{-1}\partial_{\overline z}^{-1}}{\,v}=\overline{A_0^{-1}\partial_{\overline z}^{-1}{v\,}}$,
аналогичный смысл имеет и оператор $\overline{A_0^{-1}}$. Здесь
$N=N_1(y)-iN_2(y)$, $M=N_1(y)+iN_2(y)$, $y=\varepsilon^{-1}x$,  где $N_1(y)$, $N_2(y)$ --- решения задачи
на ячейке (см. теорему \ref{T_3}); $A_\varepsilon^{-1}$, $A_0^{-1}$
--- обратные операторы к операторам соответствующих  задач Римана--Гильберта.
\end{theorem}

В случае оператора Бельтрами ($\nu=0$), ввиду теоремы \ref{T_3}, имеем
$N=2N_1$, $M=0$. Следовательно, корректоры в \eqref{1.17}, \eqref{1.18} упрощаются
и мы имеем для оператора Бельтрами оценки
\begin{equation*}
\left\|\left(A_\varepsilon^{-1}-A_0^{-1}-\varepsilon N
\partial_zA_0^{-1}\right)
\partial_{\overline z}^{-1}
\right\|_{L_2(Q;\,\mathbb{C})\to W^1_2(Q;\,\mathbb{C})}
\leq c\,\sqrt\varepsilon,
\end{equation*}
\begin{equation*}
\left\|A_\varepsilon^{-1}-A_0^{-1}-\varepsilon N\partial_zA_0^{-1}
\right\|_{W^1_2(Q;\,\mathbb{C})\to W^1_2(Q;\,\mathbb{C})}
\leq c\,\sqrt\varepsilon.
\end{equation*}

