
%====== Section 1 =========

\bibitem{Haar-Tcheb-Shar11} И.И. Шарапудинов. Приближение функций с переменной гладкостью суммами Фурье Лежандра // Математический сборник. 2000. Том 191. Вып. 5. С.143-160.

\bibitem{Haar-Tcheb-Shar13} И.И. Шарапудинов. Смешанные ряды по ортогональным полиномам // Махачкала. Издательство Дагестанского научного центра. 2004.

\bibitem{Haar-Tcheb-Shar15} И.И. Шарапудинов. Аппроксимативные свойства смешанных рядов по полиномам Лежандра на классах $W^r$ // Математический сборник. 2006. Том 197. Вып. 3. С. 135–154.

\bibitem{Haar-Tcheb-Shar16} И.И. Шарапудинов. Аппроксимативные свойства средних типа Валле-Пуссена частичных сумм смешанных рядов по полиномам Лежандра // Математические заметки. 2008. Том 84. Вып. 3. С. 452-471

\bibitem{Haar-Tcheb-Shar18} И.И. Шарапудинов, Т.И. Шарапудинов. Смешанные ряды по полиномам Якоби и Чебышева и их дискретизация // Математические заметки. 2010. Том 88. Вып. 1. С. 116-147.

\bibitem{Haar-Tcheb-Shar19} И.И. Шарапудинов,  Г.Н. Муратова. Некоторые свойства r-кратно интегрированных рядов по системе Хаара // Изв. Сарат. ун-та. Нов. сер. Сер. Математика. Механика. Информатика. 2009. Том 9. Вып. 1. С. 68-76.

\bibitem{Haar-Tcheb-IserKoch} A. Iserles, P.E. Koch, S.P. Norsett and J.M. Sanz-Serna. On polynomials  orthogonal  with respect  to certain Sobolev inner products // J. Approx. Theory. 1991. Vol. 65. p. 151-175.

\bibitem{Haar-Tcheb-MarcelAlfaroRezola } F. Marcellan, M. Alfaro and M.L. Rezola Orthogonal polynomials on Sobolev spaces: old and new directions // Journal of Computational and Applied Mathematics. 1993. Vol. 48. p. 113-131.

\bibitem{Haar-Tcheb-Meijer} H.G. Meijer Laguerre polynimials generalized to a certain discrete Sobolev inner product space // J. Approx. Theory. 1993. Vol. 73. p. 1-16.

\bibitem{Haar-Tcheb-KwonLittl1} K.H. Kwon and L.L. Littlejohn. The orthogonality of the Laguerre polynomials $\{L_n^{(-k)}(x)\}$ for positive integers $k$ // Ann. Numer. Anal. 1995. \textnumero 2. p.289-303.

\bibitem{Haar-Tcheb-KwonLittl2} K.H. Kwon and L.L. Littlejohn. Sobolev orthogonal polynomials and second-order differential equations // Rocky Mountain J. Math. 1998. Vol. 28. p. 547-594.

\bibitem{Haar-Tcheb-MarcelXu} F. Marcellan and Yuan Xu. ON SOBOLEV ORTHOGONAL POLYNOMIALS // arXiv: 6249v1 [math.C.A] 25 Mar 2014. p. 1-40.

\bibitem{Haar-Tcheb-Lopez1995} Lopez G. Marcellan F. Vanassche W. Relative Asymptotics for Polynomials Orthogonal with Respect to a Discrete Sobolev Inner-Product // Constr. Approx.1995. Vol. 11:1. p. 107-137.

\bibitem{Haar-Tcheb-Gonchar1975} А. А. Гончар. О сходимости аппроксимаций Паде для некоторых классов мероморфных функций // Математический сборник. 1975. Вып. 97(139):4(8). C. 607-629.

\bibitem{Haar-Tcheb-Tref1} L.N. Trefethen. Spectral methods in Matlab // Philadelphia. SIAM. 2000.

\bibitem{Haar-Tcheb-Tref2} L.N. Trefethen. Finite difference and spectral methods for ordinary and partial differential equation // Cornell University. 1996.

\bibitem{Haar-Tcheb-SolDmEg} В.В. Солодовников, А.Н. Дмитриев, Н.Д. Егупов. Спектральные методы расчета и проектирования систем управления // Москва. Машиностроение. 1986.

\bibitem{Haar-Tcheb-Pash} С. Пашковский. Вычислительные применения многочленов и рядов Чебышева // Москва. Наука. 1983.

\bibitem{Haar-Tcheb-Arush2010} О. Б. Арушанян, Н. И. Волченскова, С. Ф. Залеткин. Приближенное решение обыкновенных дифференциальных уравнений с использованием рядов Чебышева // Сиб. электрон. матем. изв. 1983. Вып. 7. С. 122-131.

\bibitem{Haar-Tcheb-Arush2013} О. Б. Арушанян, Н. И. Волченскова, С. Ф. Залеткин. Метод решения задачи Коши для обыкновенных дифференциальных уравнений с использованием рядов Чебышeва // Выч. мет. программирование. 2013. Вып. 14:2. С. 203-214.

\bibitem{Haar-Tcheb-Arush2014} О. Б. Арушанян, Н. И. Волченскова, С. Ф. Залеткин. Применение рядов Чебышева для интегрирования обыкновенных дифференциальных уравнений // Сиб. электрон. матем. изв. 2014. Вып. 11. С. 517-531.

\bibitem{Haar-Tcheb-Lukom2016} Д.С. Лукомский, П.А. Терехин. Применение системы Хаара к численному решению задачи Коши для линейного дифференциального уравнения первого порядка // Материалы 18-й международной Саратовской зимней школы «Современные проблемы теории функций и их приложения». Саратов. ООО «Издательство «Научная книга». 2016. С. 171-173.

\bibitem{Haar-Tcheb-MMG2016} М.Г. Магомед-Касумов. Приближенное решение обыкновенных дифференциальных уравнений с использованием смешанных рядов по системе Хаара // Материалы 18-й международной Саратовской зимней школы «Современные проблемы теории функций и их приложения». Саратов. ООО «Издательство «Научная книга». 2016. С. 176-178.

\bibitem{Haar-Tcheb-KashSaak} Б.С. Кашин, А.А. Саакян. Ортогональные ряды // Москва. АФЦ. 1999.

\bibitem{Haar-Tcheb-Sege} Г. Сеге. Ортогональные многочлены // Москва. Физматгиз. 1962.

\bibitem{Haar-Tcheb-Gasper} G. Gasper. Positivity and special function // Theory and appl.Spec.Funct. Edited by Richard A.Askey. 1975. p. 375-433.

\bibitem{Haar-Tcheb-Muckenhoupt} B. Muckenhoupt. Mean convergence of Jacobi series // Proc.Amer. Math. Soc. 1969. Vol. 23. \textnumero 2 p. 306-310.

%====== Section 2 =========


\bibitem{sob-jac-discrete-IserKoch} A. Iserles, P.E. Koch, S.P. Norsett and J.M. Sanz-Serna, On polynomials  orthogonal  with respect  to certain Sobolev inner products // J. Approx. Theory, 1991, vol. 65, pp. 151-175.



\bibitem{sob-jac-discrete-MarcelAlfaroRezola} F. Marcellan, M. Alfaro and M.L. Rezola, Orthogonal polynomials on Sobolev spaces: old and new directions // Journal of Computational and Applied Mathematics, 1993, vol. 48, pp. 113-131, North-Holland.


\bibitem{sob-jac-discrete-Meijer} H.\,G. Meijer, Laguerre polynimials generalized to a certain discrete Sobolev inner product space // J. Approx. Theory, 1993, vol. 73, pp. 1-16.

\bibitem{sob-jac-discrete-Lopez1995} Lopez G. Marcellan F. Vanassche W., Relative Asymptotics for Polynomials Orthogonal with Respect to a Discrete Sobolev Inner-Product // Constr. Approx., 1995, vol. 11:1, pp. 107-137.

\bibitem{sob-jac-discrete-KwonLittl1} K.\,H. Kwon and L.\,L. Littlejohn, The orthogonality of the Laguerre polynomials $\{L_n^{(-k)}(x)\}$ for positive integers $k$ // Ann. Numer. Anal., 1995, issue 2, pp. 289-303.

\bibitem{sob-jac-discrete-KwonLittl2} K.\,H. Kwon and L.\,L. Littlejohn, Sobolev orthogonal polynomials and second-order differential equations // Rocky Mountain J. Math., 1998, vol. 28, pp. 547-594.

\bibitem{sob-jac-discrete-MarcelXu} F. Marcellan and Yuan Xu, ON SOBOLEV ORTHOGONAL POLYNOMIALS // arXiv: 6249v1 [math.C.A] 25 Mar 2014, pp. 1-40.


\bibitem{sob-jac-discrete-Gonchar1975} А. А. Гончар, О сходимости аппроксимаций Паде для некоторых классов мероморфных функций // Матем. сб., 1975, т. 97(139):4(8), стр. 607-629.



\bibitem{sob-jac-discrete-Shar11} И.\,И. Шарапудинов, Приближение функций с переменной гладкостью суммами Фурье Лежандра // Математический сборник, 2000, т. 191, вып. 5, стр. 143-160.


\bibitem{sob-jac-discrete-Shar12} И.\,И. Шарапудинов, Аппроксимативные свойства операторов $\mathcal{Y}_{n+2r}(f)$ и их дискретных аналогов // Математические заметки, 2002, т. 72, вып. 5, стр. 765-795.

\bibitem{sob-jac-discrete-Shar13} И.\,И. Шарапудинов, Смешанные ряды по ортогональным полиномам // Издательство Дагестанского научного центра, 2004, стр. 1-176, Махачкала




\bibitem{sob-jac-discrete-Shar15} И.\,И. Шарапудинов, Аппроксимативные свойства смешанных рядов по полиномам Лежандра на классах $W^r$ // Математический сборник, 2006, т. 197, вып. 3, стр. 135–154.


\bibitem{sob-jac-discrete-Shar16} И.\,И. Шарапудинов, Аппроксимативные свойства средних типа Валле-Пуссена частичных сумм смешанных рядов по полиномам Лежандра // Математические заметки, 2008, т. 84, вып. 3, стр. 452-471.

\bibitem{sob-jac-discrete-Shar17} И.\,И. Шарапудинов, Смешанные ряды по ультрасферическим полиномам // Математический сборник, 2003, т. 194, вып. 3, стр. 115-148.




\bibitem{sob-jac-discrete-Shar18} И.\,И. Шарапудинов, Т.\, И. Шарапудинов, Смешанные ряды по полиномам Якоби и Чебышева и их дискретизация // Математические заметки, 2010, т. 88, вып. 1, стр. 116-147.





\bibitem{sob-jac-discrete-Sege} Г. Сеге, Ортогональные многочлены. Москва: Физматгиз, 1962.

\bibitem{sob-jac-discrete-Timan} А.\,Ф. Тиман, Теория приближения функций действительного переменного. Москва: Физматгиз, 1962.

\bibitem{sob-jac-discrete-Tel} С.\,А.Теляковский, Две теоремы о приближении функций
алгебраическими многочленами // Математический сборник, 1966, т. 70, вып. 2, стр. 252-265.

\bibitem{sob-jac-discrete-Gop} И.\,З. Гопенгауз, К теореме А.Ф.Тимана о приближении
функций многочленами на конечном отрезке // Математические заметки, 1967, т. 1, вып. 2, стр. 163-172. 

\bibitem{meixner-1} Iserles~A., Koch~P.~E., Norsett~S.~P. and Sanz-Serna~J.~M. On polynomials orthogonal with respect to certain Sobolev inner products, // J. Approx. Theory., 1991, vol. 65, pp. 151-175.

\bibitem{meixner-2} Marcellan~F., Alfaro~M. and Rezola~M.~L. Orthogonal polynomials on Sobolev spaces: old and new directions. // North-Holland, 1993, vol. 48, 25 Mar 2014, pp. 113-131.

\bibitem{meixner-3} Meijer~H.~G. Laguerre polynimials generalized to a certain discrete Sobolev inner product space // J. Approx. Theory, 1993, vol. 73, pp. 1-16.

\bibitem{meixner-4} Kwon~K.~H.  and Littlejohn~L.~L. The orthogonality of the Laguerre polynomials $\{L_n^{(-k)}(x)\}$ for positive integers $k$ // Ann. Numer. Anal., 1995, vol. 2, pp. 289-303.

\bibitem{meixner-5} Kwon~K.~H. and Littlejohn~L.~L. Sobolev orthogonal polynomials and second-order differential equations // Ann. Numer. Anal., 1998, vol. 28, pp. 547-594.

\bibitem{meixner-6} Marcellan~F. and Yuan Xu ON SOBOLEV ORTHOGONAL POLYNOMIALS // arXiv: 6249v1 [math.C.A] 25 Mar 2014, pp. 1-40.

\bibitem{meixner-7} Шарапудинов~И.~И. Приближение дискретных функций и многочлены Чебышева, ортогональные на равномерной сетке // Мат. заметки, 2000, т. 67, №~3, С.~460-470. DOI: 10.4213/mzm858

\bibitem{meixner-8} Шарапудинов~И.~И. Приближение функций с переменной гладкостью суммами Фурье- Лежандра // Мат. сборник, 2000, т. 191, №~5, С.~143-160. DOI: 10.4213/sm480

\bibitem{meixner-9} Шарапудинов~И.~И. Аппроксимативные свойства операторов $\mathcal{Y}_{n+2r}(f)$ и их дискретных аналогов // Мат. заметки, 2002, т. 72, №~5, С.~765-795. DOI: 10.4213/mzm466

\bibitem{meixner-10} Шарапудинов~И.~И. Смешанные ряды по ультрасферическим полиномам и их аппроксимативные свойства // Мат. сборник, 2003, т. 194, №~3, С.~115-148. DOI: 10.4213/sm723

\bibitem{meixner-11} Шарапудинов~И.~И. Смешанные ряды по ортогональным полиномам. Махачкала: Дагестан. науч. центр РАН, 2004. С.~1-176.

\bibitem{meixner-12} Шарапудинов~И.~И. Смешанные ряды по полиномам Чебышева, ортогональным на равномерной сетке // Мат. заметки, 2005, т. 78, №~3, C.~442-465. DOI: 10.4213/mzm2599

\bibitem{meixner-13} Шарапудинов~И.~И. Аппроксимативные свойства смешанных рядов по полиномам Лежандра на классах $W^r$ // Математический сборник, 2006, т. 197, №~3, C.~135-154. DOI: 10.4213/sm1539

\bibitem{meixner-14} Шарапудинов~Т.~И. Аппроксимативные свойства смешанных рядов по полиномам Чебышева, ортогональным на равномерной сетке // Вестник Дагестанского научного центра РАН, 2007, т. 29, C.~12–23.

\bibitem{meixner-15} Шарапудинов~И.~И. Аппроксимативные свойства средних типа Валле-Пуссена частичных сумм смешанных рядов по полиномам Лежандра // Мат. заметки, 2008, т. 84, №~3, C.~452-471. DOI: 10.4213/mzm5541

\bibitem{meixner-16} Шарапудинов~И.~И., Муратова~Г.~Н. Некоторые свойства r-кратно интегрированных рядов по системе Хаара // Изв. Сарат. ун-та. Нов. Сер. Математика. Механика. Информатика, 2009, т. 9, №~1, C.~68-76.

\bibitem{meixner-17} Шарапудинов~И.~И., Шарапудинов~Т.~И. Смешанные ряды по полиномам Якоби и Чебышева и их дискретизация // Мат. заметки, 2010, т. 88, №~1, C.~116-147. DOI: 10.4213/mzm6607

\bibitem{meixner-18} Шарапудинов~И.~И. Системы функций, ортогональных по Соболеву, порожденные ортогональными функциями // Современные проблемы теории функций и их приложения.  Материалы 18-й международной Саратовской зимней школы, 2016, C.329-332.

\bibitem{meixner-19} Trefethen~L.~N. Spectral methods in Matlab. Fhiladelphia: SIAM, 2000.

\bibitem{meixner-20} Trefethen~L.~N. Finite difference and spectral methods for ordinary and partial differential equation. Cornell University, 1996.

\bibitem{meixner-21} Магомед-Касумов~Р.~Г. Приближенное решение обыкновенных дифференциальных уравнений с использованием смешанных рядов по системе Хаара // Современные проблемы теории функций и их приложения.  Материалы 18-й международной Саратовской зимней школы, 2016, C.176-178.

\bibitem{meixner-22} Шарапудинов~И.~И. Многочлены, ортогональные на дискретных сетках. Издательство Даг. гос. пед. ун-та. Махачкала, 1997.

\bibitem{meixner-23} Gasper~G. Positiviti and special function // Theory and appl.Spec.Funct. Edited by Richard A.Askey, 1975, 375-433.



\bibitem{sob-tcheb-difference-IserKoch}
A. Iserles, P.E. Koch, S.P. Norsett and J.M. Sanz-Serna,
\textit{On polynomials  orthogonal  with respect  to certain Sobolev inner products},
J. Approx. Theory, 65, 1991, pp. 151-175.



\bibitem{sob-tcheb-difference-MarcelAlfaroRezola}
F. Marcellan, M. Alfaro and M.L. Rezola,
\textit{Orthogonal polynomials on Sobolev spaces: old and new directions},
Journal of Computational and Applied Mathematics, North-Holland, vol. 48, 1993, pp. 113 -- 131.



\bibitem{sob-tcheb-difference-Meijer}
H.\,G. Meijer,
\textit{Laguerre polynimials generalized to a certain discrete Sobolev inner product space},
J. Approx. Theory, vol. 73, 1993, pp. 1-16.


\bibitem{sob-tcheb-difference-KwonLittl1}
K.\,H. Kwon and L.\,L. Littlejohn,
\textit{The orthogonality of the Laguerre polynomials $\{L_n^{(-k)}(x)\}$ for positive integers $k$},
Ann. Numer. Anal., № 2, 1995, pp. 289 -- 303.

\bibitem{sob-tcheb-difference-KwonLittl2}
 K.\,H. Kwon and L.\,L. Littlejohn,
\textit{Sobolev orthogonal polynomials and second-order differential equations}, Rocky Mountain J. Math.,
vol. 28, 1998, pp. 547 –- 594.

\bibitem{sob-tcheb-difference-MarcelXu}
F. Marcellan and Yuan Xu,
\textit{On Sobolev orthogonal polynomials},
arXiv: 6249v1 [math.C.A] 25 Mar 2014, 2014, pp. 1-40.

\bibitem{sob-tcheb-difference-Shar9}
 И.\,И. Шарапудинов,
\textit{Приближение дискретных функций и многочлены Чебышева, ортогональные на равномерной сетке},
Математические заметки, т. 67, \textnumero 3, 2000, с. 460 -- 470.

\bibitem{sob-tcheb-difference-Shar1}
И.\,И. Шарапудинов,
\textit{Приближение функций с переменной гладкостью суммами Фурье Лежандра},
Математический сборник, т. 191, \textnumero 5, 2000, с. 143 -- 160.


\bibitem{sob-tcheb-difference-Shar2}
И.\,И. Шарапудинов,
\textit{Аппроксимативные свойства операторов $\textit{Y}_{n+2r}(f)$ и их дискретных аналогов},
Математические заметки, т. 72, \textnumero 5, 2002, с. 765 -- 795.

\bibitem{sob-tcheb-difference-Shar3}
И.\,И. Шарапудинов,
\textit{Смешанные ряды по ультрасферическим полиномам и их аппроксимативные свойства},
Математический сборник, т. 194, \textnumero 3, 2003, с. 115 -- 148.

\bibitem{sob-tcheb-difference-Shar4}
И.\,И. Шарапудинов,
\textit{Смешанные ряды по ортогональным полиномам},
Издательство Дагестанского научного центра, Махачкала, 2004, с. 1 -- 176.


\bibitem{sob-tcheb-difference-Shar11}
И.\,И. Шарапудинов,
\textit{Смешанные ряды по полиномам Чебышева, ортогональным на равномерной сетке},
Математические заметки, т. 78, \textnumero 3, 2005, с. 442 -- 465.


\bibitem{sob-tcheb-difference-Shar5}
И.\,И. Шарапудинов,
\textit{Аппроксимативные свойства смешанных рядов по полиномам Лежандра на классах $W^r$},
Математический сборник, т. 197, \textnumero 3, 2006, с. 135 -- 154.

\bibitem{sob-tcheb-difference-SharT1}
Т.\,И. Шарапудинов,
\textit{Аппроксимативные свойства смешанных рядов по полиномам Чебышева, ортогональным на равномерной сетке},
Вестник Дагестанского научного центра РАН, т. 29, 2007, с. 12 -- 23.


\bibitem{sob-tcheb-difference-Shar6}
И.\,И. Шарапудинов,
\textit{Аппроксимативные свойства средних типа Валле-Пуссена частичных сумм смешанных рядов по полиномам Лежандра},
Математические заметки, т. 84, \textnumero 3, 2008, с. 452 -- 471.

\bibitem{sob-tcheb-difference-Shar7}
 И.\,И. Шарапудинов,  Г.\, Н. Муратова,
\textit{Некоторые свойства r-кратно интегрированных рядов по системе Хаара},
Изв. Сарат. ун-та. Нов. сер. Сер. Математика. Механика. Информатика, vol. 9, \textnumero 1, 2009, с. 68 -- 76.

\bibitem{sob-tcheb-difference-Shar8}
И.\,И. Шарапудинов, Т.\, И. Шарапудинов,
\textit{Смешанные ряды по полиномам Якоби и Чебышева и их дискретизация},
Математические заметки, т. 88, \textnumero 1, 2010, с. 116 -- 147.


\bibitem{sob-tcheb-difference-SharII}
И.\,И. Шарапудинов,
\textit{Системы функций, ортогональных по Соболеву, порожденные ортогональными функциями},
Современные проблемы теории функций и их приложения.  Материалы 18-й международной Саратовской зимней школы,
2016, с. 329 -- 332.

\bibitem{sob-tcheb-difference-Cheb1}
П.\,Л. Чебышев,
\textit{О непрерывных дробях}, Полн.собр.соч., Изд.АН СССР, Москва, т. 2, 1947, с. 103 -- 126.

\bibitem{sob-tcheb-difference-Cheb2}
П.\,Л. Чебышев,
\textit{Об одном новом ряде}, Полн.собр.соч., Изд.АН СССР, Москва, т. 2, 1947, с. 236 -- 238.

\bibitem{sob-tcheb-difference-Cheb3}
П.\,Л. Чебышев,
\textit{Об интерполировании по способу наименьших квадратов},
Полн.собр.соч., Изд.АН СССР, Москва, т. 2, 1947, с. 314 -- 334.

\bibitem{sob-tcheb-difference-Cheb4}
П.\,Л. Чебышев,
\textit{Об интерполировании}, Полн.собр.соч., Изд.АН СССР, Москва, т. 2, 1947, с. 357 -- 374.

\bibitem{sob-tcheb-difference-Cheb5}
П.\,Л. Чебышев,
\textit{Об интерполировании величин  равноотстоящих (1875)},
Полн.собр.соч., Изд.АН СССР, Москва, т. 2, 1947, с. 66 -- 87.

\bibitem{sob-tcheb-difference-SolDmEg}
В.В. Солодовников, А.Н. Дмитриев, Н.Д. Егупов,
\textit{Спектральные методы расчета и проектирования систем управления}, 1986, Машиностроение, Москва.


\bibitem{sob-tcheb-difference-Tref1}
L.N. Trefethen,
\textit{Spectral methods in Matlab}, 2000, SIAM, Fhiladelphia.

\bibitem{sob-tcheb-difference-Tref2}
L.N. Trefethen,
\textit{Finite difference and spectral methods for ordinary and partial differential equation},
1996, Cornell University.

\bibitem{sob-tcheb-difference-Shar16}
И.\,И. Шарапудинов,
\textit{Асимптотические свойства и весовые оценки многочленов Чебышева–Хана алгебраическими многочленами},
Математический сборник, т. 183, \textnumero 3, 1991, с. 408 -- 420.

\bibitem{sob-tcheb-difference-Shar17}
И.\,И. Шарапудинов,
\textit{Об асимптотике многочленов Чебышева, ортогональных на конечной системе точек},
Вестник МГУ. Серия 1, т. 1, 1992, с. 29 -- 35.

\bibitem{sob-tcheb-difference-Shar18}
И.\,И. Шарапудинов,
\textit{Многочлены, ортогональные на дискретных сетках},
Издательство Даг. гос. пед. ун-та. Махачкала, 1997.

\bibitem{sob-tcheb-difference-MagKas}
М.\,Г. Магомед-Касумов,
\textit{Приближенное решение обыкновенных дифференциальных уравнений с использованием смешанных рядов по системе Хаара},
Современные проблемы теории функций и их приложения. Материалы 18-й международной Саратовской зимней школы.  27 января -- 3 февраля 2016, c. 176 -- 178.

\bibitem{sob-tcheb-difference-SharTim1}
Т.\,И. Шарапудинов,
\textit{Аппроксимативные свойства смешанных рядов по полиномам Чебышева, ортогональным на равномерной сетке},
Вестник Дагестанского научного центра РАН, т. 29, 2007, с. 12 -- 23.

\bibitem{sob-lag-sb-KwonLittl1} K.\,H. Kwon and L.\,L. Littlejohn, The orthogonality of the Laguerre polynomials $\{L_n^{(-k)}(x)\}$ for positive integers $k$ // Ann. Numer. Anal., 1995, issue 2, pp. 289-303.

\bibitem{sob-lag-sb-KwonLittl2} K.\,H. Kwon and L.\,L. Littlejohn, Sobolev orthogonal polynomials and second-order differential equations // Rocky Mountain J. Math., 1998, vol. 28, pp. 547-594.

\bibitem{sob-lag-sb-MarcelAlfaroRezola } F. Marcellan, M. Alfaro and M.L. Rezola, Orthogonal polynomials on Sobolev spaces: old and new directions // Journal of Computational and Applied Mathematics, 1993, vol. 48, pp. 113-131., North-Holland.

\bibitem{sob-lag-sb-IserKoch } A. Iserles, P.E. Koch, S.P. Norsett and J.M. Sanz-Serna, On polynomials  orthogonal  with respect  to certain Sobolev inner products // J. Approx. Theory, 1991, vol. 65, pp. 151-175.

\bibitem{sob-lag-sb-Meijer}
 H.\,G. Meijer, Laguerre polynimials generalized to a certain,  Laguerre polynimials generalized to a certain discrete Sobolev inner product space // J. Approx. Theory, 1993, vol. 73, pp. 1-16.

\bibitem{sob-lag-sb-MarcelXu} F. Marcellan and Yuan Xu, ON SOBOLEV ORTHOGONAL POLYNOMIALS //  arXiv: 6249v1 [math.C.A] 25 Mar 2014, pp. 1-40.

\bibitem{sob-lag-sb-Shar11} И.\,И. Шарапудинов, Приближение функций с переменной гладкостью суммами Фурье Лежандра // Математический сборник, 2000, т. 191, вып. 5, стр. 143-160.

\bibitem{sob-lag-sb-Shar12} И.\,И. Шарапудинов, Аппроксимативные свойства операторов $\mathcal{Y}_{n+2r}(f)$ и их дискретных аналогов // Математические заметки, 2002, vol. 72, issue 5, pp. 765-795.

\bibitem{sob-lag-sb-Shar13} И.\,И. Шарапудинов, Смешанные ряды по ортогональным полиномам // Издательство Дагестанского научного центра, 2004, стр. 1-176, Махачкала.




\bibitem{sob-lag-sb-Shar14} И.\,И. Шарапудинов, Смешанные ряды по полиномам Чебышева, ортогональным на равномерной сетке // Математические заметки, 2005, т. 78, вып. 3, стр. 442-465.

\bibitem{sob-lag-sb-Shar15} И.\,И. Шарапудинов, Аппроксимативные свойства смешанных рядов по полиномам Лежандра на классах $W^r$ // Математический сборник, 2006, т. 197, вып. 3, стр. 135–154.


\bibitem{sob-lag-sb-Shar16} И.\,И. Шарапудинов, Аппроксимативные свойства средних типа Валле-Пуссена частичных сумм смешанных рядов по полиномам Лежандра // Математические заметки, 2008, т. 84, вып. 3, стр. 452-471.

\bibitem{sob-lag-sb-Sege} Г. Сеге, Ортогональные многочлены. Москва:Физматгиз. 1962.

\bibitem{sob-lag-sb-AskeyWaiger} R. Askey, S. Wainger, Mean convergence of expansions in Laguerre and Hermite series // Amer. J. Mathem., 1965, vol. 87, pp. 698-708. 


\bibitem{Ram1}
Area~I., Godoy~E., Marcellan~F. Inner products involving differences: the Meixner--Sobolev polynomials // J. Difference Equations Appl. 6 (2000), pp. 1-31.

\bibitem{Ram2}
Marcellan~F., Xu~Y. On Sobolev orthogonal polynomials // Expositiones Mathematicae, 33, №~3, 2015, pp.308-352.

\bibitem{Ram3}
Fernandez~L., Teresa E. Perez, Miguel A. Pinar, Xu~Y. Weighted Sobolev orthogonal polynomials on the unit ball // Journal of Approximation Theory, 171, 2013, pp. 84–104.

\bibitem{Ram4}
Antonia M. Delgado, Fernandez~L., Doron S. Lubinsky, Teresa E. Perez, Miguel A. Pinar. Sobolev orthogonal polynomials on the unit ball via outward normal derivatives // Journal of Mathematical Analysis and Applications, 440, №~2, 2016, pp. 716–740.

\bibitem{Ram5}
Fernandez~L., Marcellan~F., Teresa E. Perez, Miguel A. Pinar, Xu~Y. Sobolev orthogonal polynomials on product domains // Journal of Computational and Applied Mathematics, 284, 2015, pp. 202–215.

\bibitem{Ram6}
Lopez~G., Marcellan~F., Vanassche~W. Relative asymptotics for polynomials orthogonal with respect to a discrete Sobolev inner-product // Constr. Approx., 11, №~1, 1995, pp. 107-137.

\bibitem{Ram7}
Iserles~A., Koch~P.~E., Norsett~S.~P., Sanz-Serna~J.~M. On polynomials  orthogonal  with respect  to certain Sobolev inner products,J. Approx. Theory. 65 (1991), pp. 151--175.

\bibitem{Ram8}
Marcellan~F., Alfaro~M., Rezola~M.L. Orthogonal polynomials on Sobolev spaces: old and new directions. North-Holland, V. 48, 1993, 25 Mar 2014, pp. 113--131.

\bibitem{Ram9}
Meijer~H.G. Laguerre polynimials generalized to a certain discrete Sobolev inner product space // J. Approx. Theory, 73, 1993, pp. 1--16.

\bibitem{Ram10}
Kwon~K.~H., Littlejohn~L.L. The orthogonality of the Laguerre polynomials $\{L_n^{(-k)}(x)\}$ for positive integers $k$ // Ann. Numer. Anal. 2, 1995. Pp. 289--303.

\bibitem{Ram11}
Kwon~K.H., Littlejohn~L.L.  Sobolev orthogonal polynomials and second-order differential equations // Ann. Numer. Anal. 28, 1998. Pp. 547--594.

\bibitem{Ram12}
Шарапудинов~И.И. Аппроксимативные свойства операторов $Y_{n+2r}(f)$ и их дискретных аналогов // Мат. заметки, 72, №~5, 2002, C.~765-795. DOI: 10.4213/mzm466

\bibitem{Ram13}
Шарапудинов~И.И. Смешанные ряды по ультрасферическим полиномам и их аппроксимативные свойства // Матем. сб., 194:3 (2003), С.~115--148. DOI: 10.4213/sm723

\bibitem{Ram14}
Гаджиева~З.~Д. Смешанные ряды по полиномам Мейкснера. Кандидатская диссертация - Саратов. Саратовский гос. ун-т. 2004.

\bibitem{Ram15}
Шарапудинов~И.И. Смешанные ряды по ортогональным полиномам. Теория и приложения. Махачкала: Дагестан. науч. центр РАН, 2004. С.~276.

\bibitem{Ram16}
Шарапудинов~И.И. Смешанные ряды по полиномам Чебышева, ортогональным на равномерной сетке // Мат. заметки, 78, №~3, 2005, C.~442-465. DOI: 10.4213/mzm2599

\bibitem{Ram17}
Шарапудинов~И.И. Смешанные ряды по полиномам Чебышева, ортогональным на равномерной сетке // Мат. заметки, 78:3 (2005), C.~442--465. DOI: 10.4213/mzm2599

\bibitem{Ram18}
Шарапудинов~И.И. Аппроксимативные свойства смешанных рядов по полиномам Лежандра на классах $W^r$ //   Математический сборник, 197, №~3, 2006, C.~135--154. DOI: 10.4213/sm1539

\bibitem{Ram19}
Шарапудинов~И.И. Аппроксимативные свойства средних типа Валле-Пуссена частичных сумм смешанных рядов по полиномам Лежандра // Мат. заметки, 84, №~3, 2008, C.~452-471. DOI: 10.4213/mzm5541

\bibitem{Ram20}
Шарапудинов~И.И., Гаджиева~З.Д. Полиномы, ортогональные по Соболеву, порожденные многочленами Мейкснера, Изв. Сарат. ун-та. Нов. сер. Сер. Математика. Механика. Информатика {16}(3), (2016), C.~310--321.

\bibitem{sob-leg-Shar11} И.\,И. Шарапудинов. Приближение функций с переменной гладкостью суммами Фурье Лежандра // Математический сборник, 2000, т. 191, вып. 5, стр. 143 – 160.

\bibitem{sob-leg-KwonLittl1} K.\,H. Kwon and L.\,L. Littlejohn. The orthogonality of the Laguerre polynomials $\{L_n^{(k)}(x)\}$ for positive integers $k$ // Ann. Numer. Anal., 1995, issue 2, pp. 289 – 303.

\bibitem{sob-leg-KwonLittl2} K.\,H. Kwon and L.\,L. Littlejohn. Sobolev orthogonal polynomials and secondorder differential equations // Rocky Mountain J. Math., 1998, vol. 28, pp. 547 – 594.

\bibitem{sob-leg-MarcelAlfaroRezola } F. Marcellan, M. Alfaro and M.L. Rezola. Orthogonal polynomials on Sobolev spaces: old and new directions // Journal of Computational and Applied Mathematics, 1993, vol. 48, pp. 113 – 131.

\bibitem{sob-leg-IserKoch } A. Iserles, P.E. Koch, S.P. Norsett and J.M. SanzSerna. On polynomials orthogonal with respect to certain Sobolev inner products //J. Approx. Theory, 1991, vol. 65, pp. 151 - 175.

\bibitem{sob-leg-Meijer} H.\,G. Meijer. Laguerre polynimials generalized to a certain discrete Sobolev inner product space // J. Approx. Theory, 1993, vol. 73, pp. 116.

\bibitem{sob-leg-Lopez1995} Lopez G. Marcellan F. Vanassche W. Relative Asymptotics for Polynomials Orthogonal with Respect to a Discrete Sobolev InnerProduct // Constr. Approx., 1995, vol. 11:1, pp. 107–137.

\bibitem{sob-leg-MarcelXu} F. Marcellan and Yuan Xu. On Sobolev orthogonal polynomials //  Expositiones Mathematicae, 2015, vol. 33, issue 3, pp. 308–352.

\bibitem{sob-leg-Shar2016} И.И. Шарапудинов. Системы функций, ортогональные по Соболеву, порожденные ортогональными функциями // Материалы 18й международной Саратовской зимней школы «Современные проблемы теории функций и их приложения», стр. 329 - 332, Саратов: ООО «Издательство «Научная книга». 2016.

\bibitem{sob-leg-Gonchar1975} А. А. Гончар. О сходимости аппроксимаций Паде для некоторых классов мероморфных функций // Матем. сб., стр. 607–629, т. 97(139):4(8), 1975.

\bibitem{sob-leg-Tref1}   L.N. Trefethen. Spectral methods in Matlab. STAM. 2000.

\bibitem{sob-leg-Tref2}   L.N. Trefethen. Finite difference and spectral methods for ordinary and partial differential equation. Cornell University. 1996.

\bibitem{sob-leg-SolDmEg}  В.В. Солодовников, А.Н. Дмитриев, Н.Д. Егупов. Спектральные методы расчета и проектирования систем управления. Машиностроение. 1986.

\bibitem{sob-leg-Pash} С. Пашковский. Вычислительные применения многочленов и рядов Чебышева. Москва: Наука. 1983.

\bibitem{sob-leg-MMG2016} М.Г. Магомед-Касумов. Приближенное решение обыкновенных дифференциальных уравнений с использованием смешанных рядов по системе Хаара // Материалы 18й международной Саратовской зимней школы «Современные проблемы теории функций и их приложения», стр. 176 - 178. Саратов: ООО «Издательство «Научная книга». 2016.

\bibitem{valle-pussen-2-NIK} С.\,М.~Никольский. О некоторых методах приближения тригонометрическими суммами // Изв. АН СССР. Сер. матем., 1940, т. 4, вып.  6, стр. 509–520.


\bibitem{valle-pussen-2-EFIM} А.\,В.~Ефимов. О приближении периодических функций суммами Валле-Пуссена // Изв. АН СССР. Сер. матем., 1959,  т. 23, вып. 5, стр. 737–770.

\bibitem{valle-pussen-2-TEl} С.\,А.~Теляковский. О приближении дифференцируемых функций линейными средними их рядов Фурье // Изв. АН СССР. Сер. матем., 1960, т. 24, вып. 2, стр. 213-242.





\bibitem{valle-pussen-2-Zhuk} В.\,В. Жук. Аппроксимация периодических функций. Ленинград. 1982.


\bibitem{valle-pussen-2-mmg} М.\,Г. Магомед-Касумов. Аппроксимативные свойства классических средних Валле-Пуссена для кусочно гладких функций
// Вестник Дагестанского научного центра РАН, 2014, т. 54, стр. 5-12.






\bibitem{valle-pussen-2-Shar7} И. \,И.~Шарапудинов. Аппроксимативные свойства средних Валле-Пуссена на классах типа Соболева с переменным показателем // Вестник Дагестанского научного центра РАН. 2012, вып. 45, стр. 5–13.

\bibitem{valle-pussen-2-Shar8} И. \,И.~Шарапудинов.  Приближение гладких функций в  $L_{2\pi}^{p(x)}$ средними Валле-Пуссена // Известия Саратовского университета. Математика. Механика. Информатика. 2012, т. 13, вып.  1, Часть 1, стр. 45–49.




\bibitem{valle-pussen-2-Shar6} И. \,И.~Шарапудинов. Приближение функций в $L^{p(x)}_{2\pi}$ тригонометрическими полиномами // Известия РАН: Серия математическая. 2013, т. 77, вып. 2, стр. 197–224.




\bibitem{valle-pussen-2-Zigmund} А. Зигмунд. Тригонометрические ряды. т. 1, Москва: Мир. 1965.

