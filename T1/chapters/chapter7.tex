


\chapter{Наиболее значимые результаты исследований}

Сконструированы новые системы функций, ортогональные в смысле Соболева, порожденные классическими ортогональными системами. В качестве порождающих систем рассматривались системы полиномов Чебышева (непрерывный и дискретный варианты) и некоторые другие классические ортогональные системы функций (система Хаара, система полиномов Лагерра и др.). Установлена связь между рядами Фурье по соболевским полиномам и смешанными рядами, введенными ранее Шарапудиновым И.И. Используя методы и техники, разработанные для исследования смешанных рядов, изучены некоторые аппроксимативные свойства рядов Фурье по вновь построенным системам функций. Результаты были доложены на ряде научных конференций и опубликованы в научных журналах и сборниках трудов.


Разработан алгоритм для численно-аналитического решения задачи Коши для обыкновенного дифференциального уравнения на основе полиномов, ортогональных по Соболеву, порожденных дискретными полиномами Чебышева и системой функций Хаара. На основе данного алгоритма была составлена программа, которая позволяет найти решение задачи Коши для линейных обыкновенных дифференциальных уравнений численным методом, основанным на разложении самой функции и ее производных в ряд Фурье по полиномам, ортогональным по Соболеву и порожденным упомянутыми системами функций. Основное преимущество данного подхода заключается в том, что в разложении самой функции и всех ее производных участвуют одни и те же коэффициенты. При этом, ввиду особенностей конструкции соболевского ряда, самым естественным образом оказываются учтенными начальные условия задачи Коши. Программа может быть использована для решения задач, возникающих в математической физике, математическом моделировании, в задачах идентификации линейных систем автоматического регулирования и управления. Созданная программа зарегистрирована как объект интеллектуальной собственности (Свидетельство № 2016617831 о государственной регистрации программы для ЭВМ «MixedHaarDeqSolver»).

\chapter{Статистические данные}

\begin{enumerate}[1.]
  \item Всего за 2016 год по теме проекта было опубликовано 6 работ.
  \item В журналы, входящие в список ВАК, отправлены 3 работы, которые находятся на рассмотрении редакции.	
  \item По теме проекта состоялось 4 выступления на конференциях.
  \item По теме проекта был сделан пленарный доклад на международной Саратовской зимней школе <<Современные проблемы теории функций и их приложения>>, 2016 г.
\end{enumerate}
































%=================Smthng==========================



%Представление функций в виде рядов по тем или иным ортонормированным системам с целью последующего их приближения
%частичными суммами выбранного ортогонального ряда является, пожалуй, одним из самых часто применяемых подходов в теории приближений и ее приложениях. Наряду с задачами математической физики, для решения которых указанный подход является традиционным, появились и продолжают появляться все новые важные задачи, для решения которых также все чаще применяются методы, основанные на представлении функций (сигналов) в виде рядов по подходящим ортонормированным системам (см., например, \cite{shii1, shii2, dedus3, pash4, arush5, tref6, tref7, muku8}).
%
%При этом часто возникает такая ситуация, когда функция (сигнал, временной ряд, изображение и т.д) $f=f(t)$ задана на достаточно длинном промежутке $[0,T]$ и нам требуется разбить этот промежуток на части $[a_j,a_{j+1}]$ $(j=0,1,\ldots,m)$, рассмотреть отдельные фрагменты функции определенные на этих частичных отрезках, представить их в виде рядов по выбранной ортонормированной системе и аппроксимировать каждый такой фрагмент частичными суммами соответствующего ряда. Такая ситуация является типичной для задач, связанных с решением нелинейных дифференциальных уравнений численно-аналитическими методами \cite{pash4, tref6}, обработкой временных рядов и изображений и других \cite{arush5, tref6, tref7}, в которых возникает необходимость разбить заданный ряд данных на части,
%аппроксимировать каждую часть и заменить приближенно исходный временный ряд (изображение) функцией, полученной в результате <<пристыковки>> функций, аппроксимирующих отдельные части. Но тогда в местах <<стыка>> возникают нежелательные разрывы (артефакты) (см.\cite{muku8}), которые искажают общий вид временного ряда (изображения). Такая картина непременно возникает при использовании для приближения <<кусков>> исходной функции сумм Фурье по классическим ортонормированным системам.
%
%В разделе \textbf{2.1} (см. \cite{shii1, shii2}) введены некоторые специальные ряды по ультрасферическим полиномам Якоби, частичные суммы $\sigma_n^\alpha(f,x)$ которых на на концах отрезка $[-1,1]$ совпадает с исходной функцией $f(x)$, т.е. $\sigma_n^\alpha(f,\pm1)=f(\pm1)$. В качестве одного из частных случаев таких рядов возникает ряд вида $\Phi(\theta)=a_\Phi(\theta)+\sin\theta \sum_{k=1}^\infty\varphi_k\sin k \theta$, где $a_\Phi(\theta)={\Phi(0)+\Phi(\pi)\over2}+{\Phi(0)-\Phi(\pi)\over2}\cos\theta$, $\varphi(\theta)=\Phi(\theta)-a_\Phi(\theta),\quad \varphi_k={\frac2\pi} \int\limits_{0}^\pi \varphi(\tau){\sin k\tau\over\sin\tau}d\tau.$
%%В разделе  исследованы аппроксимативные свойства этого ряда в пространстве $C^e_{2\pi}$, состоящем из четных непрерывных $2\pi$-периодических функций.
%Были рассмотрены дискретные аналоги таких рядов и исследованы их аппроксимативные свойства.
%
%
%%%%%%%%%%%%%%%%%%%%%%%%%%%%%%%%%%%%%%%%%
%
%Получена оценка скорости сходимости средних Валле-Пуссена для кусочно-гладких функций из пространства Соболева $2\pi$-периодических функций $W_\infty^{3,\mathds{A}}$.
%
%%%%%%%%%%%%%%%%%%%%%%%%%%%%%%%%%%%%%%%%%
%
%В $\S\S$ \ref{sect-2.1}, \ref{sect-2.3} и \ref{sect-6.2} рассмотрена задача о конструировании аппроксимирующих операторов специального вида, обладающих тем важным свойством, что в окрестностях граничных точек отрезка аппроксимации приближает исходную функцию значительно лучше, чем на внутри отрезка. При этом на всем отрезке приближение предложенными специальными операторами осуществляется не хуже, чем приближение дискретным косинус-преобразованием Фурье. Сконструированные таким образом специальные ряды характеризуются тем, что могут быть эффективно использованы как для <<сглаживания>> ошибок в наблюдениях исходного сигнала в узлах сетки, так и для решения задачи одновременного приближения дифференцируемой функции и ее нескольких производных. При численной реализации некоторых из таких операторов используются быстрые дискретные преобразования, основанные на полиномах Чебышева I и II рода.
%
%С применением указанных фундаментальных результатов, специалистами Отдела разработано четыре программных комплекса:
%\begin{enumerate}
%\item
%система обработки и сжатия сигналов (звука, изображений, временных рядов) на основе специальных дискретных преобразований со свойством <<прилипания>>;
%\item
%программа для осуществления численного дифференцирования функций, заданных в узлах дискретной сетки;
%\item
%программный комплекс для решения задачи идентификации линейных неинвариантных по времени систем;
%\item
%пакет компьютерных программ для исследования сложных динамических систем путем анализа пространственно-временных изменений передаточных функций между параметрами взаимосвязанных процессов, заданными в виде временных рядов.
%\end{enumerate}
%
%
%
%
%
%
%
%%%%%%%%%%%%%%%%%%%%%%%%%%%%%%%%%%%%%%%%%
%
%Имеется существенный класс задач, в которых составление расписания минимальной длительности сводится к правильной раскраске двудольного графа наименьшим возможным количеством цветов, а составление расписания без простоев -- к интервальной раскраске двудольного графа. Исследована сложность задачи интервальной $\Delta$-раскраски двудольного мультиграфа. Построен пример $(6,3)$–бирегулярного графа, не обладающего интервальной $\Delta$–раскраской. Эффективность применения методов теории графов для исследования задач о расписаниях хорошо известна. В свою очередь, исследования в области теории расписаний приводят к появлению новых методов и даже целых направлений в теории графов. Например, проблема $4$-х красок возникла в связи с задачами теории расписаний и разбиений {\cite[с.\,110]{lit02}}.
%
%Отметим, что интерес к задачам раскраски объясняется не только их важностью для приложений (построения и оптимизации расписаний, исследования сетей связи и др.), но и тесными связями с широким спектром \textit{NP}-полных задач. Некоторые из них приведены в пункте \textbf{6.4.2}, посвященном краткому обзору результатов о \textit{правильной} и \textit{интервальной} раскрасках графа.
%
%В пункте \textbf{6.4.3}  рассмотрен вопрос сложности задачи об интервальной раскрашиваемости двудольного мультиграфа и построен пример $(6,3)$-бирегулярного графа, не обладающего интервальной $6$-раскраской.
%
%Для расписания взаимодействия объектов множества двудольной природы \linebreak ${G=(V,E)=(X,Y,E)}$ с предписанными операциями рассмотрена проблема минимизации общей длительности расписания и индивидуальных длительностей участия в расписании
%каждого из объектов $v_i$.
%Более точно, рассмотрены вопросы достижимости этими длительностями своих <<естественных>> нижних границ ($\Delta$ -- для всего расписания и $d_G{v_i}$ -- для каждого объекта $v_i$).
%
%
%Кроме того, было доказано, что для раскрашиваемости \textit{d}-ангуляции \textit{P} по Грюнбауму (т.е. для выполнения равенства \textit{$\chi $'}( $G^*(P)$ ) = \textit{d}) достаточно, чтобы \textit{P} обладала свойством граневой 2-раскрашиваемости (скажем, в черный и белый цвета), причем без требования ориентируемости поверхности. Оказалось, что некоторые известные (и достаточно широкие) классы триангуляций гранево $2$-раскрашиваемы. В частности, гипотеза Грюнбаума (\cite{grunb1}) доказана при каждом $n\ge2$ для всех триангуляций соответствующей ориентируемой поверхности с полным трехдольным графом $K_{n,n,n}$.
%
%
%Была решена прикладная задача воссоздания вне родительской среды объектов, созданных в среде трехмерного моделирования, и выполнения преобразований над ними.
%Если исходные данные к программе являются компьютерными рельефами или иными сложными трехмерными объектами, то одним из наиболее распространенных способов их создания является разработка в среде <<Autodesk 3ds Max>> с последующим экспортом в файлы открытого формата <<.3ds>>. Сложность заключается как в воспроизведении этих трехмерных данных средствами среды программирования, так и в выполнении в интерактивном режиме тех или иных действий, не предусмотренных ранее при создании этих объектов в родительской среде.
%
%
%
%
%
%%%%%%%%%%%%%%%%%%%%%%%%%%%%%%%%%%%%%%%%%
%
%%Исследована задача о конструировании аппроксимирующих операторов, обладающих тем важным свойством, что в окрестностях граничных точек отрезка аппроксимации приближает $f(x)$ значительно лучше, чем на всем отрезке $[0,\pi]$. Кроме того, требуется, чтобы $\sigma_{n,N}(f,x)$ приближал функцию $f(x)$ на всем $[0,\pi]$ не хуже, чем частичные суммы конечного ряда \eqref{iish_gga_5} вида, а также рассматриваются дискретные аналоги таких рядов и исследованы их аппроксимативные свойства.
%
%%%%%%%%%%%%%%%%%%%%%%%%%%%%%%%%%%%%%%%%%
%
%Исследуется задача обращения лучевого преобразования симметричного тензорного поля с источниками на кривой, т.е. восстановления поля $f$ по известной функции $If$. Задача реконструкции решается при условии, что кривая удовлетворяет некоторому условию полноты, обеспечивающему допустимость комплекса прямых (см. \cite{m2-1-gel}).
%
%Вертгейм Л.В. (\cite{m2-4-ver}) получил формулу обращения в $R^{3} $ в предположении, что $If$ задана на комплексе прямых, пересекающих данную кривую $\Gamma$, обладающую свойством: любая плоскость, пересекающая носитель $f$, пересекает $\Gamma$ в 6 точках. Денисюку А.С. \cite{m2-2-den} удалось получить формулу обращения в предположении, что любая плоскость, пересекающая носитель $f$, пересекает $\Gamma$ в 3 точках (условие Кириллова-Туя). Наш результат состоит в том, что даже одной точки пересечения $\Gamma $ произвольной плоскостью достаточно, если известны первые производные функции $h\left(x;\theta \right)$.
%
%Полученная в \textbf{6.7.2} формула обращения обобщает известные формулы, устанавливающие связь между лучевым преобразованием векторных и тензорных полей и преобразованием Радона на плоскости (\cite{m1-2-gelfand, m1-3-palamodov}).
%
%В \textbf{6.7.3} вводится в рассмотрение \textit{n}-мерное семейство двумерных плоскостей; получена новая формула обращения преобразования Радона дифференциальных форм степени $2$, определенного для плоскостей этого семейства.




