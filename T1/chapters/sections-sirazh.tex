\chapter{Асимптотическое разложение по малому параметру решения уравнения Бельтрами}
Отметим, что комплексный потенциал плоско-параллельных процессов происходящих в композитах удовлетворяет
уравнению Бельтрами или обобщенному уравнению Бельтрами, коэффициенты которых периодические по малому
параметру функции.



Для описания процессов в композитах предложено много методов. Одним из эффективных методов исследования
является метод асимптотических разложений. 
Нами получены (см. \cite{sir1})
асимптотическое разложение по малому параметру решения задачи Римана -- Гильберта для уравнения Бельтрами и оценка невязки.

Рассмотрим задачу Римана-Гильберта для уравнения Бельтрами
\begin{equation}\label{Sir_eq1}
\left\{\begin{array}{l}
 A_{\varepsilon} u\equiv \partial_{\bar{z}}u_{\varepsilon}+\mu(\varepsilon^{-1}x) \partial_z u_{\varepsilon}=
 f\in L_2(Q),  \\[8pt]
 u_\varepsilon \in W_0(Q)=\Big\{u\in W^1_2(Q)|\,\,\textrm{ Re\,}u|_{\partial Q}=0, \int\limits_{Q}\textrm{ Im\,}u\, dx=0\Big\},
\end{array}\right.
\end{equation}
где $\mu (x)$ --- измеримая ограниченная комплекснозначная периодическая функция,  удовлетворяющая условию

$$
\underset{x\in\mathbb{R}^2}{\textrm{ vrai}\, \sup}\left|\mu (x)\right|\leqslant k_0 <1, \,\,\, k_0 -
\textrm{ постоянная},
$$
$Q$ --- ограниченная односвязная область плоскости, $\varepsilon$ --- малый параметр.
Задача \eqref{Sir_eq1} однозначно разрешима для любой правой части (см. \cite{sir2}), причем имеет место априорная оценка

$$
c_1 \left\| u_{\varepsilon}\right\|_{W^{1}_{2}(Q)}\leqslant \left\| A_{\varepsilon} u_{\varepsilon}
\right\|_{L_2(Q)}\leqslant c_2 \left\| u_{\varepsilon}\right\|_{W^{1}_{2}(Q)}, \,\,\,
u\in W_0 (Q)
$$
где $c_1$, $c_2>0$ - постоянные зависящие только от $k_0$ и $Q$.

Формальное асимптотическое разложение по малому параметру $\varepsilon$ для задачи \eqref{Sir_eq1},
когда $f\in C^{\infty}(Q)$, имеет вид

\begin{equation}\label{Sir_eq2}
u^{(\infty)}(x)=\sum^{\infty}_{j=1}\varepsilon^{j-1}\left(u_{j-1} (x)+\varepsilon u_j(x,y)\right),
\end{equation}
где $u_j(x,y)$ --- периодические по $y= \varepsilon^{-1}x$ функции (подробнее см. \cite{sir1}).

В работе \cite{sir1} получена оценка невязки $A_\varepsilon \left(u_{\varepsilon} -u^{(k+1)}\right)$

$$
\left\| A_{\varepsilon}\left(u_{\varepsilon}-u^{(k+1)}\right)\right\|_{L_2(Q)}\leqslant c\,\varepsilon ^{k+1},
$$
где $c>0$ -- постоянная независящая от $\varepsilon$; $u^{(k+1)}$ --- частичная сумма ряда \eqref{Sir_eq2}.

\chapter{Частичное усреднение недивергентного эллиптического оператора второго порядка}


Рассмотрим задачу Дирихле

\begin{equation}\label{Sir_eq3}
\left\{\begin{array}{l}
   A_\varepsilon u_\varepsilon \equiv \sum\limits^{2}_{i,j=1} a_{ij} (x_1, \varepsilon^{-1}x_2)=f\in L_2(Q)\\[15pt]
   u_\varepsilon=0\quad \textrm{ на}\quad \partial Q,\qquad u_\varepsilon\in {W^2_2(Q)}, \,\,\,
\end{array}\right.
\end{equation}
где $\varepsilon>0$ --- малый параметр, коэффициенты уравнения периодические гладкие функции,
удовлетворяющие условию равномерной эллиптичности.


Задача Дирихле однозначно разрешима для любой правой части (см. \cite{sir2}) и имеет место априорная оценка

$$
c\left\| u_{\varepsilon}\right\|_{\text{W}^2_2(Q)}\leqslant
\left\| A_{\varepsilon} u_{\varepsilon}\right\|_{\text{L}_2(Q)},
$$
где $c>0$ --- постоянная.

Решение задачи \eqref{Sir_eq3} равномерно в $\bar{Q}$ сходится к решению усредненной задачи (см. \cite{sir3}):

$$
\left\{\begin{array}{l}
   \sum\limits^{2}_{i,j=1} a^{0}_{ij} (x_1) \frac{\partial^2 u_0}{\partial x_i\partial x_j} =f\in L_2(Q),\\[15pt]
   u_0=0\quad \textrm{ на\,}\quad \partial Q,\qquad u_0\in W^2_2(Q), \,\,\,
\end{array}\right.
$$

\vspace{0.5cm}
$$
a^0_{ij}(x_1)=\dfrac{\int\limits^{T}_{0}\dfrac{a_{ij}(x_1, x_2)}{a_{22}(x_1, x_2)}\, dx_2}
{\int\limits^{T}_{0}\dfrac{1}{a_{22}(x_1, x_2)}\,d x_2}, \, \, i,j=1,2.
$$

Усредненное уравнение также является равномерно эллиптическим. 
Заметим, что аналогичное утверждение имеет место, когда коэффициенты локально периодичны по другой переменной.

\chapter{О гельдеровости решений задачи Римана -- Гильберта для системы уравнений Бельтрами}

Рассмотрим задачу Римана-Гильберта

\begin{equation}\label{Sir_eq4}
\left\{\begin{array}{l}
 L u\equiv \partial_{\bar{z}}u+\mu\partial_z u+c_1 u+c_2 \bar{u}=
 f\in L_q(Q;\mathbb{C}), \,\, q>2 \\[15pt]
 u \in W_0(Q)=\{u\in W^1_2(Q; \mathbb{C})|\,\,\textrm{ Re\,}(\bar{\lambda}u)=0\quad \textrm{ на\,}\quad \partial Q\},
\end{array}\right.
\end{equation}
где $\mu$ --- квадратная матрица порядка $n$ с измеримыми ограниченными коэффициентами, удовлетворяющая условию:

$$
\mathop\textrm{ vrai\,sup}_{x\in Q}\|\mu(x)\|\leqslant k_0<1,
$$
$k_0$ --- постоянная, $\|\mu(x)\|$ --- норма матрицы $\mu$, рассматриваемой как оператор умножения; $\lambda(x)$, $x\in \partial Q$ --- гладкая унитарная матрица; $c_1$, $c_2$ --- квадратные матрицы порядка $n$, элементы которых принадлежат $L_{q_0}(Q)$, $q_0>2$; $Q$ --- ограниченная $m$-связная область с гладкой границей, причем связные компоненты границы гомеоморфны окружностям.

Как известно (см. \cite{sir5}), задача \eqref{Sir_eq4} при $q=2$ нетерова. Пусть правая часть $f\in L_q(Q)$, $q>2$ и пусть $u\in W_0(Q)$ --- решение задачи \eqref{Sir_eq4}, тогда
 найдется показатель повышенной суммируемости $p>2$, $p\leqslant q$, зависящий только от постоянной эллиптичности $k_0$ и $n$, такой, что решение $u$ принадлежит
пространству Соболева $W_p^1(Q)$. Кроме того, решение $u$ гельдерово с показателем
$\alpha=(p-2)/p$ и с постоянной, зависящей только от $\|u\|_{W_p^1(Q)}$,
 $k_0$, $n$.
(Подробнее см. \cite{sir4}). 