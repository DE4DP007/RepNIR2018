\chapter{Оценка аппроксимативных свойств полиномов $L_{n,N}(f,x)$ для некоторых кусочно-линейных функций}


\section{Предварительные сведения}



Пусть $N \geq 2$ --- целое положительное число, и
$t_k = 2\pi k/N$, $(0 \leq k \leq N-1)$
--- система узловых точек. Обозначим через
$ L_{n,N}(f) = L_{n,N}(f,x)$, где $1 \leq n \leq \lfloor N/2\rfloor$,
тригонометрический полином порядка $n$ наименьшего квадратического отклонения от функции $f$ на сетке
$\{t_k\}_{k=0}^{N-1}$. Другими словами, полином $L_{n,N}(f,x)$ доставляет минимум для
суммы
$\sum_{k=0}^{N-1}|f(t_k)-T_n(t_k)|^2$
на множестве всех тригонометрических полиномов $T_n$ порядка $n$.
В частности, $L_{\lfloor N/2 \rfloor,N}(f,x)$ --- интерполяционный полином, совпадающий с функцией $f(x)$ в
точках $t_k$.
Легко показать\cite{aggshii_article}, что $L_{n,N}(f,x)$ при $n < N/2$ представляется в виде
\begin{equation*}
	L_{n,N}(f,x) = \sum\limits_{\nu = -n}^{n} c_\nu^{(N)}(f) e^{i\nu x}, \mbox{ где } c_\nu^{(N)}(f) = \frac{1}{N}\sum\limits_{k=0}^{N-1} f(t_k)e^{-i\nu t_k},
\end{equation*}
а когда $n = N/2$ (когда $N$ --- чётное)
\begin{equation} \label{L=L+a}
	L_{N/2,N}(f,x) = L_{N/2-1,N}(f,x) + a_n^{(2n)}(f)\cos nx,
\end{equation}
где
\begin{equation} \label{a_n_formula}
	a_n^{(2n)}(f) = \frac{1}{2n}\sum_{k=0}^{2n-1}f(t_k)\cos nt_k.
\end{equation}
Подробнее о приближении функций тригонометрическими полиномами можно почитать в работах \cite{agg2_bernstein,agg4_erdos,agg7_kalashnikov,agg8_krilov,agg9_marcinkiewicz,agg10_marcinkiewicz,agg11_natanson,LapVPNIK,agg17_turetsky,LapVPZigmund}.

Через
$f_1$ и $f_2$ обозначим $2\pi$-периодические функции, которые на отрезке $[-\pi, \pi]$ совпадают с функциями $|x|$ и $\mbox{sign } x$ соответственно.
В отчетном году было проведено исследование поведения величин $|L_{n,N}(f_1,x) - f_1(x)|$ и $|L_{n,N}(f_2,x) - f_2(x)|$ при $n,N \to \infty$  на отрезке $[-\pi, \pi]$.






\section{Некоторые вспомогательные утверждения}


Рассмотрим ряд Фурье функции $f(x)$:
\begin{equation*}
	f(x) = \sum\limits_{k\in \mathbb{Z}} c_k(f)e^{ikx},
\end{equation*}
где
\begin{equation*}
	c_k(f) = \frac{1}{2\pi}\int\limits_{-\pi}^{\pi}f(t)e^{-ik t} dt, \quad k \in \mathbb{Z},
\end{equation*}
и его частичную сумму порядка $n$
\begin{equation*}
	S_n(f, x) = \sum\limits_{k = -n}^{n} c_k(f)e^{ikx}.
\end{equation*}
Обозначим
$$\Delta^I(\varepsilon) = [-\pi + \varepsilon,- \varepsilon] \cup [\varepsilon, \pi - \varepsilon], \quad 0 < \varepsilon < \pi/2.$$
Кроме того, через $C$ и $C(\varepsilon)$ мы будем обозначать некоторые положительные константы, зависящие только от указанных параметров, вообще говоря, разные в разных местах.

\begin{lemma}[Шарапудинов И.И., \cite{aggshii_article}] \label{theorem sharapudinov}
	Если ряд Фурье функции $f$ сходится в точках $t_k = 2\pi k/N$, тогда имеет место представление
	\begin{equation*}
		L_{n,N}(f,x) = S_n(f,x) + R_{n,N}(f,x), \label{iishar_Ln_eq_Sn_Rn}
	\end{equation*}
	где $2n < N$, и
	\begin{equation}\label{RfromLemma}
		R_{n,N}(f,x) = \frac2\pi \sum\limits_{\mu=1}^{\infty} \int\limits_{-\pi}^{\pi} D_n(x-t) \cos \mu N t f(t) dt.
	\end{equation}
\end{lemma}
Из приведённой леммы следует, что при $n < N/2$
\begin{equation}\label{Ln = Sn + Rn}
	|L_{n,N}(f,x) - f(x)| \leq |S_n(f,x) - f(x)| + |R_{n,N}(f,x)|.
\end{equation}
При чётном $N$ возможен случай, когда $2n = N$, тогда из \eqref{L=L+a} и \eqref{Ln = Sn + Rn} можно записать
\begin{equation}\label{L<SRa}
	|L_{n,2n}(f,x) - f(x)| \leq |S_{n-1}(f,x) - f(x)| + |R_{n-1,2n}(f,x)| + |a_n^{(2n)}(f)|,\quad n=N/2.
\end{equation}



%Из \eqref{Ln = Sn + Rn} и \eqref{L<SRa} видно, что для доказательства вышеуказанных теорем требуется изучить поведение величин $|S_n(f_1,x) - f_1(x)|$, $|S_n(f_2,x) - f_2(x)|$, $|R_{n,N}(f_1,x)|$ и $|R_{n,N}(f_2,x)|$, а также $|a_n^{(2n)}(f_1)|$ и $|a_n^{(2n)}(f_2)|$, для чего были доказаны следующие леммы:

\begin{lemma} \label{lemma_Sn_f1}
	Для величины $|S_n(f_1,x) - f_1(x)|$, где $f_1(x) = |x|$, $x \in [-\pi,\pi]$, справедливы оценки
	\begin{equation} \label{Sn1 -pi pi}
		|S_n(f_1,x) - f_1(x)| \leq \frac{C}{n}, \quad x \in [-\pi, \pi],
	\end{equation}
	\begin{equation} \label{Sn1 eps pi - eps}
		|S_n(f_1,x) - f_1(x)| \leq \frac{C(\varepsilon)}{n^2}, \quad x \in \Delta^I(\varepsilon).
	\end{equation}
\end{lemma}
\begin{lemma} \label{lemma_Sn_f2}
	Для величины $|S_n(f_2,x) - f_2(x)|$, где $f_2(x) = \mbox{sign } x$, $x \in [-\pi,\pi]$, справедлива оценка
	\begin{equation*}
		|S_n(f_2,x) - f_2(x)| \leq \frac{C(\varepsilon)}{n}, \quad x \in \Delta^I(\varepsilon). \label{Sn2 eps pi - eps}
	\end{equation*}
\end{lemma}

\begin{lemma} \label{th_abs_1}
	Для $R_{n,N}(f_1,x)$, при $n < N/2$, справедливы оценки
	\begin{equation*}
		|R_{n,N}(f_1,x)| \leq \frac{\pi}{n}\left(4+\frac{1}{2n}\right) \leq \frac{C}{n}, \quad x \in [-\pi, \pi],
	\end{equation*}
	\begin{equation*}
		|R_{n,N}(f_1,x)| \leq \frac{\pi}{n^2} \left(\frac{1}{6}+\frac{4}{|\sin \varepsilon|}\right) \leq \frac{C(\varepsilon)}{n^2}, \quad x \in \Delta^I(\varepsilon).
	\end{equation*}
\end{lemma}
%Для формулировки следующих лемм необходимы некоторые формулы:
%\begin{equation}
%	R_{n,N}(f_1,x) = \frac{2}{\pi} \sum\limits_{\mu=1}^{\infty} \int\limits_{-\pi}^{\pi} D_n(x-t) \cos \mu N t |t| dt, \label{agg_1_RnN}
%\end{equation}
%где
%\begin{equation}
%	D_n(x-t) = \frac12 + \sum\limits_{k=1}^{n} \cos k (x - t) \label{agg_2_Dn}
%\end{equation}
%--- ядро Дирихле.
%Подставив формулу  \eqref{agg_2_Dn} в \eqref{agg_1_RnN} имеем
Нетрудно показать, что имеет место следующее представление
\begin{equation} \label{R=R1+R2}
	\left|R_{n,N}(f_1,x)\right| \leq \left|R^1_{n,N}(f_1,x)\right| + \left|R^2_{n,N}(f_1,x)\right|,
\end{equation}
где
\begin{equation}\label{R_1_abs}
	R^1_{n,N}(f_1,x) = \frac{1}{\pi} \sum\limits_{\mu=1}^{\infty} \int\limits_{-\pi}^{\pi} |t| \cos \mu N t dt,
\end{equation}
\begin{equation}\label{R2f1}
	R^2_{n,N}(f_1,x) = \frac{2}{\pi} \sum\limits_{\mu=1}^{\infty} \int\limits_{-\pi}^{\pi} |t| \sum\limits_{k=1}^{n} \cos k (x-t) \cos \mu N t dt.
\end{equation}
Величины $\left|R_{n,N}(f_1,x)\right| $ и $\left|R_{n,N}(f_2,x)\right| $ оценены в следующих леммах.
\begin{lemma} \label{lm1}
	Выражение \eqref{R_1_abs} имеет следующую оценку при $n < N/2$:
	\begin{equation*}
		|R^1_{n,N}(f_1,x)| \leq \frac{\pi}{6n^2}, \quad x \in [-\pi,\pi].
	\end{equation*}
\end{lemma}
\begin{lemma} \label{lm3}
	Выражение \eqref{R2f1}	имеет следующую оценку при $n < N/2$:
	\begin{equation*}
		|R^2_{n,N}(f_1,x)| \leq \frac{4\pi}{3n}, \quad x \in [-\pi, \pi],
	\end{equation*}
\end{lemma}
\begin{lemma} \label{lm2}
	Выражение \eqref{R2f1} имеет, также, следующую оценку при $n < N/2$:
	\begin{equation*}
		|R^2_{n,N}(f_1,x)| \leq \frac{4\pi}{n^2|\sin \varepsilon|}, \quad x \in \Delta^I(\varepsilon).
	\end{equation*}
\end{lemma}

\begin{lemma}\label{lemma_R2f1_transformation}
	Выражение \eqref{R2f1} можно представить в следующем виде:
	\begin{equation}\label{R_2 with two sums before abel}
		R^2_{n,N}(f_1,x) = \frac{2}{\pi} \sum\limits_{\mu=1}^{\infty} \sum_{k=1}^{n} \cos kx \left[ \frac{1}{(\mu N - k)^2} + \frac{1}{(\mu N + k)^2} \right]
		\left((-1)^{\mu N + k} - 1\right).
	\end{equation}
\end{lemma}

\begin{lemma} \label{lemma sum cos}
	Справедлива следующая оценка:
	\begin{equation*}
		\left|\sum_{j=1}^{n} ((-1)^{M + j} - 1) \cos jx\right| \leq \frac{2}{|\sin x|},\quad M \in \mathbb{Z}.
	\end{equation*}
\end{lemma}

\begin{lemma} \label{th_f2}
	Для $R_{n,N}(f_2,x)$, при $n < N/2$, справедлива оценка
	\begin{equation*} \label{Rf2_est}
		|R_{n,N}(f_2,x)| \leq \frac{2\pi}{n|\sin \varepsilon|} = \frac{C(\varepsilon)}{n}, \quad x \in \Delta^I(\varepsilon).
	\end{equation*}
\end{lemma}

\begin{lemma} \label{lm_sum_sin}
	Имеет место следующая оценка:
	\begin{equation*}
		\left| \sum\limits_{j=1}^{m} \sin j x (1 - (-1)^{j+M}) \right| \leq \frac{2}{|\sin x|},
	\end{equation*}
	где $m$ и $M$ --- произвольные натуральные числа.
\end{lemma}

\begin{lemma} \label{lemma_an_f1_f2}
	Имеют место оценки
	\begin{equation*}
		|a_n^{(2n)}(f_1)| \leq \frac{\pi}{2n^2}, \quad
		|a_n^{(2n)}(f_2)| \leq \frac{1}{2n}.
	\end{equation*}
\end{lemma}



\section{Оценка погрешности приближения функций $|x|$ и $\mbox{sign } x$ тригонометрическими полиномами $L_{n,N}(f,x)$}

Используя леммы 4.2 -- 4.12 нами получены оценки отклонения полиномов $L_{n,N}(f,x)$, построенных на узлах сетки $t_k = \left\{ \frac{2\pi k}{N} \right\}$, от функций $f_1(x)$ и $f_2(x)$.
%\begin{theorem} \label{th_abs_2}
%	Для отклонения $L_{n,N}(f)$ от функции $f_1$, где $f_1(x) = |x|$ на $[-\pi,\pi]$ и $n \leq \lfloor N/2\rfloor$, справедливы следующие оценки:
%	\begin{equation*}
%		|L_{n,N}(f_1,x) - f_1(x)| \leq \frac{C}{n}, \quad x \in [-\pi,\pi],
%	\end{equation*}
%	\begin{equation*}
%		|L_{n,N}(f_1,x) - f_1(x)| \leq \frac{C(\varepsilon)}{n^2}, \quad x \in \Delta^I(\varepsilon).
%	\end{equation*}
%\end{theorem}
%\begin{theorem} \label{th_sign_2}
%	Для отклонения $L_{n,N}(f)$ от функции $f_2$, где $f_2(x) = \mbox{sign } x$ на $[-\pi,\pi]$ и $n \leq \lfloor N/2\rfloor$, справедлива следующая оценка:
%	\begin{equation*}
%		|L_{n,N}(f,x) - f(x)| \leq \frac{C(\varepsilon)}{n}, \quad x \in \Delta^I(\varepsilon).
%	\end{equation*}
%\end{theorem}
В дальнейшем эти оценки были обобщены для случая полиномов $\hat{L}_{n,N}(f,x)$, построенных на узлах $\hat{t}_k = \left\{ \frac{2\pi k}{N} + u \right\}$, где $u$ --- произвольное действительное число.
Приведем здесь две теоремы, охватывающие оба случая.

\begin{theorem} \label{th2_abs_2}
	Для отклонения $\hat{L}_{n,N}(f)$ от функции $f_1$, где $f_1(x) = |x|$ на $[-\pi,\pi]$ и $n \leq \lfloor N/2\rfloor$, справедливы следующие оценки:
	\begin{equation*}
		|\hat{L}_{n,N}(f_1,x) - f_1(x)| \leq \frac{C}{n}, \quad x \in [-\pi,\pi],
	\end{equation*}
	\begin{equation*}
		|\hat{L}_{n,N}(f_1,x) - f_1(x)| \leq \frac{C(\varepsilon)}{n^2}, \quad x \in \Delta^I(\varepsilon).
	\end{equation*}
\end{theorem}
\begin{theorem} \label{th2_sign_2}
	Для отклонения $\hat{L}_{n,N}(f)$ от функции $f_2$, где $f_2(x) = \mbox{sign } x$ на $[-\pi,\pi]$ и $n \leq \lfloor N/2\rfloor$, справедлива следующая оценка:
	\begin{equation*}
		|\hat{L}_{n,N}(f,x) - f(x)| \leq \frac{C(\varepsilon)}{n}, \quad x \in \Delta^I(\varepsilon).
	\end{equation*}
\end{theorem} 