\section{Двумерные  специальные ряды по системе $\{\sin x\sin nx\}_{n=1}^\infty$  и их аппроксимативные свойства}
\textbf{Изучены аппроксимативные свойства двумерных специальных рядов по системе $\{\sin x\sin nx\}_{n=1}^\infty$}


Пусть $f(x,y)$ -- четная $2\pi$-периодическая по каждой из переменных $x$ и $y$ и интегрируемая на квадрате $[0,\pi]^2$ функция, которая в точках $(i\pi,j\pi)$, $i,j\in \mathbb{Z}$ принимает конечные значения. Положим
$$
S(f)=S(f)(u,v)=f(u,v)-{f(0,v)+f(\pi,v)\over2}-{f(0,v)-f(\pi,v)\over2}\cos u,
$$
$$
H(f)=H(f)(u,v)=f(u,v)-{f(u,0)+f(u,\pi)\over2}-{f(u,0)-f(u,\pi)\over2}\cos v,
$$
$$
O(f)=O(f)(u,v)=S(u,v)-{S(u,0)+S(u,\pi)\over2}-{S(u,0)-S(u,\pi)\over2}\cos v,
$$
$$
\Theta(f)=\Theta(f)(u,v)={f(0,0)-f(0,\pi)-f(\pi,0)+f(\pi,\pi)\over4}\cos u\cos v
$$
$$
+{f(0,0)+f(0,\pi)-f(\pi,0)-f(\pi,\pi)\over4}\cos u
$$
$$
+{f(0,0)-f(0,\pi)+f(\pi,0)-f(\pi,\pi)\over4}\cos v
$$
$$
+{f(0,0)+f(0,\pi)+f(\pi,0)+f(\pi,\pi)\over4}
$$
и заметим, что
$$
f(x,y)=\Theta(f)(x,y)+
$$
$$
O(f)(x,y)+A(f)(x)+B(f)(x)\cos y+c(f)(y)+D(f)(y)\cos x,
$$
где
$$
A(f)(u)={S(f)(u,0)+S(f)(u,\pi)\over2},\quad B(f)(u)={S(f)(u,0)-S(f)(u,\pi)\over2},
$$
$$
C(f)(u)={H(f)(0,v)+H(f)(\pi,v)\over2},\quad D(f)(u)={H(f)(0,v)-H(f)(\pi,v)\over2}.
$$
Определим следующие коэффициенты
$$
o_{k,l}(f)=\frac4{\pi^2}\int\limits_0^\pi\int\limits_0^\pi
 O(f)(u,v){\sin ku\sin lv\over\sin u\sin v}dudv,
 $$
$$
a_k(f)=\frac2{\pi}\int\limits_0^\pi A(f)(u){\sin ku\over\sin u}du,\quad
b_k(f)=\frac2{\pi}\int\limits_0^\pi B(f)(u){\sin ku\over\sin u}du,
$$
$$
c_l(f)=\frac2{\pi}\int\limits_0^\pi C(f)(u){\sin lv\over\sin v}dv,\quad
d_l(f)=\frac2{\pi}\int\limits_0^\pi D(f)(u){\sin lv\over\sin v}dv
$$
и рассмотрим ряды
$$
O(f)(x,y)\sim \sin x\sin y\sum_{k=1}^\infty\sum_{l=0}^\infty o_{k,l}(f)\sin kx\sin ly,
$$
$$
A(f)(x)\sim \sin x\sum_{k=1}^\infty a_{k}(f)\sin kx,
$$
$$
B(f)(x)\sim \sin x\sum_{k=1}^\infty b_{k}(f)\sin kx,
$$
$$
C(f)(y)\sim \sin y\sum_{l=1}^\infty c_{l}(f)\sin ly,
$$
$$
D(f)(y)\sim \sin y\sum_{l=1}^\infty d_{l}(f)\sin ly.
$$
Тогда  мы можем сопоставить  функции $f(x,y)$ следующий специальный ряд
$$
f(x,y)\sim \Theta(f)(x,y)+
$$
$$
 \sin x\sin y\sum_{k=1}^\infty\sum_{l=1}^\infty o_{k,l}(f)\sin kx\sin ly+
$$
$$
\sin x\sum_{k=1}^\infty (a_{k}(f)+b_{k}(f)\cos y)\sin kx+
$$
\begin{equation}
\label{eq.1.2.4.1}
 +\sin y\sum_{l=1}^\infty (c_{l}(f)+ d_{l}(f)\cos x)\sin ly.
\end{equation}
Ряд \eqref{eq.1.2.4.1} будем называть двумерным специальным рядом по системе $\{\sin x\sin nx\}_{n=1}^\infty$. В работе \cite{shii-sinkx2D} исследованы аппроксимативные свойства таких рядов. Показано, что специальные ряды вида \eqref{eq.1.2.4.1} выгодно отличаются от двумерных косинус-рядов Фурье тем, что их частичные суммы вида
$$
S_{n,m}^{\nu,\mu}= \Theta(f)(x,y)+
$$
$$
 \sin x\sin y\sum_{k=1}^{n-1}\sum_{l=1}^{m-1} o_{k,l}(f)\sin kx\sin ly+
$$
$$
\sin x\sum_{k=1}^{\nu-1} (a_{k}(f)+b_{k}(f)\cos y)\sin kx+
$$
$$
 +\sin y\sum_{l=1}^{\mu-1} (c_{l}(f)+ d_{l}(f)\cos x)\sin ly.%\eqno (2)
$$
вблизи границы квадрата $[0,\pi]^2$ обладают значительно лучшими, чем суммы Фурье вида
$$
S_{n,m}(f)(x,y)=\sum_{k=0}^n\sum_{l=0}^ma_{k,l}(f)\cos kx\cos ly.
$$
 аппроксимативными свойствами.
