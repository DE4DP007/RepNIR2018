\chapter{Положительная обратимость матриц и устойчивость
дифференциальных уравнений Ито с запаздываниями}





\section{Предварительные сведения и объект исследования}

Пусть: $(\Omega , {\mathcal F}, ({\mathcal F})_{t\ge0},P)$ ---
стохастический базис; $k^n$ --- линейное пространство $n$--мерных
${\mathcal F}_0$ --- измеримых случайных величин; $\mathcal
B_i,i=1,...,m$ --- независимые стандартные винеровские процессы; $1
\le  p < \infty $; $c_p$ --- положительное число, зависящее от $p$
(\cite{kad8}, с. 65); $E$ --- символ математического ожидания; $|.|$ ---
норма в $R^n$; $||.||$ --- норма $n\times n$--матрицы, согласованная
с нормой в $R^n$; $||.||_X$ --- норма в нормированном пространстве
$X$; $\mu $ --- мера Лебега на $[0,+ \infty)$.

Пусть  $B = (b_{ij})^m_{i,j=1}$ --- $m \times m$--матрица. Матрица
$B$ называют неотрицательной, если $b_{ij} \geq 0$, $i, j = 1,...,m$
и положительной, если $b_{ij} > 0$, $i, j = 1,...,m$.

\begin{definition}\label{kad-def1}\cite{kad9} Матрица $B = (b_{ij})^m_{i,j=1}$ называют
$\mathcal M$--матрицей, если $b_{ij} \leq 0$ при $i, j = 1,...,m$ и
$i\neq j$ и выполнено одно из следующих условий:

\noindent-- для матрицы $B$ существует положительная обратная матрица
$B^{-1}$;

\noindent-- диагональные миноры матрицы $B$ положительны.
\end{definition}
\begin{lemma}\label{kad-lem1}
\cite{kad9} Матрица $B$ является $\mathcal M$--матрицей, если
$b_{ij} \leq 0$ при $i, j = 1,...,m$ и $i\neq j$, а также выполнено
одно из следующих условий:

\noindent-- $b_{ii} > \sum \limits _{j=1\\ i \neq j}^m|b_{ij}|$, $i =
1,...,m$;

\noindent-- $b_{jj} > \sum \limits _{i=1\\ i \neq j}^m|b_{ij}|$, $j =
1,...,m$;

\noindent-- существуют положительные числа $\xi _i$, $i = 1,...,m$ такие, что
$\xi_i b_{ii} > \sum \limits _{j=1\\ i \neq j}^m\xi_j |b_{ij}|$, $i
= 1,...,m$;

\noindent-- существуют положительные числа $\xi _i$, $i = 1,...,m$ такие, что
$\xi_j b_{jj} > \sum \limits _{i=1\\ i \neq j}^m\xi_i |b_{ij}|$, $i
= 1,...,m$.
\end{lemma}
Объектом исследования является система дифференциальных уравнений
Ито с запаздываниями вида
\begin{equation}\label{kad-eq1}
\begin{array}{crl}
dx_i(t) = \left [-a_i(t)x_i(h_i(t))+ \sum
\limits_{j=1}^nF_{ij}(t,x_j(h_{ij}(t)))\right]dt +\\
 \sum \limits_{l=1}^m \left [\sum
\limits_{j=1}^nG^l_{ij}(t,x_j(h^l_{ij}(t)))\right ]d\mathcal B_l(t)
\, (t \ge 0), i = 1,...,n
\end{array}
\end{equation}

с начальными условиями
\begin{equation}\label{kad-eq1a}
x_i(t) = \varphi_i(t) {\,\,} (t <0), i = 1,...,n, %\eqno (1a)
\end{equation}
\begin{equation}\label{kad-eq1b}
x_i(0) = b_i {\,\,}, i = 1,...,n, %\eqno (1b)
\end{equation}
где:

1.  $a_i$ --- измеримая по Лебегу функция, заданная на $[0, \infty)$
такая, что $0 < \bar a_i \leq a_i \leq A_i {\,} {\,} (t \in [0,
\infty))$ $\mu $--почти всюду для некоторых  положительных чисел
$\bar a_i, A_i$ при $i = 1,...,n$;

2.$F_{ij}(.,u)$ --- измеримая по Лебегу функция, заданная на $[0,
\infty)$,  $F_{ij}(t,.)$ --- непрерывная функция на $R^1$ такая, что
$|F_{ij}(t,u)| \leq \bar F_{ij}|u| {\,} {\,} (t \in [0, \infty))$
$\mu $--почти всюду для некоторого положительного числа $\bar
F_{ij}$ при $i,j = 1,...,n$;

3.$G^l_{ij}(.,u)$ --- измеримая по Лебегу функция, заданная на $[0,
\infty)$,  $G^l_{ij}(t,.)$ --- непрерывная функция на $R^1$ такая,
что $|G^l_{ij}(t,u)| \leq \bar G^l_{ij}|u| {\,} {\,} (t \in [0,
\infty))$ $\mu $--почти всюду для некоторого положительного числа
$\bar G^l_{ij}$ при $l = 1,...,m$, $i,j = 1,...,n$;

4.  $h_i, h_{ij}, h^l_{ij}$ --- измеримые по Лебегу функции,
заданные на $[0, \infty)$ такие, что $0 \leq t- h_i(t) \leq \tau _i,
0 \leq t- h_{ij}(t) \leq \tau _{ij}, 0 \leq t- h^l_{ij}(t) \leq \tau
^l_{ij} {\,} {\,} (t \in [0, \infty))$ $\mu $--почти всюду для
некоторых положительных чисел $\tau_i, \tau_{ij}, \tau^l_{ij}$ при
$l = 1,...,m$, $i, j = 1,...,n$;

5.  $\varphi_i$ --- ${\mathcal F}_0$--измеримый скалярный случайный
процесс, заданный на $[\sigma _i, 0)$, где $\sigma _i = \max \{
\tau_i, \tau_{ij}, \tau^l_{ij}, l = 1,...,m, j = 1,...,n \}$;

6.  $b_i$ --- ${\mathcal F}_0$--измеримая скалярная случайная
величина  при  $i = 1,...,n \}$.

В дальнейшем нам понадобятся следующие обозначения:

\noindent-- $b:= col (b_1,...,b_n)$;

\noindent-- $\varphi:= col (\varphi_1,...,\varphi_n)$.

Введем следующее обозначение линейного нормированного
подпространства $k^n$:

$$
k_p^n = \left \{\alpha: \alpha \in k^n,\|\alpha\|_{k_p^n} =
(E|\alpha |^p)^{1/p} < \infty \right \}.
$$

Задача \eqref{kad-eq1}, \eqref{kad-eq1a}, \eqref{kad-eq1b} имеет единственное решение, если
дополнительно предположить, что функции $F_{ij}(t,u)$,
$G^l_{ij}t,u)$ удовлетворяют условию Липщица по $u$ при $l =
1,...,m$, $i,j = 1,...,n$ \cite{kad10}. В дальнейшем будем считать, что
функции $F_{ij}(t,u)$, $G^l_{ij}t,u)$ удовлетворяют условию Липщица
по $u$ при $l = 1,...,m$, $i,j = 1,...,n$.  Обозначим через $x(t, b,
\varphi)$ --- решение системы \eqref{kad-eq1}, удовлетворяющее условиям \eqref{kad-eq1a} и
\eqref{kad-eq1b}, т.е. $x(t, b, \varphi ) = \varphi$ при $t < 0$ и $x(0, b,
\varphi ) = b$. Очевидно, что $x(., b, \varphi ) \in D^n$
\begin{definition}\label{kad-def2}
%{\bf Определение 2.}
Будем говорить, что система \eqref{kad-eq1} глобально
экспоненциально \textit{ $p$-устой\-чи\-ва} ($1 \leq p < \infty$), если
существует положительные числа $K,\lambda$ такие, что для решения
$x(t, b, \varphi)$ задачи \eqref{kad-eq1}, \eqref{kad-eq1a}, \eqref{kad-eq1b} выполнено неравенство
$$
(E|x(t, b, \varphi)|^p)^{1/p} \leq K\exp \{-\lambda t\}\left
(||b||_{k_p^n} + vrai \sup \limits _{t < 0}(E|\varphi
(t)|^p)^{1/p}\right ){\,} {\,} (t \geq 0).
$$
\end{definition}

\begin{lemma}\label{kad-lem2}
%{\bf Лемма 2.}
Пусть $f(s)$ --- скалярный случайный процесс,
интегрируемый по винеровскому процессу $\mathcal B(s)$ на отрезке
$[0, t]$. Тогда справедливо неравенство
\begin{equation}\label{kad-eq2}
  \left (E\left |\int \limits _0^tf(s)d\mathcal B(s)\right
|^{2p}\right )^{1/2p} \leq c_p \left (E\left (\int \limits
_0^t|f(s)|^2d(s)\right )^p\right )^{1/2p}, %\eqno (2)
\end{equation}
где $c_p$ --- некоторое число, зависящее от $p$.
\end{lemma}

Справедливость неравенства \eqref{kad-eq2} следует из неравенства 4 работы \cite{kad8}
(стр. 65), где выписано и конкретное выражение для $c_p$.

\begin{lemma}\label{kad-lem3}
%{\bf Лемма 3.}
Пусть $g(s)$ --- скалярная функция на $[0, \infty)$,
квадрат которой локально суммируема, $f(s)$ --- скалярный случайный
процесс такой, что $\sup \limits _{s \geq 0}(E|f(s)|^{2p})^{1/2p} <
\infty$. Тогда справедливы следующие неравенства
\begin{equation}\label{kad-eq3}
\sup \limits _{t \geq 0}\left(E\left|\int \limits
_0^tg(s)f(s)ds\right|^{2p}\right)^{1/2p} \leq \sup \limits _{t \geq
0}\left (\int \limits _0^t|g(s)|ds\right )\sup \limits _{t \geq
0}(E|f(s)|^{2p})^{1/2p}, %\eqno (3)
\end{equation}
\begin{equation}\label{kad-eq4}
\sup \limits _{t \geq 0}\left(E\left|\int \limits
_0^t(g(s))^2(f(s))^2ds\right|^{p}\right)^{1/2p} \leq \sup \limits
_{t \geq 0}\left (\int \limits _0^t(g(s))^2ds\right)^{1/2}\sup
\limits _{t \geq 0}(E|f(s)|^{2p})^{1/2p}. %\eqno (4)
\end{equation}


\end{lemma}

\section{Основной результат}

Пусть $C$ --- $n\times n$--матрица, элементы которой определены
следующим образом
$$
c_{ii} = 1 - \frac{A_i^2\tau _i^2 + A_i\bar F_{ii}\tau _i  + c_p A_i
\sqrt{\tau _i}\sum \limits _{i=1}^m\bar G^l_{ii} + \bar F_{ii}}{\bar
a_i} - \frac{c_p\sum \limits _{l=1}^m \bar G^l_{ii}}{\sqrt{2\bar
a_i}}, i = 1,...,n,
$$
$$
c_{ij} = - \frac{A_i\bar F_{ij}\tau _i + c_p A_i \sqrt{\tau _i}\sum
\limits _{i=1}^m\bar G^l_{ij} + \bar F_{ij}}{\bar a_i} -
\frac{c_p\sum \limits _{l=1}^m \bar G^l_{ij}}{\sqrt{2\bar a_i}}, i,j
= 1,...,n, i\neq j.
$$

\begin{theorem*}
  Если матрица $C$ является $\mathcal{
M}$-матрицей, то система \eqref{kad-eq1} глобально экспоненциально
$2p$-устойчива.
\end{theorem*}

\section{Следствия основного результата}

Отдельно рассмотрим случай, когда система дифференциальных уравнений
Ито \eqref{kad-eq1} содержит только внедиагональные  нелинейности, т.е. имеет
вид
\begin{equation}\label{kad-eq14}
\begin{array}{crl}
dx_i(t) = \left [-a_i(t)x_i(h_i(t))+ \sum
\limits_{j=1, i \neq j }^nF_{ij}(t,x_j(h_{ij}(t)))\right]dt +\\
 \sum \limits_{l=1}^m \left [\sum
\limits_{j=1, i \neq j}^nG^l_{ij}(t,x_j(h^l_{ij}(t)))\right
]d\mathcal B_l(t) \, (t \ge 0), i = 1,...,n.
\end{array}
%\eqno (14 )
\end{equation}


Пусть $C$ --- $n\times n$--матрица, элементы которой определены
следующим образом
$$
c_{ii} = 1 - \frac{A_i^2\tau _i^2}{\bar a_i}, i = 1,...,n,
$$
$$
c_{ij} = - \frac{A_i\bar F_{ij}\tau _i + c_p A_i \sqrt{\tau _i}\sum
\limits _{i=1}^m\bar G^l_{ij} + \bar F_{ij}}{\bar a_i}
-\frac{c_p\sum \limits _{l=1}^m \bar G^l_{ij}}{\sqrt{2\bar a_i}},
i,j = 1,...,n, i\neq j.
$$
Тогда из теоремы получим

\begin{corollary}\label{kad-col1}
Если матрица $C$ является $\mathcal
M$-матрицей, то система \eqref{kad-eq14} глобально экспоненциально
$2p$-устойчива.
\end{corollary}
Пусть теперь в линейной части системы \eqref{kad-eq1} отсутствуют запаздывания,
$h_i(t) = t$ при $i = 1,...,n$,  т.е. имеет вид
\begin{equation}\label{kad-eq15}
\begin{array}{crl}
dx_i(t) = \left [-a_i(t)x_i(t)+ \sum
\limits_{j=1}^nF_{ij}(t,x_j(h_{ij}(t)))\right]dt +\\
 \sum \limits_{l=1}^m \left [\sum
\limits_{j=1}^nG^l_{ij}(t,x_j(h^l_{ij}(t)))\right ]d\mathcal B_l(t)
\, (t \ge 0), i = 1,...,n.
\end{array}
%\eqno (15)
\end{equation}
Пусть $C$ --- $n\times n$-матрица, элементы которой определены
следующим образом
$$
c_{ii} = 1 - \frac{\bar F_{ii}}{\bar a_i} - \frac{c_p\sum \limits
_{l=1}^m \bar G^l_{ii}}{\sqrt{2\bar a_i}}, i = 1,...,n, c_{ij} = -
\frac{\bar F_{ij}}{\bar a_i} - \frac{c_p\sum \limits _{l=1}^m \bar
G^l_{ij}}{\sqrt{2\bar a_i}}, i,j = 1,...,n, i\neq j.
$$
Тогда из теоремы получим

\begin{corollary}\label{kad-col2}
Если матрица $C$
является $\mathcal{M}$-матрицей, то система \eqref{kad-eq15} глобально
экспоненциально $2p$-устойчива.
\end{corollary}
Пусть система \eqref{kad-eq1} линейная система, т.е. имеет вид
\begin{equation}\label{kad-eq16}
\begin{array}{crl}
dx_i(t) = \left [-a_i(t)x_i(h_i(t))+ \sum
\limits_{j=1}^nF_{ij}(t)x_j(h_{ij}(t))\right]dt +\\
 \sum \limits_{l=1}^m \left [\sum
\limits_{j=1}^nG^l_{ij}(t)x_j(h^l_{ij}(t))\right ]d\mathcal B_l(t)
\, (t \ge 0), i = 1,...,n,
\end{array}
%\eqno (16)
\end{equation}
где дополнительно $F_{ij}$ --- измеримая по Лебегу функция, заданная
на $[0, \infty)$ такая, что $|F_{ij}(t)| \leq \bar F_{ij} {\,} {\,}
(t \in [0, \infty))$ $\mu $-почти всюду для некоторого
положительного числа $\bar F_{ij}$ при $i,j = 1,...,n$ и $G^l_{ij}$
--- измеримая по Лебегу функция, заданная на $[0, \infty)$ такая, что
$|G^l_{ij}(t)| \leq \bar G^l_{ij} {\,} {\,} (t \in [0, \infty))$
$\mu $--почти всюду для некоторого положительного числа $\bar
G^l_{ij}$ при $l = 1,...,m$, $i,j = 1,...,n$.
Пусть $C$ --- $n\times n$--матрица, элементы которой определены
следующим образом
$$
c_{ii} = 1 - \frac{A_i^2\tau _i^2 + A_i\bar F_{ii}\tau _i  + c_p A_i
\sqrt{\tau _i}\sum \limits _{i=1}^m\bar G^l_{ii} + \bar F_{ii}}{\bar
a_i} - \frac{c_p\sum \limits _{l=1}^m \bar G^l_{ii}}{\sqrt{2\bar
a_i}}, i = 1,...,n,
$$
$$
c_{ij} = - \frac{A_i\bar F_{ij}\tau _i + c_p A_i \sqrt{\tau _i}\sum
\limits _{i=1}^m\bar G^l_{ij} + \bar F_{ij}}{\bar a_i} -
\frac{c_p\sum \limits _{l=1}^m \bar G^l_{ij}}{\sqrt{2\bar a_i}}, i,j
= 1,...,n, i\neq j.
$$
Тогда из теоремы получим
\begin{corollary}\label{kad-col3}
Если матрица $C$ является $\mathcal
M$-матрицей, то система \eqref{kad-eq16} глобально экспоненциально
$2p$-устойчива.
\end{corollary}
Пусть система \eqref{kad-eq16} имеет вид
\begin{equation}\label{kad-eq17}
\begin{array}{crl}
dx_i(t) = \left [-a_i(t)x_i(h_i(t))+ \sum
\limits_{j=1, i\neq j}^nF_{ij}(t)x_j(h_{ij}(t))\right]dt +\\
 \sum \limits_{l=1}^m \left [\sum
\limits_{j=1, i\neq j}^nG^l_{ij}(t)x_j(h^l_{ij}(t))\right ]d\mathcal
B_l(t) \, (t \ge 0), i = 1,...,n.
\end{array}
%\eqno (17)
\end{equation}
Пусть $C$ --- $n\times n$-матрица, элементы которой определены
следующим образом
$$
c_{ii} = 1 - \frac{A_i^2\tau _i^2 }{\bar a_i} - \frac{c_p\sum
\limits _{l=1}^m \bar G^l_{ii}}{\sqrt{2\bar a_i}}, i = 1,...,n,
$$
$$
c_{ij} = - \frac{A_i\bar F_{ij}\tau _i + c_p A_i \sqrt{\tau _i}\sum
\limits _{i=1}^m\bar G^l_{ij} + \bar F_{ij}}{\bar a_i} -
\frac{c_p\sum \limits _{l=1}^m \bar G^l_{ij}}{\sqrt{2\bar a_i}}, i,j
= 1,...,n, i\neq j.
$$
Тогда из следствия \ref{kad-col3} получим
\begin{corollary}\label{kad-col4}
Если матрица $C$ является $\mathcal{M}$-матрицей, то система \eqref{kad-eq17}) глобально экспоненциально $2p$-устойчива.
\end{corollary}
Пусть система \eqref{kad-eq16} имеет вид
\begin{equation}\label{kad-eq18}
  \begin{array}{crl}
dx_i(t) = \left [-a_i(t)x_i(t)+ \sum
\limits_{j=1}^nF_{ij}(t)x_j(h_{ij}(t))\right]dt +\\
 \sum \limits_{l=1}^m \left [\sum
\limits_{j=1}^nG^l_{ij}(t)x_j(h^l_{ij}(t))\right ]d\mathcal B_l(t)
\, (t \ge 0), i = 1,...,n.
\end{array}
%\eqno (18)
\end{equation}


Пусть $C$ --- $n\times n$-матрица, элементы которой определены
следующим образом
$$
c_{ii} = 1 - \frac{\bar F_{ii}}{\bar a_i} - \frac{c_p\sum \limits
_{l=1}^m \bar G^l_{ii}}{\sqrt{2\bar a_i}}, i = 1,...,n, c_{ij} = -
\frac{\bar F_{ij}}{\bar a_i} - \frac{c_p\sum \limits _{l=1}^m \bar
G^l_{ij}}{\sqrt{2\bar a_i}}, i,j = 1,...,n, i\neq j.
$$
Тогда из следствия \ref{kad-col3} получим
\begin{corollary}\label{kad-col5}
Если матрица $C$ является $\mathcal{M}$-матрицей, то система \eqref{kad-eq18} глобально экспоненциально $2p$-устойчива.
\end{corollary}
Пусть система \eqref{kad-eq16}  имеет вид
\begin{equation}\label{kad-eq19}
\begin{array}{crl}
dx_i(t) = \left [-a_i(t)x_i(t)+ \sum
\limits_{j=1, i\neq j}^nF_{ij}(t)x_j(h_{ij}(t))\right]dt +\\
 \sum \limits_{l=1}^m \left [\sum
\limits_{j=1, i \neq j}^nG^l_{ij}(t)x_j(h^l_{ij}(t))\right
]d\mathcal B_l(t) \, (t \ge 0), i = 1,...,n.
\end{array}
%\eqno (19)
\end{equation}
Пусть $C$ --- $n\times n$-матрица, элементы которой определены
следующим образом
$$
c_{ii} = 1, i = 1,...,n, c_{ij} = - \frac{\bar F_{ij}}{\bar a_i} -
\frac{c_p\sum \limits _{l=1}^m \bar G^l_{ij}}{\sqrt{2\bar a_i}}, i,j
= 1,...,n, i\neq j.
$$
Тогда из следствия \ref{kad-col3} получим
\begin{corollary}\label{kad-col6}
Если матрица $C$ является $\mathcal{M}$-матрицей, то система \eqref{kad-eq19} глобально экспоненциально $2p$-устойчива.
\end{corollary}
\begin{corollary}\label{kad-col7}
Пусть в системе \eqref{kad-eq1} $n = 2$ и выполнены
неравенства
$$
\sqrt{2}(A_1^2\tau _1^2 + A_1\bar F_{11}\tau _1  + c_p A_1
\sqrt{\tau _1}\sum \limits _{i=1}^m\bar G^l_{11} + \bar F_{11}) +
\sqrt{\bar a_1}c_p\sum \limits _{l=1}^m \bar G^l_{11} < \sqrt{2}\bar
a_1,
$$
$$
\begin{array}{crl}
(\sqrt{2}\bar a_1 - \sqrt{2}(A_1^2\tau _1^2 + A_1\bar F_{11}\tau _1
+ c_p A_1 \sqrt{\tau _1}\sum \limits _{i=1}^m\bar G^l_{11} + \bar
F_{11}) - \sqrt{\bar a_1}c_p\sum \limits _{l=1}^m \bar G^l_{11}) \times  \\
(\sqrt{2}\bar a_2 - \sqrt{2}(A_2^2\tau _2^2 + A_2\bar F_{22}\tau _2
+ c_p A_2 \sqrt{\tau _2}\sum \limits _{i=1}^m\bar G^l_{22} + \bar
F_{22}) - \sqrt{\bar a_2}c_p\sum \limits _{l=1}^m \bar G^l_{22}) >
\\
(\sqrt{2}(A_1\bar F_{12}\tau _1 + c_p A_1 \sqrt{\tau _1}\sum \limits
_{i=1}^m\bar G^l_{12} + \bar F_{12}) + \sqrt{\bar a_1}c_p\sum
\limits _{l=1}^m \bar G^l_{12})\\
(\sqrt{2}(A_2\bar F_{21}\tau _2 + c_p A_2 \sqrt{\tau _2}\sum \limits
_{i=1}^m\bar G^l_{21} + \bar F_{21}) + \sqrt{\bar a_2} c_p\sum
\limits _{l=1}^m \bar G^l_{21}).
\end{array}
$$
Тогда система \eqref{kad-eq1} глобально экспоненциально $2p$-устойчива.
\end{corollary}
Из следствий \ref{kad-col5} и \ref{kad-col7} получим
\begin{corollary}\label{kad-col8}
Пусть в системе (18) $n = 2$ и выполнены
неравенства
$$
\sqrt{2}\bar F_{11} + \sqrt{\bar a_1}c_p\sum \limits _{l=1}^m \bar
G^l_{11} < \sqrt{2}\bar a_1,
$$
$$
\begin{array}{crl}
(\sqrt{2}\bar a_1 - \sqrt{2}\bar F_{11} - \sqrt{\bar a_1}c_p\sum
\limits _{l=1}^m \bar G^l_{11})(\sqrt{2}\bar a_2 - \sqrt{2}\bar
F_{22} - \sqrt{\bar a_2}c_p\sum \limits _{l=1}^m \bar G^l_{22}) >\\
(\sqrt{2}\bar F_{12} + \sqrt{\bar a_1}c_p\sum \limits _{l=1}^m \bar
G^l_{12})(\sqrt{2}F_{21} + \sqrt{\bar a_2} c_p\sum \limits _{l=1}^m
\bar G^l_{21}).
\end{array}
$$
Тогда система \eqref{kad-eq18} глобально экспоненциально $2p$-устойчива.
\end{corollary}
Из следствий \ref{kad-col6} и \ref{kad-col8} получим
\begin{corollary}\label{kad-col9}
Пусть в системе \eqref{kad-eq19} $n = 2$ и выполнены
условия

$$
\begin{array}{crl}
(\sqrt{2}\bar F_{12} + \sqrt{\bar a_1}c_p\sum \limits _{l=1}^m \bar
G^l_{12})(\sqrt{2}F_{21} + \sqrt{\bar a_2} c_p\sum \limits _{l=1}^m
\bar G^l_{21}) < 2\bar a_1 \bar a_2 .
\end{array}
$$
Тогда система \eqref{kad-eq19} глобально экспоненциально $2p$-устойчива.
\end{corollary}

\section{Пример}
В качестве примера рассмотрим система двух уравнений следующего
вида
\begin{equation}\label{kad-eq20}
\begin{array}{crl}
dx_1(t) = \left [-a_{1}x_1(t-h_1)+ a_{11} F_{11}(x_1(t-h_{11}))+
a_{12} F_{12}(x_2(t-h_{12})) \right ]dt+ \\
 \left [b_{11}G_{11}(x_1(t-\tau _{11})) + b_{12}G_{12}(x_2(t-\tau _{12}))\right ]d\mathcal B(t)
\, (t \ge 0),\\
dx_2(t) = \left [-a_{2}x_1(t-h_2))+ a_{21} F_{21}(x_1(t-h_{21}))+
a_{22} F_{22}(x_2(t-h_{22})) \right ]dt+ \\
 \left [b_{21}G_{21}(x_1(t-\tau _{21})) + b_{22}G_{22}(x_2(t-\tau _{22}))\right ]d\mathcal B(t)
\, (t \ge 0),
\end{array}
%\eqno (20)
\end{equation}
где $a_1, a_2, h_{ij}, \tau _{ij}, a_{ij}, b_{ij}, i, j = 1,2 $ ---
некоторые положительные числа, $F_{ij}, G_{ij}, i, j = 1, 2$ ---
непрерывные скалярные функция на $(-\infty, +\infty)$ такие, что
$|F_{ij}(u)| \leq |u|, |G_{ij}(u)| \leq |u|, i, j = 1, 2$, $\mathcal
B$ --- стандартный винеровский процесс.
Из следствия \ref{kad-col7} получим, что, если для системы \eqref{kad-eq20} выполнены
неравенства
$$
\sqrt{2}(a_1^2h_1^2 + a_1a_{11}h_1  + c_p a_1 \sqrt{h_1}b_{11} +
a_{11}) + \sqrt{a_1}c_pb_{11} < \sqrt{2}a_1,
$$
$$
\begin{array}{crl}
(\sqrt{2}a_1 - \sqrt{2}(a_1^2h_1^2 + a_1a_{11}h_1 + c_p a_1
\sqrt{h_1}b_{11} + a_{11}) - \sqrt{a_1}c_pb_{11}) \times  \\
(\sqrt{2}a_2 - \sqrt{2}(a_2^2h_2^2 + a_2a_{22}h_2 + c_p a_2
\sqrt{h_2}b_{22} + a_{22}) - \sqrt{a_2}c_pb_{22})
>
\\
(\sqrt{2}(a_1a_{12}h_1 + c_p a_1 \sqrt{h_1}b_{12} + a_{12}) +
\sqrt{a_1}c_pb_{12})(\sqrt{2}(a_2a_{21}h_2 + c_p a_2
\sqrt{h_2}b_{21} + a_{21}) + \sqrt{a_2} c_pb_{21}),
\end{array}
$$
то она глобально экспоненциально $2p$-устойчива.
Пусть в системе \eqref{kad-eq20} $a_{ii}, b_{ii}, i = 1,2 $/ , тогда, если для
нее выполнены неравенства
$$
a_1h_1^2 < 1,
$$
$$
\begin{array}{crl}
(\sqrt{2}a_1 - \sqrt{2}a_1^2h_1^2)(\sqrt{2}a_2 - \sqrt{2}a_2^2h_2^2)
> \\
(\sqrt{2}(a_1a_{12}h_1 + c_p a_1 \sqrt{h_1}b_{12} + a_{12}) +
\sqrt{a_1}c_pb_{12})(\sqrt{2}(a_2a_{21}h_2 + c_p a_2
\sqrt{h_2}b_{21} + a_{21}) + \sqrt{a_2} c_pb_{21}),
\end{array}
$$
то она глобально экспоненциально $2p$-устойчива. 