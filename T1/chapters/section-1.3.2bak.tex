\section{}
\textbf{Улучшены достаточные условия существования положительного решения двухточечной краевой задачи для одного класса нелинейных обыкновенных дифференциальных уравнений четвертого порядка.}

\textit{ Доказано существование и единственность положительного решения для одного
класса нелинейных дифференциальных уравнений четвертого порядка. Предложен
также неитерационный численный метод нахождения положительного решения.
}




%\subsection{Предварительные сведения}

Исследованиям положительных решений краевых задач для нелинейных
дифференци\-альных уравнений посвящено много работ российских и зарубежных
математиков (см., например, \cite{krasnosel, pohojOvs, pohojVariaz, gidasSpruck, kuoShung, galahov, gaponenko} и цитированную в них литературу). Во многих из
них рассматриваются вопросы существования положительного решения, его поведения,
асимптотики и другие. Работ, посвященных  единственности положительного решения
и численным методам его построения, мало, особенно в случае сильно нелинейных
уравнений вида $ u=Au $ с выпуклым оператором $ A.$  В предлагаемой работе делается попытка
устранить указанный пробел.

\subsection{Основные результаты}

Рассмотрим семейство двухточечных краевых задач:
$$
{y_{i}}^{(4)}+x^m{\vert y_{i}\vert}^n=0, 0<x<1,   \eqno (1_{i})
$$
$$
y_{i}(0)={y_{i}}^{\prime}(0)={y_{i}}^{\prime\prime}(0)=0,       \eqno (2_{i})
$$
$$
{y_{i}}^{(i)}(1)=0,\ i=0,1,2,3,                              \eqno (3_{i})
$$
где $m\geq 0, n>1$- константы.

Очевидно,$ y_i \equiv 0-$ тривиальное решение задачи $ (1_i)-(3_i)$.
Под положительным решением задачи $ (1_i)-(3_i)$ понимается функция
$y_i \in C^4[0,1]$   положительная при $x\in (0,1)$,удовлетворяющая
уравнению (1) и граничным условиям $(2_i)-(3_i)$.

В работе доказывается существование и единственность положительного решения
задачи $ (1_i)-(3_i)$. Кроме того, предлагается численный метод построения этого
решения. Отметим, что существование положительного решения также можно доказать,
пользуясь методом  расслоения Похожаева С.И.\cite{pohojVariaz}. В качестве примеров приводится
положительное решение (в виде таблиц значений) задачи $(1_0)-(3_0) $ при
$m=0,n=4,$  построенное приведенным здесь методом.

В данной работе продолжаются исследования автора, опубликованные в
работах \cite{abdurag1,abdurag2,abdurag3,abdurag4}.

\centerline {\bf 1. Вспомогательные предложения}


Пусть $ A-$ произвольное положительное число. Рассмотрим задачу Коши:

$$
    y^{(4)}+x^m{\vert y\vert}^n=0,          \eqno (1.1)
$$
$$
   y(0)=y^{\prime}(0)=y^{\prime\prime}(0)=0,       \eqno (1.2)
$$
$$
         y^{\prime\prime\prime}(0)=A.              \eqno (1.3)
$$
Интегрируя уравнение (1.1) с учетом начальных условий (1.2), (1.3), имеем
$$
y^{\prime\prime\prime}(x)=A-\int_0^xs^m{\vert y(s)\vert}^n ds, \eqno (1.4)
$$
$$
y^{\prime\prime}(x)=Ax-\int_0^x(x-s)s^m{\vert y(s)\vert}^n ds, \eqno (1.5)
$$
$$
y^{\prime}(x)=
A\frac{x^2}{2}-\int_0^x\frac{(x-s)^2}{2}s^m{\vert y(s)\vert}^n ds, \eqno(1.6)
$$
$$
y(x)=
A\frac{x^3}{6}-\int_0^x\frac{(x-s)^3}{6}s^m{\vert y(s)\vert}^n ds.\eqno(1.7)
$$

{\bf Лемма 1.1}. {\it При любом $A>0$  существует единственная
точка $x_3>0$ такая, что существует единственное решение $ y \in C^4[0,x_3] $ задачи Коши
(1.1)-(1.3) такое, что $y^{\prime\prime\prime}(x_3)=0,
y^{\prime\prime\prime}(x)>0 $ при $ x \in (0,x_3) $  и
$ y^{\prime\prime\prime}(x)<0 $ при $ x>x_3 $.}

{\bf  Доказательство.} Так как $ y^{\prime\prime\prime}(0)=A>0 $ , то
$ y^{\prime\prime\prime}(x)>0 $ в некоторой окрестности $ (0,\delta) $   нуля.
Из уравнения (1.1) следует, что $  y^{(4)}(x) \leq 0 $.  Следовательно,
$ y^{\prime\prime\prime}(x)- $ невозрастающая функция.

Предположим противное, т.е. не существует точки $ x, $ в которой
$ y^{\prime\prime\prime}(x)=0.$ Тогда $ y^{\prime\prime\prime}(x)>0 $  при всех
$ x>0. $    В силу (1.4) из (1.7) м (1.6) следует
$$
y(x)\geq \frac{x^3}{6}(A-\int_0^x s^m{\vert y(s) \vert}^n ds)=
\frac{x^2}{2}y^{\prime\prime\prime}(x)>0,
$$
$$
y^{\prime}\geq \frac{x^2}{2}(A-\int_0^x s^m{\vert y(s) \vert}^n ds)=
\frac{x^3}{6}y^{\prime\prime\prime}(x)>0.
$$

Это значит $ y(x)-$  положительная всюду на $ (0,\infty),$ возрастающая функция.
Из (1.7) при $ x>0 $    имеем
  $$
Ax^3>\int_0^x(x-s)^3s^m{\vert y(s) \vert}^n ds.             \eqno (1.8)
$$
Пусть $ x_0\geq \delta.$    Поскольку $ y(x)$   возрастающая функция,
 то в силу (1.8) при $ x>x_0 $   имеем

$$
 Ax^3>\int_{x_0}^x(x-s)^3s^m{\vert y(s) \vert}^n ds \geq x_0^my^n(x_0)
 \frac{(x-x_0)^4}{4}.            \eqno (1.9)
$$
Отсюда при $ x=Mx_0(M>1) $  имеем
$$
AM^3>{x_0}^{m+1} \frac{(M-1)^4}{4}y^n(x_0)
$$
или
$$
1>\frac{{x_0}^{m+1}(M-1)^4}{4AM^3}y^n(x_0).       \eqno (1.10)
$$
Так как $ y^n(x_0)>0 $  и $ A>0 $,  то (1.10)  при $ M\to {+\infty} $
приводит к противоречию. Следовательно, существует точка $ x_3 $  такая, что
$ y^{\prime\prime\prime}(x_3)=0.$ Пусть $ \delta- $ произвольное число из $ (0,x_3).$
Из (1.4) имеем
$$
 y^{\prime\prime\prime}(x_3-\delta)=
A-\int_0^{x_3}s^m{\vert y(s) \vert}^n ds+
\int_{x_3-\delta}^{x_3}s^m{\vert y(s) \vert}^n ds=
\int_{x_3-\delta}^{x_3}s^m{\vert y(s) \vert}^n ds>0.
$$
Пусть теперь $\delta$ -- произвольное положительное число. Снова из (1.4) имеем

$$
y^{\prime\prime\prime}(x_3+\delta)=
A-\int_0^{x_3}s^m{\vert y(s) \vert}^n ds-
\int_{x_3}^{x_3+\delta}s^m{\vert y(s) \vert}^n ds=
-\int_{x_3-\delta}^{x_3}s^m{\vert y(s) \vert}^n ds<0.
$$
Из уравнения (1.1) и равенств (1.4)--(1.7) следует ограниченность
$ \left\|y\right\|_{C^4[0,x_3]} $. Следовательно, существует единственное решение
задачи Коши (1.1)-(1.3) на $ [0,x_3].$  Лемма доказана.$ \square $

{\bf Лемма 1.2}.{\it При любом $ A>0 $ существует единственная точка
$ x_2>0 $ такая, что существует единственное решение $ y \in C^4[0,x_2] $ задачи
Коши (1.1)-(1.3) такое, что $ y^{\prime\prime}(x_2)=0,\quad y^{\prime\prime}(x)>0$ 
при $ x \in (0,x_2) $ и $ y^{\prime\prime}(x)<0 $  при $ x>x_2. $ }

{\bf  Доказательство.}   Так как по лемме 1.1 
$ y^{\prime\prime\prime}(x)>0 $    при $ x \in (0,x_3),$
то $ y^{\prime\prime}(x) $ возрастает  при $ x \in (0,x_3).$
Поскольку $ y^{\prime\prime}(0)=0, $  то $ y^{\prime\prime}(x)>0, $
на  $ (0,x_3). $

Предположим противное, т.е.  не существует точки $x,$
в которой $ y^{\prime\prime}(x)=0. $ Тогда $ y^{\prime\prime}(x)>0 $
при всех $ x>0 $ . Из (1.7) и (1.6) в силу (1.5) при $ x>0 $  имеем
$$
y^{\prime}\geq \frac{x}{2}(Ax-\int_0^x (x-s)s^m{\vert y(s) \vert}^n ds)=
\frac{x}{2}y^{\prime\prime}(x)>0,
$$
$$
  y(x)\geq \frac{x^2}{6}(Ax-\int_0^x (x-s)s^m{\vert y(s) \vert}^n ds)=
\frac{x^2}{6}y^{\prime\prime}(x)>0.
$$

Следовательно, $ y(x)>0 $ и возрастает при $ x>0 $.
Пусть $ x_0\geq x_3, $ где  $ x_3 $ определяется леммой 1.1.
Тогда (1.7) в силу (1.8) при $ x>x_0 $ приводит к неравенству (1.9),
далее --- к неравенству (1.10), из которого получаем противоречие. Следовательно,
существует точка $ x_2 $ такая, что $ y^{\prime\prime}(x_2)=0. $
Очевидно, $ x_2\geq x_3. $  Так как $ (y^{\prime\prime)})^{\prime\prime}(x)=y^{(4)}(x)<0 $
при $ x>0, $  то $ y^{\prime\prime}(x)- $ выпуклая вниз функция. Поэтому
точка $ x_2, $  в которой $ y^{\prime\prime}(x_2)=0, $    единственна.
Следовательно,$ y^{\prime\prime}(x)>0 $   при $ x \in (0,x_2) $  и
$ y^{\prime\prime}(x)<0 $  при $ x>x_2. $ Ограниченность $ \left\|y\right\|_{C^4[0,x_2]} $
следует из (1.1) и (1.4)-(1.7). Следовательно,существует единственное решение задачи
Коши (1.1)-(1.3) на $ [0,x_2] $. Лемма доказана.$ \square $

{\bf Лемма 1.3.} {\it При любом $ A>0 $  существует единственная
точка $ x_1 $   такая, что существует единственное решение $ y \in C^4[0,x_1] $ задачи
Коши (1.1)-(1.3) такое, что $ y^{\prime}(x_1)=0,\quad y^{\prime}(x)>0  $    при
$ x \in (0,x_1) $ и $ y^{\prime}(x)<0 $ при $ x>x_1, $ где $ y(x)-$ решение
задачи (1.1)-(1.3).}

{\bf  Доказательство.}    Так как по лемме   1.2 $ y^{\prime\prime}(x)>0 $
при $ x \in (0,x_2), $    то $ y^{\prime}(x) $ возрастает при
$ x \in (0,x_2). $ Поскольку $ y^{\prime}(0)=0, $   то  $ y^{\prime}(x)>0 $
на $  (0,x_2). $

Предположим противное, т.е. не существует точки $ x, $
 в которой  $ y^{\prime}(x)=0. $    Тогда $ y^{\prime}(x)>0 $   при всех
$ x>0. $ Следовательно, $ y(x) $ возрастает при $ x>0 $ .  Поскольку $ y(0)=0, $
то $ y(x)>0 $   при $ x>0. $  Пусть $ x_0 \geq x_2, $   где $ x_2 $
определяется леммой 1.2.Тогда  так же, как в лемме 1.1,  приходим к неравенству
(1.9). Полагая в нем $ x=Mx_0(M>1), $ получим неравенство (1.10), из которого
получаем противоречие.  Следовательно, существует точка $ x_1 $   такая,
что $ y^{\prime}(x_1)=0. $  Очевидно, $ x_1\geq x_2 \geq x_3.$  Так как
$ (y^{\prime}(x))^{\prime\prime}=y^{\prime\prime\prime}(x)<0 $ при $ x>x_3 $  по лемме 1.1,
то $ y^{\prime}(x)- $  выпуклая вниз при $ x>x_3 $   функция. Тем более она
выпукла вниз при $ x>x_1\geq x_2 \geq x_3. $ Поэтому  точка $ x_1 $ , в которой
$ y^{\prime}(x_1)=0, единственна, $ \quad $ y^{\prime}(x)>0 $  при
$ x \in (0,x_1) $ и  $ y^{\prime}(x)<0 $    при $ x>x_1 $  .
Ограниченность $ \left\|y\right\|_{C^4[0,x_1]} $
следует из (1.1) и (1.4)-(1.7). Следовательно,существует единственное решение задачи
Коши (1.1)-(1.3) на $ [0,x_1] $.  Лемма доказана.$ \square $

{ \bf Лемма 1.4.} { \it  При любом $ A>0 $   существует единственная точка
$ x_0>0 $ такая, что существует единственное решение $ y \in C^4[0,x_0] $
задачи Коши (1.1)-(1.3) такое, что  $ y(x_0)=0, \quad y(x)>0 $  при $ x \in (0,x_0). $}

{\bf  Доказательство.}  Так как по лемме 1.3 $ y^{\prime}(x)>0 $
при $ x \in (0,x_1) $ и $ y(0)=0,$ то $ y(x)>0 $ и возрастает при $ x \in (0,x_1).$

Предположим противное, т.е. $ y(x)>0 $  при  всех $ x>0.$
 Так как  по лемме 1.2 $ y^{\prime\prime}(x)<0 $ при $ x>x_2 $ ,
то $ y^{\prime\prime}(x)<0 $  и при  $ x>x_1 \geq x_2 $ , т.е. $ y(x)-$  выпуклая вниз
функция при $ x>x_1. $  Следовательно, существует единственная точка $ x_0 $
такая, что $ y(x_0)=0,\quad y(x)>0 $   при $ x \in (0,x_0) $   и $ y(x)<0 $
при $ x>x_0 $.  Ограниченность $ {\left \| y \right \|}_{C^4[0,x_0]} $
следует из (1.1) и (1.4)-(1.7). Следовательно,существует единственное решение задачи
Коши (1.1)-(1.3) на $ [0,x_0] $.  Лемма доказана.$ \square $
\newpage
\centerline {\bf 2.Cуществование и единственность положительного решения}

Следуя Ц.На \cite{CeNa}, введем линейную группу преобразований
$$
\begin{cases}
x=A_i^\alpha \overline {x},\\
y_i=A_i^\beta \overline{y_i}, \quad i=0,1,2,3,
\end{cases}                                              \eqno(2.1)
$$
где $ \alpha,\beta- $ константы, подлежащие определению,$ A_i- $  положительный
параметр преобразо\-вания.В новых координатах $ (\overline {x},\overline{y_i}) $
уравнение $ (1_i) $ примет вид
$$
A_i^{\beta-4 \alpha} {\overline{y_i}}^{(4)}+
A_i^{\alpha m+\beta n}{\overline {x}}^m {\overline{y_i}}^n=0.               \eqno(2.2)
$$
Выберем константы $ \alpha $ и $ \beta $  так, чтобы это уравнение не зависело
от параметра $ A_i: $
$$
\beta-4 \alpha=\alpha m+\beta n.                  \eqno(2.3)
$$
Тогда из (2.2) имеем
$$
{\overline{y_i}}^{(4)}+{\overline {x}}^m {\overline{y_i}}^n=0, \quad i=0,1,2,3.  \eqno(2.4)
$$
т.е уравнение $ (1_i) $ оказалось инвариантным относительно преобразования (2.1).

Обозначим через $ A_i $  недостающее начальное условие в задаче $ (1_i)-(3_i) $:
$$
y_i^{\prime\prime\prime}(0)=A_i.                    \eqno(2.5)
$$

Это условие в координатах $ (\overline {x},\overline{y_i}) $   запишется в виде
$$
A_i^{\beta-3 \alpha}{\overline{y_i}}^{\prime\prime\prime}(0)=A_i.   \eqno(2.6)
$$

и  оно не будет зависеть от параметра $ A_i, $  если
$$
\beta-3 \alpha=1.                                                \eqno (2.6)
$$
Тогда из (2.6) получим
$$
{\overline y}_i^{\prime\prime\prime}(0)=1.                       \eqno(2.8)
$$
Решая систему (2.3), (2.7), находим
$$
\alpha=-\frac{n-1}{m+3n+1},                                   \eqno(2.9)
$$
$$
 \beta=\frac{m+4}{m+3n+1}.                                     \eqno(2.10)
$$
 В силу (2.4), (2.8) и того, что условия $ (2_i) $ в новых
координатах $ (\overline {x},\overline{y_i}) $ будут иметь вид
$ \overline{y_i}(0)={\overline{y_i}}^{\prime}(0)=
{\overline{y_i}}^{\prime\prime}(0)=0,$
приходим к следующей  задаче Коши для $ \overline{y_i}(\overline {x}): $
$$
 {\overline{y_i}}^{(4)}+{\overline {x}}^m {\overline{y_i}}^n=0,   \eqno(2.11)
$$
$$
\overline{y_i}(0)={\overline{y_i}}^{\prime}(0)=
{\overline{y_i}}^{\prime\prime}(0)=0,                            \eqno(2.12)
$$
$$
{\overline y}_i^{\prime\prime\prime}(0)=1.                         \eqno(2.13)
$$


Из лемм 1.1-1.4 с $ A=1 $ следует, что существует единственная
 точка $ \overline {x_{i0}},i=0,1,2,3, $  такая, что решение
$ \overline{y_i}(\overline {x}) $ задачи Коши (2.11)-(2.13) на
$ [0,\overline {x_{i0}}] $ определяется единственным образом, удовлетворяет условию
$ {\overline{y_i}}^{(i)}(\overline {x_{i0}})=0, i=0,1,2,3, $ и
$ {\overline{y_i}}^{(i)}(\overline {x})>0 $ при $ \overline {x} \in (0,x_{i0}). $
Выберем параметр $ A_i $  в (2.1) так, чтобы $ x=1 $ при
$  \overline {x}=\overline {x_{i0}},i=0,1,2,3, $ т.е. из равенства
$ 1=A_i^{\alpha}\overline {x_{i0}}. $
Отсюда положительный параметр $ A_i $ определяется однозначно:
$$
A_i=(\overline {x_{i0}})^{-\frac{1}{\alpha}}, i=0,1,2,3,        \eqno(2.14)
$$
где $ \alpha $ определяется равенством (2.9). Поэтому задача $ (1_i)-(3_i) $
имеет единственное положи\-тельное решение $ y_i \in C^4[0,1] $. Доказана

{ \bf Теорема 2.1.} { \it Задача $ (1_i)-(3_i) $ имеет единственное
положительное решение $ y_i \in C^4[0,1], \quad i=0,1,2,3. $}

{ \it \underline {Замечание 1.}  Отрезок $ [0,a] $ с  произвольным
положительным $ a $ заменой $ t=\frac{x}{a} $  сводится к отрезку [0,1].
Поэтому сформулированная здесь теорема имеет место  для любого отрезка
$ [0,a] $  с заменой условия $ (3_i) $ на $ {y_i}^{(i)}(a)=0, i=0,1,2,3. $}

\centerline {\bf 3. Численный метод построения положительного решения}

Приведенные выше рассуждения позволяют сформулировать алгоритм построения
единст\-венного положительного решения задачи $ (1_i)-(3_i),$ состоящий из
следующих шагов:

1. Вычисляем $ \alpha $  и $ \beta $   по формулам (2.9), (2.10);

2. Решаем каким-либо численным методом, например, методом Рунге-Кутта
четвертого порядка задачу Коши (2.11)-(2.13), начиная с $ \overline {x}=0 $
до тех пор, пока по одной из лемм 1.1-1.4 не выполнится равенство
$ {\overline{y_i}}^{(i)}(\overline {x_{i0}})=0 $   с
$ \overline {x_{i0}}>0,\quad i=0,1,2,3; $

3. Вычисляем $ A_i $ по формуле (2.14);

4. Находим решение по формулам (2.1).

 { \it \underline {Замечание 2.} Для уменьшения вычислительной погрешности,
связанной с вычислением степени   $ A_i $ пункт 4 можно заменить пунктом

$ 4' $   . Решаем задачу Коши $ (1_i), (2_i),  (2.5) $ тем же численным методом,
что и в пункте 2, начиная с $ x=0 $  до $ x=1. $}

       В качестве примера приведем таблицу значений положительного решения задачи
$ (1_0)-(3_0) $ при $ m=0,n=4,$ полученного указанным здесь методом.
\vspace {0.5 cm}


\centerline {\bf Положительное решение задачи
$ (1_0)-(3_0) $ при $ m=0,n=4 $}
\quad \begin{tabular}[c]{|l|l|l|l|l|l|l|l|l|l|l|l|}
\hline
x& 0,00 &0,1  & 0,2 & 0,3 & 0,4 & 0,5 & 0,6 & 0,7 & 0,8 & 0,9 & 1,0 \\
\hline
y& 0,00 & 0,04 & 0,31 & 1,05 & 2,49 & 4,86 &  8,39  & 13,18
& 18,54 & 19,65 & 0,00 \\
\hline
\end{tabular}
\bigskip



\centerline { \bf Заключение }

Доказано, что каждая задача из семейства двухточечных краевых задач
$$
{y_{i}}^{(4)}+x^m{\vert y_{i}\vert}^n=0, 0<x<1,
 $$
$$
y_{i}(0)={y_{i}}^{\prime}(0)={y_{i}}^{\prime\prime}(0)=0,
$$
$$
{y_{i}}^{(i)}(1)=0,\ i=0,1,2,3,
$$
где $m\geq 0, n>1$- константы, имеет единственное положительное решение и предложен
неитерационный численный метод его построения. Для данного класса уравнений результат
о единственности является новым. Также нов, предложенный здесь неитерационный
численный метод построения положительного решения.
