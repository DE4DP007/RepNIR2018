


\chapter{Перекрывающие преобразования для приближения непрерывных и кусочно-гладких функций посредством повторных средних Валле Пуссена}





%\section{ Приближение функций повторными средними Валле Пуссена}\label{s1}

Пусть $f$ -- $2\pi$-периодическая функция, интегрируемая на периоде,
\begin{equation*}
    a_k=a_k(f)=\frac1\pi\int_{-\pi}^\pi f(t)\cos ktdt,\quad b_k=b_k(f)=\frac1\pi\int_{-\pi}^\pi f(t)\sin ktdt
\end{equation*}
-- коэффициенты Фурье,
\begin{equation}\label{1.3}
    f(x) \sim \frac{a_0}{2}+ \sum_{k=1}^\infty a_k\cos kx+b_k\sin kx
\end{equation}
-- ряд Фурье функции $f$. Далее, пусть $A_k(f)=A_k(f,x)=a_k\cos kx+b_k\sin kx$,

\begin{equation} \label{1.4}
 S_n(f)=   S_n(f,x)=\frac{a_0}{2}+ \sum_{k=1}^n A_k(f,x)
\end{equation}
-- сумма Фурье,
\begin{equation}\label{1.5}
 {}_1 V_{n,m}(f)= {}_1 V_{n,m}(f,x)=\frac1{n}[S_m(f,x)+\cdots+S_{m+n-1}(f,x)]
\end{equation}
-- средние Валле Пуссена. Повторные средние Валле Пуссена (средние Валле Пуссена второго порядка) определим с помощью равенства
\begin{equation}\label{1.6}
 _2V_{n,m}(f)= _2V_{n,m}(f,x)=\frac1{n}[ _1V_{n,m}(f,x)+\cdots+ _1V_{n,m+n-1}(f,x)].
\end{equation}
Непосредственно из равенств \eqref{1.5} и \eqref{1.6} следует, что если $T_m=T_m(x)$ -- произвольный тригонометрический полином порядка $m$, то
\begin{equation}\label{1.7}
 _1V_{n,m}(T_m)=T_m, \quad _2V_{n,m}(T_m)=T_m.
\end{equation}
Другими словами, оператор $_2V_{n,m}=$$_2V_{n,m}(f)$ также, как и оператор $_1V_{n,m}=$ $_1V_{n,m}(f)$, является проектором на  пространство тригонометрических полиномов $T_m$ порядка $m$. Отметим также, что в силу \eqref{1.4} -- \eqref{1.6} $ _2V_{n,m}(f)$ допускает следующее представление
$$
 _2V_{n,m}(f,x)=\frac{a_0}{2}+ \sum_{k=1}^m A_k(f,x)+
$$
$$
 \sum_{k=1}^{n-1}{2n^2-k(k+1)\over2n^2}A_{m+k}(f,x)+
$$
$$
 \sum_{j=0}^{n-2}{(n-j)(n-j-1)\over2n^2}A_{m+n+j}(f,x)
$$
или, что то же,
$$
 _2V_{n,m}(f,x)=\frac{a_0}{2}+ \sum_{k=1}^m A_k(f,x)+
$$
$$
 \sum_{k=m+1}^{m+n-1}{2n^2-(k-m)(k-m+1)\over2n^2}A_{k}(f,x)+
$$
\begin{equation}\label{1.8}
 \sum_{k=m+n}^{m+2n-2}{(m+2n-k)(m+2n-k-1)\over2n^2}A_{k}(f,x).
\end{equation}
Из \eqref{1.5} и \eqref{1.6} вытекают следующие интегральные представления для операторов
$_1V_{n,m}(f)$ и $_2V_{n,m}(f)$:
\begin{equation}\label{1.9}
 _1V_{n,m}(f,x)=\frac{1}{\pi}\int_{-\pi}^\pi f(x-t)_1v_{n,m}(t)dt,
 \end{equation}

\begin{equation}\label{1.10}
 _2V_{n,m}(f,x)=\frac{1}{\pi}\int_{-\pi}^\pi f(x-t)_2v_{n,m}(t)dt,
 \end{equation}
где
\begin{equation}\label{1.11}
 _1v_{n,m}(u)=\frac{1}{n}[D_m(u)+\ldots+D_{m+n-1}(u)],
 \end{equation}
\begin{equation}\label{1.12}
 _2v_{n,m}(u)=\frac{1}{n}[_1v_{n,m}(u)+\ldots+_1v_{n,m+n-1}(u)],
 \end{equation}
 а
\begin{equation}\label{1.13}
D_k(u) =\frac{1}{2}+\cos u +\ldots+\cos ku =\frac{\sin(k+\frac12)u}{2\sin\frac{u}{2}}
 \end{equation}
-- ядро Дирихле. Из \eqref{1.11} и \eqref{1.13} легко выводится следующие хорошо известные (см., например, \cite{LapVPZhuk}) равенства
\begin{equation}\label{1.14}
_1v_{n,m}(u) =\frac{\cos mu-\cos (m+n)u}{4n\sin^2\frac{u}{2}}=
\frac{\sin\frac{nu}{2}\sin(2m+n)\frac{u}{2}}{2n\sin^2\frac{u}{2}}.
 \end{equation}
Из \eqref{1.12} и \eqref{1.14} имеем
\begin{equation*}
 _2v_{n,m}(u)=\frac{1}{4n^2\sin^2\frac{u}{2}}\left(\sum_{k=m}^{m+n-1}\cos ku-
 \sum_{k=m}^{m+n-1}\cos(k+n)u\right)=\frac{1}{8n^2\sin^3\frac{u}{2}}\times
 \end{equation*}
\begin{equation*}
 \left(\sin(m+n-\frac12)u-\sin(m-\frac12)u
 -\sin(m+2n-\frac12)u+\sin(m+n-\frac12)u\right)=
 \end{equation*}
\begin{equation*}
 \frac{\sin\frac{nu}{2}}{4n\sin^3\frac{u}{2}}
 \left(\cos(2m+n-1)\frac{u}{2}-\cos(2m+3n-1)\frac{u}{2}\right),
  \end{equation*}
поэтому
\begin{equation}\label{1.15}
 _2v_{n,m}(u)=\frac{\sin^2\frac{nu}{2}\sin(m+n-\frac12)u}{2n^2\sin^3\frac{u}{2}}.
  \end{equation}
Интегральные представления \eqref{1.9} и \eqref{1.10} с учётом  \eqref{1.14} и \eqref{1.15}  принимают следующий вид
\begin{equation}\label{1.16}
 _1V_{n,m}(f,x)=\frac{1}{2\pi n}\int_{-\pi}^\pi f(x-t) \frac{\sin\frac{nt}{2}\sin(2m+n)\frac{t}{2}}{\sin^2\frac{t}{2}}dt,
 \end{equation}
\begin{equation}\label{1.17}
 _2V_{n,m}(f,x)=\frac{1}{2\pi n^2}\int_{-\pi}^\pi f(x-t) \frac{\sin^2\frac{nt}{2}\sin(2m+2n-1)\frac{t}{2}}{\sin^3\frac{t}{2}}dt.
 \end{equation}
Равенство \eqref{1.16} хорошо известно и широко использовалось в \cite{LapVPZhuk} при исследовании аппроксимативных свойств операторов $_1V_{n,m}(f)$ в различных функциональных пространствах.  Что же касается представления \eqref{1.17}, то оно, как и сами операторы  $ _2V_{n,m}(f)$, насколько известно автору, является новым.

 Отметим, что повторные средние Валле Пуссена $_kV_{n,m}(f,x)$, введенные выше для $k=2$, допускают дальнейшее обобщение на $k\ge2$    методом индукции следующим образом
\begin{equation}\label{1.18}
  {}_{k}V_{n,m}(f,x)=\frac1{n}[ _{k-1}V_{n,m}(f,x)+\cdots+ _{k-1}V_{n,m+n-1}(f,x)],
\end{equation}
например, при $k=3$ мы можем записать
\begin{equation}\label{1.19}
  _3V_{n,m}(f,x)=\frac1{n}[ _2V_{n,m}(f,x)+\cdots+ _2V_{n,m+n-1}(f,x)].
\end{equation}
Нетрудно также получить для $_kV_{n,m}(f,x)$ интегральное представление, аналогичное \eqref{1.10} или \eqref{1.17}. К примеру, для $k=3$ из \eqref{1.17} и \eqref{1.19} имеем
$$
 _3V_{n,m}(f,x)=\frac{1}{2\pi n^3}\int_{-\pi}^\pi f(x-t) \frac{\sin^2\frac{nt}{2}}{\sin^3\frac{t}{2}}\sum_{l=m}^{m+n-1}\sin(l+n-\frac12)tdt
$$
  \begin{equation}\label{1.20}
= \frac{1}{2\pi n^3}\int_{-\pi}^\pi f(x-t) \frac{\sin^3\frac{nt}{2}\sin(2m+3n-2)\frac{t}{2}}{\sin^4\frac{t}{2}}dt,
  \end{equation}
и, вообще, если $k\ge2$, то
 \begin{equation}\label{1.21}
_kV_{n,m}(f,x)= \frac{1}{2\pi n^k}\int\limits_{-\pi}^\pi f(x-t) \frac{\sin^k\frac{nt}{2}\sin(2m+k(n-1)+1)\frac{t}{2}}{\sin^{k+1}\frac{t}{2}}dt.
  \end{equation}






Обозначим через $C_{2\pi}$ пространство непрерывных $2\pi$-периодических функций $f$  с нормой $f=\max_x|f(x)|$ и рассмотрим ${}_1 V_{n,m}={}_1 V_{n,m}(f)$ и ${}_2 V_{n,m}={}_2 V_{n,m}(f)$
как операторы, действующие в нормированном пространстве $C_{2\pi}$. Из интегральных представлений \eqref{1.16} и \eqref{1.17} непосредственно следует, что нормы этих операторов равны:
\begin{equation}\label{1.22}
\|_1V_{n,m}\|=\frac{1}{2\pi n}\int_{-\pi}^\pi \frac{|\sin\frac{nt}{2}\sin(2m+n)\frac{t}{2}|}{\sin^2\frac{t}{2}}dt,
 \end{equation}
\begin{equation}\label{1.23}
\|_2V_{n,m}\|=\frac{1}{2\pi n^2}\int_{-\pi}^\pi  \frac{\sin^2\frac{nt}{2}|\sin(m+n-\frac12)t|}{|\sin^3\frac{t}{2}|}dt.
 \end{equation}
Из \eqref{1.7}, \eqref{1.16}  и \eqref{1.17} непосредственно вытекают следующие неравенства
\begin{equation}\label{1.24}
\|f-_1V_{n,m}(f)\|\le E_m(f)(1+\|_1V_{n,m}\|),
 \end{equation}
\begin{equation}\label{1.25}
\|f-_2V_{n,m}(f)\|\le E_m(f)(1+\|_2V_{n,m}\|),
 \end{equation}
где $E_m(f)$ -- наилучшее приближение функции $f\in C_{2\pi}$ тригонометрическими полиномами $T_m$ порядка $m$. Величина $\|_1V_{n,m}\|$ достаточно хорошо изучена в  \cite{LapVPZhuk,LapVPNIK}. Например, в \cite{LapVPZhuk} установлена следующая оценка
\begin{equation}\label{1.26}
\|_1V_{n,m}\|\le \frac{4}{\pi^2}\ln\left(1+\frac{m+n}{n}\right)+1,7.
 \end{equation}
С другой стороны, из \eqref{1.10} и \eqref{1.12} следует, что
\begin{equation}\label{1.27}
\|_2V_{n,m}\|\le \frac1n\sum_{k=m}^{m+n-1}\|_1V_{n,k}\|.
 \end{equation}
Из \eqref{1.26} и \eqref{1.27} выводим
\begin{equation}\label{1.28}
\|_2V_{n,m}\|\le \frac{4}{\pi^2}\ln\left(3+\frac{m-1}{n}\right)+1,7.
 \end{equation}
Оценка \eqref{1.28}, скорее всего, не является окончательной, но этот вопрос в текущем году не рассматривался. Мы сосредоточим внимание на исследовании локальных аппроксимативных свойств повторных средних Валле Пуссена $_2V_{n,m}(f,x)$ для кусочно-гладких функций, которые, как показано в п. \ref{s3}
настоящего отчета, существенно отличаются от соответствующих свойств  сумм Фурье $S_k(f,x)$ и классических средних Валле Пуссена $_1V_{n,m}(f,x)$.
%Если, к примеру, мы рассмотрим кусочно постоянную функцию $f(x)=sign\sin x$, то из результатов, полученных в п.\ref{s3} вытекает оценка
%\begin{equation}\label{1.29}
%|f(x)-_2V_{n,n}(f,x)|\le \frac{c(\varepsilon)}{n^3} \quad(|x-k\pi|>\varepsilon, k\in \mathbb{Z}),
% \end{equation}
% тогда как для сумм Фурье $S_n(f,x)$  и классических средних Валле Пуссена $_1V_{n,n}(f,x)$
%имеют место соотношения
%\begin{equation}\label{1.30}
%\max_{x\atop|x-k\pi|>\varepsilon, k\in \mathbb{Z} }|f(x)-S_n(f,x)|\asymp n^{-1},
% \end{equation}
%\begin{equation}\label{1.31}
%\max_{x\atop|x-k\pi|>\varepsilon, k\in \mathbb{Z} }|f(x)- _1V_{n,n}(f,x)|\asymp n^{-2}.
% \end{equation}
% В оценке \eqref{1.29} и всюду в дальнейшем через $c$, $c(a), c(a,b), c_k(a,b) \ldots$ обозначаются положительные постоянные, зависящие лишь от указанных параметров, вообще говоря различные в  разных местах.
%В п.\ref{s3} показано, что совершенно аналогичная ситуация имеет место для любой кусочно-гладкой $2\pi$-периодической функции $f$ с конечным числом точек разрыва первого рода на периоде.
Это свойство делает $_2V_{n,n}(f,x)$ весьма привлекательным инструментом решения важных прикладных задач таких, например, как конструирование цифровых фильтров, обработка и сжатие речи и т.д.

\section{Некоторые вспомогательные результаты}\label{s2}
Через $W_1^r(a,b)$ обозначим пространство Соболева, состоящее из функций $f$, $r-1$ раз непрерывно дифференцируемых на $[a,b]$, для которых $f^{(r-1)}$ абсолютно непрерывна, а $f^{(r)}$ интегрируема на $[a,b]$. В дальнейшем нам понадобятся некоторые классы кусочно-гладких \linebreak $2\pi$-периодических функций, которых мы определим в данном разделе. Пусть $0\le r$ -- целое, $-\pi=x_0<x_1<\cdots<x_s<x_{s+1}=\pi$, $f(x)$ -- непрерывная $2\pi$-периодическая функция такая, что $f\in W_1^{r+1}(x_j,x_{j+1})$ для каждого $j=0,\ldots,s$. Множество всех таких функций мы обозначим  через $\mathcal{ I}^r_\Omega$, где $\Omega=\{x_0,x_1,\ldots x_s,x_{s+1}\}$.
 Характерным элементом пространства $\mathcal{ I}^r_\Omega$   с $\Omega=\{-\pi,0,\pi\}$ является, например, функция   $f(x)$, равная $|x|\,\text{при}\, |x|\le \pi$ и продолженная $2\pi$-периодически.
Нетрудно увидеть, что если $f\in  \mathcal{ I}^r_\Omega$, то она абсолютно непрерывна на $[-\pi,\pi]$, но при этом вполне может быть так, что для  $x\in\Omega$ производная $f'(x)$  не существует. Если функция $f\in \mathcal{ I}^r_\Omega$, то будем называть её кусочно-гладкой (порядка $r+1$).

Изучение локальных аппроксимативных свойств повторных средних Валле Пуссена $_2V_{n,n}(f,x)$ для функций $f\in \mathcal{ I}^r_\Omega$ является основной задачей данного исследования.
При доказательстве основного результата, полученного нами в отчетном году, используется ряд вспомогательных утверждений, которые мы сформулируем в данном пункте.

Если $f\in \mathcal{ I}^r_\Omega$, то ее ряд Фурье сходится к ней равномерно по $x\in\mathbb{R}$  и допускает представление
\begin{equation}\label{2.1}
    f(x) = \frac{a_0}{2}+ \sum_{k=1}^\infty a_k\cos kx+b_k\sin kx.
\end{equation}
 Положим
\begin{equation}\label{2.2}
R_m(f,x)=f(x)-S_m(f,x)=\sum_{k=m+1}^\infty a_k\cos kx+b_k\sin kx,
\end{equation}
и заметим, что для коэффициентов  Фурье при $k>0$ можно записать следующие равенства
\begin{equation}\label{2.3}
a_k(f)=-\frac1{k \pi}\int_{-\pi}^\pi f'(t)\sin kt=-\frac1kb_k(f'),
   \end{equation}
\begin{equation}\label{2.4}
b_k(f)=\frac1{k \pi}\int_{-\pi}^\pi f'(t)\cos kt=\frac1ka_k(f'),
   \end{equation}
Из \eqref{2.2} -- \eqref{2.4} имеем
\begin{equation}\label{2.5}
R_m(f,x)=f(x)-S_m(f,x)=\sum_{k=m+1}^\infty\frac{1}{k}(a_k(f')\sin kx-b_k(f')\cos kx).
\end{equation}
Далее запишем
   \begin{equation}\label{2.6}
b_k(f')=\frac1\pi\int_{-\pi}^\pi f'(t)\sin kt=\sum_{j=0}^s b_k^j(f'),
   \end{equation}
\begin{equation}\label{2.7}
a_k(f')=\frac1\pi\int_{-\pi}^\pi f'(t)\cos kt=\sum_{j=0}^s a_k^j(f'),
   \end{equation}
 где
\begin{equation}\label{2.8}
a_k^j(f')=\frac1\pi\int_{x_j}^{x_{j+1}} f'(t)\cos kt,
   \end{equation}
\begin{equation}\label{2.9}
b_k^j(f)=\frac1\pi\int_{x_j}^{x_{j+1}} f'(t)\sin kt,
   \end{equation}
 тогда из \eqref{2.5} -- \eqref{2.9} имеем
\begin{equation}\label{2.10}
R_m(f,x)=\sum_{j=0}^s\sum_{k=m+1}^\infty\frac1k(a_k^j(f')\sin kx-b_k^j(f')\cos kx).
   \end{equation}
Если $f\in \mathcal{ I}^r_\Omega$, то, $r$-кратно применяя метод интегрирования по частям, имеем
$$
 a_k^j(f')\sin kx-b_k^j(f')\cos kx=-\frac1\pi\int_{x_j}^{x_{j+1}}f'(t)\sin k(t-x)dt=
$$
$$
\frac1\pi\int_{x_j}^{x_{j+1}}f'(t)\cos[k(t-x)+\frac\pi2]dt=
$$
$$
\frac{1}{\pi k}(f'(x_{j+1}-0)\sin[k(x-x_{j+1})+\frac\pi2]-f'(x_j+0)\sin[k(x-x_j)+\frac\pi2])
$$
$$
-\frac1{\pi k}\int\limits_{x_j}^{x_{j+1}}f''(t)\sin[k(t-x)+\frac\pi2]dt=
$$
$$
\frac{1}{\pi k}(f'(x_j+0)\cos[k(x-x_j)+\frac{2\pi}2]-f'(x_{j+1}-0)\cos[k(x-x_{j+1})+\frac{2\pi}2])
$$
$$
+\frac1{\pi k}\int\limits_{x_j}^{x_{j+1}}f''(t)\cos[k(t-x)+\frac{2\pi}2]dt=
$$
$$
\ldots
$$
$$
=\frac{1}{\pi}
\sum_{\nu=1}^{r}\frac{1}{k^\nu}f^{(\nu)}(x_j+0)\cos\left(k(x-x_j)+\frac{(\nu+1)\pi}{2}\right)
$$
$$
-\frac{1}{\pi}
\sum_{\nu=1}^{r}\frac{1}{k^\nu} f^{(\nu)}(x_{j+1}-0)\cos\left(k(x-x_{j+1})+\frac{(\nu+1)\pi}{2}\right)
$$
\begin{equation}\label{2.11}
+\frac1{\pi k^{r}}\int\limits_{x_j}^{x_{j+1}}f^{(r+1)}(t)\cos\left(k(t-x)+\frac{\pi (r+1)}{2}\right)dt.
   \end{equation}
Из \eqref{2.10} и \eqref{2.11} мы заключаем, что справедлива
\begin{lemma}\label{l2.1}
Если $f\in \mathcal{ I}^r_\Omega$, то имеет место равенство
\begin{equation}\label{2.12}
R_l(f,x)= \hat R_l(f,x)+\tilde R_l(f,x),
   \end{equation}
в котором
$$
 \hat R_l(f,x)=\frac{1}{\pi}\sum_{j=0}^s
\sum_{\nu=1}^rf^{(\nu)}(x_j+0)\sum_{k=l+1}^\infty
{\cos\left(k(x-x_j)-\frac{(\nu+1)\pi}{2}\right)\over k^{\nu+1}}
$$
\begin{equation}\label{2.13}
-\frac{1}{\pi}\sum_{j=0}^s
\sum_{\nu=1}^r f^{(\nu)}(x_{j+1}-0)\sum_{k=l+1}^\infty
{\cos\left(k(x-x_{j+1})-\frac{(\nu+1)\pi}{2}\right)\over k^{\nu+1}},
\end{equation}
\begin{equation}\label{2.14}
\tilde R_l(f,x)=\frac1{\pi}\int_{-\pi}^{\pi}f^{(r+1)}(t)\sum_{k=l+1}^\infty
{\cos\left(k(t-x)+\frac{\pi (r+1)}{2}\right)\over k^{r+1}}dt.
\end{equation}
\end{lemma}
Положим
\begin{equation}\label{2.15}
_1R_{n,m}(f,x)=f(x)-_1V_{n,m}(f,x),
\end{equation}
\begin{equation}\label{2.16}
_2R_{n,m}(f,x)=f(x)-_2V_{n,m}(f,x)
\end{equation}
и заметим, что из \eqref{1.3} -- \eqref{1.6} и \eqref{2.2}, \eqref{1.15} и \eqref{2.16} вытекают следующие равенства
$$
_1R_{n,m}(f,x)=\frac1n\sum_{k=m}^{m+n-1}R_k(f,x),
$$
$$
_2R_{n,m}(f,x)=\frac1n\sum_{k=m}^{m+n-1}
{_1}R_{n,k}(f,x)=\frac1{n^2}\sum_{k=m}^{m+n-1}
\sum_{l=k}^{k+n-1}R_l(f,x).
$$
Если теперь обратимся к лемме \ref{l2.1}, то эти равенства можно переписать так
\begin{equation}\label{2.17}
_1R_{n,m}(f,x)=_1\hat R_{n,m}(f,x)+_1\tilde R_{n,m}(f,x),
\end{equation}
\begin{equation}\label{2.18}
_2R_{n,m}(f,x)=_2\hat R_{n,m}(f,x)+_2\tilde R_{n,m}(f,x),
\end{equation}
где
\begin{equation}\label{2.19}
_1\hat R_{n,m}(f,x)=\frac1{n}
\sum_{l=m}^{m+n-1}\hat R_l(f,x),\quad_1\tilde R_{n,m}(f,x)=\frac1{n}
\sum_{l=m}^{m+n-1}\tilde R_l(f,x),
\end{equation}
\begin{equation}\label{2.20}
_2\hat R_{n,m}(f,x)=\frac1{n^2}\sum_{k=m}^{m+n-1}
\sum_{l=k}^{k+n-1}\hat R_l(f,x),
\end{equation}
\begin{equation}\label{2.21}
_2\tilde R_{n,m}(f,x)=\frac1{n^2}\sum_{k=m}^{m+n-1}
\sum_{l=k}^{k+n-1}\tilde R_l(f,x),
\end{equation}
 а величины $\hat R_l(f,x)$  и $\tilde R_l(f,x)$ определены равенствами \eqref{2.13} и \eqref{2.14}.
%В нижеследующих леммах получены оценки этих двух величин, которые нам понадобятся в дальнейшем.
\begin{lemma}\label{l2.2}
Пусть $f\in \mathcal{ I}^r_\Omega$. Тогда имеет место оценка
\begin{equation}\label{2.22}
|\tilde R_l(f,x)|\le \frac{c(r)}{l^{r}}\int_{-\pi}^{\pi}|f^{(r+1)}(t)|dt.
\end{equation}
\end{lemma}
%\begin{proof}
%Обратимся к равенству \eqref{2.21}, из которого находим
%$$
%|\tilde R_l(f,x)|=\left|\frac1{\pi}\int_{-\pi}^{\pi}f^{(r+1)}(t)\sum_{k=l+1}^\infty
%{\cos\left(k(t-x)+\frac{\pi (r+1)}{2}\right)\over k^{r+1}}dt\right|\le.
%$$
%$$
%\frac1{\pi}\int_{-\pi}^{\pi}|f^{(r+1)}(t)|dt\sum_{k=l+1}^\infty
%{1\over k^{r+1}}\le \frac{c(r)}{l^{r}}\int_{-\pi}^{\pi}|f^{(r+1)}(t)|dt.
%$$
%Лемма \ref{l2.2} доказана.
%\end{proof}

\begin{lemma}\label{l2.3}
Пусть $r\ge3$, $f\in \mathcal{ I}^r_\Omega$. Тогда имеет место оценка
\begin{equation*}
|_2\tilde R_l(f,x)|\le \frac{c(r)I_r(f)}{(m+n)^2m^{r-2}}, \quad |_1\tilde R_l(f,x)|\le \frac{c(r)I_r(f)}{(m+n)m^{r-1}}
\end{equation*}
где $I_r(f)=\int_{-\pi}^\pi|f^{(r+1)}(t)|dt$.
\end{lemma}
%\begin{proof}
%Обратимся к равенству \eqref{2.21}. Тогда в силу леммы \ref{l2.2} имеем
%$$
%|_2\tilde R_l(f,x)|\le|\frac1{n^2}\sum_{k=m}^{m+n-1}\sum_{l=k}^{k+n-1}|\tilde R_l(f,x)|\le
%$$
%$$
%\frac1{n^2}\sum_{k=m}^{m+n-1}\sum_{l=k}^{k+n-1}\frac{c(r)I_r(f)}{l^{r}}
%\le\frac{c(r)I_r(f)}{(m+n)^2m^{r-2}}.
%$$
%Аналогично,  используя второе из равенств \eqref{2.19},  выводим оценку для \linebreak $|_1\tilde R_l(f,x)|$. Лемма \ref{l2.3} доказана.
%\end{proof}

Прежде, чем перейти к вопросу об оценках для величин $|\hat R_l(f,x)|$, $|_1\hat R_l(f,x)|$ и $|_2\hat R_l(f,x)|$, мы предварительно докажем некоторые вспомогательные утверждения. Положим
\begin{equation}\label{2.23}
\mathcal{ K}_l^\nu(u)= \sum_{z=l+1}^{\infty}{\cos(zu+\frac{\nu\pi}{2})\over z^\nu},
\end{equation}
\begin{equation}\label{2.24}
_1\mathcal{ K}_{n,m}^\nu(u)=\frac1n \sum_{l=m}^{m+n-1}\mathcal{ K}_l^\nu(u),
\end{equation}
\begin{equation}\label{2.25}
_2\mathcal{ K}_{n,m}^\nu(u)=\frac1n \sum_{k=m}^{m+n-1} {_1}\mathcal{ K}_{n,k}^\nu(u)=
\frac1{n^2} \sum_{k=m}^{m+n-1}\sum_{l=k}^{k+n-1}\mathcal{ K}_l^\nu(u).
\end{equation}


\begin{lemma}\label{l2.4}
Имеют место следующие равенства
 $$
 _1\mathcal{ K}_{n,k}^{2\mu}(u)=
 $$
 $$
 {(-1)^\mu\over n}\sum\limits_{\lambda=1}^{n-1}
\sum\limits_{\kappa=1}^{\infty}\frac{\sin\frac{\lambda}{2}u\sin\frac{\kappa}{2}u
\cos(k+\frac{\kappa+\lambda}{2})u}{\sin^2\frac u2}\Delta^2g_\mu(k+\kappa+\lambda-1)
$$
 \begin{equation}\label{2.26}
    +{(-1)^{\mu-1}\over n}\sum\limits_{\kappa=1}^{\infty}
\frac{\sin\frac {n}{2}u\sin\frac{\kappa}{2}u\cos(k+\frac{\kappa+n}{2})u}{\sin^2\frac u2}
\Delta g_\mu(k+n+\kappa-1),
 \end{equation}
 $$
 _1\mathcal{ K}_{n,k}^{2\mu-1}(u)=
 $$
 $$
 {(-1)^\mu\over n}\sum\limits_{\lambda=1}^{n-1}
\sum\limits_{\kappa=1}^{\infty}\frac{\sin\frac{\lambda}{2}u\sin\frac{\kappa}{2}u
\sin(k+\frac{\kappa+\lambda}{2})u}{\sin^2\frac u2}\Delta^2q_\mu(k+\kappa+\lambda-1)
$$
 \begin{equation}\label{2.27}
    +{(-1)^{\mu-1}\over n}\sum\limits_{\kappa=1}^{\infty}
\frac{\sin\frac {n}{2}u\sin\frac{\kappa}{2}u\sin(k+\frac{\kappa+n}{2})u}{\sin^2\frac u2}
\Delta q_\mu(k+n+\kappa-1),
 \end{equation}
 где $g_\mu(t)=t^{-2\mu}$, $q_\mu(t)=t^{-2\mu+1}$, $\Delta\varphi(t)=\varphi(t+1)-\varphi(t)$,
 $\Delta^2\varphi(t)=\varphi(t+2)-2\varphi(t+1)+\varphi(t)$.
\end{lemma}
% \begin{proof}
%Из \eqref{2.23} и  \eqref{2.24} имеем
%$$
%_1\mathcal{ K}_{n,k}^{\nu}(u)={1\over n}\sum\limits_{\lambda=0}^{n-1}\sum\limits_{\kappa=0}^\infty
%\frac{\cos\left[(k+\kappa+\lambda+1)u+\frac{\pi \nu}{2}\right]}{(k+\kappa+\lambda+1)^\nu},
%$$
%поэтому, с помощью преобразования Абеля мы можем записать
%\begin{equation}\label{2.28}
%    _1\mathcal{ K}_{n,k}^{\nu}(u)={1\over n}\sum\limits_{\lambda=0}^{n-1}
%\sum\limits_{\kappa=0}^\infty\left[\frac{1}{(k+\kappa+\lambda+1)^\nu}-\frac{1}
%{(k+\kappa+\lambda+2)^\nu}\right]v^k_{\kappa,\lambda}(u),
%\end{equation}
%где
%\begin{equation}\label{2.29}
%    v^k_{\kappa,\lambda}(u)=\sum\limits_{j=0}^{\kappa}\cos[(k+j+\lambda+1)u
%    +\frac{\pi \nu}{2}].
%\end{equation}
%Мы рассмотрим два случая, когда $\nu$ является четным или нечетным. Если
%$\nu=2\mu$, то $\cos(p u+\frac{\pi \nu}{2})=(-1)^\mu\cos p u$, следовательно, \eqref{2.29} принимает вид
%$$
%v^k_{\kappa,\lambda}(u)=(-1)^\mu\sum\limits_{j=0}^\kappa\cos(k+1+\lambda+j)u=
%$$
%\begin{equation}\label{2.30}
%    =(-1)^\mu\frac{\sin(2(k+1+\lambda+\kappa)+1)\frac u2-\sin(2(k+\lambda)+1)\frac u2}{2\sin\frac u2}
%\end{equation}
%Из \eqref{2.28} и \eqref{2.30} имеем
%$$
%_1\mathcal{ K}_{n,k}^{2\mu}(u)={(-1)^{\mu-1}\over n}\times
%$$
%\begin{equation}\label{2.31}
%    \sum\limits_{\kappa=0}^\infty\sum\limits_{\lambda=0}^{n-1} \Delta
%g_\mu(k+1+\kappa+\lambda)\frac{\sin(2(k+1+\lambda+\kappa)+1)\frac u2- \sin(2(k+\lambda)+1)\frac u2}{2\sin\frac u2}.
%\end{equation}
%Применяя к внутренней сумме из \eqref{2.31} преобразование Абеля, мы получим
%$$
%_1\mathcal{ K}_{n,k}^{2\mu}(u)={(-1)^{\mu}\over n}\sum\limits_{\kappa=0}^\infty
%\sum\limits_{\lambda=0}^{n-2}\Delta^2 g_\mu(k+1+\kappa+\lambda)\frac{W^k_{\kappa,\lambda}(u)-W^k_{-1,\lambda}(u)}{2\sin\frac u2}+
%$$
%\begin{equation}\label{2.32}
%+{(-1)^{\mu-1}\over n}\sum\limits_{\kappa=0}^\infty\Delta g_\mu(n+k+\kappa)
%\frac{W^k_{\kappa,n-1}(u)-W^k_{-1,n-1}(u)}{2\sin\frac u2},
%\end{equation}
%где
%$$
%W^k_{\kappa,\lambda}(u)=\sum\limits_{\eta=0}^\lambda\sin(2(k+\kappa+1+\eta)+1)\frac u2=
%$$
%$$
%\frac{\sin^2(k+\lambda+\kappa+2)\frac u2-\sin^2(k+\kappa+1)\frac u2}{\sin\frac u2}=
%$$
%$$
%\frac{1}{\sin\frac u2}(\sin(k+\lambda+\kappa+2)\frac u2-\sin(k+\kappa+1)\frac u2)\times
%$$
%$$
%(\sin(k+\lambda+\kappa+2)\frac u2+\sin(k+\kappa+1)\frac u2)=
%$$
%$$\frac{4}{\sin\frac u2}\sin\frac{\lambda+1}{4}u\cdot\cos(k+\kappa+1+\frac{\lambda+1}{2})\frac u2\times
%$$
%\begin{equation*}
%    \sin(k+\kappa+1+\frac{\lambda+1}{2})\frac u2\cos\frac{\lambda+1}{4}u
%\end{equation*}
%и отсюда
%$$
%W_{\kappa,\lambda}^k(u)-W_{-1,\lambda}^k(u)=\frac{4}{\sin\frac u2}\sin\frac{\lambda+1}{4}u\cos\frac{\lambda+1}{4}u\times
%$$
%$$
%\left(\sin(k+\kappa+1+\frac{\lambda+1}{2})\frac u2\cos(k+\kappa+1+\frac{\lambda+1}{2})\frac u2-\right.
%$$
%$$
%\left.-\sin(k+\frac{\lambda+1}{2})\frac u2\cos(k+\frac{\lambda+1}{2})\frac u2\right)=
%$$
%$$
%\frac{\sin(\lambda+1)\frac{u}{2}}{\sin\frac u2}
%\left(\sin(2(k+\kappa+1)+\lambda+1)\frac u2-\sin(2k+\lambda+1)\frac u2\right)=
%$$
%\begin{equation}\label{2.33}
%    \frac{2}{\sin\frac u2}\sin(\lambda+1)\frac{u}{2}\sin(\kappa+1)\frac{u}{2}
%\cos(2k+\kappa+\lambda+2)\frac u2.
%\end{equation}
% Равенство \eqref{2.26}  вытекает из \eqref{2.32} и \eqref{2.33}.
%
% Докажем \eqref{2.27}. Для этого заметим, что при $\nu=2\mu-1$ имеем $\cos(p u+\frac{\pi (2\mu-1)}{2})=(-1)^\mu\sin p u$, следовательно, \eqref{2.29} принимает вид
%$$
%v^k_{\kappa,\lambda}(u)=(-1)^\mu\sum\limits_{j=0}^\kappa\sin(k+1+\lambda+j)u=
%$$
%$$
%  (-1)^\mu\frac{\sin(k+1+\lambda+\kappa)\frac u2\sin(k+2+\lambda+\kappa)\frac u2-\sin(k+\lambda)\frac u2\sin(k+\lambda+1)\frac u2} {\sin\frac u2}=
%$$
%\begin{equation}\label{2.34}
%   (-1)^\mu \frac{\cos(2k+2\lambda+1)\frac u2-\cos(2k+2\lambda+2\kappa+3)\frac u2} {2\sin\frac u2}.
%\end{equation}
%
%
%Из \eqref{2.28} и \eqref{2.34} имеем
%$$
%_1\mathcal{ K}_{n,k}^{2\mu-1}(u)={(-1)^{\mu-1}\over n}\times
%$$
%\begin{equation}\label{2.35}
%    \sum\limits_{\kappa=0}^\infty\sum\limits_{\lambda=0}^{n-1} \Delta
%q_\mu(k+1+\lambda+\kappa) \frac{\cos(2k+2\lambda+1)\frac u2-\cos(2k+2\lambda+2\kappa+3)\frac u2} {2\sin\frac u2}.
%\end{equation}
%Применяя к внутренней сумме из \eqref{2.35} преобразование Абеля, мы получим
%$$
%_1\mathcal{ K}_{n,k}^{2\mu-1}(u)={(-1)^{\mu}\over n}\sum\limits_{\kappa=0}^\infty
%\sum\limits_{\lambda=0}^{n-2}\Delta^2 q_\mu(k+1+\kappa+\lambda)\frac{Y^k_{-1,\lambda}(u)-Y^k_{\kappa,\lambda}(u)}{2\sin\frac u2}+
%$$
%\begin{equation}\label{2.36}
%+{(-1)^{\mu-1}\over n}\sum\limits_{\kappa=0}^\infty\Delta q_\mu(n+k+\kappa)
%\frac{Y^k_{-1,n-1}(u)-Y^k_{\kappa,n-1}(u)}{2\sin\frac u2},
%\end{equation}
%где
%$$Y^k_{\kappa,\lambda}(u)=\sum\limits_{j=0}^\lambda\cos(2(k+1+j+\kappa)+1)\frac u2=
%$$
%$$
%\frac{\sin(k+\kappa+\lambda+2)u-\sin(k+\kappa+1)u}{2\sin\frac u2}=
%$$
%$$
%\frac{\sin(l+1)\frac{u}{2}\cos(2n+2k+l+3)\frac{u}{2}}{\sin\frac u2}.
%$$
%Отсюда имеем
%$$
%Y_{-1,\lambda}^k(u)-Y_{\kappa,\lambda}^k(u)=
%$$
%$$
%\frac{\sin(\lambda+1)\frac{u}{2}}{\sin\frac u2}(\cos(2k+\lambda+1)\frac{u}{2}-\cos(2k+2\kappa+\lambda+3)\frac{u}{2})=
%$$
%\begin{equation}\label{2.37}
%  \frac{2}{\sin\frac u2}\sin(\lambda+1)\frac{u}{2}\sin(\kappa+1)\frac{u}{2}
%\sin(2k+\lambda+\kappa+2)\frac{u}{2}.
%\end{equation}
%Равенство \eqref{2.27}  вытекает из \eqref{2.36} и \eqref{2.37}. Лемма \ref{l2.4} доказана.
% \end{proof}

Вернемся к равенству \eqref{2.25} и рассмотрим случай $\nu=2\mu$ -- четное.
Тогда в силу леммы \ref{l2.4} мы можем записать
$$
 _2\mathcal{ K}_{n,m}^{2\mu}(u)=
 $$
 $$
 {(-1)^\mu\over n^2}\sum\limits_{\lambda=1}^{n-1}
\sum\limits_{\kappa=1}^{\infty}\frac{\sin\frac{\lambda}{2}u\sin\frac{\kappa}{2}u}
{\sin^2\frac u2}
\sum_{k=m}^{m+n-1}\cos(k+\frac{\kappa+\lambda}{2})u\Delta^2g_\mu(k+\kappa+\lambda-1)
$$
 \begin{equation}\label{2.38}
    +{(-1)^{\mu-1}\over n^2}\sum\limits_{\kappa=1}^{\infty}
\frac{\sin\frac {n}{2}u\sin\frac{\kappa}{2}u}{\sin^2\frac u2}
\sum_{k=m}^{m+n-1}\cos(k+\frac{\kappa+n}{2})u\Delta g_\mu(k+n+\kappa-1).
 \end{equation}
 К внутренним суммам вида $\sum_{k=m}^{m+n-1}$, фигурирующим в правой части равенства \eqref{2.38}, применим преобразование Абеля, что дает
 $$
 \sum_{k=m}^{m+n-1}\cos(k+\frac{\kappa+\lambda}{2})u\Delta^2g_\mu(k+\kappa+\lambda-1)=
 $$
 \begin{equation}\label{2.39}
-\sum_{k=m}^{m+n-2}\Delta^3g_\mu(k+\kappa+\lambda-1)X_k^{\lambda,\kappa}
+\Delta^2g_\mu(m+n+\kappa+\lambda-2)X_{m+n-1}^{\lambda,\kappa},
\end{equation}
$$
\sum_{k=m}^{m+n-1}\cos(k+\frac{\kappa+n}{2})u\Delta g_\mu(k+n+\kappa-1)=
$$
 \begin{equation}\label{2.40}
-\sum_{k=m}^{m+n-2}\Delta^2g_\mu(k+n+\kappa-1)X_k^{n,\kappa}
+\Delta g_\mu(m+2n+\kappa-2)X_{m+n-1}^{n,\kappa},
    \end{equation}
где
$$
X_k^{\lambda,\kappa}=\sum_{j=m}^k \cos(ju+\frac{(\kappa+\lambda)u}{2})=
$$
$$
\frac{\cos(k+\kappa+\lambda)\frac{u}{2}\sin(k+1)\frac{u}{2}-
\cos(m+\kappa+\lambda-1)\frac{u}{2}\sin\frac{mu}{2}}{\sin\frac{u}{2}}=
$$
$$
=\frac{\sin(2k+\kappa+\lambda+1)\frac{u}{2}-\sin(2m+\kappa+\lambda-1)\frac{u}{2}}{2\sin\frac{u}{2}}
$$
 \begin{equation}\label{2.41}
=\frac{\sin(k-m+1)\frac{u}{2}\cos(m+k+\kappa+\lambda)\frac{u}{2}}{\sin\frac{u}{2}}.
    \end{equation}
Из равенств \eqref{2.38} -- \eqref{2.41} мы выводим следующий результат.

\begin{lemma}\label{l2.5}
Имеет место равенство
$$
_2\mathcal{ K}_{n,m}^{2\mu}(u)={(-1)^{\mu-1}\over n^2\sin^3\frac{u}{2}}\left(\sum\limits_{\lambda=1}^{n-1}\sum_{k=1}^{n-1}
\sum\limits_{\kappa=1}^{\infty}\Delta^3g_\mu(m+k+\kappa+\lambda-2)\times\right.
$$
$$
\sin\frac{\lambda u}{2}\sin\frac{\kappa u}{2}\sin\frac{ku}{2}\cos(2m+k+\kappa+\lambda-1)\frac{u}{2}
$$
$$
 -\sum\limits_{\lambda=1}^{n-1}
\sum\limits_{\kappa=1}^{\infty}\Delta^2g_\mu(m+n+\kappa+\lambda-2)\times
$$
$$
\sin\frac{\lambda u}{2}\sin\frac{\kappa u}{2}\sin\frac{nu}{2}\cos(2m+n+\kappa+\lambda-1)\frac{u}{2}
$$
$$
 -\sum_{k=1}^{n-1}
\sum\limits_{\kappa=1}^{\infty}\Delta^2g_\mu(m+n+k+\kappa-2)\times
$$
$$
\sin\frac{nu}{2}\sin\frac{\kappa u}{2}\sin\frac{ku}{2}\cos(2m+n+k+\kappa-1)\frac{u}{2}
$$
$$
 \left.+\sum\limits_{\kappa=1}^{\infty}\Delta g_\mu(m+2n+\kappa-2)
\sin^2\frac{nu}{2}\sin\frac{\kappa u}{2}\cos(2m+2n+\kappa-1)\frac{u}{2}\right).
$$
\end{lemma}
Рассмотрим \eqref{2.20} в случае $\nu=2\mu-1$ --  нечетно. Из леммы \ref{l2.4} (равенство \eqref{2.27}) имеем
$$
 _2\mathcal{ K}_{n,m}^{2\mu-1}(u)=
 $$
 $$
 {(-1)^\mu\over n^2}\sum\limits_{\lambda=1}^{n-1}
\sum\limits_{\kappa=1}^{\infty}\frac{\sin\frac{\lambda}{2}u\sin\frac{\kappa}{2}u}
{\sin^2\frac u2}
\sum_{k=m}^{m+n-1}\Delta^2g_\mu(k+\kappa+\lambda-1)\sin(ku+(\kappa+\lambda)\frac{u}{2})
$$
 \begin{equation}\label{2.42}
    -{(-1)^{\mu}\over n^2}\sum\limits_{\kappa=1}^{\infty}
\frac{\sin\frac {n}{2}u\sin\frac{\kappa}{2}u}{\sin^2\frac u2}
\sum_{k=m}^{m+n-1}\Delta g_\mu(k+n+\kappa-1)\sin(ku+(n+\kappa)\frac{u}{2}).
 \end{equation}
К внутренним суммам вида $\sum_{k=m}^{m+n-1}$, фигурирующим в правой части равенства \eqref{2.42}, применим преобразование Абеля, что дает
 $$
 \sum_{k=m}^{m+n-1}\Delta^2g_\mu(k+\kappa+\lambda-1)\sin(ku+(\kappa+\lambda)\frac{u}{2})=
 $$
 \begin{equation}\label{2.43}
-\sum_{k=m}^{m+n-2}\Delta^3q_\mu(k+\kappa+\lambda-1)Z_k^{\lambda,\kappa}
+\Delta^2q_\mu(m+n+\kappa+\lambda-2)Z_{m+n-1}^{\lambda,\kappa},
\end{equation}
$$
\sum_{k=m}^{m+n-1}\sin(k+\frac{\kappa+n}{2})u\Delta q_\mu(k+n+\kappa-1)=
$$
 \begin{equation}\label{2.44}
-\sum_{k=m}^{m+n-2}\Delta^2q_\mu(k+n+\kappa-1)Z_k^{n,\kappa}
+\Delta q_\mu(m+2n+\kappa-2)Z_{m+n-1}^{n,\kappa},
    \end{equation}
где
$$
Z_k^{\lambda,\kappa}=\sum_{j=m}^k \sin(ju+\frac{(\kappa+\lambda)u}{2})=
$$
$$
\frac{\sin(k+\kappa+\lambda)\frac{u}{2}\sin(k+1)\frac{u}{2}-
\sin(m+\kappa+\lambda-1)\frac{u}{2}\sin\frac{mu}{2}}{\sin\frac{u}{2}}
$$
$$
=\frac{\cos(2k+\kappa+\lambda+1)\frac{u}{2}-\cos(\kappa+\lambda-1)\frac{u}{2}}{2\sin\frac{u}{2}}
$$
 \begin{equation}\label{2.45}
=\frac{\sin(k-m+1)\frac{u}{2}\sin(m+k+\kappa+\lambda)\frac{u}{2}}{\sin\frac{u}{2}}.
    \end{equation}
Из равенств \eqref{2.42} -- \eqref{2.45} мы выводим следующий результат.

\begin{lemma}\label{l2.6}
Имеет место равенство
$$
_2\mathcal{ K}_{n,m}^{2\mu-1}(u)={(-1)^{\mu-1}\over n^2\sin^3\frac{u}{2}}\left(\sum\limits_{\lambda=1}^{n-1}\sum_{k=1}^{n-1}
\sum\limits_{\kappa=1}^{\infty}\Delta^3q_\mu(m+k+\kappa+\lambda-2)\times\right.
$$
$$
\sin\frac{\lambda u}{2}\sin\frac{\kappa u}{2}\sin\frac{ku}{2}\sin(2m+k+\kappa+\lambda-1)\frac{u}{2}
$$
$$
 -\sum\limits_{\lambda=1}^{n-1}
\sum\limits_{\kappa=1}^{\infty}\Delta^2q_\mu(m+n+\kappa+\lambda-2)\times
$$
$$
\sin\frac{\lambda u}{2}\sin\frac{\kappa u}{2}\sin\frac{nu}{2}\sin(2m+n+\kappa+\lambda-1)\frac{u}{2}
$$
$$
 -\sum_{k=1}^{n-1}
\sum\limits_{\kappa=1}^{\infty}\Delta^2q_\mu(m+n+k+\kappa-2)\times
$$
$$
\sin\frac{nu}{2}\sin\frac{\kappa u}{2}\sin\frac{ku}{2}\sin(2m+n+k+\kappa-1)\frac{u}{2}
$$
$$
 \left.+\sum\limits_{\kappa=1}^{\infty}\Delta q_\mu(m+2n+\kappa-2)
\sin^2\frac{nu}{2}\sin\frac{\kappa u}{2}\sin(2m+2n+\kappa-1)\frac{u}{2}\right).
$$
\end{lemma}

\begin{lemma}\label{l2.7}
Имеют место равенства
$$
\mathcal{ K}_l^{2\mu}(u)=
(-1)^{\mu-1}\sum_{\kappa=1}^{\infty}
\Delta g_\mu(\kappa+l){\sin\frac{\kappa u}{2}\cos(2l+\kappa+1)\frac{u}{2}\over\sin\frac{u}{2}},
$$
$$
\mathcal{ K}_l^{2\mu-1}(u)=
(-1)^{\mu-1}\sum_{\kappa=1}^{\infty}
\Delta q_\mu(\kappa+l){\sin\frac{\kappa u}{2}\sin(2l+\kappa+1)\frac{u}{2}\over\sin\frac{u}{2}}.
$$
где $g_\mu(t)=t^{-2\mu}$, $q_\mu(t)=t^{-2\mu-1}$.
\end{lemma}
%\begin{proof}
%Рассмотрим сначала случай четного $\nu=2\mu$. Тогда в силу \eqref{2.23} имеем
%$$
%\mathcal{ K}_l^{2\mu}(u)= (-1)^\mu\sum_{z=l+1}^{\infty}{\cos(zu)\over z^\nu}=(-1)^\mu\sum_{\kappa=1}^{\infty}{\cos((\kappa+l)u)\over (\kappa+l)^\nu}.
%$$
%Отсюда, в результате преобразования Абеля, имеем
%$$
%\mathcal{ K}_l^{2\mu}(u)=(-1)^{\mu-1}\sum_{\kappa=1}^{\infty}
%\Delta g_\mu(\kappa+l){\sin(\kappa+l+\frac12)u)-\sin(l+\frac12)u)\over2\sin\frac{u}{2}}=
%$$
%\begin{equation}\label{2.43}
%(-1)^{\mu-1}\sum_{\kappa=1}^{\infty}
%\Delta g_\mu(\kappa+l){\sin\frac{\kappa u}{2}\cos(2l+\kappa+1)\frac{u}{2}\over\sin\frac{u}{2}  }.
%\end{equation}
%
%Перейдем к нечетному $\nu=2\mu-1$. Тогда в силу \eqref{2.20} имеем
%$$
%\mathcal{ K}_l^{2\mu-1}(u)= (-1)^\mu\sum_{z=l+1}^{\infty}{\sin(zu)\over z^\nu}=(-1)^\mu\sum_{\kappa=1}^{\infty}{\sin((\kappa+l)u)\over (\kappa+l)^\nu}.
%$$
%Отсюда, в результате преобразования Абеля, имеем
%$$
%\mathcal{ K}_l^{2\mu-1}(u)=
%$$
%$$
%(-1)^{\mu-1}\sum_{\kappa=1}^{\infty}
%\Delta q_\mu(\kappa+l)\frac{\sin(\kappa+l)\frac u2\sin(\kappa+l+1)\frac u2-\sin l\frac u2\sin(l+1)\frac u2} {\sin\frac u2}=
%$$
%$$
%(-1)^{\mu-1}\sum_{\kappa=1}^{\infty}
%\Delta q_\mu(\kappa+l)\frac{\cos(2l+1)\frac u2-\cos(2\kappa+2l+1)\frac u2}{2\sin\frac u2}=
%$$
%\begin{equation}\label{2.44}
%(-1)^{\mu-1}\sum_{\kappa=1}^{\infty}
%\Delta q_\mu(\kappa+l)\frac{\sin\frac{\kappa u}{2}\sin(2l+\kappa+1)\frac{u}{2}}{\sin\frac u2}.
%\end{equation}
%Утверждение леммы \ref{l2.7} вытекает из \eqref{2.43} и \eqref{2.44}.
%
%\end{proof}

\section{Приближение кусочно-гладких непрерывных периодических функций}\label{s3}


Рассмотрим задачу о приближении функций $f\in\mathcal{ I}^r_\Omega$ суммами Фурье
$S_n(f,x)$, средними Валле Пуссена  $_1V_{n,n}(f,x)$ и повторными средними Валле Пуссена $_2V_{n,n}(f,x)$. Для этого мы вернемся  к вопросу  об оценке величин  $ |R_n(f,x)|$,
 $| _1R_{n,m}(f,x)|$ и $|_2R_{n,m}(f,x)|$ (см. \eqref{2.2},  \eqref{2.15}, \eqref{2.16}). Поскольку для этих величин имеют место представления \eqref{2.12}, \eqref{2.17} и \eqref{2.18}, а величины $|\tilde R_n(f,x)|$, $|_1\tilde R_{n,m}(f,x)|$ и  $|_2\tilde R_{n,m}(f,x)|$ уже оценены (см. леммы \ref{l2.2} и \ref{l2.3}), то нам остаётся рассмотреть
 $|\hat R_n(f,x)|$, $|_1\hat R_{n,m}(f,x)|$ и  $|_2\hat R_{n,m}(f,x)|$ (см. \eqref{2.13}, \eqref{2.19},   \eqref{2.20}). С этой целью введем следующие обозначения. Если $f\in\mathcal{ I}^r_\Omega$, то положим
\begin{equation}\label{3.1}
J_{\Omega,r}(f)=\max\{|f^{(\nu)}(x_j+0)|,|f^{(\nu)}(x_{j+1}-0)|: 0\le j\le s, 1\le\nu\le r\},
\end{equation}
\begin{equation}\label{3.2}
Q_{\Omega,\varepsilon}=\bigcup_{j=0}^s[x_j+\varepsilon,x_{j+1}+\varepsilon],
\end{equation}
где $\quad 0<\varepsilon<\frac12\min\{x_{j+1}-x_{j}:0\le j\le s\}$ и
заметим, что в силу \eqref{2.10} и \eqref{2.20} мы можем записать
$$
 \hat R_l(f,x)=\frac{1}{\pi}\sum_{j=0}^s
\sum_{\nu=1}^rf^{(\nu)}(x_j+0)\mathcal{ K}_l^{\nu+1}(x-x_j)
$$
\begin{equation}\label{3.3}
-\frac{1}{\pi}\sum_{j=0}^s
\sum_{\nu=1}^r f^{(\nu)}(x_{j+1}-0)\mathcal{ K}_l^{\nu+1}(x-x_{j+1}),
\end{equation}
поэтому из \eqref{2.16}, \eqref{2.17}, \eqref{2.21} и \eqref{2.22} находим
$$
 _1\hat R_{n,m}(f,x)=\frac{1}{\pi}\sum_{j=0}^s
\sum_{\nu=1}^rf^{(\nu)}(x_j+0)_1\mathcal{ K}_{n,m}^{\nu+1}(x-x_j)
$$
\begin{equation}\label{3.4}
-\frac{1}{\pi}\sum_{j=0}^s
\sum_{\nu=1}^r f^{(\nu)}(x_{j+1}-0)_1\mathcal{ K}_{n,m}^{\nu+1}(x-x_{j+1}),
\end{equation}
$$
 _2\hat R_{n,m}(f,x)=\frac{1}{\pi}\sum_{j=0}^s
\sum_{\nu=1}^rf^{(\nu)}(x_j+0)_2\mathcal{ K}_{n,m}^{\nu+1}(x-x_j)
$$
\begin{equation}\label{3.5}
-\frac{1}{\pi}\sum_{j=0}^s
\sum_{\nu=1}^r f^{(\nu)}(x_{j+1}-0)_2\mathcal{ K}_{n,m}^{\nu+1}(x-x_{j+1}).
\end{equation}
Чтобы оценить величины $|_2\mathcal{ K}_{n,m}^{\nu+1}(x-x_{j})|$ обратимся к леммам \ref{l2.5} и \ref{l2.6}. Если $x\in Q_{\Omega,\varepsilon}$, то $|\sin(x-x_j)/2|>\sin(\varepsilon/2)$, поэтому в силу указанных лемм мы можем записать
$$
|_2\mathcal{ K}_{n,m}^{\nu+1}(x-x_j)|\le{c(\nu,\varepsilon)\over n^2}
\left(\sum\limits_{\lambda=1}^{n-1}\sum_{k=1}^{n-1}
\sum\limits_{\kappa=1}^{\infty}|\Delta^3F(m+k+\kappa+\lambda-2)|+\right.
$$
$$
 \sum\limits_{\lambda=1}^{n-1}
\sum\limits_{\kappa=1}^{\infty}|\Delta^2F(m+n+\kappa+\lambda-2)|
 +\sum_{k=1}^{n-1}
\sum\limits_{\kappa=1}^{\infty}\Delta^2|F(m+n+k+\kappa-2)|
$$
$$
 \left.+\sum\limits_{\kappa=1}^{\infty}|\Delta F(m+2n+\kappa-2)|\right),
$$
где $F(t)=t^{\nu+1} $, $\nu\ge1$. Отсюда следует, что если $x\in Q_{\Omega,\varepsilon}$, то
\begin{equation}\label{3.6}
|_2\mathcal{ K}_{n,m}^{\nu+1}(x-x_j)|\le{c(\nu,\varepsilon)\over n^2(m+1)^2}.
\end{equation}
Аналогично, из леммы \ref{l2.4}  выводим следующую оценку
\begin{equation}\label{3.7}
|_1\mathcal{ K}_{n,m}^{\nu+1}(x-x_j)|\le{c(\nu,\varepsilon)\over n(m+1)^2}, \quad x\in Q_{\Omega,\varepsilon}, 0\le j\le s+1,
\end{equation}
а из леммы \ref{l2.7}  вытекает оценка
\begin{equation}\label{3.8}
|\mathcal{ K}_{m}^{\nu+1}(x-x_j)|\le{c(\nu,\varepsilon)\over( m+1)^2}, \quad x\in Q_{\Omega,\varepsilon}, 0\le j\le s+1.
\end{equation}

Сопоставляя \eqref{3.6} -- \eqref{3.8} c \eqref{3.3} -- \eqref{3.5}, мы можем сформулировать следующее утверждение.
\begin{lemma}\label{l3.1}
Пусть $r \ge4$, $f\in\mathcal{ I}^r_\Omega$. Тогда если $x\in Q_{\Omega,\varepsilon}$, то
имеют место оценки
$$
|\hat R_{m}(f,x)|\le{c(r,\varepsilon)J_{\Omega,r}(f)\over (m+1)^2},
$$
$$
|_1\hat R_{n,m}(f,x)|\le{c(r,\varepsilon)J_{\Omega,r}(f)\over n(m+1)^2},
$$
$$
|_2\hat R_{n,m}(f,x)|\le{c(r,\varepsilon)J_{\Omega,r}(f)\over n^2(m+1)^2}.
$$
\end{lemma}

Из лемм \ref{l2.2}, \ref{l2.3}, \ref{l3.1} и равенств  \eqref{2.12}, \eqref{2.17},  \eqref{2.18} выводим следующий основной результат настоящей работы.
\begin{theorem}\label{theo1}
 Пусть $-\pi=x_0<x_1<\cdots<x_s<x_{s+1}=\pi$, $\Omega=\{x_0,x_1,\ldots x_s,x_{s+1}\}$,  $ 0<\varepsilon<\frac12\min\{x_{j+1}-x_{j}:0\le j\le s\}$, множество $Q_{\Omega,\varepsilon}$ определено равенством \eqref{3.2},
$f\in\mathcal{ I}^r_\Omega$. Тогда если $r\ge3$, $x\in Q_{\Omega,\varepsilon}$, то
имеют место оценки
$$
|R_{m}(f,x)|\le{c(r,\varepsilon)J_{\Omega,r}(f)\over (m+1)^2}+{c(r)I_r(f)\over (m+1)^{r}},
$$
$$
|_1R_{n,m}(f,x)|\le{c(r,\varepsilon)J_{\Omega,r}(f)\over n(m+1)^2}+{c(r)I_r(f)\over (n+m)(m+1)^{r-1}},
$$
$$
|_2R_{n,m}(f,x)|\le{c(r,\varepsilon)J_{\Omega,r}(f)\over n^2(m+1)^2}+{c(r)I_r(f)\over (n+m)^2(m+1)^{r-2}},
$$
где величина $J_{\Omega,r}(f)$ определена равенством \eqref{3.1},
 $I_r(f)=\int_{-\pi}^{\pi}|f^{(r+1)}(t)|dt$.
\end{theorem}

\section{Приближение непериодических функций посредством перекрывающих преобразований }\label{s4}
Пусть функция $f=f(x)$ задана и непрерывна на отрезке $[0,d\pi]$, где $2\le d$ -- натуральное число. Для каждого $0\le l\le 2d-2$ на $[0,\pi]$ определим функцию $f_l(x)=f(x+l\pi/2)$ ($x\in[0,\pi]$). Далее, определим  операторы $LS_n(f)$, $L_1V_n(f)$ и $L_2V_n(f)$, полагая
\begin{equation}\label{4.1}
LS_n(f)(x)=\begin{cases}S_n(f,x),&\text{при $0\le x<\pi/4$,}\\
                        S_n(f_{l},x-\frac{l\pi}{2})\, (0\le l\le 2d-2),&\text{при $x\in[(l+\frac12)\frac\pi2,(l+\frac32)\frac\pi2)$,}\\
                        S_n(f_{2d-2},x-(d-1)\pi),&\text{при $d\pi-\pi/4< x\le d\pi$,}
           \end{cases}
\end{equation}
\begin{equation}\label{4.2}
L_1V_n(f)(x)=\begin{cases}_1V_{n,n}(f,x),&\text{при $0\le x<\pi/4$,}\\
                        _1V_{n,n}(f_{l},x-\frac{l\pi}{2})\, (0\le l\le 2d-2),&\text{при $x\in[(l+\frac12)\frac\pi2,(l+\frac32)\frac\pi2)$,}\\
                        _1V_{n,n}(f_{2d-2},x-(d-1)\pi),&\text{при $d\pi-\pi/4< x\le d\pi$,}
           \end{cases}
\end{equation}
\begin{equation}\label{4.3}
L_2V_n(f)(x)=\begin{cases}_2V_{n,n}(f,x),&\text{при $0\le x<\pi/4$,}\\
                        _2V_{n,n}(f_{l},x-\frac{l\pi}{2})\, (0\le l\le 2d-2),&\text{при $x\in[(l+\frac12)\frac\pi2,(l+\frac32)\frac\pi2)$, }\\
                        _2V_{n,n}(f_{2d-2},x-(d-1)\pi),&\text{при $d\pi-\pi/4< x\le d\pi$,}
           \end{cases}
\end{equation}
и считая $LS_n(f)(x)$, $L_1V_n(f)(x)$ и $L_2V_n(f)(x)$ непрерывными слева в точке $x=d\pi-\pi/4$ . Будем рассматривать операторы как аппарат приближения дифференцируемых (вообще говоря, непериодических) функций, заданных на $[0,d\pi]$.  Из теоремы \ref{theo1} непосредственно выводим
\begin{corollary}\label{cor}
  Пусть $f\in W_1^{4}[0,d\pi]$, $\pi/4\le x\le d\pi-\pi/4$. Тогда имеют место следующие оценки
  $$
  |f(x)-LS_n(f,x)|\le \frac{c(f)}{n^2},
  $$
$$
  |f(x)-L_1V_n(f,x)|\le \frac{c(f)}{n^3},
  $$
$$
  |f(x)-L_2V_n(f,x)|\le \frac{c(f)}{n^4}.
  $$
\end{corollary}
%\begin{proof}
%В самом деле, функция $f_l(x)=f(x+l\pi/2)$ с $0\le l\le 2d-2$ можно продолжить на отрезок $[-\pi,\pi]$ по четности, а затем на всю ось $\mathbb{R}$ $2\pi$-периодически. Тогда, поскольку $f\in W_1^{4}[0,d\pi]$, то $f_l\in\mathcal{ I}^3_\Omega$ и, стало быть, мы попадаем в условия теоремы \ref{theo1} с $r=3$, откуда и вытекают утверждения следствия \ref{cor}.
%\end{proof}




%
%
%\section{Численные эксперименты }\label{s5}
%В настоящем пункте мы продемонстрируем эффективность предлагаемого алгоритма приближения гладких периодических функций последством повторных средних Валле Пуссена, путем сопоставления результатов численных экспериментов, проведенных нами одновременно для операторов $LS_{3n-2}(f)$ и $L_2V_n(f)$. При этом отметим, что множество коэффициентов Фурье всех функций $f_l$, привлекаемых  при конструировании оператора $LS_{3n-2}(f)$,  совпадает с аналогичным множеством для оператора $L_2V_n(f)$. Приведем ниже программу на языке Q-BASIC, которую мы разработали для численной реализации значений  $LS_{3n-2}(f,x)$ и $L_2V_n(f,x)$ на сетке $x_j=(2j+1)/(2M), \,j=0,\ldots,M-1$. При этом отметим, что вместо функции $f(x)=\sqrt{x+1}$, взятой  в программе для проведения численного эксперимента, может быть взята любая другая достаточно гладкая функция, заданная на отрезке $[0,d\pi]$, где $2\le d$ -- произвольное натуральное число.  Заметим также, что программу можно существенно улучшить путем применения алгоритма быстрого преобразования Фурье для вычисления коэффициентов Фурье функций $g_l(x)=g(x+l\pi/2)$ c $0\le l\le 2d-2$, играющих   в программе роль функций $f_l(x)=f(x+l\pi/2)$.
%
%{\small
%\begin{verbatim}
% REM Lapped transform
%REM  "Chislo tochek diskretizacii x_j=(2j+1)/(2M) na (0,pi); M"
%REM  "Chislo otrezkov [0,pi] v obl.opr.func. g; p1"
%REM "n usrednenie, prichem  3n-2<=M"
% M = 128
% p1 = 2
% n = 10
% n1 = 3 * n - 2
% M1 = p1 * M
% p = 2 * p1 - 2
%
%DEF fng (x) = (x + 1) ^ (1 / 2)
%pi = ATN(1) * 4
%DIM g(M1 - 1): REM Ishodnaya function
%DIM gv(M1 - 1): REM Priblizjenie vtorymi srednimi V-sena (laped)
%DIM gf(M1 - 1): REM Priblizjenie summami Fur'e (laped)
%DIM A(n1, p): REM Fourier coofficients (matrica P X n1)
%DIM V2(n1): REM mnozjiteli Valle Pussena
%DIM AV(n1, p): REM Fourier X mnojit Valle (matrica P X n1)
%
%REM Mnozjitely Valle Pussena
%FOR j = 0 TO n
%V2(j) = 1
%NEXT j
%FOR j = n + 1 TO 2 * n - 1
%V2(j) = (2 * n ^ 2 - (j - n) * (j - n + 1)) / (2 * n ^ 2)
%NEXT j
%FOR j = 2 * n TO 3 * n - 2
%V2(j) = (3 * n - j) * (3 * n - j - 1) / (2 * n ^ 2)
%NEXT j
%
%REM Znacheniya function g(x) na setke
%
%FOR j = 0 TO M1 - 1
%g(j) = fng((2 * j + 1) * pi / (2 * M))
%NEXT j
%
%REM Pryamoe preobrazovaniye Fourier
%pm = M \ 2
%FOR l = 0 TO p
%FOR k = 0 TO n1
%A(k, l) = 0
%FOR j = 0 TO M - 1
%A(k, l) = A(k, l) + COS(k * (2 * j + 1) * pi / (2 * M)) * g(l * pm + j)
%NEXT j
%A(k, l) = 2 * A(k, l) / M
%NEXT k
%NEXT l
%
%REM Umnozjenie na mnozjiteli Valle Pussena
%FOR l = 0 TO p
%FOR k = 0 TO n1
%AV(k, l) = V2(k) * A(k, l)
%NEXT k
%NEXT l
%
%REM Obratnoe  preobrazovanie Fuorier  (laped)
%qm = M \ 4
%FOR j = 0 TO 3 * qm
%s = A(0, 0) / 2
%FOR k = 1 TO n1
%s = s + A(k, 0) * COS(k * (2 * j + 1) * pi / (2 * M))
%NEXT k
%gf(j) = s
%NEXT j
%
%
%FOR j = qm + 1 TO M - 1
%s = A(0, p) / 2
%FOR k = 1 TO n1
%s = s + A(k, p) * COS(k * (2 * j + 1) * pi / (2 * M))
%NEXT k
%gf(j + p * pm) = s
%NEXT j
%
%FOR l = 1 TO p - 1
%FOR j = qm + 1 TO qm + pm
%s = A(0, l) / 2
%FOR k = 1 TO n1
%s = s + A(k, l) * COS(k * (2 * j + 1) * pi / (2 * M))
%NEXT k
%gf(j + l * pm) = s
%NEXT j
%NEXT l
%
%REM Obratnoe  preobrazovanie Valle Pussena  (laped)
%
%FOR j = 0 TO 3 * qm
%s = AV(0, 0) / 2
%FOR k = 1 TO n1
%s = s + AV(k, 0) * COS(k * (2 * j + 1) * pi / (2 * M))
%NEXT k
%gv(j) = s
%NEXT j
%
%
%FOR j = qm + 1 TO M - 1
%s = AV(0, p) / 2
%FOR k = 1 TO n1
%s = s + AV(k, p) * COS(k * (2 * j + 1) * pi / (2 * M))
%NEXT k
%gv(j + p * pm) = s
%NEXT j
%
%FOR l = 1 TO p - 1
%FOR j = qm + 1 TO qm + pm
%s = AV(0, l) / 2
%FOR k = 1 TO n1
%s = s + AV(k, l) * COS(k * (2 * j + 1) * pi / (2 * M))
%NEXT k
%gv(j + l * pm) = s
%NEXT j
%NEXT l
%
%REM Vyvod rezul
%OPEN "result.txt" FOR OUTPUT AS #1
%M2 = (M1 - 1) \ 2
%FOR j = qm TO M2
%A=ABS(gv(j) - g(j))
%B=ABS(gf(j) - g(j))
%C=ABS(gv(j+M2-qm+1) - g(j+M2-qm+1))
%D=ABS(gf(j+M2-qm+1) - g(j+M2-qm+1))
%WRITE #1,A,B,C,D
%NEXT j
%CLOSE #1
%
%Приведем таблицу отклонений  |f(x)-L_2V_n(f,x)|  и  |f(x)-LS_{3n-2}(f,x)|
%на сетке  x_j=(2j+1)/(2M), когда \pi/4 <= x_j <= d*M-M/4:
%
%
%
%
%|f-L_2V_n(f)|     |f-LS_{3n-2}(f)|         |f-L_2V_n(f|       |f-LS_{3n-2}(f)|
%1.621246E-05,"   ",2.254248E-04,"          ",1.430511E-06,"   ",1.056194E-04
%3.898144E-05,"   ",4.53949E-04,"          ",3.814697E-06,"   ",1.962185E-04
%5.877018E-05,"   ",4.652739E-04,"          ",6.67572E-06,"   ",1.966953E-04
%7.033348E-05,"   ",2.657175E-04,"          ",8.34465E-06,"   ",1.080036E-04
%7.18832E-05,"   ",4.470348E-05,"          ",7.629395E-06,"   ",2.861023E-05
%6.210804E-05,"   ",3.20673E-04,"          ",6.67572E-06,"   ",1.530647E-04
%4.518032E-05,"   ",4.388094E-04,"          ",4.768372E-06,"   ",2.062321E-04
%2.43187E-05,"   ",3.532171E-04,"          ",3.576279E-06,"   ",1.652241E-04
%4.768372E-06,"   ",1.108646E-04,"          ",7.152557E-07,"   ",4.959106E-05
%1.060963E-05,"   ",1.726151E-04,"          ",2.384186E-07,"   ",8.940697E-05
%1.966953E-05,"   ",3.664494E-04,"          ",9.536743E-07,"   ",1.878738E-04
%2.288818E-05,"   ",3.871918E-04,"          ",9.536743E-07,"   ",2.012253E-04
%2.253056E-05,"   ",2.31266E-04,"          ",1.907349E-06,"   ",1.223087E-04
%1.93119E-05,"   ",2.396107E-05,"          ",1.907349E-06,"   ",1.239777E-05
%1.597404E-05,"   ",2.598763E-04,"          ",2.384186E-06,"   ",1.423359E-04
%1.335144E-05,"   ",3.701448E-04,"          ",3.099442E-06,"   ",2.090931E-04
%1.132488E-05,"   ",3.083944E-04,"          ",4.291534E-06,"   ",1.797676E-04
%1.001358E-05,"   ",1.083612E-04,"          ",5.483627E-06,"   ",6.842613E-05
%9.179115E-06,"   ",1.347065E-04,"          ",6.914139E-06,"   ",7.605553E-05
%8.34465E-06,"   ",3.105402E-04,"          ",9.059906E-06,"   ",1.871586E-04
%7.152557E-06,"   ",3.401041E-04,"          ",1.096725E-05,"   ",2.148151E-04
%6.079674E-06,"   ",2.146959E-04,"          ",9.536743E-06,"   ",1.451969E-04
%4.649162E-06,"   ",5.483627E-06,"          ",6.914139E-06,"   ",6.914139E-06
%3.099442E-06,"   ",2.18153E-04,"          ",1.66893E-06,"   ",1.370907E-04
%0,"              ",3.265142E-04,"          ",6.914139E-06,"   ",2.205372E-04
%4.053116E-06,"   ",2.837181E-04,"          ",1.597404E-05,"   ",2.052784E-04
%8.702278E-06,"   ",1.12772E-04,"          ",2.360344E-05,"   ",9.512901E-05
%1.28746E-05,"   ",1.063347E-04,"          ",2.813339E-05,"   ",6.246567E-05
%1.645088E-05,"   ",2.732277E-04,"          ",2.908707E-05,"   ",1.964569E-04
%1.716614E-05,"   ",3.123283E-04,"          ",2.503395E-05,"   ",2.439022E-04
%1.573563E-05,"   ",2.090931E-04,"          ",1.716614E-05,"   ",1.807213E-04
%1.120567E-05,"   ",1.215935E-05,"          ",8.821487E-06,"   ",3.147125E-05
%4.649162E-06,"   ",1.875162E-04,"          ",1.192093E-06,"   ",1.378059E-04
%1.430511E-06,"   ",2.987385E-04,"          ",1.978874E-05,"   ",2.205372E-04
%6.437302E-06,"   ",2.725124E-04,"          ",2.932549E-05,"   ",2.446175E-04
%9.536743E-06,"   ",1.223087E-04,"          ",3.480911E-05,"   ",1.571178E-04
%1.049042E-05,"   ",8.177757E-05,"          ",3.457069E-05,"   ",3.33786E-06
%9.179115E-06,"   ",2.471209E-04,"          ",3.004074E-05,"   ",1.444817E-04
%7.271767E-06,"   ",2.981424E-04,"          ",2.193451E-05,"   ",2.219677E-04
%5.125999E-06,"   ",2.129078E-04,"          ",1.168251E-05,"   ",1.957417E-04
%3.576279E-06,"   ",3.147125E-05,"          ",9.536743E-07,"   ",8.249283E-05
%1.907349E-06,"   ",1.635551E-04,"          ",5.245209E-06,"   ",6.556511E-05
%1.072884E-06,"   ",2.826452E-04,"          ",1.049042E-05,"   ",1.773834E-04
%2.384186E-07,"   ",2.733469E-04,"          ",1.144409E-05,"   ",2.064705E-04
%3.576279E-07,"   ",1.38998E-04,"          ",1.0252E-05,"   ",1.425743E-04
%1.66893E-06,"   ",5.853176E-05,"          ",9.536743E-06,"   ",1.430511E-05
%1.788139E-06,"   ",2.293587E-04,"          ",7.629395E-06,"   ",1.146793E-04
%2.622604E-06,"   ",2.95639E-04,"          ",5.960464E-06,"   ",1.895428E-04
%3.814697E-06,"   ",2.273321E-04,"          ",4.291534E-06,"   ",1.759529E-04
%5.483627E-06,"   ",5.531311E-05,"          ",4.768372E-06,"   ",8.392334E-05
%7.987022E-06,"   ",1.431704E-04,"          ",3.814697E-06,"   ",4.529953E-05
%1.072884E-05,"   ",2.775192E-04,"          ",3.099442E-06,"   ",1.50919E-04
%1.323223E-05,"   ",2.857447E-04,"          ",2.861023E-06,"   ",1.852512E-04
%1.323223E-05,"   ",1.64032E-04,"          ",2.145767E-06,"   ",1.36137E-04
%1.060963E-05,"   ",3.445148E-05,"          ",9.536743E-07,"   ",2.551079E-05
%4.768372E-06,"   ",2.189875E-04,"          ",4.768372E-07,"   ",9.393692E-05
%4.410744E-06,"   ",3.073215E-04,"          ",9.536743E-07,"   ",1.692772E-04
%1.525879E-05,"   ",2.559423E-04,"          ",3.099442E-06,"   ",1.659393E-04
%2.515316E-05,"   ",8.702278E-05,"          ",5.245209E-06,"   ",8.749962E-05
%3.123283E-05,"   ",1.250505E-04,"          ",7.152557E-06,"   ",2.980232E-05
%3.290176E-05,"   ",2.833605E-04,"          ",8.583069E-06,"   ",1.33276E-04
%2.884865E-05,"   ",3.166199E-04,"          ",8.821487E-06,"   ",1.740456E-04
%2.074242E-05,"   ",2.053976E-04,"          ",6.67572E-06,"   ",1.370907E-04
%1.0252E-05,"   ",3.695488E-06,"          ",4.768372E-06,"   ",3.838539E-05
%4.768372E-07,"   ",2.17557E-04,"          ",9.536743E-07,"   ",7.748604E-05
%2.467632E-05,"   ",2.801418E-04,"          ",2.145767E-06,"   ",1.571178E-04
%3.623962E-05,"   ",3.027916E-04,"          ",4.529953E-06,"   ",1.635551E-04
%4.374981E-05,"   ",1.87993E-04,"          ",5.722046E-06,"   ",9.608269E-05
%4.36306E-05,"   ",7.152557E-06,"          ",6.437302E-06,"   ",1.573563E-05
%3.802776E-05,"   ",1.900196E-04,"          ",4.768372E-06,"   ",1.20163E-04
%2.682209E-05,"   ",2.806187E-04,"          ",3.33786E-06,"   ",1.709461E-04
%1.358986E-05,"   ",2.406836E-04,"          ",1.66893E-06,"   ",1.43528E-04
%1.549721E-06,"   ",9.393692E-05,"          ",4.768372E-07,"   ",5.054474E-05
%7.867813E-06,"   ",8.97646E-05,"          ",9.536743E-07,"   ",6.508827E-05
%1.335144E-05,"   ",2.2614E-04,"          ",9.536743E-07,"   ",1.516342E-04
%1.525879E-05,"   ",2.548695E-04,"          ",2.622604E-06,"   ",1.690388E-04
%1.442432E-05,"   ",1.679659E-04,"          ",2.384186E-06,"   ",1.091957E-04
%1.215935E-05,"   ",7.629395E-06,"          ",2.622604E-06,"   ",1.66893E-06
%1.060963E-05,"   ",1.522303E-04,"          ",2.622604E-06,"   ",1.122952E-04
%8.34465E-06,"   ",2.38061E-04,"          ",3.576279E-06,"   ",1.733303E-04
%7.033348E-06,"   ",2.144575E-04,"          ",4.291534E-06,"   ",1.554489E-04
%6.437302E-06,"   ",9.417534E-05,"          ",5.00679E-06,"   ",6.723404E-05
%5.602837E-06,"   ",6.532669E-05,"          ",7.390976E-06,"   ",5.364418E-05
%5.245209E-06,"   ",1.914501E-04,"          ",9.059906E-06,"   ",1.528263E-04
%4.410744E-06,"   ",2.276897E-04,"          ",1.001358E-05,"   ",1.821518E-04
%4.172325E-06,"   ",1.59502E-04,"          ",9.298325E-06,"   ",1.289845E-04
%2.861023E-06,"   ",2.074242E-05,"          ",6.67572E-06,"   ",1.549721E-05
%1.788139E-06,"   ",1.255274E-04,"          ",1.907349E-06,"   ",1.091957E-04
%4.768372E-07,"   ",2.118349E-04,"          ",6.198883E-06,"   ",1.859665E-04
%2.980232E-06,"   ",2.006292E-04,"          ",1.28746E-05,"   ",1.80006E-04
%5.960464E-06,"   ",9.894371E-05,"          ",2.002716E-05,"   ",9.10759E-05
%8.106232E-06,"   ",4.577637E-05,"          ",2.479553E-05,"   ",4.267693E-05
%1.001358E-05,"   ",1.678467E-04,"          ",2.479553E-05,"   ",1.604557E-04
%1.001358E-05,"   ",2.119541E-04,"          ",2.121925E-05,"   ",2.07901E-04
%8.821487E-06,"   ",1.587868E-04,"          ",1.478195E-05,"   ",1.61171E-04
%5.00679E-06,"   ",3.361702E-05,"          ",6.914139E-06,"   ",3.647804E-05
%
%
%\end{verbatim}
%}





\section{Приближение периодических кусочно-гладких  функций}


Перейдем к вопросу о локальных  аппроксимативных свойствах повторных средних Валле Пуссена на классах периодических кусочно-гладких функций, которые могут иметь разрывы первого рода.
С этой целью введем соответствующие классы кусочно-гладких $2\pi$-периодических функций.
Пусть $-\pi=x_0<x_1<\cdots<x_s<x_{s+1}=\pi$,
$f(x)$ -- $2\pi$-периодическая функция такая, что на каждом из отрезков $[x_j,x_{j+1}]$ $(j=0,1,\ldots,s)$ её можно превратить в абсолютно непрерывную функцию путем переопределения на концах $x_j$ и $x_{j+1}$. Множество всех таких функций мы обозначим  через $\mathcal{ P}_\Omega$, где $\Omega=\{x_0,x_1,\ldots x_s,x_{s+1}\}$.
%Через $W_1^r(a,b)$ обозначим пространство Соболева, состоящее из функций $f$, $r-1$-раз непрерывно дифференцируемых на $[a,b]$, для которых $f^{(r-1)}$ абсолютно непрерывна, а $f^{(r)}$ интегрируема на $[a,b]$.
Если $f\in \mathcal{  P}_\Omega$ и ее сужение на $[x_j,x_{j+1}]$ можно переопределить в точках
$x_j$ и $x_{j+1}$ так, чтобы было $f\in W_1^r(x_j,x_{j+1})$, то мы будем говорить, что $f\in {}_\Omega W_1^r$. Если функция $f\in{}_\Omega W_1^r$, то будем называть её кусочно-гладкой (порядка $r$).



\begin{theorem}
 Пусть $-\pi=x_0<x_1<\cdots<x_s<x_{s+1}=\pi$, $\Omega=\{x_0,x_1,\ldots x_s,x_{s+1}\}$,  $\quad 0<\varepsilon<\frac12\min\{x_{j+1}-x_{j}:0\le j\le s\}$, множество $Q_{\Omega,\varepsilon}$ определено равенством \eqref{3.2},
$f\in {}_\Omega W_1^r$. Тогда если $r\ge4$, $x\in Q_{\Omega,\varepsilon}$, то
имеют место оценки
$$
|R_{m}(f,x)|\le{c(r,\varepsilon)J_{\Omega,r}(f)\over m+1}+{c(r)I_r(f)\over (m+1)^{r-1}},
$$
$$
|_1R_{n,m}(f,x)|\le{c(r,\varepsilon)J_{\Omega,r}(f)\over n(m+1)}+{c(r)I_r(f)\over (n+m)(m+1)^{r-2}},
$$
$$
|_2R_{n,m}(f,x)|\le{c(r,\varepsilon)J_{\Omega,r}(f)\over n^2(m+1)}+{c(r)I_r(f)\over (n+m)^2(m+1)^{r-3}},
$$
где величина $J_{\Omega,r}(f)$ определена равенством \eqref{3.1},
 $I_r(f)=\int_{-\pi}^{\pi}|f^{(r)}(t)|dt$.
\end{theorem}

