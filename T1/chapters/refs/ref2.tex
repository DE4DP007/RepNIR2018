

В отчетном году продолжены исследования по теории систем функций, ортогональных по Соболеву, порожденных классическими ортогональными системами.
Особое внимание уделено вопросам сходимости рядов Фурье по полиномам, ортогональным по Соболеву, порожденным классическими ортогональными полиномами непрерывной и дискретной переменной.
Изучены асимптотические свойства полиномов, ортогональных по Соболеву, порожденных указанными классическими системами. В частности, изучены алгебраические и асимптотические свойства полиномов, ортогональных по Соболеву, порожденных классическими полиномами Якоби, Лагерра, полиномами Чебышева дискретной переменной, полиномами Мейкснера и полиномами Шарлье.

На основе систем функций, ортогональных по Соболеву, разработаны алгоритмы для численно-аналитического решения систем линейных и нелинейных дифференциальных и разностных уравнений. Для широкого класса систем функций, ортогональных по Соболеву, найдены условия, при соблюдении которых сходятся итерационные процессы, на которых основываются указанные алгоритмы для приближенного решения систем дифференциальных и разностных уравнений.
Ряд разработанных алгоритмов доведены до численных экспериментов (разработаны прикладные программные пакеты, реализующие указанные алгоритмы).
Проведенные эксперименты показывают высокую эффективность предлагаемого численно-аналитического подхода к решению систем дифференциальных и разностных уравнений.


%%%%%%%%%%%%%%%%
%%%%%%%%%%%%%%%%
%%%%%%%%%%%%%%%%





%Рассмотрена система функций $\mathcal{\psi}_{r,n}(x)$ $(r=1,2,\ldots, n=0,1,\ldots)$ ортонормированная по Соболеву относительно скалярного произведения  вида\linebreak $\langle f,g\rangle=\sum_{k=0}^{r-1}\Delta^kf(0)\Delta^kg(0)+
%\sum_{j=0}^\infty\Delta^rf(j)\Delta^rg(j)\rho(j)$,
%порожденная заданной ортонормированной системой функций $\mathcal{\psi}_{n}(x)$ $( n=0,1,\ldots)$. Показано, что ряды и суммы Фурье по системе
%$\mathcal{\psi}_{r,n}(x)$ $(r=1,2,\ldots, n=0,1,\ldots)$ является удобным и весьма эффективным инструментом приближенного решения задачи Коши для разностных уравнений.
%
%Для системы полиномов $l_{r,n}^{\alpha}(x)$ ($r$-натуральное число, $n=0, 1, \ldots$), ортонормированной относительно скалярного произведения типа Соболева (полиномы, ортонормированные по Соболеву) следующего вида $\langle f,g\rangle=\sum_{\nu=0}^{r-1}f^{(\nu)}(0)g^{(\nu)}(0)+\int_{0}^{\infty} f^{(r)}(x)g^{(r)}(x)\rho(x) dx$ и порожденной классическими ортонормированными полиномами Лагерра, получены рекуррентные соотношения, которые могут быть использованы для изучения различных свойств этих полиномов и вычисления их значений при любых $x$ и $n$.
%
%
%
%
%
%
%
%
%
%%%%%%%%%%%%%%%%%
%%%%%%%%%%%%%%%%%
%%%%%%%%%%%%%%%%%
%
%
%Рассмотрена задача о конструировании полиномов $s_{r,n}^\alpha(x)$, порожденных полиномами Шарлье $s_n^\alpha(x)$ и ортонормированных относительно скалярного произведения типа Соболева вида
%$  \langle f,g \rangle = \sum_{k=0}^{r-1} \Delta^k f(0) \Delta^k g(0) + \sum_{j=0}^\infty \Delta^r f(j) \Delta^r g(j) \rho(j) $, где $ \rho(x)=\alpha^x e^{-\alpha}/\Gamma(x+1)$.
%Показано, что система полиномов $s_{r,n}^\alpha(x)$, порожденная полиномами Шарлье, полна в гильбертовом пространстве $W^r_{l_\rho}$, состоящем из дискретных функций, заданных на сетке $\Omega=\{0,1,\ldots\}$, в котором введено скалярное произведение $\langle f,g \rangle$. Найдена явная формула вида $ s_{r,k+r}^{\alpha}(x) = \sum_{l=0}^{k} b_l^r x^{[l+r]} $, в которой $x^{[m]} = x(x-1)\ldots(x-m+1)$. Установлена связь полиномов $s_{r,n}^\alpha(x)$ с порождающими их ортонормированными классическими полиномами Шарлье $s_n^\alpha(x)$ вида $	s_{r,k+r}^{\alpha}(x)= U_k^r \left[s_{k+r}^{\alpha}(x) - \sum_{\nu=0}^{r-1} V_{k,\nu}^r x^{[\nu]}\right]$, в которой для чисел $U_k^r$, $V_{k,\nu}^r$ найдены явные выражения. 