\begin{thebibliography}{1} %% здесь библиографический список

%ШИИ BasesJacobiWeighted
\bibitem{ShIIBJWShar41}%1
Шарапудинов И.И. О базисности ультрасферических полиномов Якоби в весовом пространстве Лебега с переменным показателем // Математические заметки (Статья представлена в журнал)




\bibitem{ShIIBJWShar4}%2
Шарапудинов И.И. О топологии пространства  $ L^{p(x)}([0,1])$ // Математические заметки. 1979. Т. 26, № 4. С. 613--632.



\bibitem{ShIIBJWPollard1}%3
Pollard H. The mean convergence of orthogonal series // Trans. Amer. Math. Soc. 1947. Vol. 62. N 2. Pp.~387--403.

\bibitem{ShIIBJWPollard2}%4
Pollard H. The mean convergence of orthogonal series. II // Trans. Amer. Math. Soc. 1948. Vol. 63. N 2. Pp.~355 –- 367

\bibitem{ShIIBJWPollard3}%5
 Pollard H. The mean convergence of orthogonal series. III // Duke Math. J. 1949. Vol. 16. N 1. Pp.~189--191

\bibitem{ShIIBJWNewmanRudin}%6
Newman J., Rudin W. Mean convergence of orthogonal series // Proc. Amer. Math. Soc. 1952. Vol. 3. N 2. Pp.~219--222.

\bibitem{ShIIBJWMuckenhoupt1}%7
 Muckenhoupt B. Mean convergence of Jacobi series // Proc. Amer. Math. Soc. 1969. Vol. 23. N 2. Pp.~306--310

\bibitem{ShIIBJWSharap1} %8
Шарапудинов И.И. О базисности системы полиномов Лежандра в пространстве Лебега
 $L^{p(x)}(-1,1)$ с переменным показателем $p(x)$ // Математический сборник. 2000. Vol. 200, № 1. С.~137--160.


\bibitem{ShIIBJWShar1}%9
Шарапудинов И.И. О базисности системы Хаара в пространстве $L^{p(x)}([0,1])$ и принципе локализации в среднем // Математический сборник. 1986. Т. 130(172), № 2(6). С.~275--283.




\bibitem{ShIIBJWShar5}%10
 Шарапудинов И.И. О равномерной ограниченности в $L^p$ $(p=p(x))$ некоторых семейств операторов свертки // Математические заметки. 1996. Т. 59, № 2. С.~291--302



\bibitem{ShIIBJWShar2} %11
Шарапудинов И.И. Некоторые вопросы теории приближения в пространствах $L^{p(x)}$ // Analysis Mathematica. 2007. Vol. 33. N 2. С. 135--153.

\bibitem{ShIIBJWShar3} %12
Шарапудинов И.И. Некоторые вопросы теории приближений в пространствах Лебега с переменным показателем. Владикавказ: Владикавказский научный центр РАН и правительства Северной Осетии-Алании. Итоги науки. Юг России. Серия: Математическая монография. 2012. Вып. 5.



\bibitem{ShIIBJWShar6}%13
 Шарапудинов И.И. Приближение функций в $L^{p(x)}_{2\pi}$ тригонометрическими полиномами // Известия РАН: Серия математическая. 2013. Т. 77, № 2. С. 197--224.

\bibitem{ShIIBJWDiening}%14
 Lars Diening, Petteri Harjulehto, Peter Hast{\"o}, Michael Růžička. Lebesgue and Sobolev Spaces with Variable Exponents. Springer. Lecture Notes in Mathematics. 2017.






\bibitem{ShIIBJWCruz-Uribe}%15
 David V. Cruz-Uribe, Alberto Fiorenza. Variable Lebesgue Spaces: Foundations and Harmonic Analysis. Springer Basel. Applied and Numerical Harmonic Analysis. 2013. Vol. 4968

\bibitem{LapVPZigmund} Зигмунд А. Тригонометрические ряды. Москва: Мир. 1965. Т. 1.
%16
%\bibitem{Zigmund} имеется
%\by А. Зигмунд
%\book Тригонометрические ряды
%\yr 1965
%\serial
%\publ Мир
%\publaddr Москва
%\vol Т.1













\bibitem{ShIIBJWSege}%17
 Сеге Г. Ортогональные многочлены. Москва: Физматгиз. 1962.

%ШЭТН 
\bibitem{SHETN1} Магомед-Касумов М.Г. Базисность системы Хаара в весовых пространствах Лебега с переменным показателем // Владикавказский математический журнал. 2014. Т. 16, № 3. С.~38--46.

%ШИИ LappedVallePussen
\bibitem{LapVPMalvar}
Malvar H.S. Signal processing with lapped transform. Boston{ $\cdot$} London. Artech House. 1992

\bibitem{LapVPZhuk} Жук В.В. Аппроксимация периодических функций. Ленинград. 1982.

\bibitem{LapVPNIK} Никольский С.М. О некоторых методах приближения тригонометрическими суммами // Изв. АН СССР. Сер. матем. 1940. Т. 4, № 6. С. 509--520


%\bibitem{LapVPEFIM} Ефимов А.В. О приближении периодических функций суммами Валле-Пуссена // Изв. АН СССР. Сер. матем. 1959. Т. 23, № 5. С. 737--770.

%\bibitem{LapVPTEl} Теляковский С.А. О приближении дифференцируемых функций линейными средними их рядов Фурье // Изв. АН СССР. Сер. матем. 1960. Т. 24, № 2. С. 213--242.

%\bibitem{LapVPmmg} Магомед-Касумов М.Г. Аппроксимативные свойства классических средних Валле-Пуссена для кусочно гладких функций // Вестник Дагестанского научного центра РАН. 2014. Вып. 54. C. 5--12.

%\bibitem{LapVPShar7} Шарапудинов И.И. Аппроксимативные свойства средних Валле-Пуссена на классах типа Соболева с переменным показателем // Вестник Дагестанского научного центра РАН. 2012. Вып. 45 С. 5--13.

%\bibitem{LapVPShar8} Шарапудинов И.И. Приближение гладких функций в $L_{2\pi}^{p(x)}$ средними Валле-Пуссена // Известия Саратовского университета. Серия: Математика. Механика. Информатика. 2012. Т. 13, вып. 1, часть 1. С. 45--49.

%\bibitem{LapVPShar6} Шарапудинов И.И. Приближение функций в $L^{p(x)}_{2\pi}$ тригонометрическими полиномами // Известия РАН: Серия математическая. 2013. Т. 77, № 2. С. 197--224.



%Гаджимирзаев->
\bibitem{Ram1}{Шарапудинов~И.И.} Многочлены, ортогональные на сетках. Махачкала, Изд-во Даг. гос. пед. ун-та. 1997.		
\bibitem{Ram2}{Бейтмен~Г., Эрдейи~А.} Высшие трансцендентные функции. Т.2. М.: Наука, 1974.		
\bibitem{Ram3}{Никифоров~А.Ф., Суслов~С.К., Уваров~В.Б.} Классические ортогональные многочлены дискретной переменной. М. Наука. 1985.		
\bibitem{Ram5}{Шарапудинов~И.И.} Об асимптотике и весовых оценках полиномов Мейкснера, ортогональных на сетке $\{0,\delta, 2\delta, \ldots\}$ //
Матем. заметки, 62:4 (1997). С.~603--616.



%%%Акниев->
\bibitem{aggshii_article}
{Sharapudinov I.I.} On the best approximation and polynomials of the least quadratic deviation~// Analysis Mathematica. 1983. Т.~9, вып.~3. С.~223--234.
%DOI:10.7868/S0869565214260041.

\bibitem{agg2_bernstein}
{Бернштейн~С.~Н.} О тригонометрическом интерполировании по способу наименьших квадратов~// Докл. АН СССР. 1934. Т.~4. Pp.~1--5.

\bibitem{agg4_erdos}
{Erd{\"o}s~P.} Some theorems and remarks on interpolation~// Acta Sci. Math. (Szeged) Т.~12. Pp.~11--17.

\bibitem{agg7_kalashnikov}
{Калашников~М.~Д.} О полиномах наилучшего (квадратического) приближения в заданной системе точек~// Докл. АН СССР. 1955. Т.~105. С.~634--636.

\bibitem{agg8_krilov}
{Крылов~В.~И.} Сходимость алгебраического интерполирования по корням многочлена Чебышева для абсолютно непрерывных фунций и функций с ограниченным изменением~// Докл. АН СССР. 1956 Т.~107. С.~362--365.

\bibitem{agg9_marcinkiewicz}
{Marcinkiewicz~J.} Quelques remarques sur l'interpolation~// Acta Sci. Math. (Szeged). 1936. Т.~8 С.~127--130.

\bibitem{agg10_marcinkiewicz}
{Marcinkiewicz~J.} Sur la divergence des polyn{\^o}mes d'interpolation~// Acta Sci. Math. (Szeged). 1936. Т.~8. С.~131--135.

\bibitem{agg11_natanson}
{Natanson~I.~P.} On the convergence of trigonometrical interpolation at equidistant knots~// Ann. of Math. 1944. Т.~45. С.~457--471.

%\bibitem{agg12_nikolsky}
%{Никольский~С.~М.} О некоторых методах приближения тригонометрическими суммами~// Изв. АН СССР, серия матем. Т.~4 (1940). С.~509--520.

\bibitem{agg17_turetsky}
{Турецкий~А.~Х.} Теория интерполирования в задачах. Минск.~: Высшейшая школа. 1968. 320~с.

%\bibitem{agg18_zigmund}
%{Зигмунд~А.} Тригонометрические ряды. Т.~1. М.~:
%Мир,1965. 616~с.

\bibitem{aggfiht_tom3}
{Фихтенгольц~Г.~М.} Курс дифференциального и интегрального исчисления. Т.~3. М.~:
ФИЗМАТЛИТ,1969. 656~с.

\bibitem{aggmkasumov_article}
{Магомед-Касумов~М.~Г.} Аппроксимативные свойства средних Валле Пуссена для кусочно гладких функций~// Матем. заметки. Т.~100, вып.~2. С.~229--247.
DOI:10.4213/mzm10588

% Рамазанов

\bibitem{rark1} Алберг Дж.,Нилсон Э., Уолш Дж. Теория сплайнов и ее приложения М.: Мир. 1972.

\bibitem{rark2}Стечкин С.Б.,Субботин Ю.Н. Сплайны в вычислительной математике. М.: Наука. 1976.

\bibitem{rark3}Завьялов Ю.С., Квасов Б.И.,Мирошниченко В.Л. Методы сплайн-функций. М.: Наука. 1980.


\bibitem{rark4}Корнейчук Н.П. Сплайны в теории приближения. М.: Наука. 1984.


\bibitem{rark5}Малоземов В.Н., Певный А.Б. Полиномиальные сплайны. Л.: Изд-во ЛГУ. 1986.

\bibitem{rark6} {Субботин Ю.Н.} Вариации на тему сплайнов // Фундамент. и прикл. матем.
 1997. Т.~3, вып.~4. С.~1043--1058.


\bibitem{rark7}Schaback R. Spezielle rationale Splinefunktionen // J. Approx. Theory. Academic Press, Inc. 1973. V. 7. N 2.
Pp. 281--292.

\bibitem{rark8}Edeoa A.,Gofeb G.,Tefera T. Shape preserving $C^2$ rational cubic spline interpolation //
 American Scientific Research Journal for Engineering, Technology and Sciences. 2015. V. 12. N 1. Pp. 110--122.


\bibitem{rark9}Рамазанов А.-Р.К., Магомедова В.Г. Сплайны по рациональным интерполянтам // Дагестанские электронные математические известия. 2015. Вып. 4. С. 22--31.

\bibitem{rark10} Рамазанов А.-Р.К., Магомедова В.Г. Сплайны по четырехточечным рациональным интерполянтам // Труды Института матем. и механики УрО РАН. 2016. Т. 22, № 4. С. 233--246.

\bibitem{rark11} Рамазанов А.-Р.К., Магомедова В.Г. Сплайны по трехточечным
рациональным интерполянтам с автономными полюсами // Дагестанские
электронные математические известия. 2017. Вып. 7. С. 1-7.

\bibitem{rark12} {Субботин~Ю.Н., Черных~Н.И.} Порядок наилучших сплайн-приближений
некоторых классов функций // Мат. заметки. 1970. Т.~7, вып.~1. С.~31--42.

\bibitem{rark13} Рамазанов А.-Р.К., Магомедова В.Г. Оценки скорости сходимости сплайнов
по трехточечным рациональным интерполянтам для непрерывных и непрерывно дифференцируемых функций // Труды Института матем. и механики УрО РАН. 2017. Т. 23, № 3. С. 224--233.

\bibitem{rark14} {Севастьянов~Е.А.} Кусочно-монотонная аппроксимация и $\Phi$--вариации //
Analysis Math. 1975. Вып.~1. С.~141--164.

\bibitem{rark15} {Lagrange~R.} Sur oscillations d`order superior d`une functions numerique
 // Ann. sci. Escole norm. super. 1965. V.~82, no~2. P.~101--130.

\bibitem{rark16} {Чантурия~З.А.} О равномерной сходимости рядов Фурье // Мат. сб. 1976.
 Т.~100, \No~4. С.~534--554.

\bibitem{rark17} {Whitney~H.} On functions with bounded $n$-th differences //
J. Math. Pures Appl. 1957. V.~6(9), no.~36. P.~67--95.






%%%%%%%%%%%%%%%%%%%%%
%%%%%%%%%%%%%%%%%%%%%
%%%%% ПРОШЛЫЙ ГОД


%\bibitem{atpshii24} Шарапудинов И.И. Приближение функций с переменной
%гладкостью суммами Фурье-Лежандра // Матем. сб. 2000. Т. 191. Вып. 1. С. 143--160.
%
%\bibitem{atpshii25} Шарапудинов И.И. Смешанные ряды по ультрасферическим
%     полиномам и их аппроксимативные свойства //
%     Матем. сб. 2003. Т. 194. Вып. 3. С. 115--148.
%
%\bibitem{atpshii26} Шарапудинов И.И. Аппроксимативные свойства операторов
%     $\mathcal{ Y}_{n+2r}(f)$ и их дискретных аналогов //
%Матем. заметки. 2002. Т.72. Вып. 5. С.765--795.
%
%\bibitem{atpshii27} Шарапудинов И.И. Смешанные ряды по ортогональным
%    полиномам. Теория и приложения. -- Махачкала. Даг. науч. центр РАН. 2004. 276 с.
%
%\bibitem{atpshii28} Шарапудинов И.И. Аппроксимативные свойства смешанных рядов по полиномам
%Лежандра на классах   $W^r$ // Матем. сб. Т. 197. Вып. 3. 135--154.
%
%\bibitem{atpshii29} Шарапудинов И.И. Аппроксимативные свойства средних типа
%Валле-Пуссена частичных сумм смешанного ряда по полиномам
%Лежандра // Матем. заметки. Т. 84. Вып. 3. С. 452--471.
%
%\bibitem{atpshii14}Сеге Г. Ортогональные многочлены. М.: Физматгиз. 1962.
%
%\bibitem{atpshii4} Гопенгауз И.З. К теореме А.Ф. Тимана о приближении
%функций многочленами на конечном отрезке // Матем. заметки. 1967. Т. 1.
%Вып. 2. С. 163--172.
%
%\bibitem{atpshii16} Теляковский С.А. Две теоремы о приближении функций
%алгебраическими многочленами // Матем. сб. 1966. Т. 70. Вып. 2. С. 252--265.
%
%\bibitem{shiimonog} Некоторые вопросы теории приближений в пространствах Лебега с переменным показателем. Владикавказ: ЮМИ ВНЦ РАН и РСО-А. 2012. 267 с.
%
%
%
%\bibitem{vpmshiiShar5} Шарапудинов И.И. О равномерной ограниченности в $L^p$ $(p=p(x))$ некоторых семейств операторов свертки // Математические заметки. 1996. Т. 59. Вып. 2. С. 291--302.
%\bibitem{vpmshiiShar2} Шарапудинов И.И. Некоторые вопросы теории приближения в
%             пространствах $L^{p(x)}$ // Analysis Mathematica. 2007. Vol. 33. N 2. P. 135--153.
%
%
%\bibitem{vpmshiiGuvIsr} Ali  Guven  and Israfilov D.M. Trigonometric approximation in Generalized Lebesgue spaces $L^{p(x)}$ // Journal of Mathematical Inequalities. 2010. Vol. 4. N 2. P. 285--299.
%
%\bibitem{vpmshiiRamAk1} Ramazan Akgun. Polynomial approximation of function in weighted Lebesgue and Smirnov spaces whith nonstandard growth // Georgian Math. J. 2011. N 18. P. 203--235.
%
%\bibitem{vpmshiiRamAk2} Ramazan Akgun. Trigonometric approximation of functions in generalized Lebesgue  spaces with variable exponent // Ukrainian Mathematical Journal. 2011. Vol. 63. N 1. P. 1--26.
%
%
%\bibitem{vpmshiiAkVakh} Ramazan Akgun and Vakhtang Kokilashvili. On converse theorems of trigonometric  approximation in weighted variable exponent Lebesgue // Banach J. Math. Anal. 2011. Vol 5. N 1. P. 70--82.
%
%
%\bibitem{vpmshiiShar7} Шарапудинов И.И. Аппроксимативные свойства средних Валле-Пуссена на классах типа Соболева с переменным показателем // Вестник Дагестанского научного центра РАН. 2012. Вып. 45. С. 5--13.
%
%\bibitem{vpmshiiShar8} Шарапудинов И.И. Приближение гладких функций в  $L_{2\pi}^{p(x)}$ средними Валле-Пуссена // Известия Саратовского университета. Серия: Математика. Механика. Информатика. 2012. Т. 13. Вып. 1. Часть 1. С. 45--49.
%
%
%
%
%\bibitem{vpmshiiChaich} Чайченко C.О. Наилучшее приближение периодических функций в обобщенных пространствах Лебега // Укр. мат. журн. 2012. Т. 64. Вып. 9. С. 1--17.
%
%
%\bibitem{vpmshiiShar6} Шарапудинов И.И. Приближение функций в $L^{p(x)}_{2\pi}$ тригонометрическими полиномами // Известия РАН: Серия математическая. 2013. Т. 77. Вып. 2. С. 197--224.
%
%
%\bibitem{vpmshiiShar4} Шарапудинов И.И. О топологии пространства   $ L^{p(x)}([0,1])$ // Математические заметки. 1979. Т. 26. Вып. 4. С. 613--632.
%
%
%
%
%
%\bibitem{vpmshiiShar1} Шарапудинов И.И. О базисности системы Хаара в пространстве $L^{p(x)}([0,1])$ и принципе локализации в среднем // Математический сборник. 1986. Т. 130(172). Вып. 2(6). С. 275--283.
%
%
%
%\bibitem{Zigmund1} Зигмунд А. Тригонометрические ряды. 1965. М.: Мир. Т. 1.
%
%
%\bibitem{shtn1}
%Kantorovich~L.V. Sur certains developpements suivant les polyn$\hat{o}$mes de la forme de S. Bernstein I, II // C. R. Acad. Sci. URSS. 1930. P.~563--568. P.~595--600.
%\bibitem{shtn2}
%Lorentz~G.G. Bernstein Polynomials. Toronto: University of Toronto Press. 1953.
%\bibitem{shtn5} Натансон~И.П. Конструктивная теория функций. Москва, Ленинград: Государственное издательство технико-теоретической литературы. 1949. 688~c.
%\bibitem{shtn7} Вулих~Б.З. Введение в функциональный анализ. Москва: Наука. 1967. 416~c.




\end{thebibliography}
