
\chapter{Формосохраняющие свойства рациональных сплайн-функций класса $C^1$}%смерджить cо след. чаптером???
\section{Введение}

Пусть на отрезке $[a,b]$ задана сетка узлов
$\Delta: a=x_0<x_1<\dots<x_N=b$ $(N\geqslant 3)$,
в которых определена дискретная функция $f(x)$. При $i=1,2,\dots,N-1$ коэффициенты
рациональной функции
\begin{equation}\label{ark-1.1}
R_i(x)=\alpha_i+\beta_i (x-x_i)+\frac{\gamma_i}{x-g_i}
\end{equation}
с произвольным полюсом $g_i\not \in [x_{i-1},x_{i+1}]$ определим из интерполяционных условий
$R_i(x_j)=f(x_j)$ $(j=i-1,i,i+1)$.

Тогда, используя разделенные разности $f(x_{j-1}, x_j)$ и $f(x_{j-1}, x_j, x_{j+1})$, получим
\begin{equation}
\begin{array}{l}
\alpha_i=f(x_i)-f(x_{i-1}, x_i, x_{i+1})(x_{i-1}-g_i)(x_{i+1}-g_i),\\
\beta_i=f(x_{i-1}, x_{i+1})+f(x_{i-1}, x_i, x_{i+1})(x_i-g_i),\\
\gamma_i=f(x_{i-1}, x_i, x_{i+1})(x_{i-1}-g_i)(x_i-g_i)(x_{i+1}-g_i).
\end{array}\label{ark-1.2}
\end{equation}

Будем считать также $R_0(x)\equiv R_1(x)$, $R_N(x)\equiv R_{N-1}(x)$ и на отрезке $[a,b]$
определим (\cite{ark-12}) непрерывно дифференцируемую рациональную сплайн-функцию
$R_{N,1}(x)=R_{N,1} (x, f, \Delta, g)$, полагая
\begin{equation}\label{ark-1.3}
R_{N,1}(x)=R_i(x)\frac{x-x_{i-1}}{x_i-x_{i-1}}+R_{i-1} (x)\frac{x_i-x}{x_i-x_{i-1}}
\end{equation}
при $x\in[x_{i-1},x_i]$ $(i=1,2,\dots,N)$.

Для исследования поведения сплайн-функции $R_{N,1}(x)$  на всем отрезке $[a,b]$ возникает
необходимость рассмотрения  для интерполянтов $R_i(x)$ полюсов  двух видов, а именно,
полюсы $g_{i-1}$ и $g_i$ соответственно интерполянтов
$R_{i-1} (x)$ и $R_i(x)$ в одном случае должны удовлетворять неравенству $g_{i-1}<g_i$,
 а в другом случае --- неравенству $g_{i-1}>g_i$.

Определим полюсы интерполянтов через параметр $t$ и расстояния между соответствующими узлами
$h_i=x_i-x_{i-1}$ $(i=1, 2,\dots,N)$, полагая при $i=2,3,\dots,N-1$ в первом случае
\begin{equation}\label{ark-1.5}
g_{i-1}(t)=x_{i-1}-th_{i-1},\quad g_i(t)=x_i+th_{i+1},
\end{equation}
 а во втором случае
\begin{equation}\label{ark-1.6}
g_{i-1}(t)=x_{i-1}+th_i,\quad g_i(t)=x_i-th_i.
\end{equation}

Для всего отрезка $[a,b]$ получим систему полюсов
$g(t)=\{g_1(t), g_2(t),\dots,g_{N-1}(t)\}$, в которой могут встречаться, вообще говоря, полюсы
обоих видов.

Ниже будем полагать, что рассматриваемая система полюсов $g(t)$ является согласованной в том
смысле, что в каждой паре соседних полюсов системы оба полюса получаются по формулам
\eqref{ark-1.5} или оба они получаются по формулам \eqref{ark-1.6}.

Будем придерживаться также следующей терминологии. Тройку данных $f(x_{i-1})$, $f(x_i)$, $f(x_{i+1})$
будем называть строго выпуклой вниз (вверх), если соответствующая разделенная разность
$f(x_{i-1}, x_i,x_{i+1})$ больше (меньше) нуля.

Систему всех данных $f(x_0), f(x_1),\dots,f(x_N)$ будем называть строго выпуклой вниз (вверх),
если каждая тройка соседних данных в ней строго выпукла вниз (вверх).

Всюду ниже для отношений разделенных разностей и расстояний между узлами
 будем придерживаться следующих обозначений:
$$
q_i=\frac{f(x_{i-2}, x_{i-1}, x_i)}{f(x_{i-1}, x_i, x_{i+1})}, \quad
Q_i=\max\left\{\frac 2{2q_i-1}, \frac{2q_i}{2-q_i}\right\},
$$
$$
H_i=\max\left\{\left.\frac{h_k}{h_j}\right\vert\,|k-j|=1; k,j\in\{i-1,i,i+1\}\right\}
\quad (i=2,3,\dots,N-1);
$$
$$
T_0=\max_{2\leqslant i\leqslant N-1}\left\{17 H_iQ_i |q_i>0, 3H_i|q_i<0 \right\}.
$$

\section{Основные результаты}

Основные результаты сформулируем в виде следующих двух утверждений.

\begin{theorem}\label{ark-teor1}
Если на отрезке $[a,b]$ задана произвольная сетка узлов $\Delta: a=x_0<x_1<\dots<x_N=b$
$(N\geqslant 3)$, система данных $f(x_0), f(x_1),\dots,f(x_N)$  строго выпукла вниз
{(}вверх{)} и выполняются неравенства $1/2<q_i<2$ $(i=2,3,\dots,N-1)$, то при любом
значении $t\geqslant T_0$ рациональная сплайн-функция $R_{N,1} (x)=R_{N,1}(x, \Delta, f, g(t))$
выпукла вниз {(}вверх{)} на отрезке $[a,b]$.
 \end{theorem}

\begin{theorem}\label{ark-teor2}
Пусть на сетке узлов $\Delta: a=x_0<x_1<\dots<x_N=b$
$(N\geqslant 3)$ система данных $f(x_0), f(x_1),\dots,f(x_N)$  содержит перемены направления
выпуклости и при каждом $i=2,3,\dots,N-1$ выполняется двойное неравенство $1/2<|q_i|<2$ .
Тогда для любого значения $t\geqslant T_0$ рациональная сплайн-функция
 $$R_{N,1}(x)=R_{N,1} (x, f, \Delta, g(t))$$ сохраняет форму выпуклости в следующем смысле{:}

1) если при данном $i=2,3,\dots,N-1$ отношение $q_i>0$, то $R_{N,1}(x)$ выпукла вниз {(}вверх{)} на
отрезке $[x_{i-1}, x_i]$ в соответствии с положительной {(}отрицательной{)} $f(x_{i-1},x_i,x_{i+1})${;}

2) если при данном $i=2,3,\dots,N-1$ отношение $q_i<0$, то $R_{N,1}(x)$ имеет точку перегиба
$z_i$ на интервале $\left(x_{i-1}+ 1/3h_i, x_i-1/3 h_i\right)$, выпукла вверх {(}вниз{)} на
отрезке $[x_{i-1}, z_i]$ и выпукла вниз {(}вверх{)} на отрезке $[z_i, x_i]$
при положительной {(}отрицательной{)} $f(x_{i-1},x_i,x_{i+1})$.
 \end{theorem}

 \chapter{Выпуклая интерполяция рациональными сплайн-функциями класса $C^2$}
\section{Введение}

Вопросы выпуклой интерполяции полиномиальными сплайнами в достаточно полной
форме исследованы в ряде работ различными авторами (см., напр., \cite{ark-1, ark-2, ark-3, ark-4, ark-5} и цитированные
в них источники). Подобные вопросы рассматривались также для рациональных сплайнов специальных
видов, например, в работах \cite{ark-8, ark-9, ark-10, ark-11}.

Ниже вопрос выпуклой (вниз или вверх) интерполяции выпуклых (вниз или вверх, соответственно)
дискретных данных рассматривается  для дважды непрерывно дифференцируемых сплайн-функций, построенных с помощью
 трехточечных рациональных интерполянтов.

Пусть на отрезке $[a,b]$ задана сетка узлов
$\Delta: a=x_0<x_1<\dots<x_N=b$ $(N\geqslant 3)$, в которых определена дискретная функция $f(x)$.
При $i=1,2,\dots, N-1$  коэффициенты рациональной функции
\begin{equation*}
R_i(x)=\alpha_i+\beta_i (x-x_i)+\gamma_i \frac 1{x-g_i},
\end{equation*}
с произвольным полюсом $g_i\not \in [x_{i-1}, x_{i+1}]$ определим из интерполяционных условий
$R_i(x_j)=f(x_j)$ $(j=i-1,i,i+1)$.

Тогда с использованием разделенных разностей первого порядка $f(x_{i-1}, x_i)$ и второго порядка
$\delta_i=f(x_{i-1}, x_i, x_{i+1})$ при $i=1,2,\dots,N-1$ получим
\begin{equation*}
\begin{array}{l}
\alpha_i=f(x_i)-\delta_i(x_{i-1}-g_i)(x_{i+1}-g_i),\\[1ex]
\beta_i=f(x_{i-1}, x_{i+1})+\delta_i(x_i-g_i),\\[1ex]
\gamma_i=\delta_i(x_{i-1}-g_i)(x_i-g_i)(x_{i+1}-g_i).
\end{array}
\end{equation*}

Будем считать, что  $R_0(x)\equiv R_1(x)$, $R_N(x)\equiv R_{N-1}(x)$ (допускаются и другие варианты крайних интерполянтов
$R_0(x)$ и $R_N(x)$).

Для краткости при $i=1,2,\dots,N$ обозначим
\begin{equation*}
A_i(x)=\frac{(x-x_{i-1})^2}{(x-x_{i-1})^2+(x_i-x)^2)},\quad B_i(x)=1-A_i(x)
\end{equation*}
и на отрезке $[a,b]$ определим (\cite{ark-9}) рациональную сплайн-функцию
$R_{N,2}(x)=R_{N,2} (x, f, \Delta,g)$ класса $C^2[a,b]$, полагая
\begin{equation*}
R_{N,2}(x)=R_i(x) A_i(x)+R_{i-1}(x)B_i(x), \quad x\in [x_{i-1}, x_i]\quad (i=1,2,\dots,N).
\end{equation*}

Всюду ниже придерживаемся также обозначений:
$q_i=\delta_{i-1}/\delta_i$ $(i=2,3,\dots,N-1)$; $h_i=x_i-x_{i-1}$ $(i=1,2,\dots,N)$;
$\rho_\Delta=\max\{\left.h_ih_j^{-1}\right\vert |i-j|=1; i,j=1,2,\dots,N\}$;
$\gamma= (2\sqrt 2+1)/(2\sqrt 2+5)$.


\section{Основной результат}

Следующее утверждение дает условия выпуклой интерполяции дискретных данных рациональными
сплайн-функциями $R_{N,2}(x, f, \Delta, g)$ класса $C^2[a,b]$.

\begin{theorem}\label{ark-teor2.1}
Пусть для дискретных данных $f(x_i)$ $(i=0,1,\dots,N)$  на сетке узлов
$\Delta: a=x_0<x_1<\dots<x_N=b$ $(N\geqslant 3)$ выполнены неравенства $\delta_i>0$
(соответственно, $\delta_i<0$) при $i=1,2,\dots,N-1$ и $\gamma<q_i<1/\gamma$ при
$i=2,3,\dots,N-1$.

Тогда существуют полюсы  $g=\{g_1,g_2,\dots,g_{N-1}\}$, для которых рациональная сплайн-функция
$R_{N,2}(x)=R_{N,2} (x, f, \Delta,g)$ выпукла вниз (соответственно, вверх) на отрезке $[a,b]$.
\end{theorem}
