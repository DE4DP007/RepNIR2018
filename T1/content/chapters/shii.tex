
\chapter{Равномерное приближение непрерывных функций средними Валле Пуссена специального ряда}
%\subsubsec{\textbf{3.5.3.} }
Приведенные выше результаты содержат оценки \textit{взвешенного} приближения функций частичными суммами специальных рядов и их линейными средними. Однако интерес представляют и вопросы равномерного (без веса) приближения функций.

В работе \cite{15} изучены аппроксимативные свойства средних Валле-Пуссена специального ряда $V_{n,m}^\alpha(f)=V_{n,m}^{\alpha,1}(f)$ (см. определение \eqref{vp-r}) в случае $\alpha=\frac{1}{2}$ и $n \le qm$, где $q$ --- произвольное положительное фиксированное число, а именно, была получена оценка
\begin{equation}\label{est-Vp05}
|f(x)-V_{n,m}^\frac{1}{2}(f,x)| \le c(q) E_n(f), \; f \in C[-1,1].
\end{equation}

В данном проекте ставилась \hyperlink{task2}{задача} исследовать аппроксимативные свойства $V_{n,m}^\alpha(f)$ для $\frac{1}{2} < \alpha \le \frac{3}{2}$. Нам удалось получить следующие результаты в этом направлении.

Обозначим через $\Lambda_{n,m}^\alpha(x)$ функцию Лебега средних Валле Пуссена $V_{n,m}^\alpha(f)$:
\begin{equation*}
\Lambda_{n,m}^\alpha(x)=
(1-x^2)\int_{-1}^{1}(1-t^2)^{\alpha-1}\Bigl|
\frac{1}{m+1}\sum\limits_{k=n}^{n+m}K^\alpha_{k-2}(x,t)
\Bigr|dt,
\end{equation*}
Имеет место следующее утверждение.
\begin{theorem}\label{Leb-func-vp}
Если $c_1 m \le n \le c_2 m$, то
\begin{equation}
  \Lambda_{n,m}^\alpha(x)\le c(c_1, c_2, \alpha), \quad \frac12 \le \alpha < \frac32, \quad -1 \le x \le 1.
\end{equation}
\end{theorem}

Из \eqref{eq9}, \eqref{eq11} и \eqref{eq13} нетрудно получить равенство
\begin{equation*}
V_{n,m}^\alpha(f,x)=
l_1(f,x)+(1-x^2)\int\limits_{-1}^{1}
\Bigl[
f(t)-l_1(f,x)
\Bigr]
(1-t^2)^{\alpha-1}\frac{1}{m+1}\sum\limits_{k=n}^{n+m}K_{k-2}^\alpha(x,t)dt,
\end{equation*}
из которого выводится следующее неравенство:
\begin{equation*}
|V_{n,m}^\alpha(f,x)| \le (1+\Lambda_{n,m}^\alpha(f,x))\|f\|_{C[-1,1]}.
\end{equation*}
Отсюда и из теоремы \ref{Leb-func-vp} вытекает, что семейство операторов $V_{n,m}^\alpha(f)$, $\frac12\le \alpha < \frac32$, $c_1m \le n\le c_2m$, равномерно ограничено в пространстве $C[-1,1]$.

Далее, используя теорему \ref{Leb-func-vp}, мы можем оценку \eqref{est-Vp05} распространить при условии $c_1 m \le n \le c_2 m$ на случай $\frac{1}{2} < \alpha < \frac{3}{2}$.
\begin{theorem}
Для функций $f \in C[-1,1]$ справедлива следующая оценка остатка при приближении средними Валле Пуссена $V_{n,m}^\alpha(f,x)$:
\begin{equation*}
|f(x)-V_{n,m}^\alpha(f,x)| \le c E_n(f),
\quad c_1 m \le n \le c_2 m, \quad \frac{1}{2} \le \alpha < \frac{3}{2}.
\end{equation*}
\end{theorem}



\chapter{Аппроксимативные свойства операторов $\sigma^{\alpha,r}_n(f,x)$ в весовых пространствах Лебега с переменным показателем}

%\subsubsec{\textbf{3.5.6.} }
Пространства Лебега $L^{p(x)}(E)$ с переменным показателем суммируемости $p(x)$ позволяют более точно описывать свойства функций, имеющих существенно переменное поведение. %Краткое определение этих пространств и связанных с ними пространств Соболева с переменным показателем дано в подпункте \hyperlink{goals-3}{III} пункта 3.4.
Пусть $1\le p(x)$ -- измеримая функция, заданная на $[-1,1]$. Через $L^{p(x)}(-1,1)$ обозначим пространство таких измеримых функций $f(x)$, что $\int_{-1}^1|f(x)|^{p(x)}dx <\infty$. Для $f(x)\in L^{p(x)}(-1,1)$, как это было показано в \cite{20}, можно  ввести норму $\|f\|_{p(\cdot)}=\inf\{\alpha>0:\int_{-1}^1|f(x)/\alpha|^{p(x)}dx\le1\}$. Для целого $r\ge0$
через $W^{r,p(x)}(-1,1)$ обозначим пространство Соболева с переменным показателем $p(x)$, состоящее из функций $f(x)$, непрерывно дифференцируемых на $[-1,1]$ $r-1$-раз, для которых $f^{(r-1)}(x)$ абсолютно непрерывна на $[-1,1]$  и $f^{(r-1)}(x)\in L^{p(x)}(-1,1)$.
Далее, через  $W^{r,p(x)}(-1,1,M)$ обозначим класс функций из $W^{r,p(x)}(-1,1)$, для которых $\|f^{(r)}\|_{p(\cdot)}\le M$.
В последние годы наблюдается заметный всплеск интереса к изучению теории так называемых весовых пространств Лебега с переменным показателем, состоящих из измеримых функций $f(x)$, удовлетворяющих условию $\int\limits_{-1}^{1} |f(x)|^{p(x)} w(x) dx < \infty$, где $w(x)$ -- неотрицательная суммируемая функция (вес), для которых норма определяется равенством $\|f\|_{p(\cdot),w}=\inf\{\alpha>0:\int_{-1}^1|f(x)/\alpha|^{p(x)}w(x)dx\le1\}$. Аналогично безвесовому случаю определяются весовые пространства Соболева $W^{r,p(x)}_w(-1,1)$. Используя методы, разработанные в работах \cite{26, 27, 28, 29, 30, 31, 32}, в настоящем проекте предполагается исследовать вопросы, связанные с приближением функций $g(x)\in W^{r,p(x)}(-1,1)$ частичными суммами $\sigma_n^\alpha(g,x)$. В частности, ставится следующая

В ходе выполнения настоящего проекта нам  удалось установить следующий результат.

Пусть $\mathcal{\hat P}$ обозначает класс переменных показателей $p(x)$, удовлетворяющих следующим условиям:
\begin{enumerate}[1)]
\item\label{report2017-p-cond-1}
$p(x)>1$ для $x\in[-1,1]$;
\item\label{report2017-p-cond-2}
$|p(x)-p(y)|\ln\frac{2}{|x-y|}\le d$ при $x,y\in[-1,1]$;
\item\label{report2017-p-cond-3}
$p(x)$  принимает постоянные значения вблизи концов отрезка $[-1,1]$: $p(x)=p_1$ при $x\in[-1,-1+\delta_1]$ и $p(x)=p_2$ при $[1-\delta_2, 1]$, где $\delta_1$ и $\delta_2$ -- сколь угодно малые положительные числа.
\end{enumerate}
\begin{theorem}\label{report2016-th5}
Если $p(x)\in \mathcal{\hat P}$ и $p(\pm1)\in (\frac43,4)$, то для $g(x)\in W^{r,p(x)}(-1,1,1)$ имеют место оценки
\begin{equation}\label{report2016-42}
|g^{(\nu)}(x)-(\sigma_n^{r,r}(g,x))^{(\nu)}|\le c\left({\sqrt{1-x^2}\over n}\right)^{r-\nu-\frac{1}{p(x)}} ,
\end{equation}
\begin{equation}\label{report2017-29}
\|g^{(r)}-(\sigma_n^{r,r}(g))^{(r)}\|_{p(\cdot)}\le cE_{n-r}(g^{(r)})_{p(\cdot)},
\end{equation}
где $ x\in[-1,1]$, $0\le\nu\le r-1$, $E_m(f)_{p(\cdot)}$ -- наилучшее приближение функции $f\in L^{p(x)}(-1,1) $ алгебраическими полиномами степени $m$.
\end{theorem}


Оценка \eqref{report2016-42} показывает, что применение пространств Соболева  $W^{r,p(x)}$ с переменным показателем $p(x)$ позволяет учитывать существенно переменное поведение производных функции  $g(x)$  при оценке погрешности $|g^{(\nu)}(x)-(\sigma_n^{r,r}(g,x))^{(\nu)}|$ приближения  $g^{(\nu)}(x)$ посредством \linebreak $(\sigma_n^{r,r}(g,x))^{(\nu)}$. Говоря более точно, имеется в виду следующее: если $p(x)>1$, $r\ge1$, $g\in W^{r,p(x)}$, то, оставаясь в шкале пространств Соболева $W^{r,p_0}$ с постоянным показателем $p_0$, мы можем утверждать лишь, что  $g\in W^{r,p_0}$, где $p_0=\min\limits_x p(x)$. Поэтому вместо оценки \eqref{report2016-42} мы будем иметь
\begin{equation}\label{report2017-30}
|g^{(\nu)}(x)-(\sigma_n^{r,r}(g,x))^{(\nu)}|\le c\left({\sqrt{1-x^2}\over n}\right)^{r-\nu-\frac{1}{p_0}} ,
\end{equation}
Но для точки $x$, в которой $p(x)>p_0$, оценка \eqref{report2017-30}  по порядку хуже оценки \eqref{report2016-42}.

Одной из целей настоящего проекта являлась задача об исследовании аппроксимативных свойств операторов
$\sigma_{n}^{\alpha,r}=\sigma_{n}^{\alpha,r}(f,x)$ в весовых пространствах Лебега $L^{p(x)}_w(-1,1)$ с весами вида $w(x)=(1-x^2)^\alpha$.  При этом следует отметить, что основополагающую роль при доказательстве оценок \eqref{report2016-42} и \eqref{report2017-29} в безвесовом случае для функции $g(x)\in W^{r,p(x)}(-1,1,1)$ сыграл тот факт, что если переменный показатель $p(x)$ подчинен вышеуказанным условиям \ref{report2017-p-cond-1} -- \ref{report2017-p-cond-3} и $p(\pm1)\in (\frac43,4)$, то система ортонормированных полиномов Лежандра $\{\hat{P}_n^{0}(x)\}$, как было доказано в работе \cite{report2016-shar31},  является базисом Шаудера в пространстве $L^{p(x)}(-1,1)$. Основная сложность при решении задачи об аппроксимативных свойствах операторов $\sigma_{n}^{\alpha,r}$ в весовых пространствах была связана с отсутствием исследований о базисности ультрасферических полиномов Якоби $\hat{P}^{\alpha}_n(x)$ в весовом пространстве Лебега $L^{p(x)}_w(-1,1)$ с весом вида $w(x)=(1-x^2)^\alpha$.
В этом направлении нами установлены следующие результаты.

\begin{theorem}\label{report2017-jacobi-weighted-basis}
Пусть $\alpha>-1/2$, $\mu=\mu(x)=(1-x^2)^\alpha$, $p\in\mathcal{ \hat P}$,
$$4\frac{\alpha+1}{2\alpha+3}<p(\pm1)<4\frac{\alpha+1}{2\alpha+1}.$$
Тогда система ультрасферических полиномов Якоби $\{\hat{P}^{\alpha,\alpha}_n(x)\}_{n=0}^\infty$ является базисом Шаудера в весовом пространстве Лебега $L^{p(x)}_w(-1,1)$ с весом вида $w(x)=(1-x^2)^\alpha$.
\end{theorem}

Отметим, что указанная теорема была доказана и в более общем случае.
\begin{theorem}
	Пусть $\alpha,\beta>-1/2$, $w=w(x)=(1-x)^\alpha(1+x)^\beta$, $p\in\mathcal{\hat P}$,
	$$
	4\frac{\alpha+1}{2\alpha+3}<p(1)<4\frac{\alpha+1}{2\alpha+1},\quad
	4\frac{\beta+1}{2\beta+3}<p(-1)<4\frac{\beta+1}{2\beta+1},\quad
	$$
	Тогда система полиномов Якоби $\{\hat{P}^{\alpha,\beta}_n(x)\}_{n=0}^\infty$ является базисом Шаудера в весовом пространстве Лебега $L^{p(x)}_w(-1,1)$.
\end{theorem}

Рассмотрим отдельно случай $\alpha=-1/2$, который не входит в теорему \ref{report2017-jacobi-weighted-basis}. Хорошо известно, что сумма Фурье -- Якоби  $S_n^{\alpha,\alpha}(f)$ при $\alpha=-1/2$ представляет собой сумму Фурье по полиномами Чебышева $T_n(x)=\cos(n\arccos x)$ $(n=0,1,\ldots)$. Это обстоятельство позволяет доказать равномерную ограниченность сумм Фурье -- Чебышева $S_n^{-1/2,-1/2}(f)$ в пространстве $L_\mu^{p(x)}([-1,1])$ с $\mu(x)=(1-x^2)^{-\frac12}$ в том случае, когда переменный показатель подчиняется на $[-1,1]$ лишь условиям \ref{report2017-p-cond-1} и \ref{report2017-p-cond-2}.  Причём это условие является в определённом смысле также и необходимым. Чтобы сформулировать соответствующий окончательный результат, введем обозначение. Обозначим через $\mathcal{ P}^\beta$ класс переменных показателей $p(x)>1$, удовлетворяющих на $[-1,1]$ следующему условию:
\begin{equation}\label{report2017-31}
|p(x)-p(y)|\left(\ln\frac{2}{|x-y|}\right)^\beta\le d\quad(\beta, d>0, \quad x,y\in[-1,1]).
\end{equation}

\begin{theorem}\label{report2017-tcheb-weighted-basis}
Пусть  $\mu=\mu(x)=(1-x^2)^{-\frac12}$. Тогда суммы Фурье -- Якоби $S_n^{-\frac12,-\frac12}(f)$ $(n=0,1,\ldots)$
равномерно ограничены в весовом пространстве Лебега $L_\mu^{p(x)}([-1,1])$ с произвольным переменным показателем $p \in\mathcal{ P}^\beta$ тогда и только тогда, когда $\beta\ge1$. Другими словами, если $\beta\ge1$, то найдется такое положительное число $c(\beta,p)$, зависящее только от указанных параметров $\beta$ и $p\in\mathcal{P}^\beta$,     что для произвольной функции $f\in L_\mu^{p(x)}([-1,1])$ имеет место оценка
\begin{equation}\label{report2017-32}
\|S_n^{-\frac12,-\frac12}(f)\|_{p(\cdot),\mu}([-1,1])\le c(\beta,p)\|f\|_{p(\cdot),\mu}([-1,1]).
\end{equation}
Если же $0<\beta<1$, то найдется переменный показатель $p_\beta\in\mathcal{ P}^\beta$ и $f_\beta\in L_\mu^{p_\beta(x)}([-1,1])$,
для которых
\begin{equation}\label{report2017-33}
\|S_n^{-\frac12,-\frac12}(f_\beta)\|_{p_\beta(\cdot),\mu}([-1,1])\to\infty\quad(n\to\infty).
\end{equation}
\end{theorem}

Вернёмся теперь к задаче об аппроксимативных свойствах операторов
$\sigma_{n}^{\alpha,r}=\sigma_{n}^{\alpha,r}(f,x)$ в весовых пространствах Лебега $L^{p(x)}_w(-1,1)$ с весами вида $w(x)=(1-x^2)^\alpha$. Мы будем считать, что функция $f=f(x)$ подчинена двум условиям: i)  существуют производные $f^{(\nu)}(\pm1)$ при $\nu=0,\ldots r-1$; ii) функция $q_r(f)=q_r(f,x)$, определённая равенством \eqref{eq21}, принадлежит весовому пространству Лебега $L^{p(x)}_w(-1,1)$ с весом $w(x)=(1-x^2)^\alpha$, т. е. $q_r(f)\in L^{p(x)}_w(-1,1)$, что, в свою очередь, означает конечность интеграла:
\begin{equation}\label{report2017-35}
\int_{-1}^1|q_r(f,x)|^{p(x)}(1-x^2)^\alpha dx<\infty.
\end{equation}
Пространство всех функций $f=f(x)$, для которых справедливо неравенство \eqref{report2017-35}, обозначим через $\mathcal{F}_{r,\alpha,p(\cdot)}$. Далее, пусть  $H^n$ -- пространство алгебраических полиномов $Q_n$ степени не выше $n$,  $E_n(g)_{p,w}$ -- наилучшее приближение функции $g\in  L^{p(x)}_w(-1,1)$ полиномами $Q_n\in H^n$, т. е.
\begin{equation}\label{report2017-36}
E_n(g)_{p(\cdot),w}=\inf_{Q_n\in H^n}\|g-Q_n\|_{p(\cdot),w}.
\end{equation}

С помощью теоремы \ref{report2017-jacobi-weighted-basis} и \ref{report2017-tcheb-weighted-basis} в ходе выполнения проекта получены следующие результаты.

\begin{theorem}\label{report2017-weighted-spec-est-jacobi}
Пусть $\alpha>-1/2$, $\mu=\mu(x)=(1-x^2)^r$, $w=w(x)=(1-x^2)^\alpha$, $p\in\mathcal{ \hat P}$,
$$4\frac{\alpha+1}{2\alpha+3}<p(\pm1)<4\frac{\alpha+1}{2\alpha+1}.$$
Тогда если $f\in \mathcal{ F}_{r,\alpha,p(\cdot)}$, то имеет место оценка
\begin{equation}\label{report2017-37}
\left\|\frac{f-\sigma_{n}^{\alpha,r}(f)}{\mu}\right\|_{p(\cdot),w}\le c(r,\alpha,p)E_n(q_r(f))_{p(\cdot),w}.
\end{equation}
\end{theorem}

\begin{theorem}\label{report2017-weighted-spec-est-tcheb}
Пусть $\mu=\mu(x)=(1-x^2)^r$, $w=w(x)=(1-x^2)^{-\frac12}$, $p(x) \in \mathcal{ P}^1$. Тогда если $f\in \mathcal{ F}_{r,-\frac12,p(\cdot)}$, то имеет место оценка
\begin{equation}\label{report2017-38}
\left\|\frac{f-\sigma_{n}^{-\frac12,r}(f)}{\mu}\right\|_{p(\cdot),w}\le c(r,p)E_n(q_r(f))_{p(\cdot),w}.
\end{equation}
\end{theorem}