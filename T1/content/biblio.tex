\begin{thebibliography}{111}

%%%%РФФИ

\bibitem{rffi-13}
Шарапудинов И.И. Предельные ультрасферические ряды и их
аппроксимативные свойства // Математические заметки. --- 2013. --- Т. 94. --- № 2. --- С. 295---309.

\bibitem{rffi-14}
Шарапудинов И.И. Некоторые специальные ряды по ультрасферическим полиномам  и их аппроксимативные свойства //
Известия РАН. Серия математическая. --- 2014. --- Т. 78. --- № 5. --- С. 201---224.

\bibitem{rffi-15}
Шарапудинов И.И. Аппроксимативные свойства средних типа Фейера и Валле-Пуссена частичных сумм специального ряда по системе $\{ \sin x\sin kx\}_{k=1}^\infty$, Матем. сб. --- 2015. --- Т. 206. --- № 4. --- С. 131---148.

\bibitem{rffi-20}
Шарапудинов И.И. О топологии пространства
$ L^{p(x)}([0,1])$ // Математические заметки. --- 1979. --- Т. 26. --- № 4. --- С. 613---632.

\bibitem{rffi-26}
Шарапудинов И.И. Приближение функций в $L^{p(x)}_{2\pi}$ тригонометрическими полиномами // Известия РАН: Серия математическая. --- 2013. --- Т.77. --- № 2. --- С. 197---224.




\bibitem{rffi-27}
Шарапудинов И.И. Приближение функций из пространств Лебега и Соболева с переменным показателем суммами Фурье-Хаара // Математический сборник. --- 2014. --- Т. 205. --- № 2. --- С. 145---160.

\bibitem{rffi-28}
Магомед-Касумов М.Г. Особенности поведения частичных сумм Фурье -- Хаара в двоично-иррациональных точках разрыва // Сибирский математический журнал. --- 2013. --- Т. 54. --- № 6. --- С. 1331--1336.




\bibitem{rffi-29}
Магомед-Касумов М.Г. Сходимость	прямоугольных сумм Фурье -- Хаара в	пространствах Лебега с переменным показателем $L^{p(x,y)}$ // Изв. Сарат. ун-та. Нов. сер. Сер. Математика. Механика. Информатика. --- 2013. --- Т. 13. --- № 1(2). --- С. 76---81.




\bibitem{rffi-30}
Магомед-Касумов М.Г. Базисность системы	Хаара в весовых пространствах Лебега с переменным показателем // Владикавказский 	математический журнал. --- 2014. --- Т. 16. --- № 3. --- С. 38---46.




\bibitem{rffi-31}
Магомед-Касумов М.Г. Приближение функций суммами Хаара в весовых пространствах Лебега и Соболева с переменным показателем // Изв. Сарат. ун-та. Нов. сер.	Сер. Математика. Механика. Информатика. --- 2014. --- Т. 14. --- № 3. --- С. 295---304.




\bibitem{rffi-32}
Магомед-Касумов М.Г. Вопросы поточечной сходимости в среднем сумм Фурье и их линейных средних по некоторым ортогональным системам: дис. ... канд. физ-мат. наук. Южный федеральный университет, Ростов-на-Дону. 2015. URL: http://hub.sfedu.ru/diss/announcement/7cb90672-7c69-4d8e-8a21-8ee1ee849beb/ (дата обращения 10.01.2019).


\bibitem{rffi-35}
Atia M.J., Littlejohn L.L., Stewart J. Spectral Theory of $X_1$-Laguerre Polynomials // Advances in Dynamical Systems and Applications. --- 2013. --- ISSN 0973-5321. --- V. 8. --- № 2. --- P. 181---192.




\bibitem{rffi-36}
Eliana X.L. de Andrade, Cleonice F. Bracciali, A. Sri Ranga. Asymptotics of zeros of Jacobi–Sobolev orthogonal polynomials // Journal of Approximation Theory. --- November 2010. --- V. 162. --- Iss. 11. --- P. 1945---1963.




\bibitem{rffi-37}
Area I., Godoy E., Marcellan F., Moreno-Balcazar J.J. $\Delta$-Sobolev orthogonal polynomials of Meixner type: asymptotics and limit relation // Journal of Computational and Applied Mathematics. --- 2005. --- V. 178. --- № 1---2. --- P. 21---36.




\bibitem{rffi-38}
Manuel Alfaro, Teresa E. Perez, Miguel A. Pinnar, M. Luisa Rezola. Sobolev Orthogonal Polynomials: The discrete-continuous case // Methods Appl. Anal. --- 1999. --- V. 6. --- № 4. --- P. 593---616.




\bibitem{rffi-39}
Francisco Marcellan, Yuan Xu. On Sobolev orthogonal polynomials // Expositiones Mathematicae. --- 2015. --- V. 33. --- Iss. 3. --- P. 308---352. ISSN 0723-0869. URL: http://dx.doi.org/10.1016/j.exmath.2014.10.002 (дата обращения 10.01.2019).

\bibitem{rffi-report2016-shar31}
Шарапудинов И.И. О базисности системы полиномов Лежандра в пространстве Лебега $L^{p(x)}(-1,1)$ с переменным показателем $p(x)$ // Математический сборник. --- 2009. --- Т. 200. --- № 1. --- С. 137---160.

\bibitem{rffi-40}
Шарапудинов И. И. Приближение функций с переменной гладкостью суммами Фурье -- Лежандра // Матем. сб. --- 2000. --- Т. 191. --- № 5. 143---160.




\bibitem{rffi-41}
Шарапудинов И.И. Смешанные ряды по ультрасферическим полиномам и их аппроксимативные свойства // Матем. сб. --- 2003. Т. 194. --- № 3. --- С. 115---148.




\bibitem{rffi-42}
Шарапудинов И.И. Аппроксимативные свойства смешанных рядов по полиномам Лежандра на классах $W^r$ // Матем. сб. --- 2006. --- Т. 197. --- № 3. --- С. 135---154.


%%%%%%%%%%%
%%%%% Рамазанов

\bibitem{ark-1}
Schweikert D.G. An interpolation curve using a spline in tension
// J. Math. Phys. --- 1966. --- V. 45. --- P. 312---317.


\bibitem{ark-2}
Miroshnichenko V.L. Convex and monotone spline interpolation
// Constuctive Theory of Function: Proc. Int. Conf. (Varna, 1984). --- Sofia:
Publ. House of Bulgarian Acad. Sci., 1984. --- P. 610--620.

\bibitem{ark-3}
Мирошниченко В.Л. Достаточные условия монотонности и выпуклости
для интерполяционных кубических сплайнов класса $C^2$
// Вычислительные системы: сб. ст. ИМ СО АН СССР. ---
Новосибирск. --- 1990. --- № 137: Приближение сплайнами. --- С. 31---57.

\bibitem{ark-4}
Квасов Б.И. Методы изогеометрической аппроксимации сплайнами. --- М.: Физматлит, 2006. --- 360 c.

\bibitem{ark-5}
Волков Ю.С., Богданов В.В., Мирошниченко В.Л., Шевалдин В.Т.
Формосохраняющая интерполяция кубическими сплайнами // Матем. заметки. --- 2010. --- Т. 88. --- № 6. --- С. 836---844.

\bibitem{ark-6}
Волков Ю.С., Шевалдин В.Т.
Условия формосохранения при интерполяции сплайнами второй степени по Субботину и по Марсдену
// Тр. Ин-та математики и механики УрО РАН. --- 2012. --- Т. 18. --- № 4. --- С. 145---152.

\bibitem{ark-7}
Богданов В.В., Волков Ю.С.
Об условиях формосохранения при интерполяции параболическими сплайнами по Субботину
// Тр. Ин-та математики и механики УрО РАН. --- 2016. --- Т. 22. --- № 4. --- С. 102---113.

\bibitem{ark-8}
Schaback R. Spezielle rationale Splinefunktionen
// J. Approx.Theory. --- 1973. --- V. 7. --- № 2. --- P. 281--292.

\bibitem{ark-9}
Spath H. Spline algorithms for curves and surfaces. --- Winnipeg: Utilitas Mathematica Publ. Inc., 1974. --- 198 p.

\bibitem{ark-10}
Hussain M.Z., Sarfraz M., Shaikh T.S.
Shape preserving rational cubic spline for positive and convex data
// Egyptian Informatics Journal. --- 2011. --- V. 12. --- P. 231---236.

\bibitem{ark-11}
Edeo A., Gofeb G., Tefera T. Shape preserving $C^2$ rational
 cubic spline interpolation // American Scientific Research Journal for Engineering, Technology and Sciences. ---
 2015. --- V. 12. --- № 1. --- P. 110---122.

\bibitem{ark-12}
Рамазанов А.-Р.К., Магомедова В.Г.  Сплайны по рациональным интерполянтам
// Дагестанские электронные математические известия. --- 2015. --- № 4. --- C 22---31.


\bibitem{ark-13}
Рамазанов А.-Р.К., Магомедова В.Г. Оценки скорости сходимости сплайнов
 по трехточечным рациональным интерполянтам
 для непрерывных и непрерывно дифференцируемых функций
// Тр. Ин-та математики и механики УрО РАН. --- 2017. --- Т. 23. --- № 3. --- C. 224---233.


\bibitem{ark-14}
Рамазанов А.-Р.К., Магомедова В.Г.
Сплайны по трехточечным рациональным интерполянтам
// Тр. Матем. центра им. Н.И. Лобачевского. --- Казань, 2017. --- Т. 54. ---  C. 304---306.


\bibitem{ark-15}
Рамазанов А.-Р.К., Магомедова В.Г. Сплайны по трехточечным рациональным
 интерполянтам с автономными полюсами
// Дагестанские электронные математические известия. --- 2017. --- № 7. --- C. 16---28.




%%%%%%%%%%%
%%%%% Рамис


\bibitem{RamNik}{Никифоров А.Ф., Суслов С.К., Уваров В.Б.} Классические ортогональные многочлены дискретной переменной. --- М.: Наука, 1985.

\bibitem{RamBateman} {Bateman H, Erdeyi A.} Higher transcendental functions. --- Vol. 2. McGraw-Hill, New York-Toronto-London, 1953.


\bibitem{RamSharMnog} {Шарапудинов И.И.} Многочлены, ортогональные на сетках. --- Махачкала: Изд-во Даг. гос. пед. ун-та, 1997.


\bibitem{Rampetrazav}{Gadzhimirzaev R.M.} Approximative properties of Fourier–Meixner sums // Пробл. анал. Issues Anal. --- 2018. --- V. 7(25). --- № 1. --- P. 23---40.


\bibitem{Ram3}{Гаджиева З.Д.} Смешанные ряды по полиномам Мейкснера. Кандидатская диссертация - Саратов. Саратовский гос. ун-т. --- 2004.

\bibitem{Ram4}{Гаджимирзаев Р.М.} Приближение функций, заданных на сетке $\{0, \delta, 2\delta, \ldots\}$ суммами Фурье-Мейкснера // Дагестанские электронные математические известия. --- 2017. --- \No\ 7. --- С. 61---65.


\bibitem{RamVMJ} {Гаджимирзаев Р.М.} Аппроксимативные свойства специальных рядов по полиномам Мейкснера // Владикавк. матем. журн. --- 2018. --- Т. 20. --- № 3. --- С. 21---36.

\bibitem{RamSar} {Гаджимирзаев Р.М.} Ряды Фурье по полиномам Мейкснера, ортогональным по Соболеву // Изв. Сарат. ун-та. Нов. сер. Сер. Математика. Механика. Информатика. --- 2016. --- Т. 16. --- № 4. ---. С. 388–--395.














%%%%%%%%%%%
%%%%% АГГ



\bibitem{sobleg-Shar11}
{Шарапудинов И.И.} Приближение функций с переменной гладкостью суммами Фурье Лежандра // Мат. сборник. --- 2000. --- Т. 191(5) --- С. 143---160.


\bibitem{sobleg-Shar12}
{Шарапудинов И.И.} Аппроксимативные свойства операторов $\mathcal{ Y}_{n+2r}(f)$ и их дискретных аналогов // Мат. заметки. --- 2002. --- Т. 72(5). --- С. 765---795.


\bibitem{sobleg-Shar13}
{Шарапудинов И.И.} Смешанные ряды по ортогональным полиномам. --- Махачкала: Издательство Дагестанского научного центра, 2004. --- С. 1---176.


\bibitem{sobleg-Shar15}
{Шарапудинов И.И.}
Аппроксимативные свойства смешанных рядов по полиномам Лежандра на классах $W^r$ // Мат. сборник. --- 2006. --- Т. 97(3). --- С. 135---154.


\bibitem{sobleg-Shar16}
{Шарапудинов И.И.}
Аппроксимативные свойства средних типа Валле-Пуссена частичных сумм смешанных рядов по полиномам Лежандра // Мат. заметки. --- 2008. --- Т. 84(3). --- С. 452---471.


\bibitem{sobleg-Shar17}
{Шарапудинов И.И.}
 Смешанные ряды по ультрасферическим полиномам и их аппроксимативные свойства
// Мат. сборник. --- 2003. --- Т. 194(3). --- С. 115---148.


\bibitem{sobleg-Shar18}
{Шарапудинов И.И., Шарапудинов Т.И.}
 Смешанные ряды по полиномам Якоби и Чебышева и их дискретизация
// Мат. заметки. --- 2010. --- Т. 88(1). --- С. 116---147.


\bibitem{sobleg-sharap3}
{Шарапудинов И.И.}
 Некоторые специальные ряды по ультрасферическим полиномам и их аппроксимативные свойства
// Изв. РАН. Сер. матем. --- 2004. --- Т. 78(5). --- С. 201---224.


\bibitem{sobleg-SHII}
{Шарапудинов И.И.}
 Некоторые специальные ряды по общим полиномам Лагерра и ряды Фурье по полиномам Лагерра, ортогональным по Соболеву
// Дагестанские электронные математические известия. --- 2015. --- № 4.


\bibitem{sobleg-Sege}
{Сеге Г.} Ортогональные многочлены. --- Москва: Физматгиз, 1962.


\bibitem{sobleg-Gasper}
{Gasper G.}
 Positiviti and special function
// Theory and appl.Spec.Funct. Edited by Richard A.Askey. --- 1975. --- Pp. 375--433.


\bibitem{sobleg-KwonLittl1}
{Kwon K.H., Littlejohn L.L.}
 The orthogonality of the Laguerre polynomials $\{L_n^{(-k)}(x)\}$ for positive integers $k$
// Ann. Numer. Anal. Iss. --- 1995. --- № 2. P. 289---303.


\bibitem{sobleg-KwonLittl2}
{Kwon K.H., Littlejohn L.L.}
 Sobolev orthogonal polynomials and second-order differential equations
// Rocky Mountain J. Math. --- 1998. --- V. 28. --- P. 547---594.


\bibitem{sobleg-MarcelAlfaroRezola}
{Marcellan F., Alfaro M., Rezola M.L.} Orthogonal polynomials on Sobolev spaces: old and new directions
// Journal of Computational and Applied Mathematics. --- 1993. --- V. 48. --- P. 113---131.


\bibitem{sobleg-IserKoch}
{ Iserles A., Koch P.E., Norsett S.P., Sanz-Serna J.M.}
 On polynomials  orthogonal  with respect  to certain Sobolev inner products
// J. Approx. Theory. --- 1991. --- № 65. --- P. 151---175.


\bibitem{sobleg-Meijer}
{Meijer H.G.} Laguerre polynimials generalized to a certain.
Laguerre polynimials generalized to a certain discrete Sobolev inner product space
// J. Approx. Theory. --- 1993. --- № 73. --- Pp. 1---16.


\bibitem{sobleg-Lopez1995}
{Lopez G. Marcellan F. Vanassche W.}
 Relative Asymptotics for Polynomials Orthogonal with Respect to a Discrete Sobolev Inner-Product
// Constr. Approx. --- 1995. --- № 11:1. --- Pp. 107---137.


\bibitem{sobleg-MarcelXu}
{Marcellan F., Xu Y.}
 On Sobolev orthogonal polynomials
// Expositiones Mathematicae. --- 2015. --- № 33(3). --- Pp. 308---352.


\bibitem{sobleg-Shar2016}
И.И. Шарапудинов
 Системы функций, ортогональные по Соболеву, порожденные ортогональными функциями
// Материалы 18-й международной Саратовской зимней школы «Современные проблемы теории функций и их приложения». --- 2016. --- С. 329---332.


\bibitem{sobleg-Tref1}
{Trefethen  L.N.} Spectral methods in Matlab. --- Fhiladelphia: SIAM, 2000.


\bibitem{sobleg-Tref2}
{Trefethen  L.N.}
Finite difference and spectral methods for ordinary and partial differential equation. --- Cornell University, 1996.


\bibitem{sobleg-SolDmEg}
{Солодовников В.В., Дмитриев А.Н., Егупов Н.Д.}
Спектральные методы расчета и проектирования систем управления. --- Москва: Машиностроение, 1986.


\bibitem{sobleg-Pash}
{Пашковский С.} Вычислительные применения многочленов и рядов Чебышева. --- Москва: Наука, 1983. --- С. 143---160.


\bibitem{sobleg-MMG2016}
{Магомед-Касумов М.Г.}
 Приближенное решение обыкновенных дифференциальных уравнений с использованием смешанных рядов по системе Хаара
// Материалы 18-й международной Саратовской зимней школы «Современные проблемы теории функций и их приложения». --- 2016. --- С. 176---178.


\bibitem{sobleg-Gonchar1975}
{Гончар А.А.}
 О сходимости аппроксимаций Паде для некоторых классов мероморфных функций
// Мат. сборник. --- 1975. --- Т. 97(139). --- № 4(8). --- С. 607---629.


\bibitem{sobleg-TEL}
{Теляковский С.А.}
 Две теоремы о приближении функций алгебраическими многочленами
// Мат. сборник. --- 1966. --- № 70(2). --- С. 252---265.


\bibitem{sobleg-GOP}
{Гопенгауз И.З.}
 К теореме А. Ф. Тимана о приближении функций многочленами на конечном отрезке
// Мат. заметки. --- 1967. --- № 1(2). --- С. 163---172.


\bibitem{sobleg-OSK}
{Осколков К.И.}
 К неравенству Лебега в равномерной метрике и на множестве полной меры
// Мат.  заметки. --- 1975. --- № 18(4). --- С. 515---526.


\bibitem{sobleg-sharap1}
{Sharapudinov I.I.}
 On the best approximation and polinomial of the least quadratic deviation
// Analysis Mathematica. --- 1983. --- № 9(3). --- P. 223---234.


\bibitem{sobleg-sharap2}
{Шарапудинов И.И.}
 О наилучшем приближении и суммах Фурье-Якоби
//Мат. заметки. --- 1983. --- № 34(5). --- С. 651---661.


\bibitem{sobleg-Timan}
{Тиман А.Ф.} Теория приближения функций действительного переменного. --- Москва: Физматгиз, 1960.
%%
%% end lit. section2-sobleg


\bibitem{laplas-Shar13}
{Шарапудинов И.И.}
Смешанные ряды по ортогональным полиномам // --- Махачкала: Издательство Дагестанского научного центра, 2004. --- С. 1---176.


\bibitem{laplas-Shar14}
{Шарапудинов И.И.}
Смешанные ряды по полиномам Чебышева, ортогональным на равномерной сетке // Математические заметки. --- 2005. --- Т. 78. --- № 3. --- С. 442–--465.


\bibitem{laplas-Shar11}
{Шарапудинов И.И.}
Специальные ряды по полиномам Лагерра и их аппроксимативные свойства // Сибирский математический журнал. --- 2017. --- Т. 58. № 2. --- С. 440---467.


\bibitem{laplas-Meijer}
{Meijer H.G.}
Laguerre polynimials generalized to a certain discrete Sobolev inner product space // J. Approx. Theory. --- 1993. --- V. 73. --- P. 1---16.


\bibitem{laplas-MarcelXu}
{Marcellan F., Yuan Xu}
ON SOBOLEV ORTHOGONAL POLYNOMIALS. arXiv: 6249v1 [math.C.A] 25 Mar 2014. Pp. 1--40


\bibitem{laplas-MarcelVanash}
{Lopez G., Marcellan F., Van Assche W.}
Relative asymptotics for polynomials orthogonal with respect to a discrete Sobolev inner product // Constr. Approx. --- 1995. --- V. 11. --- Iss. 1. --- P. 107---137.


\bibitem{laplas-Sege}
{Сеге Г.}
Ортогональные многочлены. --- Москва: Физматгиз, 1962.


\bibitem{laplas-AskeyWaiger}
{Askey R., Wainger S.}
Mean convergence of expansions in Laguerre and Hermite series // Amer. J. Mathem. --- 1965. --- V. 87. --- P. 698---708.


\bibitem{laplas-DitPrud}
{Диткин В.А., Прудников А.П.}
Операционное исчисление. --- Москва: Высшая школа, 1975.


\bibitem{laplas-KrylovSkob}
{Крылов В.И., Скобля Н.С.}
Методы приближенного преобразования Фурье и обращения преобразования Лапласа. --- Москва: Наука. 1974.
%%
%% end lit. section2-laplas


\bibitem{equ102-Shar20}
Шарапудинов И.И. Ортогональные  по Соболеву системы, порожденные ортогональными функциями // Изв. РАН. Сер. Математическая. --- 2018. --- Т. 82. (Принята к печати)


\bibitem{equ102-Tref1}
{Trefethen  L.N.}
Spectral methods in Matlab. --- Fhiladelphia: SIAM, 2000.


\bibitem{equ102-Arush2014}
{Арушанян О.Б., Волченскова Н.И., Залеткин С.Ф.}
Применение рядов Чебышева для интегрирования обыкновенных дифференциальных уравнений // Сиб. электрон. матем. изв. --- 2014. --- № 11. --- С. 517---531.


\bibitem{equ102-Lukom2016}
{Лукомский Д.С., Терехин П.А.}
Применение системы Хаара к численному решению задачи Коши для линейного дифференциального уравнения первого порядка // Материалы 18-й международной Саратовской зимней школы «Современные проблемы теории функций и их приложения». Саратов. ООО «Издательство «Научная книга». --- 2016. --- С. 171---173.


\bibitem{equ102-DiffUr2017}
{Шарапудинов И.И., Магомед-Касумов М.Г.}
О представлении решения задачи Коши  рядом Фурье  по полиномам, ортогональным по  Соболеву, порожденным многочленами Лагерра. Дифференциальные уравнения. --- 2017 (принята к печати)


\bibitem{equ102-KashSaak}
{Кашин Б.С., Саакян А.А.}
Ортогональные ряды. Москва: АФЦ, 1999.


\bibitem{equ102-Shar19}
{Шарапудинов И.И., Муратова Г.Н.}
Некоторые свойства r-кратно интегрированных рядов по системе Хаара // Изв. Сарат. ун-та. Нов. сер. Сер. Математика. Механика. Информатика. --- 2009. --- Т. 9. № 1. --- С. 68---76.


\bibitem{equ102-Faber}
{G. Faber}
Ober die Orthogonalfunktionen des Herrn Haar // Jahresber. Deutsch. Math. Verein. --- 1910. --- V. 19. --- P. 104---112.


\bibitem{equ102-Shar25}
{Шарапудинов И.И.}
Асимптотические свойства полиномов, ортогональных по Соболеву, порожденных полиномами Якоби // Дагестанские электронные математические известия. --- 2016. --- № 6. --- С. 1---24.


\bibitem{equ130-Shar13}
{Шарапудинов И.И.}
Смешанные ряды по ортогональным полиномам. --- Махачкала: Издательство Дагестанского научного центра, 2004. --- С. 1--176.


\bibitem{equ130-Tref1}
{Trefethen L.N.}
Spectral methods in Matlab. --- Philadelphia: SIAM, 2000.


\bibitem{equ130-DiffUr2017}
{Шарапудинов И.И., Магомед-Касумов М.Г.}
О представлении решения задачи Коши  рядом Фурье  по полиномам, ортогональным по  Соболеву, порожденным многочленами Лагерра // Дифференциальные уравнения. --- 2017. (принята к печати)








\bibitem{equ130-Sege}
Сеге Г. Ортогональные многочлены. --- Москва. Физматгиз, --- 1962.
%%
%% end lit. section2-equ130


\bibitem{sobcheb_urav-Arush2010}
{Арушанян О.Б., Волченскова Н.И., Залеткин С.Ф.}
Приближенное решение обыкновенных дифференциальных уравнений с использованием рядов Чебышева // Сиб. электрон. матем. изв. --- 1983. --- № 7. --- С. 122---131


\bibitem{sobcheb_urav-Arush2013}
{Арушанян О.Б., Волченскова Н.И., Залеткин С.Ф.}
 Метод решения задачи Коши для обыкновенных дифференциальных уравнений с использованием рядов Чебышeва // Выч. мет. программирование. --- 2013. --- № 14:2. --- С. 203---214.


\bibitem{sobcheb_urav-fiht2}
{Фихтенгольц Г.М.}
Курс дифференциального и интегрального исчисления // Физматлит. Москва. --- 2001. --- Т. 2. --- С. 810.


\bibitem{ramis-Ram1}
{Шарапудинов И.И.} Многочлены, ортогональные на сетках. Махачкала, Изд-во Даг. гос. пед. ун-та. --- 1997.		


\bibitem{ramis-shGadj}
{Шарапудинов И.И., Гаджиева З.Д.}
Полиномы, ортогональные по Соболеву, порожденные многочленами Мейкснера // Изв. Сарат. ун-та. Нов. сер. Сер. Математика. Механика. Информатика. ---
2016. --- Т.16. --- № 3. --- С. 310---321.


\bibitem{ramis-shGadjGadjMir}
{Шарапудинов И.И., Гаджиева З.Д., Гаджимирзаев Р.М.}
Разностные уравнения и полиномы, ортогональные по Соболеву, порожденные многочленами Мейкснера //
Владикавказский Мат. журнал. --- 2017. --- Т.19. --- № 2. --- С. 58---72.


\bibitem{ramis-Shar9}
{Шарапудинов И.И.}
Приближение дискретных функций и многочлены Чебышева, ортогональные на равномерной сетке //
Мат. заметки. --- 2000. --- Т. 67. № 3. --- С. 460---470.


\bibitem{ramis-SharT1}
{Шарапудинов~Т.И.}
Аппроксимативные свойства смешанных рядов по полиномам Чебышева, ортогональным на равномерной сетке //
Вестник Дагестанского научного центра РАН. --- 2007. --- Т. 29. --- С. 12---23.


\bibitem{ramis-SharII}
{Шарапудинов И.И.}
Системы функций, ортогональных по Соболеву, порожденные ортогональными функциями //
Современные проблемы теории функций и их приложения. // Материалы 18-й международной Саратовской зимней школы. --- 2016. --- С. 329---332.


\bibitem{ramis-Gadz12}
{Fernandez L., Teresa E. Perez, Miguel A. Pinar, Xu Y.} Weighted Sobolev orthogonal polynomials on the unit ball // Journal of Approximation Theory. --- 2013. --- № 171. --- P. 84---104.


\bibitem{ramis-Gadz13}
{Antonia M. Delgado, Fernandez L., Doron S. Lubinsky, Teresa E. Perez, Miguel A. Pinar.} Sobolev orthogonal polynomials on the unit ball via outward normal derivatives // Journal of Mathematical Analysis and Applications. --- 2016. --- V. 440. --- № 2. --- P. 716---740.


\bibitem{ramis-Gadz14}
{Fernandez L., Marcellan F., Teresa E. Perez, Miguel A. Pinar, Xu Y.} Sobolev orthogonal polynomials on product domains // Journal of Computational and Applied Mathematics. --- 2015. --- № 284. --- P. 202---215.


\bibitem{ramis-Gadz16}
{Шарапудинов И.И., Шарапудинов Т.И.} Полиномы, ортогональные по Соболеву, порожденные многочленами Чебышева, ортогональными на сетке // Изв. вузов. Матем. --- 2017. --- № 8. --- С. 67---79.


\bibitem{ramis-Gadz17}
{Гаджимирзаев Р.М.} Ряды Фурье по полиномам Мейкснера, ортогональным по Соболеву // Изв. Сарат. ун-та. Нов. сер. Сер. Математика. Механика. Информатика. --- 2016. --- № 16:4. --- С. 388---395.


\bibitem{ramis-Gadz1}
{Шарапудинов И.И., Гаджиева З.Д., Гаджимирзаев Р.М.} Системы функций, ортогональных относительно скалярных произведений типа Соболева с дискретными массами, порожденных классическими ортогональными системами // Дагестанские электронные математические известия. --- 2016. --- № 6. --- С. 31---60.


\bibitem{ramis-Gadz3}
{Сеге Г.} Ортогональные многочлены. --- М.: Физматгиз, 1962.
%%
%% end lit. ramis

\bibitem{charlier-Shar3}
Meijer H.G. Laguerre polynomials generalized to a certain discrete Sobolev inner product space // J. Approx. Theory. --- 1993. --- V. 73. --- Iss. 1. --- P. 1---16.

\bibitem{charlier-Shar8}
Шарапудинов И.И. Смешанные ряды по ортогональным полиномам. --- Махачкала: Изд-во ДНЦ РАН, 2004.

\bibitem{charlier-Shar10}
Бейтмен Г., Эрдейи А. Высшие трансцендентные функции. Том 2. --- Москва: Наука, 1974.

\bibitem{charlier-Shar11}
Ширяев А.Н. Вероятность-1. --- Москва. Изд-во МЦНМО, 2007.

%AGG

\bibitem{2_bernstein}
\textit{Bernshtein S.N.}
On trigonometric interpolation by the method of least squares
%O trigonometricheskom interpolirovanii po sposobu naimen'shih kvadratov
 // Dokl. Akad. Nauk USSR. --- 1934. --- V. 4 --- P. 1---5. (in Russian)

\bibitem{4_erdos}
\textit{Erd{\"o}s~P.} Some theorems and remarks on interpolation // Acta Sci. Math. (Szeged) --- 1950. --- V. 12. --- P. 11---17.

%		\bibitem{fiht-1969}
%		\textit{Fikhtengol'ts~G.~M.} Course of differential and integral calculus Vol.~3. Moscow.~:
%		FIZMATLIT,1969. 656~p. (in Russian)

\bibitem{7_kalashnikov}
\textit{Kalashnikov M.D.}
On polynomials of best (quadratic) approximation on a given system of points.
%O polinomah nailuchshego (kvadraticheskogo) priblizheniya v zadannoy sisteme tochek
 // Dokl. Akad. Nauk USSR. --- 1955. --- V. 105. --- P. 634---636. (in Russian)

\bibitem{8_krilov}
\textit{Krilov V.I.}
Convergence of algebraic interpolation with respect to the roots of a Chebyshev polynomial for absolutely continuous functions and functions with bounded variation.
%Shodimost algebraicheskogo interpolirovaniya po kornyam mnogochlena Chebisheva dlya absolutno neprerivnih funkciy i funkciy s ogranichennim izmeneniyem
 // Dokl. Akad. Nauk USSR. --- 1956. --- V. 107. --- P. 362---365. (in Russian)

\bibitem{9_marcinkiewicz}
\textit{Marcinkiewicz J.} Quelques remarques sur l'interpolation // Acta Sci. Math. (Szeged) --- 1936. --- V. 8. --- P. 127---130. (in French)

\bibitem{10_marcinkiewicz}
\textit{Marcinkiewicz J.} Sur la divergence des polyn{\^o}mes d'interpolation // Acta Sci. Math. (Szeged) --- 1936. --- V. 8. --- P. 131---135. (in French)

\bibitem{11_natanson}
\textit{Natanson I.P.} On the Convergence of Trigonometrical Interpolation at Equi-Distant Knots. // Annals of Mathematics, Second Series, --- 1944. --- V. 45. --- \No\ 3. --- P. 457---471. DOI:10.2307/1969188.
%On the convergence of trigonometrical interpolation at equi-distant knots~// Ann. Math. Vol.~45 (1944). pp.~457--471.

\bibitem{12_nikolsky}
\textit{Nikol'skii S.M.} On some methods of approximation by trigonometric sums. // Mathematics of the USSR - Izvestiya. --- 1940. --- V. 4. --- P. 509---520. (in Russian)

\bibitem{17_turetsky}
\textit{Turetskiy A.H.} Interpolation theory in exercises. --- Minsk: Vissheyshaya shkola, 1968. --- 320 p. (in Russian).

\bibitem{18_zigmund}
\textit{Zygmund A.} Trigonometric Series. Vol 1. --- Cambridge:
Cambridge University Press, 1959. --- 747 p.	

\bibitem{akniyev}
\textit{Akniyev G.G.} Discrete least squares approximation of piecewise-linear functions by trigonometric polynomials // Issues Anal. --- 2017. --- V. 6 (24), --- Iss. 2. --- P. 3---24. DOI: 10.15393/j3.art.2017.4070.

\bibitem{shii-1983}
\textit{Sharapudinov I.I.} On the best approximation and polynomials of the least quadratic deviation // Anal. Math. --- 1983. --- V. 9. --- Iss. 3. --- P. 223---234.

\bibitem{shii-2017}
\textit{Sharapudinov I.I.} Overlapping transformations for approximation of continuous functions by means of repeated mean Valle Poussin~// Daghestan Electronic Mathematical Reports. --- 2017. Iss. 8. --- P. 70---92.

\bibitem{mkasumov_disc_funcs}
\textit{Magomed-Kasumov M.G.}
Approximation of piecewise smooth functions by the trigonometric Fourier series //
Materials of XIX International Saratov Winter School "Contemporary Problems of Function Theory and Their Applications". --- 2018. --- P. 190---193.
%		\bibitem{mkasumov}
%		\textit{Magomed-Kasumov~M.~G.} Approximation properties of de la Valle-Poussin means for piecewise smooth functions~// Mat. Zametki. Vol.~100, Issue~2, 2016. pp.~229--244.
%		DOI:10.1134/S000143461607018X

\bibitem{courant}
\textit{Courant R.} Differential and Integral Calculus Vol. 1. --- NewJersey:
Wiley-Interscience, 1988. --- 704 p.



\end{thebibliography}
