\begin{thebibliography}{111}
\bibitem{rffi-13}
Шарапудинов И.И. Предельные ультрасферические ряды и их
аппроксимативные свойства // Математические заметки. --- 2013. --- Т. 94. --- № 2. --- С. 295---309.



\bibitem{rffi-14}
Шарапудинов И.И. Некоторые специальные ряды по ультрасферическим полиномам  и их аппроксимативные свойства //
Известия РАН. Серия математическая. --- 2014. --- Т. 78. --- № 5. --- С. 201---224.



\bibitem{rffi-15}
Шарапудинов И.И. Аппроксимативные свойства средних типа Фейера и Валле-Пуссена частичных сумм специального ряда по системе $\{ \sin x\sin kx\}_{k=1}^\infty$, Матем. сб. --- 2015. --- Т. 206. --- № 4. --- С. 131---148.



\bibitem{laplas-Shar11}
{Шарапудинов И.И.}
Специальные ряды по полиномам Лагерра и их аппроксимативные свойства // Сибирский математический журнал. --- 2017. --- Т. 58. № 2. --- С. 440---467.




\bibitem{RamVMJ}
{Гаджимирзаев Р.М.} Аппроксимативные свойства специальных рядов по полиномам Мейкснера // Владикавк. матем. журн. --- 2018. --- Т. 20. --- № 3. --- С. 21---36.



\bibitem{ark-1}
Schweikert D.G. An interpolation curve using a spline in tension
// J. Math. Phys. --- 1966. --- V. 45. --- P. 312---317.




\bibitem{ark-2}
Miroshnichenko V.L. Convex and monotone spline interpolation
// Constuctive Theory of Function: Proc. Int. Conf. (Varna, 1984). --- Sofia:
Publ. House of Bulgarian Acad. Sci., 1984. --- P. 610--620.



\bibitem{ark-3}
Мирошниченко В.Л. Достаточные условия монотонности и выпуклости
для интерполяционных кубических сплайнов класса $C^2$
// Вычислительные системы: сб. ст. ИМ СО АН СССР. ---
Новосибирск. --- 1990. --- № 137: Приближение сплайнами. --- С. 31---57.



\bibitem{ark-4}
Квасов Б.И. Методы изогеометрической аппроксимации сплайнами. --- М.: Физматлит, 2006. --- 360 c.



\bibitem{ark-5}
Волков Ю.С., Богданов В.В., Мирошниченко В.Л., Шевалдин В.Т.
Формосохраняющая интерполяция кубическими сплайнами // Матем. заметки. --- 2010. --- Т. 88. --- № 6. --- С. 836---844.



\bibitem{ark-6}
Волков Ю.С., Шевалдин В.Т.
Условия формосохранения при интерполяции сплайнами второй степени по Субботину и по Марсдену
// Тр. Ин-та математики и механики УрО РАН. --- 2012. --- Т. 18. --- № 4. --- С. 145---152.



\bibitem{ark-7}
Богданов В.В., Волков Ю.С.
Об условиях формосохранения при интерполяции параболическими сплайнами по Субботину
// Тр. Ин-та математики и механики УрО РАН. --- 2016. --- Т. 22. --- № 4. --- С. 102---113.



\bibitem{ark-8}
Schaback R. Spezielle rationale Splinefunktionen
// J. Approx.Theory. --- 1973. --- V. 7. --- № 2. --- P. 281--292.



\bibitem{ark-9}
Spath H. Spline algorithms for curves and surfaces. --- Winnipeg: Utilitas Mathematica Publ. Inc., 1974. --- 198 p.



\bibitem{ark-10}
Hussain M.Z., Sarfraz M., Shaikh T.S.
Shape preserving rational cubic spline for positive and convex data
// Egyptian Informatics Journal. --- 2011. --- V. 12. --- P. 231---236.



\bibitem{ark-11}
Edeo A., Gofeb G., Tefera T. Shape preserving $C^2$ rational
 cubic spline interpolation // American Scientific Research Journal for Engineering, Technology and Sciences. ---
 2015. --- V. 12. --- № 1. --- P. 110---122.



\bibitem{rffi-20}
Шарапудинов И.И. О топологии пространства
$ L^{p(x)}([0,1])$ // Математические заметки. --- 1979. --- Т. 26. --- № 4. --- С. 613---632.



\bibitem{rffi-26}
Шарапудинов И.И. Приближение функций в $L^{p(x)}_{2\pi}$ тригонометрическими полиномами // Известия РАН: Серия математическая. --- 2013. --- Т. 77. --- № 2. --- С. 197---224.






\bibitem{rffi-27}
Шарапудинов И.И. Приближение функций из пространств Лебега и Соболева с переменным показателем суммами Фурье-Хаара // Математический сборник. --- 2014. --- Т. 205. --- № 2. --- С. 145---160.



\bibitem{rffi-28}
Магомед-Касумов М.Г. Особенности поведения частичных сумм Фурье -- Хаара в двоично-иррациональных точках разрыва // Сибирский математический журнал. --- 2013. --- Т. 54. --- № 6. --- С. 1331--1336.






\bibitem{rffi-29}
Магомед-Касумов М.Г. Сходимость	прямоугольных сумм Фурье -- Хаара в	пространствах Лебега с переменным показателем $L^{p(x,y)}$ // Изв. Сарат. ун-та. Нов. сер. Сер. Математика. Механика. Информатика. --- 2013. --- Т. 13. --- № 1(2). --- С. 76---81.






\bibitem{rffi-30}
Магомед-Касумов М.Г. Базисность системы	Хаара в весовых пространствах Лебега с переменным показателем // Владикавказский 	математический журнал. --- 2014. --- Т. 16. --- № 3. --- С. 38---46.






\bibitem{rffi-31}
Магомед-Касумов М.Г. Приближение функций суммами Хаара в весовых пространствах Лебега и Соболева с переменным показателем // Изв. Сарат. ун-та. Нов. сер.	Сер. Математика. Механика. Информатика. --- 2014. --- Т. 14. --- № 3. --- С. 295---304.






\bibitem{rffi-35}
Atia M.J., Littlejohn L.L., Stewart J. Spectral Theory of $X_1$-Laguerre Polynomials // Advances in Dynamical Systems and Applications. --- 2013. --- ISSN 0973-5321. --- V. 8. --- № 2. --- P. 181---192.






\bibitem{rffi-36}
Eliana X.L. de Andrade, Cleonice F. Bracciali, A. Sri Ranga. Asymptotics of zeros of Jacobi–Sobolev orthogonal polynomials // Journal of Approximation Theory. --- November 2010. --- V. 162. --- Iss. 11. --- P. 1945---1963.






\bibitem{rffi-37}
Area I., Godoy E., Marcellan F., Moreno-Balcazar J.J. $\Delta$-Sobolev orthogonal polynomials of Meixner type: asymptotics and limit relation // Journal of Computational and Applied Mathematics. --- 2005. --- V. 178. --- № 1---2. --- P. 21---36.






\bibitem{rffi-38}
Manuel Alfaro, Teresa E. Perez, Miguel A. Pinnar, M. Luisa Rezola. Sobolev Orthogonal Polynomials: The discrete-continuous case // Methods Appl. Anal. --- 1999. --- V. 6. --- № 4. --- P. 593---616.






\bibitem{rffi-39}
Francisco Marcellan, Yuan Xu. On Sobolev orthogonal polynomials // Expositiones Mathematicae. --- 2015. --- V. 33. --- Iss. 3. --- P. 308---352. ISSN 0723-0869. URL: http://dx.doi.org/10.1016/j.exmath.2014.10.002 (дата обращения 10.01.2019).



\bibitem{rffi-40}
Шарапудинов И. И. Приближение функций с переменной гладкостью суммами Фурье -- Лежандра // Матем. сб. --- 2000. --- Т. 191. --- № 5. 143---160.






\bibitem{rffi-41}
Шарапудинов И.И. Смешанные ряды по ультрасферическим полиномам и их аппроксимативные свойства // Матем. сб. --- 2003. Т. 194. --- № 3. --- С. 115---148.






\bibitem{rffi-42}
Шарапудинов И.И. Аппроксимативные свойства смешанных рядов по полиномам Лежандра на классах $W^r$ // Матем. сб. --- 2006. --- Т. 197. --- № 3. --- С. 135---154.


%%%%%%%%%%%
%%%%% Рамазанов



\bibitem{rffi-32}
Магомед-Касумов М.Г. Вопросы поточечной сходимости в среднем сумм Фурье и их линейных средних по некоторым ортогональным системам: дис. ... канд. физ-мат. наук. Южный федеральный университет, Ростов-на-Дону. 2015. URL: http://hub.sfedu.ru/diss/announcement/7cb90672-7c69-4d8e-8a21-8ee1ee849beb/ (дата обращения 10.01.2019).




\bibitem{rffi-report2016-shar31}
Шарапудинов И.И. О базисности системы полиномов Лежандра в пространстве Лебега $L^{p(x)}(-1,1)$ с переменным показателем $p(x)$ // Математический сборник. --- 2009. --- Т. 200. --- № 1. --- С. 137---160.



\bibitem{ark-12}
Рамазанов А.-Р.К., Магомедова В.Г.  Сплайны по рациональным интерполянтам
// Дагестанские электронные математические известия. --- 2015. --- № 4. --- C 22---31.




\bibitem{RamSharMnog}
{Шарапудинов И.И.} Многочлены, ортогональные на сетках. --- Махачкала: Изд-во Даг. гос. пед. ун-та, 1997.




\bibitem{Rampetrazav}
{Gadzhimirzaev R.M.} Approximative properties of Fourier–Meixner sums // Пробл. анал. Issues Anal. --- 2018. --- V. 7(25). --- № 1. --- P. 23---40.




\bibitem{Ram3}
{Гаджиева З.Д.} Смешанные ряды по полиномам Мейкснера. Кандидатская диссертация - Саратов. Саратовский гос. ун-т. --- 2004.



\bibitem{RamNik}
{Никифоров А.Ф., Суслов С.К., Уваров В.Б.} Классические ортогональные многочлены дискретной переменной. --- М.: Наука, 1985.



\bibitem{RamBateman}
{Bateman H, Erdeyi A.} Higher transcendental functions. --- Vol. 2. McGraw-Hill, New York-Toronto-London, 1953.




\bibitem{Ram4}
{Гаджимирзаев Р.М.} Приближение функций, заданных на сетке $\{0, \delta, 2\delta, \ldots\}$ суммами Фурье-Мейкснера // Дагестанские электронные математические известия. --- 2017. --- \No\ 7. --- С. 61---65.




\bibitem{RamSar}
{Гаджимирзаев Р.М.} Ряды Фурье по полиномам Мейкснера, ортогональным по Соболеву // Изв. Сарат. ун-та. Нов. сер. Сер. Математика. Механика. Информатика. --- 2016. --- Т. 16. --- № 4. ---. С. 388–--395.














%%%%%%%%%%%
%%%%% АГГ





\bibitem{2_bernstein}
\textit{Bernshtein S.N.}
On trigonometric interpolation by the method of least squares
%O trigonometricheskom interpolirovanii po sposobu naimen'shih kvadratov
 // Dokl. Akad. Nauk USSR. --- 1934. --- V. 4 --- P. 1---5. (in Russian)



\bibitem{4_erdos}
\textit{Erd{\"o}s~P.} Some theorems and remarks on interpolation // Acta Sci. Math. (Szeged) --- 1950. --- V. 12. --- P. 11---17.

%		\bibitem{fiht-1969}
%		\textit{Fikhtengol'ts~G.~M.} Course of differential and integral calculus Vol.~3. Moscow.~:
%		FIZMATLIT,1969. 656~p. (in Russian)



\bibitem{7_kalashnikov}
\textit{Kalashnikov M.D.}
On polynomials of best (quadratic) approximation on a given system of points.
%O polinomah nailuchshego (kvadraticheskogo) priblizheniya v zadannoy sisteme tochek
 // Dokl. Akad. Nauk USSR. --- 1955. --- V. 105. --- P. 634---636. (in Russian)



\bibitem{8_krilov}
\textit{Krilov V.I.}
Convergence of algebraic interpolation with respect to the roots of a Chebyshev polynomial for absolutely continuous functions and functions with bounded variation.
%Shodimost algebraicheskogo interpolirovaniya po kornyam mnogochlena Chebisheva dlya absolutno neprerivnih funkciy i funkciy s ogranichennim izmeneniyem
 // Dokl. Akad. Nauk USSR. --- 1956. --- V. 107. --- P. 362---365. (in Russian)



\bibitem{9_marcinkiewicz}
\textit{Marcinkiewicz J.} Quelques remarques sur l'interpolation // Acta Sci. Math. (Szeged) --- 1936. --- V. 8. --- P. 127---130. (in French)



\bibitem{10_marcinkiewicz}
\textit{Marcinkiewicz J.} Sur la divergence des polyn{\^o}mes d'interpolation // Acta Sci. Math. (Szeged) --- 1936. --- V. 8. --- P. 131---135. (in French)



\bibitem{11_natanson}
\textit{Natanson I.P.} On the Convergence of Trigonometrical Interpolation at Equi-Distant Knots. // Annals of Mathematics, Second Series, --- 1944. --- V. 45. --- \No\ 3. --- P. 457---471. DOI:10.2307/1969188.
%On the convergence of trigonometrical interpolation at equi-distant knots~// Ann. Math. Vol.~45 (1944). pp.~457--471.



\bibitem{12_nikolsky}
\textit{Nikol'skii S.M.} On some methods of approximation by trigonometric sums. // Mathematics of the USSR - Izvestiya. --- 1940. --- V. 4. --- P. 509---520. (in Russian)



\bibitem{18_zigmund}
\textit{Zygmund A.} Trigonometric Series. Vol 1. --- Cambridge:
Cambridge University Press, 1959. --- 747 p.	



\bibitem{17_turetsky}
\textit{Turetskiy A.H.} Interpolation theory in exercises. --- Minsk: Vissheyshaya shkola, 1968. --- 320 p. (in Russian).



\bibitem{akniyev}
\textit{Akniyev G.G.} Discrete least squares approximation of piecewise-linear functions by trigonometric polynomials // Issues Anal. --- 2017. --- V. 6 (24), --- Iss. 2. --- P. 3---24. DOI: 10.15393/j3.art.2017.4070.



\bibitem{shii-1983}
\textit{Sharapudinov I.I.} On the best approximation and polynomials of the least quadratic deviation // Anal. Math. --- 1983. --- V. 9. --- Iss. 3. --- P. 223---234.



\bibitem{mkasumov_disc_funcs}
\textit{Magomed-Kasumov M.G.}
Approximation of piecewise smooth functions by the trigonometric Fourier series //
Materials of XIX International Saratov Winter School "Contemporary Problems of Function Theory and Their Applications". --- 2018. --- P. 190---193.
%		\bibitem{mkasumov}
%		\textit{Magomed-Kasumov~M.~G.} Approximation properties of de la Valle-Poussin means for piecewise smooth functions~// Mat. Zametki. Vol.~100, Issue~2, 2016. pp.~229--244.
%		DOI:10.1134/S000143461607018X



\bibitem{ark-13}
Рамазанов А.-Р.К., Магомедова В.Г. Оценки скорости сходимости сплайнов
 по трехточечным рациональным интерполянтам
 для непрерывных и непрерывно дифференцируемых функций
// Тр. Ин-та математики и механики УрО РАН. --- 2017. --- Т. 23. --- № 3. --- C. 224---233.




\bibitem{ark-14}
Рамазанов А.-Р.К., Магомедова В.Г.
Сплайны по трехточечным рациональным интерполянтам
// Тр. Матем. центра им. Н.И. Лобачевского. --- Казань, 2017. --- Т. 54. ---  C. 304---306.




\bibitem{ark-15}
Рамазанов А.-Р.К., Магомедова В.Г. Сплайны по трехточечным рациональным
 интерполянтам с автономными полюсами
// Дагестанские электронные математические известия. --- 2017. --- № 7. --- C. 16---28.




%%%%%%%%%%%
%%%%% Рамис



\end{thebibliography}
