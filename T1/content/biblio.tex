\begin{thebibliography}{111}

%%%%РФФИ

\bibitem{rffi-13}
Шарапудинов И.И. Предельные ультрасферические ряды и их
аппроксимативные свойства // Математические заметки. 2013. Т. 94. Вып. 2. 295–309.

\bibitem{rffi-14}
Шарапудинов И.И. Некоторые специальные ряды по ультрасферическим полиномам  и их аппроксимативные свойства //
Известия РАН. Серия математическая, 78:5 (2014),  201-224.

\bibitem{rffi-15}
Шарапудинов И.И. Аппроксимативные свойства средних типа Фейера и Валле-Пуссена частичных сумм специального ряда по системе $\{ \sin x\sin kx\}_{k=1}^\infty$, Матем. сб., 206:4 (2015), 131-148.

\bibitem{rffi-20}
Шарапудинов И.И. О топологии пространства
$ L^{p(x)}([0,1])$ // Математические заметки.1979. Т. 26. Вып. 4. С.~613-632.

\bibitem{rffi-26}
Шарапудинов И.И. Приближение функций в $L^{p(x)}_{2\pi}$ тригонометрическими полиномами // Известия РАН: Серия математическая. 2013. Т.77. \textnumero~2. С.~197-224.




\bibitem{rffi-27}
Шарапудинов И.И. Приближение функций из пространств Лебега и Соболева с переменным показателем суммами Фурье-Хаара // Математический сборник. 2014. Т. 205. \textnumero2. С.~145-160.

\bibitem{rffi-28}
Магомед-Касумов~М.Г. Особенности поведения частичных сумм Фурье -- Хаара в двоично-иррациональных точках разрыва~//	Сибирский математический журнал. 2013. Т.~54, \textnumero~6. {С.}~1331--1336.




\bibitem{rffi-29}
Магомед-Касумов~М.Г. Сходимость	прямоугольных сумм Фурье -- Хаара в	пространствах Лебега с переменным показателем $L^{p(x,y)}$~// Изв. Сарат. ун-та. Нов.	сер. Сер. Математика. Механика.	Информатика. 2013. Т.~13, {\textnumero}~1(2). {С.}~76--81.




\bibitem{rffi-30}
Магомед-Касумов~М.Г. Базисность системы	Хаара в весовых пространствах Лебега с переменным показателем~// Владикавказский 	математический журнал. 2014. Т.~16, {\textnumero}~3. {С.}~38--46.




\bibitem{rffi-31}
Магомед-Касумов~М.Г. Приближение функций суммами Хаара в весовых пространствах Лебега и Соболева с переменным показателем~// Изв. Сарат. ун-та. Нов. сер.	Сер. Математика. Механика. Информатика. 2014. Т.~14, {\textnumero}~3. {С.}~295--304.




\bibitem{rffi-32}
Магомед-Касумов М.Г. Вопросы поточечной сходимости в среднем сумм Фурье и их линейных средних по некоторым ортогональным системам: дис. ... канд. физ-мат. наук. Южный федеральный университет, Ростов-на-Дону, 2015.	http://hub.sfedu.ru/diss/announcement/7cb90672-7c69-4d8e-8a21-8ee1ee849beb/.


\bibitem{rffi-report2016-shar31}
Шарапудинов И.\,И. О базисности системы полиномов Лежандра в пространстве Лебега $L^{p(x)}(-1,1)$ с переменным показателем $p(x)$// Математический сборник, 200:1 (2009), 137 -- 160.


%%%%%%%%%%%
%%%%% Рамазанов

\bibitem{ark-1}
Schweikert~D.G. An interpolation curve using a spline in tension
//~J. Math. Phys. 1966. Vol.~45. pp.~312--317.


\bibitem{ark-2}
Miroshnichenko~V.L. Convex and monotone spline interpolation
//~Constuctive Theory of Function: Proc. Int. Conf. (Varna, 1984).  Sofia:
Publ. House of Bulgarian Acad. Sci., 1984. pp.~610--620.

\bibitem{ark-3}
Мирошниченко~В.Л. Достаточные условия монотонности и выпуклости
для интерполяционных кубических сплайнов класса $C^2$
//~Вычислительные системы: сб. ст. / ИМ СО АН СССР.
Новосибирск, 1990. Вып.~137: Приближение сплайнами. С.~31--57.

\bibitem{ark-4}
Квасов~Б.И. Методы изогеометрической аппроксимации сплайнами.
 М.: Физматлит, 2006. 360~c.

\bibitem{ark-5}
Волков~Ю.С., Богданов~В.В., Мирошниченко~В.\,Л., Шевалдин~В.\,Т.
Формосохраняющая интерполяция кубическими сплайнами //~Матем.~заметки. 2010. Т.~88, \No~6. С.~836--844.

\bibitem{ark-6}
Волков~Ю.С., Шевалдин~В.Т.
Условия формосохранения при интерполяции сплайнами второй степени по Субботину и по Марсдену
//~Тр. Ин-та математики и механики УрО РАН. 2012. Т.~18, \No~4. С.~145--152.

\bibitem{ark-7}
Богданов~В.В., Волков~Ю.С.
Об условиях формосохранения при интерполяции параболическими сплайнами по Субботину
//~Тр. Ин-та математики и механики УрО РАН. 2016. Т.~22, \No~4. С.~102--113.

\bibitem{ark-8}
Schaback~R. Spezielle rationale Splinefunktionen
//~J. Approx.Theory.  1973. Vol.~7, no.~2. pp.~281--292.

\bibitem{ark-9}
Spath~H. Spline algorithms for curves and surfaces
//~Winnipeg: Utilitas Mathematica Publ. Inc., 1974. 198~p.

\bibitem{ark-10}
Hussain~M.Z., Sarfraz~M., Shaikh~T.S.
Shape preserving rational cubic spline for positive and convex data
//~Egyptian Informatics Journal. 2011. Vol.~12. pp.~231--236.

\bibitem{ark-11}
Edeo~A., Gofeb~G., Tefera~T. Shape preserving $C^2$ rational
 cubic spline interpolation //~American Scientific Research Journal for Engineering, Technology and Sciences.
 2015. Vol.~12, no.~1. pp.~110--122.

\bibitem{ark-12}
Рамазанов~А.-Р.К., Магомедова~В.Г.  Сплайны по рациональным интерполянтам
//~Дагестанские электронные математические известия. 2015. Вып.~4. C.~22--31.


\bibitem{ark-13}
Рамазанов~А.-Р.К., Магомедова~В.Г. Оценки скорости сходимости сплайнов
 по трехточечным рациональным интерполянтам
 для непрерывных и непрерывно дифференцируемых функций
//~Тр. Ин-та математики и механики УрО РАН. 2017. Т.~23, \No~3. C.~224--233.


\bibitem{ark-14}
Рамазанов~А.-Р.К., Магомедова~В.Г.
Сплайны по трехточечным рациональным интерполянтам
//~ Тр. Матем. центра им. Н.И. Лобачевского. Казань, 2017. Т.~54. C.~304--306.


\bibitem{ark-15}
Рамазанов~А.-Р.К., Магомедова~В.Г. Сплайны по трехточечным рациональным
 интерполянтам с автономными полюсами
//~Дагестанские электронные математические известия. 2017. Вып.~7. C.~16--28.




%%%%%%%%%%%
%%%%% Рамис


\bibitem{RamNik}{Никифоров А.Ф., Суслов С.К., Уваров В.Б.} Классические ортогональные многочлены дискретной переменной. М.: Наука. 1985.

\bibitem{RamBateman} {Bateman H, Erdeyi A.} Higher transcendental functions. Vol. 2. McGraw-Hill, New York-Toronto-London, 1953.


\bibitem{RamSharMnog} {Шарапудинов И.И.} Многочлены, ортогональные на сетках. Махачкала: Изд-во Даг. гос. пед. ун-та. 1997.


\bibitem{Rampetrazav}{Gadzhimirzaev R.M.} Approximative properties of Fourier–Meixner sums // Пробл. анал. Issues Anal., 7(25):1, 2018. Pp. 23--40.


\bibitem{Ram3}{Гаджиева З.Д.} Смешанные ряды по полиномам Мейкснера. Кандидатская диссертация - Саратов. Саратовский гос. ун-т. 2004.

\bibitem{Ram4}{Гаджимирзаев Р.М.} Приближение функций, заданных на сетке $\{0, \delta, 2\delta, \ldots\}$ суммами Фурье-Мейкснера // Дагестанские электронные математические известия, вып. 7, 2017. С. 61--65.


\bibitem{RamVMJ} {Гаджимирзаев Р.М.} Аппроксимативные свойства специальных рядов по полиномам Мейкснера // Владикавк. матем. журн., 20:3, 2018. С. 21--36.

\bibitem{RamSar} {Гаджимирзаев Р.М.} Ряды Фурье по полиномам Мейкснера, ортогональным по Соболеву // Изв. Сарат. ун-та. Нов. сер. Сер. Математика. Механика. Информатика, 16:4, 2016. С. 388–-395.














%%%%%%%%%%%
%%%%% АГГ



\bibitem{sobleg-Shar11}
{Шарапудинов И.И.} Приближение функций с переменной гладкостью суммами Фурье Лежандра // Мат. сборник,
191(5), 2000. С. 143--160.


\bibitem{sobleg-Shar12}
{Шарапудинов И.И.} Аппроксимативные свойства операторов $\mathcal{ Y}_{n+2r}(f)$ и их дискретных аналогов // Мат. заметки, 72(5), 2002. С.765--795.


\bibitem{sobleg-Shar13}
{Шарапудинов И.И.} Смешанные ряды по ортогональным полиномам. Издательство Дагестанского научного центра.
Махачкала 2004. С.1--176.


\bibitem{sobleg-Shar15}
{Шарапудинов И.И.}
Аппроксимативные свойства смешанных рядов по полиномам Лежандра на классах $W^r$ //
Мат. сборник, 97(3), 2006. С. 135--154.


\bibitem{sobleg-Shar16}
{Шарапудинов И.И.}
Аппроксимативные свойства средних типа Валле-Пуссена частичных сумм смешанных рядов по полиномам Лежандра // Мат. заметки, 84(3), 2008. С.452--471.


\bibitem{sobleg-Shar17}
{Шарапудинов И.И.}
 Смешанные ряды по ультрасферическим полиномам и их аппроксимативные свойства
// Мат. сборник, 194(3), 2003. С. 115--148.


\bibitem{sobleg-Shar18}
{Шарапудинов И.И., Шарапудинов Т.И.}
 Смешанные ряды по полиномам Якоби и Чебышева и их дискретизация
// Мат. заметки, 88(1), 2010. С. 116--147.


\bibitem{sobleg-sharap3}
{Шарапудинов И.И.}
 Некоторые специальные ряды по ультрасферическим полиномам и их аппроксимативные свойства
// Изв. РАН. Сер. матем. 78(5), 2014. С. 201--224.


\bibitem{sobleg-SHII}
{Шарапудинов И.И.}
 Некоторые специальные ряды по общим полиномам Лагерра и ряды Фурье по полиномам Лагерра, ортогональным по Соболеву
// Дагестанские электронные математические известия. 2015. Вып. 4.


\bibitem{sobleg-Sege}
{Сеге Г.} Ортогональные многочлены. Физматгиз. Москва. 1962.


\bibitem{sobleg-Gasper}
{Gasper G.}
 Positiviti and special function
// Theory and appl.Spec.Funct. Edited by Richard A.Askey. 1975. Pp. 375--433.


\bibitem{sobleg-KwonLittl1}
{Kwon K.H., Littlejohn L.L.}
 The orthogonality of the Laguerre polynomials $\{L_n^{(-k)}(x)\}$ for positive integers $k$
// Ann. Numer. Anal. Iss. 2. 1995. Pp. 289--303.


\bibitem{sobleg-KwonLittl2}
{Kwon K.H., Littlejohn L.L.}
 Sobolev orthogonal polynomials and second-order differential equations
// Rocky Mountain J. Math. Vol. 28. 1998. Pp. 547--594.


\bibitem{sobleg-MarcelAlfaroRezola}
{Marcellan F. , Alfaro M., Rezola M.L.} Orthogonal polynomials on Sobolev spaces: old and new directions
// Journal of Computational and Applied Mathematics. Vol. 48. 1993. Pp. 113--131.


\bibitem{sobleg-IserKoch}
{ Iserles A., Koch P.E., Norsett S.P., Sanz-Serna J.M.}
 On polynomials  orthogonal  with respect  to certain Sobolev inner products
// J. Approx. Theory, 65. 1991. Pp. 151--175.


\bibitem{sobleg-Meijer}
{Meijer H.G.} Laguerre polynimials generalized to a certain.
Laguerre polynimials generalized to a certain discrete Sobolev inner product space
// J. Approx. Theory, 73. 1993. Pp. 1--16.


\bibitem{sobleg-Lopez1995}
{Lopez G. Marcellan F. Vanassche W.}
 Relative Asymptotics for Polynomials Orthogonal with Respect to a Discrete Sobolev Inner-Product
// Constr. Approx. 11:1. 1995. Pp. 107--137.


\bibitem{sobleg-MarcelXu}
{Marcellan F., Xu Y.}
 On Sobolev orthogonal polynomials
// Expositiones Mathematicae, 33(3). 2015. Pp. 308--352.


\bibitem{sobleg-Shar2016}
И.И. Шарапудинов
 Системы функций, ортогональные по Соболеву, порожденные ортогональными функциями
// Материалы 18-й международной Саратовской зимней школы «Современные проблемы теории функций и их приложения». 2016. С. 329--332.


\bibitem{sobleg-Tref1}
{Trefethen  L.N.} Spectral methods in Matlab. Fhiladelphia. SIAM. 2000.


\bibitem{sobleg-Tref2}
{Trefethen  L.N.}
Finite difference and spectral methods for ordinary and partial differential equation. Cornell University. 1996.


\bibitem{sobleg-SolDmEg}
{Солодовников В.В., Дмитриев А.Н., Егупов Н.Д.}
Спектральные методы расчета и проектирования систем управления. Машиностроение. Москва. 1986.


\bibitem{sobleg-Pash}
{Пашковский С.} Вычислительные применения многочленов и рядов Чебышева. Наука. Москва. 1983. С. 143--160.


\bibitem{sobleg-MMG2016}
{Магомед-Касумов М.Г.}
 Приближенное решение обыкновенных дифференциальных уравнений с использованием смешанных рядов по системе Хаара
// Материалы 18-й международной Саратовской зимней школы «Современные проблемы теории функций и их приложения». 2016. С. 176--178.


\bibitem{sobleg-Gonchar1975}
{Гончар А.А.}
 О сходимости аппроксимаций Паде для некоторых классов мероморфных функций
// Мат. сборник, 97(139):4(8), 1975. С. 607--629.


\bibitem{sobleg-TEL}
{Теляковский С.А.}
 Две теоремы о приближении функций алгебраическими многочленами
// Мат. сборник, 70(2), 1966. С.252--265.


\bibitem{sobleg-GOP}
{Гопенгауз И.З.}
 К теореме А. Ф. Тимана о приближении функций многочленами на конечном отрезке
// Мат. заметки, 1(2), 1967. С. 163--172.


\bibitem{sobleg-OSK}
{Осколков К.И.}
 К неравенству Лебега в равномерной метрике и на множестве полной меры
// Мат.  заметки, 18(4), 1975. С. 515--526.


\bibitem{sobleg-sharap1}
{Sharapudinov I.I.}
 On the best approximation and polinomial of the least quadratic deviation
// Analysis Mathematica, 9(3), 1983. Pp. 223--234.


\bibitem{sobleg-sharap2}
{Шарапудинов И.И.}
 О наилучшем приближении и суммах Фурье-Якоби
//Мат. заметки, 34(5), 1983. С. 651--661.


\bibitem{sobleg-Timan}
{Тиман А.Ф.} Теория приближения функций действительного переменного. Физматгиз, Москва. 1960.
%%
%% end lit. section2-sobleg


\bibitem{laplas-Shar13}
{Шарапудинов И.И.}
Смешанные ряды по ортогональным полиномам // Издательство Дагестанского научного центра. Махачкала. 2004. Стр. 1--176.


\bibitem{laplas-Shar14}
{Шарапудинов И.И.}
Смешанные ряды по полиномам Чебышева, ортогональным на равномерной сетке // Математические заметки. 2005. Т. 78. Вып. 3. Стр. 442–-465.


\bibitem{laplas-Shar11}
{Шарапудинов И.И.}
Специальные ряды по полиномам Лагерра и их аппроксимативные свойства // Сибирский математический журнал. 2017. Т. 58. Вып. 2. Стр. 440--467.


\bibitem{laplas-Meijer}
{Meijer H.G.}
Laguerre polynimials generalized to a certain discrete Sobolev inner product space // J. Approx. Theory. 1993. Vol. 73. Pp. 1--16.


\bibitem{laplas-MarcelXu}
{Marcellan F., Yuan Xu}
ON SOBOLEV ORTHOGONAL POLYNOMIALS. arXiv: 6249v1 [math.C.A] 25 Mar 2014. Pp. 1--40


\bibitem{laplas-MarcelVanash}
{Lopez G., Marcellan F., Van Assche W.}
Relative asymptotics for polynomials orthogonal with respect to a discrete Sobolev inner product // Constr. Approx. 1995. Vol. 11. Issue 1. Pp. 107--137.


\bibitem{laplas-Sege}
{Сеге Г.}
Ортогональные многочлены. Москва. Физматгиз. 1962.


\bibitem{laplas-AskeyWaiger}
{Askey R., Wainger S.}
Mean convergence of expansions in Laguerre and Hermite series // Amer. J. Mathem. 1965. Vol. 87. Pp. 698--708.


\bibitem{laplas-DitPrud}
{Диткин В.А., Прудников А.П.}
Операционное исчисление. Москва. Высшая школа. 1975.


\bibitem{laplas-KrylovSkob}
{Крылов В.И., Скобля Н.С.}
Методы приближенного преобразования Фурье и обращения преобразования Лапласа. Москва. Наука. 1974.
%%
%% end lit. section2-laplas


\bibitem{equ102-Shar20}
Шарапудинов И.И. Ортогональные  по Соболеву системы, порожденные ортогональными функциями // Изв. РАН. Сер. Математическая. 2018. Том. 82. (Принята к печати)


\bibitem{equ102-Tref1}
{Trefethen  L.N.}
Spectral methods in Matlab. Fhiladelphia. SIAM. 2000.


\bibitem{equ102-Arush2014}
{Арушанян О.Б., Волченскова Н.И., Залеткин С.Ф.}
Применение рядов Чебышева для интегрирования обыкновенных дифференциальных уравнений // Сиб. электрон. матем. изв. 2014. Вып. 11. Стр. 517--531.


\bibitem{equ102-Lukom2016}
{Лукомский Д.С., Терехин П.А.}
Применение системы Хаара к численному решению задачи Коши для линейного дифференциального уравнения первого порядка // Материалы 18-й международной Саратовской зимней школы «Современные проблемы теории функций и их приложения». Саратов. ООО «Издательство «Научная книга». 2016. Стр. 171--173.


\bibitem{equ102-DiffUr2017}
{Шарапудинов И.И., Магомед-Касумов М.Г.}
О представлении решения задачи Коши  рядом Фурье  по полиномам, ортогональным по  Соболеву, порожденным многочленами Лагерра. Дифференциальные уравнения. 2017 (принята к печати)


\bibitem{equ102-KashSaak}
{Кашин Б.С., Саакян А.А.}
Ортогональные ряды. Москва. АФЦ 1999.


\bibitem{equ102-Shar19}
{Шарапудинов И.И., Муратова Г.Н.}
Некоторые свойства r-кратно интегрированных рядов по системе Хаара // Изв. Сарат. ун-та. Нов. сер. Сер. Математика. Механика. Информатика. 2009. Т. 9. Вып. 1. Стр. 68 -- 76


\bibitem{equ102-Faber}
{G. Faber}
Ober die Orthogonalfunktionen des Herrn Haar // Jahresber. Deutsch. Math. Verein. 1910. Vol. 19. Pp. 104--112.


\bibitem{equ102-Shar25}
{Шарапудинов И.И.}
Асимптотические свойства полиномов, ортогональных по Соболеву, порожденных полиномами Якоби // Дагестанские электронные математические известия. 2016. Вып. 6.	Стр. 1–-24.


\bibitem{equ130-Shar13}
{Шарапудинов И.И.}
Смешанные ряды по ортогональным полиномам. Издательство Дагестанского научного центра. Махачкала. 2004. С. 1--176.


\bibitem{equ130-Tref1}
{Trefethen L.N.}
Spectral methods in Matlab. SIAM. Philadelphia. 2000.


\bibitem{equ130-DiffUr2017}
{Шарапудинов И.И., Магомед-Касумов М.Г.}
О представлении решения задачи Коши  рядом Фурье  по полиномам, ортогональным по  Соболеву, порожденным многочленами Лагерра // Дифференциальные уравнения. 2017. (принята к печати)








\bibitem{equ130-Sege}
Сеге Г. Ортогональные многочлены. Москва. Физматгиз. 1962.
%%
%% end lit. section2-equ130


\bibitem{sobcheb_urav-Arush2010}
{Арушанян О.Б., Волченскова Н.И., Залеткин С.Ф.}
Приближенное решение обыкновенных дифференциальных уравнений с использованием рядов Чебышева // Сиб. электрон. матем. 1983. изв. Вып. 7. Стр. 122–-131


\bibitem{sobcheb_urav-Arush2013}
{Арушанян О.Б., Волченскова Н.И., Залеткин С.Ф.}
 Метод решения задачи Коши для обыкновенных дифференциальных уравнений с использованием рядов Чебышeва // Выч. мет. программирование. 2013. Вып. 14:2. Стр. 203-214.


\bibitem{sobcheb_urav-fiht2}
{Фихтенгольц Г.М.}
Курс дифференциального и интегрального исчисления // Физматлит. Москва. 2001. Т. 2. Стр. 810.


\bibitem{ramis-Ram1}
{Шарапудинов~И.И.} Многочлены, ортогональные на сетках. Махачкала, Изд-во Даг. гос. пед. ун-та. 1997.		


\bibitem{ramis-shGadj}
{Шарапудинов И.И., Гаджиева З.Д.}
Полиномы, ортогональные по Соболеву, порожденные многочленами Мейкснера // Изв. Сарат. ун-та. Нов. сер. Сер. Математика. Механика. Информатика,
2016. Т.16. Вып. 3. С. 310--321.


\bibitem{ramis-shGadjGadjMir}
{Шарапудинов И.И., Гаджиева З.Д., Гаджимирзаев Р.М.}
Разностные уравнения и полиномы, ортогональные по Соболеву, порожденные многочленами Мейкснера //
Владикавказский Мат. журнал, 2017. Т.19. Вып. 2. С. 58--72.


\bibitem{ramis-Shar9}
{Шарапудинов~И.И.}
Приближение дискретных функций и многочлены Чебышева, ортогональные на равномерной сетке //
Мат. заметки, 2000. Т. 67. Вып. 3. С. 460--470.


\bibitem{ramis-SharT1}
{Шарапудинов~Т.И.}
Аппроксимативные свойства смешанных рядов по полиномам Чебышева, ортогональным на равномерной сетке //
Вестник Дагестанского научного центра РАН, 2007. Т. 29. С. 12-–23.


\bibitem{ramis-SharII}
{Шарапудинов~И.И.}
Системы функций, ортогональных по Соболеву, порожденные ортогональными функциями //
Современные проблемы теории функций и их приложения.  Материалы 18-й международной Саратовской зимней школы. 2016. С. 329--332.


\bibitem{ramis-Gadz12}
{Fernandez~L., Teresa E. Perez, Miguel A. Pinar, Xu~Y.} Weighted Sobolev orthogonal polynomials on the unit ball~// Journal of Approximation Theory, 171, 2013, pp.~84--104.


\bibitem{ramis-Gadz13}
{Antonia~M. Delgado, Fernandez~L., Doron~S. Lubinsky, Teresa~E. Perez, Miguel~A. Pinar.} Sobolev orthogonal polynomials on the unit ball via outward normal derivatives~// Journal of Mathematical Analysis and Applications, 440, №~2, 2016, pp.~716--740.


\bibitem{ramis-Gadz14}
{Fernandez~L., Marcellan~F., Teresa~E. Perez, Miguel~A. Pinar, Xu~Y.} Sobolev orthogonal polynomials on product domains~// Journal of Computational and Applied Mathematics, 284, 2015, pp.~202--215.


\bibitem{ramis-Gadz16}
{Шарапудинов~И.~И., Шарапудинов~Т.~И.} Полиномы, ортогональные по Соболеву, порожденные многочленами Чебышева, ортогональными на сетке~// Изв. вузов. Матем., 2017, №~8, 67--79.


\bibitem{ramis-Gadz17}
{Гаджимирзаев~Р.~М.} Ряды Фурье по полиномам Мейкснера, ортогональным по Соболеву~// Изв. Сарат. ун-та. Нов. сер. Сер. Математика. Механика. Информатика, 16:4 (2016), 388--395.


\bibitem{ramis-Gadz1}
{Шарапудинов~И.~И., Гаджиева~З.~Д., Гаджимирзаев~Р.~М.} Системы функций, ортогональных относительно скалярных произведений типа Соболева с дискретными массами, порожденных классическими ортогональными системами~// Дагестанские электронные математические известия. 2016. Вып.~6. С.~31--60.


\bibitem{ramis-Gadz3}
{Сеге~Г.} Ортогональные многочлены. М.: Физматгиз, 1962.
%%
%% end lit. ramis

\bibitem{charlier-Shar3}
Meijer~H.~G. Laguerre polynomials generalized to a certain discrete Sobolev inner product space~// J. Approx. Theory. 1993. Vol.~73. Iss.~1. Pp.~1--16.

\bibitem{charlier-Shar8}
Шарапудинов~И.~И. Смешанные ряды по ортогональным полиномам. Махачкалаю. Изд-во ДНЦ РАН. 2004.

\bibitem{charlier-Shar10}
Бейтмен~Г., Эрдейи~А. Высшие трансцендентные функции. Том 2. Москва. Наука. 1974.

\bibitem{charlier-Shar11}
Ширяев~А.~Н. Вероятность-1. Москва. Изд-во МЦНМО. 2007.

%AGG

\bibitem{2_bernstein}
\textit{Bernshtein~S.~N.}
On trigonometric interpolation by the method of least squares
%O trigonometricheskom interpolirovanii po sposobu naimen'shih kvadratov
~// Dokl. Akad. Nauk USSR. Vol.~4 (1934). pp.~1--5. (in Russian)

\bibitem{4_erdos}
\textit{Erd{\"o}s~P.} Some theorems and remarks on interpolation~// Acta Sci. Math. (Szeged) Т.~12 (1950), pp.~11--17.

%		\bibitem{fiht-1969}
%		\textit{Fikhtengol'ts~G.~M.} Course of differential and integral calculus Vol.~3. Moscow.~:
%		FIZMATLIT,1969. 656~p. (in Russian)

\bibitem{7_kalashnikov}
\textit{Kalashnikov~M.~D.}
On polynomials of best (quadratic) approximation on a given system of points.
%O polinomah nailuchshego (kvadraticheskogo) priblizheniya v zadannoy sisteme tochek
~// Dokl. Akad. Nauk USSR. Vol.~105 (1955). pp.~634--636. (in Russian)

\bibitem{8_krilov}
\textit{Krilov~V.~I.}
Convergence of algebraic interpolation with respect to the roots of a Chebyshev polynomial for absolutely continuous functions and functions with bounded variation.
%Shodimost algebraicheskogo interpolirovaniya po kornyam mnogochlena Chebisheva dlya absolutno neprerivnih funkciy i funkciy s ogranichennim izmeneniyem
~// Dokl. Akad. Nauk USSR. Vol.~107 (1956). pp.~362--365. (in Russian)

\bibitem{9_marcinkiewicz}
\textit{Marcinkiewicz~J.} Quelques remarques sur l'interpolation~// Acta Sci. Math. (Szeged) Vol.~8 (1936). pp.~127--130. (in French)

\bibitem{10_marcinkiewicz}
\textit{Marcinkiewicz~J.} Sur la divergence des polyn{\^o}mes d'interpolation~// Acta Sci. Math. (Szeged) Vol.~8 (1936). pp.~131--135. (in French)

\bibitem{11_natanson}
\textit{Natanson~I.~P.} On the Convergence of Trigonometrical Interpolation at Equi-Distant Knots. // Annals of Mathematics, Second Series, Vol.~45, no. 3 (1944). pp.~457-471. DOI:10.2307/1969188.
%On the convergence of trigonometrical interpolation at equi-distant knots~// Ann. Math. Vol.~45 (1944). pp.~457--471.

\bibitem{12_nikolsky}
\textit{Nikol'skii~S.~M.} On some methods of approximation by trigonometric sums. // Mathematics of the USSR - Izvestiya. Vol.~4 (1940). pp.~509--520. (in Russian)

\bibitem{17_turetsky}
\textit{Turetskiy~A.~H.} Interpolation theory in exercises. Minsk.~: Vissheyshaya shkola, 1968. 320~p (in Russian).

\bibitem{18_zigmund}
\textit{Zygmund~A.} Trigonometric Series. Vol~1. Cambridge.~:
Cambridge University Press,1959. 747~p.	

\bibitem{akniyev}
\textit{Akniyev~G.~G.} Discrete least squares approximation of piecewise-linear functions by trigonometric polynomials // Issues Anal., Vol.~6 (24), Issue 2 (2017). pp.~3-24. DOI: 10.15393/j3.art.2017.4070.

\bibitem{shii-1983}
\textit{Sharapudinov I.I.} On the best approximation and polynomials of the least quadratic deviation~// Anal. Math. V.~9 (1983), Issue~3. pp.~223--234.

\bibitem{shii-2017}
\textit{Sharapudinov I.I.} Overlapping transformations for approximation of continuous functions by means of repeated mean Valle Poussin~// Daghestan Electronic Mathematical Reports, Issue~8 (2017). pp.~70--92.

\bibitem{mkasumov_disc_funcs}
\textit{Magomed-Kasumov~M.~G.}
Approximation of piecewise smooth functions by the trigonometric Fourier series //
Materials of XIX International Saratov Winter School "Contemporary Problems of Function Theory and Their Applications" (2018). pp.~190-193.
%		\bibitem{mkasumov}
%		\textit{Magomed-Kasumov~M.~G.} Approximation properties of de la Valle-Poussin means for piecewise smooth functions~// Mat. Zametki. Vol.~100, Issue~2, 2016. pp.~229--244.
%		DOI:10.1134/S000143461607018X

\bibitem{courant}
\textit{Courant~R.} Differential and Integral Calculus Vol.~1. New~Jersey.~:
Wiley-Interscience, 1988. 704~p.



\end{thebibliography}
