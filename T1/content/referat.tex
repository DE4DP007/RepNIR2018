\Referat %Реферат отчёта, не более 1 страницы
Отчет содержит 43~с., X~источников.
 %\bigskip

%\textbf{ Ключевые слова:}
%функциональные пространства Лебега и Соболева с переменным показателем;
ФУНКЦИОНАЛЬНЫЕ ПРОСТРАНСТВА ЛЕБЕГА И СОБОЛЕВА,
ВЕСО-\linebreak ВЫЕ ПРОСТРАНСТВА С ПЕРЕМЕННЫМ ПОКАЗАТЕЛЕМ,
ТЕОРИЯ ПРИБЛИЖЕНИЙ,
%ультрасферические полиномы Якоби;
ПОЛИНОМЫ ЯКОБИ,
ПОЛИНОМЫ МЕЙКСНЕРА,
%специальные ряды по классическим ортогональным полиномам;
СПЕЦИАЛЬНЫЕ РЯДЫ ПО ОРТОГОНАЛЬНЫМ ПОЛИНОМАМ,
СРЕДНИЕ ВАЛЛЕ ПУССЕНА
 %\bigskip

%Настоящий отчёт содержит итоги работы за 2018 год Отдела математики и информатики ДНЦ РАН по теме
%№ 0202-2018-0004
%%<<Функциональные пространства с переменным показателем и их приложения. Некоторые вопросы теории приближений полиномами, рациональными функциями, сплайнами и вейвлетами>>
%%осуществлению фундаментальных научных исследований в соответствии с
%из Программы фундаментальных научных исследований государственных академий наук на 2013–2020 годы.

Объектом исследования являются частичные суммы специального ряда по ультрасферическим полиномам Якоби и их линейные средние, ряды Фурье и специальные ряды по полиномам Мейкснера, интерполяционные рациональные сплайн-функции, фрагментами которых служат трехточечные  рациональные интерполянты, дискретные суммы Фурье для кусочно-гладких функций

Цель работы -- исследовать аппроксимативные свойства частичных сумм специального ряда по ультрасферическим полиномам Якоби и их линейных средних, получить поточечную оценку для функции Лебега сумм Фурье-Мейкснера и специального ряда по полиномам Мейкснера, исследовать вопросы сохранения интерполяционными рациональными
сплайн-функциями выпуклости и ковыпуклости с произвольной переменой направления выпуклости дискретных функций, определенных в узлах сплайн-функций, получить оценки отклонения дискретных сумм Фурье от $2\pi$--периодической кусочно-гладкой функции.

В процессе работы использовались общие методы функционального анализа, конструктивной теории функций, а также методы теории ортогональных полиномов.

В результате исследования получены оценки для приближения дифференцируемых и аналитических функций частичными суммами специальных рядов по ультрасферическим полиномам Якоби со свойством прилипания в точках $\pm1$. Эти результаты являются новыми и носят окончательный характер.
%Показано, что система полиномов Якоби является базисом Шаудера в весовом пространстве Лебега с переменным показателем.
%Доказано, что при определенных условиях на переменный показатель суммы Фурье -- Якоби равномерно ограничены в весовом пространстве Лебега с переменным показателем.
Исследовано поведение функции Лебега частичных сумм Фурье-Мейкснера. Этот результат является обобщением результатов других авторов на эту же тему.
Получено решение открытой задачи о ковыпуклой сплайн-интерполяции с переменой направления выпуклости заданной функции в случае рациональных сплайн-функций.
Исследованы аппроксимативные свойства дискретных сумм Фурье для кусочно-гладких функций и показано, что полученные оценки неулучшаемы по порядку.

%Все эти
Полученные результаты находят применение при описании физических и биологических явлений, при обработке изображений, в картографии, в некоторых вопросах теории приближений, при спектральном методе решения дифференциальных и разностных уравнений.

Интерполяционные рациональные сплайн-функции разной степени гладкости могут применяться при проектировании
сложных кривых и поверхностей для сохранения геометрических свойств исходных данных,
в частности, выпуклости или ковыпуклости, а их обобщение -- эффективно использовано в
 численных методах при решении дифференциальных уравнений, вычислении интегралов.




