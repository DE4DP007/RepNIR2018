\Referat %Реферат отчёта, не более 1 страницы

Отчет содержит X~с., X~источников.

 \bigskip
 \textbf{ Ключевые
  слова:}
  функциональные пространства Лебега и Соболева с переменным показателем; весовые пространства Лебега с переменным показателем; теория приближений; ультрасферические полиномы Якоби; полиномы Мейкснера; специальные ряды по классическим ортогональным полиномам; наилучшее приближение и скорость сходимости; повторные средние Валле Пуссена; ортогональные функции.

 \bigskip

Настоящий отчёт содержит итоги работы за 2018 год Отдела математики и информатики ДНЦ РАН по теме
№ 0202-2018-0004
%<<Функциональные пространства с переменным показателем и их приложения. Некоторые вопросы теории приближений полиномами, рациональными функциями, сплайнами и вейвлетами>>
%осуществлению фундаментальных научных исследований в соответствии с
из Программы фундаментальных научных исследований государственных академий наук на 2013–2020 годы.

%%%%%%%%%%%%%%%%
%%%%%%%%%%%%%%%%
%%%%%%%%%%%%%%%%

%Целями ... являются следущие

% begin Ramazanov  A.-R.K.
1. \textbf{ Объект исследования.} Интерполяционные рациональные сплайн-функции, фрагментами
которых служат трехточечные  рациональные интерполянты.  Функция Лебега частичных сумм Фурье-Мейкснера;
                     частичные суммы специального ряда полиномам Мейкснера и их аппроксимативные свойства.

2. \textbf{ Цель работы.} Исследовать вопросы сохранения интерполяционными рациональными
сплайн-функциями выпуклости (вверх или вниз) и ковыпуклости с
произвольной переменой направления выпуклости дискретных функций, определенных в узлах
сплайн-функций. Получить поточечную оценку для функции Лебега сумм Фурье-Мейкснера;
             получить оценки сверху отклонения частичной суммы специального ряда от функции, заданной на равномерной сетке и принадлежащей пространству $l^2_\rho$.

3. \textbf{ Методы или методология проведения работы.} Конструирование непрерывно дифференцируемых
или дважды непрерывно дифференцируемых на данном отрезке локальных сплайн-функций
из трехточечных рациональных интерполянтов с сохранением выпуклости или
ковыпуклости исходной дискретной функции, различные методы теории функций и функционального анализа, а также методы теории ортогональных полиномов.

4. \textbf{ Результаты работы и их новизна.} Для  сплайн-функций класса $C^1$ по трехточечным
 рациональным интерполянтам исследованы вопросы сохранения ими выпуклости (вверх или вниз)
 и ковыпуклости заданной дискретной функции.

Получены достаточные условия ковыпуклой интерполяции с произвольной переменой направления
выпуклости дискретной функции, определенной в узлах сплайн-функции такими сплайн-функциями
в случае  произвольных сеток узлов. Этот результат представляет собой решение
открытой задачи о ковыпуклой сплайн-интерполяции с переменой направления выпуклости заданной
функции в случае рациональных сплайн-функций.

Получены также достаточные условия выпуклости (вверх или вниз) интерполяционных
сплайн-функций класса $C^2$ по рациональным интерполянтам по произвольным сеткам узлов.
Получены более общие условия на систему узлов, при которых интерполяционные
рациональные сплайн-функции сохраняют выпуклость (или вогнутость) исходной функции. 

получена оценка для функции Лебега частичных сумм Фурье-Мейкснера. Этот результат является обобщением результатов других авторов на эту же тему.
                                        Исследованы аппроксимативные свойства частичных сумм специального ряда по полиномам Мейкснера.
                                        Главным преимуществом специальных рядов, по сравнению с рядами Фурье, является то, что они совпадают с исходной функцией,
                                        заданной на равномерной сетке,  в первых $r$ точках и являются хорошим аппаратом для одновременного приближения
                                        дискретных функций и их конечных разностей.


5. \textbf{ Область применения результатов.} Методы формосохраняющих сплайн-функций являются одним
из основных аппаратов автоматизированного геометрического проектирования. Они находят
также применение при описании физических и биологических явлений, при обработке изображений,
в картографии и в других областях. Полученные результаты могут быть использованы в некоторых вопросах теории приближений и
                                численного анализа при спектральном методе решения задачи Коши для разностного уравнения.

6. \textbf{ Рекомендации по внедрению  или итоги внедрения результатов исследования.} Результаты
работы могут находить применения как в самой теории сплайнов, так и при проектировании
сложных кривых и поверхностей для сохранения геометрических свойств исходных данных,
в частности, выпуклости или ковыпуклости.


8. \textbf{ Прогнозные предположения о развитии объекта исследования.} Интерполяционные рациональные
сплайн-функции разной степени гладкости могут быть обобщены и эффективно использованы в
 численных методах решения дифференциальных уравнений, вычисления интегралов.




%Для произвольной непрерывной на отрезке числовой оси функции $f(x)$  были построены $k$ раз($k=1, 2,\ldots$) непрерывно дифференцируемые на данном отрезке интерполяционные рациональные сплайн-функции $R_{N,k}(x,f)$, последовательность которых (в отличие от классических полиномиальных сплайнов) для любой последовательности сеток с диаметром, стремящимся к нулю, равномерно на этом отрезке сходится к самой функции $f(x)$.  Аналогичным свойством безусловной сходимости обладают также последовательности производных построенных сплайн-функций.
%Интерполяционные сплайн-функции $R_{N,1}(x,f)$  применены для исследования вопросов сохранения ими выпуклости (вверх или вниз) и ковыпуклости  с произвольной переменой направления выпуклости дискретной функции $f(x)$. Как известно, конструкция классических полиномиальных сплайнов не позволяет эффективно применять их для решения второй части данной задачи, касающейся сохранения ковыпуклости с произвольной переменой направления выпуклости дискретных данных.     Получены достаточно легко проверяемые условия ковыпуклой интерполяции дискретной функции  $f(x)$  с возможной переменой направления выпуклости рациональными сплайн-функциями $R_{N,1}(x,f)$ по произвольным сеткам узлов.
%
%Получены достаточные условия выпуклости (вверх или вниз) интерполяционных рациональных сплайн-функций $R_{N,2}(x,f)$  класса $C^2$ в случае выпуклой (вверх или вниз) функции $f(x)$ и произвольных сеток узлов.
%
%Для дважды непрерывно дифференцируемых на отрезке функций получены оценки совместного   равномерного приближения самих функций и их производных до второго порядка посредством интерполяционных рациональных сплайн-функций и их соответствующих производных.
%
%%Гасан
%Пусть $N \geq 2$ --- некоторое натуральное число. Выберем на вещественной оси $N$ равномерно расположенных точек $t_k=2\pi k / N + u$ $(0 \leq k \leq N-1)$.
%Обозначим через $L_{n,N}(f)=L_{n,N}(f,x)$ $(1\leq n\leq N/2)$ тригонометрический полином порядка $n$, обладающий наименьшим квадратичным отклонением от $f$ относительно системы $\{t_k\}_{k=0}^{N-1}$. Выберем $m+1$ точку $-\pi=a_{0}<a_{1}<\ldots<a_{m-1}<a_{m}=\pi$, где $m\geq 2$, и обозначим $\Omega=\left\{a_i\right\}_{i=0}^{m}$.
%Через $C_{\Omega}^{0,r}$ обозначим класс $2\pi$-периодических функций $f$, $r$-раз дифференцируемых на каждом интервале  $(a_{i},a_{i+1})$, причем
%производная $f^{(r)}$ на каждом $(a_{i},a_{i+1})$ абсолютно непрерывна.
%Через $C^r_\Omega$ обозначим множество непрерывных функций из $C^{0,r}_\Omega$.
%Мы рассмотрим задачу приближения функций из $C_{\Omega}^{0,1}$ и $C_{\Omega}^{2}$ полиномами $L_{n,N}(f,x)$.
%Была найдена точная по порядку оценка
%$\left|f(x) - L_{n,N}(f,x)\right| \leq c(f,\varepsilon)/n$, $\left|x - a_i\right| \geq \varepsilon$ для функий из $C_{\Omega}^{0,1}$.
%Для функий из $C_{\Omega}^{2}$, вместо оценки $\left|f(x)-L_{n,N}(f,x)\right| \leq c\ln n/n$, которая следует из известного неравенства Лебега, найдена точная по порядку оценка $\left|f(x)-L_{n,N}(f,x)\right| \leq c/n$ ($x \in \mathbb{R}$), которая равномерна относительно $1 \leq n \leq N/2$.
%Кроме того, для них была найдена локальная оценка $\left|f(x)-L_{n,N}(f,x)\right| \leq c(\varepsilon)/n^2$ ($\left|x - a_i\right| \geq \varepsilon$), которая также равномерна относительно $1 \leq n \leq N/2$.
%Доказательства этих оценок основаны на сравнении дискретных и непрерывных конечных сумм ряда Фурье.






%В отчетном году продолжены исследования по теории систем функций, ортогональных по Соболеву, порожденных классическими ортогональными системами.
%Особое внимание уделено вопросам сходимости рядов Фурье по полиномам, ортогональным по Соболеву, порожденным классическими ортогональными полиномами непрерывной и дискретной переменной.
%Изучены асимптотические свойства полиномов, ортогональных по Соболеву, порожденных указанными классическими системами. В частности, изучены алгебраические и асимптотические свойства полиномов, ортогональных по Соболеву, порожденных классическими полиномами Якоби, Лагерра, полиномами Чебышева дискретной переменной, полиномами Мейкснера и полиномами Шарлье.
%
%На основе систем функций, ортогональных по Соболеву, разработаны алгоритмы для численно-аналитического решения систем линейных и нелинейных дифференциальных и разностных уравнений. Для широкого класса систем функций, ортогональных по Соболеву, найдены условия, при соблюдении которых сходятся итерационные процессы, на которых основываются указанные алгоритмы для приближенного решения систем дифференциальных и разностных уравнений.
%Ряд разработанных алгоритмов доведены до численных экспериментов (разработаны прикладные программные пакеты, реализующие указанные алгоритмы).
%Проведенные эксперименты показывают высокую эффективность предлагаемого численно-аналитического подхода к решению систем дифференциальных и разностных уравнений.


%%%%%%%%%%%%%%%%
%%%%%%%%%%%%%%%%
%%%%%%%%%%%%%%%%



%Рассмотрена система функций $\mathcal{\psi}_{r,n}(x)$ $(r=1,2,\ldots, n=0,1,\ldots)$ ортонормированная по Соболеву относительно скалярного произведения  вида\linebreak $\langle f,g\rangle=\sum_{k=0}^{r-1}\Delta^kf(0)\Delta^kg(0)+
%\sum_{j=0}^\infty\Delta^rf(j)\Delta^rg(j)\rho(j)$,
%порожденная заданной ортонормированной системой функций $\mathcal{\psi}_{n}(x)$ $( n=0,1,\ldots)$. Показано, что ряды и суммы Фурье по системе
%$\mathcal{\psi}_{r,n}(x)$ $(r=1,2,\ldots, n=0,1,\ldots)$ является удобным и весьма эффективным инструментом приближенного решения задачи Коши для разностных уравнений.
%
%Для системы полиномов $l_{r,n}^{\alpha}(x)$ ($r$-натуральное число, $n=0, 1, \ldots$), ортонормированной относительно скалярного произведения типа Соболева (полиномы, ортонормированные по Соболеву) следующего вида $\langle f,g\rangle=\sum_{\nu=0}^{r-1}f^{(\nu)}(0)g^{(\nu)}(0)+\int_{0}^{\infty} f^{(r)}(x)g^{(r)}(x)\rho(x) dx$ и порожденной классическими ортонормированными полиномами Лагерра, получены рекуррентные соотношения, которые могут быть использованы для изучения различных свойств этих полиномов и вычисления их значений при любых $x$ и $n$.
%
%
%
%
%
%
%
%
%
%%%%%%%%%%%%%%%%%
%%%%%%%%%%%%%%%%%
%%%%%%%%%%%%%%%%%
%
%
%Рассмотрена задача о конструировании полиномов $s_{r,n}^\alpha(x)$, порожденных полиномами Шарлье $s_n^\alpha(x)$ и ортонормированных относительно скалярного произведения типа Соболева вида
%$  \langle f,g \rangle = \sum_{k=0}^{r-1} \Delta^k f(0) \Delta^k g(0) + \sum_{j=0}^\infty \Delta^r f(j) \Delta^r g(j) \rho(j) $, где $ \rho(x)=\alpha^x e^{-\alpha}/\Gamma(x+1)$.
%Показано, что система полиномов $s_{r,n}^\alpha(x)$, порожденная полиномами Шарлье, полна в гильбертовом пространстве $W^r_{l_\rho}$, состоящем из дискретных функций, заданных на сетке $\Omega=\{0,1,\ldots\}$, в котором введено скалярное произведение $\langle f,g \rangle$. Найдена явная формула вида $ s_{r,k+r}^{\alpha}(x) = \sum_{l=0}^{k} b_l^r x^{[l+r]} $, в которой $x^{[m]} = x(x-1)\ldots(x-m+1)$. Установлена связь полиномов $s_{r,n}^\alpha(x)$ с порождающими их ортонормированными классическими полиномами Шарлье $s_n^\alpha(x)$ вида $	s_{r,k+r}^{\alpha}(x)= U_k^r \left[s_{k+r}^{\alpha}(x) - \sum_{\nu=0}^{r-1} V_{k,\nu}^r x^{[\nu]}\right]$, в которой для чисел $U_k^r$, $V_{k,\nu}^r$ найдены явные выражения. 