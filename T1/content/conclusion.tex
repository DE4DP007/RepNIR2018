\Conclusion

В 2018 году в Отделе математики и информатики Дагестанского научного центра РАН проведены научно-исследовательские работы по теме
<<Функциональные пространства с переменным показателем и их приложения. Некоторые вопросы теории приближений полиномами, рациональными функциями, сплайнами и вейвлетами>>.



%%%%%%%%%%%%%%%%
%%%%%%%%%%%%%%%%
%%%%%%%%%%%%%%%%


В теоремах \ref{ark-teor1} и \ref{ark-teor2} было показано, что интерполяционные рациональные
сплайн-функции $R_{N,1}(x)$ позволяют при выполнении определенных условий
(приведенных в формулировках этих теорем) интерполировать дискретные функции
с сохранением формы ковыпуклости.

Ранее в работах \cite{ark-12,ark-13,ark-14,ark-15} для рациональных
сплайн-функций $R_{N,1}(x)$ исследованы их аппроксимативные свойства
 (в частности, доказана безусловная сходимость для функций и производных, получены
 оценки скорости сходимости), а также исследованы вопросы отсутствия или наличия
явления Гиббса.

Отметим, что найдены приложения рациональных сплайн-функций $R_{N,1}(x)$ к  приближенному
решению начальных и краевых задач для дифференциальных уравнений.

Выше теоремы \ref{ark-teor1} и \ref{ark-teor2} получены для сплайн-функций $R_{N,1}(x)$ класса
$C^1_{[a,b]}$.
Теорема \ref{ark-teor2.1} представляет собой аналог теоремы \ref{ark-teor1} для сплайн-функций
$R_{N,2}(x)$ класса $C^2_{[a,b]}$. При этом теорема \ref{ark-teor1} справедлива
при условии $1/2<q_i<2$, тогда как теорема \ref{ark-teor2.1} справедлива при более
слабом условии $c<q_i<1/c$ для $c=(2\sqrt 2+1)/(2\sqrt 2+5)$.

Однако остается открытым вопрос о справедливости аналога теоремы  \ref{ark-teor2} для сплайн-функций
$R_{N,2}(x)$.

Исследовано поведение функции Лебега сумм Фурье по модифицированным полиномам Мейкснера $m_{n,N}^{\alpha}(x)$, $\alpha>-1$. При соблюдении условий, указанных в теореме \ref{Ramtheo1}, получена верняя оценка для функции Лебега для $x\in\left[\frac{\theta_n}{2}, \infty\right)$. Этот результат является обобщением (относительно параметра $\alpha$) результатов, полученных в работе \cite{Ram3}. Далее для произвольного натурального $r$  и функции $f$ из пространства $l_{\rho_N}^2(\Omega_\delta)$ построены специальные ряды по полиномам $m_{n,N}^{\alpha}(x)$. Рассмотрена задача об исследовании аппроксимативных свойств частичных сумм специального ряда с уделением основного внимания на получение поточечной оценки для соответствующей функции Лебега. При этом отметим, что частичные суммы этих рядов, в отличие от сумм Фурье по тем же полиномам, совпадают с значениями исходной функции $f$ в точках $\{0, \delta, 2\delta, \ldots, (r-1)\delta\}$.


%%%%%%%%%%%%%%%%
%%%%%%%%%%%%%%%%
%%%%%%%%%%%%%%%%





%Для  произвольного натурального $r$ рассмотрены полиномы $p^{\alpha,\beta}_{r,k}(x)$ $(k=0,1,\ldots)$, ортонормированные относительно скалярного произведения типа Соболева следующего вида
%$$
%<f,g>=\sum_{\nu=0}^{r-1}f^{(\nu)}(-1)g^{(\nu)}(-1)+
%\int_{-1}^{1}f^{(r)}(t)g^{(r)}(t)(1-t)^\alpha(1+t)^\beta dt
%$$
%и изучены  их свойства. Введены в рассмотрение ряды Фурье по полиномам $p_{r,k}(x)=p^{0,0}_{r,k}(x)$ и некоторые их обобщения, частичные суммы которых  сохраняют некоторые важные  свойства частичных сумм ряда Фурье по полиномам $p_{r,k}(x)$, в том числе и свойство $r$-кратного совпадения (<<прилипания>>) частичных сумм ряда Фурье по полиномам $p_{r,k}(x)$  в  точках $-1$ и $1$ между собой и с исходной функцией $f(x)$.  Основное внимание уделено  исследованию вопросов приближения гладких и аналитических функций  частичными суммами упомянутых обобщений, представляющих собой   специальные ряды  по ультрасферическим полиномам Якоби со свойством <<прилипания>> их частичных сумм  точках $-1$ и $1$ .
%
%Рассмотрены системы функций $\mathcal{ \varphi}_{r,n}(x)$ $(r=1,2,\ldots, n=0,1,\ldots)$,
%ортонормированные по Соболеву относительно скалярного произведения  вида
%\begin{equation*}
%\langle f,g\rangle=\sum_{\nu=0}^{r-1}f^{(\nu)}(a)g^{(\nu)}(a)+\int_{a}^{b}f^{(r)}(t)g^{(r)}(x)\rho(x)dx,
%\end{equation*}
%порожденные заданной ортонормированной системой функций $\mathcal{ \varphi}_{n}(x)$ $( n=0,1,\ldots)$.  Показано, что ряды и суммы Фурье по системе $\mathcal{ \varphi}_{r,n}(x)$ $(r=1,2,\ldots, n=0,1,\ldots)$ являются удобным и весьма эффективным инструментом приближенного решения задачи Коши для обыкновенных дифференциальных уравнений (ОДУ).
%
%Рассмотрены полиномы $T_{r,n}(x)$ $(n=0,1,\ldots)$, порожденные многочленами Чебышева $T_{n}(x)=\cos( n\arccos x)$, образующие ортонормированную систему по Соболеву относительно скалярного произведения
%следующего вида
%\begin{equation*}
%<f,g>=\sum_{\nu=0}^{r-1}f^{(\nu)}(-1)g^{(\nu)}(-1)+\int_{-1}^{1}f^{(r)}(t)g^{(r)}(x)\mu(x)dx,
%\end{equation*}
%где $\mu(x)=\frac2\pi(1-x^2)^{-\frac12}$. Для $T_{r,n}(x)$ $(n=0,1,\ldots)$  установлена связь с многочленами Чебышева $T_{n}(x)$ и получены явные представления, которые могут быть использованы при вычислении значений полиномов $T_{r,n}(x)$ и исследовании их асимптотических свойств.
%
%Рассмотрена задача об обращении преобразования Лапласа
% посредством специального ряда по полиномам Лагерра, который в  частном случае совпадает с рядом Фурье по   полиномам $l_{r,k}^{\gamma}(x)$ $(r\in \mathbb{N}, k=0,1,\ldots)$, ортогональным относительно скалярного произведения типа Соболева следующего вида
%\begin{equation*}
%<f,g>=\sum\nolimits_{\nu=0}^{r-1}f^{(\nu)}(0)g^{(\nu)}(0)+\int_0^\infty f^{(r)}(t)g^{(r)}(t)t^\gamma e^{-t}dt, \gamma>-1.
%\end{equation*}
%  Даны оценки приближения функций частичными суммами специального ряда по полиномам Лагерра.
%
%Получены рекуррентные соотношения для системы полиномов $l_{r,n}^{\alpha}(x)$ ($r$-натуральное число, $n=0, 1, \ldots$), ортонормированной относительно скалярного произведения типа Соболева $\langle f,g\rangle=\sum_{\nu=0}^{r-1}f^{(\nu)}(0)g^{(\nu)}(0)+\int_{0}^{\infty} f^{(r)}(x)g^{(r)}(x)\rho(x) dx$ и порожденной классическими ортонормированными полиномами Лагерра.
%
%Рассмотрены системы дискретных функций $\mathcal{\psi}_{r,n}(x)$ $(r=1,2,\ldots, n=0,1,\ldots)$, ортонормированные по Соболеву относительно скалярного произведения  вида
%\begin{equation*}
%\langle f,g\rangle=\sum_{k=0}^{r-1}\Delta^kf(0)\Delta^kg(0)+\sum_{j=0}^\infty\Delta^rf(j)\Delta^rg(j)\rho(j),
%\end{equation*}
%порожденные заданной ортонормированной системой функций $\mathcal{\psi}_{n}(x)$ $( n=0,1,\ldots)$.  Показано, что ряды и суммы Фурье по системе $\mathcal{\psi}_{r,n}(x)$ $(r=1,2,\ldots, n=0,1,\ldots)$ является удобным и весьма эффективным инструментом приближенного решения задачи Коши для  разностных уравнений.




