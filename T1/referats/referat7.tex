\Referat %Реферат отчёта, не более 1 страницы

Отчет содержит 12~с., 11~источников.

 \bigskip
 \textbf{ Ключевые
  слова:}
  ПРИКЛАДНОЙ ПАКЕТ ПРОГРАММ;
  ПОЛИНОМЫ ЧЕБЫШЕВА ПЕРВОГО РОДА;
  ПОЛИНОМЫ, ОРТОГОНАЛЬНЫЕ ПО СОБОЛЕВУ;
  ЧИСЛЕННО-АНАЛИТИЧЕСКОЕ РЕШЕНИЕ ОДУ;
  ЗАДАЧА КОШИ ДЛЯ ОДУ;
  ОБЫКНОВЕННОЕ ДИФФЕРЕНЦИАЛЬНОЕ УРАВНЕНИЕ.


 \bigskip

Настоящий отчёт содержит итоги работы за 2016 год Отдела математики и информатики ДНЦ РАН по теме
<<Разработка алгоритма для численно-аналитического решения задачи Коши для обыкновенного дифференциального уравнения на основе полиномов, ортогональных по Соболеву, порожденных полиномами Чебышева первого рода>>
%осуществлению фундаментальных научных исследований в соответствии с
из Программы фундаментальных научных исследований государственных академий наук на 2013–2020 годы.


В отчетный период сотрудниками ОМИ были сконструированы новые системы функций, ортогональные в смысле Соболева, порожденные классическими ортогональными системами. Установлена связь между рядами Фурье по соболевским полиномам и смешанными рядами, введенными в работах Шарапудинова И.И. Изучены аппроксимативные свойства рядов Фурье по построенным системам функций. Разработан алгоритм для численно-аналитического решения задачи Коши для обыкновенных дифференциальных уравнений (ОДУ) на основе полиномов, ортогональных по Соболеву, порожденных дискретными полиномами Чебышева и системой функций Хаара. Была составлена программа, которая позволяет найти решение задачи Коши для линейных ОДУ численным методом, основанным
на разложении функции и ее производных в ряд Фурье по полиномам, ортогональным по Соболеву и порожденным упомянутыми системами функций. Одно из преимуществ данного подхода заключается в том, что в разложении  функции и всех ее производных участвуют одни и те же коэффициенты. При этом
оказываются учтенными начальные условия задачи Коши. Программа может найти применение в прикладных задачах, возникающих в таких областях, как математическая физика, математическое моделирование и др. Данная программа зарегистрирована (Свидетельство № 2016617831 о государственной регистрации программы для ЭВМ <<MixedHaarDeqSolver>>).



%Сконструированы новые системы функций, ортогональные в смысле Соболева, порожденные классическими ортогональными системами. В качестве порождающих систем рассматривались системы полиномов Чебышева (непрерывный и дискретный варианты) и некоторые другие классические ортогональные системы функций (система Хаара, система полиномов Лагерра и др.). Установлена связь между рядами Фурье по соболевским полиномам и смешанными рядами, введенными ранее Шарапудиновым И.И. Используя методы и техники, разработанные для исследования смешанных рядов, изучены некоторые аппроксимативные свойства рядов Фурье по вновь построенным системам функций. Результаты были доложены на ряде научных конференций и опубликованы в научных журналах и сборниках трудов.
%
%Разработан алгоритм для численно-аналитического решения задачи Коши для обыкновенного дифференциального уравнения на основе полиномов, ортогональных по Соболеву, порожденных дискретными полиномами Чебышева и системой функций Хаара. На основе данного алгоритма была составлена программа, которая позволяет найти решение задачи Коши для линейных обыкновенных дифференциальных уравнений численным методом, основанным на разложении самой функции и ее производных в ряд Фурье по полиномам, ортогональным по Соболеву и порожденным упомянутыми системами функций. Основное преимущество данного подхода заключается в том, что в разложении самой функции и всех ее производных участвуют одни и те же коэффициенты. При этом ввиду особенностей конструкции соболевского ряда самым естественным образом оказываются учтенными начальные условия задачи Коши. Программа может быть использована для решения задач, возникающих в математической физике, математическом моделировании, в задачах идентификации линейных систем автоматического регулирования и управления. Созданная программа зарегистрирована как объект интеллектуальной собственности (Свидетельство № 2016617831 о государственной регистрации программы для ЭВМ «MixedHaarDeqSolver»).

\input chapters/refs/ref7.tex


