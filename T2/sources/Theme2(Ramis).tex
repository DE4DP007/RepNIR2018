\documentclass[a4paper,12pt]{article}
\usepackage[T2A]{fontenc}				% Поддержка русских букв
\usepackage{cmap}
\usepackage[T2A]{fontenc}
\usepackage[utf8]{inputenc}
\usepackage[russian]{babel}
\usepackage{cmap}

\usepackage{mathtools}
\usepackage{amsfonts, amssymb, amsthm}
\usepackage{amscd}
\usepackage{dsfont}
\usepackage{cite}
\usepackage[plainpages=false,pdfpagelabels=false,colorlinks=false,pdfborder={0 0 0}]{hyperref}

\newtheorem{theorem}{Теорема}
\newtheorem{proposition}{Предложение}
\newtheorem{lemma}{Лемма}
\newtheorem{corollary}{Следствие}
\newtheorem{notation}{Обозначения}
\newtheorem{definition}{Определение}
\newtheorem{remark}{Замечание}

\newenvironment{Rtwocolbib}
{%
\vspace*{3mm} %
\noindent
%{\normalfont\bfseries\sffamily Библиографический список}%
\def\bibname{}
\small
%\begin{multicols}{2}%
\vspace*{-12mm}%
\begin{thebibliography}{99}
\setlength{\itemsep}{-4pt}
}{%
\end{thebibliography}
%\end{multicols}%
\normalsize}%

\textwidth=16.5cm
\hoffset=-2.5cm
\textheight=25.5cm
\voffset=-2.5cm

\righthyphenmin=2
\parindent=0mm
\parskip=3.5mm


% Заменяем библиографию с квадратных скобок на точку:
\makeatletter
\renewcommand{\@biblabel}[1]{#1.}
\makeatother

\bibliographystyle{utf8gost705u}
%\bibliographystyle{unsrt}

\begin{document}
\section{Система функций, ортонормированная по Соболеву и порождённая системой функций Лагерра}

{\bf Аннотация.} В настоящей работе рассматривается система функций $\lambda_{r,n}^{\alpha}(x)$ ($r\in\mathbb{N}$, $n=0, 1, \ldots$), ортонормированная при $\alpha>-1$ относительно скалярного произведения типа Соболева следующего вида $\langle f,g\rangle=\sum_{\nu=0}^{r-1}f^{(\nu)}(0)g^{(\nu)}(0)+\int_{0}^{\infty} f^{(r)}(x)g^{(r)}(x) dx$ и порождённая ортонормированными функциями Лагерра.
Показано, что ряд Фурье по системе $\{\lambda_{r,n}^{\alpha}(x)\}_{k=0}^\infty$ при $\alpha\geq0$ сходится равномерно относительно $x\in[0, A]$, $A\geq0,$ к функции $f\in W^r_{L^p}$ для $\frac{4}{3}<p<4$.
Для системы функций $\lambda_{r,n}^{\alpha}(x)$ получены рекуррентные соотношения.
Кроме того, исследованы асимптотические свойства функций $\lambda_{1,n}^0(x)$ при $0\leq x\leq\omega$, где $\omega$--некоторое фиксированное положительное число.


\subsection{Введение}
Пусть $p>1$, $L^p$ -- пространство измеримых функций $f$, определенных на полуоси $[0, \infty)$ и таких, что
$$
\|f\|_{L^p}=\left(\int\limits_0^{\infty}|f(x)|^pdx\right)^\frac{1}{p}<\infty,
$$
$W^r_{L^p}$ -- пространство функций $f$, непрерывно дифференцируемых $r-1$ раз, для которых $f^{(r-1)}$ абсолютно непрерывна на произвольном сегменте $[a, b]\subset[0, \infty)$, а $f^{(r)}\in L^p$.
Через $\lambda_n^\alpha(x)$ $(n=0, 1, \ldots)$ обозначим функции Лагерра, которые определяются равенством
\begin{equation}\label{funcLag}
\lambda_n^\alpha(x)=\sqrt{\rho(x)}l_n^\alpha(x),
\end{equation}
где $\rho(x)=e^{-x}x^\alpha$, $l_n^\alpha(x)$ -- ортонормированный полином Лагерра, определенный равенством \eqref{Ram_eq5}.
Хорошо известно, что система функций $\{\lambda_n^\alpha(x)\}_{n=0}^\infty$ при $\alpha>-1$ ортонормирована относительно скалярного произведения
$$
\langle \lambda_m^\alpha, \lambda_n^\alpha\rangle=\int\limits_0^\infty \lambda_m^\alpha(x)\lambda_n^\alpha(x)dx.
$$

Система функций Лагерра $\{\lambda_n^\alpha(x)\}_{n=0}^\infty$ порождает на $[0, \infty)$ систему функций $\lambda_{r,n}^\alpha(x)$ ($r\in\mathbb{N}$, $n=0, 1, \ldots$), ортонормированную при $\alpha>-1$ относительно скалярного произведения Соболева вида:
\begin{equation}\label{Ram_eq1}
\langle f,g\rangle=\sum_{\nu=0}^{r-1}f^{(\nu)}(0)g^{(\nu)}(0)+\int_{0}^{\infty} f^{(r)}(x)g^{(r)}(x)dx.
\end{equation}

Функции $\lambda_{r,n}^{\alpha}(x)$, порожденные ортонормированными функциями Лагерра $\lambda_{n}^{\alpha}(x)$, определяются посредством равенств \eqref{Ram_eq12} и \eqref{Ram_eq13}.
В настоящей работе показано, что ряд Фурье по системе $\{\lambda_{r,n}^{\alpha}(x)\}_{k=0}^\infty$ сходится равномерно относительно $x\in[0, A]$, $0\leq A<\infty$, к функции $f\in W^r_{L^p}$ для $\frac{4}{3}<p<4$.
Для системы функций $\lambda_{r,n}^{\alpha}(x)$ получены рекуррентные соотношения, которые могут быть использованы для вычисления их значений при любых $x$ и $n$.
Исследованы асимптотические свойства функций $\lambda_{1,n}^0(x)$ при $0\leq x\leq\omega$, где $\omega$ -- некоторое фиксированное положительное число. Используя эти асимптотические свойства, получены оценки для функций $\lambda_{1,n}^0(x)$ на промежутке $[0,\omega]$.

\subsection{Некоторые сведения о полиномах и функциях Лагерра}

Пусть $\alpha$ -- произвольное действительное число. Тогда для полиномов Лагерра имеют место \cite{Ramlib2}:
\begin{itemize}
\item
формула Родрига
\begin{equation*}
L_n^{\alpha}(x) = \frac{1}{n!}x^{-\alpha}e^{x} \left\{ x^{n+\alpha} e^{-x} \right\}^{(n)};
\end{equation*}

\item
соотношение ортогональности
\begin{equation}\label{Ram_eq3}
\int_0^{\infty} L^{\alpha}_{n}(x) L^{\alpha}_{m}(x)\rho(x) dx = \delta_{n,m} h^{\alpha}_n \quad (\alpha > -1),
\end{equation}
где $\rho(x)=x^{\alpha} e^{-x}$, $\delta_{n,m}$ -- символ Кронекера, $h^{\alpha}_n = \left( n+\alpha \atop n \right) \Gamma(\alpha +1)$;

\item
равенства
\begin{equation*}
\frac{d}{dx} L_n^{\alpha}(x) = -L_{n-1}^{\alpha+1}(x);
\end{equation*}

\begin{equation*}
L_{n}^{-k}(x) =\frac{(-x)^k}{n^{[k]}}L_{n-k}^{k}(x),
\end{equation*}
где $k$--целое и $1\leq k\leq n$, $n^{[0]}=1$, $n^{[k]}=n(n-1)\cdots(n-k+1)$;
\begin{equation*}
xL_{n}^{\alpha+1}(x) =(n+\alpha+1)L_{n}^{\alpha}(x) -(n+1)L_{n+1}^{\alpha}(x);
\end{equation*}

\item
рекуррентная формула
\begin{equation}\label{Ram_eq4}
\left.\begin{gathered}
L_{0}^{\alpha}(x)=1, \quad L_1^{\alpha}(x)=-x+\alpha+1,\\
nL_n^{\alpha}(x)=(-x+2n+\alpha-1)L_{n-1}^{\alpha}(x)-(n+\alpha-1)L_{n-2}^{\alpha}(x), \quad n=2, 3, \ldots
\end{gathered}\right\};
\end{equation}

\item
асимптотическая формула
\begin{equation*}
\left.\begin{gathered}
e^{-\frac{x}{2}}x^{\frac{\alpha}{2}}L_n^\alpha(x)=N^{-\frac{\alpha}{2}}\frac{\Gamma(n+\alpha+1)}{n!}J_\alpha\left(2(Nx)^\frac{1}{2}\right)
+O\left(n^{\frac{\alpha}{2}-\frac{3}{4}}\right),\\
N=n+\frac{\alpha+1}{2},\ x>0, \ \alpha>-1,
\end{gathered}\right\}
\end{equation*}
где оценка равномерна в промежутке $0<x\leq\omega$ ($\omega$ -- фиксированное положительное число), $J_\alpha(x)$ -- функция Бесселя, для которой, в свою очередь, справедлива следующая асимптотическая формула
\begin{equation*}
J_\alpha(x)=\left(\frac{2}{\pi x}\right)^{\frac{1}{2}}\cos\left(x-\frac{\alpha\pi}{2}-\frac{\pi}{4}\right)+O\left(x^{-\frac{3}{2}}\right),\ x\rightarrow+\infty;
\end{equation*}

\item
весовая оценка \cite{Ramlib1}
\begin{equation}\label{Ram_est1}
e^{-\frac{x}{2}}|L_n^\alpha(x)|\leq c(\alpha)A_n^\alpha(x),\ \alpha>-1,
\end{equation}
где $c$, $c(\alpha)$ -- положительные числа, зависящие только от указанных параметров,
\begin{equation}\label{Ram_est2}
A_n^\alpha(x)=
\begin{cases}
  \theta_n^\alpha, & 0\leq x\leq \frac{1}{\theta_n} \\
  \theta_n^{\frac{\alpha}{2}-\frac{1}{4}}x^{-\frac{\alpha}{2}-\frac{1}{4}}, & \frac{1}{\theta_n}< x\leq \frac{\theta_n}{2} \\
  \left[\theta_n\left(\theta_n^{\frac{1}{3}}+|x-\theta_n|\right)\right]^{-\frac{1}{4}}, &  \frac{\theta_n}{2}< x\leq \frac{3\theta_n}{2} \\
  e^{-\frac{x}{4}}, & \frac{3\theta_n}{2}<x,
\end{cases}
\end{equation}
$\theta_n=\theta_n(\alpha)=4n+2\alpha+2$;

\item
формула дифференцирования \cite{Ramlib3}
\begin{equation*}
\left[x^\alpha L_n^\alpha(x)\right]^{(m)}=(n-m+\alpha+1)_m x^{\alpha-m}L_n^{\alpha-m}(x),
\end{equation*}
где $(n)_0=1$, $(n)_m=n(n+1)\cdots(n+m-1)$.
\end{itemize}

Из \eqref{Ram_eq3} следует, что соответствующая ортонормированная система полиномов Лагерра имеет вид:
\begin{equation}\label{Ram_eq5}
l_n^{\alpha}(x)=(h^{\alpha}_n)^{-\frac{1}{2}}L_n^{\alpha}(x), \quad n=0, 1, \ldots,
\end{equation}
т.е.
\begin{equation*}
\int_0^{\infty} l^{\alpha}_{n}(x) l^{\alpha}_{m}(x)\rho(x) dx = \delta_{n,m} \quad (\alpha > -1).
\end{equation*}

Из \eqref{Ram_eq4} и \eqref{Ram_eq5} легко можно получить рекуррентную формулу для $l_n^\alpha(x)$:
\begin{equation*}
\left.\begin{gathered}
l_{0}^{\alpha}(x)=\frac{1}{\sqrt{\Gamma(\alpha+1)}}, \quad l_1^{\alpha}(x)=\frac{-x+\alpha+1}{\sqrt{\Gamma(\alpha+2)}},\\
l_n^{\alpha}(x)=(a_n-b_n x)l_{n-1}^{\alpha}(x)-c_n l_{n-2}^{\alpha}(x), \quad n=2, 3, \ldots
\end{gathered}\right\},
\end{equation*}
где
\begin{equation*}
a_n=a_n(\alpha)=\frac{2n+\alpha-1}{[n(n+\alpha)]^\frac{1}{2}},\
b_n=b_n(\alpha)=\frac{1}{[n(n+\alpha)]^\frac{1}{2}},\
c_n=c_n(\alpha)=\Big[\frac{(n-1)(n+\alpha-1)}{n(n+\alpha)}\Big]^\frac{1}{2}.
\end{equation*}

Так как функции $\lambda_n^\alpha(x)$ отличаются от полиномов $l_n^\alpha(x)$ множителем, не зависящим от номера функции, следовательно, аналогичная рекуррентная формула справедлива и для функций $\lambda_n^\alpha(x)$:
\begin{equation*}
\left.\begin{gathered}
\lambda_{0}^{\alpha}(x)=\frac{\sqrt{\rho(x)}}{\sqrt{\Gamma(\alpha+1)}}, \quad \lambda_1^{\alpha}(x)=\frac{\sqrt{\rho(x)}(-x+\alpha+1)}{\sqrt{\Gamma(\alpha+2)}},\\
\lambda_n^{\alpha}(x)=(a_n-b_n x)\lambda_{n-1}^{\alpha}(x)-c_n \lambda_{n-2}^{\alpha}(x), \quad n=2, 3, \ldots
\end{gathered}\right\}.
\end{equation*}
В дальнейшем нам понадобится следующее свойство функций $\lambda_{n}^{\alpha}(x)$.

\textbf{Теорема A.}
\textit{(\cite{Ramlib1}, Theorem 1)
Let $f\in L^p$, $\frac{4}{3}<p<4$. Define $a_n=\int_{0}^{\infty}\lambda^\alpha_n(x)f(x)dx$ and set $S_n=\sum_{k=0}^{n}a_k\lambda^\alpha_k(x)$. Then $\|S_n-f\|_{L^p}\rightarrow0$ as $n\rightarrow\infty$.
}

\subsection{Функции, ортонормированные по Соболеву и порождённые функциями Лагерра}\label{Ram_sec3}

Система функций Лагерра $\lambda_n^\alpha(x)$, определенная равенством \eqref{funcLag}, порождает на $[0, \infty)$ систему функций $\lambda_{r,n}^\alpha(x)$ ($r\in\mathbb{N}$, $n=0, 1, \ldots$) посредством равенств
\begin{equation}\label{Ram_eq12}
\lambda_{r,r+n}^{\alpha}(x) =\frac{1}{(r-1)!}\int\limits_{0}^x(x-t)^{r-1}\lambda_{n}^{\alpha}(t)dt, \quad n=0,1,\ldots.
\end{equation}

\begin{equation}\label{Ram_eq13}
\lambda_{r,n}^{\alpha}(x) =\frac{x^n}{n!}, \quad n=0,1,\ldots, r-1.
\end{equation}

Системы вида \eqref{Ram_eq12}, \eqref{Ram_eq13} в общем случае, когда в качестве порождающей системы используется произвольная ортонормированная система $\varphi_k(x)$ $(k=0,1,\ldots)$, были рассмотрены в работах \cite{Ramlib4,Ramlib41,Ramlib42,Ramlib43,Ramlib44}.
В частности, в работе \cite{Ramlib4} была доказана следующая

\textbf{Теорема B.}
\textit{
Предположим, что функции $\varphi_k(x)$ $(k=0,1,\ldots)$ образуют полную в $L^2_{\rho}(a,b)$ ортонормированную c весом $\rho(x)$ систему на отрезке $[a,b]$. Тогда система $\{\varphi_{r,k}(x)\}_{k=0}^\infty$, порожденная системой $\{\varphi_{k}(x)\}_{k=0}^\infty$ посредством равенств
$$
\varphi_{r,r+k}(x) =\frac{1}{(r-1)!}\int\limits_a^x(x-t)^{r-1}\varphi_{k}(t)dt, \quad k=0,1,\ldots.
$$
$$
\varphi_{r,k}(x) =\frac{(x-a)^k}{k!}, \quad k=0,1,\ldots, r-1,
$$
полна  в $W^r_{L^2_{\rho}(a,b)}$ и ортонормирована относительно скалярного произведения
$$
\langle f,g\rangle=\sum_{\nu=0}^{r-1}f^{(\nu)}(a)g^{(\nu)}(a)+\int_{a}^{b} f^{(r)}(t)g^{(r)}(t)\rho(t) dt.
$$
}
Из этой теоремы следует, что система функций $\lambda_{r,n}^\alpha(x)$, определенная равенствами \eqref{Ram_eq12}, \eqref{Ram_eq13}, полна в $W^r_{L^2}$ и ортонормирована относительно скалярного произведения \eqref{Ram_eq1}.

Далее, из \eqref{Ram_eq12}, \eqref{Ram_eq13} и формулы дифференцирования под знаком интеграла \cite[п. 509, с. 667]{Ramlib5} следует, что для п.в. $x\in[0,\infty)$
\begin{equation}\label{Rameqformu}
(\lambda_{r,k}^\alpha(x))^{(\nu)} =
\begin{cases}
\lambda_{r-\nu,k-\nu}^\alpha(x),&\text{если $0\le\nu\le r-1$, $r\le k$,}\\
\lambda_{k-r}^\alpha(x),&\text{если $\nu=r\le k$,}\\
\lambda_{r-\nu,k-\nu}^\alpha(x),&\text{если $\nu\le k< r$,}\\
0,&\text{если $k< \nu\le r$}.
\end{cases}
\end{equation}

Из \eqref{Ram_eq1}, \eqref{Ram_eq12}--\eqref{Rameqformu} нетрудно увидеть, что ряд Фурье функции $f\in W^r_{L^2}$ по системе  $\{\lambda^\alpha_{r,k}(x)\}_{k=0}^\infty$:
\begin{equation*}
f(x)\sim \sum_{k=0}^{\infty}c_{r,k}^\alpha(f)\lambda_{r,k}^\alpha(x)
\end{equation*}
имеет следующий вид
\begin{equation}\label{RamFourierseries2}
f(x)\sim \sum_{k=0}^{r-1}f^{(k)}(0)\frac{x^k}{k!}+\sum_{k=r}^{\infty} c_{r,k}^\alpha(f)\lambda_{r,k}^\alpha(x),
\end{equation}
где
\begin{equation}\label{RamFouriercoeff2}
c_{r,k}^\alpha(f)=\int\limits_0^\infty f^{(r)}(t)\lambda_{k-r}^\alpha(t)dt, \quad k=r, r+1, \ldots.
\end{equation}
Заметим, что ряд Фурье \eqref{RamFourierseries2} можно определить для любой функции $f\in W^r_{L^p}$, $p>1$. С этой целью покажем существование коэффициентов $c_{r,k}^\alpha(f)$, определенных равенством \eqref{RamFouriercoeff2}:
$$
|c_{r,k}^\alpha(f)|\leq \left(\int\limits_0^\infty|f^{(r)}(t)|^p dt\right)^{\frac{1}{p}}
\left(\int\limits_0^\infty|\lambda_{k-r}^\alpha(t)|^q dt\right)^{\frac{1}{q}}\leq M\|f^{(r)}\|_{L^p},\ k=r, r+1, \ldots,
$$
где $M$ некоторое положительное число.
Рассмотрим вопрос о равномерной сходимости ряда Фурье \eqref{RamFourierseries2} к функции $f\in W^r_{L^p}$.

\begin{theorem}\label{Ram_thm1}
Пусть $\alpha\geq0$, $0\leq A<\infty$, $\frac{4}{3}<p<4$, $f\in W^r_{L^p}$. Тогда ряд \eqref{RamFourierseries2} равномерно на $[0, A]$ сходится к функции $f$.
\end{theorem}

\subsection{Рекуррентные соотношения для функций $\lambda_{r,r+n}^{\alpha}(x)$}\label{Ram_sec4}
Заметим, что по построению справедливы следующие равенства: $\lambda_{0,n}^\alpha(x)=\lambda_{n}^\alpha(x),$ $\lambda_{1,0}^\alpha(x)=1,$ $\lambda_{1,1}^\alpha(x)=\int\limits_0^x\lambda^\alpha_0(t)dt$.

\begin{theorem}\label{Ram_thm2}
Пусть $\alpha>-1$. Тогда справедливы следующие рекуррентные соотношения:
\begin{equation*}
\lambda_{r,n}^\alpha(x)=\frac{x}{n}\lambda_{r,n-1}^\alpha(x), \ \ 1\leq n\leq r-1;
\end{equation*}
\begin{equation*}
r\lambda_{r+1,r+1}^\alpha(x)=(x-2r-\alpha)\lambda_{r,r}^\alpha(x)+2x\lambda_{r-1,r-1}^\alpha(x), \ \ r\geq 1;
\end{equation*}
\begin{equation}\label{Ram_eq16}
\sqrt{(n+1)(n+\alpha+1)}\lambda_{1,n+2}^\alpha(x)= 2x\lambda_{n}^{\alpha}(x)-\lambda_{1,n+1}^{\alpha}(x)+
\sqrt{n(n+\alpha)} \lambda_{1,n}^{\alpha}(x), \ n\geq 1;
\end{equation}
$$
r\lambda_{r+1,r+n}^\alpha(x)=\sqrt{n(n+\alpha)}\lambda_{r,r+n}^{\alpha}(x)+
$$
\begin{equation*}
\left(x - 2n-\alpha+1\right)\lambda_{r,r+n-1}^{\alpha}(x)
+\sqrt{(n-1)(n+\alpha-1)}\lambda_{r,r+n-2}^{\alpha}(x), \ r\geq 1, \ n=2, 3, \ldots.
\end{equation*}
\end{theorem}

\begin{remark}
Формула \eqref{Ram_eq16} справедлива и для $n=0$.
\end{remark}

\subsection{Асимптотические свойства функций $\lambda_{1,1+n}^{0}(x)$}
В этом пункте мы рассмотрим вопрос о поведении функций $\lambda_{1,1+n}^{0}(x)$ при $n\rightarrow\infty$, $0\leq x\leq \omega$, где $\omega$ -- некоторое фиксированное положительное число.
\begin{theorem}\label{Ramtheo3}
Справедлива следующая асимптотическая формула
\begin{equation*}
\lambda_{1,1+n}^{0}(x) = \frac{xe^{-\frac{x}{2}}}{n+1}L_n^1(x)+\frac{x^2e^{-\frac{x}{2}}}{2(n+1)(n+2)}L_n^2(x)+R_n(x),
\end{equation*}
в которой для остаточного члена $R_n(x)=\frac{1}{4(n+1)(n+2)}\int\limits_0^x t^2e^{-\frac{t}{2}}L_{n}^{2}(t)dt$ справедливы следующие оценки:
\begin{equation*}
|R_n(x)|\leq
c \left\{\begin{gathered}
\frac{1}{n^3},\ если 0\leq x\leq \frac{1}{n},\\
\frac{1}{n^\frac{7}{4}},\ \frac{1}{n}\leq x\leq \omega.
\end{gathered}\right.
\end{equation*}
\end{theorem}
Далее из теоремы \ref{Ramtheo3} и оценок \eqref{Ram_est1}, \eqref{Ram_est2} вытекает следующее утверждение.
\begin{corollary}
Имеют место следующие оценки
$$
|\lambda_{1,n}^{0}(x)|\leq c
\begin{cases}
  \frac{1}{n}, & 0\leq x\leq \frac{1}{n} \\
  \frac{1}{n^{3/4}}, & \frac{1}{n}< x\leq \omega.
\end{cases}
$$
\end{corollary}

\subsection{Заключение}
Найдены достаточные условия на параметры $\alpha$ и $p$, которые обеспечивают равномерную сходимость на отрезке $[0,A]$ ряда Фурье функции $f\in W_{L^p}^r$ по системе $\{\lambda_{r,n}^{\alpha}(x)\}_{k=0}^\infty$. Для системы функций $\{\lambda_{r,n}^{\alpha}(x)\}_{k=0}^\infty$ получены рекуррентные соотношения, которые можно использовать при вычислении их значений при любом $x$ и $n$. Кроме того, для функций $\lambda_{1,n}^{0}(x)$ получена асимптотическая формула и для $x\in[o,\omega]$ получена оценка для остаточного члена. На основе этой асимптотической формулы и весовых оценок для классических полиномов Лагерра получены оценки для функций $\lambda_{1,n}^{0}(x)$ при $x\in[o,\omega]$.


\begin{Rtwocolbib}
\bibitem{Ramlib1} {Askey R., Wainger S.} Mean convergence of expansions in Laguerre and Hermite series // Amer. J. Math., vol. 87, 1965, pp.~698--708.

\bibitem{Ramlib3} {Bateman H, Erdeyi A.} Higher transcendental functions. Vol. 2. McGraw-Hill, New York-Toronto-London, 1953.

\bibitem{Ramlib5} {Фихтенгольц Г.М.} Курс дифференциального и интегрального исчисления. Том. 2 Москва: Физматлит, 2001.

\bibitem{Ramlib4} {Шарапудинов И.И.} Системы функций, ортогональные по Соболеву, ассоциированные с ортогональной системой // Изв. РАН. Сер. матем., 82:1, 2018. С. 225--258.

\bibitem{Ramlib41} {Шарапудинов И.И., Шарапудинов Т.И.} Полиномы, ортогональные по Соболеву, порожденные многочленами Чебышева, ортогональными на сетке // Изв. вузов. Матем., № 8, 2017. С. 67--79.

\bibitem{Ramlib42} {Шарапудинов И.И.} Аппроксимативные свойства рядов Фурье по многочленам, ортогональным по Соболеву с весом Якоби и дискретными массами // Матем. заметки, 101:4, 2017. С. 611--629.

\bibitem{Ramlib43} {Шарапудинов И.И., Гаджиева З.Д., Гаджимирзаев Р.М.} Системы функций, ортогональных относительно скалярных произведений типа Соболева с дискретными массами, порожденных классическими ортогональными системами // Дагестанские электронные математические известия, вып. 6, 2016. С. 31--60. 

\bibitem{Ramlib44} {Шарапудинов И.И., Гаджиева З.Д., Гаджимирзаев Р.М.} Разностные уравнения и полиномы, ортогональные по Соболеву, порожденные многочленами Мейкснера // Владикавк. матем. журн., 19:2, 2017. С. 58--72.

\bibitem{Ramlib2} {Сеге Г.} Ортогональные многочлены. Физматгиз. Москва. 1962.

\end{Rtwocolbib}

\end{document}
