\documentclass{article}
\usepackage[T2A]{fontenc}
\usepackage[utf8]{inputenc}
\usepackage[english,russian]{babel}
\usepackage[tbtags]{amsmath}
\usepackage{amsfonts,amssymb,mathrsfs,amscd}

%для подключения графики используются стандартные команды, но кроме файла *.eps
%необходимо наличие в текущей директории соответствующего файла *.pdf
%-------------------------------------------------
\usepackage[hyper]{demi}
\newcommand{\No}{\textnumero}
\JournalName{Дагестанские Электронные Математические Известия}

%Пустой аргумент приводит к исчезновению всех атрибутов журнала "Математический
%сборник", файл можно представить в любой другой журнал
%-------------------------------------------------
\numberwithin{equation}{section}
%-------------------------------------------------
\theoremstyle{plain}
\newtheorem{theorem}{Теорема}

\newtheorem{lemma}{Лемма}[section]
\newtheorem{propos}{Предложение}
\newtheorem{corollary}{Следствие}
%-------------------------------------------------
\theoremstyle{definition}
\newtheorem{definition}{Определение}
\newtheorem{proof}{Доказательство}\def\theproof{}
\newtheorem{remark}{Замечание}
%-------------------------------------------------
\def\Re{\operatorname{Re}}
\def\Im{\operatorname{Im}}
\def\const{\mathrm{const}}
\def\RR{\mathbb R}
\def\CC{\mathbb C}
\def\NN{\mathbb N}
\def\RS{\mathfrak R}
\def\bz{\mathbf z}
\def\sH{\mathscr H}
\def\HH{\mathscr H}
\def\mro{\widehat\rho}
%-------------------------------------------------

\setcounter{page}{33}
\begin{document}

\title{О приближенном решении задачи Коши для системы ОДУ посредством системы
$1,\, x,\, \{\frac{\sqrt{2}}{\pi n}\sin(\pi nx)\}_{n=1}^\infty$}



\author[I.\,I.~Sharapudinov]{И.\,И.~Шарапудинов}

\address{Дагестанский научный центр РАН\\
Владикавказский научный центр РАН }


\email{sharapud@mail.ru}
%второй автор

\date{07.06.2018}
\udk{517.538}

\maketitle

\begin{fulltext}


\begin{abstract}
Рассмотрена система функций  $\xi_0(x)=1,\, \{\xi_n(x)=\sqrt{2}\cos(\pi nx)\}_{n=1}^\infty$ и порожденная ею система
$$
 \xi_{1,0}(x)=1,\, \xi_{1,1}(x)=x,\, \xi_{1,n+1}(x)=\int_0^x \xi_{n}(t)dt=\frac{\sqrt{2}}{\pi n}\sin(\pi nx),\, n=1,2,\ldots,
$$
 которая является ортонормированной по Соболеву относительно скалярного произведения  вида $<f,g>=f'(0)g'(0)+\int_{0}^{1}f'(t)g'(t)dt$. Показано, что ряды и суммы Фурье по системе $\{\xi_{1,n}(x)\}_{n=0}^\infty$  является удобным и весьма эффективным инструментом приближенного решения задачи Коши для систем нелинейных обыкновенных дифференциальных уравнений (ОДУ).

Библиография:  8 названий

$$$$

We consider a system of functions $\xi_0(x)=1,\, \{\xi_n(x)=\sqrt{2}\cos(\pi nx)\}_{n=1}^\infty$ and the system 
$$
\xi_{1,0}(x)=1,\, \xi_{1,1}(x)=x,\, \xi_{1,n+1}(x)=\int_0^x \xi_{n}(t)dt=\frac{\sqrt{2}}{\pi n}\sin(\pi nx),\, n=1,2,\ldots,
$$
generated by it, which is Sobolev orthonormal with respect to a scalar product of the form $<f,g>=f'(0)g'(0)+\int_{0}^{1}f'(t)g'(t)dt$. It is shown that the Fourier series and sums with respect to the system $\{\xi_{1,n}(x)\}_{n=0}^\infty$ are a convenient and very effective tool for the approximate solution of the Cauchy problem for systems of nonlinear ordinary differential equations (ODEs).

Bibliography: 8 items

\end{abstract}

\begin{keywords}
	задача Коши, ОДУ, ряды Фурье, суммы Фурье, приближенное решение
	
	{\bf Keywords:}
	Cauchy problem, ODE, Fourier series, Fourier sums, \\ approximate solution
\end{keywords}

\markright{ Функции, ортогональные по Соболеву }

\footnotetext[0]{Работа выполнена при финансовой поддержке РФФИ (проект 16-01-00486a)}








\section{Введение}\label{s1}
В настоящей работе мы продолжаем исследования, связанные с применением ортогональных по Соболеву систем функций при приближенном решении задачи Коши для системы ОДУ вида
\begin{equation}\label{1.1}
y'(x)=F(x,y), \quad y(a)=y^0,
\end{equation}
где $F=(F_1,\ldots,F_m)$, $y=(y_1,\ldots,y_m)$, $y(a)=(y_1^0,\ldots,y_m^0)$,
начатое в работах \cite{SharDagElec7} -- \cite{SharIzv2018}.
  Рассматривая задачу Коши \eqref{1.1}, мы будем накладывать на вектор-функцию   $F(x,y)$ определенные условия, соблюдение которых гарантирует сходимость некоторых итерационных процессов, сконструированных с целью найти приближенные  значения коэффициентов  в разложении искомого решения задачи Коши в ряд  по ортонормированным по Соболеву   функциям $\xi_{1,k}$. Вектор-функцию   $F(x,y)$  будем считать непрерывной в некоторой замкнутой  области $\bar G$ переменных $(x,y)$, содержащей точку $(a,y_0)$ и такой, что  $[a,b]\times\mathbb{R}^m\subset\bar G$.  Один из возможных подходов к достижению намеченной цели основан на предположении о возможности  разбиения отрезка $[a,b]$ на несколько частей точками $a=a_0<a_1<\cdots<a_n=b$ так, чтобы на каждом из промежутков $[a_i,a_{i+1}]$ вектор-функцию   $F(x,y)$ по переменной $y$ удовлетворяла условию Липшица. Через $y_i(x)$ обозначим решение уравнения $y'(x)=F(x,y)$ с начальным условием $y(a_i)= y^0$ при $i=0$ и  $y_i(a_i)= y_i^0= \tilde y_{i-1}(a_i)$,  если $1\le i\le n-1$, где $\tilde y_i(x)$ -- некоторая функция, которая является аппроксимирующей для $y_i(x)$ при $x\in[a_i,a_{i+1}]$.   Конструирование функции $\tilde y_i(x)$  в \cite{SharDagElec7,SharMagDagElec8}  базируется на преобразовании уравнения \eqref{1.1} путем замены переменных
   $x=a_i+ht$ к виду
   \begin{equation}\label{1.2}
\phi'_i(t)=hf(t,\phi_i), \quad \phi_i(0)=y_i^0,\quad 0\le t\le1,
\end{equation}
где $h=a_{i+1}-a_i$, $\phi_i(t)=y_i(a_i+ht)$, $f(t,\phi_i)=F(a_i+ht,\phi_i)$, и взятии в качестве  $\tilde y_i(a_i+ht)$   частичной суммы разложения на $[0,1]$ функции $\phi_i(t)$ с $i=0,\ldots,n-1$ в ряд Фурье по системе  $\varphi^1=\{\varphi_{1,k}\}$, образующей ортонормированную с некоторым весом $\rho(x)$ систему  относительно скалярного произведения типа Соболева следующего вида
  \begin{equation}\label{1.3}
\langle f,g\rangle=f(0)g(0)+\int_{0}^1f'(x)g'(x)\rho(x)dx.
\end{equation}
  Что касается функции $\tilde y(x)$, аппроксимирующей решение $y(x)$ исходной задачи Коши $y'(x)=F(x,y)$ на всем отрезке $[a,b]$  с начальным условием $y(0)=y^0$, то она может быть сконструирована  \cite{SharDagElec7,SharMagDagElec8} путем <<склеивания>> аппроксимирующих функций $\tilde y_i(x)$ с $i=0,1,\ldots, n-1$.

  В настоящей статье предлагается аналогичный подход решения поставленной задачи с той лишь разницей, что вместо  $x=a_i+ht$ предлагается использовать другую замену переменных. А именно, рассмотрим, к примеру, задачу Коши \eqref{1.1} на отрезке $[0,h]$.  Произведем замену переменной $x=h\sin\pi t$ и положим $\eta(t)=y(h\sin\pi t)-y^0$. Тогда относительно новой переменной $t$ система \eqref{1.1} принимает вид $\eta'(t)=h\pi F(h\sin \pi t, y^0+\eta)\cos\pi t, \quad \eta(0)=0,\quad 0\le t\le1/2$. Но нам будет удобно рассмотреть эту задачу на всем отрезке $[0,1]$, другими словами, мы рассмотрим задачу Коши вида
\begin{equation}\label{1.4}
\eta'(t)=hf(t,\eta), \quad \eta(0)=0,\quad 0\le t\le1,
\end{equation}
где $f(t,\eta)=\pi F(h\sin \pi t, y^0+\eta)\cos\pi t$, причем $\eta(1)=y(h\sin\pi)-y^0=y(0)-y^0=0$.  Существенная разница между  системами уравнений \eqref{1.2} и \eqref{1.4} заключается в том, что решение $\eta(t)$ задачи Коши \eqref{1.4} можно   2-периодически продолжить на всю ось $\mathbb{R}$ так, чтобы продолженная вектор-функция $\eta(t)$ будет нечетной и непрерывно дифференцируемой.
Поэтому, ряд Фурье для решения уравнения \eqref{1.4} по ортонормированной  по Соболеву относительно скалярного произведения   \eqref{1.3} системе $\{\xi_{1,k}\}_{k=0}^\infty$, которая определяется \cite{SharDagElec7,SharMagDagElec8} посредством равенств
\begin{equation*}
 \xi_{1,0}(t)=1,\quad \xi_{1,1}(t)=t,\quad \xi_{1,n+1}(t)=\frac{\sqrt{2}}{\pi n}\sin(\pi nt),\quad n=1,2,\ldots,
\end{equation*}
 приобретает вид
\begin{equation}\label{1.5}
 \eta(t)=\eta(0)+\sum_{j=1}^\infty \hat\eta_{1,j}\xi_{1,j}(t)=\sqrt{2} \sum\nolimits_{k=1}^\infty \hat \eta_{1,k+1}\frac{\sin(\pi kt)}{\pi k},
 \end{equation}
где $\hat \eta_{1,1}=\int_0^1\eta'(\tau)d\tau=0$,
  \begin{equation}\label{1.6}
\hat \eta_{1,k+1}=(\widehat{\eta_1}_{1,k+1},
\ldots,\widehat{\eta_m}_{1,k+1})=\sqrt{2}\int_{0}^1 \eta'(\tau)\cos(\pi k\tau)d\tau\quad(k\ge1).
\end{equation}
Полагая
\begin{equation*}
b_k(\eta)=2\int_{0}^1 \eta(\tau)\sin(\pi k\tau)d\tau=\int_{-1}^1 \eta(\tau)\sin(\pi k\tau)d\tau,
\end{equation*}
мы можем придать равенству \eqref{1.5}  еще такой вид
\begin{equation*}
 \eta(t)=\sum\nolimits_{k=1}^\infty b_k(\eta)\sin(\pi kt).
 \end{equation*}
Отсюда видно, что задачи об исследовании в пространствах $L^2(0,1)$ и $C[0,1]$ поведения при $N\to\infty$  величины $ V_N(\eta,t)= \eta(t)-Y_{1,N}(\eta,t)=\sum_{j=N+1}^\infty \widehat{\eta}_{1,j}\xi_{1,j}(t)$,  которая играет ключевую роль в вопросе о скорости сходимости итерационного процесса, сконструированного с целью нахождения приближенных значений коэффициентов \eqref{1.6} в разложении решения $\eta(t)$ в ряд \eqref{1.5},       сводятся  к исследованию в этих же пространствах поведения при $N\to\infty$ остаточного члена
\begin{equation*}
  R_N(\eta,t)= \sum\nolimits_{k=N}^\infty b_k(\eta)\sin(\pi kt)
\end{equation*}
тригонометрического ряда Фурье  функции $\eta(t)$, которая, как уже отмечалось, является нечетной 2-периодической и непрерывно дифференцируемой на всей оси $\mathbb{R}$. Но это -- детально исследованная классическая задача.

\section{Системы ортогональных по Соболеву функций, порожденные косинусами}\label{s2}

  Рассмотрим  систему функций
\begin{equation}\label{2.1}
 \xi_0(x)=1,\quad \{\xi_n(x)=\sqrt{2}\cos(\pi nx)\}_{n=1}^\infty,
 \end{equation}
 которая является  ортонормированной относительно скалярного произведения $<f,g>=\int_0^1f(x)g(x)dx$, т.е.
\begin{equation}\label{2.2}
 <\xi_n,\xi_m>=\int_0^1\xi_n(x)\xi_m(x)dx=\delta_{nm}.
 \end{equation}
Для каждого натурального $r$ система \eqref{1.1} порождает новую систему функций
\begin{equation}\label{2.3}
 \xi_{1,n+r}(x)=\frac{1}{(r-1)!}\int_0^x (x-t)^{r-1}\xi_{n}(t)dt,  n=0,1,2,\ldots.
\end{equation}
Кроме того определим $r$ функций
\begin{equation}\label{2.4}
 \xi_{1,k}(x)=\frac{x^k}{k!},\quad  k=0,1,2,\ldots, r-1.
\end{equation}
В частности, для $r=1$
\begin{equation}\label{2.5}
 \xi_{1,0}(x)=1,\quad \xi_{1,1}(x)=x,\quad \xi_{1,n+1}(x)=\frac{\sqrt{2}}{\pi n}\sin(\pi nx),\quad n=1,2,\ldots.
\end{equation}
Из \eqref{2.3} и \eqref{2.4} следует, что
 \begin{equation}\label{2.6}
(\xi_{r,k}(x))^{(\nu)} =\begin{cases}\xi_{r-\nu,k-\nu}(x),&\text{если $0\le\nu\le r-1$, $r\le k$,}\\
\xi_{k-r}(x)&\text{если  $\nu=r\le k$,}\\
\xi_{r-\nu,k-\nu}(x),&\text{если $\nu\le k< r$,}\\
0,&\text{если $k< \nu\le r$}.
  \end{cases}
\end{equation}
Из \eqref{2.3} нетрудно вывести также следующее равенство $(k=1,2,\ldots)$
\begin{equation}\label{2.7}
\xi_{r,r+k}(t) =\frac{(-1)^r\sqrt{2}}{(\pi k)^r}\left[
\cos\left(\pi kt+\frac{\pi r}{2}\right)-\sum_{\nu=0}^{r-1} \frac{(\pi kt)^\nu}{\nu!}\cos^{(\nu)}(\pi r/2)\right].
\end{equation}
В частности, если $r=2$, то система $\{\xi_{2,n}(t)\}_{n=0}^\infty$
имеет вид
 \begin{equation}\label{2.8}
\xi_{2,0}(t) =1, \quad \xi_{2,1}(t)=t, \quad \xi_{2,2}(t)=\frac{t^2}2,\,\,\left\{ \xi_{2,k+2}(t)=  \frac{2\sqrt{2}}{(k\pi)^2}\sin^2\frac{k\pi t}{2}\right\}_{k=1}^\infty.
\end{equation}



\section{Дальнейшие свойства системы $\{\xi_{r,k}(x)\}_{k=0}^\infty$ }\label{s3}
При изучении дальнейших свойств системы функций $\{\xi_{r,k}(x)\}$ нам понадобится  пространство Соболева $W^r_{L^p(0,1)}$, состоящее из функций $f(x)$, непрерывно дифференцируемых на $[0,1]$ $r-1$ раз, причем $f^{(r-1)}(x)$ абсолютно непрерывна на $[0,1]$  и $f^{(r)}(x)\in L^p(0,1)$.
Скалярное произведение в пространстве $W^r_{L^2(0,1)}$ определим с помощью равенства
\begin{equation}\label{3.1}
<f,g>=\sum_{\nu=0}^{r-1}f^{(\nu)}(0)g^{(\nu)}(0)+\int_{0}^{1} f^{(r)}(t)g^{(r)}(t) dt.
\end{equation}
 Тогда для $f\in W^r_{L^2(0,1)}$ мы можем определить норму $\|f\|_{W^r_{L^2(0,1)}}=\sqrt{\langle f,g\rangle}$, которая превращает $W^r_{L^2(0,1)}$ в банахово пространство и, стало быть, $W^r_{L^2(0,1)}$ -- гильбертово пространство со скалярным произведением \eqref{3.1}. Следующее утверждение, которое мы приведем вместе с его кратким доказательством,  для общих систем $\{\varphi_{r,k}\}_{k=0}^\infty$, порожденных заданной ортонормированной системой $\{\varphi_{k}\}_{k=0}^\infty$, было доказано  в \cite{SharIzv2018}.

 \begin{lemma}\label{lem3.1} Система $\{\xi_{r,k}(x)\}_{k=0}^\infty$ , порожденная системой $\{\xi_{k}(x)\}_{k=0}^\infty$ посредством равенств \eqref{2.3} и \eqref{2.4}, полна  в $W^r_{L^2(0,1)}$ и ортонормирована относительно скалярного произведения \eqref{3.1}.
 \end{lemma}
\begin{proof}
Из равенства \eqref{2.3} следует, что если $r\le k$ и $0\le\nu\le r-1$, то  $(\xi_{r,k}(x))^{(\nu)}_{x=0}=0$, поэтому
в силу \eqref{2.6},  имеем
$$
<\xi_{r,k},\xi_{r,l}>= \int_{0}^1(\xi_{r,k}(x))^{(r)}(\xi_{r,l}(x))^{(r)} dx=
$$
\begin{equation}\label{3.2}
    \int_{0}^1\xi_{k-r}(x)\xi_{l-r}(x) dx=\delta_{kl},\quad k,l\ge r ,
  \end{equation}
  а из \eqref{2.4} и \eqref{2.6} имеем
\begin{equation}\label{3.3}
  <\xi_{r,k},\xi_{r,l}>=
  \sum_{\nu=0}^{r-1}(\xi_{r,k}(x))^{(\nu)}|_{x=0}(\xi_{r,l}(x))^{(\nu)}|_{x=0}=\delta_{kl},\quad k,l< r.
  \end{equation}
  Очевидно также, что
  \begin{equation}\label{3.4}
  <\xi_{r,k},\xi_{r,l}>=0,\quad \text{если}\quad k< r\le l\quad \text{или} \quad l< r\le k.
  \end{equation}
 Это означает, что функции  $\xi_{r,k}(t)\, (k=0,1,\ldots) $ образуют   в $W^r_{L^2(0,1)}$ ортонормированную  систему относительно скалярного произведения \eqref{3.1}.  Остается убедиться в ее полноте в $W^r_{L^2(0,1)}$. С этой целью покажем, что если для некоторой функции $f\in W^r_{L^2(0,1)}$ и для  всех $k=0,1,\ldots$ справедливы равенства $<f,\xi_k>=0$, то $f(x)\equiv0$. В самом деле, если $k\le r-1$, то  $<f,\xi_{r,k}>=f^{(k)}(0)$, поэтому с учетом того, что $<f,\xi_{r,k}>=0$,  для нашей функции  $f(x)$ формула Тейлора приобретает вид
\begin{equation}\label{3.5}
f(x)={1\over (r-1)!}\int\limits_{0}^x(x-t)^{r-1} f^{(r)}(t)dt.
     \end{equation}
С другой стороны, для всех $k\ge r$ имеем
$$
 0= <f,\xi_{r,k}>=\int_{0}^1f^{(r)}(x) (\xi_{r,k}(x))^{(r)}dx=
  \int_{0}^1 f^{(r)}(x)\xi_{k-r}(x) dx .
$$
Отсюда и из того, что $\xi_m(x)$ ($m=0,1,\ldots$)  образуют в $L^2(0,1)$ полную ортонормированную систему имеем $f^{(r)}(x)=0$ почти всюду на $[0,1]$. Поэтому   $f(x)\equiv0$. Лемма \ref{lem3.1} доказана.
\end{proof}
   Следуя \cite{SharIzv2018}, мы будем называть систему $\{\xi_{r,k}(x)\}_{k=0}^\infty$ {\it ортонормированной по Соболеву } относительно скалярного произведения \eqref{3.1} и  {\it порожденной} ортонормированной системой $\{\xi_{k}(x)\}_{k=0}^\infty$. Ряд Фурье функции $f(x)\in W^r_{L^2(0,1)}$ по системе  $\{\xi_{r,k}(x)\}_{k=0}^\infty$ имеет смешанный характер, а, более точно, имеет следующий вид
  \begin{equation}\label{3.6}
f(x)= \sum\nolimits_{k=0}^{r-1} f^{(k)}(0)\frac{x^k}{k!}+ \sum\nolimits_{k=r}^\infty \hat f_{r,k}\xi_{r,k}(x),
\end{equation}
где
  \begin{equation}\label{3.7}
\hat f_{r,k}=\int_0^1 f^{(r)}(t) \xi^{(r)}_{r,k}(t)dt=\int_0^1 f^{(r)}(t) \xi_{k-r}(t)dt.
\end{equation}

Следует отметить, что теория  систем функций, ортогональных по Соболеву в последнее время получила \cite{SharIzv2018}  интенсивное развитие. Особенно это касается \cite{MarcelXu} полиномов, ортогональных по Соболеву. Ряды вида \eqref{3.2} для полиномов, ортогональных по Соболеву, порожденных классическими ортогональными полиномами были подробно  исследованы в работах \cite{Shar17}, \cite{Shar13}, следуя которым  будем  называть ряд \eqref{3.2} {\it смешанным рядом} по  системе $\{\xi_{k}(x)\}_{k=0}^\infty$, считая это название условным и сокращенным обозначением полного названия: <<{\it ряд Фурье по системе  $\{\xi_{r,k}(x)\}_{k=0}^\infty$, ортонормированной по Соболеву, порожденной ортонормированной системой $\{\xi_{k}(x)\}_{k=0}^\infty$}>>.

 Заметим, что для  для построения смешанного ряда \eqref{3.6} требуется, чтобы функция $\phi$ была абсолютно непрерывной на $[0,1]$, поэтому следующая теорема носит окончательный характер.
\begin{theorem}\label{th1}
  Если $\phi\in W^1_{L^1(0,1)}$, то ряд Фурье (смешанный ряд) \eqref{3.6} для $r=1$ сходится к функции $\phi(t)$ равномерно относительно $t\in[0,1]$.
 \end{theorem}
\begin{proof} С учетом равенств \eqref{2.5} мы можем переписать \eqref{3.6} для $r=1$ в следующем виде
$$
\phi(t)= \phi(0)+\hat \phi_{1,1}t+\sum\nolimits_{k=1}^\infty \hat \phi_{1,k+1}\xi_{1,k+1}(t)=
$$
\begin{equation}\label{3.8}
 \phi(0)+\hat \phi_{1,1}t+\sqrt{2} \sum\nolimits_{k=1}^\infty \hat \phi_{1,k+1}\frac{\sin(\pi kt)}{\pi k},
\end{equation}
где  в силу \eqref{3.7}
  $$
\hat \phi_{1,k+1}=\sqrt{2}\int_{0}^1 \phi'(t)\cos(\pi kt)dt=
$$
$$
\sqrt{2}\int_{0}^1 [\phi(t)-\phi(0)-(\phi(1)-\phi(0))t]'\cos(\pi kt)dt=
$$
\begin{equation}\label{3.9}
\sqrt{2}\pi k\int_{0}^1 [\phi(t)-\phi(0)-(\phi(1)-\phi(0))t]\sin(\pi kt)dt.
\end{equation}
Из \eqref{3.8} и  \eqref{3.9} имеем
\begin{equation}\label{3.10}
\bar\phi(t)=\phi(t)-\phi(0)-(\phi(1)-\phi(0))t= \sum\nolimits_{k=1}^\infty b_k\sin(\pi kt),
\end{equation}
где
\begin{equation}\label{3.11}
b_k=2\int_{0}^1 \bar\phi(\tau)\sin(\pi k\tau)d\tau=\int_{-1}^1 \bar\phi(\tau)\sin(\pi k\tau)d\tau.
\end{equation}
Правая часть равенства \eqref{3.10} представляет собой  тригонометрический ряд Фурье функции $\bar\phi(t)=  \phi(t)-\phi(0)-(\phi(1)-\phi(0))t$, продолженной на всю  числовую ось по нечетности и $2$-периодически. Далее заметим, из абсолютной непрерывности функции $\phi(t)$ на $[0,1]$ следует, что функция $\bar\phi(t)$ абсолютно непрерывна на $[-1,1]$, поэтому ряд Фурье \eqref{3.10} равномерно на $[-1,1]$ сходится к $\bar\phi(t)$. Это равносильно тому, что ряд \eqref{3.8}  сходится равномерно на $[0,1]$ к своей сумме $\phi(t)$. Теорема \ref{th1} доказана.
\end{proof}

Для порожденной  системы $\{\xi_{1,n}(x)\}_{n=0}^\infty$ мы рассмотрим следующую величину
\begin{equation}\label{3.12}
\kappa_{\xi}=\left(\int_0^1\sum_{k=2}^{\infty}
(\xi_{1,k}(t))^2\cos^2\pi tdt\right)^{\frac12},
\end{equation}
которая играет ключевую роль для  скорости  сходимости итерационных процессов, возникающих при приближении решения задачи Коши для ОДУ суммами  Фурье по системе $\{\xi_{1,n}(x)\}_{n=0}^\infty$.
\begin{lemma}\label{lem3.2}
  Имеет место равенство
 $
\kappa^2_{\xi}=\frac{1}{12}-\frac{1}{2\pi^2}.
$
\end{lemma}
\begin{proof} Из \eqref{1.5} имеем
 $$
\kappa_{\xi}^2=\sum\nolimits_{n=2}^\infty \int_{0}^1 (\xi_{1,n}(x))^2\cos^2\pi xdx=
\sum\nolimits_{n=2}^\infty\frac{2}{(n\pi)^2} \int_{0}^1\cos^2\pi x \sin^2(\pi nx)dx
$$
$$
=\frac1{2\pi^2}(\sum\nolimits_{n=1}^\infty\frac1{n^2}-1)=\frac1{2\pi^2}(\frac{\pi^2}{6}-1)=
\frac{1}{12}-\frac{1}{2\pi^2}.
 $$
 Лемма \ref{lem3.2}  доказана.
\end{proof}

 Отметим некоторые свойства смешанного ряда \eqref{3.6}, непосредственно вытекающие из \eqref{2.6}. Важное значение имеет свойство, которое заключается в том, что его частичная сумма вида
\begin{equation}\label{3.13}
Y_{r,N}(f,x)=\sum_{k=0}^{r-1} f^{(k)}(a)\frac{x^k}{k!}+ \sum_{k=r}^{N} \hat f_{r,k}\xi_{r,k}(x)
\end{equation}
 при   $N\ge r$, $r$-кратно совпадает с исходной функцией $f(x)$ в точке $x=0$, т.е.
\begin{equation}\label{3.14}
(Y_{r,N}(f,x))^{(\nu)}_{x=0}=f^{(\nu)}(0)\quad (0\le\nu\le r-1).
\end{equation}
Кроме того, из \eqref{2.6}  следует, что $(0\le\nu\le r-1)$
$$
 Y_{r,N}^{(\nu)}(f,x)=\sum_{n=0}^{r-1-\nu} f^{(n+\nu)}(0)\frac{x^n}{n!}+
 $$
\begin{equation}\label{3.15}
  \sum_{n=r-\nu}^{N-\nu} \widehat{f^{(\nu)}}_{r-\nu,n}\xi_{r-\nu,n}(x)=Y_{r-\nu,N-\nu}(f^{(\nu)},x),
 \end{equation}
\begin{equation}\label{3.16}
 f^{(\nu)}(x)=\sum_{n=0}^{r-1-\nu} f^{(n+\nu)}(0)\frac{x^n}{n!}+ \sum_{n=r-\nu}^\infty \widehat{f^{(\nu)}}_{r-\nu,n}\xi_{r-\nu,n}(x),
 \end{equation}
\begin{equation}\label{3.17}
 f^{(r)}(x)=\sum_{n=0}^\infty \widehat{f^{(r)}}_{0,n}\xi_{n}(x).
 \end{equation}

Последнее равенство  означает, что ряд, фигурирующий  в его правой части сходится к $f^{(r)}$ в метрике пространства  $L^2(0,1)$. Из предпоследнего равенства и из  \eqref{3.15}, в свою очередь, выводим $(0\le\nu\le r-2)$
 $$
f^{(\nu)}(x)-Y_{r,N}^{(\nu)}(f,x)= \frac{1}{(r-\nu-2)!}\int_0^x (x-t)^{r-\nu-2}(f^{(r-1)}(t)-Y_{r,N}^{(r-1)}(f,t))dt=
$$
  \begin{equation}\label{3.18}
\frac{1}{(r-\nu-2)!}\int_0^x (x-t)^{r-\nu-2}(f^{(r-1)}(t)-Y_{1,N-r+1}(f^{(r-1)},t))dt.
 \end{equation}
Дифференциальные свойства смешанных рядов \eqref{3.6}, выраженные равенствами \eqref{3.15} и \eqref{3.16}, показывают, что их частичные суммы $Y_{r,N}(f,x)$  могут быть использованы в задачах, в которых требуется одновременно приближать заданную дифференцируемую функцию и несколько ее производных. В дальнейшем мы существенно воспользуемся этими свойствами при рассмотрении вопроса о представлении решения задачи Коши для систем дифференциальных уравнений в виде ряда Фурье по системе $\{\xi_{r,k}(x)\}_{k=0}^\infty$.



\section{Система функций $\xi_{1,n}(x)$ и задача Коши для ОДУ  }\label{s4}
 В  вводном параграфе \S \ref{s1}  было  отмечено, что система функций  $\xi_{1,n}(x)$ тесно связана с задачей Коши \eqref{1.1} для систем ОДУ (вообще говоря, нелинейных). Согласно общей концепции, предложенной в работах   \cite{SharDagElec7,SharMagDagElec8}, функция $\tilde y(x)$, аппроксимирующая решение $y(x)$ исходной задачи Коши \eqref{1.1} на всем отрезке $[a,b]$  с начальным условием $y(a)=y^0$ конструируется путем <<склеивания>>  функций $\tilde y_i(x)$ с $i=0,1,\ldots, n-1$, аппроксимирующих решения этого же уравнения на частичных отрезках $[a_i,a_{i+1}]$ с указанными там соответствующими начальными условиями. При этом возникает задача об оценке погрешности $\max_{x\in[a,b]}|y(x)-\tilde y(x)|$. Для простоты выкладок мы ограничимся в настоящем параграфе рассмотрением этого вопроса в одномерном случае для задачи Коши вида
\begin{equation}\label{4.1}
u'(x)=F(x,u), \quad u(a)=u_0,
\end{equation}
 в которой функцию   $F(x,u)$  будем считать непрерывной в некоторой области $\bar G$, которая в качестве своего подмножества содержит множество $[a,b]\times\mathbb{R}$. Будем также предполагать, что функция $F(x,u)$ по переменной $u$  удовлетворяет условиям
 \begin{equation}\label{4.2}
|F(x,q')-F(x,q'')|\le \lambda_i|q'-q''|, \quad a_i\le x \le a_{i+1},
\end{equation}
где $a=a_0<a_1<\cdots<a_n=b$,  $\lambda_i\geq0$ ($0\leq i\leq n-1$). Требуется с заданной точностью приблизить на $[a,b]$ функцию  $u=u(x)$, которая является   решением задачи Коши \eqref{4.1}. Для этого нам понадобятся некоторые вспомогательные утверждения.

\begin{lemma}\label{lem4.1} Пусть непрерывная функция  $F(x,u)$, заданная в области $\bar  G$ переменных $(x,u)$, удовлетворяет условию Липшица $|F(x,q')-F(x,q'')|\le\lambda|q'-q''|$, в котором постоянная $\lambda$ не зависит от $x\in [\alpha,\beta]$.  Далее, пусть  $(\alpha,u_0)\in \bar G$, $(\alpha,v_0)\in \bar  G$, $u= u(x)$ и $v =v(x)$ два решения уравнения $u'(x)=F(x,u)$   на $[\alpha,\beta]$  с соответствующими начальными условиями  $u(\alpha)=u_0$,  $v(\alpha)=v_0$. Тогда если $(\beta-\alpha)\lambda<1$, то имеет место неравенство
$$
\max_{t\in[\alpha,\beta]}|u(t)-v(t)|\le \frac{|u_0-v_0|}{1-(\beta-\alpha)\lambda}.
$$
\end{lemma}
\begin{proof} Применяя формулу Ньютона-Лейбница к разности \linebreak $u(t)-v(t)$, мы можем  записать
$$
u(t)-v(t)=u_0-v_0+ \int_{\alpha}^t(u'(\tau)-v'(\tau))d\tau=
$$
$$
u_0-v_0+ \int_{\alpha}^t(F(\tau,u(\tau))-F(\tau,v(\tau)))d\tau.
$$
Воспользуемся теперь условием $|F(t,q')-F(t,q'')|\le\lambda|q'-q''|$, тогда из предыдущего равенства находим
$$
|u(t)-v(t)|\le|u_0-v_0|+\lambda\int_{\alpha}^t|u(\tau)-v(\tau)|d\tau\le
$$
$$
|u_0-v_0|+(\beta-\alpha) \lambda\max_{\tau\in[\alpha,\beta]}|u(\tau)-v(\tau)|,
$$
а отсюда, в свою очередь, получаем
$$
 \max_{t\in[\alpha,\beta]}|u(t)-v(t)|\le |u_0-v_0|+(\beta-\alpha)\lambda\max_{t\in[\alpha,\beta]}|u(t)-v(t)|.
$$
Лемма \ref{lem4.1} доказана.
\end{proof}
\begin{lemma}\label{lem4.2} Пусть непрерывная функция  $F(x,u)$, задана в области $\bar  G$ переменных $(x,u)$. Предположим, что $[a,b]\times\mathbb{R}\subset \bar G$ и пусть отрезок  $[a,b]$ разбит на части точками $a=a_0<a_1<\cdots<a_n=b$.   Далее, пусть  $(a_i,u_i^0)\in \bar G$ при $i=0,\ldots,m-1$, $u_i= u_i(x)$ -- решения уравнения $u'(x)=F(x,u)$   на $[a_i,b]$, удовлетворяющие соответствующим начальным условиям  $u_i(a_i)=u_i^0$. Будем считать, что $F(x,u)$ по переменной $u$  удовлетворяет при каждом $i=0,\ldots,n-1$ условию Липшица \eqref{4.2}.   Тогда если $\lambda_{k}(a_{k+1}-a_k)<1$ при всех $k=1,\ldots,i$, то для $u(x)=u_0(x)$ имеет место неравенство
$$
\max_{t\in[a_i,a_{i+1}]}|u(t)-u_i(t)|\le \sum_{l=1}^{i}Q_{l}^{i}|u_{l-1}(a_{l})-u_{l}(a_l)|,
$$
где $1\le i\le n-1$,
$$
Q_l^i=\prod_{j=0}^{i-l}\frac{1}{1-\lambda_{i-j}(a_{i-j+1}-a_{i-j})}.
$$
\end{lemma}
\begin{proof}
  Для $1\le i\le m-1$, $t\in[a_i,a_{i+1}]$ имеем
  $$
  |u(t)-u_i(t)|\le \sum_{l=1}^{i}|u_{l-1}(t)-u_l(t)|,
  $$
  а в силу леммы \ref{lem4.1}
  $$
  |u_{l-1}(t)-u_l(t)|\le \frac{|u_{l-1}(a_i)-u_{l}(a_i)|}{1-\lambda_i(a_{i+1}-a_i)}\le
  $$
  $$
  \frac{|u_{l-1}(a_{i-1})-u_{l}(a_{i-1})|}
  {(1-\lambda_i(a_{i+1}-a_i))(1-\lambda_{i-1}(a_{i}-a_{i-1}))}\le\ldots
  $$
  $$
  \le\frac{|u_{l-1}(a_{l})-u_{l}(a_{l})|}
  {(1-\lambda_i(a_{i+1}-a_i))\cdots(1-\lambda_{l}(a_{l+1}-a_{l}))}.
  $$
 Утверждение леммы \ref{lem4.2} немедленно вытекает из приведенных неравенств.
\end{proof}
Рассмотрим вопрос о конструировании функции $\tilde u(x)$, определенной на $[a,b]$, которая с заданной точностью будет   приближать  решение $u(x)$ уравнения  $u'(x)=F(x,u)$  на $[a,b]$  c начальным условием  $u(a)=u^0_0$. С этой целью мы обратимся к лемме \ref{lem4.2}.
Предположим, что на каждом из отрезков $[a_i,a_{i+1}]$, фигурирующих в лемме \ref{lem4.2}, мы построили непрерывную функцию $\tilde u_i(x)$,  приближающую решение $u_i(x)$ уравнения $u'(x)=F(x,u)$   на $[a_i,a_{i+1}]$  c начальным условием  $u_i(a_i)=u_i^0$ так, чтобы  было  $\tilde u_i(a_i)= u_i(a_i)$ ($0\le i\le n-1$), а при $1\le i\le n-1$ выполняются условия $\tilde u_i(a_i)=\tilde u_{i-1}(a_i)$.  На всем отрезке $[a,b]$ определим непрерывную функцию $\tilde u(x)$, полагая
$\tilde u(x)=\tilde u_i(x)\quad \text{при}\quad x\in [a_i,a_{i+1}], i=0,\ldots, n-1$.
 Положим $\delta_i=\max_{x\in[a_i,a_{i+1}]}|\tilde u_i(x)- u_i(x)|$. Тогда из леммы \ref{lem4.2} имеем
$$
\max_{t\in[a_i,a_{i+1}]}|u(t)-\tilde u_i(t)|\le \max_{t\in[a_i,a_{i+1}]}|u(t)- u_i(t)|+\delta_i\le
$$
$$
 \sum_{l=1}^{i}Q_{l}^{i}|u_{l-1}(a_{l})-u_{l}(a_l)|+\delta_i=
  \sum_{l=1}^{i}Q_{l}^{i}|u_{l-1}(a_{l})-\tilde u_{l-1}(a_{l})| +\delta_i\le
 \sum_{l=1}^{i}Q_{l}^{i}\delta_{l-1}+\delta_i.
$$
 Поэтому, мы можем сформулировать следующее утверждение.

\begin{lemma}\label{lem4.3}
  Пусть соблюдены условия леммы \ref{lem4.2}   и, кроме того,\\ $\lambda_{k}(a_{k+1}-a_k)<1$ при всех $k=1,\ldots,n-1$ . Тогда для $u(t)=u_0(t)$  имеют место следующие неравенства
  $$
  \max_{t\in[a_0,a_1]}|u(t)-\tilde u(t)|\le \delta_0,
  $$
$$
\max_{t\in[a_i,a_{i+1}]}|u(t)-\tilde u(t)|\le\sum\nolimits_{l=1}^iQ_{l}^i\delta_{l-1}+\delta_i, \quad 1\le i\le n-1.
$$
\end{lemma}
Перейдем к конструированию аппроксимирующих  функций $\tilde u_i$, удовлетворяющих условиям  леммы \ref{lem4.2}. Мы покажем, что удобным и весьма эффективным инструментом  решения этой задачи являются ряды Фурье по ортогональным по Соболеву функциям $\xi_{1,k}(x)$, рассмотренные нами в \S \ref{s3}. Следует при этом отметить, что выбор разбиения $a=a_0<a_1<\cdots<a_n=b$, фигурирующего в леммах \ref{lem4.2} и \ref{lem4.3}, оптимального в том или ином смысле, существенно зависит от свойств функции $F(x,u)$, в том числе  и от того, насколько большими являются константы $\lambda_i$ в условиях Липшица $|F(x,q')-F(x,q'')|\le\lambda_i|q'-q''|$ (  $x\in [a_i,a_{i+1}]$) с $i=0,\ldots, n-1$. Не останавливаясь на подробном обсуждении этого вопроса, мы перейдем к задаче о приближении  искомого решения $u_i(x)$ задачи Коши на отрезке $[a_i,a_{i+1}]$ с заданным начальным условием $u_i(a_i)$, где $0\le i\le n-1$.
  Идея построения функций $\tilde u_i(x)$, аппроксимирующих на соответствующих отрезках $[a_i,a_{i+1}]$ искомые  решения $u_i(x)$ и таких, что $u_i(a_i)=\tilde u_i(a_i)$ при $0\le i\le n-1$, а если $1\le i\le m-1$ удовлетворяют условиям
  $\tilde u_i(a_i)= \tilde u_{i-1}(a_i)$, состоит в следующем.  Предположим, что $[a,b]\times\mathbb{R}\subset \bar G$, а  функция $F(x,u)$ непрерывна в области $\bar G$ и по переменной $u$  удовлетворяет условиям \eqref{4.2}. Положим $h=a_{i+1}-a_{i}$, $x=a_i+h\sin\pi t$, $\phi(t)=u_i(a_i+h\sin\pi t)-u_i(a_i)$. Относительно новой переменной $t\in [0,1]$ уравнение \eqref{4.1}, рассматриваемое на отрезке $[a_i,a_i+h]$,  принимает вид $\phi'(t)=h \pi F(a_i+h\sin\pi t,u_i(a_i)+\phi)\cos\pi t$,  $0\le t\le 1/2$. Но нам будет удобно его рассмотреть на всем отрезке $[0,1]$:
\begin{equation}\label{4.3}
\phi'(t)=h\pi F(a_i+h\sin\pi t,u_i(a_i)+\phi)\cos\pi t, \quad \phi(0)= 0,\quad 0\le t\le 1,
\end{equation}
где  $u_0(a_0)=u_0(a)$, $ u_i(a_i)= \tilde u_{i-1}(a_i)$ при $1\le i\le n-1$. Для конструирования функции $\tilde u_0(x)$, аппроксимирующей решение $u_0(x)$ уравнения \eqref{4.1} на отрезке $[a_0,a_0+h]$ с начальным условием $u_0(a_0)=u(a)$ мы представим функцию $f(t)=\phi(t)=u_0(a+h\sin\pi t)-u_0(a)$ в виде ряда Фурье \eqref{3.6}, который, в силу теоремы \ref{th1} равномерно на $[0,1]$ сходится и  принимает вид \eqref{3.8}. Поскольку, как нетрудно показать, решение задачи Коши \eqref{4.3} удовлетворяет условию $\phi(1)= 0$, то ряд \eqref{3.8}, в свою очередь, можно записать в следующем виде
$$
\phi(t)= \sqrt{2} \sum\nolimits_{k=1}^\infty \hat \phi_{1,k+1}\frac{\sin(\pi kt)}{\pi k},
$$
где
$$
\hat \phi_{1,k+1}=\int_{0}^1 \phi'(t)\xi_{k}(t)dt=
\sqrt{2}\pi k\int_{0}^1 \phi(t)\sin(\pi kt)dt.
$$
Отсюда имеем
\begin{equation}\label{4.4}
\phi(t)=  \sum\nolimits_{k=1}^\infty b_k(\phi)\sin(\pi kt),
\end{equation}
где
$$
b_k(\phi)=\int_{-1}^1 \phi(t)\sin(\pi kt)dt.
$$
Положим для $x=a_0+h\sin\pi t$
$$
\tilde u_0(x)=u_0(a_0)+Y_{1,N_0}(\phi, t)=u_0(a_0)+ \sum\nolimits_{k=1}^{N_0-1} b_k(\phi)\sin(\pi kt),\quad 0\le t\le 1,
$$
где $Y_{1,N_0}(\phi, t)$ -- частичная сумма ряда Фурье \eqref{4.4}, $N_0$ -- произвольное натуральное число, которое при решении конкретной задачи следует  выбрать, в зависимости от требуемой точности приближения искомого решения $u_0(x)=u_0(a_0)+\phi(t)=  u_0(a_0)+\sum\nolimits_{k=1}^\infty b_k(\phi)\sin(\pi kt)$ на $[a_0,a_1]$ аппроксимирующей функцией $\tilde u_0(x)$.  Аппроксимирующая функция $\tilde u_i(x)$ с  $1\le i\le n-1$ конструируем совершенно аналогично, выбирая в качестве  начального значения искомого решения $u_i(x)$ на отрезке $[a_i,a_{i+1}]$ число $\tilde u_{i-1}(a_i)$.
Из свойства \eqref{3.14}, которым обладают частичные суммы $Y_{1,N}(f,t)$ смешанного ряда \eqref{3.6}, непосредственно вытекает, что если $1\le i\le m-1$, то $\tilde u_{i-1}(a_i)=u_i(a_i)=\tilde u_i(a_i)$. Тем самым, аппроксимирующие функции  $\tilde u_i(x)$ подчиняются условиям леммы \ref{lem4.2} и, стало быть, справедливо неравенство для погрешности $ |u_i(x)-\tilde u_i(x)|$, установленное в этой лемме. Далее, заметим, что погрешности   $\delta_i $, фигурирующие в лемме  \ref{lem4.3}, приобретают вид $\delta_i=\max_{x\in[a_i,a_{i+1}]}|u_i(x)-\tilde u_i(x)|=\max_{t\in[0,1]}|V_{N_i}(\phi,t)|$, где
 $V_{N_i}(\phi,t)=\sum\nolimits_{k=N_i}^\infty b_k(\phi)\sin(\pi kt)$. Отсюда непосредственно возникает задача об исследовании поведения $\max_{t\in[0,1]}|V_{N_i}(\phi,t)|$ при $N_i\to\infty$.   Если мы определим  с помощью равенств $\tilde u(x)= \tilde u_i(x)=u_i(a_i)+Y_{1,N_i}(\phi, t)$  ($x\in [a_i,a_{i+1}]$)  аппроксимирующую функцию для искомого решения $u(x)$  задачи Коши \eqref{3.1}, то для оценки  разности $|u(x)-\tilde u(x)|$  можем использовать неравенства, полученные в лемме \ref{lem4.3}. Следует при этом отметить, что гладкость функции $u(x)$, представляющей собой решение задачи Коши для уравнения \eqref{3.1}, а также   гладкость функций $\phi_i(t)=u_i(a_i+ht)$ по переменной  $t\in[0,1]$,  следовательно, и скорость стремления к нулю величины $\max_{t\in[0,1]}|V_{N_i}(\phi,t)|$ при $N_i\to\infty$,  в свою очередь, существенно зависят от свойств функции $F(x,u)$. На подробном обсуждении этого вопроса мы не остановимся, поскольку это выходит за рамки настоящей работы.


\section{О представлении решения задачи Коши для систем ОДУ рядом Фурье по функциям $\xi_{1,n}(x)$}\label{s5}

Как было показано в предыдущем параграфе на примере одномерной задачи Коши \eqref{4.1}, проблема о конструировании  функции
$\tilde u(x)$, аппроксимирующей ее решение  $u(x)$ на отрезке $[a,b]$ при соблюдении условий Липшица \eqref{4.2} может быть может быть решена путем <<склеивания>> аппроксимирующих функций $\tilde u_i(x)$ для решений $u_i(x)$  той же задачи на частичных отрезках $[a_i,b_i]$, причем конструирование функций $\tilde u_i(x)$ для $i=1,\ldots, n-1$ можно осуществить также, как это сделано для $i=0$ на отрезке $[a,a+h]$. Совершенно аналогично можно поступить и в том случае, когда мы имеем дело с  задачей Коши для системы ОДУ \eqref{1.1}. При этом, не умаляя общности, мы можем считать, что $a=0$. Поэтому, для простоты выкладок, мы ограничимся рассмотрением следующей задачи Коши
\begin{equation}\label{5.1}
y'(x)=F(x,y), \quad y(0)=y^0,\quad 0\le x\le h,
\end{equation}
где $F=(F_1,\ldots,F_m)$, $y=(y_1,\ldots,y_m)$, $y(0)=(y_1^0,\ldots,y_m^0)$.  Рассмотрим вопрос о приближении решения задачи Коши \eqref{5.1},  в которой функцию   $F(x,y)$  будем считать непрерывной в некоторой замкнутой  области $\bar G$ переменных $(x,y)$,  которая содержит точку $(0,y_0)$. При этом предполагается, что   $F(x,y)$ по переменной $y$ удовлетворяет нижеследующему условию Липшица \eqref{5.2} равномерно относительно  $x\in[0,h]$. Кроме того мы будем считать, что $[0,h]\times\mathbb{R}^m\subset\bar G$. Это требование не сужает дальнейшие рассмотрения, так как, не ограничивая  общности,  мы можем, в случае необходимости, продолжить функцию $F(x,y)$ по переменной $y$ на всё $\mathbb{R}^m$, сохраняя свойство ее подчиненности  нижеследующему условию Липшица \eqref{4.3}. Например, если область $\bar G$ такова, что  прямая в $\mathbb{R}^{m+1}$ вида $(x,ty_1,\ldots,ty_m)$ ($t\in\mathbb{R}$) для каждого $x\in[0,h]$ и $(y_1,\ldots,y_m)\in\mathbb{R}^{m}$ пересекается с границей области $\bar G$ не более, чем в двух (граничных для $\bar G$) точках $(x,y')$ и $(x,y'')$, то  $F(x,y)$ можно непрерывно продолжить   на $[0,h]\times\mathbb{R}^m$, считая ее  постоянной на лучах, выходящих из точек  $(x,y')$ и $(x,y'')$ в противоположные направления вдоль прямой $(x,ty_1,\ldots,ty_m)$ ($t\in\mathbb{R}$).

Кроме того мы будем считать, что по переменной $y$ функция $F(x,y)$ удовлетворяет условию Липшица
 \begin{equation}\label{5.2}
\|F(x,a)-F(x,b)\|\le \lambda_0\|a-b\|, \quad 0\le x \le h,
\end{equation}
где $\|(a_1,\ldots,a_m)\|=\sqrt{\sum_{l=1}^ma_l^2}$.

Положим  $x=h\sin\pi t$, $\phi(t)=y(h\sin\pi t)-y^0$. Относительно новой переменной $t$ уравнение \eqref{5.1}   принимает вид $\phi'(t)=h \pi F(h\sin\pi t,y^0+\phi)\cos\pi t$,  $0\le t\le 1/2$. Но нам будет удобно его рассмотреть на всем отрезке $[0,1]$:
\begin{equation}\label{5.3}
\phi'(t)=h\pi F(h\sin\pi t,y^0+\phi)\cos\pi t, \quad \phi(0)= 0,\quad 0\le t\le 1.
\end{equation}
     Для конструирования функции $\tilde y(x)$, аппроксимирующей решение $y(x)$ уравнения \eqref{5.1} на отрезке $[0,h]$ с начальным условием $y(0)=y^0$ мы представим функцию $\phi(t)=y(h\sin\pi t)-y^0$ в виде ряда Фурье \eqref{3.8}. Поскольку, как нетрудно показать, решение задачи Коши \eqref{5.3} удовлетворяет условию $\phi(1)= 0$ и, следовательно, $\hat \phi_{1,1}=\int_0^1\phi'(\tau)d\tau=0$, то ряд \eqref{3.8}, в свою очередь, можно записать в следующем виде
\begin{equation}\label{5.4}
\phi(t)=\sum\nolimits_{k=1}^\infty \hat \phi_{1,k+1}\xi_{1,k+1}(t),
\end{equation}
где $\phi=(\phi_1,\ldots,\phi_m)$, $\phi'=(\phi_1',\ldots,\phi_m')$,
\begin{equation}\label{5.5}
\hat \phi_{1,k+1}=(\widehat{\phi_1}_{1,k+1},\ldots,\widehat{\phi_m}_{1,k+1})=\int_{0}^1 \phi'(\tau)\xi_k(\tau)d\tau.
\end{equation}
 Поскольку, по предположению, функция $F(x,y)$ непрерывна в области $\bar G$, то из \eqref{5.3} следует, что  функция $\phi'(t)$ непрерывна на $[0,1]$ и, следовательно, $\phi_l\in W_{L^2(0,1)}^1$ для всех $1\le l\le m$, поэтому в силу теоремы \ref{th1} ряд \eqref{5.4} сходится равномерно  на $[0,1]$, а его сумма $\phi(t)$ представляет собой нечетную непрерывно дифференцируемую  2-периодическую функцию. Пользуясь свойствами \eqref{2.6} и \eqref{3.16},  мы можем записать равенство
\begin{equation}\label{5.6}
\phi'(t)=\sum_{k=1}^\infty \hat \phi_{1,k+1}\xi_kt),
\end{equation}
 которое справедливо в метрике пространства $L^2(0,1)$.
Положим
$$q(t)=\pi F(h\sin\pi t,y^0+\phi(t))\cos\pi t=\phi'(t)/h$$
и заметим, что в силу  \eqref{5.5}  коэффициенты Фурье функции $q=q(t)$ по системе  $\{\xi_{n}(t)\}_{n=0}^\infty$ имеют вид
\begin{equation}\label{5.7}
 c_k(q)=\int_{0}^1 q(t)\xi_{k}(t)dt=\hat \phi_{1,k+1}/h \quad (k\ge0),
\end{equation}
причем $c_0=\hat \phi_{1,1}/h=0$. С учетом этих равенств мы можем переписать \eqref{5.4} в следующем виде
\begin{equation}\label{5.8}
\phi(t)=  h\sum\nolimits_{k=1}^\infty c_k(q){\xi}_{1,k+1}(t).
\end{equation}



Наша цель состоит в том, чтобы сконструировать итерационный процесс для нахождения приближенных значений коэффициентов $c_k=\hat \eta_{1,k+1}/h$ $(k=1,\ldots)$.
Из  \eqref{5.6} и \eqref{5.7} выводим следующие равенства
\begin{equation}\label{5.9}
c_k(q)=\pi\int\limits_{0}^1 F(h\sin\pi t,y^0+ h\sum_{j=1}^\infty c_j(q){\xi}_{1,j+1}(t))\xi_j(t)\cos\pi t dt,\ k=1,\ldots.
\end{equation}
Введем в рассмотрение гильбертово пространство $l_2^m$, состоящее из $m$-мерных последовательностей $C=(c_1,\ldots, c_j,\ldots)$, где $c_j=(c_j^1,\ldots,c_j^m)$, для которых определена норма
$$\|C\|=\left(\sum\nolimits_{j=1}^\infty \sum\nolimits_{l=1}^{m}(c_j^l)^2\right)^\frac12.$$

Заметим, что если $C\in l_2^m$, то ряд  $\sum_{j=1}^\infty c_j{\xi}_{1,j+1}(t)$ сходится равномерно на $[0,1]$ и, стало быть, функция $g=g(t)=F(h\sin\pi t,y^0+ h\sum_{j=1}^\infty c_j{\xi}_{1,j+1}(t))\cos\pi t$ непрерывна при $t\in [0,1]$. Отсюда, в свою очередь, следует, что $g\in L^2(0,1)$. Это позволяет рассмотреть в пространстве $l_2^m$  оператор $A$, сопоставляющий точке $C\in l_2^m$ точку $C'\in l_2^m$ по правилу
\begin{equation}\label{5.10}
c_k'=\pi\int\limits_{0}^1 F(h\sin\pi t,y^0+ h\sum_{j=1}^\infty c_j{\xi}_{1,j+1}(t))\cos\pi t\xi_j(t) dt,\ k=1,\ldots.
\end{equation}
Из  \eqref{5.9} следует, что точка $C(q)=(c_1(q),\ldots)$ является неподвижной точкой оператора $A:l_2^m\to l_2^m$. Для того чтобы найти точку $C(q)$ методом простых итераций, достаточно показать, что оператор $A:l_2^m\to l_2^m$ является сжимающим в метрике пространства $l_2^m$. С этой целью рассмотрим две точки $P,Q\in l_2^m$, где $P=(p_1,\ldots)$, $Q=(q_1,\ldots)$, и положим $P'=A(P)$, $Q'=A(Q)$. Имеем
\begin{equation}\label{5.11}
p'_k-q'_k=\pi\int_{0}^1 F_{P,Q}(t)\cos\pi t\xi_k(t)dt,\quad k=1,\ldots
\end{equation}
где
\begin{multline}\label{5.12}
F_{P,Q}(t)=F\left[h\sin\pi t,y^0+ h\sum\nolimits_{j=1}^\infty p_j{\xi}_{1,j+1}(t)\right] \\
  -F\left[h\sin\pi t,y^0+ h\sum\nolimits_{j=1}^\infty p'_j{\xi}_{1,j+1}(t)\right].
\end{multline}
Из \eqref{5.11}, пользуясь неравенством Бесселя, находим
 \begin{equation}\label{5.13}
\sum\nolimits_{k=1}^\infty \sum_{l=1}^m((p^l_k)'-(q^l_k)')^2\le\pi^2\int_{0}^1F_{P,Q}(t)(F_{P,Q}(t))^* \cos^2\pi tdt,
\end{equation}
где $(a_1,\ldots,a_m)^*$ -- вектор-столбец, полученный в результате транспонирования строки $(a_1,\ldots,a_m)$.
Из \eqref{5.2} и \eqref{5.13}  имеем
 \begin{equation}\label{5.14}
F_{P,Q}(t)(F_{P,Q}(t))^*\le (h\lambda_0)^2 \sum\nolimits_{l=1}^m  \left(\sum\nolimits_{j=1}^\infty( p^l_j-q^l_j)\xi_{1,j+1}(t)\right)^2,
\end{equation}
откуда,  воспользовавшись неравенством Коши-Буняковского, выводим
$$
F_{P,Q}(t)(F_{P,Q}(t))^*\le(h\lambda_0)^2  \sum\nolimits_{j=1}^\infty(\xi_{1,j+1}(t))^2 \sum\nolimits_{j=1}^\infty\sum\nolimits_{l=1}^m( p^l_j-q^l_j)^2.
$$
Сопоставляя \eqref{5.14} с \eqref{5.13}, находим
$$
\sum\nolimits_{j=1}^\infty\sum\nolimits_{l=1}^m((p^l_j)'-(q^l_j)')^2\le
$$
\begin{equation}\label{5.15}
(\pi h\lambda_0)^2 \sum\nolimits_{j=1}^\infty\sum\nolimits_{l=1}^m( p^l_j-q^l_j)^2\int_{0}^1 \sum\nolimits_{j=1}^\infty(\xi_{1,j+1}(t))^2\cos^2\pi t dt.
\end{equation}
Отсюда, с учетом \eqref{3.12} находим
\begin{equation}\label{5.16}
(\sum\nolimits_{j=1}^\infty\sum\nolimits_{l=1}^m((p^l_j)'-(q^l_j)')^2)^{1/2}\le h\pi\kappa_\xi\lambda_0(\sum\nolimits_{j=1}^\infty\sum\nolimits_{l=1}^m(p^l_j-q^l_j)^2)^{1/2}
\end{equation}

Неравенство \eqref{5.16} показывает, что если $h\pi\kappa_\varphi\lambda_0<1$, то оператор  $A:l_2^m\to l_2^m$ является сжимающим и, как следствие, итерационный процесс $C^{\nu+1}=A(C^{\nu})$  сходится к точке $C(q)$ при $\nu\to\infty$. Однако с точки зрения приложений важно рассмотреть конечномерный аналог оператора $A$. Обозначим через $\mathbb{R}^N_m$ пространство матриц $C$ размерности $m\times N$, в котором определена норма
$$\|C\|_N^m=\left(\sum\nolimits_{j=1}^{N} \sum\nolimits_{l=1}^{m}(c_j^l)^2\right)^\frac12.$$
 Мы рассмотрим оператор $A_N:\mathbb{R}^N_m\to \mathbb{R}^N_m$, сопоставляющий точке\\
$C_N=(c_1,\ldots,c_{N})\in \mathbb{R}^N_m $ точку  $C'_N=(c_1',\ldots,c_{N}')\in \mathbb{R}^N_m $ по правилу
\begin{equation}\label{5.17}
c_k'=\pi\int_{0}^1 F(h\sin\pi t,y^0+ h\sum\nolimits_{j=1}^N c_j{\xi}_{1,j+1}(t))\xi_j(t) \cos\pi tdt,\ k=1,\ldots,N.
\end{equation}

 Рассмотрим две точки $P_N,Q_N\in \mathbb{R}^N_m$, где $P_N=(p_1,\ldots,p_{N})$,   $Q_N=\linebreak (q_1,\ldots,p_{N})$ и положим $P'_N=A_N(P_N)$, $Q'_N=A_N(Q_N)$. Дословно повторяя рассуждения, которые привели нас к неравенству \eqref{5.16}, мы получим
\begin{equation}\label{5.18}
\left(\sum\nolimits_{j=1}^{N}\sum\nolimits_{l=1}^m((p^l_j)'-(q^l_j)')^2\right)^\frac12\le h\pi\kappa_\xi\lambda_0 \left(\sum\nolimits_{j=1}^{N}\sum\nolimits_{l=1}^m( p^l_j-q^l_j)^2\right)^\frac12.
\end{equation}

Неравенство \eqref{5.18} показывает, что если $h\pi\kappa_\xi\lambda_0<1$, то оператор  $A_N:\mathbb{R}^N_m\to \mathbb{R}^N_m$ является сжимающим и, как следствие, итерационный процесс $C_N^{\nu+1}=A_N(C_N^{\nu})$  при $\nu\to\infty$ сходится к его неподвижной точке, которую мы обозначим через  $\bar C_N(q)=(\bar c_1(q),\ldots,\bar c_{N}(q))$. С другой стороны, рассмотрим точку $C_N(q)=(c_1(q),\ldots,c_{N}(q))$, составленную из искомых коэффициентов Фурье вектор-функции $q$ по системе $\xi$, определяемой равенствами \eqref{2.1}. Нам остается оценить погрешность, проистекающую в результате замены точки $C_N(q)$ точкой $\bar C_N(q)$. Другими словами, требуется оценить величину
$\|C_N(q)-\bar C_N(q)\|_N^m= \left(\sum_{j=1}^{N}\sum\nolimits_{l=1}^m(c_j^l(q)-\bar c_j^l(q))^2\right)^\frac12$. С этой целью рассмотрим точку $C'_N(q)=A_N(C_N(q))=(c_1'(q),\ldots,c_{N}'(q))$ и запишем
\begin{equation}\label{5.19}
\|C_N(q)-\bar C_N(q)\|_N^m\le \|C_N(q)- C_N'(q)\|_N^m+\|C_N'(q)-\bar C_N(q)\|_N^m.
\end{equation}
Далее, пользуясь неравенством \eqref{5.18}, имеем
$$
\|C_N'(q)-\bar C_N(q)\|_N^m=\|A_N(C_N(q))-A_N(\bar C_N(q))\|_N^m\le
$$
\begin{equation}\label{5.20}
h\pi\kappa_\xi\lambda_0\|C_N(q)-\bar C_N(q)\|_N^m.
\end{equation}
Из \eqref{5.19} и \eqref{5.20} выводим
\begin{equation}\label{5.21}
\|C_N(q)-\bar C_N(q)\|_N^m\le \frac1{1-h\pi\kappa_\xi\lambda_0}\|C_N(q)- C_N'(q)\|_N^m.
\end{equation}
Чтобы оценить норму в правой части неравенства \eqref{5.21}, заметим, что в силу неравенства Бесселя
\begin{equation}\label{5.22}
(\|C_N(q)- C_N'(q)\|_N^m)^2\le \pi^2\int_{0}^1 F_{C(q),C_N(q)}(t)( F_{C(q),C_N(q)}(t))^*\cos^2\pi t dt,
\end{equation}
где
\begin{multline}\label{5.23}
  F_{C(q),C_N(q)}(t)= F\left[h\sin\pi t,y^0+ h\sum\nolimits_{j=1}^\infty c_j(q)\xi_{1,j+1}(t)\right] \\
  - F\left[h\sin\pi t,y^0+ h\sum\nolimits_{j=1}^{N}c_j(q)\xi_{1,j+1}(t)\right].
\end{multline}
Из \eqref{5.23} и \eqref{5.2} следует, что
$$
F_{C(q),C_N(q)}(t)(F_{C(q),C_N(q)}(t))^*\le \lambda_0^2 \sum\nolimits_{l=1}^m  \left(\sum\nolimits_{j=N+1}^\infty hc_j^l(q)\xi_{1,j+1}(t)\right)^2,
$$
отсюда с учетом того, что $hc_k(q)=\hat \phi_{1,k+1}$ $(k=1,\ldots)$, имеем
\begin{equation}\label{5.24}
F_{C(q),C_N(q)}(t)(F_{C(q),C_N(q)}(t))^*\le \lambda_0^2   \sum\nolimits_{l=1}^m \left(\sum\nolimits_{j=N+1}^\infty  \widehat {\phi_l}_{1,j+1}\xi_{1,j+1}(t)\right)^2.
\end{equation}
Сопоставляя \eqref{5.24} с \eqref{5.22}, получаем
\begin{equation}\label{5.25}
(\|C_N(q)- C_N'(q)\|_N^m)^2\le (\pi\lambda_0)^2\sum_{l=1}^m\int\limits_{0}^1\left(\sum_{j=N+1}^\infty \widehat {\phi_l}_{1,j+1}\xi_{1,j+1}(t)\right)^2\cos^2\pi t dt,
\end{equation}







Подводя итоги, из \eqref{5.21} и \eqref{5.25}  мы можем вывести следующий результат.
\begin{theorem}\label{th2} Пусть область $\bar G$ такова, что $[0,h]\times\mathbb{R}^m\subset \bar G$, вектор-функ-\linebreak ция $F(x,y)$ непрерывна в области $\bar G$ и удовлетворяет условию Липшица \eqref{5.2}, а $h$ и $\lambda_0$ удовлетворяет неравенству $h\pi\lambda_0\kappa_\xi<1$, где величина $\kappa_\xi$ определена равенством \eqref{3.12}. Далее, пусть $l_2^m$ гильбертово пространство, состоящее из $m$-мерных последовательностей $C=(c_1,\ldots)$, для которых введена норма $\|C\|=\left(\sum\nolimits_{j=1}^{\infty} \sum\nolimits_{l=1}^{m}(c_j^l)^2\right)^\frac12$,   оператор $A: l_2^m\to l_2^m$ сопоставляет точке $C\in l_2^m$ точку $C'\in l_2^m$ по правилу \eqref{5.10}. Кроме того, пусть $A_N:\mathbb{R}^N_m\to \mathbb{R}^N_m$ -- конечномерный аналог оператора $A$, сопоставляющий точке $C_N=(c_1,\ldots,c_{N})\in \mathbb{R}^N_m $ точку  $C'_N=(c_1',\ldots,c_{N}')\in \mathbb{R}^N_m $ по правилу \eqref{5.17}.
Тогда операторы $A: l_2^m\to l_2^m$ и $A_N:\mathbb{R}^N_m\to \mathbb{R}^N_m$ являются сжимающими и, следовательно, существуют  их неподвижные точки $C(q)=(c_1(q),\ldots)=A(C(q))\in l_2^m$ и $\bar C_N(q)=(\bar c_1(q),\ldots,\bar c_{N}(q))=A_N(\bar C_N(q))\in \mathbb{R}^N_m$, для которых имеет место неравенство
\begin{equation}\label{5.26}
\|C_N(q)-\bar C_N(q)\|_N^m\le \frac{\pi\lambda_0 \sigma_N^\xi(\phi)}{1-h\pi\kappa_\xi\lambda_0},
\end{equation}
где $\phi=\phi(t)$ --- решение задачи \eqref{5.3},
\begin{equation}\label{5.27}
\sigma_N^\xi(\phi)=\left(\sum\nolimits_{\nu=1}^m\int_{0}^1\left(\sum\nolimits_{j=N+2}^\infty \widehat {\phi_\nu}_{1,j}\xi_{1,j}(t)\right)^2\cos^2\pi t dt\right)^\frac12,
\end{equation}
 a $C_N(q)=(c_1(q),\ldots,c_{N}(q))$ -- конечная последовательность, составленная из первых $N$ компонент точки  $C(q)$.
\end{theorem}

\section{Аппроксимативные свойства частичных сумм ряда  Фурье по системе
$\{\xi_{1,n}(x)\}$}\label{s6}

Неравенство \eqref{5.26} непосредственно приводит к  задаче об исследовании поведения величины $\sigma_N^\xi(\phi)$, определяемой равенством \eqref{5.27}, которая, в свою очередь, приводит к вопросу об исследовании поведения при $N\to\infty$ остаточного члена $V_{N+1}(t)=V_{N+1}(\phi,t)=\phi(t)-Y_{1,N+1}(\phi,t)=\sum\nolimits_{j=N+2}^\infty \hat \phi_{1,j}\xi_{1,j}(t)$  ряда Фурье \eqref{5.4} по функциям $\xi_{1,j}(t)$ $(k=0,1,\ldots)$. Мы рассмотрим эту задачу
для одной (произвольной) компоненты $\phi_l(t)$ вектор-функции $\phi(t)=(\phi_1(t),\ldots,\phi_m(t))$. Как было показано выше, если $\phi_l(t)$ является решением уравнения \eqref{4.3}, то  ряд Фурье \eqref{3.8} (одномерный вариант ряда  \eqref{5.4}) допускает его преобразование к виду \eqref{4.4} и, соответственно, имеем
\begin{equation}\label{6.1}
Y_{1,N+1}(\phi_l,t)=\sum\nolimits_{k=1}^{N} b_k\sin(\pi kt),
\end{equation}
\begin{equation}\label{6.2}
V_{N+1}(\phi_l,t)=\phi_l(t)-Y_{1,N+1}(\phi_l,t)= \sum\nolimits_{k=N}^\infty b_k\sin(\pi kt),
\end{equation}
\begin{equation*}
b_k==\int_{-1}^1 \phi_l(\tau)\sin(\pi k\tau)d\tau.
\end{equation*}

Отсюда мы замечаем, что задача об исследовании поведения величины \linebreak $V_{N+1}(\phi_l,t)$ сводится к изучению остаточного члена тригонометрического ряда Фурье нечетной непрерывно дифференцируемой 2-периодической функции $\phi(t)$.
Обозначим через $E_n(\phi_l)_2$ наилучшее приближение функции $\phi_l$ в $L^2(-1,1)$ тригонометрическими полиномами вида $T_N(x)=\sum_{k=1}^{N}\beta_k\sin(\pi kx)$. Поскольку $Y_{1,N+1}(\phi_l,x)$ является полиномом наилучшего приближения к функции $\phi_l$ в $L^2(-1,1)$, то из \eqref{5.27} и  \eqref{6.2}, с учетом того, что функция $\phi_l$ нечетная, имеем
\begin{equation}\label{6.3}
\|V_{N+1}(\phi_l)\|_{L^2(0,1)}=\frac12E_{N}(\phi_l)_2.
\end{equation}
Вернемся к вектор-функции $\phi=(\phi_1,\ldots,\phi_m)$ и положим
\begin{equation}\label{6.4}
\|V_{N+1}(\phi)\|_{L^2(0,1)}=\left(\sum\nolimits_{l=1}^m\int_{0}^1\left(\sum\nolimits_{j=N+2}^\infty \widehat {\phi_l}_{1,j}\xi_{1,j}(t)\right)^2 dt\right)^\frac12,
\end{equation}
\begin{equation}\label{6.5}
    E_{N}(\phi)_2=\left(\sum\nolimits_{l=1}^mE^2_{N}(\phi_l)_2\right)^{1/2}.
\end{equation}
Из \eqref{6.3} -- \eqref{6.5} имеем
\begin{equation}\label{6.6}
\|V_{N+1}(\phi)\|_{L^2(0,1)}=\frac12E_{N}(\phi)_2.
\end{equation}
С другой стороны, из \eqref{5.27} и \eqref{6.4} следует, что
\begin{equation}\label{6.7}
\sigma_N^\xi(\phi)=\left(\sum_{\nu=1}^m\int_{0}^1\left(\sum_{j=N+2}^\infty \widehat {\phi_\nu}_{1,j}\xi_{1,j}(t)\right)^2\cos^2\pi t dt\right)^\frac12\le \|V_{N+1}(\phi)\|_{L^2(0,1)}
\end{equation}



Из теоремы \ref{th2}, леммы \ref{lem3.2} и равенства \eqref{6.6} и неравенства \eqref{6.7}  непосредственно вытекает
\begin{corollary}\label{cor1}
  При соблюдении условий теоремы 1 имеет место неравенство
  \begin{equation}\label{6.8}
\|C_N(q)-\bar C_N(q)\|_N\le \frac{\pi\lambda_0 E_{N}(\phi)_2} {2-(1/6-1/\pi^2)h\pi\lambda_0}.
\end{equation}
\end{corollary}

 Теорема \ref{th2}, в которой получена оценка скорости сходимости частичных сумм  ряда Фурье по функциям  $\xi_{1,k}(t)$ к функции $\phi$ в метрике пространства $L^2(0,1)$, дает, как это показано в следствие \ref{cor1}, один из возможных подходов к оценке погрешности, которая проистекает в результате замены искомой матрицы $C_N(q)$, составленной из неизвестных коэффициентов  разложения решения задачи Коши \eqref{5.3}, приближенной матрицей $\bar C_N(q)$, являющейся неподвижной точкой конечномерного оператора $A_N$, сконструированного по правилу \eqref{5.17}. Но, после того, как  матрица   $C_N(q)$ будет  найдена с требуемой точностью, немедленно возникнет вопрос о том, какова максимальная по $t\in[0,1]$ погрешность замены точного решения  задачи Коши \eqref{5.3} теми или иными линейными средними частичных сумм его разложения по функциям $\xi_{1,k}(x)$, которые могут быть сконструированы посредством найденных коэффициентов, составляющих матрицу $C_N(q)$.
  Как показывают равенства \eqref{6.1} и \eqref{6.2}, эта задача сводится к  глубоко исследованной  задаче о равномерном приближении дифференцируемых периодических функций тригонометрическими суммами Фурье и  их линейными средними и нам нет особой необходимости ее рассматривать.

\end{fulltext}
%\newpage
\begin{thebibliography}{20}

\bibitem{SharDagElec7}
Шарапудинов И.И. О приближении решения задачи Коши для нелинейных систем ОДУ посредством рядов Фурье по функциям, ортогональным по Соболеву // Дагестанские электронные математические известия. 2017. Вып. 7. С. 66--76.

%\RBibitem{SharDagElec7}
%\by И.\,И. Шарапудинов
%\paper О приближении решения задачи Коши для нелинейных систем ОДУ посредством рядов Фурье по функциям, ортогональным по Соболеву
%\inbook Дагестанские электронные математические известия
%\vol
%\issue 7
%\yr 2017
%\pages 66–76

\bibitem{SharMagDagElec8}
Шарапудинов И.И., Магомед-Касумов М.Г. Численный метод решения задачи Коши для систем обыкновенных дифференциальных уравнений с помощью ортогональной в смысле Соболева системы, порожденной системой косинусов // Дагестанские электронные математические известия. 2017. Вып. 8. С. 53--60.

%\RBibitem{SharMagDagElec8}
%\by И.\,И. Шарапудинов, М.\,Г. Магомед-Касумов
%\paper Численный метод решения задачи Коши для систем обыкновенных дифференциальных уравнений с помощью ортогональной в смысле Соболева системы, порожденной системой косинусов
%\inbook Дагестанские электронные математические известия
%\vol
%\issue 8
%\yr 2017
%\pages 53–60

\bibitem{Shar2016}
Шарапудинов И.И. Системы функций, ортогональные по Соболеву, порожденные ортогональными функциями // Материалы 18-й международной Саратовской зимней школы «Современные проблемы теории функций и их приложения». Саратов. ООО «Издательство «Научная книга». 2016. С. 329--332.

%\RBibitem{Shar2016}
%\by И.И. Шарапудинов
%\paper Системы функций, ортогональные по Соболеву, порожденные ортогональными функциями
%\inbook Материалы 18-й международной Саратовской зимней школы «Современные проблемы теории функций и их приложения»
%\publ ООО «Издательство «Научная книга»
%\yr 2016
%\pages 329-332
%\publaddr Саратов

\bibitem{MMG2016}
Магомед-Касумов М.Г. Приближенное решение обыкновенных дифференциальных уравнений с использованием смешанных рядов по системе Хаара // Материалы 18-й международной Саратовской зимней школы «Современные проблемы теории функций и их приложения». Саратов. ООО «Издательство «Научная книга». 2016. С. 176--178.

%\RBibitem{MMG2016}
%\by М.Г. Магомед-Касумов
%\paper Приближенное решение обыкновенных дифференциальных уравнений с использованием смешанных рядов по системе Хаара
%\inbook Материалы 18-й международной Саратовской зимней школы «Современные проблемы теории функций и их приложения»
%\publ ООО «Издательство «Научная книга»
%\yr 2016
%\pages 176-178
%\publaddr Саратов

\bibitem{SharIzv2018}
Шарапудинов И.И. Системы функций, ортогональные по Соболеву, ассоциированные с ортогональной системой // Изв. РАН. Сер. матем. 2018. Т. 82, вып. 1. С. 225--258.

%\RBibitem{SharIzv2018}
%\by И.И.Шарапудинов
%\paper Системы функций, ортогональные по Соболеву, ассоциированные с ортогональной системой
%\inbook Изв. РАН. Сер. матем.
%\vol 82
%\issue 1
%\yr 2018
%\pages 225 –- 258

\bibitem{MarcelXu}
Marcellan F. and Yuan Xu On Sobolev orthogonal polynomials // Expositiones Mathematicae. 2015. Vol. 33. Issue 3. Pp. 308--352.

%\RBibitem{MarcelXu}
%\by F. Marcellan and Yuan Xu
%\paper On Sobolev orthogonal polynomials
%\inbook  Expositiones Mathematicae
%\vol 33
%\issue 3
%\yr 2015
%\pages 308--352

\bibitem{Shar17}
Шарапудинов И.И. Смешанные ряды по ультрасферическим полиномам и их аппроксимативные свойства // Математический сборник. 2003. Т. 194, вып. 3. С. 115--148.

%\RBibitem{Shar17}
%\by И.\,И. Шарапудинов
%\paper Смешанные ряды по ультрасферическим полиномам и их аппроксимативные свойства
%\inbook Математический сборник
%\vol 194
%\issue 3
%\yr 2003
%\pages 115--148

\bibitem{Shar13}
Шарапудинов И.И. Смешанные ряды по ортогональным полиномам. Махачкала. Издательство Дагестанского научного центра. 2004. 176 стр.

%\RBibitem{Shar13}
%\by И.\,И. Шарапудинов
%\paper Смешанные ряды по ортогональным полиномам
%\inbook
%\publ Издательство Дагестанского научного центра
%\yr 2004
%\pages 1 --176
%\publaddr Махачкала

\end{thebibliography}

\end{document}

89896587588
