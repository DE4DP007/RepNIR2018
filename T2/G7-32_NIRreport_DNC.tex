% This package designed and commented in russian koi8-r encoding.
%
% Класс документов по ГОСТ 7.32-2001 "Отчёт о научно-исследовательской работе"
% на основе ГОСТ 2.105-95
% Автор - Алексей Томин, с помощью списка рассылки latex-gost-request@ice.ru,
%  "extreport.cls", "lastpage.sty" и конференции RU.TEX
% Лицензия GPL
% Все вопросы, замечания и пожелания сюда: mailto:alxt@yandex.ru
% Дальнейшая разработка и поддержка - Михаил Конник,
% связаться можно по адресу mydebianblog@gmail.com


\documentclass[utf8,usehyperref,12pt]{G7-32}
\usepackage{amsthm,amsfonts,amsmath,amssymb,amscd} % Математические дополнения от AMS
\usepackage[T2A]{fontenc}
\usepackage[utf8]{inputenc} %% ваша любимая кодировка здесь
\usepackage[english,russian]{babel} %% это необходимо для включения переносов
\usepackage{float}
\usepackage{graphicx}
\usepackage{cmap}
\usepackage{color}
\usepackage[all,cmtip]{xy}
\graphicspath{{pictures/}}


\usepackage{cite} % Красивые ссылки на литературу
%\usepackage[plainpages=false,pdfpagelabels=false]{hyperref}
%%% Списки %%%

\usepackage{dsfont}
\usepackage{mathrsfs}



\TableInChaper % таблицы будут нумероваться в пределах раздела
\PicInChaper   % рисунки будут нумероваться в пределах раздела
\setlength\GostItemGap{2mm}% для красоты можно менять от 0мм

% Определяем заголовки для титульной страницы
\NirOrgLongName{\textsc{ФЕДЕРАЛЬНО АГЕНТСТВО НАУЧНЫХ ОРГАНИЗАЦИЙ\\
ФЕДЕРАЛЬНОЕ ГОСУДАРСТВЕННОЕ БЮДЖЕТНОЕ УЧРЕЖДЕНИЕ НАУКИ \\ ДАГЕСТАНСКИЙ НАУЧНЫЙ ЦЕНТР РОССИЙСКОЙ АКАДЕМИИ НАУК}} %% Полное название организации

\NirBoss{Врио председателя ДНЦ РАН}{Гаджиев М.С.} %% Заказчик, утверждающий НИР
\NirManager{Зав. Отделом математики и информатики ДНЦ РАН, доктор физ.-мат. наук}{Шарапудинов И.И.} %% Название организации

\NirYear{2015}%% если нужно поменять год отчёта; если закомментировано, ставится текущий год
\NirTown{г. Махачкала,} %% город, в котором написан отчёт
% по проекту \No8550:

% \NirIsAnnotacion{АННОТАЦИОННЫЙ } %% Раскомментируйте, если это аннотационный отчёт

\NirUdk{УДК \No }
\NirGosNo{Регистрационный \No }

\NirStage{%Этап \No 1.1
}{итоговый отчет за 2015 г.}{%<<Обзор современного состояния торсионных наногенераторов>>
} %%% Этап НИР: {номер этапа}{вид отчёта - промежуточный или заключительный}{название этапа}

\bibliographystyle{unsrt} %Стиль библиографических ссылок БибТеХа

\input commands.tex
\usepackage[shortlabels]{enumitem}
%%%%%%%<------------- НАЧАЛО ДОКУМЕНТА
\begin{document}
\usefont{T2A}{ftm}{m}{} %%% Использование шрифтов Т2 для возможности скопировать текст из PDF-файлов.

\frontmatter %%% <-- это выключает нумерацию ВСЕГО; здесь начинаются ненумерованные главы типа Исполнители, Обозначения и прочее

\NirTitle{%\textbf{<<Торсионные наногенераторы плазменных стволовых клеток с протонной накачкой>>}

%\begin{figure}[h]
%\center{\includegraphics[natwidth=2cm, natheight=6cm, width=2cm, height=6cm]{black.bmp}}
%\caption{Зависимость сигнала от шума для данных.}
%\label{ris:image}
%\end{figure}

\input annots/annot.tex


} %%% Название НИР и генерация титульного листа


\Executors %% Список исполнителей здесь
%% это рисует линию размера 3мм и толщиной 0.1 пункт
\begin{longtable}{p{0.35\linewidth}p{0.2\linewidth}p{0.35\linewidth}}

\input executors/executors.tex


Нормоконтроллер, н.с. ОМИ,  &		&	\\
Султанахмедов М.С. & \rule{1\linewidth}{0.1pt}& \\
\end{longtable}

\input chapters/referat.tex

\setcounter{tocdepth}{2} %hide subsections

\tableofcontents

%\NormRefs % Нормативные ссылки
%\Defines % Необходимые определения


\Abbreviations %% Список обозначений и сокращений в тексте
\begin{abbreviation}
\item[ДНЦ] Дагестанский научный центр
\item[ОМИ] Отдел математики и информатики
\item[РАН] Российская академия наук
\end{abbreviation}


\input chapters/introduction.tex

\mainmatter %% это включает нумерацию глав и секций в документе ниже


%\input chapters/chapter1.tex

\input chapters/chapter2.tex
%
%\input chapters/chapter3.tex
%
%\input chapters/chapter4.tex
%
%\input chapters/chapter5.tex
%
%\input chapters/chapter6.tex



\backmatter %% Здесь заканчивается нумерованная часть документа и начинаются заключение и ссылки

\input chapters/conclusion.tex% заключение к отчёту

\input chapters/biblio.tex

% \bibliography{biblio/filosofy} %% вместо вставки библиографии можно использовать базы BiBTeX - просто раскомментируйте эту строку.
\end{document}
