\Conclusion

В 2018 году в Отделе математики и информатики Дагестанского научного центра РАН проведены научно-исследовательские работы по теме
<<Теория полиномов, ортогональных по Соболеву. аппроксимативные свойства рядов Фурье по полиномам, ортогональным по Соболеву. приложения полиномов, ортогональных по Соболеву>>.

Для  произвольного натурального $r$ рассмотрены полиномы $p^{\alpha,\beta}_{r,k}(x)$ $(k=0,1,\ldots)$, ортонормированные относительно скалярного произведения типа Соболева следующего вида
$$
<f,g>=\sum_{\nu=0}^{r-1}f^{(\nu)}(-1)g^{(\nu)}(-1)+
\int_{-1}^{1}f^{(r)}(t)g^{(r)}(t)(1-t)^\alpha(1+t)^\beta dt
$$
и изучены  их свойства. Введены в рассмотрение ряды Фурье по полиномам $p_{r,k}(x)=p^{0,0}_{r,k}(x)$ и некоторые их обобщения, частичные суммы которых  сохраняют некоторые важные  свойства частичных сумм ряда Фурье по полиномам $p_{r,k}(x)$, в том числе и свойство $r$-кратного совпадения (<<прилипания>>) частичных сумм ряда Фурье по полиномам $p_{r,k}(x)$  в  точках $-1$ и $1$ между собой и с исходной функцией $f(x)$.  Основное внимание уделено  исследованию вопросов приближения гладких и аналитических функций  частичными суммами упомянутых обобщений, представляющих собой   специальные ряды  по ультрасферическим полиномам Якоби со свойством <<прилипания>> их частичных сумм  точках $-1$ и $1$ .

Рассмотрены системы функций $\mathcal{ \varphi}_{r,n}(x)$ $(r=1,2,\ldots, n=0,1,\ldots)$,
ортонормированные по Соболеву относительно скалярного произведения  вида  
\begin{equation*}
\langle f,g\rangle=\sum_{\nu=0}^{r-1}f^{(\nu)}(a)g^{(\nu)}(a)+\int_{a}^{b}f^{(r)}(t)g^{(r)}(x)\rho(x)dx,
\end{equation*}
порожденные заданной ортонормированной системой функций $\mathcal{ \varphi}_{n}(x)$ $( n=0,1,\ldots)$.  Показано, что ряды и суммы Фурье по системе $\mathcal{ \varphi}_{r,n}(x)$ $(r=1,2,\ldots, n=0,1,\ldots)$ являются удобным и весьма эффективным инструментом приближенного решения задачи Коши для обыкновенных дифференциальных уравнений (ОДУ).

Рассмотрены полиномы $T_{r,n}(x)$ $(n=0,1,\ldots)$, порожденные многочленами Чебышева $T_{n}(x)=\cos( n\arccos x)$, образующие ортонормированную систему по Соболеву относительно скалярного произведения
следующего вида 
\begin{equation*}
<f,g>=\sum_{\nu=0}^{r-1}f^{(\nu)}(-1)g^{(\nu)}(-1)+\int_{-1}^{1}f^{(r)}(t)g^{(r)}(x)\mu(x)dx,
\end{equation*}
где $\mu(x)=\frac2\pi(1-x^2)^{-\frac12}$. Для $T_{r,n}(x)$ $(n=0,1,\ldots)$  установлена связь с многочленами Чебышева $T_{n}(x)$ и получены явные представления, которые могут быть использованы при вычислении значений полиномов $T_{r,n}(x)$ и исследовании их асимптотических свойств. 

Рассмотрена задача об обращении преобразования Лапласа
 посредством специального ряда по полиномам Лагерра, который в  частном случае совпадает с рядом Фурье по   полиномам $l_{r,k}^{\gamma}(x)$ $(r\in \mathbb{N}, k=0,1,\ldots)$, ортогональным относительно скалярного произведения типа Соболева следующего вида
\begin{equation*}
<f,g>=\sum\nolimits_{\nu=0}^{r-1}f^{(\nu)}(0)g^{(\nu)}(0)+\int_0^\infty f^{(r)}(t)g^{(r)}(t)t^\gamma e^{-t}dt, \gamma>-1.
\end{equation*}
  Даны оценки приближения функций частичными суммами специального ряда по полиномам Лагерра.

Получены рекуррентные соотношения для системы полиномов $l_{r,n}^{\alpha}(x)$ ($r$-натуральное число, $n=0, 1, \ldots$), ортонормированной относительно скалярного произведения типа Соболева $\langle f,g\rangle=\sum_{\nu=0}^{r-1}f^{(\nu)}(0)g^{(\nu)}(0)+\int_{0}^{\infty} f^{(r)}(x)g^{(r)}(x)\rho(x) dx$ и порожденной классическими ортонормированными полиномами Лагерра.

Рассмотрены системы дискретных функций $\mathcal{\psi}_{r,n}(x)$ $(r=1,2,\ldots, n=0,1,\ldots)$, ортонормированные по Соболеву относительно скалярного произведения  вида
\begin{equation*}
\langle f,g\rangle=\sum_{k=0}^{r-1}\Delta^kf(0)\Delta^kg(0)+\sum_{j=0}^\infty\Delta^rf(j)\Delta^rg(j)\rho(j),
\end{equation*}
порожденные заданной ортонормированной системой функций $\mathcal{\psi}_{n}(x)$ $( n=0,1,\ldots)$.  Показано, что ряды и суммы Фурье по системе $\mathcal{\psi}_{r,n}(x)$ $(r=1,2,\ldots, n=0,1,\ldots)$ является удобным и весьма эффективным инструментом приближенного решения задачи Коши для  разностных уравнений.




