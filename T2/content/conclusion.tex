\Conclusion

В ходе выполнения НИР получены следующие основные результаты.

1. Рассмотрена задача о представлении решения задачи Коши для системы  обыкновенных  дифференциальных  уравнений (вообще говоря нелинейных) в виде  ряда  Фурье по полиномам $T_{r,k}(x)$ $(k=0,1,\ldots)$, ортонормированным по Соболеву относительно скалярного произведения
$<f,g>=\sum_{\nu=0}^{r-1}f^{(\nu)}(-1)g^{(\nu)}(-1)+\int_{-1}^1 f^{(r)}(t)g^{(r)}(t)\rho(t)dt$, где $\rho(x)=\frac2\pi(1-x^2)^{-\frac12}$, порожденным полиномами Чебышева $T_n(x)=\cos(n\arccos x)$ посредством равенств $T_{r,k}(x) =\frac{(x+1)^k}{k!}\, ( k=0,1,\ldots, r-1)$,\, $T_{r,r}(x) =\frac{(x+1)^r}{\sqrt{2}r!}$,\, $\quad T_{r,r+n}(x) =\frac{1}{(r-1)!}\int_{-1}^x(x-t)^{r-1}T_{n}dt\,\, ( n=1,\ldots)$.  В бесконечномерном гильбертовом  пространстве $l_2^m$ $m$-мерных последовательностей  $C=(c_0,c_1,\ldots)$, для которых определена норма
$\|C\|=\left(\sum\nolimits_{j=0}^\infty \sum\nolimits_{l=1}^{m}(c_j^l)^2\right)^\frac12$, сконструирован сжимающий нелинейный оператор  $A: l_2^m\to l_2^m$,  неподвижная точка $\hat C=(\hat c_0,\hat c_1,\ldots)$ которого совпадает с последовательностью  искомых неизвестных коэффициентов разложения решения рассматриваемой задачи Коши в ряд Фурье по системе $T_{1,k}(x)$ $(k=0,1,\ldots)$. Сконструирован также соотвествующий конечномерный аналог $A_N:\mathbb{R}^N_m\to \mathbb{R}^N_m$ оператора $A$, который действует в кечномерном пространстве  $\mathbb{R}^N_m$ матриц $C$ размерности $m\times N$, в котором определена норма
$\|C\|_N^m=\left(\sum\nolimits_{j=0}^{N-1} \sum\nolimits_{l=1}^{m}(c_j^l)^2\right)^\frac12$. Неподвижная точка $\bar C=(\bar c_0,\bar c_1,\ldots, \bar c_{N-1})$ оператора $A_N$ представляет собой оценку (приближенное значение) искомой точки $\hat C_N=(\hat c_0,\hat c_1,\ldots, \hat c_{N-1})$. Установлена оценка погрешности $\|\hat C_N-\bar C_N\|_N^m$. Рассмотрены задачи о приближении дифференцируемых и аналитических функций суммами Фурье по полиномам $T_{1,k}(x)$ $(k=0,1,\ldots)$, ортогональным относительно указанного скалярного произведения.


2. Рассмотрена система функций  $\xi_0(x)=1,\, \{\xi_n(x)=\sqrt{2}\cos(\pi nx)\}_{n=1}^\infty$ и порожденная ею система
$$
\xi_{1,0}(x)=1,\, \xi_{1,1}(x)=x,\, \xi_{1,n+1}(x)=\int_0^x \xi_{n}(t)dt=\frac{\sqrt{2}}{\pi n}\sin(\pi nx),\, n=1,2,\ldots,
$$
которая является ортонормированной по Соболеву относительно скалярного произведения  вида $<f,g>=f'(0)g'(0)+\int_{0}^{1}f'(t)g'(t)dt$. Показано, что ряды и суммы Фурье по системе $\{\xi_{1,n}(x)\}_{n=0}^\infty$  является удобным и весьма эффективным инструментом приближенного решения задачи Коши для систем нелинейных обыкновенных дифференциальных уравнений (ОДУ).

3. Рассмотрена система функций $\lambda_{r,n}^{\alpha}(x)$ ($r\in\mathbb{N}$, $n=0, 1, \ldots$), ортонормированная при $\alpha>-1$ относительно скалярного произведения типа Соболева следующего вида $\langle f,g\rangle=\sum_{\nu=0}^{r-1}f^{(\nu)}(0)g^{(\nu)}(0)+\int_{0}^{\infty} f^{(r)}(x)g^{(r)}(x) dx$ и порождённая ортонормированными функциями Лагерра.
Показано, что ряд Фурье по системе $\{\lambda_{r,n}^{\alpha}(x)\}_{k=0}^\infty$ при $\alpha\geq0$ сходится равномерно относительно $x\in[0, A]$, $A\geq0,$ к функции $f\in W^r_{L^p}$ для $\frac{4}{3}<p<4$.
Для системы функций $\lambda_{r,n}^{\alpha}(x)$ получены рекуррентные соотношения.
Кроме того, исследованы асимптотические свойства функций $\lambda_{1,n}^0(x)$ при $0\leq x\leq\omega$, где $\omega$--некоторое фиксированное положительное число.

4. Изучены свойства функций из ортогональной по Соболеву системы $\mathfrak{W}_r$, порожденной системой Уолша. 
В частности, получены рекуррентные соотношения для функций из $\mathfrak{W}_1$.
Доказана равномерная сходимость рядов Фурье по системе $\mathfrak{W}_r$ к функциям $f$ из пространств Соболева $W^r_{L^p}$, $p \ge 1$,  $r=1,2,\ldots$.

5. Введено понятие решения задачи Коши для системы ОДУ вида
$ y'(x)=f(x,y),\quad y(0)=y_0, \quad 0\le x\le 1$, в которой правая часть  $f=(f_1,\ldots,f_m)$ необязательно непрерывна в области своего определения $G\subset\mathbb{R}^{m+1}$. Рассмотрены вопросы существования и единственности решения задачи Коши. Для того, чтобы определить понятие  решения задачи Коши, введен класс  $AC^m[0,1]$, состоящий из всех абсолютно непрерывных вектор-функций $y=y(x)=(y_1(x),\ldots,y_m(x))$, заданных на $[0,1]$.
Вектор-функция $y\in AC^m[0,1]$ называется решением задачи Коши, если выполнено начальное условие $y(0)=y_0$ и имеет место равенство $y'(x)=f(x,y(x))$ для почти всех $x\in[0,1]$.
При изучении вопросов, связанных с существованием и единственностью задачи Коши в смысле приведенного определения, ключевую роль сыграли системы функций, ортонормированные по Соболеву и порожденные заданной системой  $\{\varphi_k(x)\}_{k=0}^\infty$, ортонормированной в весовом пространстве Лебега $L_\rho^2(0,1)$  с весом $\rho=\rho(x)$. 