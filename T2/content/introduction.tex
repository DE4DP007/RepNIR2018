\Introduction
В работе Фабера \cite{Faber} впервые была построена система функций, которые с точностью до нормирующих множителей совпадают с функциями
\begin{equation}\label{1.1}
\phi_{0}(x)=1, \quad \phi_{k}(x)=\int_{0}^{x}\chi_k(t)dt, \quad k=1,2,\ldots,
\end{equation}
где  $\chi=\{\chi_k\}_{k=1}^\infty$ --   классическая система Хаара.
В \cite{Faber} было показано (см. также \cite{Shauder}), что  система $\phi=\{\phi_k(x)\}_{k=0}^\infty$  представляет собой базис в пространстве $C[0,1]$, состоящем из непрерывных на $[0,1]$ функций $f$, для которых определена норма $\|f\|=\max_{x\in[0,1]}|f(x)|$. С другой стороны, из определения \eqref{1.1} следует, что система $\phi$, как это показано в \cite{SharIzv2018},  является ортонормированным базисом в  пространстве Соболева $W^1_{L^2(0,1)}$, состоящем из абсолютно непрерывных на $[0,1]$ функций $f$, таких что $f'\in L^2(0,1)$, в котором введено скалярное произведение
\begin{equation}\label{1.2}
\langle f,g\rangle=f(0)g(0)+\int_{0}^1f'(x)g'(x)dx.
\end{equation}
Это свойство системы Фабера --- Шаудера $\phi$ до недавнего времени \cite{SharIzv2018} оставалось неотмеченным и невостребованным. В работе \cite{SharIzv2018} было показано (см. также \cite{SharDiffur2018}), что системы функций, ортогональные относительно скалярных произведений типа Соболева, частным случаем которых и является \eqref{1.2}, тесно связаны с задачей Коши для линейных и нелинейных систем ОДУ, а в работе  \cite{MMG2016} на основе этой идеи был разработан конкретный алгоритм для численно-аналитического решения задачи Коши для  ОДУ путем представления ее решения в виде ряда Фурье по системе Фабера --- Шаудера \eqref{1.1} как по ортонормированной по Соболеву системе в $W^1_{L^2(0,1)}$ относительно скалярного произведения \eqref{1.2}. Такой ряд Фурье для функции $f\in W^1_{L^2(0,1)}$, как это показано в \cite{SharIzv2018}, имеет вид
\begin{equation}\label{1.3}
f(x)=f(0)+\sum\nolimits_{k=1}^\infty \hat f_k \phi_{k}(x),
\end{equation}
где
\begin{equation}\label{1.4}
\hat f_k=\int_{0}^1f'(t)\chi_{k}(t)dt, \quad k=1,2,\ldots
\end{equation}
-- коэффициенты Фурье-Хаара функции $f'$. Отличительная особенность смешанного ряда \eqref{1.3} от обычных рядов Фурье по классическим ортонормированным системам, включая и систему Хаара, заключается в том, что любая его частичная сумма вида
\begin{equation}\label{1.5}
Y_N(x)=f(0)+\sum_{k=1}^N \hat f_k \phi_{k}(x)
\end{equation}
в начальной точке $x=0$ совпадает с исходной функцией $f$, т. е. $Y_N(0)=f(0)$. Именно это свойство делает системы функций, ортогональных по Соболеву относительно скалярных произведений типа \eqref{1.2}, удобным инструментом  численно-аналитического решения задачи Коши для ОДУ.

Однако мы не можем утверждать, что теория систем функций, ортогональных по Соболеву, берет свое начало с указанных работ \cite{Faber} и \cite{Shauder}, так как скалярные произведения, содержащие производные и соответствующие ортогональные системы появились   \cite{Althammer} позже -- в начале 60-ых -- в связи с задачами теории приближений алгебраическими полиномами. Интенсивное развитие теории полиномов, ортогональных по Соболеву, началось   \cite{IserKoch} -- \cite{KwonLittl2} в  90-ые. Одно из направлений зародившейся новой теории ассоциируется со следующим скалярным произведением соболевского типа
\begin{equation}\label{1.6}
\langle f,g\rangle=\sum_{\nu=0}^r\int_{\mathbb{R}}f^{(\nu)}(x)g^{(\nu)}(x)d\rho_\nu(x),
\end{equation}
в котором $\rho_\nu$ ( $\nu=0,\ldots,r$) --  борелевские меры, заданные на $\mathbb{R}$. Если конечная или бесконечная  последовательность полиномов $p_k=p_k(x)$, $0\le k< N\le\infty$, удовлетворяет условию
\begin{equation}\label{1.7}
\langle p_k,p_n\rangle=\delta_{k,n}=\begin{cases}0,&\text{если $k\ne n$, }\\1,&\text{если $k= n$ }, \end{cases}
\end{equation}
то говорят, что система  $\{p_k(x)\}_{k=0}^{N-1}$ ортонормирована по Соболеву относительно скалярного произведения \eqref{1.6}. В настоящее время теория полиномов, ортогональных по Соболеву, интенсивно развивается. Достаточно отметить, что число работ, опубликованных  на эту тему, уже превышает нескольких сотен. По поводу более подробных сведений, касающихся развития общей теории полиномов, ортогональных относительно скалярного произведения \eqref{1.6}, мы отсылаем к обзору \cite{MarcelXu}. При этом следует отметить, что, по-видимому, еще рано говорить о завершенной теории полиномов, ортогональных относительно общего скалярного произведения \eqref{1.6}. В настоящее время мы имеем дело с большим количеством разрозненных результатов, касающихся различного рода рекуррентных соотношений, дифференциальных или разностных уравнений, которым удовлетворяют те или иные системы полиномов, ортогональных относительно скалярного произведения \eqref{1.6}. Характер исследований и глубина достигнутых к настоящему времени результатов этой теории существенно зависят от свойств борелевских мер $\rho_k$, $k=0,\ldots,r$, фигурирующих в скалярном произведении \eqref{1.6}. Существенно используя идеи и методы, разработанные А.А. Гончаром \cite{Gonchar1975}, в работе \cite{Lopez1995} авторы исследовали свойства сравнительной асимптотики полиномов, ортогональных по Соболеву относительно скалярного произведения вида
$$
\langle h,g\rangle=\sum\nolimits_{j=1}^m\sum\nolimits_{i=0}^{N_j}M_{j,i}h^{(i)}(c_j)g^{(i)}(c_j)+\int_{\mathbb{R}}hgd\mu.
$$

К настоящему времени наиболее продвинутые результаты, касающиеся как  свойств систем  функций, ортогональных по Соболеву, так и их приложений,  получены в том случае, когда \eqref{1.6} допускает представление
\begin{equation}\label{1.8}
\langle f,g\rangle=\sum\nolimits_{k=0}^{r-1}f^{(k)}(a)g^{(k)}(a)+\int_{a}^{b}f^{(r)}(x)g^{(r)}(x)\rho(x)dx,
\end{equation}
которое является обобщением скалярного произведения \eqref{1.2}. В настоящей научно-исследовательской работе мы акцентировали внимание на скалярном произведении \eqref{1.8} и свойствах систем полиномов и функций, ортогональных относительно этого скалярного произведения. Найдены глубинные связи таких систем с соответствующими классическими ортогональными системами, такими как системы полиномов Якоби, Лагерра, Уолша и др. Рассмотрены системы функций, ортогональных по Соболеву относительно скалярного произведения \eqref{1.8}, ассоциированные с системами косинусов и системой функций Лагерра.  Показано, что  ряды Фурье по системам функций и полиномов, ортогональным по Соболеву относительно скалярного произведения \eqref{1.8}, и их частичные суммы тесно связаны с задачей Коши для систем обыкновенных дифференциальных уравнений, а также рядом других задач, в которых требуется одновременно приближать дифференцируемые функции и несколько их производных.


