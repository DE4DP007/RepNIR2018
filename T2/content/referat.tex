\Referat %Реферат отчёта, не более 1 страницы

Отчет содержит X~с., X~источников.

 \bigskip
 \textbf{ Ключевые
  слова:}
  СИСТЕМЫ ФУНКЦИЙ, ОРТОГОНАЛЬНЫЕ ПО СОБОЛЕВУ; РЯДЫ ФУРЬЕ; СМЕШАННЫЕ РЯДЫ; ЗАДАЧА КОШИ; РАЗРЫВНАЯ ПРАВАЯ ЧАСТЬ;
  ТЕОРИЯ ПРИБЛИЖЕНИЙ; ОРТОГОНАЛЬНЫЕ ПОЛИНОМЫ; ПОЛИНОМЫ ЯКОБИ; ПОЛИНОМЫ ЧЕБЫШЕВА; ПОЛИНОМЫ ЛЕЖАНДРА; ПОЛИНОМЫ ЛАГЕРРА; ФУНКЦИИ УОЛША; НАИЛУЧШЕЕ ПРИБЛИЖЕНИЕ И СКОРОСТЬ СХОДИМОСТИ.

 \bigskip

Объект исследования --- ряды Фурье по системам функций, ортогональным в смысле Соболева и ассоциированным с классическими ортогональными системами.

Цель работы --- исследование свойств ортогональных по Соболеву систем функций, изучение аппроксимативных свойств рядов Фурье по указанным системам и применение рядов Фурье по этим системам к решению краевых задач для нелинейных систем ОДУ.

В ходе выполнения НИР получены следующие результаты. 

1. Изучены алгебраические и асимптотические свойства функций, ортогональных по Соболеву, порожденных системой полиномов Чебышева первого рода, системой косинусов, системой функций Лагерра, системой модифицированных полиномов Лагерра, системой функций Уолша. 

2. Исследованы вопросы сходимости рядов Фурье по этим системам и в некоторых случаях изучены аппроксимативные свойства частичных сумм указанных рядов Фурье.

3. На основе систем функций, ортогональных по Соболеву, разработаны итерационные алгоритмы для численно-аналитического решения задачи Коши для систем линейных и нелинейных дифференциальных. Для широкого класса систем функций, ортогональных по Соболеву, найдены условия, при соблюдении которых сходятся итерационные процессы, на которых основываются указанные алгоритмы для приближенного решения систем дифференциальных уравнений.
Ряд разработанных алгоритмов доведены до численных экспериментов (разработаны прикладные программные пакеты, реализующие указанные алгоритмы).
Проведенные эксперименты показывают высокую эффективность предлагаемого численно-аналитического подхода к решению систем дифференциальных и разностных уравнений.

4. Рассмотрены вопросы существования и единственности решения задачи Коши для ОДУ с разрывной правой частью.
  
Полученные результаты могут найти применение в задачах обработки и сжатия временных рядов и изображений, приближённого решения краевых задач для систем дифференциальных уравнений численно-аналитическими методами, численного обращения преобразования Лапласа, идентификации линейных и нелинейных систем автоматического регулирования и др.
