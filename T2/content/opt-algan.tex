\section{О представлении решения задачи Коши для систем ОДУ рядом Фурье по полиномам $T_{r,n}(x)$}
В данном подразделе мы рассмотрим задачу о приближении решения задачи Коши для систем ОДУ  суммами  Фурье по системе $\{T_{r,n}(x)\}_{n=0}^\infty$, ортогональной по Соболеву и порожденной ортонормированной системой полиномов Чебышева $T=\{T_{n}(x)\}_{n=0}^\infty$ посредством равенств \eqref{du2018cheb-1.4} и \eqref{du2018cheb-1.5} с $r=1$.
 Мы будем рассматривать задачу Коши для систем ОДУ вида

\begin{equation}\label{algan-3.1}
y'(x)=f(x,y), \quad y(a)=y^0,
\end{equation}
где  $f=(f_1, \ldots, f_m)$, $y=(y_1, \ldots, y_m)$. Как и в подразделе \ref{SobSystemsAndCauchyProblem}, будем считать вектор-функцию $f(x,y)$  непрерывной в некоторой замкнутой  области $\bar G$ переменных $(x,y)$, содержащей точку $(a,y_0)$. Кроме того, предположим, что $[a,b]\times\mathbb{R}^m\subset\bar G$. Положим также, что по переменной $y$ функция $f(x,y)$ удовлетворяет условию Липшица
 \begin{equation}\label{algan-3.2}
\|f(x,u)-f(x,v)\|\le \lambda\|u-v\|, \quad a\le x \le b,
\end{equation}
где $\|(u_1,\ldots,u_m)\|=\sqrt{\sum_{l=1}^m u_l^2}$.

Через $h$ обозначим положительное число, для которого $h\lambda\kappa_T<1$, где  величина $\kappa_T$ определена равенством \eqref{du2018cheb-3.6} и имеет значение, приведенное в теореме \ref{du2018cheb-th1}. Если, в частности, $\lambda\kappa_T<1$, то мы можем взять $h=1$. Будем  решать уравнение \eqref{algan-3.1} на отрезках вида $[a+2jh,a+2(j+1)h]$ $(j=0,1,\ldots)$, остановившись для определенности на $[a,a+2h]$. Полагая $x=a+h(t+1)$, отобразим линейно отрезок $[-1,1]$ на $[a,a+2h]$   Относительно новой переменной $t\in [-1,1]$ уравнение \eqref{algan-3.1} принимает следующий вид
\begin{equation}\label{algan-3.3}
\eta'(t)=hf(a+h(t+1),\eta(t)), \quad \eta(-1)=y^0,\quad -1\le t\le  1,
\end{equation}
где $\eta(t)=y(a+h(t+1))$.  
Повторяя рассуждения, аналогичные приведенным в подразделе \ref{CauchyProblemSolutionRepr}, мы можем получить утверждение, аналогичное тому, что содержит теорема \ref{th3}.


Положим
\begin{equation}\label{algan-3.30}
  Y_{1,N}(\phi,x)= \phi(-1)+ \sum\nolimits_{k=1}^N  \phi_{1,k}T_{1,k}(x)
\end{equation}
и заметим,  что величина вида
\begin{equation}\label{algan-3.31}
 V_N(\phi,x)=\phi(x)- Y_{1,N}(\phi,x)
=\sum\nolimits_{j=N+1}^\infty  \phi_{1,j}T_{1,j}(x),
\end{equation}
фигурирующая в правой части равенства \eqref{4.29}, представляет собой остаточный член ряда Фурье функции $\phi$ по функциям $T_{1,k}(t)$ $(k=0,1,\ldots)$, образующим ортонормированную систему по Соболеву относительно скалярного произведения \eqref{du2018cheb-1.3}, порожденным  полиномами Чебышева $T_k$ посредством равенств \eqref{du2018cheb-1.4} и \eqref{du2018cheb-1.5} с $r=1$.  Неравенство \eqref{4.28} непосредственно приводит к  задаче об исследовании поведения величины $$R_N(\phi)=\left(\int_{-1}^1(\phi(t)- Y_{1,N}(\phi,t))^2 \rho(t)dt\right)^\frac12.$$
Другими словами, требуется исследовать задачу об оценке приближения функции $\phi$ в метрике пространства $L^2_{\rho}$ частичными суммами $ Y_{1,N}(\phi,t)$  ряда Фурье
$$
\phi(x)= \phi(-1)+ \sum\nolimits_{k=1}^\infty  \phi_{1,k}T_{1,k}(x),
$$
по полиномам $T_{1,k}(x)$, в котором
\begin{equation}\label{algan-3.32}
  \phi_{1,k}=\int_{-1}^1\phi'(t)T_{k-1}(t)\rho(t)dt \quad(k=1,2,\ldots).
\end{equation}
При этом следует заметить, что  полиномы $T_{1,k}(x)$ $(k=0,1,\ldots)$ не образуют ортонормированной системы в $L^2_{\rho}$, поэтому указанная задача не сводится к простому применению равенства Парсеваля или неравенства Бесселя, с помощью которых легко оценивается норма остаточного члена ряда Фурье в гильбертовом пространстве. И тем не менее, с помощью леммы \ref{du2018cheb-lemA} удается свести поставленную задачу об оценке величины $R_N(\phi)$  к задаче об оценке  остаточного члена некоторого  ряда Фурье по полиномам Чебышева $T_{k}(x)$ $(k=0,1,\ldots)$ в пространстве $L^2_{\rho}$. Справедлива следующая
 \begin{lemma}\label{algan-lemB}
 Если $\phi\in W^1_{L^2_{\rho}}$, то имеет место  равенство
 $$
\int_{-1}^{1}(V_N(\phi,t))^2\rho(t)dt=
2\left(\sum_{k=N+1}^{\infty}\phi_{1,k}\frac{(-1)^{k-1}}{(k-1)^2-1}\right)^2+
$$
\begin{equation}\label{algan-3.33}
{(\phi_{1,N+1})^2\over4(N-1)^2} +{(\phi_{1,N+2})^2\over4N^2}
+\sum_{k=N+1}^{\infty}\frac{(\phi_{1,k}-\phi_{1,k+2})^2}{4k^2}.
\end{equation}

 \end{lemma}

Обозначим через $E^\rho_ m(g)$ -- наилучшее приближение функции $g\in L^2_\rho$ алгебраическими полиномами степени $m$, т.е.
$$
 E^\rho_ m(g)=\inf_{p_m}\left(\int_{-1}^1(g(t)- p_m(t))^2 \rho(t)dt\right)^\frac12,
$$
где нижняя грань берется по всем алгебраическим полиномам $p_m$ степени не выше $m$.

\begin{theorem}\label{algan-th4}
     Имеет место  неравенство
$$
R_N(\phi)\le I_NE^\rho_ {N-1}(\phi'),
$$
где
$$
I_N=\left(\frac{1}{(N-1)^2}+\sum_{k=N}^{\infty}\frac{2}{(k^2-1)^2}\right)^\frac12.
 $$
\end{theorem}

Для вектор-функции $\eta'=(\eta'_1,\ldots,\eta'_m)$ положим
$$
E^\rho_ m(\eta')=\left(\sum\nolimits_{l=1}^m (E^\rho_ m(\eta'_l))^2\right)^\frac12.
$$
Тогда из \eqref{4.28}, \eqref{4.29} и теоремы \ref{algan-th4} вытекает
\begin{corollary}\label{algan-cor1}
  Пусть вектор-функция $\eta=(\eta_1,\ldots,\eta_m)$ представляет собой решение задачи Коши для системы ОДУ \eqref{algan-3.3} и выполнены условия теоремы 5. Тогда имеет место неравенство
$$
\|C_N(q)-\bar C_N(q)\|_N^m\le \frac{\lambda I_NE^\rho_{N-1}(\eta') }{1-h\kappa_T\lambda}.
$$
  \end{corollary}

Теорема \ref{algan-th4}, в которой получена оценка скорости сходимости частичных сумм  ряда Фурье по полиномам  $T_{1,k}(x)$ к функции $\phi$ в метрике пространства $L^2_\rho$, дает, как это показано в следствии \ref{algan-cor1}, один из возможных подходов к оценке погрешности, которая проистекает в результате замены искомой матрицы $C_N(q)$, составленной из неизвестных коэффициентов  разложения решения задачи Коши \eqref{algan-3.3}, приближенной матрицей $\bar C_N(q)$, являющейся неподвижной точкой конечномерного оператора $A_N$, сконструированного по правилу \eqref{4.18}. Но после того, как эта матрица будет  найдена с требуемой точностью, немедленно возникает вопрос о том, какова максимальная по $x\in[-1,1]$ погрешность замены точного решения  задачи Коши \eqref{algan-3.3} частичной суммой его разложения по полиномам $T_{1,k}(x)$. Ответ на этот вопрос для $r$ раз дифференцируемых и аналитических функций дается в теоремах \ref{du2018cheb-th3} и \ref{du2018cheb-th4}. 