\begin{thebibliography}{111}
\bibitem{Faber}
Faber G. Uber die Orthogonalenfunctionen des Herrn Haar // Jahresber. Deutsch Math. Verein. --- 1910. --- № 19. --- P. 104---112.








\bibitem{Shauder}
Shauder J. Zur Theorir stetiger Abbildungen in Functionalreumen // Math. Z. --- 1927. --- № 26. --- P. 47---65.








\bibitem{SharIzv2018}
Шарапудинов И.И. Системы функций, ортогональные по Соболеву, ассоциированные с ортогональной системой // Изв. РАН. Сер. матем. --- 2018. --- Т. 82. --- № 1. --- С. 225---258.








\bibitem{SharDiffur2018}
Шарапудинов И.И., Магомед-Касумов M.Г. О представлении решения задачи Коши  рядом Фурье  по полиномам, ортогональным по  Соболеву, порожденным многочленами Лагерра // Дифференциальные уравнения. --- 2018. --- Т. 54. --- № 1. --- С. 51---68.








\bibitem{MMG2016}
Магомед-Касумов М.Г. Приближенное решение обыкновенных дифференциальных уравнений с использованием смешанных рядов по системе Хаара // Материалы 18-й международной Саратовской зимней школы «Современные проблемы теории функций и их приложения». --- 2016. --- С. 176---178.








\bibitem{Althammer}
Althammer P. Eine Erweiterung des Orthogonalitatsbegriffes bei Polynomen und deren Anwendung  auf die beste Approximation // J. Reine Angew. Math. --- 1962. --- № 211. P. 192---204.








\bibitem{IserKoch}
Iserles A., Koch P.E., Norsett S.P. and Sanz-Serna J.M. On polynomials  orthogonal  with respect  to certain Sobolev inner products // J. Approx. Theory. --- 1991. --- № 65. --- P. 151---175.








\bibitem{KwonLittl2}
Kwon K.H. and Littlejohn L.L. Sobolev orthogonal polynomials and second-order differential equations // Rocky Mountain J. Math. --- 1998. --- № 28. --- P. 547---594.








\bibitem{MarcelXu}
Marcellan F. and Yuan Xu On Sobolev orthogonal polynomials // Expositiones Mathematicae. --- 2015. --- V. 33. --- № 3. --- P. 308---352.








\bibitem{Gonchar1975}
Гончар А.А. О сходимости аппроксимаций Паде для некоторых классов мероморфных функций // Матем. сб. --- 1975. --- Т. 97(139). --- № 4(8). --- С. 607---629.








\bibitem{Lopez1995}
Lopez G. Marcellan F. Vanassche W. Relative Asymptotics for Polynomials Orthogonal with Respect to a Discrete Sobolev Inner-Product // Constr. Approx. --- 1995. --- № 111. --- P. 107---137.







\bibitem{Shar11}
Шарапудинов И.И. Приближение функций с переменной гладкостью суммами Фурье Лежандра // Математический сборник. --- 2000. --- Т. 191. --- № 5. --- С. 143---160.








\bibitem{Shar2003}
Шарапудинов И.И. Смешанные ряды по ультрасферическим полиномам и их аппроксимативные свойства // Математический сборник. --- 2003. --- Т. 194. --- № 3. --- С. 115---148.








\bibitem{Shar2006}
Шарапудинов И.И. Аппроксимативные свойства смешанных рядов по полиномам Лежандра на классах $W^r$ // Математический сборник. --- 2006. --- Т. 197. --- № 3. --- С. 135---154.







\bibitem{Shar2008}
Шарапудинов И.И. Аппроксимативные свойства средних типа Валле-Пуссена частичных сумм смешанных рядов по полиномам Лежандра // Математические заметки. --- 2008. --- Т. 84. --- № 3. --- С. 452---471.







\bibitem{Shar19}
Шарапудинов И.И., Муратова Г.Н. Некоторые свойства r-кратно интегрированных рядов по системе Хаара // Изв. Сарат. ун-та. Нов. сер. Сер. Математика. Механика. Информатика. --- 2009. --- Т. 9. --- № 1. 68---76.








\bibitem{Shar18}
Шарапудинов И.И., Шарапудинов Т.И. Смешанные ряды по полиномам Якоби и Чебышева и их дискретизация // Математические заметки. --- 2010. --- Т. 88. --- № 1. --- С. 116---147.








\bibitem{sharap3}
Шарапудинов И.И. Некоторые специальные ряды по ультрасферическим полиномам и их аппроксимативные свойства // Изв. РАН. Сер. матем. --- 2014. --- Т. 78. --- № 5. --- С. 201---224.








\bibitem{Shar_Dag_Elec}
Шарапудинов И.И. Смешанные ряды по классическим ортогональным полиномам // Дагестанские электронные математические известия. --- 2015. --- № 3. --- С. 1---254.








\bibitem{SHII}
Шарапудинов И.И. Некоторые специальные ряды по общим полиномам Лагерра и ряды Фурье по полиномам Лагерра, ортогональным по Соболеву // Дагестанские электронные математические известия. --- 2015. --- №. 4. --- С. 31---73.








\bibitem{Shar2017}
Шарапудинов И.И. Аппроксимативные свойства рядов Фурье по многочленам, ортогональным по Соболеву с весом Якоби и дискретными массами // Математические заметки. --- 2017. --- Т. 101. --- № 4. --- С. 611---629.








\bibitem{SharSMJ2017}
Шарапудинов И.И. Специальные ряды по полиномам Лагерра и их аппроксимативные свойства // Сиб. матем. журн. --- 2017. --- Т. 58. --- № 2. --- С. 440---467.








\bibitem{Shar12}
Шарапудинов И.И. Аппроксимативные свойства операторов $\mathcal{Y}_{n+2r}(f)$ и их дискретных аналогов // Математические заметки. --- 2002. --- Т. 72. № 5. --- С. 765---795.








\bibitem{Shar13}
Шарапудинов И.И. Смешанные ряды по ортогональным полиномам. --- 2004. --- С. 1---176.








\bibitem{Shar14}
Шарапудинов И.И. Смешанные ряды по полиномам Чебышева, ортогональным на равномерной сетке // Математические заметки. --- 2005. --- Т. 78. --- № 3. --- С. 442---465.







\bibitem{Sege}
Сеге Г. Ортогональные многочлены. --- 1962.








\bibitem{Pash}
Пашковский С. Вычислительные применения многочленов и рядов Чебышева. --- 1983.








\bibitem{Timan}
Тиман А.Ф. Теория приближения функций действительного переменного. --- 1960.








\bibitem{du2018cheb-Ahiezer}
Ахиезер Н.И. Лекции по теории аппроксимации. --- 1965.

%=========Ramis







\bibitem{du2018cheb-SHII-MMG2018}
Шарапудинов И.И., Магомед-Касумов М.Г. Численный метод решения задачи Коши для систем обыкновенных дифференциальных уравнений с помощью ортогональной в смысле Соболева системы, порожденной системой косинусов // Дагестанские Электронные Математические Известия. --- 2017. --- №. 8. --- С. 53---60.








\bibitem{du2018cheb-Arush2010}
Арушанян О.Б., Волченскова Н.И., Залеткин С.Ф. Приближенное решение обыкновенных дифференциальных уравнений с использованием рядов Чебышева // Сиб. электрон. матем. изв. --- 1983. --- № 7. --- С. 122---131.








\bibitem{du2018cheb-Arush2013}
Арушанян О.Б., Волченскова Н.И., Залеткин С.Ф. Метод решения задачи Коши для обыкновенных дифференциальных уравнений с использованием рядов Чебышeва // Выч. мет. программирование. --- 2013. --- № 142. --- С. 203---214.








\bibitem{du2018cheb-Arush2014}
Арушанян О.Б., Волченскова Н.И., Залеткин С.Ф. Применение рядов Чебышева для интегрирования обыкновенных дифференциальных уравнений // Сиб. электрон. матем. изв. --- 2014. --- № 11. --- С. 517---531.








\bibitem{Dzjadyc}
Дзядык В.К. Введение в теорию равномерного приближения функций полиномами. --- 1977.








\bibitem{AskeyWaiger}
Askey R. Wainger S. Mean convergence of expansions in Laguerre and Hermite series // Amer. J. Mathem. --- 1965. --- № 87. --- С. 698---708.

%%%==========report 2017========







\bibitem{rep2017-ramis-Gadz16}
Шарапудинов И.И., Шарапудинов Т.И. Полиномы, ортогональные по Соболеву, порожденные многочленами Чебышева, ортогональными на сетке // Изв. вузов. Матем. --- 2017. --- № 8. --- С. 67---79.









\bibitem{rep2017-ramis-Gadz1}
Шарапудинов И.И., Гаджиева З.Д., Гаджимирзаев Р.М. Системы функций, ортогональных относительно скалярных произведений типа Соболева с дискретными массами, порожденных классическими ортогональными системами // Дагестанские электронные математические известия. 2016. --- Вып. 6. --- С. 31---60.









\bibitem{rep2017-ramis-shGadjGadjMir}
Шарапудинов И.И., Гаджиева З.Д., Гаджимирзаев Р.М.
Разностные уравнения и полиномы, ортогональные по Соболеву, порожденные многочленами Мейкснера //
Владикавказский Мат. журнал. --- 2017. --- Т.19. --- Вып. 2. --- С. 58---72.









\bibitem{rep2017-sobcheb_urav-fiht2}
Фихтенгольц Г.М.
Курс дифференциального и интегрального исчисления. --- Москва: Физматлит. --- 2001. --- Т. 2. --- Стр. 810.









\bibitem{walsh-GolubovBook}
Голубов Б.И., Ефимов А.В., Скворцов В.А. Ряды и преобразования Уолша: Теория и применения. --- Гл. ред. физ.-мат. лит. Москва: Наука, 1987. --- 344 с.








\bibitem{walsh-ShII-2015-demi}
Шарапудинов И.И., Магомед-Касумов М.Г., Магомедов С.Р. Полиномы, ортогональные по Соболеву, ассоциированные с полиномами Чебышева первого рода // Дагестанские электронные математические известия. --- 2015. --- № 4. --- С. 1---14.








\bibitem{SharIZVUZ}
Шарапудинов И.И., Шарапудинов Т.И. Полиномы, ортогональные по Соболеву, поржденные многогчленами Чебышева, ортогональными на сетке // Изв. вузов. Матем. --- 2017. --- № 8. --- С. 67---79.








\bibitem{walsh-ShII-meix-2016}
Шарапудинов И.И., Гаджиева  З.Д. Полиномы, ортогональные по Соболеву, порожденные многочленами Мейкснера // Изв. Сарат. ун-та. Нов. сер. Сер. Математика. Механика. Информатика. --- 2016. --- Т. 16. --- № 3. --- С. 310---321.








\bibitem{walsh-Gadzh-2016-saratov}
Гаджимирзаев Р.М. Ряды Фурье по полиномам Мейкснера, ортогональным по Соболеву // Изв. Сарат. ун-та. Нов. сер. Сер. Математика. Механика. Информатика. --- 2016. --- Т. 16. --- № 4. --- С. 388---395.






\end{thebibliography} 