\begin{thebibliography}{20}
	
\bibitem{SharIzv2018}
И.И.Шарапудинов,"Системы функций, ортогональные по Соболеву, ассоциированные с ортогональной системой", Изв. РАН. Сер. матем., 82:1 (2018), 225 –- 258.



\bibitem{Shar11}
И.\,И. Шарапудинов,"Приближение функций с переменной гладкостью суммами Фурье Лежандра", Математический сборник, 191:5 (2000), 143 -- 160..



\bibitem{Shar2003}
И.\,И. Шарапудинов,"Смешанные ряды по ультрасферическим полиномам и их аппроксимативные свойства", Математический сборник, 194:3 (2003), 115--148.



\bibitem{Shar2006}
И.\,И. Шарапудинов,"Аппроксимативные свойства смешанных рядов по полиномам Лежандра на классах $W^r$", Математический сборник, 197:3 (2006), 135–154..



\bibitem{Shar2008}
И.\,И. Шарапудинов,"Аппроксимативные свойства средних типа Валле-Пуссена частичных сумм смешанных рядов по полиномам Лежандра", Математические заметки, 84:3 (2008), 452 -- 471..



\bibitem{Shar19}
И.\,И. Шарапудинов,  Г.\, Н. Муратова,"Некоторые свойства r-кратно интегрированных рядов по системе Хаара", Изв. Сарат. ун-та. Нов. сер. Сер. Математика. Механика. Информатика, 9:1 (2009), 68 -- 76.



\bibitem{Shar18}
И.\,И. Шарапудинов, Т.\, И. Шарапудинов,"Смешанные ряды по полиномам Якоби и Чебышева и их дискретизация", Математические заметки, 88:1 (2010), 116 -- 147.



\bibitem{sharap3}
И.\,И. Шарапудинов,"Некоторые специальные ряды по ультрасферическим полиномам и их аппроксимативные свойства", Изв. РАН. Сер. матем., 78:5 (2014), 201 -- 224.



\bibitem{Shar_Dag_Elec}
И.\,И. Шарапудинов,"Смешанные ряды по классическим ортогональным полиномам", Дагестанские электронные математические известия, 3 (2015), 1 -- 254.



\bibitem{SHII}
И.\,И. Шарапудинов,"Некоторые специальные ряды по общим полиномам Лагерра и ряды Фурье по полиномам Лагерра, ортогональным по Соболеву", Дагестанские электронные математические известия, 4 (2015), 31 -- 73.



\bibitem{Shar2017}
И.\,И. Шарапудинов,"Аппроксимативные свойства рядов Фурье по многочленам, ортогональным по Соболеву с весом Якоби и дискретными массами", Математические заметки, 101:4 (2017), 611 -– 629..



\bibitem{SharSMJ2017}
И.И.Шарапудинов,"Специальные ряды по полиномам Лагерра и их аппроксимативные свойства", Сиб. матем. журн., 58:2 (2017), 440–467.



\bibitem{Shar12}
И.\,И. Шарапудинов,"Аппроксимативные свойства операторов $\mathcal{Y}_{n+2r}(f)$ и их дискретных аналогов", Математические заметки, 72:5 (2002), 765–-795..



\bibitem{Shar13}
И.\,И. Шарапудинов,"Смешанные ряды по ортогональным полиномам", ,  (2004), 1 --176.



\bibitem{Shar14}
И.\,И. Шарапудинов,"Смешанные ряды по полиномам Чебышева, ортогональным на равномерной сетке", Математические заметки, 78:3 (2005), 442–-465..



\bibitem{Sege}
Г. Сеге,"", Ортогональные многочлены,  (1962), .



\bibitem{Pash}
С. Пашковский,"", Вычислительные применения многочленов и рядов Чебышева,  (1983), .



\bibitem{Timan}
А.Ф. Тиман,"", Теория приближения функций действительного переменного,  (1960), .



\bibitem{du2018cheb-Ahiezer}
Н.И. Ахиезер,"", Лекции по теории аппроксимации,  (1965), .

%=========Ramis


\bibitem{MMG2016}
М.Г. Магомед-Касумов,"Приближенное решение обыкновенных дифференциальных уравнений с использованием смешанных рядов по системе Хаара", Материалы 18-й международной Саратовской зимней школы «Современные проблемы теории функций и их приложения»,  (2016), 176-178.



\bibitem{du2018cheb-SHII-MMG2018}
И.И. Шарапудинов, М.Г. Магомед-Касумов,"Численный метод решения задачи Коши для систем обыкновенных дифференциальных уравнений с помощью ортогональной в смысле Соболева системы, порожденной системой косинусов", Дагестанские Электронные Математические Известия, 8 (2017), 53 -- 60.



\bibitem{du2018cheb-Arush2010}
О. Б. Арушанян, Н. И. Волченскова, С. Ф. Залеткин,"Приближенное решение обыкновенных дифференциальных уравнений с использованием рядов Чебышева", Сиб. электрон. матем. изв., 7 (1983), 122–131.



\bibitem{du2018cheb-Arush2013}
О. Б. Арушанян, Н. И. Волченскова, С. Ф. Залеткин,"Метод решения задачи Коши для обыкновенных дифференциальных уравнений с использованием рядов Чебышeва", Выч. мет. программирование, 142 (2013), 203--214.



\bibitem{du2018cheb-Arush2014}
О. Б. Арушанян, Н. И. Волченскова, С. Ф. Залеткин,"Применение рядов Чебышева для интегрирования обыкновенных дифференциальных уравнений", Сиб. электрон. матем. изв., 11 (2014), 517--531.



\bibitem{Dzjadyc}
В.К. Дзядык,"", Введение в теорию равномерного приближения функций полиномами,  (1977), .



\bibitem{Tref1}
L.N. Trefethen,Spectral methods in Matlab. SIAM, Fhiladelphia, 2000,  с..



\bibitem{Tref2}
L.N. Trefethen,Finite difference and spectral methods for ordinary and partial differential equation. Cornell University, , 1996,  с..



\bibitem{SolDmEg}
В.В. Солодовников, А.Н. Дмитриев, Н.Д. Егупов,Спектральные методы расчета и проектирования систем управления. Машиностроение, Москва, 1986,  с..



\bibitem{AskeyWaiger}
R. Askey, S. Wainger,"Mean convergence of expansions in Laguerre and Hermite series", Amer. J. Mathem., 87 (1965), 698 -- 708.

%%%==========report 2017========


\bibitem{rep2017-laplas-AskeyWaiger}
{Askey R., Wainger S.}
Mean convergence of expansions in Laguerre and Hermite series // Amer. J. Mathem. 1965. Vol. 87. Pp. 698--708.




\bibitem{SharDiffur2018}
И.\,И. Шарапудинов, M.\,Г. Магомед-Касумов,"О представлении решения задачи Коши  рядом Фурье  по полиномам, ортогональным по  Соболеву, порожденным многочленами Лагерра", Дифференциальные уравнения, 54:1 (2018), 51 -- 68.



\bibitem{Ramlib4}
{Шарапудинов И.И.} Системы функций, ортогональные по Соболеву, ассоциированные с ортогональной системой // Изв. РАН. Сер. матем., 82:1, 2018. С. 225--258.



\bibitem{rep2017-ramis-Gadz16}
{Шарапудинов~И.~И., Шарапудинов~Т.~И.} Полиномы, ортогональные по Соболеву, порожденные многочленами Чебышева, ортогональными на сетке~// Изв. вузов. Матем., 2017, №~8, 67--79.




\bibitem{Ramlib42}
{Шарапудинов И.И.} Аппроксимативные свойства рядов Фурье по многочленам, ортогональным по Соболеву с весом Якоби и дискретными массами // Матем. заметки, 101:4, 2017. С. 611--629.



\bibitem{rep2017-ramis-Gadz1}
{Шарапудинов~И.~И., Гаджиева~З.~Д., Гаджимирзаев~Р.~М.} Системы функций, ортогональных относительно скалярных произведений типа Соболева с дискретными массами, порожденных классическими ортогональными системами~// Дагестанские электронные математические известия. 2016. Вып.~6. С.~31--60.




\bibitem{rep2017-ramis-shGadjGadjMir}
{Шарапудинов И.И., Гаджиева З.Д., Гаджимирзаев Р.М.}
Разностные уравнения и полиномы, ортогональные по Соболеву, порожденные многочленами Мейкснера //
Владикавказский Мат. журнал, 2017. Т.19. Вып. 2. С. 58--72.




\bibitem{rep2017-sobcheb_urav-fiht2}
{Фихтенгольц Г.М.}
Курс дифференциального и интегрального исчисления // Физматлит. Москва. 2001. Т. 2. Стр. 810.




\bibitem{walsh-GolubovBook}
Б.\,И. Голубов, А.\,В. Ефимов, В.\,А. Скворцов,Ряды и преобразования Уолша: Теория и применения. Наука. Гл. ред. физ.-мат. лит, Москва, 1987, 344 с.



\bibitem{walsh-ShII-2018-IzvRan}
И.\,И. Шарапудинов,"Системы функций, ортогональные по Соболеву, ассоциированные с ортогональной системой", Изв. РАН. Сер. матем., 82:1 (2018), 225--258.



\bibitem{walsh-ShII-2015-demi}
И.~И.~Шарапудинов, М.~Г.~Магомед-Касумов, С.~Р.~Магомедов,"Полиномы, ортогональные по Соболеву, ассоциированные с полиномами Чебышева первого рода", Дагестанские электронные математические известия, 4 (2015), 1--14.



\bibitem{SharIZVUZ}
И.\,И. Шарапудинов, Т.\, И. Шарапудинов,"Полиномы, ортогональные по Соболеву, поржденные многогчленами Чебышева, ортогональными на сетке", Изв. вузов. Матем., 8 (2017), 67--79.



\bibitem{walsh-ShII-meix-2016}
И.~И.~Шарапудинов, З.~Д.~Гаджиева,"Полиномы, ортогональные по Соболеву, порожденные многочленами Мейкснера", Изв. Сарат. ун-та. Нов. сер. Сер. Математика. Механика. Информатика, 16:3 (2016), 310--321.



\bibitem{walsh-Gadzh-2016-saratov}
Р.~М.~Гаджимирзаев,"Ряды Фурье по полиномам Мейкснера, ортогональным по Соболеву", Изв. Сарат. ун-та. Нов. сер. Сер. Математика. Механика. Информатика, 16:4 (2016), 388--395.



\bibitem{walsh-fiht2}
Г.\,М. Фихтенгольц,Курс дифференциального и интегрального исчисления. Физматлит, Москва, 2001, 810 с.




\bibitem{Faber}
G. Faber,"Uber die Orthogonalenfunctionen des Herrn Haar", Jahresber. Deutsch Math. Verein., 19 (1910), 104 –- 112.



\bibitem{Shauder}
J. Shauder,"Zur Theorir stetiger Abbildungen in Functionalreumen", Math. Z., 26 (1927), 47 –- 65.



\bibitem{Althammer}
P. Althammer,"Eine Erweiterung des Orthogonalitatsbegriffes bei Polynomen und deren Anwendung  auf die beste Approximation", J. Reine Angew. Math. 211 (1962), 192–204., 211 (1962), 192-204.



\bibitem{IserKoch}
A. Iserles, P.E. Koch, S.P. Norsett and J.M. Sanz-Serna,"On polynomials  orthogonal  with respect  to certain Sobolev inner products", ,  J. Approx. Theory, 65 (1991), 151-175..



\bibitem{KwonLittl2}
K.\,H. Kwon and L.\,L. Littlejohn,"Sobolev orthogonal polynomials and second-order differential equations", Rocky Mountain J. Math., 28 (1998), 547 –- 594..



\bibitem{MarcelXu}
F. Marcellan and Yuan Xu,"On Sobolev orthogonal polynomials", Expositiones Mathematicae, 33:3 (2015), 308--352.



\bibitem{Gonchar1975}
А. А. Гончар,"О сходимости аппроксимаций Паде для некоторых классов мероморфных функций", Матем. сб., 97(139)4(8) (1975), 607–629.



\bibitem{Lopez1995}
Lopez G. Marcellan F. Vanassche W.,"Relative Asymptotics for Polynomials Orthogonal with Respect to a Discrete Sobolev Inner-Product", Constr. Approx., 111 (1995), 107–137.



\bibitem{Bav}
H. Bavinck,"On polynomials orthogonal with respect to an inner product", J. Comput. Appl. Math., 57 (1995), 17--27.



\bibitem{SharGus}
И.\,И. Шарапудинов, И.Г. Гусейнов,"Полиномы, ортогональные по Соболеву, порожденные полиномами Шарлье", Изв. Сарат. ун-та. Нов. сер. Сер. Математика. Механика. Информатика, 18:2 (2018), 196 -- 205.


\end{thebibliography}