\begin{thebibliography}{20}


\bibitem{Faber}
G. Faber,"Uber die Orthogonalenfunctionen des Herrn Haar", Jahresber. Deutsch Math. Verein., 19 (1910), 104 –- 112.

\bibitem{Shauder}
J. Shauder,"Zur Theorir stetiger Abbildungen in Functionalreumen", Math. Z., 26 (1927), 47 –- 65.

\bibitem{SharIzv2018}
И.И.Шарапудинов,"Системы функций, ортогональные по Соболеву, ассоциированные с ортогональной системой", Изв. РАН. Сер. матем., 82:1 (2018), 225 –- 258.

\bibitem{SharDiffur2018}
И.\,И. Шарапудинов, M.\,Г. Магомед-Касумов,"О представлении решения задачи Коши  рядом Фурье  по полиномам, ортогональным по  Соболеву, порожденным многочленами Лагерра", Дифференциальные уравнения, 54:1 (2018), 51 -- 68.

\bibitem{MMG2016}
М.Г. Магомед-Касумов,"Приближенное решение обыкновенных дифференциальных уравнений с использованием смешанных рядов по системе Хаара", Материалы 18-й международной Саратовской зимней школы «Современные проблемы теории функций и их приложения»,  (2016), 176-178.

\bibitem{Althammer}
P. Althammer,"Eine Erweiterung des Orthogonalitatsbegriffes bei Polynomen und deren Anwendung  auf die beste Approximation", J. Reine Angew. Math. 211 (1962), 192–204., 211 (1962), 192-204.

\bibitem{IserKoch}
A. Iserles, P.E. Koch, S.P. Norsett and J.M. Sanz-Serna,"On polynomials  orthogonal  with respect  to certain Sobolev inner products", ,  J. Approx. Theory, 65 (1991), 151-175..

\bibitem{Meijer}
H.\,G. Meijer,"Laguerre polynimials generalized to a certain discrete Sobolev inner product space", ,  J. Approx. Theory, 73 (1993), 1-16..

\bibitem{MarcelAlfaroRezola }
F. Marcellan, M. Alfaro and M.L. Rezola,"Orthogonal polynomials on Sobolev spaces: old and new directions", Journal of Computational and Applied Mathematics, 48 (1993), 113 -- 131..

\bibitem{Lopez1995}
Lopez G. Marcellan F. Vanassche W.,"Relative Asymptotics for Polynomials Orthogonal with Respect to a Discrete Sobolev Inner-Product", Constr. Approx., 111 (1995), 107–137.

\bibitem{KwonLittl1}
K.\,H. Kwon and L.\,L. Littlejohn,"The orthogonality of the Laguerre polynomials $\{L_n^{(-k)}(x)\}$ for positive integers $k$", Ann. Numer. Anal., 2 (1995), 289 –- 303..

\bibitem{KwonLittl2}
K.\,H. Kwon and L.\,L. Littlejohn,"Sobolev orthogonal polynomials and second-order differential equations", Rocky Mountain J. Math., 28 (1998), 547 –- 594..

\bibitem{MarcelXu}
F. Marcellan and Yuan Xu,"On Sobolev orthogonal polynomials", Expositiones Mathematicae, 33:3 (2015), 308--352.

\bibitem{Gonchar1975}
А. А. Гончар,"О сходимости аппроксимаций Паде для некоторых классов мероморфных функций", Матем. сб., 97(139)4(8) (1975), 607–629.

\bibitem{Bav}
H. Bavinck,"On polynomials orthogonal with respect to an inner product", J. Comput. Appl. Math., 57 (1995), 17--27.

\bibitem{Bav1}
H. Bavinck,"On polynomials orthogonal with respect to an inner product involving differences (The general case)", Appl. Anal., 59 (1995), 233--240.

\bibitem{BavKoe}
H. Bavinck, R. Koekoek,"Difference operators with Sobolev-type Meixner", Comput. Math. Appl., 36:10-12 (1998), 163--177.

\bibitem{AreaGobMar}
I. Area, E. Godoy, F. Marcellan,"Inner products involving differences: the Meixner--Sobolev polynomials", J. Difference Equations Appl., 6 (2000), 1--31.

\bibitem{Shar12}
И.\,И. Шарапудинов,"Аппроксимативные свойства операторов $\mathcal{Y}_{n+2r}(f)$ и их дискретных аналогов", Математические заметки, 72:5 (2002), 765–-795..

\bibitem{Shar13}
И.\,И. Шарапудинов,"Смешанные ряды по ортогональным полиномам", ,  (2004), 1 --176.

\bibitem{AreaGobMarMor}
I. Area, E. Godoy, F. Marcellan, J.J. Moreno-Balcazar,"$\Delta$-Sobolev orthogonal polynomials of Meixner type: asymptotics and limit relation", Journal of Computational and Applied Mathematics, 178:1-2 (2005), 21--36.

\bibitem{Shar14}
И.\,И. Шарапудинов,"Смешанные ряды по полиномам Чебышева, ортогональным на равномерной сетке", Математические заметки, 78:3 (2005), 442–-465..

\bibitem{Shar15}
И.\,И. Шарапудинов,"Приближение дискретных функций и многочлены Чебышева, ортогональные на равномерной сетке", Математические заметки, 67:3 (2000), 460–-470..

\bibitem{SharIZVUZ}
И.\,И. Шарапудинов, Т.\, И. Шарапудинов,"Полиномы, ортогональные по Соболеву, поржденные многогчленами Чебышева, ортогональными на сетке", Изв. вузов. Матем., 8 (2017), 67--79.

\bibitem{SharGadjieva}
И.\,И. Шарапудинов, З.\, Д.Гаджиева,"Полиномы, ортогональные по Соболеву, порожденные многочленами Мейкснера", Изв. Сарат. ун-та. Нов. сер. Сер. Математика. Механика. Информатика, 16:3 (2016), 310--321.

\bibitem{SharGadGad}
И.\,И. Шарапудинов, З.\, Д.Гаджиева, Р.\, М. Гаджимирзаев,"Разностные уравнения и полиномы, ортогональные по Соболеву, порожденные многочленами Мейкснера", Владикавк. матем. журн., 19:2 (2017), 58--72.

\bibitem{SharGus}
И.\,И. Шарапудинов, И.Г. Гусейнов,"Полиномы, ортогональные по Соболеву, порожденные полиномами Шарлье", Изв. Сарат. ун-та. Нов. сер. Сер. Математика. Механика. Информатика, 18:2 (2018), 196 -- 205.

\bibitem{Shar11}
И.\,И. Шарапудинов,"Приближение функций с переменной гладкостью суммами Фурье Лежандра", Математический сборник, 191:5 (2000), 143 -- 160..

\bibitem{Shar2003}
И.\,И. Шарапудинов,"Смешанные ряды по ультрасферическим полиномам и их аппроксимативные свойства", Математический сборник, 194:3 (2003), 115--148.

\bibitem{Shar2006}
И.\,И. Шарапудинов,"Аппроксимативные свойства смешанных рядов по полиномам Лежандра на классах $W^r$", Математический сборник, 197:3 (2006), 135–154..

\bibitem{Shar2008}
И.\,И. Шарапудинов,"Аппроксимативные свойства средних типа Валле-Пуссена частичных сумм смешанных рядов по полиномам Лежандра", Математические заметки, 84:3 (2008), 452 -- 471..

\bibitem{Shar19}
И.\,И. Шарапудинов,  Г.\, Н. Муратова,"Некоторые свойства r-кратно интегрированных рядов по системе Хаара", Изв. Сарат. ун-та. Нов. сер. Сер. Математика. Механика. Информатика, 9:1 (2009), 68 -- 76.

\bibitem{Shar18}
И.\,И. Шарапудинов, Т.\, И. Шарапудинов,"Смешанные ряды по полиномам Якоби и Чебышева и их дискретизация", Математические заметки, 88:1 (2010), 116 -- 147.

\bibitem{sharap3}
И.\,И. Шарапудинов,"Некоторые специальные ряды по ультрасферическим полиномам и их аппроксимативные свойства", Изв. РАН. Сер. матем., 78:5 (2014), 201 -- 224.

\bibitem{Shar_Dag_Elec}
И.\,И. Шарапудинов,"Смешанные ряды по классическим ортогональным полиномам", Дагестанские электронные математические известия, 3 (2015), 1 -- 254.

\bibitem{SHII}
И.\,И. Шарапудинов,"Некоторые специальные ряды по общим полиномам Лагерра и ряды Фурье по полиномам Лагерра, ортогональным по Соболеву", Дагестанские электронные математические известия, 4 (2015), 31 -- 73.

\bibitem{Shar2017}
И.\,И. Шарапудинов,"Аппроксимативные свойства рядов Фурье по многочленам, ортогональным по Соболеву с весом Якоби и дискретными массами", Математические заметки, 101:4 (2017), 611 -– 629..

\bibitem{SharSMJ2017}
И.И.Шарапудинов,"Специальные ряды по полиномам Лагерра и их аппроксимативные свойства", Сиб. матем. журн., 58:2 (2017), 440–467.

\bibitem{Dzjadyc}
В.К. Дзядык,"", Введение в теорию равномерного приближения функций полиномами,  (1977), .

\bibitem{KashSaak}
Б.\,С.~Кашин, А.\,А.~Саакян,Ортогональные ряды. АФЦ, Москва, 1999,  с..

\bibitem{SharDag2016}
И.\,И. Шарапудинов,"Асимптотические свойства полиномов, ортогональных по Соболеву, порожденных полиномами Якоби", Дагестанские электронные математические известия, 6 (2016), 1 -- 24.

\bibitem{Sege}
Г. Сеге,"", Ортогональные многочлены,  (1962), .

\bibitem{TEL}
С.\, А. Теляковский,"Две теоремы о приближении функций алгебраическими многочленами", Математический сборник, 70:2 (1966), 252 -- 265.

\bibitem{GOP}
И.\, З. Гопенгауз,"К теореме А. Ф. Тимана о приближении функций многочленами на", Математические  заметки, 1:2 (1967), 163 -- 172.

\bibitem{OSK}
К.\, И. Осколков,"К неравенству Лебега в равномерной метрике и на множестве полной меры", Математические  заметки, 18:4 (1975), 515 -- 526.

\bibitem{sharap1}
I.\,I. Sharapudinov,"On the best approximation and polinomial of the least quadratic deviation", Analysis Mathematica, 9:3 (1983), 223 -- 234.

\bibitem{sharap2}
И.\,И. Шарапудинов,"О наилучшем приближении и суммах Фурье-Якоби", Математические заметки, 34:5 (1983), 651 -- 661.

\bibitem{Timan}
А.Ф. Тиман,"", Теория приближения функций действительного переменного,  (1960), .

\bibitem{Gasper}
G. Gasper,"Positiviti and special function", Theory and appl.Spec.Funct. Edited by Richard A.Askey.,  (1975), 375 -- 433.

\bibitem{Tref1}
L.N. Trefethen,Spectral methods in Matlab. SIAM, Fhiladelphia, 2000,  с..

\bibitem{Tref2}
L.N. Trefethen,Finite difference and spectral methods for ordinary and partial differential equation. Cornell University, , 1996,  с..

\bibitem{SolDmEg}
В.В. Солодовников, А.Н. Дмитриев, Н.Д. Егупов,Спектральные методы расчета и проектирования систем управления. Машиностроение, Москва, 1986,  с..

\bibitem{Pash}
С. Пашковский,"", Вычислительные применения многочленов и рядов Чебышева,  (1983), .

\bibitem{AskeyWaiger}
R. Askey, S. Wainger,"Mean convergence of expansions in Laguerre and Hermite series", Amer. J. Mathem., 87 (1965), 698 -- 708.

%%%==========report 2017========
\bibitem{rep2017-sobleg-Shar11}
{Шарапудинов И.И.} Приближение функций с переменной гладкостью суммами Фурье Лежандра // Мат. сборник,
191(5), 2000. С. 143--160.


\bibitem{rep2017-sobleg-Shar12}
{Шарапудинов И.И.} Аппроксимативные свойства операторов $\mathcal{ Y}_{n+2r}(f)$ и их дискретных аналогов // Мат. заметки, 72(5), 2002. С.765--795.


\bibitem{rep2017-sobleg-Shar13}
{Шарапудинов И.И.} Смешанные ряды по ортогональным полиномам. Издательство Дагестанского научного центра.
Махачкала 2004. С.1--176.


\bibitem{rep2017-sobleg-Shar15}
{Шарапудинов И.И.}
Аппроксимативные свойства смешанных рядов по полиномам Лежандра на классах $W^r$ //
Мат. сборник, 97(3), 2006. С. 135--154.


\bibitem{rep2017-sobleg-Shar16}
{Шарапудинов И.И.}
Аппроксимативные свойства средних типа Валле-Пуссена частичных сумм смешанных рядов по полиномам Лежандра // Мат. заметки, 84(3), 2008. С.452--471.


\bibitem{rep2017-sobleg-Shar17}
{Шарапудинов И.И.}
Смешанные ряды по ультрасферическим полиномам и их аппроксимативные свойства
// Мат. сборник, 194(3), 2003. С. 115--148.


\bibitem{rep2017-sobleg-Shar18}
{Шарапудинов И.И., Шарапудинов Т.И.}
Смешанные ряды по полиномам Якоби и Чебышева и их дискретизация
// Мат. заметки, 88(1), 2010. С. 116--147.


\bibitem{rep2017-sobleg-sharap3}
{Шарапудинов И.И.}
Некоторые специальные ряды по ультрасферическим полиномам и их аппроксимативные свойства
// Изв. РАН. Сер. матем. 78(5), 2014. С. 201--224.


\bibitem{rep2017-sobleg-SHII}
{Шарапудинов И.И.}
Некоторые специальные ряды по общим полиномам Лагерра и ряды Фурье по полиномам Лагерра, ортогональным по Соболеву
// Дагестанские электронные математические известия. 2015. Вып. 4.


\bibitem{rep2017-sobleg-Sege}
{Сеге Г.} Ортогональные многочлены. Физматгиз. Москва. 1962.


\bibitem{rep2017-sobleg-Gasper}
{Gasper G.}
Positiviti and special function
// Theory and appl.Spec.Funct. Edited by Richard A.Askey. 1975. Pp. 375--433.


\bibitem{rep2017-sobleg-KwonLittl1}
{Kwon K.H., Littlejohn L.L.}
The orthogonality of the Laguerre polynomials $\{L_n^{(-k)}(x)\}$ for positive integers $k$
// Ann. Numer. Anal. Iss. 2. 1995. Pp. 289--303.


\bibitem{rep2017-sobleg-KwonLittl2}
{Kwon K.H., Littlejohn L.L.}
Sobolev orthogonal polynomials and second-order differential equations
// Rocky Mountain J. Math. Vol. 28. 1998. Pp. 547--594.


\bibitem{rep2017-sobleg-MarcelAlfaroRezola}
{Marcellan F. , Alfaro M., Rezola M.L.} Orthogonal polynomials on Sobolev spaces: old and new directions
// Journal of Computational and Applied Mathematics. Vol. 48. 1993. Pp. 113--131.


\bibitem{rep2017-sobleg-IserKoch}
{ Iserles A., Koch P.E., Norsett S.P., Sanz-Serna J.M.}
On polynomials  orthogonal  with respect  to certain Sobolev inner products
// J. Approx. Theory, 65. 1991. Pp. 151--175.


\bibitem{rep2017-sobleg-Meijer}
{Meijer H.G.} Laguerre polynimials generalized to a certain.
Laguerre polynimials generalized to a certain discrete Sobolev inner product space
// J. Approx. Theory, 73. 1993. Pp. 1--16.


\bibitem{rep2017-sobleg-Lopez1995}
{Lopez G. Marcellan F. Vanassche W.}
Relative Asymptotics for Polynomials Orthogonal with Respect to a Discrete Sobolev Inner-Product
// Constr. Approx. 11:1. 1995. Pp. 107--137.


\bibitem{rep2017-sobleg-MarcelXu}
{Marcellan F., Xu Y.}
On Sobolev orthogonal polynomials
// Expositiones Mathematicae, 33(3). 2015. Pp. 308--352.


\bibitem{rep2017-sobleg-Shar2016}
И.И. Шарапудинов
Системы функций, ортогональные по Соболеву, порожденные ортогональными функциями
// Материалы 18-й международной Саратовской зимней школы «Современные проблемы теории функций и их приложения». 2016. С. 329--332.


\bibitem{rep2017-sobleg-Tref1}
{Trefethen  L.N.} Spectral methods in Matlab. Fhiladelphia. SIAM. 2000.


\bibitem{rep2017-sobleg-Tref2}
{Trefethen  L.N.}
Finite difference and spectral methods for ordinary and partial differential equation. Cornell University. 1996.


\bibitem{rep2017-sobleg-SolDmEg}
{Солодовников В.В., Дмитриев А.Н., Егупов Н.Д.}
Спектральные методы расчета и проектирования систем управления. Машиностроение. Москва. 1986.


\bibitem{rep2017-sobleg-Pash}
{Пашковский С.} Вычислительные применения многочленов и рядов Чебышева. Наука. Москва. 1983. С. 143--160.


\bibitem{rep2017-sobleg-MMG2016}
{Магомед-Касумов М.Г.}
Приближенное решение обыкновенных дифференциальных уравнений с использованием смешанных рядов по системе Хаара
// Материалы 18-й международной Саратовской зимней школы «Современные проблемы теории функций и их приложения». 2016. С. 176--178.


\bibitem{rep2017-sobleg-Gonchar1975}
{Гончар А.А.}
О сходимости аппроксимаций Паде для некоторых классов мероморфных функций
// Мат. сборник, 97(139):4(8), 1975. С. 607--629.


\bibitem{rep2017-sobleg-TEL}
{Теляковский С.А.}
Две теоремы о приближении функций алгебраическими многочленами
// Мат. сборник, 70(2), 1966. С.252--265.


\bibitem{rep2017-sobleg-GOP}
{Гопенгауз И.З.}
К теореме А. Ф. Тимана о приближении функций многочленами на конечном отрезке
// Мат. заметки, 1(2), 1967. С. 163--172.


\bibitem{rep2017-sobleg-OSK}
{Осколков К.И.}
К неравенству Лебега в равномерной метрике и на множестве полной меры
// Мат.  заметки, 18(4), 1975. С. 515--526.


\bibitem{rep2017-sobleg-sharap1}
{Sharapudinov I.I.}
On the best approximation and polinomial of the least quadratic deviation
// Analysis Mathematica, 9(3), 1983. Pp. 223--234.


\bibitem{rep2017-sobleg-sharap2}
{Шарапудинов И.И.}
О наилучшем приближении и суммах Фурье-Якоби
//Мат. заметки, 34(5), 1983. С. 651--661.


\bibitem{rep2017-sobleg-Timan}
{Тиман А.Ф.} Теория приближения функций действительного переменного. Физматгиз, Москва. 1960.
%%
%% end lit. section2-sobleg


\bibitem{rep2017-laplas-Shar13}
{Шарапудинов И.И.}
Смешанные ряды по ортогональным полиномам // Издательство Дагестанского научного центра. Махачкала. 2004. Стр. 1--176.


\bibitem{rep2017-laplas-Shar14}
{Шарапудинов И.И.}
Смешанные ряды по полиномам Чебышева, ортогональным на равномерной сетке // Математические заметки. 2005. Т. 78. Вып. 3. Стр. 442–-465.


\bibitem{rep2017-laplas-Shar11}
{Шарапудинов И.И.}
Специальные ряды по полиномам Лагерра и их аппроксимативные свойства // Сибирский математический журнал. 2017. Т. 58. Вып. 2. Стр. 440--467.


\bibitem{rep2017-laplas-Meijer}
{Meijer H.G.}
Laguerre polynimials generalized to a certain discrete Sobolev inner product space // J. Approx. Theory. 1993. Vol. 73. Pp. 1--16.


\bibitem{rep2017-laplas-MarcelXu}
{Marcellan F., Yuan Xu}
ON SOBOLEV ORTHOGONAL POLYNOMIALS. arXiv: 6249v1 [math.C.A] 25 Mar 2014. Pp. 1--40


\bibitem{rep2017-laplas-MarcelVanash}
{Lopez G., Marcellan F., Van Assche W.}
Relative asymptotics for polynomials orthogonal with respect to a discrete Sobolev inner product // Constr. Approx. 1995. Vol. 11. Issue 1. Pp. 107--137.


\bibitem{rep2017-laplas-Sege}
{Сеге Г.}
Ортогональные многочлены. Москва. Физматгиз. 1962.


\bibitem{rep2017-laplas-AskeyWaiger}
{Askey R., Wainger S.}
Mean convergence of expansions in Laguerre and Hermite series // Amer. J. Mathem. 1965. Vol. 87. Pp. 698--708.


\bibitem{rep2017-laplas-DitPrud}
{Диткин В.А., Прудников А.П.}
Операционное исчисление. Москва. Высшая школа. 1975.


\bibitem{rep2017-laplas-KrylovSkob}
{Крылов В.И., Скобля Н.С.}
Методы приближенного преобразования Фурье и обращения преобразования Лапласа. Москва. Наука. 1974.
%%
%% end lit. section2-laplas


\bibitem{rep2017-equ102-Shar20}
Шарапудинов И.И. Ортогональные  по Соболеву системы, порожденные ортогональными функциями // Изв. РАН. Сер. Математическая. 2018. Том. 82. (Принята к печати)


\bibitem{rep2017-equ102-Tref1}
{Trefethen  L.N.}
Spectral methods in Matlab. Fhiladelphia. SIAM. 2000.


\bibitem{rep2017-equ102-Arush2014}
{Арушанян О.Б., Волченскова Н.И., Залеткин С.Ф.}
Применение рядов Чебышева для интегрирования обыкновенных дифференциальных уравнений // Сиб. электрон. матем. изв. 2014. Вып. 11. Стр. 517--531.


\bibitem{rep2017-equ102-Lukom2016}
{Лукомский Д.С., Терехин П.А.}
Применение системы Хаара к численному решению задачи Коши для линейного дифференциального уравнения первого порядка // Материалы 18-й международной Саратовской зимней школы «Современные проблемы теории функций и их приложения». Саратов. ООО «Издательство «Научная книга». 2016. Стр. 171--173.


\bibitem{rep2017-equ102-DiffUr2017}
{Шарапудинов И.И., Магомед-Касумов М.Г.}
О представлении решения задачи Коши  рядом Фурье  по полиномам, ортогональным по  Соболеву, порожденным многочленами Лагерра. Дифференциальные уравнения. 2017 (принята к печати)


\bibitem{rep2017-equ102-KashSaak}
{Кашин Б.С., Саакян А.А.}
Ортогональные ряды. Москва. АФЦ 1999.


\bibitem{rep2017-equ102-Shar19}
{Шарапудинов И.И., Муратова Г.Н.}
Некоторые свойства r-кратно интегрированных рядов по системе Хаара // Изв. Сарат. ун-та. Нов. сер. Сер. Математика. Механика. Информатика. 2009. Т. 9. Вып. 1. Стр. 68 -- 76


\bibitem{rep2017-equ102-Faber}
{G. Faber}
Ober die Orthogonalfunktionen des Herrn Haar // Jahresber. Deutsch. Math. Verein. 1910. Vol. 19. Pp. 104--112.


\bibitem{rep2017-equ102-Shar25}
{Шарапудинов И.И.}
Асимптотические свойства полиномов, ортогональных по Соболеву, порожденных полиномами Якоби // Дагестанские электронные математические известия. 2016. Вып. 6.	Стр. 1–-24.


\bibitem{rep2017-equ130-Shar13}
{Шарапудинов И.И.}
Смешанные ряды по ортогональным полиномам. Издательство Дагестанского научного центра. Махачкала. 2004. С. 1--176.


\bibitem{rep2017-equ130-Tref1}
{Trefethen L.N.}
Spectral methods in Matlab. SIAM. Philadelphia. 2000.


\bibitem{rep2017-equ130-DiffUr2017}
{Шарапудинов И.И., Магомед-Касумов М.Г.}
О представлении решения задачи Коши  рядом Фурье  по полиномам, ортогональным по  Соболеву, порожденным многочленами Лагерра // Дифференциальные уравнения. 2017. (принята к печати)








\bibitem{rep2017-equ130-Sege}
Сеге Г. Ортогональные многочлены. Москва. Физматгиз. 1962.
%%
%% end lit. section2-equ130


\bibitem{rep2017-sobcheb_urav-Arush2010}
{Арушанян О.Б., Волченскова Н.И., Залеткин С.Ф.}
Приближенное решение обыкновенных дифференциальных уравнений с использованием рядов Чебышева // Сиб. электрон. матем. 1983. изв. Вып. 7. Стр. 122–-131


\bibitem{rep2017-sobcheb_urav-Arush2013}
{Арушанян О.Б., Волченскова Н.И., Залеткин С.Ф.}
Метод решения задачи Коши для обыкновенных дифференциальных уравнений с использованием рядов Чебышeва // Выч. мет. программирование. 2013. Вып. 14:2. Стр. 203-214.


\bibitem{rep2017-sobcheb_urav-fiht2}
{Фихтенгольц Г.М.}
Курс дифференциального и интегрального исчисления // Физматлит. Москва. 2001. Т. 2. Стр. 810.


\bibitem{rep2017-ramis-Ram1}
{Шарапудинов~И.И.} Многочлены, ортогональные на сетках. Махачкала, Изд-во Даг. гос. пед. ун-та. 1997.		


\bibitem{rep2017-ramis-shGadj}
{Шарапудинов И.И., Гаджиева З.Д.}
Полиномы, ортогональные по Соболеву, порожденные многочленами Мейкснера // Изв. Сарат. ун-та. Нов. сер. Сер. Математика. Механика. Информатика,
2016. Т.16. Вып. 3. С. 310--321.


\bibitem{rep2017-ramis-shGadjGadjMir}
{Шарапудинов И.И., Гаджиева З.Д., Гаджимирзаев Р.М.}
Разностные уравнения и полиномы, ортогональные по Соболеву, порожденные многочленами Мейкснера //
Владикавказский Мат. журнал, 2017. Т.19. Вып. 2. С. 58--72.


\bibitem{rep2017-ramis-Shar9}
{Шарапудинов~И.И.}
Приближение дискретных функций и многочлены Чебышева, ортогональные на равномерной сетке //
Мат. заметки, 2000. Т. 67. Вып. 3. С. 460--470.


\bibitem{rep2017-ramis-SharT1}
{Шарапудинов~Т.И.}
Аппроксимативные свойства смешанных рядов по полиномам Чебышева, ортогональным на равномерной сетке //
Вестник Дагестанского научного центра РАН, 2007. Т. 29. С. 12-–23.


\bibitem{rep2017-ramis-SharII}
{Шарапудинов~И.И.}
Системы функций, ортогональных по Соболеву, порожденные ортогональными функциями //
Современные проблемы теории функций и их приложения.  Материалы 18-й международной Саратовской зимней школы. 2016. С. 329--332.


\bibitem{rep2017-ramis-Gadz12}
{Fernandez~L., Teresa E. Perez, Miguel A. Pinar, Xu~Y.} Weighted Sobolev orthogonal polynomials on the unit ball~// Journal of Approximation Theory, 171, 2013, pp.~84--104.


\bibitem{rep2017-ramis-Gadz13}
{Antonia~M. Delgado, Fernandez~L., Doron~S. Lubinsky, Teresa~E. Perez, Miguel~A. Pinar.} Sobolev orthogonal polynomials on the unit ball via outward normal derivatives~// Journal of Mathematical Analysis and Applications, 440, №~2, 2016, pp.~716--740.


\bibitem{rep2017-ramis-Gadz14}
{Fernandez~L., Marcellan~F., Teresa~E. Perez, Miguel~A. Pinar, Xu~Y.} Sobolev orthogonal polynomials on product domains~// Journal of Computational and Applied Mathematics, 284, 2015, pp.~202--215.


\bibitem{rep2017-ramis-Gadz16}
{Шарапудинов~И.~И., Шарапудинов~Т.~И.} Полиномы, ортогональные по Соболеву, порожденные многочленами Чебышева, ортогональными на сетке~// Изв. вузов. Матем., 2017, №~8, 67--79.


\bibitem{rep2017-ramis-Gadz17}
{Гаджимирзаев~Р.~М.} Ряды Фурье по полиномам Мейкснера, ортогональным по Соболеву~// Изв. Сарат. ун-та. Нов. сер. Сер. Математика. Механика. Информатика, 16:4 (2016), 388--395.


\bibitem{rep2017-ramis-Gadz1}
{Шарапудинов~И.~И., Гаджиева~З.~Д., Гаджимирзаев~Р.~М.} Системы функций, ортогональных относительно скалярных произведений типа Соболева с дискретными массами, порожденных классическими ортогональными системами~// Дагестанские электронные математические известия. 2016. Вып.~6. С.~31--60.


\bibitem{rep2017-ramis-Gadz3}
{Сеге~Г.} Ортогональные многочлены. М.: Физматгиз, 1962.
%%
%% end lit. ramis

\bibitem{rep2017-charlier-Shar3}
Meijer~H.~G. Laguerre polynomials generalized to a certain discrete Sobolev inner product space~// J. Approx. Theory. 1993. Vol.~73. Iss.~1. Pp.~1--16.

\bibitem{rep2017-charlier-Shar8}
Шарапудинов~И.~И. Смешанные ряды по ортогональным полиномам. Махачкалаю. Изд-во ДНЦ РАН. 2004.

\bibitem{rep2017-charlier-Shar10}
Бейтмен~Г., Эрдейи~А. Высшие трансцендентные функции. Том 2. Москва. Наука. 1974.

\bibitem{rep2017-charlier-Shar11}
Ширяев~А.~Н. Вероятность-1. Москва. Изд-во МЦНМО. 2007.

%============diffur2018-cheb
\bibitem{du2018cheb-KwonLittl1}
K.\,H. Kwon and L.\,L. Littlejohn,"The orthogonality of the Laguerre polynomials $\{L_n^{(-k)}(x)\}$ for positive integers $k$", Ann. Numer. Anal., 2 (1995), 289--303.

\bibitem{du2018cheb- KwonLittl2}
K.\,H. Kwon and L.\,L. Littlejohn,"Sobolev orthogonal polynomials and second-order differential equations", Rocky Mountain J. Math., 28 (1998), 547–-594.

\bibitem{du2018cheb-MarcelAlfaroRezola }
F. Marcellan, M. Alfaro and M.L. Rezola,"Orthogonal polynomials on Sobolev spaces: old and new directions", Journal of Computational and Applied Mathematics, 48 (1993), 113--131.

\bibitem{du2018cheb-IserKoch }
A. Iserles, P.E. Koch, S.P. Norsett and J.M. Sanz-Serna,"On polynomials  orthogonal  with respect  to certain Sobolev inner products", J. Approx. Theory, 65 (1991), 151--175.

\bibitem{du2018cheb-Meijer}
H.\,G. Meijer,"Laguerre polynomials generalized to a certain discrete Sobolev inner product space", J. Approx. Theory, 73 (1993), 1--16.

\bibitem{du2018cheb-Lopez1995}
Lopez G. Marcellan F. Vanassche W.,"Relative Asymptotics for Polynomials Orthogonal with Respect to a Discrete Sobolev Inner-Product", Constr. Approx., 111 (1995), 107–-137.

\bibitem{du2018cheb-MarcelXu}
F. Marcellan and Yuan Xu,"On Sobolev orthogonal polynomials", Expositiones Mathematicae, 33:3 (2015), 308--352.

\bibitem{du2018cheb-SHII}
И.\,И. Шарапудинов,"Некоторые специальные ряды по общим полиномам Лагерра и ряды Фурье по полиномам Лагерра, ортогональным по Соболеву", Дагестанские электронные математические известия, 4 (2015), 31 -- 73.

\bibitem{du2018cheb-SharSMJ2017}
И.И.Шарапудинов,"Специальные ряды по полиномам Лагерра и их аппроксимативные свойства", Сиб. матем. журн., 58:2 (2017), 440–467.

\bibitem{du2018cheb-SharIzv2018}
И.И.Шарапудинов,"Системы функций, ортогональные по Соболеву, ассоциированные с ортогональной системой", Изв. РАН. Сер. матем., 82:1 (2018), 225--258.

\bibitem{du2018cheb-SHIIDiff2018}
И.\,И. Шарапудинов, M.-P.\,Г. Магомед-Касумов,"О представлении решения задачи Коши  рядом Фурье  по полиномам, ортогональным по  Соболеву, порожденным многочленами Лагерра", Дифференциальныеуравнения, 54:1 (2018), 51 -- 68.

\bibitem{du2018cheb-Tref1}
L.N. Trefethen,Spectral methods in Matlab. SIAM, Fhiladelphia, 2000,  с..

\bibitem{du2018cheb-Tref2}
L.N. Trefethen,Finite difference and spectral methods for ordinary and partial differential equations. Cornell University, , 1996,  с..

\bibitem{du2018cheb-SolDmEg}
В.В. Солодовников, А.Н. Дмитриев, Н.Д. Егупов,Спектральные методы расчета и проектирования систем управления. Машиностроение, Москва, 1986,  с..

\bibitem{du2018cheb-Pash}
С. Пашковский,"", Вычислительные применения многочленов и рядов Чебышева,  (1983), .

\bibitem{du2018cheb-Arush2010}
О. Б. Арушанян, Н. И. Волченскова, С. Ф. Залеткин,"Приближенное решение обыкновенных дифференциальных уравнений с использованием рядов Чебышева", Сиб. электрон. матем. изв., 7 (1983), 122–131.

\bibitem{du2018cheb-Arush2013}
О. Б. Арушанян, Н. И. Волченскова, С. Ф. Залеткин,"Метод решения задачи Коши для обыкновенных дифференциальных уравнений с использованием рядов Чебышeва", Выч. мет. программирование, 142 (2013), 203--214.

\bibitem{du2018cheb-Arush2014}
О. Б. Арушанян, Н. И. Волченскова, С. Ф. Залеткин,"Применение рядов Чебышева для интегрирования обыкновенных дифференциальных уравнений", Сиб. электрон. матем. изв., 11 (2014), 517--531.

\bibitem{du2018cheb-Lukom2016}
Д.С. Лукомский, П.А. Терехин,"Применение системы Хаара к численному решению задачи Коши для линейного дифференциального уравнения первого порядка", Материалы 18-й международной Саратовской зимней школы «Современные проблемы теории функций и их приложения»,  (2016), 171--173.

\bibitem{du2018cheb-MMG2016}
М.Г. Магомед-Касумов,"Приближенное решение обыкновенных дифференциальных уравнений с использованием смешанных рядов по системе Хаара", Материалы 18-й международной Саратовской зимней школы «Современные проблемы теории функций и их приложения»,  (2016), 176--178.

\bibitem{du2018cheb-SHII-MMG2018}
И.И. Шарапудинов, М.Г. Магомед-Касумов,"Численный метод решения задачи Коши для систем обыкновенных дифференциальных уравнений с помощью ортогональной в смысле Соболева системы, порожденной системой косинусов", Дагестанские Электронные Математические Известия, 8 (2017), 53 -- 60.

\bibitem{du2018cheb-Shar11}
И.\,И. Шарапудинов,"Приближение функций с переменной гладкостью суммами Фурье Лежандра", Математический сборник, 191:5 (2000), 143--160.

\bibitem{du2018cheb-Shar12}
И.\,И. Шарапудинов,"Аппроксимативные свойства операторов $\mathcal{Y}_{n+2r}(f)$ и их дискретных аналогов", Математические заметки, 72:5 (2002), 765--795.

\bibitem{du2018cheb-Shar13}
И.\,И. Шарапудинов,"Смешанные ряды по ортогональным полиномам", ,  (2004), 1--176.

\bibitem{du2018cheb-Shar14}
И.\,И. Шарапудинов,"Смешанные ряды по полиномам Чебышева, ортогональным на равномерной сетке", Математические заметки, 78:3 (2005), 442--465.

\bibitem{du2018cheb-Shar15}
И.\,И. Шарапудинов,"Аппроксимативные свойства смешанных рядов по полиномам Лежандра на классах $W^r$", Математический сборник, 197:3 (2006), 135–154.

\bibitem{du2018cheb-Shar16}
И.\,И. Шарапудинов,"Аппроксимативные свойства средних типа Валле-Пуссена частичных сумм смешанных рядов по полиномам Лежандра", Математические заметки, 84:3 (2008), 452--471.

\bibitem{du2018cheb-Shar17}
И.\,И. Шарапудинов,"Смешанные ряды по ультрасферическим полиномам и их аппроксимативные свойства", Математический сборник, 194:3 (2003), 115--148.

\bibitem{du2018cheb-Shar18}
И.\,И. Шарапудинов, Т.\, И. Шарапудинов,"Смешанные ряды по полиномам Якоби и Чебышева и их дискретизация", Математические заметки, 88:1 (2010), 116--147.

\bibitem{du2018cheb-Sege}
Г. Сеге,"", Ортогональные многочлены,  (1962), .

\bibitem{du2018cheb-Timan}
А.Ф. Тиман,"", Теория приближения функций действительного переменного,  (1960), .

\bibitem{du2018cheb-Ahiezer}
Н.И. Ахиезер,"", Лекции по теории аппроксимации,  (1965), .

%=========Ramis
\bibitem{Ramlib1} {Askey R., Wainger S.} Mean convergence of expansions in Laguerre and Hermite series // Amer. J. Math., vol. 87, 1965, pp.~698--708.

\bibitem{Ramlib3} {Bateman H, Erdeyi A.} Higher transcendental functions. Vol. 2. McGraw-Hill, New York-Toronto-London, 1953.

\bibitem{Ramlib5} {Фихтенгольц Г.М.} Курс дифференциального и интегрального исчисления. Том. 2 Москва: Физматлит, 2001.

\bibitem{Ramlib4} {Шарапудинов И.И.} Системы функций, ортогональные по Соболеву, ассоциированные с ортогональной системой // Изв. РАН. Сер. матем., 82:1, 2018. С. 225--258.

\bibitem{Ramlib41} {Шарапудинов И.И., Шарапудинов Т.И.} Полиномы, ортогональные по Соболеву, порожденные многочленами Чебышева, ортогональными на сетке // Изв. вузов. Матем., № 8, 2017. С. 67--79.

\bibitem{Ramlib42} {Шарапудинов И.И.} Аппроксимативные свойства рядов Фурье по многочленам, ортогональным по Соболеву с весом Якоби и дискретными массами // Матем. заметки, 101:4, 2017. С. 611--629.

\bibitem{Ramlib43} {Шарапудинов И.И., Гаджиева З.Д., Гаджимирзаев Р.М.} Системы функций, ортогональных относительно скалярных произведений типа Соболева с дискретными массами, порожденных классическими ортогональными системами // Дагестанские электронные математические известия, вып. 6, 2016. С. 31--60. 

\bibitem{Ramlib44} {Шарапудинов И.И., Гаджиева З.Д., Гаджимирзаев Р.М.} Разностные уравнения и полиномы, ортогональные по Соболеву, порожденные многочленами Мейкснера // Владикавк. матем. журн., 19:2, 2017. С. 58--72.

\bibitem{Ramlib2} {Сеге Г.} Ортогональные многочлены. Физматгиз. Москва. 1962.

%==============walsh
\bibitem{walsh-GolubovBook}
Б.\,И. Голубов, А.\,В. Ефимов, В.\,А. Скворцов,Ряды и преобразования Уолша: Теория и применения. Наука. Гл. ред. физ.-мат. лит, Москва, 1987, 344 с.

\bibitem{walsh-ShII-2018-IzvRan}
И.\,И. Шарапудинов,"Системы функций, ортогональные по Соболеву, ассоциированные с ортогональной системой", Изв. РАН. Сер. матем., 82:1 (2018), 225--258.

\bibitem{walsh-ShII-2017-mz}
И.\,И. Шарапудинов,"Аппроксимативные свойства рядов Фурье по многочленам, ортогональным по Соболеву с весом Якоби и дискретными массами", Матем. заметки, 101:4 (2017), 611--629.

\bibitem{walsh-ShII-2017-smj}
И.~И.~Шарапудинов,"Специальные ряды по полиномам Лагерра и~их аппроксимативные свойства", Сиб. матем. журн., 58:2 (2017), 440--467.

\bibitem{walsh-ShII-2015-demi}
И.~И.~Шарапудинов, М.~Г.~Магомед-Касумов, С.~Р.~Магомедов,"Полиномы, ортогональные по Соболеву, ассоциированные с полиномами Чебышева первого рода", Дагестанские электронные математические известия, 4 (2015), 1--14.

\bibitem{walsh-ShII-2017-izvuz}
И.~И.~Шарапудинов, Т.~И.~Шарапудинов,"Полиномы, ортогональные по Соболеву, порожденные многочленами Чебышева, ортогональными на сетке", Изв. вузов. Матем., 61:8 (2017), 67--79.

\bibitem{walsh-ShII-meix-2016}
И.~И.~Шарапудинов, З.~Д.~Гаджиева,"Полиномы, ортогональные по Соболеву, порожденные многочленами Мейкснера", Изв. Сарат. ун-та. Нов. сер. Сер. Математика. Механика. Информатика, 16:3 (2016), 310--321.

\bibitem{walsh-Gadzh-2016-saratov}
Р.~М.~Гаджимирзаев,"Ряды Фурье по полиномам Мейкснера, ортогональным по Соболеву", Изв. Сарат. ун-та. Нов. сер. Сер. Математика. Механика. Информатика, 16:4 (2016), 388--395.

\bibitem{walsh-fiht2}
Г.\,М. Фихтенгольц,Курс дифференциального и интегрального исчисления. Физматлит, Москва, 2001, 810 с.



\end{thebibliography}