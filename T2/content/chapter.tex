\chapter{Ортонормированные по Соболеву системы функций }
\section{Определение и основные свойства}
Система Хаара $\chi=\{\chi_k(x)\}_{k=1}^\infty$, ортонормированная на отрезке $[0,1]$, порождает систему Фабера -- Шаудера $\phi=\{\phi_k(x)\}_{k=0}^\infty$ посредством равенств \eqref{1.1}, которая, как отмечалось, является ортонормированной по Соболеву относительно скалярного произведения \eqref{1.2}. Если вместо системы Хаара взять произвольную ортонормированную
с весом $\rho(x)$ на $(a,b)$  систему $\{\varphi_k(x)\}_{k=0}^\infty$, состоящую из интегрируемых функций $\varphi_k(x)$, то, как показано в рамках данной НИР (см. \cite{SharIzv2018}),  можно для каждого натурального $r$ сконструировать новую систему функций $\{\varphi_{r,k}(x)\}_{k=0}^\infty$ посредством равенств
\begin{equation}\label{2.1}
\varphi_{r,r+k}(x) =\frac{1}{(r-1)!}\int_a^x(x-t)^{r-1}\varphi_{k}(t)dt, \quad k=0,1,\ldots,
\end{equation}

\begin{equation}\label{2.2}
\varphi_{r,k}(x) =\frac{(x-a)^k}{k!}, \quad k=0,1,\ldots, r-1,
\end{equation}
которая является   ортонормированной по Соболеву относительно скалярного произведения \eqref{1.8}. Чтобы сформулировать  этот результат более точно, нам понадобятся некоторые обозначения. Для заданной на $(a,b)$ почти всюду положительной интегрируемой  (весовой) функции $\rho$ обозначим через $L^p_\rho(a,b)$  пространство  функций $f(x)$, измеримых  на  $(a,b)$, для которых конечен интеграл $\int_a^b|f(x)|^p\rho(x)dx$. Если $\rho(x)\equiv1$, то будем писать $L^p_\rho(a,b)=L^p(a,b)$ и $L(a,b)=L^1(a,b)$. Нетрудно показать, что если весовая функция $\rho(x)$ удовлетворяет некоторым (естественным) условиям, то пространство $L^p_\rho(a,b)$ представляет собой банахово пространство с нормой $\|f\|_{p,\rho}=(\int_a^b|f(x)|^p\rho(x)dx)^\frac1p$. Отметим, что если $1/\rho(x)\in L(a,b)$, то из  неравенства Гельдера вытекает, что $L^2_\rho(a,b)\subset L(a,b)$. Через $W^r_{L^p_\rho(a,b)}$ обозначим пространство Соболева, которое состоит из функций $f$, $r-1$-раз непрерывно дифференцируемых на $[a,b]$, причем $f^{(r-1)}$ абсолютно непрерывна и  $f^{(r)}\in L^p_\rho(a,b)$. Если система функций $\{\varphi_k(x)\}_{k=0}^\infty$ ортонормирована на $(a,b)$ с весом $\rho$, то $\varphi_k(x)\in L^2_\rho(a,b)$ $(k=0,1,\ldots)$, но мы добавим к этому условию еще одно, считая, что $\varphi_k(x)\in L(a,b)$ $(k=0,1,\ldots)$. Тогда мы можем определить новую систему функций $\varphi^r=\{\varphi_{r,k}(x)\}_{k=0}^\infty$ с помощью равенств \eqref{2.1} и \eqref{2.2}, которая ортонормирована относительно скалярного произведения \eqref{1.8}, т. е.
\begin{equation}\label{2.3}
\langle\varphi_{r,k},\varphi_{r,n}\rangle=\sum\nolimits_{\nu=0}^{r-1}\varphi_{r,k}^{(\nu)}(a)\varphi_{r,n}^{(\nu)}(a)+
\int_{a}^{b}\varphi_{r,k}^{(r)}(x)\varphi_{r,n}^{(r)}(x)\rho(x)dx=\delta_{k,n}.
\end{equation}
Справедливость этого утверждения вытекает из следующей теоремы,   установленной в \cite{SharIzv2018}:
\begin{theorem}\label{th1} Предположим, что    функции $\varphi_k(x)$ $(k=0,1,\ldots)$ образуют полную в $L^2_\rho(a,b)$ ортонормированную   c весом   $\rho(x)$ систему на отрезке $[a,b]$. Тогда система $\{\varphi_{r,k}(x)\}_{k=0}^\infty$, порожденная системой $\{\varphi_{k}(x)\}_{k=0}^\infty$ посредством равенств \eqref{2.1} и \eqref{2.2}, полна  в $W^r_{L^2_\rho(a,b)}$, и справедливы равенства \eqref{2.3}.
\end{theorem}

Ряд Фурье функции $f\in W^r_{L^2_\rho(a,b)}$ можно записать в следующем смешанном виде
\begin{equation}\label{2.4}
f(x)\sim \sum\nolimits_{k=0}^{r-1} f^{(k)}(a)\frac{(x-a)^k}{k!}+ \sum\nolimits_{k=r}^\infty \hat f_{r,k}\varphi_{r,k}(x),
\end{equation}
где
\begin{equation}\label{2.5}
\hat f_{r,k}=\int_a^b f^{(r)}(t) \varphi^{(r)}_{r,k}(t)\rho(t)dt=\int_a^b f^{(r)}(t) \varphi_{k-r}(t)\rho(t)dt,
\end{equation}
поэтому ряд  вида \eqref{2.4} будем  называть \textit{смешанным рядом} по  системе $\{\varphi_{k}(x)\}_{k=0}^\infty$, считая это название условным и сокращенным обозначением полного названия: <<\textit{ряд Фурье по системе  $\{\varphi_{r,k}(x)\}_{k=0}^\infty$, ортонормированной по Соболеву, порожденной ортонормированной системой $\{\varphi_{k}(x)\}_{k=0}^\infty$}>>.


Если от весовой функции $\rho$ потребовать дополнительно условие $1/\rho(x)\in L(a,b)$, то   справедлива \cite{SharIzv2018}
следующая
\begin{theorem}\label{th2} Предположим, что  $ \frac{1}{\rho(x)}\in L(a,b) $, а  функции $\varphi_k(x)$ $(k=0,1,\ldots)$  образуют полную в $L^2_\rho(a,b)$ ортонормированную   c весом   $\rho(x)$ систему на отрезке $[a,b]$, $\{\varphi_{r,k}(x)\}_{k=0}^\infty$ -- система, ортонормированная в $W^r_{L^2_\rho(a,b)}$ относительно скалярного произведения \eqref{1.8},  порожденная системой $\{\varphi_{k}(x)\}_{k=0}^\infty$ посредством равенств \eqref{2.1} и \eqref{2.2}. Тогда если $f(x)\in W^r_{L^2_\rho(a,b)}$, то ряд Фурье (смешанный ряд) \eqref{2.4} сходится  равномерно относительно $x\in[a,b]$, и имеет место равенство
	\begin{equation*}
	f(x)= \sum\nolimits_{k=0}^{r-1} f^{(k)}(a)\frac{(x-a)^k}{k!}+ \sum\nolimits_{k=r}^\infty \hat f_{r,k}\varphi_{r,k}(x),\quad x\in[a,b].
	\end{equation*}
	
\end{theorem}

В  \cite{SharIzv2018} показано, что условие равномерной сходимости ряда \eqref{2.4} для конкретных систем $\varphi^r$ может быть значительно ослаблено и доведено в отдельных случаях до $f\in W^r_{L^1_\rho(a,b)}$.

Важное свойство систем функций $\varphi^r$, определенных равенствами  \eqref{2.1} и \eqref{2.2}, заключается в том, что
\begin{equation}\label{2.6}
(\varphi_{r,k}(x))^{(\nu)} =\begin{cases}\varphi_{r-\nu,k-\nu}(x),&\text{если $0\le\nu\le r-1$, $r\le k$,}\\
\varphi_{k-r}(x)\quad\text{для п.в. $x\in (a,b)$},&\text{если  $\nu=r\le k$,}\\
\varphi_{r-\nu,k-\nu}(x),&\text{если $\nu\le k< r$,}\\
0,&\text{если $k< \nu\le r$}.
\end{cases}
\end{equation}
Отметим некоторые свойства смешанного ряда \eqref{2.4}, непосредственно вытекающие из \eqref{2.6}. Важное значение имеет свойство, которое состоит в том, что его частичная сумма вида
\begin{equation}\label{2.7}
Y_{r,N}(f,x)=\sum_{k=0}^{r-1} f^{(k)}(a)\frac{(x-a)^k}{k!}+ \sum_{k=r}^{N} \hat f_{r,k}\varphi_{r,k}(x)
\end{equation}
при   $N\ge r$ совпадает  $r$-кратно с исходной функцией $f(x)$ в точке $x=a$, т. е.
\begin{equation}\label{2.8}
(Y_{r,N}(f,x))^{(\nu)}_{x=a}=f^{(\nu)}(a)\quad (0\le\nu\le r-1).
\end{equation}
Кроме того, из \eqref{2.6}  следует, что $(0\le\nu\le r-1)$
$$
Y_{r,N}^{(\nu)}(f,x)=\sum_{n=0}^{r-1-\nu} f^{(n+\nu)}(a)\frac{(x-a)^n}{n!}+
$$
\begin{equation}\label{2.9}
\sum_{n=r-\nu}^{N-\nu} \widehat{f^{(\nu)}}_{r-\nu,n}\varphi_{r-\nu,n}(x)=Y_{r-\nu,N-\nu}(f^{(\nu)},x),
\end{equation}
\begin{equation*}
f^{(\nu)}(x)=\sum_{n=0}^{r-1-\nu} f^{(n+\nu)}(a)\frac{(x-a)^n}{n!}+ \sum_{n=r-\nu}^\infty \widehat{f^{(\nu)}}_{r-\nu,n}\varphi_{r-\nu,n}(x),
\end{equation*}
\begin{equation*}
f^{(r)}(x)=\sum_{n=0}^\infty \widehat{f^{(r)}}_{0,n}\varphi_{n}(x).
\end{equation*}

Последнее равенство  означает, что ряд, фигурирующий  в его правой части сходится к $f^{(r)}$ в метрике пространства  $L^2_\rho(a,b)$. Из предпоследнего равенства и из  \eqref{2.9}, в свою очередь, выводим $(0\le\nu\le r-2)$
$$
f^{(\nu)}(x)-Y_{r,N}^{(\nu)}(f,x)= \frac{1}{(r-\nu-2)!}\int_a^x (x-t)^{r-\nu-2}(f^{(r-1)}(t)-Y_{r,N}^{(r-1)}(f,t))dt=
$$
\begin{equation}\label{2.10}
\frac{1}{(r-\nu-2)!}\int_a^x (x-t)^{r-\nu-2}(f^{(r-1)}(t)-Y_{1,N-r+1}(f^{(r-1)},t))dt.
\end{equation}
Дифференциальные свойства смешанных рядов \eqref{2.4}, выраженные равенствами \eqref{2.8} -- \eqref{2.10}, показывают, что их частичные суммы $Y_{r,N}(f,x)$  могут быть использованы в задачах, в которых требуется одновременно приближать заданную дифференцируемую функцию и несколько ее производных. В дальнейшем мы существенно воспользуемся этими свойствами при рассмотрении вопроса о представлении решения задачи Коши для систем дифференциальных уравнений в виде ряда Фурье по системе $\varphi^r$. При этом нам понадобятся некоторые вспомогательные утверждения, которые рассмотрены в следующем подразделе.

\section{Задача Коши для систем ОДУ}\label{SobSystemsAndCauchyProblem}
Рассматривая задачу Коши для систем ОДУ вида $y'(x)=F(x,y), \quad y(a)=y^0$, где  $F=(F_1, \ldots, F_m)$, $y=(y_1, \ldots, y_m)$, мы будем накладывать на вектор-функцию   $F(x,y)$ определенные условия, соблюдение которых гарантирует сходимость некоторых итерационных процессов, сконструированных с целью найти приближенные  значения коэффициентов  в разложении искомого решения задачи Коши в ряд  по ортонормированным по Соболеву   функциям $\varphi_{r,k}$. Вектор-функцию   $F(x,y)$  будем считать непрерывной в некоторой замкнутой  области $\bar G$ переменных $(x,y)$, содержащей точку $(a,y_0)$, и такой, что  $[a,b]\times\mathbb{R}^m\subset\bar G$.  Один из возможных подходов к достижению намеченной цели основан на предположении о возможности  разбиения отрезка $[a,b]$ на несколько частей точками $a=a_0<a_1<\cdots<a_n=b$ так, чтобы на каждом из промежутков $[a_i,a_{i+1}]$ вектор-функция   $F(x,y)$ по переменной $y$
удовлетворяла условию Липшица. Через $y_i(x)$ обозначим решение уравнения $y'(x)=F(x,y)$ с начальным условием $y(a_i)= y^0$ при $i=0$ и  $y_i(a_i)= y_i^0= \tilde y_{i-1}(a_i)$,  если $1\le i\le n-1$, где $\tilde y_i(x)$ -- некоторая функция, которая является аппроксимирующей для $y_i(x)$ при $x\in[a_i,a_{i+1}]$.  Конструирование функции $\tilde y_i(x)$  осуществляется путем разложения на $[0,1]$ функции $\phi_i(t)=y_i(a_i+(a_{i+1}-a_i)t)$ с $i=0,\ldots,n-1$ в ряд Фурье по системе  $\varphi^1=\{\varphi_{1,k}\}$.
Наконец,  строится функция $\tilde y(x)$, аппроксимирующая решение $y(x)$ исходной задачи Коши $y'(x)=F(x,y)$ на всем отрезке $[a,b]$  с начальным условием $y(a)=y^0$ путем <<склеивания>> аппроксимирующих функций $\tilde y_i(x)$ с $i=0,1,\ldots, n-1$. При этом возникает задача об оценке погрешности $\max_{x\in[a,b]}|y(x)-\tilde y(x)|$. Для простоты выкладок мы ограничимся в настоящем параграфе рассмотрением этого вопроса в одномерном случае для задачи Коши вида
\begin{equation}\label{3.1}
u'(x)=F(x,u), \quad u(a)=u_0,
\end{equation}
в которой функцию   $F(x,u)$  будем считать непрерывной в некоторой области $\bar G$, которая в качестве своего подмножества содержит множество $[a,b]\times\mathbb{R}$. Будем также предполагать, что функция $F(x,u)$ по переменной $u$  удовлетворяет условиям
\begin{equation}\label{3.2}
|F(x,q')-F(x,q'')|\le \lambda_i|q'-q''|, \quad a_i\le x \le a_{i+1},
\end{equation}
где $a=a_0<a_1<\cdots<a_n=b$,  $\lambda_i\geq0$ ($0\leq i\leq n-1$). Требуется с заданной точностью приблизить на $[a,b]$ функцию  $u=u(x)$, которая является   решением задачи Коши \eqref{3.1}. Для этого нам понадобятся некоторые вспомогательные утверждения.
\begin{lemma}\label{lem1} Пусть непрерывная функция  $F(x,u)$, заданная в области $\bar  G$ переменных $(x,u)$, удовлетворяет условию Липшица $|F(x,q')-F(x,q'')|\le\lambda|q'-q''|$, в котором постоянная $\lambda$ не зависит от $x\in [\alpha,\beta]$.  Далее, пусть  $(\alpha,u_0)\in \bar G$, $(\alpha,v_0)\in \bar  G$, $u= u(x)$ и $v =v(x)$ два решения уравнения $u'(x)=F(x,u)$   на $[\alpha,\beta]$  с соответствующими начальными условиями  $u(\alpha)=u_0$,  $v(\alpha)=v_0$. Тогда если $(\beta-\alpha)\lambda<1$, то имеет место неравенство
	$$
	\max_{t\in[\alpha,\beta]}|u(t)-v(t)|\le \frac{|u_0-v_0|}{1-(\beta-\alpha)\lambda}.
	$$
\end{lemma}
\begin{lemma}\label{lem2} Пусть непрерывная функция  $F(x,u)$ задана в области $\bar  G$ переменных $(x,u)$. Предположим, что $[a,b]\times\mathbb{R}\subset \bar G$, и пусть отрезок  $[a,b]$ разбит на части точками $a=a_0<a_1<\cdots<a_n=b$.   Далее, пусть  $(a_i,u_i^0)\in \bar G$ при $i=0,\ldots,n-1$, $u_i= u_i(x)$ -- решения уравнения $u'(x)=F(x,u)$   на $[a_i,b]$, удовлетворяющие соответствующим начальным условиям  $u_i(a_i)=u_i^0$. Будем считать, что $F(x,u)$ по переменной $u$  удовлетворяет при каждом $i=0,\ldots,n-1$ условию Липшица \eqref{3.2}.   Тогда если $\lambda_{k}(a_{k+1}-a_k)<1$ при всех $k=1,\ldots,i$, то для $u(x)=u_0(x)$ имеет место неравенство
	$$
	\max_{t\in[a_i,a_{i+1}]}|u(t)-u_i(t)|\le \sum_{l=1}^{i}Q_{l}^{i}|u_{l-1}(a_{l})-u_{l}(a_l)|,
	$$
	где $1\le i\le n-1$,
	$$
	Q_l^i=\prod_{j=0}^{i-l}\frac{1}{1-\lambda_{i-j}(a_{i-j+1}-a_{i-j})}.
	$$
\end{lemma}
\begin{proof}
	Для $1\le i\le n-1$, $t\in[a_i,a_{i+1}]$ имеем
	$$
	|u(t)-u_i(t)|\le \sum_{l=1}^{i}|u_{l-1}(t)-u_l(t)|,
	$$
	а в силу леммы \ref{lem1}
	$$
	|u_{l-1}(t)-u_l(t)|\le \frac{|u_{l-1}(a_i)-u_{l}(a_i)|}{1-\lambda_i(a_{i+1}-a_i)}\le
	$$
	$$
	\frac{|u_{l-1}(a_{i-1})-u_{l}(a_{i-1})|}
	{(1-\lambda_i(a_{i+1}-a_i))(1-\lambda_{i-1}(a_{i}-a_{i-1}))}\le\ldots\le
	$$
	$$
	\frac{|u_{l-1}(a_{l})-u_{l}(a_{l})|}
	{(1-\lambda_i(a_{i+1}-a_i))\cdots(1-\lambda_{l}(a_{l+1}-a_{l}))}.
	$$
	Утверждение леммы \ref{lem2} немедленно вытекает из приведенных неравенств.
\end{proof}
Рассмотрим вопрос о конструировании функции $\tilde u(x)$, определенной на $[a,b]$, которая с заданной точностью будет   приближать  решение $u(x)$ уравнения  $u'(x)=F(x,u)$  на $[a,b]$  c начальным условием  $u(a)=u^0_0$. С этой целью мы обратимся к лемме \ref{lem2}.
Предположим, что на каждом из отрезков $[a_i,a_{i+1}]$, фигурирующих в лемме \ref{lem2}, мы построили непрерывную функцию $\tilde u_i(x)$,  приближающую решение $u_i(x)$ уравнения $u'(x)=F(x,u)$   на $[a_i,a_{i+1}]$  c начальным условием  $u_i(a_i)=u_i^0$ так, чтобы  было  $\tilde u_i(a_i)= u_i(a_i)$ ($0\le i\le n-1$), а при $1\le i\le n-1$ выполняются условия $\tilde u_i(a_i)=\tilde u_{i-1}(a_i)$.  На всем отрезке $[a,b]$ определим непрерывную функцию $\tilde u(x)$, полагая
$\tilde u(x)=\tilde u_i(x)\quad \text{при}\quad x\in [a_i,a_{i+1}], i=0,\ldots, n-1$.
Положим $\delta_i=\max_{x\in[a_i,a_{i+1}]}|\tilde u_i(x)- u_i(x)|$. Тогда из леммы \ref{lem2} имеем
$$
\max_{t\in[a_i,a_{i+1}]}|u(t)-\tilde u_i(t)|\le \max_{t\in[a_i,a_{i+1}]}|u(t)- u_i(t)|+\delta_i\le
$$
$$
\sum_{l=1}^{i}Q_{l}^{i}|u_{l-1}(a_{l})-u_{l}(a_l)|+\delta_i=
\sum_{l=1}^{i}Q_{l}^{i}|u_{l-1}(a_{l})-\tilde u_{l-1}(a_{l})| +\delta_i\le
\sum_{l=1}^{i}Q_{l}^{i}\delta_{l-1}+\delta_i.
$$
Поэтому мы можем сформулировать следующее утверждение.
\begin{lemma}\label{lem3}
	Пусть соблюдены условия леммы \ref{lem2}   и, кроме того,\\ $\lambda_{k}(a_{k+1}-a_k)<1$ при всех $k=1,\ldots,n-1$. Тогда для $u(t)=u_0(t)$  имеют место следующие неравенства
	$$
	\max_{t\in[a_0,a_1]}|u(t)-\tilde u(t)|\le \delta_0,
	$$
	$$
	\max_{t\in[a_i,a_{i+1}]}|u(t)-\tilde u(t)|\le\sum\nolimits_{l=1}^iQ_{l}^i\delta_{l-1}+\delta_i, \quad 1\le i\le n-1.
	$$
\end{lemma}
Перейдем к конструированию аппроксимирующих  функций $\tilde u_i$, удовлетворяющих условиям  леммы \ref{lem2}. Мы покажем, что удобным и весьма эффективным инструментом  решения этой задачи являются ряды Фурье по ортогональным по Соболеву системам функций, рассмотренным нами в подразделе \ref{SobSystemsAndCauchyProblem}. Следует при этом отметить, что выбор разбиения $a=a_0<a_1<\cdots<a_n=b$, фигурирующего в леммах \ref{lem2} и \ref{lem3}, оптимального в том или ином смысле, существенно зависит от свойств функции $F(x,u)$, в том числе  и от того, насколько большими являются константы $\lambda_i$ в условиях Липшица $|F(x,q')-F(x,q'')|\le\lambda_i|q'-q''|$ ($x\in [a_i,a_{i+1}]$) с $i=0,\ldots, n-1$. Не останавливаясь на подробном обсуждении этого вопроса, мы перейдем к задаче о приближении  искомого решения $u_i(x)$ задачи Коши на отрезке $[a_i,a_{i+1}]$ с заданным начальным условием $u_i(a_i)$, где $0\le i\le n-1$.
Идея построения функций $\tilde u_i(x)$, аппроксимирующих на соответствующих отрезках $[a_i,a_{i+1}]$ искомые  решения $u_i(x)$ и таких, что $u_i(a_i)=\tilde u_i(a_i)$ при $0\le i\le n-1$ и $\tilde u_i(a_i)= \tilde u_{i-1}(a_i)$ при $1\le i\le n-1$,
состоит в следующем.  Предположим, что $[a,b]\times\mathbb{R}\subset \bar G$, а  функция $F(x,u)$ непрерывна в области $\bar G$ и по переменной $u$  удовлетворяет условиям \eqref{3.2}. Положим $h=a_{i+1}-a_{i}$ и   отобразим линейно отрезок $[0,1]$ на $[a_i,a_{i+1}]=[a_i,a_i+h]$ посредством замены переменных  $x=a_i+ht$. Относительно новой переменной $t\in [0,1]$ уравнение \eqref{3.1}, рассматриваемое на отрезке $[a_i,a_i+h]$,  принимает вид
\begin{equation}\label{3.3}
\phi'(t)=hF(a_i+ht,\phi(t)), \quad \phi(0)= u_i(a_i),\quad 0\le t\le 1,
\end{equation}
где $\phi(t)=\phi_i(t)=u_i(a_i+ht)$, $u_0(a_0)=u_0(a)$, $ u_i(a_i)= \tilde u_{i-1}(a_i)$ при $1\le i\le n-1$. Для конструирования функции $\tilde u_0(x)$, аппроксимирующей решение $u_0(x)$ уравнения \eqref{3.1} на отрезке $[a_0,a_0+h]$ с начальным условием $u_0(a_0)=u(a)$, мы представим функцию $f(t)=\phi(t)=u_0(a_0+ht)$ в виде ряда Фурье \eqref{2.4} с $r=1$:
\begin{equation}\label{3.4}
\phi(t)= \phi(0)+ \sum\nolimits_{k=1}^\infty \hat \phi_{1,k}\varphi_{1,k}(t),
\end{equation}
где
\begin{equation*}
\hat \phi_{1,1+j}=\int_{0}^1 \phi'(t)\varphi_{j}(t)\rho(t)dt\quad(j\ge0).
\end{equation*}
и положим для $x=a_0+ht$
$$
\tilde u_0(x)=Y_{1,N_0}(\phi, t)= \phi(0)+ \sum\nolimits_{k=1}^{N_0} \hat \phi_{1,k}\varphi_{1,k}(t),\quad 0\le t\le 1,
$$
где $Y_{1,N_0}(\phi, t)$ -- частичная сумма ряда Фурье \eqref{3.4}, $N_0$ -- произвольное натуральное число, которое при решении конкретной задачи следует  выбрать в зависимости от требуемой точности приближения искомого решения $u_0(x)=\phi(0)+ \sum\nolimits_{k=1}^\infty \hat \phi_{1,k}\varphi_{1,k}(t)$ на $[a_0,a_1]$ аппроксимирующей функцией $\tilde u_0(x)$. Если  $1\le i\le n-1$, то в качестве начального значения искомого решения $u_i(x)$ на отрезке $[a_i,a_{i+1}]$ мы берем число $\tilde u_{i-1}(a_i)$ и конструируем $\tilde u_i(x)$ совершенно аналогично тому, как была построена $\tilde u_{i-1}(x)$.
Из свойства \eqref{2.8}, которым обладают частичные суммы $Y_{1,N}(f,t)$ смешанного ряда \eqref{3.4}, непосредственно вытекает, что если $1\le i\le n-1$, то $\phi_{i-1}(1)=\tilde u_{i-1}(a_i)=u_i(a_i)=\tilde u_i(a_i)=\phi_i(0)$. Тем самым, аппроксимирующие функции  $\tilde u_i(x)$ подчиняются условиям леммы \ref{lem2}, и, стало быть, справедливо неравенство для погрешности $ |u_i(x)-\tilde u_i(x)|$, установленное в этой лемме. Далее, заметим, что погрешности   $\delta_i $, фигурирующие в лемме  \ref{lem3}, приобретают вид $\delta_i=\max_{x\in[a_i,a_{i+1}]}|u_i(x)-\tilde u_i(x)|=\max_{t\in[0,1]}|V_{N_i}(\phi,t)|$, где
$V_{N_i}(\phi,t)=\sum\nolimits_{j=N_i+1}^\infty \hat \phi_{1,j}\varphi_{1,j}(t)$. Отсюда непосредственно возникает задача об исследовании поведения величины $\max_{t\in[0,1]}|V_{N_i}(\phi,t)|$ при $N_i\to\infty$.  Этот вопрос  для   ортонормированных по Соболеву систем, порожденных некоторыми классическими ортонормированными системами, рассмотрен в работах  \cite{Shar11, Shar2003, Shar2006, Shar2008, Shar19, Shar18, sharap3, Shar_Dag_Elec, SHII, Shar2017, SharSMJ2017}, \cite{SharIzv2018}.
Если мы определим  с помощью равенств $\tilde u(x)=\tilde u_i(x)=Y_{1,N_i}(\phi, t)$  ($x\in [a_i,a_{i+1}]$, $i=0,1,\ldots, n-1$)  аппроксимирующую функцию для искомого решения $u(x)$  задачи Коши \eqref{3.1}, то для оценки  разности $|u(x)-\tilde u(x)|$  можно использовать неравенства, полученные в лемме \ref{lem3}, а для оценки погрешностей $\delta_l$, фигурирующих в этой лемме, можно использовать соответствующие оценки, установленные для выбранной ортонормированной по Соболеву системы $\{\varphi_{1,k}(t)\}_{k=0}^\infty$. Следует при этом отметить, что гладкость функции $u(x)$, представляющей собой решение задачи Коши для уравнения \eqref{3.1}, а также   гладкость функций $\phi_i(t)=u_i(a_i+ht)$ по переменной  $t\in[0,1]$,  следовательно, и скорость стремления к нулю величины $\max_{t\in[0,1]}|V_{N_i}(\phi,t)|$ при $N_i\to\infty$,  в свою очередь, существенно зависят от свойств функции $F(x,u)$.


\section{О представлении решения задачи Коши для систем ОДУ рядом Фурье по функциям $\varphi_{1,n}(x)$}\label{CauchyProblemSolutionRepr}
В предыдущем подразделе на примере одномерной задачи Коши \eqref{3.1} было показано, что если правая часть уравнения \eqref{3.1} удовлетворяет условиям \eqref{3.2}, то вопрос о конструировании аппроксимирующей функции $\tilde u(x)$ для ее решения $u(x)$ на отрезке $[a,b]$ может быть  сведен к задаче о последовательном  конструировании аппроксимирующих функций $\tilde u_i(x)$ с $i=0, 1,\ldots, n-1$ для соответствующих   решений $u_i(x)$ уравнения \eqref{3.1} с указанными там начальными условиями на  отрезках $[a_i,a_{i+1}]$, где $a=a_0<a_1<\ldots<a_{n-1}<a_n=b$. Аппроксимирующая функция $\tilde u(x)$ для $u(x)$ на всем отрезке $[a,b]$ строится путем <<склеивания>> функций $\tilde u_i(x)$, т. е. требуется выполнение условий $\tilde u_0(a)=u(a)$,
$\tilde u_{i-1}(a_i)=\tilde u_i(a_i)=u_i(a_i)$ ($1\le i\le n-1$). Совершенно аналогично можно поступить и при решении задачи Коши для систем ОДУ.

Перейдем теперь к вопросу о конструировании аппроксимирующих функций $\tilde u_i(x)$. При этом, считая для простоты записи $a_0=0$, $a_1=h$,  мы ограничимся  рассмотрением этого вопроса лишь на отрезке $[a_0, a_1]=[0, h]$ для задачи Коши вида
\begin{equation}\label{4.1}
y'(x)=f(x,y), \quad y(0)=y^0,
\end{equation}
где   $f=(f_1, \ldots, f_m)$, $y=(y_1, \ldots, y_m)$, вектор-функция  $f(x,y)$  непрерывна в некоторой замкнутой  области $\bar G$ переменных $(x,y)$, такой что  $[0,h]\times\mathbb{R}^m\subset\bar G$ для  некоторого числа $h>0$. Предположим также, что   $f(x,y)$ по переменной $y$ удовлетворяет нижеследующему условию Липшица \eqref{4.3} равномерно относительно  $x\in[0,h]$. Заметим, что требование $[0,h]\times\mathbb{R}^m\subset\bar G$ не сужает дальнейшие рассмотрения, так как, не ограничивая  общности,  мы можем в случае необходимости продолжить функцию $f(x,y)$ по переменной $y$ на всё $\mathbb{R}^m$, сохраняя свойство ее подчиненности  условию Липшица \eqref{4.3}. Например, если область $\bar G$ такова, что  прямая в $\mathbb{R}^{m+1}$ вида $(x,ty_1,\ldots,ty_m)$ ($t\in\mathbb{R}$) для каждого $x\in[0,h]$ и $(y_1,\ldots,y_m)\in\mathbb{R}^{m}$ пересекается с границей области $\bar G$ не более, чем в двух (граничных для $\bar G$) точках $(x,y')$ и $(x,y'')$, то  $f(x,y)$ можно непрерывно продолжить   на $[0,h]\times\mathbb{R}^m$, считая ее  постоянной на лучах, выходящих из точек  $(x,y')$ и $(x,y'')$ в противоположных направлениях вдоль прямой $(x,ty_1,\ldots,ty_m)$ ($t\in\mathbb{R}$).

Мы рассмотрим задачу о приближении решения задачи Коши \eqref{4.1}     суммами  Фурье по системе $\{\varphi_{1,n}(x)\}_{n=0}^\infty$, ортонормированной по Соболеву и порожденной ортонормированной системой функций $\{\varphi_{n}(x)\}_{n=0}^\infty$ посредством равенств \eqref{2.1} и \eqref{2.2} с $a=0$, $b=1$, $r=1$.
Требуется аппроксимировать с заданной точностью вектор-функцию $y=y(x)$, определенную на $[0,h]$, которая является решением задачи Коши \eqref{4.1}.
Будем считать, что весовая функция $\rho(x)$ интегрируема на $(0,1)$, система $\{\varphi_{n}(x)\}_{n=0}^\infty$ удовлетворяет условиям теоремы \ref{th2}, а порожденная система $\{\varphi_{1,n}(x)\}_{n=0}^\infty$ -- условиям $(0\le x\le 1)$
\begin{equation}\label{4.2}
\delta_\varphi(x)=\sum\nolimits_{k=1}^{\infty}(\varphi_{1,k}(x))^2<\infty,\quad
\kappa_{\varphi}=\left(\int_0^1\sum_{k=1}^{\infty}
(\varphi_{1,k}(t))^2\rho(t)dt\right)^{\frac12}<\infty.
\end{equation}
Кроме того, мы предположим, что по переменной $y$ функция $f(x,y)$ удовлетворяет условию Липшица
\begin{equation}\label{4.3}
\|f(x,a)-f(x,b)\|\le \lambda_0\|a-b\|, \quad 0\le x \le h,
\end{equation}
где $\|(a_1,\ldots,a_m)\|=\sqrt{\sum_{l=1}^ma_l^2}$. Полагая $x=ht$, отобразим линейно отрезок $[0,1]$ на $[0,h]$. Относительно новой переменной $t\in [0,1]$ уравнение \eqref{4.1} принимает следующий вид
\begin{equation}\label{4.4}
\eta'(t)=hf(ht,\eta(t)), \quad \eta(0)=y_0,\quad 0\le t\le 1,
\end{equation}
где $\eta(t)=y(ht)$.  Поскольку по предположению вектор-функция $f(x,y)$ непрерывна в области $\bar G$, то из \eqref{4.4} следует, что  вектор-функция $$\eta'(t)=(\eta'_1(t),\ldots,\eta'_l(t),\ldots,\eta'_m(t))$$ непрерывна на $[0,1]$ и, следовательно, $\eta(t)=(\eta_1(t),\ldots,\eta_l(t),\ldots,\eta_m(t))$, где $\eta_l\in W_{L_\rho^2(0,1)}^1$ при всех $l=1,\ldots,m$. Поэтому в силу теоремы \ref{th2}  мы можем представить  функцию $\eta(t)$ в виде равномерно сходящегося на $[0,1]$ ряда Фурье по порожденной системе $\{\varphi_{1,n}(t)\}_{n=0}^\infty$:
\begin{equation}\label{4.5}
\eta(t)= \eta(0)+ \sum\nolimits_{k=1}^\infty \hat \eta_{1,k}\varphi_{1,k}(t),
\end{equation}
где
\begin{equation}\label{4.6}
\hat \eta_{1,k}=(\widehat{\eta_1}_{1,k},\ldots,\widehat {\eta_l}_{1,k},\ldots,\widehat{\eta_m}_{1,k})=\int_{0}^1 \eta'(t)\varphi_{k-1}(t)\rho(t)dt\quad(k\ge1).
\end{equation}
Наша цель состоит в том, чтобы сконструировать итерационный процесс для нахождения приближенных значений коэффициентов $c_k=\hat \eta_{1,k+1}/h$ $(k=0,1,\ldots)$, где $c_k=(c_k^1,\ldots,c_k^m)$. Для этого обратимся к равенству \eqref{4.5} и запишем
\begin{equation}\label{4.7}
\eta'(t)=  \sum\nolimits_{k=0}^\infty \hat \eta_{1,k+1}\varphi_k(t),
\end{equation}
где равенство понимается в том смысле, что ряд в правой части равенства \eqref{4.7} покомпонентно сходится к $\eta'=(\eta'_1,\ldots,\eta'_m)$ в метрике пространства $L^2_{\rho}(0,1)$ для всех $\nu=1,\ldots,m$. Положим $s=1/h$, $q(t)=f(ht,\eta(t))=s\eta'(t)$ и заметим, что в силу  \eqref{4.6} (см. также \eqref{4.7}) коэффициенты Фурье вектор-функции $q=q(t)$ по системе  $\{\varphi_{n}(t)\}_{n=0}^\infty$ имеют вид
\begin{equation}\label{4.8}
c_k(q)=\int_{0}^1 q(t)\varphi_{k}(t)\rho(t)dt=s\hat \eta_{1,k+1} \quad (k\ge0).
\end{equation}
С учетом этих равенств мы можем переписать \eqref{4.5} в следующем виде
\begin{equation}\label{4.9}
\eta(t)= \eta(0)+ h\sum\nolimits_{k=0}^\infty c_k(q){\varphi}_{1,k+1}(t).
\end{equation}
Из  \eqref{4.8} и \eqref{4.9}, в свою очередь, выводим следующие соотношения
\begin{equation}\label{4.10}
c_k(q)=\int_{0}^1f\left[ht,\eta(0)+ h\sum\nolimits_{j=0}^\infty c_j(q)\varphi_{1,j+1}(t)\right]\varphi_k(t)\rho(t) dt,\, k=0,1,\ldots.
\end{equation}
Введем в рассмотрение гильбертово пространство $l_2^m$, состоящее из $m$-мерных последовательностей $C=(c_0,c_1,\ldots)$, для которых определена норма
$$\|C\|=\left(\sum\nolimits_{j=0}^\infty \sum\nolimits_{l=1}^{m}(c_j^l)^2\right)^\frac12.$$

Покажем, что в пространстве $l_2^m$ можно определить   оператор $A$, сопоставляющий точке $C\in l_2^m$ точку $C'\in l_2^m$ по правилу
\begin{equation}\label{4.11}
c_k'=\int_{0}^1f\left[ht,\eta(0)+ h\sum\nolimits_{j=0}^\infty c_j
\varphi_{1,j+1}(t)\right]\varphi_k(t)\rho(t) dt,\quad k=0,1,\ldots.
\end{equation}
Для этого достаточно показать, что точка $C'=(c_0',c_1',\ldots)\in l_2^m$. С этой целью для каждого $1\le l\le m$ рассмотрим функцию
$\phi_l'=\phi_l'(t)= \sum\nolimits_{j=0}^\infty hc_j^l\varphi_j(t)$. Поскольку $C=(c_0,c_1,\ldots)\in l_2^m$, то $\phi_l'\in L^2_\rho(0,1)\subset L(0,1)$, следовательно, функция $\phi_l(x)=\eta_l(0)+\int_0^x\phi_l'(t)dt\in W^1_{L^2_\rho(0,1)}$ и, в силу теоремы \ref{th2},  $\phi_l(x)=\eta_l(0)+\sum_{j=1}^\infty{\phi_l}_{1,j}\varphi_{1,j}(x)$, причем этот ряд сходится равномерно на $[0,1]$, а  коэффициенты ${\phi_l}_{1,j+1}$ $(j=0,1,\ldots)$ имеют вид
${\phi_l}_{1,j+1}=\int_0^1\phi_l'(t)\varphi_{j}(t)\rho(t)dt=hc_j^l$, другими словами, функция
$\phi_l(x)=\eta_l(0)+h\sum_{j=0}^\infty c_j^l\varphi_{1,j+1}(x)$ абсолютно непрерывна на $[0,1]$. Поэтому функция $f(ht,\phi(t))$ непрерывна на $[0,1]$ и, стало быть,
$f_l\left[ht,\eta(0)+ h\sum\nolimits_{j=0}^\infty c_j
\varphi_{1,j+1}(t)\right]\in L^2_\rho(0,1)$, поэтому\underline{} из \eqref{4.11} заключаем, что $C'=(c_0',c_1',\ldots)\in l_2^m$. Отсюда следует, что мы можем определить  оператор $A:l_2^m\to l_2^m$ по правилу \eqref{4.11}.
Из  \eqref{4.10} следует, что точка $C(q)=(c_0(q),c_1(q),\ldots)$ является неподвижной точкой оператора $A:l_2^m\to l_2^m$. Для того чтобы найти точку $C(q)$ методом простых итераций, достаточно показать, что оператор $A:l_2^m\to l_2^m$ является сжимающим в метрике пространства $l_2^m$. С этой целью рассмотрим две точки $P,Q\in l_2^m$, где $P=(p_0,p_1\ldots)$, $Q=(q_0,q_1\ldots)$, и положим $P'=A(P)$, $Q'=A(Q)$. Имеем
\begin{equation}\label{4.12}
p'_k-q'_k=\int_{0}^1F_{P,Q}(t)\varphi_k(t)\rho(t)dt,\quad k=0,1,\ldots
\end{equation}
где
\begin{multline}\label{4.13}
F_{P,Q}(t)=f\left[ht,\eta(0)+ h\sum\nolimits_{j=0}^\infty p_j\varphi_{1,j+1}(t)\right] \\
-f\left[ht,\eta(0)+ h\sum\nolimits_{j=0}^\infty q_j\varphi_{1,j+1}(t)\right].
\end{multline}
Из \eqref{4.12}, пользуясь неравенством Бесселя, находим
\begin{equation}\label{4.14}
\sum\nolimits_{k=0}^\infty \sum_{l=1}^m((p^l_k)'-(q^l_k)')^2\le\int_{0}^1F_{P,Q}(t)(F_{P,Q}(t))^*\rho(t) dt,
\end{equation}
где $(a_1,\ldots,a_m)^*$ -- вектор-столбец, полученный в результате транспонирования строки $(a_1,\ldots,a_m)$.
Из \eqref{4.13} и \eqref{4.3}  имеем
$$
F_{P,Q}(t)(F_{P,Q}(t))^*\le (h\lambda_0)^2 \sum\nolimits_{l=1}^m  \left(\sum\nolimits_{j=0}^\infty( p^l_j-q^l_j)\varphi_{1,j+1}(t)\right)^2,
$$
откуда,  воспользовавшись неравенством Коши-Буняковского, выводим
\begin{equation}\label{4.15}
F_{P,Q}(t)(F_{P,Q}(t))^*\le(h\lambda_0)^2  \sum\nolimits_{j=0}^\infty(\varphi_{1,j+1}(t))^2 \sum\nolimits_{j=0}^\infty\sum\nolimits_{l=1}^m( p^l_j-q^l_j)^2.
\end{equation}
Сопоставляя \eqref{4.15} с \eqref{4.14}, находим
$$
\sum\nolimits_{j=0}^\infty\sum\nolimits_{l=1}^m((p^l_j)'-(q^l_j)')^2\le
$$
\begin{equation}\label{4.16}
(h\lambda_0)^2 \sum\nolimits_{j=0}^\infty\sum\nolimits_{l=1}^m( p^l_j-q^l_j)^2\int_{0}^1 \sum\nolimits_{j=0}^\infty(\varphi_{1,j+1}(t))^2\rho(t) dt.
\end{equation}
Из  \eqref{4.16}  и \eqref{4.2} имеем
\begin{equation}\label{4.17}
\left(\sum\nolimits_{j=0}^\infty\sum\nolimits_{l=1}^m((p^l_j)'-(q^l_j)')^2\right)^\frac12\le h\kappa_\varphi\lambda_0 \left(\sum\nolimits_{j=0}^\infty\sum\nolimits_{l=1}^m( p^l_j-q^l_j)^2\right)^\frac12.
\end{equation}
Неравенство \eqref{4.17} показывает, что если $h\kappa_\varphi\lambda_0<1$, то оператор  $A:l_2^m\to l_2^m$ является сжимающим и, как следствие, итерационный процесс $C^{\nu+1}=A(C^{\nu})$  сходится к точке $C(q)$ при $\nu\to\infty$. Однако с точки зрения приложений важно рассмотреть конечномерный аналог оператора $A$. Обозначим через $\mathbb{R}^N_m$ пространство матриц $C$ размерности $m\times N$, в котором определена норма
$$\|C\|_N^m=\left(\sum\nolimits_{j=0}^{N-1} \sum\nolimits_{l=1}^{m}(c_j^l)^2\right)^\frac12.$$
Мы рассмотрим оператор $A_N:\mathbb{R}^N_m\to \mathbb{R}^N_m$, сопоставляющий точке\\
$C_N=(c_0,\ldots,c_{N-1})\in \mathbb{R}^N_m $ точку  $C'_N=(c_0',\ldots,c_{N-1}')\in \mathbb{R}^N_m $ по правилу
\begin{equation}\label{4.18}
c_k'=\int\limits_{0}^1f\left[ht,\eta(0)+ h\sum\nolimits_{j=0}^{N-1} c_j\varphi_{1,j+1}(t)\right]\varphi_k(t)\rho(t) dt,\,k=0,1,\ldots, N-1.
\end{equation}
Рассмотрим две точки $P_N,Q_N\in \mathbb{R}^N_m$, где $P_N=(p_0,p_1,\ldots,p_{N-1})$,\\   $Q_N=(q_0,q_1,\ldots,p_{N-1})$ и положим $P'_N=A_N(P_N)$, $Q'_N=A_N(Q_N)$. Дословно повторяя рассуждения, которые привели нас к неравенству \eqref{4.17}, мы получим
\begin{equation}\label{4.19}
\left(\sum\nolimits_{j=0}^{N-1}\sum\nolimits_{l=1}^m((p^l_j)'-(q^l_j)')^2\right)^\frac12\le h\kappa_\varphi\lambda_0 \left(\sum\nolimits_{j=0}^{N-1}\sum\nolimits_{l=1}^m( p^l_j-q^l_j)^2\right)^\frac12.
\end{equation}

Неравенство \eqref{4.19} показывает, что если $h\kappa_\varphi\lambda_0<1$, то оператор  $A_N:\mathbb{R}^N_m\to \mathbb{R}^N_m$ является сжимающим и, как следствие, итерационный процесс $C_N^{\nu+1}=A_N(C_N^{\nu})$  при $\nu\to\infty$ сходится к его неподвижной точке, которую мы обозначим через  $\bar C_N(q)=(\bar c_0(q),\ldots,\bar c_{N-1}(q))$. С другой стороны, рассмотрим точку $C_N(q)=(c_0(q),\ldots,c_{N-1}(q))$, составленную из искомых коэффициентов Фурье вектор-функции $q$ по системе $\varphi$. Нам остается оценить погрешность, проистекающую в результате замены точки $C_N(q)$ точкой $\bar C_N(q)$. Другими словами, требуется оценить величину \linebreak
$\|C_N(q)-\bar C_N(q)\|_N^m= \left(\sum_{j=0}^{N-1}\sum\nolimits_{l=1}^m(c_j^l(q)-\bar c_j^l(q))^2\right)^\frac12$. С этой целью рассмотрим точку $C'_N(q)=A_N(C_N(q))=(c_0'(q),\ldots,c_{N-1}'(q))$ и запишем
\begin{equation}\label{4.20}
\|C_N(q)-\bar C_N(q)\|_N^m\le \|C_N(q)- C_N'(q)\|_N^m+\|C_N'(q)-\bar C_N(q)\|_N^m.
\end{equation}
Далее, пользуясь неравенством \eqref{4.19}, имеем
$$
\|C_N'(q)-\bar C_N(q)\|_N^m=\|A_N(C_N(q))-A_N(\bar C_N(q))\|_N^m\le
$$
\begin{equation}\label{4.21}
h\kappa_\varphi\lambda_0\|C_N(q)-\bar C_N(q)\|_N^m.
\end{equation}
Из \eqref{4.20} и \eqref{4.21} выводим
\begin{equation}\label{4.22}
\|C_N(q)-\bar C_N(q)\|_N^m\le \frac1{1-h\kappa_\varphi\lambda_0}\|C_N(q)- C_N'(q)\|_N^m.
\end{equation}
Чтобы оценить норму в правой части неравенства \eqref{4.22}, заметим, что в силу неравенства Бесселя
\begin{equation}\label{4.23}
(\|C_N(q)- C_N'(q)\|_N^m)^2\le \int_{0}^1F_{C(q),C_N(q)}(t)(F_{C(q),C_N(q)}(t))^*\rho(t) dt,
\end{equation}
где
\begin{multline}\label{4.24}
F_{C(q),C_N(q)}(t)=f\left[ht,\eta(0)+ h\sum\nolimits_{j=0}^\infty c_j(q)\varphi_{1,j+1}(t)\right] \\
-f\left[ht,\eta(0)+ h\sum\nolimits_{j=0}^{N-1}c_j(q)\varphi_{1,j+1}(t)\right].
\end{multline}
Из \eqref{4.24} и \eqref{4.3} следует, что
$$
F_{C(q),C_N(q)}(t)(F_{C(q),C_N(q)}(t))^*\le \lambda_0^2 \sum\nolimits_{l=1}^m  \left(\sum\nolimits_{j=N}^\infty hc_j^l(q)\varphi_{1,j+1}(t)\right)^2,
$$
отсюда с учетом того, что $hc_k=\hat \eta_{1,k+1}$ $(k=0,1,\ldots)$, имеем
\begin{equation}\label{4.25}
F_{C(q),C_N(q)}(t)(F_{C(q),C_N(q)}(t))^*\le \lambda_0^2   \sum\nolimits_{l=1}^m \left(\sum\nolimits_{j=N}^\infty  \widehat {\eta_l}_{1,j+1}\varphi_{1,j+1}(t)\right)^2.
\end{equation}
Сопоставляя \eqref{4.25} с \eqref{4.23}, получаем
\begin{equation}\label{4.26}
(\|C_N(q)- C_N'(q)\|_N^m)^2\le \lambda_0^2\sum\nolimits_{l=1}^m\int_{0}^1\left(\sum\nolimits_{j=N}^\infty \widehat {\eta_l}_{1,j+1}\varphi_{1,j+1}(t)\right)^2\rho(t) dt,
\end{equation}
где согласно \eqref{4.6}
\begin{equation}\label{4.27}
\hat \eta_{1,j+1}=\int_{0}^1\eta'(t)\varphi_j(t)\rho(t)dt \quad(j=0,1,\ldots)
\end{equation}
-- коэффициенты Фурье вектор-функции $\eta'(t)=hf(ht,\eta(t))$.

Подводя итоги, из \eqref{4.22} и \eqref{4.26}  мы можем сформулировать следующий результат.
\begin{theorem}\label{th3} Пусть область $\bar G$ такова, что $[0,h]\times\mathbb{R}^m\subset \bar G$, вектор-функция $f(x,y)$ непрерывна в области $\bar G$ и удовлетворяет условию Липшица \eqref{4.3}, а $h$ и $\lambda_0$ удовлетворяет неравенству $h\lambda_0\kappa_\varphi<1$, где величина $\kappa_\varphi$ определена равенством \eqref{4.2}. Далее, пусть $l_2^m$ гильбертово пространство, состоящее из $m$-мерных последовательностей $C=(c_0,c_1\ldots)$, для которых введена норма $\|C\|=\left(\sum\nolimits_{j=0}^{\infty} \sum\nolimits_{l=1}^{m}(c_j^l)^2\right)^\frac12$,   оператор $A: l_2^m\to l_2^m$ сопоставяет точке $C\in l_2^m$ точку $C'\in l_2^m$ по правилу \eqref{4.11}. Кроме того, пусть $A_N:\mathbb{R}^N_m\to \mathbb{R}^N_m$ -- конечномерный аналог оператора $A$, сопоставляющий точке $C_N=(c_0,\ldots,c_{N-1})\in \mathbb{R}^N_m $ точку  $C'_N=(c_0',\ldots,c_{N-1}')\in \mathbb{R}^N_m $ по правилу \eqref{4.18}.
	Тогда операторы $A: l_2^m\to l_2^m$ и $A_N:\mathbb{R}^N_m\to \mathbb{R}^N_m$ являются сжимающими и, следовательно, существуют  их неподвижные точки $C(q)=(c_0(q),c_1(q),\ldots)=A(C(q))\in l_2^m$ и $\bar C_{N}(q)=(\bar c_0(q),\bar c_1(q),\ldots,\bar c_{N-1}(q))=A_N(\bar C_N(q))\in \mathbb{R}^N_m$, для которых имеет место неравенство
	\begin{equation}\label{4.28}
	\|C_N(q)-\bar C_N(q)\|_N^m\le \frac{\lambda_0 \sigma_N^\varphi(\eta)}{1-h\kappa_\varphi\lambda_0},
	\end{equation}
	где
	\begin{equation}\label{4.29}
	\sigma_N^\varphi(\eta)=\left(\sum\nolimits_{\nu=1}^m\int_{0}^1\left(\sum\nolimits_{j=N+1}^\infty \widehat {\eta_\nu}_{1,j}\varphi_{1,j}(t)\right)^2\rho(t) dt\right)^\frac12,
	\end{equation}
	a $C_N(q)=(c_0(q),\ldots,c_{N-1}(q))$ -- конечная последовательность, составленная из первых $N$ компонент точки  $C(q)$.
\end{theorem}


Заметим,  что величина
\begin{equation}\label{4.30}
V_N(\phi,x)=V_N^\varphi(\phi,x)=\phi(x)- Y_{1,N}(\phi,x)
=\sum\nolimits_{j=N+1}^\infty \hat \phi_{1,j}\varphi_{1,j}(x),
\end{equation}
фигурирующая в равенстве \eqref{4.29}, в котором $\phi(x)=\eta_\nu(x)$, представляет собой остаточный член ряда Фурье функции $\phi$ по системе $\varphi_{1,k}(t)$ $(k=0,1,\ldots)$, ортонормированной  по Соболеву относительно скалярного произведения \eqref{2.3}, порожденной  системой $\varphi=\{\varphi_k\}_{k=0}^\infty$ посредством равенств \eqref{2.1} и \eqref{2.2} с $r=1$, $a=0$, $b=1$.  Неравенство \eqref{4.28} непосредственно приводит к  задаче об исследовании поведения величины $\sigma_N^\varphi(\phi)=(\int_{0}^1(\phi(x)- Y_{1,N}(\phi,x))^2 dx)^\frac12$.
При этом следует заметить, что  функции $\varphi_{1,k}(x)$ $(k=0,1,\ldots)$ не образуют ортонормированной системы в $L^2_{\rho}$, поэтому указанная задача не сводится к простому применению равенства Парсеваля или неравенства Бесселя, с помощью которых легко оценивается норма остаточного члена ряда Фурье в гильбертовом пространстве. И тем не менее, для ряда классических систем $\varphi_k(x)$ $(k=0,1,\ldots)$ удается свести поставленную задачу об оценке величины $\sigma_N^\varphi(\phi)$  к задаче об оценке  остаточного члена некоторого  ряда Фурье по системе $\varphi_k(x)$ $(k=0,1,\ldots)$  в пространстве $L^2_{\rho}$.   Еще одна задача, которая непосредственно связана с неравенством \eqref{4.28}, состоит в том, чтобы найти точное значение величины  $\kappa_\varphi$, определенной вторым из равенств \eqref{4.2}, для различных классических ортонормированных систем $\varphi$. В ходе выполнения НИР эти задачи были рассмотрены для таких систем, как система Хаара, система косинусов, система функций Уолша, система полиномов Якоби и других.

В теореме \ref{th3} получена оценка  погрешности, которая проистекает в результате замены искомой матрицы $C_N(q)$, составленной из неизвестных коэффициентов  разложения решения задачи Коши \eqref{4.4}, приближенной матрицей $\bar C_N(q)$, являющейся неподвижной точкой конечномерного оператора $A_N$, сконструированного по правилу \eqref{4.18}.
Как было показано выше, этот вопрос непосредственно приводит к задаче об оценке величины  $\sigma_N^\varphi(\phi)=(\int_{0}^1(\phi(x)- Y_{1,N}(\phi,x))^2 dx)^\frac12$. Другими словами, возникает задача о приближении функции $\phi(t)$ частичными суммами $Y_{1,N}(\phi,t)$  в метрике пространства $L^2_\rho(0,1)$.
Но после того, как  матрица $C_N(q)$ будет найдена с требуемой точностью, немедленно возникнет вопрос о том, какова максимальная по $t\in[0,1]$ погрешность замены точного решения $\eta(t)$ задачи Коши \eqref{4.4} частичной суммой его ряда Фурье \eqref{4.5}, которую мы обозначим через  $Y_{1,N}(\eta,t)$. Как было показано в предыдущем подразделе (см. лемму \ref{lem3}), знание такой погрешности важно для оценки погрешности, проистекающей в результате замены решения задачи Коши для ОДУ (или системы ОДУ) на заданном отрезке  аппроксимирующей функцией, сконструированной по схеме, описанной выше.  Таким образом, возникает задача о приближении  дифференцируемой функции $\eta(t)$ частичными суммами $Y_{1,N}(\eta,t)$ в метрике пространства $C[0,1]$, которая является одной из центральных проблем теории рядов Фурье по ортонормированным по Соболеву системам функций. Как уже отмечалось выше, эта проблема  была предметом исследования  работ  \cite{Shar11, Shar2003, Shar2006, Shar2008, Shar19, Shar18, sharap3, Shar_Dag_Elec, SHII, Shar2017, SharSMJ2017}, \cite{SharIzv2018},  в которых  рассмотрены аппроксимативные свойства смешанных рядов, ассоциированных с некоторыми классическими ортогональными системами.

Мы перейдем к более подробному рассмотрению отмеченных выше задач для некоторых  систем вида $\{\varphi_{r,k}(x)\}$, ортонормированных относительно скалярного произведения типа \eqref{2.3} и порожденных классическими ортонормированными системами $\{\varphi_k(x)\}$ посредством равенств  \eqref{2.1} и \eqref{2.2}.



\section{О существовании и единственности решений ОДУ с разрывной правой частью}

В данном подразделе вводится понятие решения задачи Коши для системы обыкновенных дифференциальных уравнений вида
\begin{equation}\label{discont-1.1}
y'(x)=f(x,y),\quad y(0)=y_0, \quad 0\le x\le 1,
\end{equation}
в которой правая часть  $f=(f_1,\ldots,f_m)$ необязательно непрерывна в области своего определения $G\subset\mathbb{R}^{m+1}$. Рассматриваются задачи о существовании и единственности решения задачи Коши \eqref{discont-1.1}. Для того, чтобы определить понятие  решения задачи Коши для уравнения \eqref{discont-1.1}, введем класс  $AC^m[0,1]$, состоящий из всех абсолютно непрерывных вектор-функций $y=y(x)=(y_1(x),\ldots,y_m(x))$, заданных на $[0,1]$.

\textbf{Определение.} \textit{Вектор-функцию $y\in AC^m[0,1]$ будем называть решением задачи Коши  \eqref{discont-1.1}, если $y(0)=y_0$ и имеет место равенство $y'(x)=f(x,y(x))$ для почти всех $x\in[0,1]$.}

При изучении вопросов, связанных с существованием и единственностью решения задачи Коши  \eqref{discont-1.1} в смысле приведенного определения, ключевую роль играют системы функций, ортонормированные по Соболеву и порожденные заданной системой  $\{\varphi_k(x)\}_{k=0}^\infty$, ортонормированной в весовом пространстве Лебега $L_\rho^2(0,1)$  с весом $\rho=\rho(x)$. 

Мы будем считать, что область $G\subset\mathbb{R}^{m+1}$, в котором определена правая часть этой системы $f(x,y)$ содержит в качестве своего подмножества полосу $[0,1]\times\mathbb{R}^m$. Кроме того будем предполагать, что функция $f(x,y)$ подчиняется нижеследующим  требованиям $A)$ и $B)$:

$A)$ если $y=(y_1,\ldots,y_m)$, где $y_l\in  W^1_{L^2_\rho(0,1)}$ для  $l=1,\ldots,m$, то сложные функции $g_l(x)=f_\nu(x,y(x))$ принадлежат пространству $L^2_\rho(0,1)$, т.е.\\ $\int_0^1g^2_l(x)\rho(x)dx<\infty$ для всех $1\le l\le m$;

$B)$ найдется  такое постоянное число $\delta$, что для произвольных    $u=(u_1,\ldots,\linebreak u_m)$, $v=(v_1,\ldots,v_m)$ таких, что  $u_l,v_l\in W^1_{L^2_\rho(0,1)}$ при $l=1,\ldots,m$  имеет место неравенство (усредненное  условие Липшица)
\begin{equation}\label{discont-3.1}
\sum_{\nu=1}^m \int_0^1[f_\nu(x,u(x))-f_\nu(x,v(x))]^2\rho(x)dx\le \delta^2 \sum_{\nu=1}^m\int_0^1[u_\nu(x)-v_\nu(x)]^2\rho(x)dx.
\end{equation}

Рассмотрим систему ортонормированных по Соболеву функций $\{\varphi_{1,k}(x)\}_{k=0}^\infty$, которые определены  посредством равенств \eqref{2.1} и \eqref{2.2} с $r=1$, $a=0$, $b=1$,  т.е.
\begin{equation}\label{discont-3.2}
\varphi_{1,0}(x)=1, \quad \varphi_{1,k}(x)= \int_0^x\varphi_{k-1}(t)dt,\quad k=1,2,\ldots,
\end{equation}
где $\{\varphi_{k}(x)\}_{k=0}^\infty$ -- ортонормированная система, полная в $L_{\rho}^2$ и такая, что порожденная ею система $\{\varphi_{1,k}(x)\}_{k=0}^\infty$ удовлетворяет  условию
\begin{equation}\label{discont-3.3}
\kappa_\varphi=\left(\sum\nolimits_{k=1}^\infty \int_0^1\varphi^2_{1,k}(x)\rho(x)dx\right)^\frac12<\infty.
\end{equation}
Мы можем   теперь сформулировать следующий результат.
\begin{theorem}\label{discont-th3.3}
	Предположим, что весовая функция $\rho = \rho(x)$ подчиняется условиям теоремы \ref{th2}, и пусть выполнены следующие условия:
	
	1) вектор-функция $f(x,y)$ удовлетворяет условиям $A)$ и $B)$;
	2) $\delta\kappa_\varphi<1$.
	Тогда задача Коши \eqref{discont-1.1} имеет  единственное решение $y=(y_1,\ldots,y_m)$ с компонентами $y_\nu\in W^1_{L^2_\rho(0,1)}$ при всех $1\le\nu\le m$. Оно
	представимо в виде равномерно сходящегося ряда
	\begin{equation}\label{discont-3.4}
	y(x)=y(0)+\sum_{k=1}^\infty y_{1,k} \varphi_{1,k}(x), \quad x\in[0,1],
	\end{equation}
	где
	$$
	y_{1,k}=\int_{0}^{1} y'(t)\varphi_{k-1}(t)dt=\int_{0}^{1} f(t,y(t))\varphi_{k-1}(t)dt.
	$$
\end{theorem}

\chapter{Полиномы, ортогональные по Соболеву, ассоциированные с полиномами Чебышева первого рода, и задача Коши для ОДУ}

%В данном разделе рассмотрены полиномы $T_{r,n}(x)$ $(n=0,1,\ldots)$, порожденные многочленами Чебышева $T_{n}(x)=\cos( n\arccos x)$, образующие ортонормированную систему по Соболеву относительно скалярного произведения вида $\langle f,g\rangle=\sum_{\nu=0}^{r-1}f^{(\nu)}(-1)g^{(\nu)}(-1)+\int_{-1}^{1}f^{(r)}(x)g^{(r)}(x)\mu(x)dx$, где $\mu(x)=\frac2\pi(1-x^2)^{-\frac12}$. Показано, что суммы Фурье по полиномам $T_{r,n}(x)$ $(n=0,1,\ldots)$ являются удобным и весьма эффективным инструментом приближенного решения задачи Коши для обыкновенных дифференциальных уравнений (ОДУ).


\section{Введение}

Как уже отмечалось, (см. также \cite{Shar11, Shar12, Shar13, Shar14, Shar2006, Shar2008, Shar2003, Shar18}), частичные суммы смешанных рядов по классическим ортогональным полиномам, в отличие от сумм Фурье по этим же полиномам, успешно могут быть использованы в задачах, в которых требуется одновременно приближать дифференцируемую функцию и ее несколько производных. В частности, такие задачи непосредственно возникают, например, в связи с решением краевых задач для дифференциальных уравнений спектральными методами.

В данном разделе мы рассмотрим систему полиномов $T_{r,k}(x)$,  $(k=0,1,\ldots)$, определенных равенствами
\begin{equation}\label{du2018cheb-1.4}
T_{r,k}(x) =\frac{(x+1)^k}{k!}, \quad k=0,1,\ldots, r-1,
\end{equation}
\begin{equation}\label{du2018cheb-1.5}
T_{r,r}(x) =\frac{(x+1)^r}{\sqrt{2}r!},\quad T_{r,r+n}(x) =\frac{1}{(r-1)!}\int_{-1}^x(x-t)^{r-1}T_{n}(t)dt, \quad n=1,\ldots,
\end{equation}
ортонормированных по Соболеву относительно скалярного произведения
\begin{equation}\label{du2018cheb-1.3}
\langle f,g\rangle=\sum_{\nu=0}^{r-1}f^{(\nu)}(-1)g^{(\nu)}(-1)+\int_{-1}^{1}f^{(r)}(t)g^{(r)}(x)\mu(x)dx,
\end{equation}
где $\mu(x)=\frac2\pi(1-x^2)^{-\frac12}$, и ряды Фурье по этим полиномам. Особое внимание мы уделим вопросу о представлении  решения задачи Коши для ОДУ  в виде ряда Фурье по системе $\{T_{r,k}(x)\}_{k=0}^\infty$  и об аппроксимации указанного решения их частичными суммами.

\section{Некоторые свойства полиномов $T_{r,k}(x)$ и рядов Фурье по ним}
Через $L^p_\mu$ обозначим пространство  функций $f$, измеримых  на  $(-1,1)$, для которых
$$\int_{-1}^1|f(x)|^p\mu(x)dx<\infty.$$
$L^p_\mu$ представляет собой банахово пространство с нормой $\|f\|_{p,\mu}=(\int_{-1}^1|f(x)|^p\mu(x)dx)^\frac1p$. Из неравенства Гельдера вытекает, что $L^2_\mu\subset L$, где $L$ -- пространство всех суммируемых функций, заданных на $(-1,1)$. Полиномы Чебышева
\begin{equation}\label{du2018cheb-3.1}
T_0(x)=\frac{1}{\sqrt{2}},\quad T_k(x)=\cos(k\arccos x), \quad k=1,2,\ldots
\end{equation}
образуют ортонормированную  в $L_\mu^2$ с весом  $\mu(x)=\frac2\pi(1-x^2)^{-\frac12}$ систему, которую обозначим через $T$. Как хорошо известно \cite{Sege}, система $T$ полна в $L_\mu^2$.   Для произвольного натурального $r$ эта система порождает  систему полиномов $T_{r,k}(x)$ $(k=0,1,\ldots)$, определенных равенствами \eqref{du2018cheb-1.4} и \eqref{du2018cheb-1.5}. При изучении свойств системы $\{T_{r,k}(x)\}$ нам понадобится весовое пространство Соболева $W^r_{L^p_\mu}$, состоящее из функций $f$, непрерывно дифференцируемых на $[-1,1]$ $r-1$ раз, причем $f^{(r-1)}$ абсолютно непрерывна на $[-1,1]$  и $f^{(r)}\in L^p_\mu$.
Скалярное произведение в пространстве $W^r_{L^2_\mu}$ определим с помощью равенства
\eqref{du2018cheb-1.3}, превратив тем самым его в гильбертово пространство. Из теоремы \ref{th1}, в частности, вытекает

\begin{theorem}\label{du2018cheb-thC}
	Система полиномов $\{T_{r,k}(x)\}_{k=0}^\infty$, порожденная системой ортонормированных полиномов Чебышева \eqref{du2018cheb-3.1} посредством равенств \eqref{du2018cheb-1.4} и \eqref{du2018cheb-1.5}, полна  в $W^r_{L^2_\mu}$ и ортонормирована относительно скалярного произведения \eqref{du2018cheb-1.3}.
\end{theorem}

Ряд Фурье функции $f\in W_{L_\mu^1}^r$ по системе   $\{T_{r,k}(x)\}_{k=0}^\infty$ имеет вид
\begin{equation}\label{du2018cheb-3.2}
f(x)\sim \sum_{k=0}^{r-1} f^{(k)}(-1)\frac{(x+1)^k}{k!}+ \sum_{k=r}^\infty f_{r,k}T_{r,k}(x),
\end{equation}
где
\begin{equation}\label{du2018cheb-3.3}
f_{r,r+j}=\int_{-1}^1 f^{(r)}(t)T_{j}(t)\mu(t)dt\quad(j\ge0).
\end{equation}
В ходе выполнения НИР получен следующий результат, который значительно усиливает теорему \ref{th2} для системы  $\{T_{r,k}(x)\}_{k=0}^\infty$.

\begin{theorem}\label{du2018cheb-thD}
	Если $r\ge1$ и  $f\in W^r_{L^1_\mu}$, то ряд Фурье (смешанный ряд) \eqref{du2018cheb-3.2} сходится к функции $f(x)$ равномерно относительно $x\in[-1,1]$.
\end{theorem}
Этот результат окончателен в том смысле, что при определении смешанного ряда \eqref{du2018cheb-3.2} требуется выполнение условия $f\in W^r_{L^1_\mu}$, которое необходимо для существования коэффициентов $f_{r,k}$, определенных равенствами \eqref{du2018cheb-3.3}.

В доказательстве приведенных далее теорем важное значение имеет следующий результат из \cite{SharIzv2018}.
\begin{lemma}\label{du2018cheb-lemA}
	Имеют место равенства
	\begin{equation}\label{du2018cheb-3.4}
	T_{1,k+1}(x)={T_{k+1}(x)\over2(k+1)}- {T_{k-1}(x)\over2(k-1)} -\frac{(-1)^k}{k^2-1}\quad (k\ge 2),
	\end{equation}
	\begin{equation}\label{du2018cheb-3.5}
	T_{1,0}(x)=1, \quad T_{1,1}(x)=\frac{1+x}{\sqrt{2}}, \quad T_{1,2}(x)=\frac12(x^2-1).
	\end{equation}
\end{lemma}
По поводу равенства \eqref{du2018cheb-3.4} мы можем отослать также к  \cite{Pash}.

Рассмотрим величину
\begin{equation}\label{du2018cheb-3.6}
\kappa_T=\left(\int_{-1}^1\sum_{k=1}^{\infty}(T_{1,k}(x))^2\mu(x)dx\right)^{\frac12},
\end{equation}
которая играет ключевую роль при исследовании вопроса о  сходимости итерационного процесса, конструируемого ниже с целью приближенного решения задачи Коши для    нелинейного ОДУ путем представления ее решения в виде ряда Фурье по полиномам $T_{1,n}(x)$. Лемма \ref{du2018cheb-lemA} позволяет найти точное значение этой величины.
\begin{theorem}\label{du2018cheb-th1}
	Имеет место равенство
	$$
	\kappa_ T=\left(\frac{11}{8}+\frac{\pi^2}{12}+\sum_{k=2}^\infty\frac{2}{(k^2-1)^2}\right)^\frac12.
	$$
\end{theorem}

Положим $r=1$ и обозначим через $Y_{1,N}(f,x)$ частичную cумму ряда \eqref{du2018cheb-3.2} вида
\begin{equation}\label{du2018cheb-3.12}
Y_{1,N}(f,x)= f(-1)+ \sum\nolimits_{k=1}^N  f_{1,k}T_{1,k}(x).
\end{equation}
Заметим,  что величина
\begin{equation}\label{du2018cheb-3.13}
V_N(f,x)=f(x)- Y_{1,N}(f,x)
=\sum\nolimits_{j=N+1}^\infty  f_{1,j}T_{1,j}(x),
\end{equation}
которая часто будет встречаться в дальнейшем,   представляет собой остаточный член ряда Фурье функции $f$ по полиномам $T_{1,k}(t)$ $(k=0,1,\ldots)$. Поскольку из \eqref{du2018cheb-1.5} следует, что $T_{1,k}(0)=0$ при $k\ge1$, то справедливы равенства
\begin{equation}\label{du2018cheb-3.14}
Y_{1,N}(f,-1)= f(-1),\quad V_N(f,-1)=0.
\end{equation}

Непосредственно из определения \eqref{du2018cheb-1.3} следует, что $(T_{1,k+1}(x))'=T_k(x)$, $k=0,1,\ldots$,
поэтому почленное дифференцирование ряда \eqref{du2018cheb-3.2} c $r=1$ приводит к следующему соотношению
$f'(x)\sim f_{1,1}/\sqrt{2} + \sum_{k=1}^\infty f_{1,k+1}T_k(x)$, в правой части которого фигурирует ряд Фурье --- Чебышева функции $f'$. Если при этом $f\in W^1_{L^2_\mu}$, то в силу полноты системы
\eqref{du2018cheb-3.1} в $L^2_\mu$ имеет место равенство
\begin{equation}\label{du2018cheb-3.15}
f'(x)= \frac{f_{1,1}}{\sqrt{2}} + \sum_{k=1}^\infty f_{1,k+1}T_k(x),
\end{equation}
где равенство понимается в смысле сходимости ряда \eqref{du2018cheb-3.15} в метрике пространства $L^2_\mu$.
\begin{theorem}\label{du2018cheb-th2}
	Если  $-1\le x\le1$, $1\le l$, то имеют место неравенства
	\begin{equation}\label{du2018cheb-3.16}
	|T_{1,2l}(x)|\le \frac{1}{l}(1-x^2)^\frac12,
	\end{equation}
	\begin{equation}\label{du2018cheb-3.17}
	|T_{1,2l+1}(x)|\le\frac{\sqrt{1+x}}{l+1/2}\left(\sqrt{1-x}+\frac{\sqrt{2}}{2l-1}\right).
	\end{equation}
\end{theorem}

\section{ Некоторые сведения о полиномах Якоби}

Для удобства ссылок мы соберем здесь необходимые для дальнейшего
свойства полиномов Якоби $P_n^{\alpha,\beta}(x)$, которые могут быть
определены~\cite{Sege}  с помощью формулы Родрига
\begin{equation}\label{du2018cheb-4.1}
P_n^{\alpha,\beta}(x) = {(-1)^n\over2^nn!}{1\over\kappa(x)}{d^n\over
	dx^n} \left\{\kappa(x)\sigma^n(x)\right\},
\end{equation}
где $\alpha,\beta$ -- произвольные действительные числа, $\kappa(x)=
(1-x)^\alpha(1+x)^\beta,\,\,\sigma(x)=1-x^2$. Если
$\alpha,\beta>-1$, то полиномы Якоби \eqref{du2018cheb-4.1} образуют ортогональную
систему с весом $\kappa(x)$, т.е.
\begin{equation}\label{du2018cheb-4.2}
\int_{-1}^1P_n^{\alpha,\beta}(x)P_m^{\alpha,\beta}(x)\kappa(x)dx =
h_n^{\alpha,\beta}\delta_{nm},
\end{equation}
где
\begin{equation}\label{du2018cheb-4.3}
h_n^{\alpha,\beta} =
{\Gamma(n+\alpha+1)\Gamma(n+\beta+1)2^{\alpha+\beta+1} \over
	n!\Gamma(n+\alpha+\beta+1)(2n+\alpha+\beta+1)}.
\end{equation}
Ниже нам понадобятся еще следующие свойства полиномов
Якоби~\cite{Sege}:


\textit{обобщенная формула Родрига}
\begin{equation}\label{du2018cheb-4.4}
\kappa(x) P_n^{\alpha,\beta}(x)
={(-1)^m\over2^mn^{[m]}}{d^m\over dx^m}
\left\{(1-x)^{m+\alpha}(1+x)^{m+\beta} P_{n-m}^{m+\alpha,m+\beta}(x)
\right\},
\end{equation}
где $k^{[0]}=1$, $k^{[r]}=k(k-1)\dots(k-r+1)$;

\textit{связь с полиномами Чебышева $T_n(x)=\cos(n\arccos x)$}
\begin{equation}\label{du2018cheb-4.5}
\left(h_n^{-\frac12,-\frac12}\right)^{-\frac12}P_n^{-\frac{1}{2},-\frac{1}{2}}(x)
=\sqrt{2/\pi}T_n(x)\quad (n\ge1).
\end{equation}



\section{Равномерное приближение на $[-1,1]$ дифференцируемых функций частичными суммами Фурье по полиномам $T_{1,k}(x)$ }
Рассмотрим вопрос об оценке уже упоминавшейся выше величины $|V_N(f,x)|$ $=|f(x)- Y_{1,N}(f,x)|$, определенной равенством \eqref{du2018cheb-3.13}. При этом нам понадобятся некоторые обозначения. Для натурального $s$ положим $\nu(x)=\nu_s(x)=(1-x^2)^{s-3/2}$ и обозначим через $L_{\nu}^2$ пространство измеримых функций $f$, для которых $\int_{-1}^1f^2(x)\nu(x)dx<\infty$, а соответствующее весовое пространство Соболева,  состоящее из функций $f$, непрерывно дифференцируемых на $[-1,1]$ $(r-1)$-раз, причем $f^{(r-1)}$ абсолютно непрерывна на $[-1,1]$  и $f^{(r)}\in L^2_{\nu}$, обозначим через $W^s_{L^2_{\nu}}$. Через $E_n(f)_{L^2_{\nu}}$  обозначим наилучшее приближение функции
$f\in L^2_{\nu}$ алгебраическими полиномами степени $n$.

\begin{theorem}\label{du2018cheb-th3}    Пусть $N\ge s\ge1$, $f\in W_{L^2_{\nu}}^s$, $x\in[-1,1]$. Тогда имеет  место оценка
	$$
	|f(x)- Y_{1,N}(f,x)|\le c(s) {E_{N-s}(f^{(s)})_{L^2_\nu}\over N^{s-1/2}}
	\left((1-x)^\frac12+\frac1{N+1}\right)(1+x)^\frac12.
	$$
\end{theorem}


\section{Равномерное приближение на $[-1,1]$ аналитических функций частичными суммами Фурье по полиномам $T_{1,k}(x)$}
Пусть $0<q<1$, $\mathcal{E}_q$ -- эллипс с фокусами в точках
$-1$ и $1$, сумма полуосей которого равна $1/q$.
Через $A_q(B)$ мы обозначим класс функций $f=f(z)$,
принимающих действительные значения на $[-1,1]$, аналитических
внутри эллипса $\mathcal{E}_q$ и ограниченных там по модулю числом
$B$. Хорошо известно \cite{Timan}, что если
$f\in A_q(B)$, то для коэффициентов Фурье --- Чебышева этой функции
\begin{equation}\label{du2018cheb-6.1}
a_k(f)=\frac{2}{\pi}\int_{-1}^{1}\frac{f(t)T_k(t)}{\sqrt{1-t^2}}dt
\end{equation}
имеет место оценка
\begin{equation}\label{du2018cheb-6.2}
|a_k(f)|\le2Bq^k,
\end{equation}
и как следствие
\begin{equation}\label{du2018cheb-6.3}
E_N(f)_{C[-1,1]}\le{2B\over1-q}q^{N+1},
\end{equation}
где $E_N(f)_{C[-1,1]}$ -- наилучшее приближение функции $f\in C[-1,1]$ алгебраическими полиномами степени $N$. Как показано в \cite{du2018cheb-Ahiezer}, эта оценка на всем классе  $A_q(B)$ не улучшаема по порядку.  Ниже мы  покажем, что среди алгебраических полиномов степени $N$ частичная сумма $Y_{1,N}(f,x)$ ряда Фурье функции $f$ по полиномам $T_{1,k}(x)$    доставляет для $f$ приближение в $C[-1,1]$ наилучшего на классе $A_q(B)$ порядка \eqref{du2018cheb-6.3}. Нам понадобится следующее    вспомогательное утверждение, доказательство которого можно найти  в \cite{Pash}.
\begin{lemma}\label{du2018cheb-lemC} Если $f\in A_q(B)$, то имеет место неравенство
	$$
	a_k(f')=2\sum_{j=0}^\infty(k+2j+1)a_{k+2j+1}(f).
	$$
\end{lemma}

\begin{theorem} \label{du2018cheb-th4}    Пусть  $f\in A_q(B)$ , $x\in[-1,1]$. Тогда имеет  место неравенство
	$$
	|f(x)- Y_{1,N}(f,x)|\le
	$$
	$$
	\frac{8Bq^{N+1}\sqrt{1+x}}{(1-q^2)(1-q)}\left[\sqrt{1-x}+\frac{3\sqrt{2}}{N+1}\right]\left[1+
	\frac{2q^2}{(1-q^2)(N+1)}\right].
	$$
\end{theorem}

\input content/opt-algan.tex

\section{Замечание о численных экспериментах}
Компьютерные эксперименты, проведенные исполнителями НИР, показали \cite{MMG2016}, \cite{du2018cheb-SHII-MMG2018}  весьма высокую эффективность метода приближенного решения ОДУ вида \eqref{algan-3.1} и их систем, основанного на представлении его решения в виде ряда Фурье по функциям, ортогональным по Соболеву, порожденным классическими ортогональными системами. Среди таких систем особое место занимает система полиномов $T_{1,k}(x)$ $(k=0,1,\ldots)$, порожденных полиномами Чебышева   $T_k(x)=\cos(k\arccos x)$ посредством равенств \eqref{du2018cheb-1.4} и \eqref{du2018cheb-1.5}. Дело в том, что в этом случае основные этапы описанного выше алгоритма решения задачи Коши для ОДУ удается численно реализовать с помощью быстрого дискретного косинусного преобразования Фурье. Это особенно важно при осуществлении итерационного процесса, направленного на нахождение приближенных значений неизвестных коэффициентов $\varphi_{1,j}$ разложения искомого решения по системе $\{T_{1,k}(x)\}_{k=0}^\infty$. Отметим также работы
\cite{du2018cheb-Arush2010, du2018cheb-Arush2013, du2018cheb-Arush2014}, в которых для приближенного решения уравнений вида \eqref{algan-3.1} применяется  метод, основанный на разложении искомого решения уравнения \eqref{algan-3.1} в ряд Фурье по самим полиномам Чебышева
$T_k(x)=\cos(k\arccos x)$, а не по порожденным ими полиномам $\{T_{1,k}(x)\}_{k=0}^\infty$. В этом случае возникают трудности, связанные учетом начальных условий задачи Коши. Тем не менее в \cite{du2018cheb-Arush2010, du2018cheb-Arush2013, du2018cheb-Arush2014} на конкретных примерах продемонстрированы явные преимущества спектрального метода решения задачи Коши \eqref{algan-3.1}, основанного на базисе полиномов Чебышева $T_k(x)=\cos(k\arccos x)$, по сравнению c сеточными методами решения этой задачи, такими как метод Рунге --- Кутты, метод Адамса и другие. Проведенные нами компьютерные эксперименты, которые упоминались выше, приводят к аналогичным выводам при сопоставлении результатов, полученных при приближенном  решении конкретных систем уравнений типа \eqref{algan-3.1} методом, описанным выше, с классическими сеточными методами.




\chapter{Ортонормированная по Соболеву система функций, ассоциированная с косинусами}

\section{Ортонормированная по Соболеву  система, порожденная косинусами в случае $r=1$}
Рассмотрим систему функций
\begin{equation}\label{5.1}
\xi_0(x)=1,\quad \{\xi_n(t)=\sqrt{2}\cos(\pi nt)\}_{n=1}^\infty,
\end{equation}
которая является ортонормированной относительно скалярного произведения $\langle f,g \rangle=\int_0^1f(t)g(t)dt$, т. е.
\begin{equation}\label{5.2}
\langle\xi_n,\xi_m\rangle=\int_0^1\xi_n(t)\xi_m(t)dt=\delta_{nm}.
\end{equation}
Соответствующая ей порожденная система с $r=1$ имеет вид
$$
\xi_{1,0}(t)=1,\quad \xi_{1,n}(t)=\int_0^t \xi_{n-1}(\tau)d\tau, \quad n=1,2,\ldots,
$$
т. е.
\begin{equation}\label{5.3}
\xi_{1,0}(t)=1,\quad \xi_{1,1}(t)=t,\quad \xi_{1,n+1}(t)=\frac{\sqrt{2}}{\pi n}\sin(\pi nt),\quad n=1,2,\ldots.
\end{equation}
В силу теоремы \ref{th2} система \eqref{5.3} является полной в $W^1_{L^2(0,1)}$ и ортонормированной относительно скалярного произведения $\langle f,g\rangle=f(0)g(0)+\int_0^1 f'(t)g'(t)dt$:
\begin{equation*}
\langle\xi_{1,n},\xi_{1,m}\rangle=\xi_{1,n}(0)\xi_{1,m}(0)+\int_0^1\xi'_{1,n}(t)\xi'_{1,m}(t)dt=\delta_{nm}.
\end{equation*}

\begin{theorem}\label{th4}
	Пусть $\xi=\{\xi_k(t)\}_{k=0}^\infty$ -- система  \eqref{5.1}, для которой  \eqref{5.3} является порожденной. Тогда для величин $\delta_{\xi}(t)$ и    $\kappa_\xi$, определенных равенствами \eqref{4.2},  имеют место соотношения
	$$
	\delta_{\xi}(t)=t,\quad\kappa_{\xi}=\sqrt{1/2}.
	$$
\end{theorem}


Произвольной функции $\phi\in W^1_{L^1(0,1)}$ мы можем сопоставить ее ряд Фурье по функциям \eqref{5.3}, который имеет вид
\begin{equation}\label{5.4}
\phi(t)=\phi(0)+\sum_{j=1}^\infty \hat\phi_{1,j}\xi_{1,j}(t),
\end{equation}
где
$$
\hat\phi_{1,1}=\phi(1)-\phi(0),\quad
\hat\phi_{1,j+1}=\sqrt{2}\int_{0}^{1}\phi'(t)\cos(j\pi t)dt \quad(j\ge1).
$$


Заметим, что для  для построения смешанного ряда \eqref{5.4} требуется, чтобы функция $\phi$ была абсолютно непрерывной на $[0,1]$, поэтому следующая теорема носит окончательный характер.
\begin{theorem}\label{th5}
	Если $\phi\in W^1_{L^1(0,1)}$, то ряд Фурье (смешанный ряд) \eqref{5.4} сходится к функции $\phi(t)$ равномерно относительно $t\in[0,1]$.
\end{theorem}

\section{Аппроксимативные свойства сумм Фурье по системе $\{\xi_{1,n}\}_{n=0}^\infty$ и некоторые следствия}
Вернемся к задаче об исследовании поведения остаточных членов ряда Фурье \eqref{5.4}   вида $V_N(t)=V_N(\phi,t)=\sum\nolimits_{j=N+1}^\infty \hat \phi_{1,j}\xi_{1,j}(t)$, фигурирующих в правой части неравенства \eqref{4.30}. Нетрудно показать, что ряд \eqref{5.4} допускает его преобразование к виду
\begin{equation}\label{5.7}
\bar\phi(t)=\phi(t)-\phi(0)-(\phi(1)-\phi(0))t= \sum\nolimits_{k=1}^\infty b_k\sin(\pi kt),
\end{equation}
где
\begin{equation}\label{5.8}
b_k=2\int_{0}^1 \bar\phi(\tau)\sin(\pi k\tau)d\tau=\int_{-1}^1 \bar\phi(\tau)\sin(\pi k\tau)d\tau.
\end{equation}
Отсюда имеем
\begin{equation}\label{5.9}
V_N(\phi,t)=\phi(t)-Y_{1,N}(\phi,t)= \sum\nolimits_{k=N}^\infty b_k\sin(\pi kt),
\end{equation}
где  $Y_{1,N}(\phi,t)$ частичная сумма ряда \eqref{5.4} вида
$$
Y_{1,N}(\phi,t)= \phi(0)+\phi_{1,k}\xi_{1,1}(t)+ \sum\nolimits_{k=2}^N \hat \phi_{1,k}\xi_{1,k}(t)
$$
\begin{equation}\label{5.10}
= \phi(0)+(\phi(1)-\phi(0))t+\sum\nolimits_{k=1}^{N-1} b_k\sin(\pi kt).
\end{equation}
Рассмотрим частичную сумму
\begin{equation}\label{5.11}
S_{N-1}(\bar\phi,t)= \sum\nolimits_{k=1}^{N-1} b_k\sin(\pi kt)
\end{equation}
тригонометрического ряда Фурье \eqref{5.7} для функции $\bar\phi(x)$ и заметим, что в силу \eqref{5.10} и \eqref{5.11} имеем
\begin{equation}\label{5.12}
V_N(\phi,t)= \sum\nolimits_{k=N}^{\infty} b_k\sin(\pi kt)=\bar\phi(t)-S_{N-1}(\bar\phi,t).
\end{equation}
Отсюда мы замечаем, что задача об исследовании поведения величины $V_N(\phi,t)$ сводится к аналогичной задаче для остатка ряда Фурье
\begin{equation}\label{5.13}
R_N(\bar\phi,t)= \bar\phi(t)-S_{N-1}(\bar\phi,t).
\end{equation}
В частности, для величины $\|V_N(\phi)\|_{L^2(0,1)}$, фигурирующей в равенстве \eqref{4.29}, имеем
\begin{equation}\label{5.14}
\|V_N(\phi)\|_{L^2(0,1)}=\left(\int_{0}^1\left(\sum\nolimits_{j=N}^\infty \hat \phi_{1,j+1}\xi_{1,j+1}(t)\right)^2 dt\right)^\frac12=\|R_N(\bar\phi)\|_{L^2(0,1)}.
\end{equation}
Обозначим через $E_n(\bar\phi)_2$ наилучшее приближение функции $\bar\phi$ в $L^2(-1,1)$ тригонометрическими полиномами вида $T_n(t)=a_0+\sum_{k=1}^{n}a_k\cos(\pi kt)+b_k\sin(\pi kt)$. Поскольку $S_{N-1}(\bar\phi,t)$ является полиномом наилучшего приближения к функции $\bar\phi$ в $L^2(-1,1)$, то из \eqref{5.13}, с учетом того, что функция $\bar\phi$ нечетная, имеем
\begin{equation}\label{5.15}
\|V_N(\phi)\|_{L^2(0,1)}=\frac12\|R_N(\bar\phi)\|_{L^2(-1,1)}=\frac12E_{N-1}(\bar\phi)_2.
\end{equation}


Для вектор-функции $\bar\eta=(\bar\eta_1,\ldots,\bar\eta_m)$ положим
$$
E_ {N-1}(\bar\eta)_2=\left(\sum\nolimits_{\nu=1}^m [E_ {N-1}(\bar\eta_\nu)_2]^2\right)^\frac12.
$$

Из теоремы \ref{th4} и равенства \eqref{5.15} непосредственно вытекает
\begin{corollary}\label{cor1}Пусть вектор-функция $\eta=(\eta_1,\ldots,\eta_m)$ представляет собой решение задачи Коши для системы ОДУ \eqref{4.4} и выполнены условия теоремы \ref{th3}. Тогда имеет место неравенство
	\begin{equation}\label{5.16}
	\|C_N(q)-\bar C_N(q)\|_N\le \frac{\lambda_0 E_{N-1}(\bar\eta)_2} {\sqrt{2}(\sqrt{2}-h\lambda_0)},
	\end{equation}
	где $\bar\eta_\nu(t)=\eta_\nu(t)-\eta_\nu(0)-(\eta_\nu(1)-\eta_\nu(0))t$.
\end{corollary}

Рассмотрим  задачу о приближении  непрерывной функции $\phi(t)$ частичными суммами $Y_{1,N}(\phi,t)$ в метрике пространства $C[0,1]$, другими словами, ставится задача об исследовании поведения величины
\begin{equation}\label{5.17}
\|V_N(\phi)\|_{C[0,1]}=\max_{t\in[0,1]}|\phi(t)-Y_{1,N}(\phi,t)|.
\end{equation}
Из \eqref{5.12} мы можем заметить, что эта задача совпадает с задачей о приближении 2-периодической нечетной непрерывной функции $\bar \phi(t)$ тригонометрическими полиномами вида $T_{N-1}(t)=a_0+\sum_{k=1}^{N-1}a_k\cos(\pi kt)+b_k\sin(\pi kt)$ в метрике пространства  $C[-1,1]$. Обозначим через $E_{N-1}(\bar\phi)$ наилучшее приближение функции $\bar\phi$ тригонометрическими полиномами $T_{N-1}(t)$. Тогда, используя стандартные методы теории приближений,  имеем
\begin{equation}\label{5.18}
\|V_N(\phi)\|_{C[0,1]}=\max_{t\in[-1,1]}|\bar\phi(t)-S_{N-1}(\bar\phi,t)|\le (1+L_{N-1})E_{N-1}(\bar\phi),
\end{equation}
где $L_{N-1}$ -- постоянная Лебега для частичных сумм Фурье $S_{N-1}(\bar\phi,t)$, для которой, как хорошо известно \cite{Dzjadyc}, имеет место неравенство
\begin{equation}\label{5.19}
L_{N-1}\le \frac{4}{\pi^2}\ln(N-1)+3.
\end{equation}
Сопоставляя \eqref{5.18} и  \eqref{5.19}, мы убеждаемся в справедливости следующего утверждения.
\begin{corollary}\label{cor2} Имеет место следующее неравенство
	\begin{equation}\label{5.20}
	\|V_N(\phi)\|_{C[0,1]}\le \left(4+\frac{4}{\pi^2}\ln(N-1)\right)E_{N-1}(\bar\phi).
	\end{equation}
\end{corollary}

Рассмотрим задачу Коши \eqref{4.1} ещё раз.  Произведем замену переменной $x=h\sin\pi t$ и положим $\eta(t)=y(h\sin\pi t)-y^0$. Тогда относительно новой переменной $t$ система \eqref{4.1} принимает вид
\begin{equation}\label{5.21}
\eta'(t)=hF(t,\eta), \quad \eta(0)=0,\quad 0\le t\le1,
\end{equation}
где $F(t,\eta)=\pi f(h\sin \pi t,y^0+\eta)\cos\pi t$, причем $\eta(1)=y(h\sin\pi)-y^0=y(0)-y^0=0$. Из \eqref{4.3} для $F(t,\eta)$ вытекает условие Липшица вида
\begin{equation*}
\|F(t,a)-F(t,b)\|\le \bar\lambda\|a-b\|, \quad 0\le t \le 1, \quad\bar\lambda=\pi\lambda_0.
\end{equation*}
Мы можем  в отношении системы уравнений \eqref{5.21} повторить схему рассуждений, описанную в подразделе \ref{CauchyProblemSolutionRepr}  для приближения решения системы ОДУ \eqref{4.4}, в которых вместо $\lambda_0$  фигурирует $\bar\lambda$.  Существенная разница между этими двумя системами уравнений заключается в том, что решение $\eta(t)$ задачи Коши \eqref{5.21} можно   2-периодически продолжить на всю ось $\mathbb{R}$ так, что продолженная вектор-функция $\eta(t)$ будет нечетной и непрерывно дифференцируемой, в частности, как уже отмечалось,  $\eta(0)=\eta(1)=0$ и, как следствие, $\hat\eta_{1,1}=0$. Поэтому ее разложение  в ряд Фурье по системе \eqref{5.3} приобретает вид
\begin{equation}\label{5.22}
\eta(t)=\sum_{j=2}^\infty \hat\eta_{1,j}\xi_{1,j}(t)=\sqrt{2} \sum\nolimits_{k=1}^\infty \hat \eta_{1,k+1}\frac{\sin(\pi kt)}{\pi k},
\end{equation}
где согласно  \eqref{4.6} и \eqref{5.1}
\begin{equation}\label{5.23}
\hat \eta_{1,k+1}=(\widehat{\eta_1}_{1,k+1},\ldots,\widehat{\eta_m}_{1,k+1})=\sqrt{2}\int_{0}^1 \eta'(\tau)\cos(\pi k\tau)d\tau\quad(k\ge1).
\end{equation}
Полагая
\begin{equation}\label{5.24}
b_k(\eta)=2\int_{0}^1 \eta(\tau)\sin(\pi k\tau)d\tau=\int_{-1}^1 \eta(\tau)\sin(\pi k\tau)d\tau,
\end{equation}
мы можем придать равенству \eqref{5.22}  еще такой вид
\begin{equation}\label{5.25}
\eta(t)=\sum\nolimits_{k=1}^\infty b_k(\eta)\sin(\pi kt).
\end{equation}
Отсюда видно, что отмеченные выше задачи об исследовании в пространствах $L^2(0,1)$ и $C[0,1]$ поведения при $N\to\infty$  величины $ V_N(\eta_\nu,t)= \eta_\nu(t)-Y_{1,N}(\eta_\nu,t)=\sum_{j=N+1}^\infty \widehat{\eta_\nu}_{1,j}\xi_{1,j}(t)$ для вектор -функции $\eta(t)$, представляющей собой решение задачи Коши \eqref{5.21},     сводятся  к исследованию в этих же пространствах поведения при $N\to\infty$ остаточного члена
\begin{equation}\label{5.26}
R_N(\eta_\nu,t)= \sum\nolimits_{k=N}^\infty b_k(\eta_\nu)\sin(\pi kt)
\end{equation}
тригонометрического ряда Фурье  функции $\eta_\nu(t)$, которая, как уже отмечалось, является нечетной 2-периодической и непрерывно дифференцируемой на всей оси $\mathbb{R}$. Но это -- детально исследованная классическая задача. В частности, если мы положим $\phi(t)=\eta_\nu(t)$, то $\bar \phi(t)=\phi(t)-\phi(0)-(\phi(1)-\phi(0))t=\phi(t)$, то в утверждении следствия  \ref{cor2} вместо  $\bar \phi$ можно взять $\phi$. Утверждение следствия \ref{cor1} также справедливо с заменой  $\bar\eta(t)$  на $\eta(t)$, а   $\lambda_0$ на $\bar\lambda$.

\section{Ортонормированная по Соболеву  система, порожденная косинусами в случае $r>1$}

Пусть $1<r$ -- натуральное. Рассмотрим систему функций $\{\xi_{r,n}\}_{n=0}^\infty$, определенных равенствами
\begin{equation}\label{5.27}
\xi_{r,r+k}(x) =\frac{1}{(r-1)!}\int_0^x(x-t)^{r-1}\xi_{k}(t)dt, \quad k=1,2,\ldots,
\end{equation}
\begin{equation}\label{5.28}
\xi_{r,k}(x) =\frac{x^k}{k!}, \quad k=0,1,\ldots, r.
\end{equation}
В силу теоремы \ref{th1} система $\{\xi_{r,n}(t)\}_{n=0}^\infty$ является полной в $W^r_{L^2(0,1)}$ и ортонормированной относительно скалярного произведения $\langle f,g\rangle=\sum_{\nu=0}^{r-1}f^{(\nu)}(0)g^{(\nu)}(0)+\int_0^1 f^{(r)}(t)g^{(r)}(t)dt$, т. е.
\begin{equation*}
\langle\xi_{r,n},\xi_{r,m}\rangle=
\sum_{\nu=0}^{r-1}(\xi_{r,n}(t))^{(\nu)}(0)(\xi_{r,m}(t))^{(\nu)}(0)
+\int_0^1\xi^{(r)}_{r,n}(t)\xi^{(r)}_{r,m}(t)dt=\delta_{nm}.
\end{equation*}
Из \eqref{5.27} нетрудно вывести следующее равенство $(k=1,2,\ldots)$
\begin{equation}\label{5.29}
\xi_{r,r+k}(t) =\frac{(-1)^r\sqrt{2}}{(\pi k)^r}\left[
\cos\left(\pi kt+\frac{\pi r}{2}\right)-\sum_{\nu=0}^{r-1} \frac{(\pi kt)^\nu}{\nu!}\cos^{(\nu)}(\pi r/2)\right].
\end{equation}
В частности, если $r=2$, то система $\{\xi_{r,n}(t)\}_{n=0}^\infty$, ортонормированная по Соболеву относительно скалярного произведения $\langle f,g\rangle=\sum_{\nu=0}^1f^{(\nu)}(0)g^{(\nu)}(0)+\int_0^1 f''(t)g''(t)dt$,
имеет вид
\begin{equation}\label{5.30}
\xi_{2,0}(t) =1, \quad \xi_{2,1}(t)=t, \quad \xi_{2,2}(t)=\frac{t^2}2,\,\,\left\{ \xi_{2,k+2}(t)=  \frac{2\sqrt{2}}{(k\pi)^2}\sin^2\frac{k\pi t}{2}\right\}_{k=1}^\infty.
\end{equation}



%======================================
\chapter{Ортогональные по Соболеву полиномы, ассоциированные с полиномами Лагерра}

%\section{Введение}
%
%Рассмотрим  задачу Коши  для обыкновенного дифференциального уравнения (ОДУ) вида
%\begin{equation}\label{lag-1.1}
%a_r(x)y^{(r)}(x)+a_{r-1}(x)y^{(r-1)}(x)+\cdots+a_0(x)y(x)=h(x)
%\end{equation}
%с начальными условиями $y^{(k)}(0)=y_k$, $k=0,1,\ldots,r-1$.  Наряду с различными сеточными методами для решения этой задачи часто применяют так называемые спектральные методы \cite{Tref1}--\cite{Pash}. Напомним, что суть спектрального метода решения задачи Коши  для ОДУ  заключается в том, что в первую очередь искомое решение $y(x)$ представляется в виде ряда Фурье
%\begin{equation}\label{lag-1.2}
%y(x)=\sum_{k=0}^\infty \hat y_k\psi_k(x)
%\end{equation}
%по подходящей ортонормированной системе $\{\psi_k(x)\}_{k=0}^\infty$. На втором этапе осуществляется подстановка вместо $y(x)$ ряда \eqref{lag-1.2} в уравнение \eqref{lag-1.1}. Это приводит к системе уравнений относительно неизвестных коэффициентов $\hat y_k$ ($k=0,1,\ldots$). На третьем этапе требуется решить эту систему с учетом начальных условий  $y^{(k)}(0)=y_k$, $k=0,1,\ldots,r-1$, исходной задачи Коши.
%Одна из основных трудностей, которые возникают на этом этапе, состоит в том, чтобы
%выбрать такой ортонормированный базис $\{\psi_k(x)\}_{k=0}^\infty$, для которого искомое решение $y(x)$ уравнения \eqref{lag-1.1}, представленное в виде ряда  \eqref{lag-1.2}, удовлетворяло бы начальным условиям $y^{(k)}(0)=y_k$, $k=0,1,\ldots,r-1$. Более того, поскольку в результате решения системы уравнений относительно неизвестных коэффициентов $\hat y_k$  будет найдено только конечное их число с $k=0,1,\ldots, n$, то весьма важно, чтобы частичная сумма ряда \eqref{lag-1.2}  вида $y_n(x)=\sum_{k=0}^n\hat y_k\psi_k(x)$,
%будучи приближенным решением рассматриваемой задачи Коши, также удовлетворяла бы начальным условиям $y_n^{(k)}(0)=y_k$, $k=0,1,\ldots,r-1$. Как  показано в \cite{SharDiffur2018} (см. также ниже пункт 11.6),   базис $\{\psi_k(x)=l_{r,k}^{\alpha}(x)\}_{k=0}^\infty  $, состоящий из полиномов $l_{r,k}^{\alpha}(x)$, ортонормированных по Соболеву относительно   скалярного произведения
%\begin{equation}\label{lag-1.3}
%\langle f,g \rangle=\sum_{s=0}^{r-1}f^{(s)}(0)g^{(s)}(0)+\int_0^\infty f^{(r)}(t)g^{(r)}(t)t^\alpha e^{-t}dt,
%\end{equation}
%и порожденных ортонормированными с весом $\rho(x)=x^\alpha e^{-x}$ ($-1<\alpha<1$) полиномами Лагерра $l_{k}^{\alpha}(x)$  посредством равенств \eqref{lag-3.1} и \eqref{lag-3.2}, обладает указанными свойствами. Таким образом, полиномы, ортогональные по Соболеву относительно скалярного произведения \eqref{lag-1.3}, тесно связаны с задачей Коши для уравнения \eqref{lag-1.1}. Как показано в  \cite{SharDiffur2018},   ряд Фурье по полиномам  $l^\alpha_{r,k}(x)$ является весьма удобным и естественным  инструментом  представления решения задачи Коши для обыкновенного дифференциального уравнения \eqref{lag-1.1}. А в настоящей работе мы покажем, что ряд Фурье по полиномам $l^\alpha_{r,k}(x)$ является удобным средством, позволяющим численно-аналитическими методами решать  задачи Коши  для систем нелинейных ОДУ. С этой целью в пространстве $l^2_m$ (см.\S 4) будут сконструированы некоторые итерационные процессы, направленные на приближенное нахождение неизвестных коэффициентов разложения в ряд Фурье по полиномам $l^\alpha_{r,k}(x)$ искомого решения задачи Коши для систем ОДУ. Скорость сходимости указанных итерационных процессов непосредственно зависит от аппроксимативных свойств  сумм Фурье решения рассматриваемой задачи Коши  по полиномам $l^\alpha_{r,k}(x)$, изучение которых, в свою очередь, приводит к задаче об асимптотических свойствах самих полиномов $l^\alpha_{r,k}(x)$ при $k\to\infty$. Некоторые из этих вопросов былы рассмотрены в \cite{Shar13}, \cite{SHII},  \cite{SharSMJ2017} и \cite{SharDiffur2018}. В дальнейшем мы  обзорно остановимся на  результатах, полученных в этих работах.

\section{Некоторые сведения о полиномах Лагерра}
При рассмотрении свойств систем функций, ортогональных по Соболеву, порожденных полиномами и функциями Лагерра, нам понадобится ряд свойств самих полиномов Лагерра $L_n^\alpha(t)$, которые для удобства ссылок мы соберем в данном разделе.

Пусть $\alpha$ -- произвольное действительное число. Тогда для полиномов Лагерра  имеют место \cite{Sege}:

\textit{Формула Родрига}
\begin{equation}\label{lag-2.1}
L_n^{\alpha}(t) = \frac{1}{n!}t^{-\alpha}e^{t} \left\{ t^{n+\alpha} e^{-t} \right\}^{(n)};
\end{equation}

\textit{Явный вид}
\begin{equation}\label{lag-2.2}
L_n^\alpha(t) =
\sum\limits_{\nu=0}^{n}
\binom{n+\alpha}{n-\nu}
\frac{(-t)^\nu}{\nu!};
\end{equation}

\textit{Соотношение ортогональности}

\begin{equation}
\label{lag-2.3}
\int_0^{\infty} t^{\alpha} e^{-t} L^{\alpha}_{n}(t) L^{\alpha}_{m}(t) dt = \delta_{nm} h^{\alpha}_n \quad (\alpha > -1),
\end{equation}
где $\delta_{nm}$ --- символ Кронекера,
$h^{\alpha}_n = \left( n+\alpha \atop n \right) \Gamma(\alpha +1)$,
в частности, если обозначить $L_{n}(t) = L^{0}_{n}(t)$, то $\int_0^{\infty} e^{-t} L_{n}(t) L_{m}(t) dt = \delta_{nm}$;

\textit{ Равенства}

\begin{equation} \label{lag-2.6}
\frac{d^r}{dt^r} L_{k+r}^{\alpha-r}(t) = (-1)^{r} L_{k}^{\alpha}(t),
\end{equation}
\begin{equation}\label{lag-2.7}
L_{k}^{-r}(t) = \frac{(-t)^{r}}{k^{[r]}} L_{k-r}^{r}(t),
\end{equation}
где $k^{[r]} = k(k-1)\ldots(k-r+1)$;


\textit{Весовая оценка} \cite{AskeyWaiger}
\begin{equation}\label{lag-2.10}
e^{-\frac{t}{2}}|L_n^\alpha(t)| \le c(\alpha) B_n^\alpha(t), \quad \alpha>-1,
\end{equation}
где здесь и далее $c,c(\alpha),c(\alpha,\ldots,\beta)$ -- положительные числа, зависящие лишь от указанных параметров,
\begin{equation}\label{lag-2.11}
B_n^\alpha(t)=
\begin{cases}
\theta^\alpha, &0 \le t \le \frac{1}{\theta},\\
\theta^{\frac{\alpha}{2} - \frac{1}{4}}\,t^{-\frac{\alpha}{2} - \frac{1}{4}}, & \frac{1}{\theta} < t \le \frac{\theta}{2},\\
\Bigl[
\theta(\theta^{\frac{1}{3}}+|t-\theta|)
\Bigr]^{-\frac{1}{4}}, & \frac{\theta}{2} < t \le \frac{3\theta}{2},\\
e^{-\frac{t}{4}}, &\frac{3\theta}{2}< t,
\end{cases}
\end{equation}
где $\theta=\theta_n=\theta_n(\alpha)=4n+2\alpha+2$.

\textit{рекуррентная формула}
\begin{equation}\label{Ram_eq4}
\left.\begin{gathered}
L_{0}^{\alpha}(x)=1, \quad L_1^{\alpha}(x)=-x+\alpha+1,\\
nL_n^{\alpha}(x)=(-x+2n+\alpha-1)L_{n-1}^{\alpha}(x)-(n+\alpha-1)L_{n-2}^{\alpha}(x), \quad n=2, 3, \ldots
\end{gathered}\right\};
\end{equation}


\textit{асимптотическая формула}
\begin{equation*}
\left.\begin{gathered}
e^{-\frac{x}{2}}x^{\frac{\alpha}{2}}L_n^\alpha(x)=N^{-\frac{\alpha}{2}}\frac{\Gamma(n+\alpha+1)}{n!}J_\alpha\left(2(Nx)^\frac{1}{2}\right)
+O\left(n^{\frac{\alpha}{2}-\frac{3}{4}}\right),\\
N=n+\frac{\alpha+1}{2},\ x>0, \ \alpha>-1,
\end{gathered}\right\}
\end{equation*}
где оценка равномерна в промежутке $0<x\leq\omega$ ($\omega$ -- фиксированное положительное число), $J_\alpha(x)$ -- функция Бесселя, для которой, в свою очередь, справедлива следующая асимптотическая формула
\begin{equation*}
J_\alpha(x)=\left(\frac{2}{\pi x}\right)^{\frac{1}{2}}\cos\left(x-\frac{\alpha\pi}{2}-\frac{\pi}{4}\right)+O\left(x^{-\frac{3}{2}}\right),\ x\rightarrow+\infty;
\end{equation*}

Для нормированных полиномов Лагерра
\begin{equation}\label{lag-2.12}
l_n^\alpha(t)=
\Bigl\{h_n^\alpha \Bigr\}^{-\frac{1}{2}} L_n^\alpha(t)
\end{equation}
имеют место следующая оценка \cite{AskeyWaiger}:
\begin{equation*}\label{lag-2.13}
e^{-\frac{t}{2}}
|l_n^\alpha(t)|\le
c(\alpha)\theta_n^{-\frac{\alpha}{2}}B_n^\alpha(t), \quad t \ge 0.
\end{equation*}
Отметим также следующие свойства:

\textit{Свертка}
\begin{equation}
\label{lag-2.15}
\int_0^{t} L_{n}(t-\tau) L_{m}(\tau) d\tau = L_{n+m}(t) - L_{n+m+1}(t);
\end{equation}

\textit{Формула Кристоффеля -- Дарбу}
\begin{equation}\label{lag-2.16}
\mathcal{K}_n^\alpha(t,\tau)=
\sum\limits_{k=0}^{n}\frac{L_\nu^\alpha(t)L_\nu^\alpha(\tau)}{h_\nu^\alpha}=
\frac{n+1}{h_n^\alpha}
\frac{L_n^\alpha(t)L_{n+1}^\alpha(\tau) - L_n^\alpha(\tau)L_{n+1}^\alpha(t)}{t-\tau}.
\end{equation}

\begin{lemma}\label{lem11.1}
	Если $\alpha\ge-1/2$, то имеет место оценка
	\begin{equation*}
	\tau^{\alpha+1/2}e^{-\tau}\mathcal{K}_{n}^\alpha(\tau,\tau)\le c(\alpha)n^{1/2}, \quad 0\le \tau<\infty.
	\end{equation*}
\end{lemma}
Эта лемма для $\tau\ge3/(4n+2\alpha+2)$ доказана в \cite{SharSMJ2017} при $\alpha>-1$. Если же $\alpha\ge-1/2$, то из оценки \eqref{lag-2.10} непосредственно следует, что утверждение леммы \ref{lem11.1} справедливо  также и при  $0\le\tau\le3/(4n+2\alpha+2)$.

Из \eqref{Ram_eq4} и \eqref{lag-2.12} легко можно получить рекуррентную формулу для $l_n^\alpha(x)$:
\begin{equation}\label{lag-recurr}
\left.\begin{gathered}
l_{0}^{\alpha}(x)=\frac{1}{\sqrt{\Gamma(\alpha+1)}}, \quad l_1^{\alpha}(x)=\frac{-x+\alpha+1}{\sqrt{\Gamma(\alpha+2)}},\\
l_n^{\alpha}(x)=(a_n-b_n x)l_{n-1}^{\alpha}(x)-c_n l_{n-2}^{\alpha}(x), \quad n=2, 3, \ldots
\end{gathered}\right\},
\end{equation}
где
\begin{equation*}
a_n=a_n(\alpha)=\frac{2n+\alpha-1}{[n(n+\alpha)]^\frac{1}{2}},\
b_n=b_n(\alpha)=\frac{1}{[n(n+\alpha)]^\frac{1}{2}},\
c_n=c_n(\alpha)=\Big[\frac{(n-1)(n+\alpha-1)}{n(n+\alpha)}\Big]^\frac{1}{2}.
\end{equation*}

Через $\lambda_n^\alpha(x)$ $(n=0, 1, \ldots)$ обозначим функции Лагерра, которые определяются равенством
\begin{equation}\label{funcLag}
\lambda_n^\alpha(x)=\sqrt{\rho(x)}l_n^\alpha(x),
\end{equation}
где $\rho(x)=e^{-x}x^\alpha$, $l_n^\alpha(x)$ -- ортонормированный полином Лагерра, определенный равенством \eqref{lag-2.12}.
Так как функции $\lambda_n^\alpha(x)$ отличаются от полиномов $l_n^\alpha(x)$ множителем, не зависящим от номера функции, следовательно, аналогичная \eqref{lag-recurr} рекуррентная формула справедлива и для функций $\lambda_n^\alpha(x)$:
\begin{equation*}
\left.\begin{gathered}
\lambda_{0}^{\alpha}(x)=\frac{\sqrt{\rho(x)}}{\sqrt{\Gamma(\alpha+1)}}, \quad \lambda_1^{\alpha}(x)=\frac{\sqrt{\rho(x)}(-x+\alpha+1)}{\sqrt{\Gamma(\alpha+2)}},\\
\lambda_n^{\alpha}(x)=(a_n-b_n x)\lambda_{n-1}^{\alpha}(x)-c_n \lambda_{n-2}^{\alpha}(x), \quad n=2, 3, \ldots
\end{gathered}\right\}.
\end{equation*}
В дальнейшем нам понадобится следующее свойство функций $\lambda_{n}^{\alpha}(x)$ \cite[Theorem 1]{AskeyWaiger}.

\begin{theoremA}\label{Ram_thA}
	Let $f\in L^p$, $\frac{4}{3}<p<4$. Define $a_n=\int_{0}^{\infty}\lambda^\alpha_n(x)f(x)dx$ and set $S_n=\sum_{k=0}^{n}a_k\lambda^\alpha_k(x)$. Then $\|S_n-f\|_{L^p}\rightarrow0$ as $n\rightarrow\infty$.
\end{theoremA}


\section{Ортогональные по Соболеву полиномы, порожденные полиномами Лагерра}
Пусть $-1<\alpha$,  $\rho=\rho(x)=x^\alpha e^{-x}$, $1\le p<\infty $,  $\mathcal{L}_{\rho}^p$ -- пространство измеримых функций $f(x)$, определенных на полуоси $[0,\infty)$ и таких, что
$$
\|f\|_{\mathcal{L}_{\rho}^p}=
\left(\int_0^\infty|f(x)|^p\rho(x)dx\right)^{1/p}<\infty.
$$
Из равенства \eqref{lag-2.3} следует, что если $\alpha>-1$, то полиномы $l_n^{\alpha}(x)$ $(n=0,1,\ldots)$ (см. \eqref{lag-2.12})
образуют ортонормированную в $\mathcal{L}_\rho^2$ систему. Как хорошо известно \cite{Sege}, система полиномов Лагерра  \eqref{lag-2.12} полна в $\mathcal{L}_\rho^2$.   Эта система порождает на $[0,\infty)$ систему полиномов $l_{r,k}^{\alpha}(x)$ $(r \in \mathbb{N}, k=0,1,\ldots)$, определенных равенствами

\begin{equation}\label{lag-3.1}
l_{r,k}^{\alpha}(x) =\frac{x^k}{k!}, \quad k=0,1,\ldots, r-1,
\end{equation}
\begin{equation}\label{lag-3.2}
l_{r,r+k}^{\alpha}(x) =\frac{1}{(r-1)!}\int\limits_{0}^x(x-t)^{r-1}l_{k}^{\alpha}(t)dt, \quad k=0,1,\ldots.
\end{equation}
и обладающих следующим важным свойством

\begin{equation}\label{lag-3.3}
(l^\alpha_{r,k}(x))^{(\nu)} =
\begin{cases}
l^\alpha_{r-\nu,k-\nu}(x), &\nu \le \min\{k,r\},\\
(l^\alpha_{k-r}(x))^{(\nu-r)}, &r<\nu \le k\\
0, &\nu>k,
\end{cases}
\end{equation}
где полагаем $l^\alpha_{0,k}(x)=l^\alpha_{k}(x)$.

Через $W_{\mathcal{L}_{\rho}^p}^r$ обозначим  подкласс функций $f=f(x)$,
непрерывно дифференцируемых $r-1$ раз, для которых $f^{(r-1)}(x)$
абсолютно непрерывна на произвольном сегменте $[a,b]\subset[0,\infty)$,
а $f^{(r)}\in \mathcal{L}_{\rho}^p$. В $W_{\mathcal{L}_{\rho}^2}^r$ мы введем скалярное произведение
\begin{equation}\label{lag-1.3}
\langle f,g \rangle=\sum_{s=0}^{r-1}f^{(s)}(0)g^{(s)}(0)+\int_0^\infty f^{(r)}(t)g^{(r)}(t)t^\alpha e^{-t}dt,
\end{equation}
которое превращает $W_{\mathcal{L}_{\rho}^2}^r$ в гильбертово пространство.
В работах \cite{SHII}, \cite{SharSMJ2017},  \cite{SharDiffur2018} было показано, что система полиномов $\{l_{r,k}^{\alpha}(x)\}_{k=0}^\infty$ является полной и ортонормированной в $W_{\mathcal{L}_{\rho}^2}^r$. В частности, в \cite{SHII} (см. также \cite{SharDiffur2018}) доказана следующая

\begin{theorem}\label{th13}
	Пусть $\alpha>-1$. Тогда система полиномов $\{l_{r,k}^{\alpha}(x)\}_{k=0}^\infty$, порожденная системой ортонормированных полиномов Лагерра \eqref{lag-2.12} посредством равенств \eqref{lag-3.1} и \eqref{lag-3.2}, полна  в $W^r_{\mathcal{L}^2_\rho}$ и ортонормирована относительно скалярного произведения \eqref{lag-1.3}.
\end{theorem}

Ряд Фурье функции $f\in W^r_{\mathcal{L}^2_\rho}$ по системе $\{l_{r,k}^{\alpha}(x)\}_{k=0}^\infty$
мы можем записать в виде
\begin{equation}\label{lag-3.8}
f(x)\sim  \sum_{k=0}^\infty \langle f,l_{r,k}^\alpha \rangle  l_{r,k}^\alpha(x),
\end{equation}
где
\begin{equation}\label{lag-3.9}
\langle f,l_{r,k}^\alpha \rangle = f^{(k)}(0),\quad k=0,\ldots, r-1,
\end{equation}
\begin{equation}\label{lag-3.10}
\langle f,l_{r,k}^\alpha \rangle = \int\limits_0^\infty f^{(r)}(t) l_{k-r}^\alpha(t)e^{-t}t^\alpha dt=f_{r,k}^\alpha,\quad k=r,r+1,\ldots.
\end{equation}
В силу \eqref{lag-3.9}  и \eqref{lag-3.10} мы можем \eqref{lag-3.8} переписать еще так
\begin{equation}\label{lag-3.11}
f(x)\sim \sum_{k=0}^{r-1} f^{(k)}(0)\frac{x^k}{k!}+ \sum_{k=r}^\infty f_{r,k}^\alpha l_{r,k}^\alpha(x).
\end{equation}
Ряд, фигурирующий в правой части соотношения \eqref{lag-3.11}, впервые был исследован в работе \cite{Shar13}, где он был назван \textit{смешанным рядом по полиномам Лагерра $L_{k}^\alpha(x)$}. Из теоремы \ref{th13} следует, что если   $f\in W^r_{\mathcal{L}^2_\rho}$, то ряд \eqref{lag-3.11}, будучи  рядом Фурье  по системе $\{l_{r,k}^{\alpha}(x)\}_{k=0}^\infty$, сходится к $f$ в метрике гильбертова пространства $W^r_{\mathcal{L}^2_\rho}$ со скалярным произведением \eqref{lag-1.3}.
Однако это не означает, что ряд Фурье \eqref{lag-3.11} сходится к $f(x)$ в заданной точке $x\in[0,\infty)$.  При исследовании этого вопроса существенную роль играют асимптотические свойства полиномов $l_{r,k}^{\alpha}(x)$ при $k\to\infty$. С целью изучения асимптотических свойств  полиномов $l_{r,k}^{\alpha}(x)$ в работах \cite{SHII}, \cite{SharSMJ2017},  \cite{SharDiffur2018} (см. также \cite{Shar13}) получены некоторые их представления, содержащие   классические полиномы Лагерра $L_{m}^{\alpha}(x)$, а также представление $l_{r,k}^{\alpha}(x)$  в явном виде.

\begin{theorem}
	Пусть $\alpha>-1$, $k\ge0$. Тогда имеет место равенство
	\begin{equation}\label{lag-3.19}
	l_{r,r+k}^{\alpha}(x)=\frac{(-1)^r}{\sqrt{h_k^\alpha}}\left[L_{k+r}^{\alpha-r}(x)-
	\sum_{\nu=0}^{r-1}
	{B_{k,\nu}^\alpha x^\nu\over\nu!}\right],
	\end{equation}
	в котором
	$$
	B_{k,\nu}^\alpha={(-1)^\nu\Gamma(k+\alpha+1)\over\Gamma(\nu-r+ \alpha+1)(k+r-\nu)!}.
	$$
\end{theorem}

\begin{corollary}\label{cor8}
	Пусть  $k\ge0$. Тогда
	$$
	l_{r,r+k}^{0}(x)=(-1)^rL_{k+r}^{-r}(x)=\frac{x^{r}L_{k}^{r}(x)}{(k+r)^{[r]}}.
	$$
\end{corollary}



Еще одно важное представление для полиномов $l_{r,n+r}^{\alpha}(x)$ можно получить, если мы обратимся к равенствам \eqref{lag-2.2} и \eqref{lag-2.12} и запишем
\begin{equation}\label{lag-3.21}
l_n^\alpha(x) =\frac{1}{(h_n^\alpha)^{1/2}}
\sum\limits_{\nu=0}^{n}
\binom{n+\alpha}{n-\nu}
\frac{(-x)^\nu}{\nu!}.
\end{equation}
Поскольку, очевидно,
\begin{equation*}
{1\over (r-1)!}\int\limits_{0}^x(x-t)^{r-1}t^\nu dt=\frac{x^{\nu+r}}{(\nu+r)^{[r]}},
\end{equation*}
то из \eqref{lag-3.21} и \eqref{lag-3.2} вытекает справедливость следующего утверждения.

\begin{statement}
	Для произвольного $\alpha>-1$ и натурального $r$ имеет место равенство
	\begin{equation*}\label{lag-3.22}
	l_{r,n+r}^{\alpha}(x)=
	\frac{1}{(h_n^\alpha)^{1/2}}
	\sum\limits_{\nu=0}^{n}(-1)^\nu \binom{n+\alpha}{n-\nu}
	\frac{x^{\nu+r}}{\nu!(\nu+r)^{[r]}}\quad (n=0,1,\ldots).
	\end{equation*}
\end{statement}












\section{Ортогональные по Соболеву полиномы, порожденные модифицированными полиномами Лагерра и задача Коши для систем ОДУ}
Мы рассмотрим некоторые новые вопросы, касающиеся представления решения задачи Коши для систем нелинейных ОДУ рядами Фурье по ортогональным по Соболеву полиномам, порожденным модифицированными полиномами Лагерра. Прежде всего определим  модифицированные полиномы Лагерра $l_n^\alpha(x;b)$ и порожденные ими полиномы, ортогональные по Соболеву. С этой целью  для $b>0$ мы положим
\begin{equation}\label{12.1}
\rho_b(x)=\rho(bx)=(bx)^\alpha e^{-bx},\quad l_n^\alpha(x;b)=\sqrt{b}l_n^\alpha(bx),\quad l_n(x;b)=l_n^0(x;b)
\end{equation}
и заметим, что из \eqref{12.1} в силу \eqref{lag-2.3} и \eqref{lag-2.12}
\begin{equation}\label{12.2}
\int_0^\infty\rho_b(x) {l}_n^\alpha(x;b) {l}_k^\alpha(x;b)dx=\delta_{nk},
\end{equation}
откуда следует, что система $\{l_n^\alpha(x;b)\}_{n=0}^\infty$ ортонормирована на $[0,\infty)$ с весом $\rho_b(x)$. Она порождает при каждом натуральном $r$ новую систему посредством равенств
\begin{equation}\label{12.3}
l_{r,n+r}^\alpha(x;b)=\frac{1}{(r-1)!}\int_{0}^x(x-t)^{r-1}{l}_{n}^\alpha(t;b)dt, \quad n=0,1, \ldots
\end{equation}
и
\begin{equation}\label{12.4}
{l}_{r,k}^{\alpha}(x;b) =\frac{x^k}{k!}, \quad k=0,1,\ldots, r-1,
\end{equation}
которая ортонормирована по Соболеву относительно скалярного произведения
\begin{equation}\label{12.5}
\langle f,g\rangle_b=\sum\nolimits_{\nu=0}^{r-1}f^{(\nu)}(0)g^{(\nu)}(0)+\int_{0}^\infty f^{(r)}(t)g^{(r)}(t)\rho_b(t)dt.
\end{equation}

Из \eqref{12.3} и \eqref{12.4} имеем
\begin{equation}\label{12.6}
(l^\alpha_{r,k}(x;b))^{(\nu)} =\begin{cases}l^\alpha_{r-\nu,k-\nu}(x;b),&\text{если $0\le\nu\le r-1$, $r\le k$,}\\
l^\alpha_{k-r}(x;b),&\text{если  $\nu=r\le k$,}\\
l^\alpha_{r-\nu,k-\nu}(x;b),&\text{если $\nu\le k< r$,}\\
0,&\text{если $k< \nu\le r-1$}.
\end{cases}
\end{equation}
Далее, сопоставляя \eqref{12.1}, \eqref{12.3} с \eqref{lag-3.2}, мы замечаем, что
\begin{equation}\label{12.7}
{l}_{r,k}^{\alpha}(x;b)=b^{\frac12-r}l_{r,k}^\alpha(bx), \quad k\ge r.
\end{equation}
Если, кроме того,  мы положим $\alpha=0$, то из \eqref{12.7} и следствия \ref{cor8} находим
\begin{equation}\label{12.8}
{l}_{r,k+r}(x;b)={l}_{r,k+r}^{0}(x;b)=\frac{\sqrt{b}x^rL_{k}^r(bx)}{(k+r)^{[r]}}, \,k=0,1,\ldots,
\end{equation}
где $s^{[r]}=s(s-1)\ldots (s-r+1)$.

Введем для $a>0$ следующую величину
\begin{equation}\label{12.9}
I(a,b)=\int_0^\infty e^{-(2a+b)t}\sum\nolimits_{k=1}^\infty({l}_{1,k}(t;b) )^2dt.
\end{equation}
\begin{lemma}\label{lem12.1}
	Для произвольных $a,b>0$ имеет место неравенство
	$$
	I(a,b)\le \frac{\sqrt{\pi}s}{4\sqrt{2ba^3}}\sum\nolimits_{n=0}^\infty(n+1)^{-3/2},
	$$
	где
	\begin{equation}\label{12.10}
	s=\sup_{n,x\ge0}\frac{\mathcal{K}_n^1(x,x)e^{-x}x^{3/2}}{\sqrt{n+1}}.
	\end{equation}
\end{lemma}
\begin{proof}
	Из \eqref{12.9} с учетом \eqref{12.8}   имеем
	$$
	I(a,b)=\int_0^\infty e^{-2at}\sum\nolimits_{k=1}^\infty\frac{bt^2e^{-bt}}{k^2}(L_{k-1}^1(bt))^2dt=
	$$
	$$
	\frac{1}{b^2}\int_0^\infty e^{-2a\tau/b}\sum\nolimits_{k=1}^\infty\frac{\tau^2e^{-\tau}}{k^2}(L_{k-1}^1(\tau))^2d\tau=
	$$
	$$
	\frac{1}{b^2}\int_0^\infty e^{-2a\tau/b}\sum\nolimits_{k=0}^\infty\frac{\tau^2e^{-\tau}}{k+1}(l_k^1(\tau))^2d\tau.
	$$
	Отсюда, воспользовавшись преобразованием Абеля, находим
	$$
	I(a,b)=\frac{1}{b^2}\int_0^\infty e^{-2a\tau/b}\sum_{k=0}^\infty\frac{\tau^2e^{-\tau}}{(k+1)(k+2)}\mathcal{K}_k^1(\tau,\tau)d\tau=
	$$
	$$
	$$
	$$
	\frac{1}{b^2}\int_0^\infty \sum_{k=0}^\infty\frac{\tau^{1/2}e^{-2a\tau/b}}{(k+1)^{1/2}(k+2)}\frac{e^{-\tau}\tau^{3/2} \mathcal{K}_k^1(\tau,\tau)}{(k+1)^{1/2}}d\tau\le
	$$
	\begin{equation}\label{12.11}
	\frac{s}{b^2}\sum_{k=0}^\infty(k+1)^{-3/2}\int_0^\infty
	\tau^{1/2}e^{-2a\tau/b}d\tau,
	\end{equation}
	где число $s$ определено равенством \eqref{12.10} и в силу леммы \ref{lem11.1} конечно.
	Далее,
	\begin{equation}\label{12.12}
	\int_0^\infty\tau^{1/2}e^{-2a\tau/b}d\tau=(b/2a)^{3/2}\int_0^\infty\tau^{1/2}e^{-\tau}d\tau=
	(b/2a)^{3/2}\Gamma(3/2).
	\end{equation}
	Утверждение леммы \ref{lem12.1} вытекает из \eqref{12.11} и \eqref{12.12}.
\end{proof}

\begin{lemma}\label{lem12.2}
	Если $A\ge0$, $b>0$, то ряд
	$\sum_{k=1}^\infty (l_{1,k}(x;b))^2$ сходится равномерно относительно $x\in[0,A]$.
\end{lemma}

%++++++++++++++++++++++++++++++++++++++++++++++++++++++++++++++++++

Введем в рассмотрение пространства $\mathcal{L}_{\rho_b}^2$ и $W^r_{\mathcal{L}_{\rho_b}^2}$ совершенно аналогично тому, как были определены пространства $\mathcal{L}_{\rho}^2$ и $W^r_{\mathcal{L}_{\rho}^2}$, с той лишь разницей, что вместо веса $\rho(x)=x^\alpha e^{-x}$ берется вес $\rho_b(x)=\rho(bx)$.  Тогда ряд Фурье функции $f\in W^r_{\mathcal{L}_{\rho_b}^2}$ по системе  $l_{r,k}^\alpha(x;b)$ примет вид
\begin{equation}\label{12.13}
f(x)= \sum_{k=0}^{r-1} f^{(k)}(0)\frac{x^k}{k!}+ \sum_{k=r}^\infty {f}_{r,k}^{\alpha,b} {l}_{r,k}^\alpha(x;b),
\end{equation}
\begin{equation}\label{12.14}
{f}_{r,k}^{\alpha,b}=\langle f,l_{k}^\alpha(*;b) \rangle_b=\int_{0}^\infty f^{(r)}(t){l}_{k-r}^\alpha(t;b)\rho_b(t)dt, \quad k=r,r+1,\ldots.
\end{equation}
Отсюда и из \eqref{12.6}, в частности, следует, что
\begin{equation}\label{12.15}
f^{(r)}(x)=  \sum\nolimits_{k=0}^\infty f_{r,k}^{\alpha,b} l_{k}^\alpha(x;b),
\end{equation}
где равенство \eqref{12.15} понимается в смысле сходимости ряда из правой части этого равенства  к функции $f^{(r)}(x)$ в метрике пространства $\mathcal{L}_{\rho_b}^2$. Кроме того,  в силу \eqref{12.6} и \eqref{12.13} для $0\le\nu\le r-1$ находим
(${f}_{r,k}^{\alpha,b}=(f^{(\nu)})_{r-\nu,k-\nu}^{\alpha,b} $)
\begin{equation}\label{12.16}
f^{(\nu)}(x)= \sum\nolimits_{k=0}^{r-\nu-1} f^{(k+\nu)}(0)\frac{x^k}{k!}+ \sum\nolimits_{k=r-\nu}^\infty (f^{(\nu)})_{r-\nu,k}^{\alpha,b} l_{r-\nu,k}^\alpha(x;b).
\end{equation}

\begin{theorem}\label{cor12.1}
	Пусть $-1<\alpha<1$, $0\le\nu\le r-1$, $f\in W^r_{\mathcal{L}_{\rho_b}^2}$, $0\le A<\infty$. Тогда для произвольного $x\in[0,\infty)$ имеет место равенство \eqref{12.16}, в котором ряд Фурье функции $f^{(\nu)}(x)$ по полиномам $l_{r-\nu,k}^\alpha(x;b)$ сходится равномерно относительно $x\in[0,A]$.
\end{theorem}
%++++++++++++++++++++++++++++++++++++++++++++++++++++++++++++++++++++++++++++



Рассмотрим задачу о приближении решения задачи Коши для систем ОДУ  суммами  Фурье по системе $\{l_{1,n}(x;b)\}_{n=0}^\infty$, ортогональной по Соболеву и порожденной ортонормированной модифицированной системой полиномов Лагерра $\{l_{n}(x;b)\}_{n=0}^\infty$ посредством равенств \eqref{12.3} и \eqref{12.4} с $r=1$.
Мы будем рассматривать задачу Коши для систем ОДУ вида
\begin{equation}\label{125.1}
y'(x)=f(x,y), \quad y(0)=y^0,\quad 0\le x\le 1,
\end{equation}
где  $f=(f_1, \ldots, f_m)$, $y=(y_1, \ldots, y_m)$. Метод приближения решения этой задачи, предлагаемый в настоящем параграфе, существенно отличается от подхода к решению той же проблемы, применявшегося в подразделе \ref{SobSystemsAndCauchyProblem}, согласно которому  отрезок $[0,1]$ разбивался на несколько частей точками $0=a_0<a_1<\ldots<a_s=1$ и рассматривался вопрос об аппроксимации решения соответствующей задачи Коши на каждом из отрезков $[a_i,a_{i+1}]$ $(i=0,1,\ldots,s-1)$.  <<Склеивая>> сконструированные на отрезках $[a_i,a_{i+1}]$ приближенные решения $y_i(x)$, определялось приближенное решение $\tilde y(x)$ для исходной задачи Коши \eqref{125.1} на всем отрезке $[0,1]$.   Предлагаемый ниже метод, основанный на разложении искомого решения задачи Коши  в ряд по полиномам, ортогональным по Соболеву и порожденным  модифицированными полиномами Лагерра, позволяет сконструировать функцию, аппроксимирующую решение задачи Коши, сразу на всем полуинтервале $[0,1)$.  Вектор-функцию   $f(x,y)$  будем считать непрерывной в некоторой замкнутой  области $\bar G$ переменных $(x,y)$, содержащей точку $(0,y_0)$. Кроме того, мы будем  считать, что  $[0,1]\times\mathbb{R}^m\subset\bar G$. Это требование, как уже отмечалось в подразделе \ref{SobSystemsAndCauchyProblem}, не сужает дальнейшие рассмотрения.
Мы будем считать, что по переменной $y$ функция $f(x,y)$ удовлетворяет условию Липшица
\begin{equation}\label{125.2}
\|f(x,u)-f(x,v)\|\le \lambda\|u-v\|, \quad 0\le x \le 1,
\end{equation}
где $\|(u_1,\ldots,u_m)\|=\sqrt{\sum_{l=1}^mu_l^2}$.

Если $a>0$, то мы можем отобразить  полуось $[0,\infty)$ на $[0,1)$ посредством  замены переменных $x=1-e^{-at}$. Относительно новой переменной $t\in [0,\infty)$ уравнение \eqref{125.1} принимает следующий вид
\begin{equation}\label{125.3}
\eta'(t)=ae^{-at}f(1-e^{-at},\eta(t)), \quad \eta(0)=y^0,\quad 0\le t<  \infty,
\end{equation}
где $\eta(t)=y(1-e^{-at})$. Пусть $y=y(x)=(y_1(x), \ldots, y_m(x))$ решение уравнения \eqref{125.1}, удовлетворяющее начальному условию $y(0)=y^0$. Стало быть, вектор-функция $y=y(x)$ непрерывно дифференцируема и ограничена на $[0,1]$. Отсюда следует, что функция $\eta=\eta(t)=y(1-e^{-at})=(\eta_1(t),\ldots,\eta_m(t))$ непрерывно дифференцируема и ограничена на полуоси $[0,\infty)$.
Следовательно, $\eta_l\in W^1_{\mathcal{L}_{\rho_b}^2}$ для любого $b>0$ и $l=1,\ldots,m$. Поэтому из теоремы \ref{cor12.1} вытекает, что для функции $\eta=\eta(t)$ справедливо разложение
\begin{equation}\label{125.4}
\eta(t)= \eta(0)+ \sum\nolimits_{k=1}^\infty {\eta}_{1,k} {l}_{1,k}(t;b),\quad t\in[0,\infty),
\end{equation}
где ${\eta}_{1,k}={\eta}_{1,k}^{0,b}$, т.е.
\begin{equation}\label{125.5}
{\eta}_{1,k}=({\eta_1}_{1,k},\ldots, {\eta_\nu}_{1,k},\ldots,{\eta_m}_{1,k})=\int_{0}^\infty \eta'(\tau){l}_{k-1}(\tau;b)\rho_b(\tau)d\tau, k=1,2,\ldots.
\end{equation}
Обратимся к равенствам \eqref{12.6} и  \eqref{125.4} и запишем
\begin{equation}\label{125.6}
\eta'(t)=  \sum\nolimits_{k=0}^\infty {\eta}_{1,k+1}l_{k}(t;b),
\end{equation}
где равенство понимается в том смысле, что ряд в правой части равенства \eqref{125.6} сходится к $\eta'_l$ в метрике пространства $L^2_{\rho_b}$ для всех $l=1,\ldots,m$. Положим $q(t)=e^{-at}f(1-e^{-at},\eta(t))=\eta'(t)/a$ и заметим, что в силу  \eqref{125.6}  коэффициенты Фурье вектор-функции $q=q(t)$ по системе  $\{l_{k}(t;b)\}_{k=0}^\infty$ имеют вид
\begin{equation}\label{125.7}
c_k(q)=\int_{0}^\infty q(\tau)l_{k}(\tau;b)\rho_b(\tau)d\tau=\eta_{1,k+1}/a \quad (k\ge0).
\end{equation}
С учетом этих равенств мы можем переписать \eqref{125.4} в следующем виде
\begin{equation}\label{125.8}
\eta(t)= \eta(0)+ a\sum\nolimits_{k=0}^\infty c_k(q)l_{1,k+1}(t;b).
\end{equation}
Наша цель состоит в том, чтобы сконструировать итерационный процесс для нахождения приближенных значений коэффициентов $c_k$ $(k=0,1,\ldots)$, где $c_k=(c_k^1,\ldots,c_k^m)$.
Из  \eqref{125.7} и \eqref{125.8} имеем следующие равенства ($k=0,1,\ldots$)

\begin{equation}\label{125.9}
c_k(q)=\int_{0}^\infty e^{-a\tau}f\left[1-e^{-a\tau},\eta(0)+ a\sum\nolimits_{j=0}^\infty c_j(q)l_{1,j+1}(\tau;b)\right]l_{k}(\tau;b)e^{-b\tau} d\tau.
\end{equation}
Введем в рассмотрение гильбертово пространство $l_2^m$, состоящее из последовательностей  $C=(c_0,\ldots,c_j,\ldots)$ $m$-мерных векторов $c_j=(c_j^1,\ldots,c_j^m)$,  для которых определена норма
\begin{equation}\label{125.10}
\|C\|=\left(\sum\nolimits_{j=0}^\infty \sum\nolimits_{\nu=1}^{m}(c_j^\nu)^2\right)^\frac12.
\end{equation}
Заметим, что если $C=(c_0,\ldots,c_j,\ldots)\in l_2^m$, то  в силу  неравенства Коши-Шварца и леммы \ref{lem12.2} ряд $\sum\nolimits_{j=0}^\infty c_jl_{1,j+1}(t;b)$ равномерно сходится на $[0,\infty)$ к некоторой непрерывной на $[0,A]$ вектор-функции, что, в свою очередь, гарантирует непрерывность по $t$  сложной вектор-функции
\begin{equation}\label{125.11}
g(t)=(g_1(t),\ldots,g_m(t))= e^{-at}f\left[1-e^{-at},\eta(0)+ a\sum\nolimits_{j=0}^\infty c_jl_{1,j+1}(t;b)\right].
\end{equation}
Покажем, что $g_i(t)\in L^2_{\rho_b}$, где $\rho_b(t)=e^{-bt}$. В самом деле, из условия \eqref{125.2} имеем
$$
|g_i(t)|\le e^{-at}\left|f_i\left[1-e^{-at},\eta(0)+ a\sum\nolimits_{j=0}^\infty c_jl_{1,j+1}(t;b)\right]-f_i\left[1-e^{-at},0\right]\right|+
$$
$$
e^{-at}\left|f_i\left[1-e^{-at},0\right]\right|\le e^{-at}\left|f_i\left[1-e^{-at},0\right]\right|+
$$
$$
a\lambda e^{-at} \left(\sum\nolimits_{\nu=1}^m
\left(\sum\nolimits_{j=0}^\infty c_j^\nu l_{1,j+1}(t;b)\right)^2 \right)^\frac12,
$$
а отсюда находим
$$
e^{-bt}(g_i(t))^2\le
$$
$$
2e^{-(2a+b)t}\left(f_i\left[1-e^{-at},0\right]\right)^2+
2a^2\lambda^2 e^{-(2a+b)t} \sum_{\nu=1}^m
\left(\sum\nolimits_{j=0}^\infty c_j^\nu l_{1,j+1}(t;b)\right)^2\le
$$
$$
2e^{-(2a+b)t}\left(f_i\left[1-e^{-at},0\right]\right)^2+
2a^2\lambda^2 e^{-(2a+b)t} \sum_{j=0}^\infty(l_{1,j+1}(t;b))^2\sum_{\nu=1}^m
\sum_{j=0}^\infty (c_j^\nu)^2.
$$
Поэтому, обратившись к равенству \eqref{12.9} и лемме \ref{lem12.1}, убеждаемся в том, что $g_i(t)\in L^2_{\rho_b}$.
Это позволяет рассмотреть в пространстве $l_2^m$  оператор $A$, сопоставляющий точке $C\in l_2^m$ точку $C'\in l_2^m$ по правилу ($k=0,1,\ldots$)

\begin{equation}\label{125.12}
c_k'=\int_{0}^\infty e^{-a\tau}f\left[1-e^{-a\tau},\eta(0)+ a\sum\nolimits_{j=0}^\infty c_jl_{1,j+1}(\tau;b)\right]l_{k}(\tau;b)e^{-b\tau} d\tau.
\end{equation}
Из  \eqref{125.9} следует, что точка $C(q)=(c_0(q),c_1(q),\ldots)$ является неподвижной точкой оператора $A:l_2^m\to l_2^m$.  Покажем, что оператор $A:l_2^m\to l_2^m$ является сжимающим в метрике пространства $l_2^m$. С этой целью рассмотрим две точки $P,Q\in l_2^m$, где $P=(p_0,\ldots)$, $Q=(q_0,\ldots)$, и положим $P'=A(P)$, $Q'=A(Q)$. Имеем
\begin{equation}\label{125.13}
p'_k-q'_k=\int_{0}^\infty F_{P,Q}(\tau)l_k(\tau;b)\rho_b(\tau)d\tau,\quad k=0,1,\ldots,
\end{equation}
где
\begin{multline}\label{125.14}
F_{P,Q}(\tau)=e^{-a\tau}f\left[1-e^{-a\tau},\eta(0)+ a\sum\nolimits_{j=0}^\infty p_jl_{1,j+1}(\tau;b)\right] \\
-e^{-a\tau}f\left[1-e^{-a\tau},\eta(0)+ a\sum\nolimits_{j=0}^\infty q_jl_{1,j+1}(\tau;b)\right].
\end{multline}
Из \eqref{125.13}, пользуясь неравенством Бесселя, находим
\begin{equation}\label{125.15}
\sum\nolimits_{k=0}^\infty \sum_{\nu=1}^m((p^\nu_k)'-(q^\nu_k)')^2\le\int_{0}^\infty F_{P,Q}(\tau)(F_{P,Q}(\tau))^*\rho_b(\tau) d\tau,
\end{equation}
где $(a_1,\ldots,a_m)^*$ -- вектор-столбец, полученный в результате транспонирования строки $(a_1,\ldots,a_m)$.
Из \eqref{125.14} и \eqref{125.2}  имеем
$$
F_{P,Q}(\tau)(F_{P,Q}(\tau))^*\le (a\lambda)^2e^{-2a\tau} \sum\nolimits_{\nu=1}^m  \left(\sum\nolimits_{j=0}^\infty( p^\nu_j-q^\nu_j)l_{1,j+1}(\tau;b)\right)^2,
$$
откуда,  воспользовавшись неравенством Коши-Буняковского, выводим
\begin{equation}\label{125.16}
F_{P,Q}(\tau)(F_{P,Q}(\tau))^*\le(a\lambda)^2e^{-2a\tau}  \sum\nolimits_{j=0}^\infty(l_{1,j+1}(\tau))^2 \sum\nolimits_{j=0}^\infty\sum\nolimits_{\nu=1}^m( p^\nu_j-q^\nu_j)^2.
\end{equation}
Сопоставляя \eqref{125.15} с \eqref{125.16}, находим
$$
\sum\nolimits_{j=0}^\infty\sum\nolimits_{\nu=1}^m((p^\nu_j)'-(q^\nu_j)')^2\le
$$
\begin{equation}\label{125.17}
(a\lambda)^2 \sum\nolimits_{j=0}^\infty\sum\nolimits_{\nu=1}^m( p^\nu_j-q^\nu_j)^2\int_{0}^\infty \sum\nolimits_{j=0}^\infty(l_{1,j+1}(\tau;b))^2e^{-(2a+b)\tau} d\tau.
\end{equation}
Из  \eqref{125.17}  и \eqref{12.9} имеем
\begin{equation}\label{125.18}
\left(\sum\nolimits_{j=0}^\infty\sum\nolimits_{\nu=1}^m((p^\nu_j)'-(q^\nu_j)')^2\right)^\frac12\le a\lambda I^\frac12(a,b) \left(\sum\nolimits_{j=0}^\infty\sum\nolimits_{\nu=1}^m( p^\nu_j-q^\nu_j)^2\right)^\frac12.
\end{equation}
С другой стороны, из леммы \ref{lem12.1} следует, что
\begin{equation}\label{125.19}
a\lambda I^\frac12(a,b)\le\lambda \frac{\sqrt{s}}2\left(\frac{\pi a}{2b}\right)^\frac14 \sum\nolimits_{n=1}^\infty n^{-3/2}=\delta(\lambda,a,b),
\end{equation}
где число $s$ определено равенством \eqref{12.10}. Сопоставляя \eqref{125.19} с \eqref{125.18}, окончательно получаем
\begin{equation}\label{125.20}
\left(\sum\nolimits_{j=0}^\infty\sum\nolimits_{\nu=1}^m((p^\nu_j)'-(q^\nu_j)')^2\right)^\frac12\le \delta(\lambda,a,b)\left(\sum\nolimits_{j=0}^\infty\sum\nolimits_{\nu=1}^m( p^\nu_j-q^\nu_j)^2\right)^\frac12.
\end{equation}

Из равенства в \eqref{125.19} видно, что, каково бы ни было число $\lambda>0$, параметры $0<a<1$ и $b>0$ могут быть выбраны так, чтобы было $\delta(\lambda,a,b)<1$. И тогда, в силу неравенства \eqref{125.20}, соответствующий оператор $A:l^m_2\to l_2^m$, действующий по правилу \eqref{125.12}, является сжимающим, и, следовательно, существует единственная неподвижная точка этого оператора $C(q)\in l^m_2$, к которой сходится итерационный процесс $C^{\nu+1}=A(C^\nu)$ $(\nu=0,1,\ldots)$ в метрике пространства $l^m_2$.
Однако с точки зрения приложений важно рассмотреть конечномерный аналог оператора $A$. Обозначим через $\mathbb{R}^N_m$ пространство матриц $C$ размерности $m\times N$, в котором определена норма
$$\|C\|_N^m=\left(\sum\nolimits_{j=0}^{N-1} \sum\nolimits_{\nu=1}^{m}(c_j^\nu)^2\right)^\frac12.$$
Мы рассмотрим оператор $A_N:\mathbb{R}^N_m\to \mathbb{R}^N_m$, cопоставляющий точке\\
$C_N=(c_0,\ldots,c_{N-1})\in \mathbb{R}^N_m $ точку  $C'_N=(c_0',\ldots,c_{N-1}')\in \mathbb{R}^N_m $ по правилу
\begin{equation}\label{125.21}
c_k'=\int\limits_{0}^\infty e^{-a\tau}f\left[1-e^{-a\tau},\eta(0)+ a\sum_{j=0}^{N-1} c_jl_{1,j+1}(\tau;b)\right]l_k(\tau;b)e^{-b\tau} d\tau,\,k=0,1,\ldots, N-1.
\end{equation}

Рассмотрим две точки $P_N,Q_N\in \mathbb{R}^N_m$, где $P_N=(p_0,p_1,\ldots, p_{N-1})$,\\   $Q_N=(q_0,q_1,\ldots, q_{N-1})$ и положим $P'_N=A_N(P_N)$, $Q'_N=A_N(Q_N)$. Дословно повторяя рассуждения, которые привели нас к неравенству \eqref{125.20}, мы получим
\begin{equation}\label{125.22}
\left(\sum\nolimits_{j=0}^{N-1}\sum\nolimits_{\nu=1}^m((p^\nu_j)'-(q^\nu_j)')^2\right)^\frac12\le \delta(\lambda,a,b)\left(\sum\nolimits_{j=0}^{N-1}\sum\nolimits_{\nu=1}^m( p^\nu_j-q^\nu_j)^2\right)^\frac12.
\end{equation}

Неравенство \eqref{125.22} показывает, что если $\delta(\lambda,a,b)<1$, то оператор  $A_N:\mathbb{R}^N_m\to \mathbb{R}^N_m$ является сжимающим и, как следствие, итерационный процесс $C_N^{\nu+1}=A_N(C_N^{\nu})$  при $\nu\to\infty$ сходится к его неподвижной точке, которую мы обозначим через  $\bar C_N(q)=(\bar c_0(q),\ldots,\bar c_{N-1}(q))$. С другой стороны, рассмотрим точку $C_N(q)=(c_0(q),\ldots,c_{N-1}(q))$, составленную из искомых коэффициентов Фурье вектор-функции $q$ по системе $\{l_{1,k}(t;b)\}$. Нам остается оценить погрешность, проистекающую в результате замены точки $C_N(q)$ точкой $\bar C_N(q)$. Другими словами, требуется оценить величину
$$\|C_N(q)-\bar C_N(q)\|_N^m= \left(\sum\nolimits_{j=0}^{N-1}\sum\nolimits_{\nu=1}^m(c_j^\nu(q)-\bar c_j^\nu(q))^2\right)^\frac12.$$
С этой целью рассмотрим точку $C'_N(q)=A_N(C_N(q))=(c_0'(q),\ldots,c_{N-1}'(q))$ и запишем
\begin{equation}\label{125.23}
\|C_N(q)-\bar C_N(q)\|_N^m\le \|C_N(q)- C_N'(q)\|_N^m+\|C_N'(q)-\bar C_N(q)\|_N^m.
\end{equation}
Далее, пользуясь неравенством \eqref{125.22}, имеем
$$
\|C_N'(q)-\bar C_N(q)\|_N^m=\|A_N(C_N(q))-A_N(\bar C_N(q))\|_N^m\le
$$
\begin{equation}\label{125.24}
\delta(\lambda,a,b)\|C_N(q)-\bar C_N(q)\|_N^m.
\end{equation}
Из \eqref{125.23} и \eqref{125.24} выводим
\begin{equation}\label{125.25}
\|C_N(q)-\bar C_N(q)\|_N^m\le \frac1{1-\delta(\lambda,a,b)}\|C_N(q)- C_N'(q)\|_N^m.
\end{equation}
Чтобы оценить норму в правой части неравенства \eqref{125.25}, заметим, что в силу неравенства Бесселя
\begin{equation}\label{125.26}
(\|C_N(q)- C_N'(q)\|_N^m)^2\le \int_{0}^\infty F_{C(q),C_N(q)}(\tau)(F_{C(q),C_N(q)}(\tau))^*\rho_b(\tau) d\tau,
\end{equation}
где
\begin{multline}\label{125.27}
F_{C(q),C_N(q)}(\tau)=e^{-a\tau}f\left[1-e^{-a\tau},\eta(0)+ a\sum\nolimits_{j=0}^\infty c(q)_jl_{1,j+1}(\tau;b)\right] \\
-e^{-a\tau}f\left[1-e^{-a\tau},\eta(0)+ a\sum\nolimits_{j=0}^{N-1} c(q)_jl_{1,j+1}(\tau;b)\right].
\end{multline}
Из \eqref{125.27} и \eqref{125.2} следует, что
$$
F_{C(q),C_N(q)}(t)(F_{C(q),C_N(q)}(t))^*\le \lambda^2 \sum\nolimits_{\nu=1}^m  \left(\sum\nolimits_{j=N}^\infty ac_j^\nu(q)l_{1,j+1}(t;b)\right)^2.
$$
Отсюда с учетом того, что $ac_k= \eta_{1,k+1}$ $(k=0,1,\ldots)$, имеем
\begin{equation}\label{125.28}
F_{C(q),C_N(q)}(t)(F_{C(q),C_N(q)}(t))^*\le \lambda^2   \sum\nolimits_{\nu=1}^m \left(\sum\nolimits_{j=N}^\infty  {\eta_\nu}_{1,j+1}l_{1,j+1}(t;b)\right)^2.
\end{equation}
Сопоставляя \eqref{125.28} с \eqref{125.26}, получаем
\begin{equation}\label{125.29}
(\|C_N(q)- C_N'(q)\|_N^m)^2\le \lambda^2\sum\nolimits_{\nu=1}^m\int_{0}^\infty\left(V_N(\eta_\nu,t)\right)^2\rho_b(t) dt,
\end{equation}
где величина
\begin{equation}\label{125.30}
V_N(\eta_\nu,t)= \sum\nolimits_{j=N}^\infty  {\eta_\nu}_{1,j+1}l_{1,j+1}(t;b),
\end{equation}
представляет собой остаточный член ряда Фурье функции $\eta_\nu(t)=y_\nu(1-e^{-at})$ по полиномам $l_{1,k}(t;b)$, ортогональным по Соболеву относительно скалярного произведения \eqref{12.5} c  $\rho_b(t)=e^{-bt}$, определенным равенствами \eqref{12.3} и \eqref{12.4} для $r=1$, где
$y(x)=(y_1(x),\ldots,y_m(x))$ -- решение задачи Коши  \eqref{125.1}.

Исходя из \eqref{125.25} и \eqref{125.29},  мы можем сформулировать следующий результат.
\begin{theorem}\label{th20} Пусть область $\bar G$ такова, что $[0,1]\times\mathbb{R}^m\subset  \bar G$, вектор-функция $f(x,y)$ непрерывна в области $\bar G$ и удовлетворяет условию Липшица \eqref{125.2}, а $0< a<1$ и $b>1$ таковы, что  $\delta(\lambda,a,b)<1$. Далее, пусть $l_2^m$ гильбертово пространство, состоящее из последовательностей $C=(c_0,\ldots,c_j,\ldots)$ $m$-мерных векторов $c_j=(c_j^1,\ldots,c_j^m)$, для которых введена норма \eqref{125.10}, оператор $A: l_2^m\to l_2^m$ сопоставлят точке $C\in l_2^m$ точку $C'\in l_2^m$ по правилу \eqref{125.12}. Кроме того, пусть $A_N:\mathbb{R}^N_m\to \mathbb{R}^N_m$ -- конечномерный аналог оператора $A$, cопоставляющий точке $C_N=(c_0,\ldots,c_{N-1})\in \mathbb{R}^N_m $ точку  $C'_N=(c_0',\ldots,c_{N-1}')\in \mathbb{R}^N_m $ по правилу \eqref{125.21}.
	Тогда операторы $A: l_2^m\to l_2^m$ и $A_N:\mathbb{R}^N_m\to \mathbb{R}^N_m$ являются сжимающими, и, следовательно, существуют  их неподвижные точки $C(q)=(c_0(q),c_1(q),\ldots)=A(C(q))\in l_2^m$ и $\bar C_N(q)=(\bar c_0(q),\bar c_1(q),\ldots,\bar c_{N-1}(q))=A_N(\bar C_N(q))\in \mathbb{R}^N_m$, для которых имеет место неравенство
	\begin{equation}\label{125.31}
	\|C_N(q)-\bar C_N(q)\|_N^m\le \frac{\lambda \sigma_N(\eta)}{1-\delta(\lambda,a,b)},
	\end{equation}
	где
	\begin{equation}\label{125.32}
	\sigma_N(\eta)=\left(\sum\nolimits_{\nu=1}^m\int_{0}^\infty\left(V_N(\eta_\nu,t)\right)^2\rho_b(t) dt\right)^\frac12,
	\end{equation}
	a $C_N(q)=(c_0(q),\ldots,c_{N-1}(q))$ -- конечная последовательность, составленная из первых $N$ компонент точки  $C(q)$.
\end{theorem}

В связи с неравенством \eqref{125.31} возникает задача об оценке величины $\sigma_N(\eta)$, определенной равенством \eqref{125.32}. В нижеследующей теореме рассмотрен этот вопрос в терминах
последовательности  коэффициентов Фурье $\eta_{1,k}$ функции
$\eta'(t)=(\eta_1'(t),\ldots,\eta_m'(t))=ae^{-at}f(1-e^{-at},\eta(t))$ по модифицированным полиномам Лагерра $l_k(t;b)$ (см. \eqref{125.5}), где $\eta(t)$ -- решение задачи Коши \eqref{125.3}.
\begin{theorem}\label{th21}
	Для произвольных $a$ и $b$, таких что $0<a<1$, $b>0$, справедливо равенство
	\begin{equation}\label{125.33}
	\int_{0}^{\infty}(V_N(\eta_\nu,t))^2\rho_b(t)dt=
	\frac1{b}\left({\eta_\nu}_{1,N+1}\right)^2+
	\frac1{b}\sum\nolimits_{j=N+1}^\infty  ({\eta_\nu}_{1,j+1}-{\eta_\nu}_{1,j})^2.
	\end{equation}
	
\end{theorem}

Если $b>0$, $\rho_b(t)=e^{-bt}$ и функция $g\in \mathcal{L}_{\rho_b}^2$, то можно ее разложить в ряд Фурье по модифицированным ортонормированным полиномам Лагерра $l_{k}(t;b) (k=0,1,\ldots)$, который имеет вид
\begin{equation}\label{125.36}
g(t)=\sum\nolimits_{k=0}^{\infty}g_k^bl_{k}(t;b),
\end{equation}
где $g_k^b=\int_{0}^{\infty}g(\tau)l_{k}(\tau;b)\rho_b(\tau)d\tau \quad (k=0,1,\ldots)$ -- коэффициенты Фурье-Лагерра функции $g$. Частичную сумму ряда \eqref{125.36} обозначим через $S_N^b(g,t)$, т.е. $S_N^b(g,t)=\sum_{k=0}^Ng_k^bl_{k}(t;b)$. Тогда величина
\begin{equation}\label{125.37}
E_N(g)_{\mathcal{L}_{\rho_b}^2}=\left(\int_{0}^{\infty} (g(t)-S_N^b(g,t))^2\rho_b(t)dt\right)^\frac12=\left(\sum\nolimits_{k=N+1}^{\infty}(g_k^b)^2\right)^\frac12
\end{equation}
представляет собой наилучшее приближение функции $g\in \mathcal{L}_{\rho_b}^2$ алгебраическими полиномами степени $N$. Из теоремы \ref{th21}, пользуясь равенством \eqref{125.37}, выводим
\begin{corollary}\label{cor3} Имеет место неравенство
	$$
	\int_{0}^{\infty}(V_N(\eta_\nu,t))^2\rho_b(t)dt\le \frac5bE^2_N(\eta'_\nu)_{\mathcal{L}_{\rho_b}^2}.
	$$
\end{corollary}
Для $\eta'(t)=(\eta'_1(t),\ldots,\eta'_m(t))$ положим
\begin{equation}\label{125.38}
E_N(\eta')_{\mathcal{L}_{\rho_b}^2}=\left(\sum\nolimits_{\nu=1}^{m}E^2_N(\eta'_\nu)_{\mathcal{L}_{\rho_b}^2}\right)^\frac12.
\end{equation}
Из \eqref{125.32}, \eqref{125.38} и следствия \ref{cor3} выводим
\begin{corollary}\label{cor11} Имеет место неравенство
	$$
	\sigma_N(\eta)\le  \sqrt{5/b}E_N(\eta')_{\mathcal{L}_{\rho_b}^2}.
	$$
\end{corollary}

\section{Ортогональная по Соболеву система функций, порождённая системой функций Лагерра}
Пусть $p>1$, $L^p$ -- пространство измеримых функций $f$, определенных на полуоси $[0, \infty)$ и таких, что
$$
\|f\|_{L^p}=\left(\int\limits_0^{\infty}|f(x)|^pdx\right)^\frac{1}{p}<\infty,
$$
$W^r_{L^p}$ -- пространство функций $f$, непрерывно дифференцируемых $r-1$ раз, для которых $f^{(r-1)}$ абсолютно непрерывна на произвольном сегменте $[a, b]\subset[0, \infty)$, а $f^{(r)}\in L^p$.

Хорошо известно, что система функций $\{\lambda_n^\alpha(x)\}_{n=0}^\infty$ при $\alpha>-1$ ортонормирована относительно скалярного произведения
$$
\langle \lambda_m^\alpha, \lambda_n^\alpha\rangle=\int\limits_0^\infty \lambda_m^\alpha(x)\lambda_n^\alpha(x)dx.
$$

Система функций Лагерра $\{\lambda_n^\alpha(x)\}_{n=0}^\infty$ порождает на $[0, \infty)$ систему функций $\lambda_{r,n}^\alpha(x)$ ($r\in\mathbb{N}$, $n=0, 1, \ldots$), ортонормированную при $\alpha>-1$ относительно скалярного произведения Соболева вида:
\begin{equation}\label{Ram_eq1}
\langle f,g\rangle=\sum_{\nu=0}^{r-1}f^{(\nu)}(0)g^{(\nu)}(0)+\int_{0}^{\infty} f^{(r)}(x)g^{(r)}(x)dx.
\end{equation}

Функции $\lambda_{r,n}^{\alpha}(x)$, порожденные ортонормированными функциями Лагерра $\lambda_{n}^{\alpha}(x)$, определяются посредством равенств \eqref{Ram_eq12} и \eqref{Ram_eq13}.
В данном разделе показано, что ряд Фурье по системе $\{\lambda_{r,n}^{\alpha}(x)\}_{k=0}^\infty$ сходится равномерно относительно $x\in[0, A]$, $0\leq A<\infty$, к функции $f\in W^r_{L^p}$ для $\frac{4}{3}<p<4$.
Для системы функций $\lambda_{r,n}^{\alpha}(x)$ получены рекуррентные соотношения, которые могут быть использованы для вычисления их значений при любых $x$ и $n$.
Исследованы асимптотические свойства функций $\lambda_{1,n}^0(x)$ при $0\leq x\leq\omega$, где $\omega$ -- некоторое фиксированное положительное число. Используя эти асимптотические свойства, получены оценки для функций $\lambda_{1,n}^0(x)$ на промежутке $[0,\omega]$.

Система функций Лагерра $\lambda_n^\alpha(x)$, определенная равенством \eqref{funcLag}, порождает на $[0, \infty)$ систему функций $\lambda_{r,n}^\alpha(x)$ ($r\in\mathbb{N}$, $n=0, 1, \ldots$) посредством равенств
\begin{equation}\label{Ram_eq12}
\lambda_{r,r+n}^{\alpha}(x) =\frac{1}{(r-1)!}\int\limits_{0}^x(x-t)^{r-1}\lambda_{n}^{\alpha}(t)dt, \quad n=0,1,\ldots.
\end{equation}

\begin{equation}\label{Ram_eq13}
\lambda_{r,n}^{\alpha}(x) =\frac{x^n}{n!}, \quad n=0,1,\ldots, r-1.
\end{equation}

Системы вида \eqref{Ram_eq12}, \eqref{Ram_eq13} в общем случае, когда в качестве порождающей системы используется произвольная ортонормированная система $\varphi_k(x)$ $(k=0,1,\ldots)$, были рассмотрены в работах \cite{SharIzv2018, rep2017-ramis-Gadz16, Shar2017, rep2017-ramis-Gadz1, rep2017-ramis-shGadjGadjMir}.
В частности, из теоремы \ref{th1} \cite{SharIzv2018} следует, что система функций $\lambda_{r,n}^\alpha(x)$, определенная равенствами \eqref{Ram_eq12}, \eqref{Ram_eq13}, полна в $W^r_{L^2}$ и ортонормирована относительно скалярного произведения \eqref{Ram_eq1}.

Далее, из \eqref{Ram_eq12}, \eqref{Ram_eq13} и формулы дифференцирования под знаком интеграла \cite[п. 509, с. 667]{rep2017-sobcheb_urav-fiht2} следует, что для п.в. $x\in[0,\infty)$
\begin{equation}\label{Rameqformu}
(\lambda_{r,k}^\alpha(x))^{(\nu)} =
\begin{cases}
\lambda_{r-\nu,k-\nu}^\alpha(x),&\text{если $0\le\nu\le r-1$, $r\le k$,}\\
\lambda_{k-r}^\alpha(x),&\text{если $\nu=r\le k$,}\\
\lambda_{r-\nu,k-\nu}^\alpha(x),&\text{если $\nu\le k< r$,}\\
0,&\text{если $k< \nu\le r$}.
\end{cases}
\end{equation}

Из \eqref{Ram_eq1}, \eqref{Ram_eq12}--\eqref{Rameqformu} нетрудно увидеть, что ряд Фурье функции $f\in W^r_{L^2}$ по системе  $\{\lambda^\alpha_{r,k}(x)\}_{k=0}^\infty$:
\begin{equation*}
f(x)\sim \sum_{k=0}^{\infty}c_{r,k}^\alpha(f)\lambda_{r,k}^\alpha(x)
\end{equation*}
имеет следующий вид
\begin{equation}\label{RamFourierseries2}
f(x)\sim \sum_{k=0}^{r-1}f^{(k)}(0)\frac{x^k}{k!}+\sum_{k=r}^{\infty} c_{r,k}^\alpha(f)\lambda_{r,k}^\alpha(x),
\end{equation}
где
\begin{equation}\label{RamFouriercoeff2}
c_{r,k}^\alpha(f)=\int\limits_0^\infty f^{(r)}(t)\lambda_{k-r}^\alpha(t)dt, \quad k=r, r+1, \ldots.
\end{equation}
Заметим, что ряд Фурье \eqref{RamFourierseries2} можно определить для любой функции $f\in W^r_{L^p}$, $p>1$. Это вытекает из соотношения 
$$
|c_{r,k}^\alpha(f)|\leq \left(\int\limits_0^\infty|f^{(r)}(t)|^p dt\right)^{\frac{1}{p}}
\left(\int\limits_0^\infty|\lambda_{k-r}^\alpha(t)|^q dt\right)^{\frac{1}{q}}\leq M\|f^{(r)}\|_{L^p},\ k=r, r+1, \ldots,
$$
где $M$ некоторое положительное число, которое можно получить из \eqref{RamFouriercoeff2} с помощью неравенства Коши -- Буняковского. 
Возможность построить ряд Фурье еще не гарантирует сходимости данного ряда к самой функции. В ходе выполнения НИР в этом направлении был получен следующий результат.

\begin{theorem}\label{Ram_thm1}
	Пусть $\alpha\geq0$, $0\leq A<\infty$, $\frac{4}{3}<p<4$, $f\in W^r_{L^p}$. Тогда ряд \eqref{RamFourierseries2} равномерно на $[0, A]$ сходится к функции $f$.
\end{theorem}

\subsection{Рекуррентные соотношения для функций $\lambda_{r,r+n}^{\alpha}(x)$}\label{Ram_sec4}
Для классических ортогональных систем существуют трехчленные рекуррентные формулы. В случае соболевских систем таких формул нет. Однако в ходе выполнения НИР нам удалось получить следующие рекуррентные соотношения. 

Заметим, что по построению справедливы следующие равенства: $\lambda_{0,n}^\alpha(x)=\lambda_{n}^\alpha(x),$ $\lambda_{1,0}^\alpha(x)=1,$ $\lambda_{1,1}^\alpha(x)=\int\limits_0^x\lambda^\alpha_0(t)dt$.

\begin{theorem}\label{Ram_thm2}
	Пусть $\alpha>-1$. Тогда справедливы следующие рекуррентные соотношения:
	\begin{equation*}
	\lambda_{r,n}^\alpha(x)=\frac{x}{n}\lambda_{r,n-1}^\alpha(x), \ \ 1\leq n\leq r-1;
	\end{equation*}
	\begin{equation*}
	r\lambda_{r+1,r+1}^\alpha(x)=(x-2r-\alpha)\lambda_{r,r}^\alpha(x)+2x\lambda_{r-1,r-1}^\alpha(x), \ \ r\geq 1;
	\end{equation*}
	\begin{equation}\label{Ram_eq16}
	\sqrt{(n+1)(n+\alpha+1)}\lambda_{1,n+2}^\alpha(x)= 2x\lambda_{n}^{\alpha}(x)-\lambda_{1,n+1}^{\alpha}(x)+
	\sqrt{n(n+\alpha)} \lambda_{1,n}^{\alpha}(x), \ n\geq 1;
	\end{equation}
	$$
	r\lambda_{r+1,r+n}^\alpha(x)=\sqrt{n(n+\alpha)}\lambda_{r,r+n}^{\alpha}(x)+
	$$
	\begin{equation*}
	\left(x - 2n-\alpha+1\right)\lambda_{r,r+n-1}^{\alpha}(x)
	+\sqrt{(n-1)(n+\alpha-1)}\lambda_{r,r+n-2}^{\alpha}(x), \ r\geq 1, \ n=2, 3, \ldots.
	\end{equation*}
\end{theorem}

\begin{remark}
	Формула \eqref{Ram_eq16} справедлива и для $n=0$.
\end{remark}

\subsection{Асимптотические свойства функций $\lambda_{1,1+n}^{0}(x)$}
В этом пункте мы рассмотрим вопрос о поведении функций $\lambda_{1,1+n}^{0}(x)$ при $n\rightarrow\infty$, $0\leq x\leq \omega$, где $\omega$ -- некоторое фиксированное положительное число.
\begin{theorem}\label{Ramtheo3}
	Справедлива следующая асимптотическая формула
	\begin{equation*}
	\lambda_{1,1+n}^{0}(x) = \frac{xe^{-\frac{x}{2}}}{n+1}L_n^1(x)+\frac{x^2e^{-\frac{x}{2}}}{2(n+1)(n+2)}L_n^2(x)+R_n(x),
	\end{equation*}
	в которой для остаточного члена $R_n(x)=\frac{1}{4(n+1)(n+2)}\int\limits_0^x t^2e^{-\frac{t}{2}}L_{n}^{2}(t)dt$ справедливы следующие оценки:
	\begin{equation*}
	|R_n(x)|\leq
	c \left\{\begin{gathered}
	\frac{1}{n^3},\ если 0\leq x\leq \frac{1}{n},\\
	\frac{1}{n^\frac{7}{4}},\ \frac{1}{n}\leq x\leq \omega.
	\end{gathered}\right.
	\end{equation*}
\end{theorem}
Далее из теоремы \ref{Ramtheo3} и оценок \eqref{lag-2.10}, \eqref{lag-2.11} вытекает следующее утверждение.
\begin{corollary}
	Имеют место следующие оценки
	$$
	|\lambda_{1,n}^{0}(x)|\leq c
	\begin{cases}
	\frac{1}{n}, & 0\leq x\leq \frac{1}{n} \\
	\frac{1}{n^{3/4}}, & \frac{1}{n}< x\leq \omega.
	\end{cases}
	$$
\end{corollary}







\chapter{Система функций, ортогональная в смысле Соболева и порожденная системой Уолша}

\section{Основные результаты}
Система функций Уолша $\mathfrak{W}=\mathfrak{W}_0=\{w_n(x)\}_{n=0}^\infty$ определяется с помощью функций Радемахера $r_k(x)=\operatorname{sign}\sin 2^{k+1}\pi x$ следующим образом \cite{walsh-GolubovBook}. Положим $w_0(x)=1$. Представляя $n \ge 1$ в виде двоичного разложения $n=\sum_{i=0}^{k}\varepsilon_i2^i$, где $\varepsilon_k=1$ и $\varepsilon_i \in \{0,1\}$, определим
\begin{equation}
w_n(x)=\prod_{i=0}^{k}(r_i(x))^{\varepsilon_i}.
\end{equation}

Введем в рассмотрение новую систему функций $\mathfrak{W}_r$, $r \ge 1$, порожденную системой функций Уолша $\mathfrak{W}$ ($x \in [0,1]$):
\begin{gather}
\label{walsh-wrk-1}
w_{r,k}(x) = \frac{x^k}{k!}, \quad 0 \le k \le r-1,\\
\label{walsh-wrk-2}
w_{r,k}(x) = \frac{1}{(r-1)!}\int_{0}^{x}(x-t)^{r-1}w_{k-r}(t)dt, \quad k \ge r.
\end{gather}



Пусть $L^p$ --- пространство функций, суммируемых на отрезке $[0,1]$, т. е.
\begin{equation}
L^p=\Bigl\{
f: \int_{0}^{1}|f(x)|^pdx < \infty
\Bigr\}.
\end{equation}
Через $W_{L^p}^r$, $r \ge 1$, обозначим пространство Соболева, состоящее из непрерывно дифференцируемых $r-1$ раз функций $f=f(x)$, заданных на отрезке $[0,1]$, для которых $f^{(r-1)}(x)$ абсолютно непрерывна, а $f^{(r)}\in L^p$. Заметим, что $\mathfrak{W}_r \subset W_{L^1}^r$ при любом $r$. В пространстве $W_{L^2}^r$ определим скалярное произведение соболевского типа
\begin{equation}\label{walsh-inner-prod}
\langle f,g \rangle=\sum_{s=0}^{r-1}f^{(s)}(0)g^{(s)}(0)+\int_0^1 f^{(r)}(t)g^{(r)}(t)dt,
\end{equation}
которое превращает $W_{L^2}^r$ в гильбертово пространство.
Нетрудно показать (см. подраздел \ref{walsh-sec-Wr-props}), что ряд Фурье функции $f \in W_{L^2}^r$ по системе $\mathfrak{W}_r$ принимает следующий вид:
\begin{equation}\label{walsh-fourier-series-intro}
f(x) \sim \sum_{k=0}^{r-1}f^{(k)}(0)\frac{x^k}{k!}+
\sum_{k=r}^{\infty}c_{r,k}(f)w_{r,k}(x),
\end{equation}
где $c_{r,k}=\int_{0}^{1}f^{(r)}(t)w_{k-r}(t)dt$.

Общая теория систем функций, ортогональных относительно соболевского скалярного произведения вида \eqref{walsh-inner-prod} и порожденных классическими ортогональными системами, разработана в работах Шарапудинова И. И. (см., например, \cite{SharIzv2018}). В статьях \cite{SharIzv2018, Shar2017, SharSMJ2017, walsh-ShII-2015-demi} рассмотрены конкретные соболевские системы $\Phi_r$, $r \ge 1$, порожденные различными классическими ортогональными системами: системой Хаара, системой полиномов Чебышева, Лежандра, Якоби, Лагерра. Для функций из систем $\Phi_r$ получены явные представления, исследованы их асимптотические свойства и изучены аппроксимативные свойства частичных сумм рядов Фурье по указанным системам. 
Аналогичные результаты получены и для систем, порожденных классическими дискретными ортогональными системами, такими как система полиномов Чебышева дискретной переменной \cite{SharIZVUZ}, система полиномов Мейкснера \cite{walsh-ShII-meix-2016, walsh-Gadzh-2016-saratov}. Нерассмотренной оставалась ортогональная по Соболеву система функций $\mathfrak{W}_r$, порожденная системой Уолша. В данном разделе предпринята попытка частично восполнить этот пробел.

Основная цель работы заключается в исследовании вопроса о равномерной сходимости рядов Фурье по системе $\mathfrak{W}_r$ для функций $f \in W^r_{L^p}$, $p \ge 1$.
Прежде всего, отметим, что применяя теорему \ref{th1}, доказанную в общем случае в \cite{SharIzv2018}, к системе $\mathfrak{W}_r$, можно получить утверждение:
\begin{statement}
	Система функций $\mathfrak{W}_r$ полна в $W_{L^2}^r$ и ортонормирована относительно скалярного произведения \eqref{walsh-inner-prod}.
\end{statement}
В той же работе исследованы вопросы равномерной сходимости рядов Фурье по системам, ортогональным относительно скалярных произведений вида \eqref{walsh-inner-prod}, для функций из пространств Соболева. Из теоремы \ref{th2} работы \cite{SharIzv2018} вытекает следующее утверждение.
\begin{statement}
	Ряд Фурье по системе $\mathfrak{W}_r$ функции $f \in W_{L^2}^r$ равномерно на $[0,1]$ сходится к $f$.
\end{statement}



Используя базисность системы Уолша в пространствах $L^p$, $p>1$, данное утверждение можно усилить, распространив его на более широкий класс функций --- $W^r_{L^p}$, $p>1$.
\begin{theorem}\label{walsh-st-uni-conv-pg1}
	Для любой функции $f \in W_{L^p}^r$, $p>1$, $r=1,2,\ldots$, ряд Фурье этой функции по системе $\mathfrak{W}_r$ равномерно на $[0,1]$ сходится к $f$.
\end{theorem}

Отметим, что ряд Фурье по системе $\mathfrak{W}_r$ можно определить для любой функции $f \in W^r_{L^1}$. В связи с этим возникает естественный вопрос о том, справедливо ли утверждение теоремы \ref{walsh-st-uni-conv-pg1} для случая $p=1$. Ответ на этот вопрос является основным результатом этого раздела.
\begin{theorem}\label{walsh-st-main}
	Для любой функции $f \in W_{L^1}^r$, $r=1,2,\ldots$, ряд Фурье этой функции по системе $\mathfrak{W}_r$ равномерно на $[0,1]$ сходится к $f$.
\end{theorem}

Доказательство указанных теорем опирается на свойства функций системы $\mathfrak{W}_r$, приведенные в следующем подразделе.

\section{Некоторые свойства систем функций $\mathfrak{W}_r$}\label{walsh-sec-Wr-props}
Используя формулу дифференцирования под знаком интеграла \cite{walsh-fiht2}, из \eqref{walsh-wrk-1} и \eqref{walsh-wrk-2} выводим:
\begin{equation}\label{walsh-wrk-deriv-prop}
(w_{r,k}(x))^{(\nu)} =
\begin{cases}
w_{r-\nu,k-\nu}(x), &\nu \le \min\{k,r\},\\
0, &\nu > \min\{k,r\},
\end{cases}
\end{equation}
где полагаем $w_{0,k}(x)=w_{k}(x)$, $r \ge 0$, причем если $k > r$ и $\nu \ge r$, то равенство \eqref{walsh-wrk-deriv-prop} верно лишь почти всюду.

Покажем, что коэффициенты Фурье $c_{r,k}(f)$, $k, r \ge 0$, функции $f \in W^r_{L^2}$ по системе $\mathfrak{W}_r$ имеют вид
\begin{equation}\label{walsh-crk}
c_{r,k}(f)=
\begin{cases}
f^{(k)}(0), &k<r,\\
\int\limits_{0}^{1}f^{(r)}(t)w_{k-r}(t)dt, &k \ge r.
\end{cases}
\end{equation}
В самом деле, из \eqref{walsh-inner-prod} имеем
\begin{equation*}
c_{r,k}(f)=\langle f, w_{r,k} \rangle =
\sum\limits_{s=0}^{r-1}f^{(s)}(0)w^{(s)}_{r,k}(0)+\int\limits_{0}^{1}f^{(r)}(t)w^{(r)}_{r,k}(t)dt.
\end{equation*}
Если $k < r$, то $w^{(r)}_{r,k}(x)=(x^k/k!)^{(r)}=0$ и $w^{(s)}_{r,k}(0)=(x^k/k!)^{(s)}\big|_{x=0}=\delta_{ks}$, $0 \le s \le r-1$. Следовательно, $c_{r,k}(f)=f^{(k)}(0)$, $k < r$. При $k \ge r$ в силу \eqref{walsh-wrk-deriv-prop} имеем равенство $w^{(s)}_{r,k}(0)=w_{r-s,k-s}(0)=0$, поэтому $c_{r,k}(f)=\int\limits_{0}^{1}f^{(r)}(t)w^{(r)}_{r,k}(t)dt$. Если еще раз применить \eqref{walsh-wrk-deriv-prop}, то получим вторую строку в \eqref{walsh-crk}.

Непосредственно с помощью \eqref{walsh-crk} можно установить связь между коэффициентами $c_{r,k}(f)$ при различных $r$:
\begin{equation}\label{walsh-crk-deriv}
c_{r,k}(f)=c_{r-\nu,k-\nu}(f^{(\nu)}), \quad 0 \le \nu \le \min\{r,k\}.
\end{equation}

Частичная сумма ряда Фурье функции $f$ по системе $\mathfrak{W}_r$ порядка $n$ имеет вид:
\begin{equation*}
S_{r,n}(f,x)=
\sum_{k=0}^{n-1}c_{r,k}(f)w_{r,k}(x)=
\sum_{k=0}^{r-1}f^{(k)}(0)\frac{x^k}{k!}+
\sum_{k=r}^{n-1}c_{r,k}(f)w_{r,k}(x), \quad n > r \ge 1.
\end{equation*}

Через $S_{0,n}(f)=S_n(f)=\sum\limits_{k=0}^{n-1}c_k(f)w_k(x)$ будем обозначать частичную сумму Фурье функции $f$ порядка $n$ по системе Уолша.

Используя соотношения \eqref{walsh-wrk-deriv-prop} и \eqref{walsh-crk-deriv}, выводим:
\begin{equation*}
S_{r,n}^{(\nu)}(f,x)=
\sum_{k=0}^{n-1}c_{r,k}(f)w^{(\nu)}_{r,k}(x)=
\sum_{k=\nu}^{n-1}c_{r-\nu,k-\nu}(f^{(\nu)})w_{r-\nu,k-\nu}(x).
\end{equation*}
Последнее выражение представляет собой частичную сумму Фурье функции $f^{(\nu)}$ по системе $\mathfrak{W}_{r-\nu}$ порядка $n-\nu$. Таким образом, имеет место равенство
\begin{equation}\label{walsh-Srn-deriv}
S_{r,n}^{(\nu)}(f,x)=
S_{r-\nu,n-\nu}(f^{(\nu)},x), \quad 0 \le \nu \le r.
\end{equation}

Остановимся более подробно на свойствах функций системы $\mathfrak{W}_1$.
Приведем явный вид нескольких первых функций этой системы при $x \in [0,1]$:
\begin{gather}\label{walsh-w12}
w_{1,0}(x)=1, \;
w_{1,1}(x)=x, \;
w_{1,2}(x)=\frac{1}{2}-|x-\frac{1}{2}|=
\begin{cases}
x, &0 \le x < \frac{1}{2},\\
1-x, &\frac{1}{2} \le x < 1,
\end{cases}\\
\notag
w_{1,3}(x)=
\begin{cases}
x, &0 \le x < 1/4,\\
\frac{1}{2}-x, &1/4 \le x < 1/2,\\
x-\frac{1}{2}, &1/2 \le x < 3/4,\\
1-x, &3/4 \le x < 1,
\end{cases}
\quad
w_{1,4}(x)=
\begin{cases}
x, &0 \le x < 1/4,\\
\frac{1}{2}-x, &1/4 \le x < 1/2,\\
\frac{1}{2}-x, &1/2 \le x < 3/4,\\
x-1 , &3/4 \le x < 1.
\end{cases}
\end{gather}

Далее, отметим, что из ортогональности функций Уолша на $[0,1]$ вытекает соотношение
\begin{equation*}
w_{1,n}(0)=w_{1,n}(1)=0, \quad n \ge 2.
\end{equation*}
Это свойство позволяет нам без дополнительных оговорок периодически с периодом $T=1$ продолжить $w_{1,k}(x)$, $k \ge 2$, на всю ось. Поэтому
всюду в дальнейшем мы будем считать функции $w_{1,k}(x)$, $k \ge 2$, определенными на всей оси.
\begin{theorem}
	Если $k \ge 0$ и $0 \le p \le k$, то имеет место следующее рекуррентное соотношение
	\begin{equation*}
	w_{1,1+2^k}(x)=\frac{1}{2^p}w_{1,1+2^{k-p}}(2^p x).
	\end{equation*}
\end{theorem}

Полагая $p=k$, из приведенной выше теоремы получаем
\begin{equation}\label{walsh-w1pow2k-w12}
w_{1,1+2^k}(x)=\frac{1}{2^k}w_{1,2}(2^k x).
\end{equation}
Отсюда с учетом \eqref{walsh-w12} имеем следующую оценку:
\begin{equation}\label{walsh-w12k-est}
0 \le w_{1,1+2^k}(x) \le \frac{1}{2^{k+1}}, \quad k \ge 0.
\end{equation}
Из равенства \eqref{walsh-w1pow2k-w12} также следует, что период функции $w_{1,1+2^k}(x)$ равен $\frac{1}{2^k}$.
Кроме того, указанное равенство дает простой способ вычисления значений функции $w_{1,1+2^k}(x)$ при произвольном $k \ge 1$.

\begin{theorem}\label{walsh-st-w1n-w12k}
	Если $n=2^k+i$, $k \ge 0$, $0 \le i \le 2^k-1$, то справедливо следующее равенство:
	\begin{equation*}
	w_{1,1+n}(x)=w_i(x)w_{1,1+2^k}(x).
	\end{equation*}
\end{theorem}
Доказательство основывается на использовании формулы $w_n(x)=w_i(x)w_{2^k}(x)$ \cite[с. 10, формула (1.1.5)]{walsh-GolubovBook} и следующих двух утверждений относительно свойств системы Уолша, доказательство которых можно найти в \cite[с. 11, теорема 1.1.3]{walsh-GolubovBook} и \cite[с. 11, теорема 1.1.4]{walsh-GolubovBook} соответственно.
\begin{statement}\label{walsh-w-const-on-delta}
	Функция $w_n(x)$ при $0 \le n < 2^{k+1}$ принимает постоянное значение, равное $\pm 1$, на каждом двоичном интервале 
	$\Delta_m^{(k+1)}=\Bigl[\frac{m}{2^{k+1}},\frac{m+1}{2^{k+1}}\Bigr)$, $0 \le m < 2^{k+1}$, причем $w_n(x)=1$ при $x \in \Delta_0^{(k+1)}$ .
\end{statement}
\begin{statement}\label{walsh-int-deltak-zero}
	Для любого $m$, $0 \le m \le 2^k-1$, и для любого $k \ge 0$ справедливо равенство 
	\begin{equation*}
	\int_{m/2^k}^{(m+1)/2^k}
	w_n(x)dx = 0, \quad 2^k \le n < 2^{k+1}.
	\end{equation*}
\end{statement}








