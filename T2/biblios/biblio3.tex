\begin{thebibliography}{1} %% здесь библиографический список
\bibitem{1} Сиражудинов М.М. Асимптотический метод усреднения  обобщенных операторов  Бельтрами // Математический сборник. 2017. Т. 208, №4, С.~87--110.
\bibitem{2} Sirazhudinov M.M. An asymptotic method for homogenization for generalized Beltrami operators
 // SB MATH. 2017. Vol. 208. N 4. DOI:10.1070/SM8742
\bibitem{3} Сиражудинов М.М., Тихомирова С.В. О гельдеровости решений обобщенных уравнений Бельтрами // Дифференц. уравнения (статья находится в редакции журнала).
\bibitem{4} Сиражудинов М.М. Операторные оценки усреднения  обобщенных уравнений  Бельтрами // Дагестанские электронные математические известия. 2017. Вып. 7. С. 40-46.
\bibitem{5} Sirazhudinov M.M. An asymptotic method for homogenization for a generalized system of Beltrami operators // International Conference of Mathematical Physics Kezenoi-Am 2017
\bibitem{6} Сиражудинов М.М. Операторные оценки усреднения обобщенных уравнений Бельтрами// Международная  научная  конференция «Дифференциальные уравнения, математический анализ  и теория чисел», Республика Таджикистан, Кургантюбе,  29-30 сентября 2017 г.
\bibitem{7} Сиражудинов М.М., Джамалудинова С.П. О G-компактности одного класса эллиптических операторов второго порядка// International Conference of Modern methods, problems and applications of operator theory and harmonic analysis, Rostov-on-Don, 23 - 28 April, 2017 г.
\bibitem{8} Сиражудинов М.М. О гельдеровости решений краевой задачи Римана-Гильберта для обобщенной системы Бельтрами с нормальными матрицами коэффициентов // Материалы XII международной конференции «Фундаментальные и прикладные проблемы математики и информатики». 2017. С.191
\bibitem{9} Сиражудинов М.М., Джамалудинова С.П. О G-компактности одного класса эллиптических систем второго порядка // Материалы XII международной конференции «Фундаментальные и прикладные проблемы математики и информатики». 2017. С. 192.
\bibitem{10} Сиражудинов М.М., Рамазанов Ш.Р. О гельдеровости решений одной задачи Римана -- Гильберта для треугольных системы // Материалы XII международной конференции «Фундаментальные и прикладные проблемы математики и информатики». 2017. С. 193.

\bibitem{kad1}Колмановский В.Б., Носов В.Р. Устойчивость и периодические режимы регулируемых систем с последействием. М.: Наука, 1981.
\bibitem{kad2} Царьков Е.Ф. Случайные возмущения дифференционально-функциональных уравнений. Рига: Зинате, 1989.
\bibitem{kad3} Mao X. Stochastic Differential  Equations and Applications. Chichester: Horwood Publishing Itd. 1997. 360~p.
\bibitem{kad4} Mohammed S.-E.F. Stochastic Functional Differential Equations With Memory. Theory, Examples and Applications // Proceeding of The
Sixth on Stochastic Analysis. Geilo.  Norway. 1996. Pp.~1--91.
\bibitem{kad5} Кадиев Р.И. К вопросу об устойчивости стохастических
функционально-диффе\-ренциальных уравнений по первому приближению
// Изв. Вузов. Математика. 1999. №~10. С.~3--8.
\bibitem{kad6} Кадиев Р.И. Устойчивость решений нелинейных
функционально-дифференциаль\-ных уравнений с импульсными
воздействиями по линейному приближению // Дифференц. уравнения.
Минск. Т.~49, №~8. 2013. С.~963--970.
\bibitem{kad7} Berezansky L., Braverman E., Idels L. New global exponential
stability criteria for nonlinear delay differential systems with
applications to bam neural networks // Applied Mathematics and Computation. 2014. N~243.  Pp.~899--910.
\bibitem{kad8} Липцер Р.Ш., Ширяев А.Н. Теория мартингалов. М.: Наука, 1986.
\bibitem{kad9} A. Berman and R. Plemmons, Nonnegative Matrices in the
Mathematical Sciences. New York-London. Academic Press, Computer Science and Applied Mathematics. 1979.
\bibitem{kad10}Кадиев Р.И. Существование и единственность решения задачи Коши
для функци\-онально-дифференциальных уравнений по семимартингалу
//Изв. Вузов. Математика. 1995. № 10. С.~35--40.
\bibitem{kad11} Кадиев Р.И., Поносов А.В. Устойчивость относительно начальных
данных по части переменных решений линейных импульсных систем
дифференциальных уравнений  Ито с последействием // Дифференц. уравнения. Минск. 2017.
Т. 53, № 1. С.~20--34.
\bibitem{kad12}Кадиев Р.И., Поносов А.В. Положительная обратимость матриц
устойчивость дифференциальных уравнений Ито с запаздываниями // Дифференц. уравнения. Минск. 2017. Т. 53, № 5. С.~579--590.
\bibitem{kad13}Кадиев Р.И., Шахбанова З.И. Применение теории положительно
обратимых матриц  при исследовании устойчивости решений систем
линейных дифференциальных  уравнений Ито  // Вестник Дагестанского
государственного университета. 2017.  № 1. C.~30--36.
\bibitem{kad14} Кадиев Р.И., Поносов А.В. Использование теории положительно обратимых матриц при исследовании устойчивости решений линейных стохастических дифференциальных уравнений с последействием // Прикладная математика и вопросы управления. Пермь. 2017. № 2. С.~32--48.

\bibitem {LitEI1}
{Красносельский М.А.} Положительные решения операторных уравнений. М.: Наука. 1962.

\bibitem {LitEI2}
{Красносельский М.А., Забрейко П.П.} Геометрические методы нелинейного анализа.   М.: Наука. 1975.

\bibitem {LitEI3}
{Похожаев С.И.} О собственных функциях уравнения $\Delta u+\lambda f(u)=0.$ // ДАН СССР. 1965. Том 165, № 1.  C.~36--39.

\bibitem {LitEI4}
 {Похожаев С.И.} Об одной задаче Овсянникова // ПМТФ. 1989. № 2. C.~5--10.

\bibitem {LitEI5}
 {Gidas B., Spruck  J.} Global and local behavior of positive solutions
of nonlineare elliptic equations // Communications on Pure and
Applied Mathematics.  1982. Vol. 34. N 4. Pp.~525--598.

\bibitem {LitEI6}
{Kuo-Shung Cheng and Jenn-Tsann Lin.} On the elliptic equations
$ \Delta u=K(x)u^{\alpha} $   and $ \Delta u=K(x)\exp^{2u} $ // Transactions of American mathematical society. 1987. Vol 304. N 2. Pp.~633--668.

\bibitem {LitEI7}
{Абдурагимов Э.И.} О положительном радиально-симметричном
решении задачи Дирихле для одного нелинейного уравнения и
численном методе его получения // Изв. вузов. Математика. 1997. № 5. C.~3--6.

\bibitem {LitEI8}
{Абдурагимов Э.И.} О единственности положительного
радиально-симетричного решения в шаре
 задачи Дирихле для одного нелинейного дифференциального уравнения
 второго порядка // Изв. вузов. Математика. 2008. № 12. C.~3--6.

\bibitem {LitEI9}
{Абдурагимов Э.И.} Единственность положительного решения задачи Дирихле в шаре для
квазилинейного уравнения с  $p$ - лапласианом // Изв. вузов. Математика. 2011. № 10. C.~3--11.

\bibitem {LitEI10}
{Zhanbing Bai, Haishen Li.} Positive solutions for boundary value
problem of nonlinear fractional differential equation // J. Math. Anal. Appl. 2005. Vol. 311. Pp.~495--505.

\bibitem {LitEI11}
{Changyou Wang, Ruifang Wang, Shu Wang and Chunde Yang.} Positive
solution of singular Boundary Value Problem for a nonlinear Fraction
Differential Equation // Hindawi Publishing Corporation Boundary
Value Problems, 2011. Pp.~1--12.

\bibitem {LitEI12}
{Zhang S.} Existence of solution for a boundary value problem of
fractional order.  2006, Vol. 26. N 2. Pp.~220--228.

\bibitem {LitEI13}
{Bai Z.B., Li H.S.} Positive solutions for boundary  value
problem of nonlinear fractional differential equation // Journal
of  Mathematical  Analysis and Applications. 2005. Vol. 311. N~2. Pp.~495--505.

\bibitem {LitEI14}
{Абдурагимов Э.И., Омарова Р.А.} Численный метод построения
положительного решения двухточечной краевой задачи для одного
дифференциального уравнения второго порядка с дробной
производной // Вестник ДГУ. 2014. № 6. С.~40--46.

\bibitem {LitEI15}
{Абдурагимов Э.И., Омарова Р.А.} Положительное решение граничной
задачи для одного нелинейного дифференциального уравнения  с
дробными производными // Вестник ДГУ. 2015. Т. 30, № 6. С.~90--104.

\bibitem {LitEI16}
{Бейбалаев В.Д.} О численном решении задачи Дирихле для уравнения
Пуассона с производными дробного порядка // Вест. Сам. гос.
техн. ун-та. Сер. Физ.-мат. Науки. 2012. № 2. C.~183--187.

\bibitem {LitEI17}
{Алероев Т.С.} Краевые задачи для дифференциальных уравнений с
дробными производными. Дисс. доктора физ.-мат. наук. МГУ. 2000.

\bibitem {LitEI18}
{Ahmed ANBER, Soumia BELARBI and Zoubir DAHMANI.}  New Existence and
Uniqueness Results for Fractional Differential Equations // An.
St. Univ. Ovidius Constanta, versita. 2013. Vol. 21(3). Pp.~33--41.

\bibitem {LitEI19}
{Youji Xu.} Existence and Multiple of  Positive Solution for
Nonlinear Fractional Difference Equations with Parameter // Journal of Applied Mathematics and Physics. 2015. N 3. Pp.~757--760.

\bibitem{med-metka1} Braun H., Hauk A. Tomographic reconstruction of vector fields // IEEE Trans. Signal Proc. 1991. Vol. 39, № 2. Pp. 464--471.
\bibitem{med-metka2} Деревцов Е.Ю. Вычислительные технологии решения задач рефракционной, векторной и тензорной томографии. Дисс. на соиск. уч. ст. докт. ф-м. н. Новосибирск. 2014. 357 с.
\bibitem{med-metka3} Бегматов А.Х. Задача интегральной геометрии для семейства конусов в $n$-мерном пространстве // СМЖ. 1996. Том 37. С. 851--857.
\bibitem{med-metka4} Truong T.T., Nguen M. K. On V-line Radon transform in $\mathbb R^2$ and thear inversion //  J. Phis. A: Math. Theor. 2011. V. 44, №075206. 13 pp.
\bibitem{med-metka5} Бегматов А.Х., Пиримбетов А. О., Сейдуллаев А. К. Слабо некорректные задачи интегральной геометрии с возмущениями на семействе ломаных //  Изв. Сарат. ун-та. Нов. Сер. Математика. Механика. Информатика. 2015. Том 15, вып. 1. С. 5--12.
\bibitem{med-metka6} Бегматов А.Х., Джайков Г. М. Линейная задача интегральной геометрии с гладкими весовыми функциями и возмущением //  Владикавказский математический журнал.  2015.  Том 17, вып. 3. С. 14--22.




%%%%%%%%%%%%%%%%%%%%%
%%%%%% ПРОШЛЫЙ ГОД %%
%%%%%%%%%%%%%%%%%%%%%
%\bibitem{Abd13}
%Acerbi E. and  Mingione G. Regularity results for stationary electrorheological
%fluids // Arch. Ration. Mech. Anal. 2002. № 164. P. 213--259.
%
%\bibitem{Abd14}
%Halsey T.C. Electrorheological fluids // Science. 1992. № 258. P. 761--766.
%
%\bibitem{Abd15}
%Ruzicka. Electrorheological Fluids: Modeling and
%Mathematical Theory, Lecture Notes in Mathematics //
%Springer-Verlag. Berlin. 2000. N 1748.
%
%
%\bibitem{Abd16}
% Fan X.L. On the sub-super solution method for
%p(x)-Laplacian equaions // Mathematical Analysis and Application. 2007. V. 330. P. 665--582.
%
%\bibitem{Abd17}
%Qihu Zhang.  \emph{Existence of positive solutions for a class of
%$p(x)$-Laplacian systems} // Mathematical Analysis and Applications,
%2007. V. 333. P. 591--603.
%
%\bibitem{Abd18} Samira Ala, Ghasem Alizadeh Afrouzi and Asadollah Niknam.
%Existence of positive solutions for variable exponent
%elliptic systems // Boundary Value Problems. 2012. V. 37. P. 1--12.
%
%
%\bibitem{sir1} Сиражудинов М.М., Джамалудинова С.П. {Асимптотическое разложение решения уравнения Бельтрами, зависящего от малого параметра} //
%Вестник Дагестанского государственного университета. Серия 1: Естественные науки. 2016. Т. 31. Вып. 1. С. 65--72.
%
%\bibitem{sir2} Ладыженская О.А., Уральцева Н.Н. {Линейные и квазилинейные уравнения эллиптического типа.} М.: Наука. 1973. 576 с.
%
%\bibitem{sir3} Сиражудинов М.М., Джамалудинова С.П., Махмудова М.Э. {Частичное
%усреднение недивергентного эллиптического уравнения второго порядка} //
%Вестник Дагестанского государственного университета. Серия 1: Естественные науки. 2016. Т. 31. Вып. 3. С. 47--53.
%
%\bibitem{sir4} Сиражудинов М.\,М., Алиев Ш.\,Г., Джамалудинова С.\,П.
%\emph{О гельдеровости решений системы уравнений Бельтрами} //
%Вестник Дагестанского государственного университета. Серия 1: Естественные науки. 2016. Т. 31, вып. 4. С. 77--83.
%
%\bibitem{sir5} Сиражудинов М.М. \emph{О задаче Римана–Гильберта для эллиптических систем первого порядка в многосвязной области} // Математический сборник, 1993. Т. 184, № 11. С. 39–62.
%
%
%\bibitem{Abd3}
%Абдурагимов Э.И. Положительное решение двухточечной
%краевой задачи для одного нелинейного ОДУ четвертого порядка и
%численный метод его построения // Вестник Самарского
%государственного университета. 2010. Т. 76, вып 2. С. 5--12.
%
%\bibitem{Abd4}
%Абдурагимов Э.И., Гаджиева Т.Ю., Магомедова П.К. Численный  метод  построения положительного решения
%двухточечной краевой задачи для одного нелинейного оду четвертого порядка // Вестник ДГУ. № 6. Махачкала. С. 85--92.
%
%\bibitem{Abd5}
%Абдурагимов Э.И. Существование положительного решения
%двухточечной краевой задачи для одного нелинейного ОДУ четвертого
%порядка // Вестник СамГу. 2014. Т. 121, вып. 10. С. 9--16.
%


\end{thebibliography} 