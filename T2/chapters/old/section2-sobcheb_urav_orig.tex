\chapter{Полиномы, ортогональные по Соболеву, ассоциированные с полиномами Чебышева первого рода и задача Коши для ОДУ }

%Рассмотрены полиномы $T_{r,n}(x)$ $(n=0,1,\ldots)$, порожденные многочленами Чебышева $T_{n}(x)=\cos( n\arccos x)$, образующие ортонормированную систему по Соболеву относительно скалярного произведения
%следующего вида $<f,g>=\sum_{\nu=0}^{r-1}f^{(\nu)}(-1)g^{(\nu)}(-1)+\int_{-1}^{1}f^{(r)}(t)g^{(r)}(x)\mu(x)dx$,
%где $\mu(x)=\frac2\pi(1-x^2)^{-\frac12}$. Для $T_{r,n}(x)$ $(n=0,1,\ldots)$  установлена связь с многочленами Чебышева $T_{n}(x)$ и получены явные представления, которые могут быть использованы при вычислении значений полиномов $T_{r,n}(x)$ и исследовании их асимптотических свойств. Показано, что суммы Фурье по полиномам $T_{r,n}(x)$ $(n=0,1,\ldots)$ является удобным и весьма эффективным инструментом приближенного решения задачи Коши для систем обыкновенных дифференциальных уравнений (ОДУ). 



%%%%%%%%%%%%%%%%%%%%%%%%%%%%%
%%%%%%%%%%%%%%%%%%%%%%%%%%%%%
%%%%%%%%%%%%%%%%%%%%%%%%%%%%%





\section{Введение}
Интерес к теории полиномов, ортогональных относительно скалярных произведений, в которых присутствуют одна или несколько точек  с дискретными массами,  в последнее время интенсивно растет  (см. \cite{KwonLittl1} -- \cite{Shar2016}  и цитированную там литературу).
 Это новое направление принято обозначать словами: <<полиномы, ортогональные по Соболеву>>. Не ослабивающее внимание специалистов  к этому направлению теории ортогональных полиномов можно объяснить в том числе и тем обстоятельством, что ряды Фурье по полиномам, ортогональным по Соболеву, оказались естественным и весьма удобным инструментом для представления решений  дифференциальных (разностных) уравнений. Это можно показать, в частности, на примере  задачи Коши для  дифференциального уравнения
\begin{equation}\label{1.1}
F(x,y,y',\ldots,y^{(r)})=0
 \end{equation}
с начальными условиями $y^{(k)}(-1)=y_k$, $k=0,1,\ldots,r-1$.  Наряду с различными сеточными методами, для решения этой задачи часто применяют так называемые спектральные методы \cite{Tref1} -- \cite{MMG2016}. Напомним, что суть спектрального метода решения задачи Коши  для ОДУ \eqref{1.1} заключается в том, что в первую очередь искомое решение $y(x)$ представляется в виде ряда Фурье
\begin{equation}\label{1.2}
 y(x)=\sum_{k=0}^\infty \hat y_k\psi_k(x)
 \end{equation}
по подходящей ортонормированной системе $\{\psi_k(x)\}_{k=0}^\infty$ (чаще всего в качестве $\{\psi_k(x)\}_{k=0}^\infty$ используют    тригонометрическую систему, ортогональные полиномы, вэйвлеты, корневые функции того или иного дифференциального оператора  и некоторые другие). На втором этапе осуществляется подстановка вместо $y(x)$ ряда \eqref{1.2} в уравнение \eqref{1.1}. Это приводит к системе уравнений относительно неизвестных коэффициентов $\hat y_k$ ($k=0,1,\ldots$). На третьем этапе требуется решить эту систему с учетом начальных условий  $y^{(k)}(-1)=y_k$, $k=0,1,\ldots,r-1$ исходной задачи Коши.
Одна из основных трудностей, которая возникает на этом этапе, состоит в том, чтобы
выбрать такой ортонормированный базис $\{\psi_k(x)\}_{k=0}^\infty$, для которого искомое решение $y(x)$ уравнения \eqref{1.1}, представленное в виде ряда  \eqref{1.2}, удовлетворяло бы начальным условиям $y^{(k)}(-1)=y_k$, $k=0,1,\ldots,r-1$. Более того, поскольку в результате решения системы уранений относительно неизвестных коэффициентов $\hat y_k$  будет найдено только конечное их число с $k=0,1,\ldots, n$, то весьма важно, чтобы частичная сумма ряда \eqref{1.2} вида $ y_n(x)=\sum_{k=0}^n\hat y_k\psi_k(x)$,
 будучи приближенным решением рассматриваемой задачи Коши, также удовлетворяла начальным условиям $y_n^{(k)}(-1)=y_k$, $k=0,1,\ldots,r-1$. Покажем, что  базис $\{\psi_k(x)=T_{r,k}(x)\}_{k=0}^\infty$, состоящий из полиномов
$T_{r,k}(x)$, ортонормированных по Соболеву относительно скалярного произведения
\begin{equation}\label{1.3}
<f,g>=\sum_{\nu=0}^{r-1}f^{(\nu)}(-1)g^{(\nu)}(-1)+\int_{-1}^{1}f^{(r)}(t)g^{(r)}(x)\mu(x)dx,
\end{equation}
где $\mu(x)=\frac2\pi(1-x^2)^{-\frac12}$ и порожденных многочленами Чебышева  $T_{k}(x)$ посредством равенств
   \begin{equation}\label{1.4}
T_{r,k}(x) =\frac{(x+1)^k}{k!}, \quad k=0,1,\ldots, r-1,
\end{equation}
  \begin{equation}\label{1.5}
 T_{r,r}(x) =\frac{(x+1)^r}{\sqrt{2}r!},\quad T_{r,r+n}(x) =\frac{1}{(r-1)!}\int\limits_{-1}^x(x-t)^{r-1}T_{n}dt, \quad n=1,\ldots.
\end{equation}
   обладает указанными свойствами. С этой целью заметим, что ряд Фурье \eqref{1.2}, в случае, когда $\{\psi_k(x)=T_{r,k}(x)\}_{k=0}^\infty$, приобретает, как показано ниже, следующий вид
   \begin{equation}\label{1.6}
y(x)= \sum_{k=0}^{r-1} y^{(k)}(-1)\frac{(x+1)^k}{k!}+ \sum_{k=r}^\infty \hat y_{r,k}T_{r,k}(x),
\end{equation}
где
  \begin{equation}\label{1.7}
 \hat y_{r,k}=\int_{-1}^1 y^{(r)}(t)T_{k-r}(t)\mu(t)dt.
\end{equation}
С другой стороны, из определения \eqref{1.5} вытекает, что $(T_{r,k}(-1))^{(\nu)}=0$ для всех $0\le\nu\le r-1$, поэтому функция $y(x)$, представленная в виде ряда \eqref{1.6}, так и частичная  сумма этого ряда вида
 \begin{equation}\label{1.8}
y_n(x)= \sum_{k=0}^{r-1} y^{(k)}(-1)\frac{(x+1)^k}{k!}+ \sum_{k=r}^n \hat y_{r,k}T_{r,k}(x)
\end{equation}
удовлетворяют начальным условиям задачи Коши для уравнения \eqref{1.1}.

Таким образом, полиномы, ортогональные по Соболеву относительно скалярного произведения \eqref{1.3}, тесно связаны с задачей Коши для уравнения \eqref{1.1}.

         Следует отметить, что смешанные ряды вида \eqref{1.6}, ассоциированные с классическими ортогональными полиномами, являлись основным объектом исследования работ  \cite{Shar11} -- \cite{Shar18}.
Полиномы $T_{r,k}(x)$ (и их обобщения), определямые равенствами типа \eqref{1.4} и \eqref{1.5}, в этих работах существенно использовались в качестве вспомогательного аппарата при исследовании вопросов сходимости смешанных рядов вида \eqref{1.6},  не отмечая при  этом свойство их ортогональности по Соболеву относительно скалярных произведений типа \eqref{1.3}. В настоящей работе, напротив, смешанные ряды будут исталковываться как ряды Фурье по полиномам, ортогональным по Соболеву. С точки зрения геометрии гильбертовых пространств общая идея, которая лежит в основе построения смешанных рядов, заключается в следующем. Предположим, что система функций  $\left\{\varphi_k(x)\right\}$ ортонормирована  на $(a,b)$  c весом   $\rho(x)$, т.е.
 \begin{equation}\label{1.9}
\int\limits_a^b\varphi_k(x)\varphi_l(x)\rho(x)dx=\delta_{kl},
\end{equation}
где $\delta_{kl}$ -- символ Кронекера. Через $L^p_\rho(a,b)$ обозначим пространство  функций $f(x)$, измеримых  на  $(a,b)$, для которых
 \begin{equation*}
\int\limits_a^b|f(x)|^p\rho(x)dx<\infty.
\end{equation*}
Если $\rho(x)\equiv1$, то будем писать $L^p_\rho(a,b)=L^p(a,b)$ и $L(a,b)=L^1(a,b)$.
Из \eqref{1.9} следует, что $\varphi_k(x)\in L^2_\rho(a,b)$ $(k=0,1,\ldots)$. Мы добавим к этому условию еще одно, считая, что $\varphi_k(x)\in L(a,b)$ $(k=0,1,\ldots)$. Тогда мы можем определить следующие порожденные системой $\{\varphi_k(x)\}$ функции
 \begin{equation}\label{1.10}
\varphi_{r,r+k}(x) =\frac{1}{(r-1)!}\int\limits_a^x(x-t)^{r-1}\varphi_{k}(t)dt, \quad k=0,1,\ldots.
\end{equation}
 Кроме того, определим конечный набор функций
  \begin{equation}\label{1.11}
\varphi_{r,k}(x) =\frac{(x-a)^k}{k!}, \quad k=0,1,\ldots, r-1.
\end{equation}

 Из \eqref{1.10} и \eqref{1.11} следует, что для п.в. $x\in (a,b)$
 \begin{equation}\label{1.12}
(\varphi_{r,k}(x))^{(\nu)} =\begin{cases}\varphi_{r-\nu,k-\nu}(x),&\text{если $0\le\nu\le r-1$, $r\le k$,}\\
\varphi_{k-r}(x),&\text{если  $\nu=r\le k$,}\\
\varphi_{r-\nu,k-\nu}(x),&\text{если $\nu\le k< r$,}\\
0,&\text{если $k< \nu\le r-1$}.
  \end{cases}
\end{equation}
Через $W^r_{L^p_\rho(a,b)}$ обозначим пространство Соболева $W^r_{L^p_\rho(a,b)}$, состоящее из функций $f(x)$, непрерывно дифференцируемых на $[a,b]$ $r-1$ раз, причем $f^{(r-1)}(x)$ абсолютно непрерывна на $[a,b]$  и $f^{(r)}(x)\in L^p_\rho(a,b)$.
Скалярное произведение в пространстве $W^r_{L^2_\rho(a,b)}$ определим с помощью равенства
\begin{equation}\label{1.13}
<f,g>=\sum_{\nu=0}^{r-1}f^{(\nu)}(a)g^{(\nu)}(a)+\int_{a}^{b} f^{(r)}(t)g^{(r)}(t)\rho(t) dt.
\end{equation}
Тогда, пользуясь определением функций  $\varphi_{r,k}(x)$ (см. \eqref{1.10} и \eqref{1.11}) и равенством  \eqref{1.12} нетрудно увидеть(см. теорему 2.1),  что система $\{\varphi_{r,k}(x)\}_{k=0}^\infty$ является ортонормированной в пространстве $W^r_{L^2_\rho(a,b)}$.  Мы будем называть систему $\{\varphi_{r,k}(x)\}_{k=0}^\infty$ \textit{ортонормированной по Соболеву } относительно скалярного произведения \eqref{1.13} и  \textit{порожденной} ортонормированной системой $\{\varphi_{k}(x)\}_{k=0}^\infty$.
В дальнейшем будет показано (см.\S2),  что ряд Фурье функции $f(x)\in W^r_{L^2_\rho(a,b)}$ по системе  $\{\varphi_{r,k}(x)\}_{k=0}^\infty$ имеет смешанный характер, а, более точно, имеет следующий вид
  \begin{equation}\label{1.14}
f(x)\sim \sum_{k=0}^{r-1} f^{(k)}(a)\frac{(x-a)^k}{k!}+ \sum_{k=r}^\infty \hat f_{r,k}\varphi_{r,k}(x),
\end{equation}
где
  \begin{equation}\label{1.15}
 \hat f_{r,k}=\int\limits_a^b f^{(r)}(t) \varphi^{(r)}_{r,k}(t)\rho(t)dt=\int\limits_a^b f^{(r)}(t) \varphi_{k-r}(t)\rho(t)dt,
\end{equation}
поэтому ряд  вида \eqref{1.14} будем (следуя \cite{Shar11} -- \cite{Shar16})  называть \textit{смешанным рядом} по  системе $\{\varphi_{k}(x)\}_{k=0}^\infty$, считая это название условным и сокращенным обозначением полного названия: <<\textit{ряд Фурье по системе  $\{\varphi_{r,k}(x)\}_{k=0}^\infty$, ортонормированной по Соболеву, порожденной ортонормированной системой $\{\varphi_{k}(x)\}_{k=0}^\infty$}>>.



В цитированных выше   работах  \cite{Shar11} -- \cite{Shar18} (см. также  \cite{Shar19})   были рассмотрены некоторые частные случаи ортонормированных систем функций вида $\{\varphi_{r,k}(x)\}_{k=0}^\infty$, порожденных классическими ортонормированными системами Якоби, Лежандра, Чебышева, Лагерра и Хаара. С другой стороны, как уже отмечалось, в последние годы интенсивное развитие получила  теория полиномов, ортогональных относительно различных скалярных произведений соболевского типа (полиномы, ортогональные по Соболеву). Скалярные произведения соболевского типа характеризуются тем, что они включают в себя слагаемые, которые <<контролируют>>, поведение соответствующих ортогональных полиномов  в нескольких заданных точках числовой оси. Например, в некоторых случаях оказывается так, что полиномы, ортогональные по Соболеву на интервале $(a,b)$, могут иметь нули, совпадающие с одним или с обоими концами этого интервала. Это обстоятельство имеет важное значение для некоторых приложений, в которых требуется, чтобы значения  частичных сумм ряда Фурье функции $f(x)$ по рассматриваемой системе ортогональных полиномов совпали в концах интервала $(a,b)$ со значениями $f(a)$ и $f(b)$.  Заметим, что обычные ортогональные с положительным на  $(a,b)$ весом полиномы этим важным свойством не обладают. Скалярное произведение \eqref{1.13}, рассматриваемое в настоящей работе, имеет одну особую точку, а именно, точку $a$, в окрестности которой <<контролируется>>, поведение функций $\varphi_{r,k}(x)$, ортогональных по Соболеву и порожденных исходной ортонормированной системой $\{\varphi_{k}(x)\}_{k=0}^\infty$ посредством равенства \eqref{1.10}.


Отметим некоторые важные свойства смешанного ряда \eqref{1.14}, непосредственно вытекающие из \eqref{1.12}. Первое из них связано с дифференциальным свойством смешанного ряда, а именно, если $r>1$, то в результате почленного дифференцирования смешанного ряда \eqref{1.14} мы получим смешанный ряд для производной $f'(x)$, соответствующий случаю, когда вместо $r$ фигурирует $r-1$, другими словами
\begin{equation}\label{1.16}
f'(x)\sim  \sum_{k=1}^\infty f'_{r-1,k-1}\varphi_{r-1,k-1}(x)=\sum_{k=1}^\infty (\hat f_{r,k}\varphi_{r,k}(x))'.
\end{equation}
Второе свойство связано c почленным интегрированием с переменным верхним пределом и имеет вид
\begin{equation}\label{1.17}
\int\limits_a^xf'(t)dt\sim \sum_{k=1}^\infty f'_{r-1,k-1}\int\limits_a^x\varphi_{r-1,k-1}(t)dt=\sum_{k=1}^\infty \hat f_{r,k}\varphi_{r,k}(x).
\end{equation}
Важное значение имеет свойство  смешанного ряда \eqref{1.14}, котрое заключается в том, что его частичная сумма вида
\begin{equation}\label{1.18}
Y_{r,N}(f,x)=\sum_{k=0}^{r-1} f^{(k)}(a)\frac{(x-a)^k}{k!}+ \sum_{k=r}^{N} \hat f_{r,k}\varphi_{r,k}(x)
\end{equation}
 при   $r\le N$  совпадает с исходной функцией $f(x)$   в точке $x=a$ $r$-кратно , т.е.
\begin{equation}\label{1.19}
(Y_{r,N}(f,x))^{(\nu)}_{x=a}=f^{(\nu)}(a)\quad (0\le\nu\le r-1).
\end{equation}
Кроме того, из \eqref{1.12} и \eqref{1.18} следует, что $(0\le\nu\le r-1)$
\begin{equation}\label{1.20}
 Y_{r,N}^{(\nu)}(f,x)=\sum_{n=0}^{r-1-\nu} f^{(n+\nu)}(a)\frac{(x-a)^n}{n!}+ \sum_{n=r-\nu}^{N-\nu} f_{r-\nu,n}^{(\nu)}\varphi_{r-\nu,n}(x)=Y_{r-\nu,N-\nu}(f^{(\nu)},x),
 \end{equation}
отсюда, в свою очередь, выводим $(0\le\nu\le r-2)$
 $$
f^{(\nu)}(x)-Y_{r,N}^{(\nu)}(f,x)= \frac{1}{(r-\nu-2)!}\int_a^x (x-t)^{r-\nu-2}(f^{(r-1)}(t)-Y_{r,N}^{(r-1)}(f,t))dt=
$$
  \begin{equation}\label{1.21}
\frac{1}{(r-\nu-2)!}\int_a^x (x-t)^{r-\nu-2}(f^{(r-1)}(t)-Y_{1,N-r+1}(f^{(r-1)},t))dt.
 \end{equation}


  В \cite{Shar11} -- \cite{Shar18}, воспользовавшись равенствами типа \eqref{1.20} и \eqref{1.21}, было показано, что частичные суммы смешанных рядов по классическим ортогональным полиномами, в отличие от сумм Фурье по этим же полиномам, успешно могут быть использованы в задачах, в которых требуется одновременно приближать дифференцируемую функцию и ее несколько производных. Как было показано выше, такие задачи непосредственно возникают, например, в связи с решением краевых задач для дифференциальных уравнений спектральными методами.  Основное внимание  в \cite{Shar11} -- \cite{Shar16}  уделялось исследованию аппроксимативных свойств смешанных рядов по ультрасферическим полиномам Якоби  $P_n^{\alpha,\alpha}(x)$, тогда как в работе \cite{Shar18} были найдены условия на параметры $\alpha$ и $\beta$, которые обеспечивают равномерную сходимость смешанных рядов по общим полиномам Якоби $P_n^{\alpha,\beta}(x)$ с $\alpha,\beta>-1$.

   В связи с отмеченными выше и рядом других  задач, в которых полиномы, ортогональные по Соболеву, предстают перед нами как естественный и эффективный инструмент для их решения, возникают важные задачи об изучении различных  свойств самых этих полиномов. Наиболее трудными из них, как это и бывает в теории ортогональных полиномов, являются задачи, связанные с их асимптотическим поведением.   В связи с этой проблемой  отметим  работу
\cite{Lopez1995}, в которой, используя  идеи и технику А.\,А. Гончара \cite{Gonchar1975}, исследована задача о сравнительной асимптотике полиномов, ортогональных относительно скалярного произведения типа  Соболева с дискретными массами.

Одной из целей настоящей статьи является получение различных выражений для полиномов $T_{r,k}(x)$, ортогональных по Соболеву относительно скалярного произведения \eqref{1.3} и порожденных полиномами  $T_{k}(x)$ посредством  равенства \eqref{1.5}. Эти результаты могут быть использованы при вычислении значений полиномов  $T_{r,k}(x)$ и изучении асимптотических свойств полиномов $T_{r,k}(x)$ при $x\in[-1,1]$ и $n\to\infty$. А это, в свою очередь, позволит исследовать аппроксимативные свойства частичных сумм \eqref{1.8} ряда Фурье \eqref{1.6} по полиномам $T_{r,k}(x)$ для гладких и аналитических функций $y=y(x) $(решений дифференциального уравнения \eqref{1.1}).

Прежде, чем  перейти к исследованию свойств полиномов $T_{r,k}(x)$, мы рассмотрим некоторые общие свойства  функций $\varphi_{r,k}(x)$, $(k=0,1,\ldots)$, определенных равенствами \eqref{1.10} и \eqref{1.11},  ортонормированных по Соболеву относительно скалярного произведения \eqref{1.13}.
\section{Некоторые результаты общего характера }

 Рассмотрим сначала задачу о полноте в $W^r_{L^2_\rho(a,b)}$ системы $\{\varphi_{r,k}(x)\}_{k=0}^\infty$, состоящей из функций, определенных равенствами   \eqref{1.10} и \eqref{1.11}.
 \begin{theorem} Предположим, что    функции $\varphi_k(x)$ $(k=0,1,\ldots)$ образуют полную в $L^2_\rho(a,b)$ ортонормированную   c весом   $\rho(x)$ систему на  $(a,b)$. Тогда система $\{\varphi_{r,k}(x)\}_{k=0}^\infty$, порожденная системой $\{\varphi_{k}(x)\}_{k=0}^\infty$ посредством равенств \eqref{1.10} и \eqref{1.11}, полна  в $W^r_{L^2_\rho(a,b)}$ и ортонормирована относительно скалярного произведения \eqref{1.13}.
 \end{theorem}
%\begin{proof}
%Из равенства \eqref{1.12} следует, что если $r\le k$ и $0\le\nu\le r-1$, то  $(\varphi_{r,k}(x))^{(\nu)}_{x=a}=0$, поэтому
%в силу \eqref{1.9}, \eqref{1.12},  имеем
%$$
%<\varphi_{r,k},\varphi_{r,l}>= \int_{a}^b(\varphi_{r,k}(x))^{(r)}(\varphi_{r,l}(x))^{(r)}\rho(x) dx=
%$$
%\begin{equation}\label{2.1}
%    \int_{a}^b\varphi_{k-r}(x)\varphi_{l-r}(x)\rho(x) dx=\delta_{kl},\quad k,l\ge r ,
%  \end{equation}
%  а из \eqref{1.11} и \eqref{1.12} имеем
%\begin{equation}\label{2.2}
%  <\varphi_{r,k},\varphi_{r,l}>=\sum_{\nu=0}^{r-1}(\varphi_{r,k}(x))^{(\nu)}|_{x=a}(\varphi_{r,l}(x))^{(\nu)}|_{x=a}=\delta_{kl},\quad k,l< r.
%  \end{equation}
%  Очевидно также, что
%  \begin{equation}\label{2.3}
%  <\varphi_{r,k},\varphi_{r,l}>=0,\quad \text{если}\quad k< r\le l\quad \text{или} \quad l< r\le k.
%  \end{equation}
% Это означает, что функции  $\varphi_{r,k}(t)\, (k=0,1,\ldots) $ образуют   в $W^r_{L^2_\rho(a,b)}$ ортонормированную  систему относительно скалярного произведения \eqref{1.13}.  Остается убедиться в ее полноте в $W^r_{L^2_\rho(a,b)}$. С этой целью покажем, что если для некоторой функции $f=f(x)\in W^r_{L^2_\rho(a,b)}$ и для  всех $k=0,1,\ldots$ справедливы равенства $<f,\varphi_k>=0$, то $f(x)\equiv0$. В самом деле, если $k\le r-1$, то  $<f,\varphi_{r,k}>=f^{(k)}(a)$, поэтому с учетом того, что $<f,\varphi_{r,k}>=0$,  для нашей функции  $f(x)$ формула Тейлора приобретает вид
%\begin{equation}\label{2.4}
%f(x)={1\over (r-1)!}\int\limits_{a}^x(x-t)^{r-1} f^{(r)}(t)dt.
%     \end{equation}
%С другой стороны, для всех $k\ge r$ имеем
%$$
% 0= <f,\varphi_{r,k}>=\int_{a}^bf^{(r)}(x) (\varphi_{r,k}(x))^{(r)}\rho(x) dx=
%  \int_{a}^b f^{(r)}(x)\varphi_{k-r}(x) \rho(x) dx .
%$$
%Отсюда и из того, что $\varphi_m(x)$ ($m=0,1,\ldots$)  образуют в $L^2_{\rho}(a,b)$ полную ортонормированную систему имеем $f^{(r)}(x)=0$ почти всюду на $(a,b)$. Поэтому   $f(x)\equiv0$. Теорема 1 доказана.
%\
%\end{proof}

\begin{theorem} Предположим, что  $ \frac{1}{\rho(x)}\in L(a,b) $, а  функции $\varphi_k(x)$ $(k=0,1,\ldots)$  образуют полную в $L^2_\rho(a,b)$ ортонормированную   c весом   $\rho(x)$ систему на $(a,b)$, $\{\varphi_{r,k}(x)\}_{k=0}^\infty$ -- система, ортонормированная в $W^r_{L^2_\rho(a,b)}$ относительно скалярного произведения \eqref{1.13},  порожденная системой $\{\varphi_{k}(x)\}_{k=0}^\infty$ посредством равенств \eqref{1.10} и \eqref{1.11}.
Тогда, если $f(x)\in W^r_{L^2_\rho(a,b)}$, то ряд Фурье (смешанный ряд) \eqref{1.14} сходится к функции $f(x)$ равномерно относительно $x\in[a,b]$.
 \end{theorem}
%\begin{proof}
%Обозначим через $S_n(f^{(r)})=S_n(f^{(r)},x)$ частичную сумму ряда Фурье функции $f^{(r)}(x)\in L^2_\rho(a,b) $ по системе $\{\varphi_k(x)\}_{k=0}^n$, т.е.
%\begin{equation}\label{2.5}
%S_n(f^{(r)},x)=\sum_{k=0}^n f_{r,k+r}\varphi_k(x),
%     \end{equation}
%где коэффициенты $f_{r,k+r}$ $(k=0,1,\ldots)$ определены равенством \eqref{1.15}.  Из условий теоремы 2 следует, что при $n\to\infty$
% \begin{equation}\label{2.6}
%\|f^{(r)}-S_n(f^{(r)})\|_{L^2_\rho(a,b)}\to0.
% \end{equation}
%Далее, обозначим через $ Y_{n+r}(f,x)$ частичную сумму смешанного ряда \eqref{1.14} следующего вида
% \begin{equation}\label{2.7}
% Y_{n+r}(f,x)= \sum_{k=0}^{r-1} f^{(k)}(a)\frac{(x-a)^k}{k!}+ \sum_{k=r}^{n+r} \hat f_{r,k}\varphi_{r,k}(x),
%\end{equation}
%и запишем формулу Тейлора
%\begin{equation}\label{2.8}
% f(x)= \sum_{k=0}^{r-1} f^{(k)}(a)\frac{(x-a)^k}{k!}+ {1\over (r-1)!}\int\limits_{a}^x(x-t)^{r-1} f^{(r)}(t)dt.
%\end{equation}
%Из \eqref{2.7} и \eqref{2.8} имеем
%\begin{equation}\label{2.9}
% f(x)-Y_{n+r}(f,x)=  {1\over (r-1)!}\int\limits_{a}^x(x-t)^{r-1} f^{(r)}(t)dt-\sum_{k=r}^{n+r} \hat f_{r,k}\varphi_{r,k}(x).
%\end{equation}
%Обратимся к равенству \eqref{1.10}, тогда \eqref{2.9} можно переписать так
%$$
%f(x)-Y_{n+r}(f,x)=
%$$
%$$
%  {1\over (r-1)!}\int\limits_{a}^x(x-t)^{r-1} f^{(r)}(t)dt-{1\over (r-1)!}\int\limits_{a}^x(x-t)^{r-1}\sum_{k=r}^{n+r} \hat f_{r,k}\varphi_{k-r}(t)dt=
%$$
%\begin{equation}\label{2.10}
% {1\over (r-1)!}\int\limits_{a}^x(x-t)^{r-1}[f^{(r)}(t)-S_n(f^{(r)},t)]dt .
%\end{equation}
%Из \eqref{2.10} и неравенства Гельдера имеем
%$$
%|f(x)-Y_{n+r}(f,x)|\le
%$$
%\begin{equation}\label{2.11}
% {1\over (r-1)!} \left(\int\limits_{a}^b\frac{|x-t|^{2(r-1)}}{\rho(t)}dt\right)^\frac12\left(\int\limits_{a}^b [f^{(r)}(t)-S_n(f^{(r)},t)]^2\rho(t)dt\right)^\frac12.
% \end{equation}
%Сопоставляя \eqref{2.6} с \eqref{2.11}, убеждаемся в справедливости теоремы 2.
%
%\end{proof}

\section{Некоторые сведения о полиномах Якоби}
При изучении свойств полиномов, ортогональных по Соболеву, порожденных полиномами Чебышева первого рода $T_n(x)=\cos(n\arccos x)$, нам понадобится ряд свойств классических полиномов Якоби.  Для произвольных действительных $\alpha$ и $\beta$ полиномы Якоби  $P_n^{\alpha,\beta}(x)$ можно определить \cite{Sege}  с помощью формулы Родрига
 \begin{equation}\label{3.1}
P_n^{\alpha,\beta}(x) = {(-1)^n\over2^nn!}{1\over\rho(x)}{d^n\over
dx^n} \left\{\rho(x)\sigma^n(x)\right\},
\end{equation}
где $\alpha,\beta$ -- произвольные действительные числа, $\rho(x)=
\rho(x;\alpha,\beta) =
(1-x)^\alpha(1+x)^\beta,\,\,\sigma(x)=1-x^2$. Если
$\alpha,\beta>-1$, то полиномы Якоби образуют ортогональную
систему с весом $\rho(x)$, т.е.
\begin{equation}\label{3.2}
\int_{-1}^1P_n^{\alpha,\beta}(x)P_m^{\alpha,\beta}(x)\rho(x)dx =
h_n^{\alpha,\beta}\delta_{nm},
\end{equation}
где
\begin{equation}\label{3.3}
h_n^{\alpha,\beta} =
{\Gamma(n+\alpha+1)\Gamma(n+\beta+1)2^{\alpha+\beta+1} \over
n!\Gamma(n+\alpha+\beta+1)(2n+\alpha+\beta+1)}.
\end{equation}
Нам понадобятся еще следующие свойства полиномов Якоби~\cite{Sege}, \cite{Gasper}:


\begin{equation}\label{3.4}
{d\over dx}P_n^{\alpha,\beta}(x) =
{1\over2}(n+\alpha+\beta+1)P_{n-1}^{\alpha+1,\beta+1}(x),
\end{equation}
\begin{equation}\label{3.5}
{d^\nu\over dx^\nu}P_n^{\alpha,\beta}(x) =
{(n+\alpha+\beta+1)_\nu\over2^\nu} P_{n-\nu}^{\alpha+\nu,\beta+\nu}(x),
\end{equation}
где $(a)_0=1$, $(a)_\nu=a(a+1)\dots(a+\nu-1)$, $a^{[0]}=1$,

 \begin{equation}\label{3.6}
 {n\choose l}P_n^{\alpha,-l}(x)= {n+\alpha\choose
l}\left({x+1\over2}\right)^lP_{n-l}^{\alpha,l}(x),
     \quad 1\le l \le n,
\end{equation}
\begin{equation}\label{3.7}
P_n^{\alpha,\beta}(t) ={n+\alpha\choose n}
\sum_{k=0}^n{(-n)_k(n+\alpha+\beta+1)_k\over k!(\alpha+1)_k}
\left({1-t\over 2}\right)^k,
\end{equation}
\begin{equation}\label{3.8}
(1-x)^\alpha(1+x)^\beta P_n^{\alpha,\beta}(x)
={(-1)^m\over2^mn^{[m]}}{d^m\over dx^m}
\left\{(1-x)^{m+\alpha}(1+x)^{m+\beta} P_{n-m}^{m+\alpha,m+\beta}(x)
\right\},
\end{equation}
где $k^{[0]}=1$, $k^{[r]}=k(k-1)\dots(k-r+1)$,
\begin{equation}\label{3.9}
P_n^{\alpha,\beta}(-1)=(-1)^n {n+\beta\choose n},\quad P_n^{\alpha,\beta}(1)= {n+\alpha\choose n},
\end{equation}

$$
{P_n^{a,a}(x)\over P_n^{a,a}(1)}=\sum_{j=0}^{[n/2]}
   { n!(\alpha+1)_{n-2j}(n+2a+1)_{n-2j}(1/2)_j(a-\alpha)_j
    \over (n-2j)!(2j)!(a+1)_{n-2j}(n-2j+2\alpha+1)_{n-2j}}
     $$
\begin{equation}\label{3.10}
    \times{1\over (n-2j+a+1)_j(n-2j+\alpha+3/2)_j}
     {P_{n-2j}^{\alpha,\alpha}(x)\over
P_{n-2j}^{\alpha,\alpha}(1)},
\end{equation}
где $[b]$ -- целая часть числа $b$.


\begin{lemma} Пусть $\alpha>-1$, $k,r$ -- целые, $r\ge1$,
     $k\ge r+1$. Тогда
     $$
P_{k+r}^{\alpha-r,\alpha-r}(x)=\sum_{j=0}^r\lambda_j^\alpha
P_{k+r-2j}^{\alpha,\alpha}(x),
     $$
где
     $$
 \lambda_j^\alpha=\lambda_j^\alpha(r,k)=
{(-1)^j(k-r+2\alpha+1)_{k+r-2j}(1/2)_jr^{[j]}
     (\alpha+k)^{[j]}
\over(k+r-2j+2\alpha+1)_{k+r-2j} (k+r-2j+\alpha+3/2)_j(2j)!}.
     $$
\end{lemma}
%\begin{proof}
%Предположим сначала, что
%$\alpha-r>-1$, тогда, полагая
%     $a=\alpha-r$, мы можем воспользоваться формулой \eqref{3.10}.
% Поскольку  при $j\ge r+1$ выполняется равенство $(a-\alpha)_j=(-r)_j=0$, то из \eqref{3.10} мы имеем
%     $$
%     P_{k+r}^{\alpha-r,\alpha-r}(x)=\sum_{j=0}^r\lambda_j^\alpha
%P_{k+r-2j}^{\alpha,\alpha}(x),
%     $$
%где
%     $$
%\lambda_j^\alpha={P_{k+r}^{\alpha-r,\alpha-r}(1)\over
%     P_{k+r-2j}^{\alpha,\alpha}(1)}
%{(k+r)!(\alpha+1)_{k+r-2j}(k-r+2\alpha+1)_{k+r-2j}\over
%     (k+r-2j)!(2j)!(\alpha-r+1)_{k+r-2j}}\times
%     $$
%     $$
%     {(1/2)_j(-r)_j\over(k+r-2j+2\alpha+1)_{k+r-2j}(k-2j+\alpha+1)_j
%     (k+r-2j+\alpha+3/2)_j}=
%     $$
%     $$
%{(-1)^j(k-r+2\alpha+1)_{k+r-2j}(1/2)_jr^{[j]}(\alpha+k)^{[j]}
%\over(k+r-2j+2\alpha+1)_{k+r-2j}
%(k+r-2j+\alpha+3/2)_j(2j)!}.
%     $$
%     Отсюда следует справедливость утверждения леммы 3.1 в случае
%     $\alpha>r-1$. Но поскольку  $P_{k+r}^{\alpha-r,\alpha-r}(x)$,
% $\lambda_j^\alpha$ и $P_{k+r-2j}^{\alpha,\alpha}(x)$ представляют
%собой аналитические функции относительно $\alpha$, то утверждение
%леммы 3.1 вытекает из уже доказанного случая.
%\end{proof}

\begin{lemma} Пусть  $k,r$ -- целые, $r\ge1$,
     $k\ge r+1$. Тогда
     $$
P_{k+r}^{-\frac12-r,-\frac12-r}(x)=\sum_{j=0}^r\lambda_j^{-\frac12}(k,r)
P_{k+r-2j}^{-\frac12,-\frac12}(x),
     $$
где
$$
 \lambda_j^{-\frac12}(k,r)=
{(-1)^j(k-r)_{k+r-2j}(1/2)_jr^{[j]}
     (k-1/2)^{[j]}
\over(k+r-2j)_{k+r-2j} (k+r-2j+1)_j(2j)!}=
$$
\begin{equation}\label{3.11}
(-1)^j{((k+r-2j)!)^22^{2k+2r-4j}\over(2(k+r-2j))!}
{(2k)!\over(k!)^22^{2k+2r}}{r^{[j]}\over j!}
{k^{[r+1]}\over(k+r-j)^{[r+1]}}.
\end{equation}
\end{lemma}
%\begin{proof}
%Чтобы убедиться в справедливости утверждения леммы 3.2 достаточно в лемме 3.1 взять $\alpha=-\frac12$.
%\end{proof}

Пусть $T_n(x)=\cos(n\arccos x)$ -- полином Чебышева первого рода. Тогда \cite{Sege}
\begin{equation}\label{3.12}
P_n^{-\frac{1}{2},-\frac{1}{2}}(x)=\frac{1\cdot3\cdot\ldots\cdot(2n-1)}
{2\cdot4\cdot\ldots\cdot2n}T_n(x)=\frac{(2n)!}{2^{2n}{n!}^2}T_n(x);
\end{equation}
Имеет место следующая

\begin{lemma} Пусть  $k,r$ -- целые, $r\ge1$,
     $k\ge r+1$. Тогда
 $$
P_{k+r}^{-\frac12-r,-\frac12-r}(x)={(2k)!\over(k!)^22^{2k+2r}}\sum_{j=0}^r{(-1)^j\over j!}
{r^{[j]}k^{[r+1]}\over(k+r-j)^{[r+1]}}T_{k+r-2j}(x).
     $$
\end{lemma}
%\begin{proof}
%Утверждение леммы 3.3 непосредственно вытекает из леммы 3.2 и равенств \eqref{3.12} и \eqref{3.11}.
%\end{proof}

Пусть $\alpha$ и $\beta$ -- произвольные вещественные числа,
$$
s(\theta)=\pi^{-\frac12}\left(\sin\frac{\theta}{2}\right)^{-\alpha-\frac12}
\left(\cos\frac{\theta}{2}\right)^{-\beta-\frac12},
$$
$$
\lambda_n=n+\frac{\alpha+\beta+1}{2}, \quad\gamma=-
\left(\alpha+\frac{1}{2}\right)\frac{\pi}{2}.
$$
Тогда для $0<\theta<\pi$ имеет место асимптотическая формула
\begin{equation}\label{3.13}
P_n^{\alpha,\beta}(\cos\theta)=n^{-\frac12}s(\theta)
\left(\cos(\lambda_n\theta+\gamma)+
\frac{v_n(\theta)}{n\sin\theta}\right),
\end{equation}
в которой для функции $v_n(\theta)=v_n(\theta;\alpha,\beta)$
справедлива оценка
\begin{equation}\label{3.14}
|v_n(\theta)|\le c(\alpha,\beta,\delta)\quad
 \left(0<\frac{\delta}{n}\le\theta\le\pi-\frac{\delta}{n}\right).
\end{equation}





 \section{Ортогональные по Соболеву полиномы, порожденные полиномами Якоби}
Из равенства \eqref{3.2} следует, что если $\alpha,\beta>-1$, то полиномы
\begin{equation}\label{4.1}
p_n^{\alpha,\beta}(x)={P_n^{\alpha,\beta}(x)\over\sqrt{ h_n^{\alpha,\beta}}}\quad(n=0,1,\ldots)
\end{equation}
образуют ортонормированную  в $L_\kappa^2(-1,1)$ с весом $\rho(x)=(1-x)^\alpha(1+x)^\beta$ систему. Как хорошо известно \cite{Sege}, система полиномов Якоби \eqref{4.1}полна в $L_\kappa^2(-1,1)$.   Эта система порождает на $[-1,1]$ систему полиномов  $p_{r,k}^{\alpha,\beta}(x)=\varphi_{r,k}(x)$ $(k=0,1,\ldots)$, определенных равенствами \eqref{1.10} и \eqref{1.11} с $a=-1$, $\varphi_k(x)=p_{k}^{\alpha,\beta}(x)$. Из теоремы 1 непосредственно вытекает
\begin{corollary}
Пусть $a=-1$, $\alpha,\beta>-1$. Тогда система полиномов $\{p_{r,k}^{\alpha,\beta}(x)\}_{k=0}^\infty$, порожденная системой ортонормированных полиномов Якоби \eqref{4.1} посредством равенств \eqref{1.10} и \eqref{1.11}, полна  в $W^r_{L^2_\rho(-1,1)}$ и ортонормирована относительно скалярного произведения \eqref{1.13}.
\end{corollary}

Ряд Фурье \eqref{1.14} для системы   $\{p_{r,k}^{\alpha,\beta}(x)\}_{k=0}^\infty$ приобретает вид
\begin{equation}\label{4.2}
f(x)\sim \sum_{k=0}^{r-1} f^{(k)}(-1)\frac{(x+1)^k}{k!}+ \sum_{k=r}^\infty \hat f_{r,k}p_{r,k}^{\alpha,\beta}(x),
\end{equation}
где
  \begin{equation}\label{4.3}
 \hat f_{r,k}=\int_{-1}^1 f^{(r)}(t)p_{k-r}^{\alpha,\beta}(t)\rho(t)dt.
\end{equation}

\begin{corollary}
Пусть $-1<\alpha,\beta<1$. Тогда, если $f(x)\in W^r_{L^2_\rho(-1,1)}$, то ряд Фурье (смешанный ряд) \eqref{4.2} сходится к функции $f(x)$ равномерно относительно $x\in[-1,1]$.
\end{corollary}
%\begin{proof}
%Заметим, что если $-1<\alpha,\beta<1$, то $\frac{1}{\rho(x)}\in L(-1,1)$, где $\rho(x)=(1-x)^\alpha(1+x)^\beta$. Поэтому утверждение следствия 2 вытекает из теоремы 2 и следствия 1.
%\end{proof}

В следующей теореме установлен явный вид полиномов $p_{r,k}^{\alpha,\beta}(x)$ при $k\ge r$, который играет  важную роль при исследовании асимптотических свойств полиномов $p_{r,n+r}^{\alpha,\beta}(x)$ в окрестности точки $x=-1$.



\begin{theorem} Для произвольных $\alpha, \beta>-1$ и $n\ge0$
имеет место следующее равенство
\begin{equation}\label{4.4}
p_{r,n+r}^{\alpha,\beta}(x)=\frac{(-1)^n2^{r}}{\sqrt{ h_n^{\alpha,\beta}}}
{n+\beta\choose n}
\sum_{k=0}^n{(-n)_k(n+\alpha+\beta+1)_k\over (\beta+1)_k(k+r)!}
\left({1+x\over 2}\right)^{k+r}.
\end{equation}
\end{theorem}
%\begin{proof}
%Воспользуемся равенством \eqref{3.7} и запишем
%$$
%P_{n}^{\alpha,\beta}(t)=(-1)^nP_{n}^{\beta,\alpha}(-t)=(-1)^n{n+\beta\choose n}
%\sum_{k=0}^n{(-n)_k(n+\alpha+\beta+1)_k\over k!(\beta+1)_k}
%\left({1+t\over 2}\right)^k,
%$$
%поэтому, в силу \eqref{4.1} имеем
%\begin{equation}\label{4.5}
%p_{n}^{\alpha,\beta}(t)=\frac{(-1)^n}{\sqrt{ h_n^{\alpha,\beta}}}{n+\beta\choose n}
%\sum_{k=0}^n{(-n)_k(n+\alpha+\beta+1)_k\over k!(\beta+1)_k}
%\left({1+t\over 2}\right)^k.
%\end{equation}
%С другой стороны в силу формулы Тейлора
%\begin{equation}\label{4.6}
%\left({1+x\over 2}\right)^{k+r}={(k+r)^{[r]}\over 2^r(r-1)!}\int\limits_{-1}^x(x-t)^{r-1} \left({1+t\over 2}\right)^{k}dt.
%\end{equation}
%Сопоставляя \eqref{4.5} и \eqref{4.6} с определением
%\begin{equation}\label{4.7}
%  p_{r,n+r}^{\alpha,\beta}(x) =\frac{1}{(r-1)!}\int\limits_{-1}^x(x-t)^{r-1}p_{n}^{\alpha,\beta}(t)dt, \quad n=1,\ldots,
%\end{equation}
% убеждаемся в справедливости утверждения теоремы 3.
%\end{proof}

 \section{Ортогональные по Соболеву полиномы, порожденные полиномами Чебышева первого рода}
 В настоящем разделе мы отдельно и, большей частью, независимо от предыдущего раздела, исследуем полиномы $T_{r,k}(x)\,(k=0,1,\ldots)$, ортогональные по Соболеву, порожденные полиномами Чебышева первого рода
\begin{equation}\label{5.1}
T_0(x)=T_0=\frac{1}{\sqrt{2}},\quad T_k(x)=\cos(k\arccos x), \quad k=1,2,\ldots,
\end{equation}
образующми  ортонормированную  в $L_\mu^2(-1,1)$ с весом  $\mu(x)=\frac2\pi(1-x^2)^{-\frac12}$ систему. Как хорошо известно \cite{Sege}, система полиномов Чебышева \eqref{5.1} полна в $L_\mu^2(-1,1)$.   Она порождает на $[-1,1]$ систему полиномов $T_{r,k}(x)$ $(k=0,1,\ldots)$, определенных равенствами \eqref{1.4} и\eqref{1.5}.
Из теоремы 1 непосредственно вытекает
\begin{corollary}
  Система полиномов $\{T_{r,k}(x)\}_{k=0}^\infty$, порожденная системой ортонормированных полиномов Чебышева \eqref{5.1} посредством равенств \eqref{1.4} и \eqref{1.5}, полна  в $W^r_{L^2_\mu(-1,1)}$ и ортонормирована относительно скалярного произведения \eqref{1.3}.
\end{corollary}

Ряд Фурье по системе $\{T_{r,k}(x)\}_{k=0}^\infty$ для функции системы   $f\in W^r_{L^2_\mu(-1,1)}$ приобретает вид
\begin{equation}\label{5.2}
f(x)\sim \sum_{k=0}^{r-1} f^{(k)}(-1)\frac{(x+1)^k}{k!}+ \sum_{k=r}^\infty \hat f_{r,k}T_{r,k}(x),
\end{equation}
где
  \begin{equation}\label{5.3}
 \hat f_{r,r+j}=\int_{-1}^1 f^{(r)}(t)T_{j}(t)\mu(t)dt\quad(j\ge0).
\end{equation}

\begin{corollary}
 Если $f(x)\in W^r_{L^2_\mu(-1,1)}$, то ряд Фурье (смешанный ряд) \eqref{5.2} сходится к функции $f(x)$ равномерно относительно $x\in[-1,1]$.
\end{corollary}
%\begin{proof}
% Так как $\frac{1}{\mu(x)}\in L(-1,1)$, то утверждение следствия 4 вытекает из теоремы 2 и следствия 3.
%\end{proof}

 Рассмотрим дальнейшие свойства полиномов $T_{r,k}(x)$.
Пусть $\alpha=\beta=-\frac{1}{2}$, тогда $\lambda=\alpha+\beta=-1$ и если $(k-1)^{[r]}\neq0$, то мы можем воспользоваться равенством \eqref{3.5} и записать
\begin{equation}\label{5.4}
P^{-\frac{1}{2},-\frac{1}{2}}_k(t)={2^r \over (k-1 )^{[r]}}\frac{d^r}{dt^r}P_{k+r}^{-\frac{1}{2}-r,-\frac{1}{2}-r}(t).
\end{equation}
 Если  $k\ge r+1$, то, очевидно, $(k-1)^{[r]}\neq0$ и для таких $k$ мы
можем  воспользоваться равенством \eqref{5.4}. Итак, пусть $(k-1)^{[r]}\neq0$. Тогда в силу  \eqref{5.4}
$$
\frac{1}{(r-1)!}\int\limits^x_{-1}(x-t)^{r-1}P_k^{-\frac{1}{2},-\frac{1}{2}}(t)\,dt=
$$
$$
\frac{2^r}{(k-1)^{[r]}}\frac{1}{(r-1)!}\int\limits^x_{-1}(x-t)^{r-1}
\frac{d^r}{dt^r}P_{k+r}^{-\frac{1}{2}-r,-\frac{1}{2}-r}(t)\,dt=
$$
\begin{equation}\label{5.5}
\frac{2^r}{(k-1)^{[r]}}\left[P_{k+r}^{-\frac{1}{2}-r,-\frac{1}{2}-r}(x)-\sum^{r-1}_{\nu=0}
\frac{(1+x)^\nu}{\nu!}\left\{P_{k+r}^{-\frac{1}{2}-r,-\frac{1}{2}-r}(t)
\right\}_{t=-1}^{(\nu)}\right].
\end{equation}
 Далее, в силу \eqref{3.5}
 \begin{equation}\label{5.6}
\left\{P_{k+r}^{-\frac{1}{2}-r,-\frac{1}{2}-r}(t)\right\}^{(\nu)}=
\frac{(k-r)_\nu}{2^\nu}P_{k+r-\nu}^{-\frac{1}{2}+\nu-r,-\frac{1}{2}+\nu-r}(t),
\end{equation}
а из \eqref{3.9} имеем
\begin{equation}\label{5.7}
P_{k+r-\nu}^{-\frac{1}{2}+\nu-r,-\frac{1}{2}+\nu-r}(-1)=\frac{(-1)^{k+r-\nu}\Gamma(k+\frac{1}{2})}{\Gamma(\nu-r+\frac{1}{2})(k+r-\nu)!}.
\end{equation}
Из \eqref{5.6}  и \eqref{5.7} находим
\begin{equation}\label{5.8}
\left\{P_{k+r}^{-\frac{1}{2}-r,-\frac{1}{2}-r}(t)\right\}_{t=-1}^{(\nu)}=
\frac{(-1)^{k+r-\nu}\Gamma(k+\frac{1}{2})(k-r)_{\nu}}
{\Gamma(\nu-r+\frac{1}{2})(k+r-\nu)!2^\nu}=A_{\nu,k,r}.
\end{equation}
Сопоставляя \eqref{5.5} и \eqref{5.8}, мы можем записать
$$\frac{1}{(r-1)!}\int\limits^x_{-1}(x-t)^{r-1}P_k^{-\frac{1}{2},-\frac{1}{2}}(t)\,dt=$$
\begin{equation}\label{5.9}
\frac{2^r}{(k-1)^{[r]}}\left[P_{k+r}^{-\frac{1}{2}-r,-\frac{1}{2}-r}(x)-\sum^{r-1}_{\nu=0}
\frac{A_{\nu,k,r}}{\nu!}(1+x)^{\nu}\right].
\end{equation}



Из \eqref{1.5}, \eqref{3.12} и  \eqref{5.5}    имеем
$$
T_{r,r+k}(x)=\frac{1}{(r-1)!}\int\limits^x_{-1}(x-t)^{r-1}T_k(t)\,dt=
$$
$$
\frac{2^{2k}k!^2}{(2k)!(r-1)!}\int\limits^x_{-1}(x-t)^{r-1}P_k^{-\frac12,-\frac12}(t)\,dt=
$$
\begin{equation}\label{5.10}
\frac{k!^2}{(2k)!}
\frac{2^{r+2k}}{(k-1)^{[r]}}\left[P_{k+r}^{-\frac12-r,-\frac12-r}(x)-\sum^{r-1}_{\nu=0}
\frac{A_{\nu,k,r}}{\nu!}(1+x)^{\nu}\right],
\end{equation}
где в силу \eqref{5.8} и равенств
$$
\Gamma(z)\Gamma(1-z)=\frac{\pi}{\sin(\pi z)},\quad \Gamma(z+1/2)=\frac{\sqrt{\pi}\Gamma(2z)}{\Gamma(z)2^{2z-1}}
$$
 для $k\ge r+1$ находим
$$
A_{\nu,k,r}=
\frac{(-1)^{k+r-\nu}\Gamma(k+1/2)(k-r)_{\nu}}{\Gamma(\nu-r+1/2)(k+r-\nu)!2^\nu}
$$
$$
=\frac{(-1)^{k}\Gamma(k+1/2)(k-r)_{\nu}\Gamma(r-\nu+1/2)}{\pi (k+r-\nu)!2^\nu}=
$$
\begin{equation}\label{5.11}
\frac{(-1)^{k}(2k-1)!(2(r-\nu)-1)!(k-r)_{\nu}}{(k-1)!(r-\nu-1)! (k+r-\nu)!2^{2(k+r-1)-\nu}}.
\end{equation}
Таким образом, при $k\ge r+1$ мы получаем следующее представление
\begin{equation}\label{5.12}
T_{r,r+k}(x)=\frac{k!^2}{(2k)!}
\frac{2^{r+2k}}{(k-1)^{[r]}}\left[P_{k+r}^{-\frac12-r,-\frac12-r}(x)-
\sum^{r-1}_{\nu=0}\frac{A_{\nu,k,r}}{\nu!}(1+x)^{\nu}\right].
\end{equation}
Теперь обратимся к лемме 3.3, из которой выводим
\begin{equation}\label{5.13}
\frac{k!^2}{(2k)!}
\frac{2^{r+2k}}{(k-1)^{[r]}}P_{k+r}^{-\frac12-r,-\frac12-r}(x)=
\sum_{j=0}^r(-1)^j{r\choose j}
{kT_{k+r-2j}(x)\over2^r(k+r-j)^{[r+1]}}.
\end{equation}
Сопоставляя \eqref{5.12} и  \eqref{5.13}, мы приходим к следующему результату.
\begin{theorem}
Если  $k\ge r+1$, то
\begin{equation}\label{5.14}
T_{r,r+k}(x)=\sum_{j=0}^r{r\choose j}
{(-1)^jkT_{k+r-2j}(x)\over2^r(k+r-j)^{[r+1]}}
-\frac{k!^22^{r+2k}}{(2k)!(k-1)^{[r]}}
\sum^{r-1}_{\nu=0}\frac{A_{\nu,k,r}}{\nu!}(1+x)^{\nu}.
\end{equation}
\end{theorem}

Рассмотрим два важных частных случая, соответствующие  значениям   $r=1$  и $r=2$.

\noindent\textbf{1)} Пусть $r=1$. Тогда из \eqref{5.10} имеем
\begin{equation}\label{5.15}
A_{0,k,1}=\frac{(-1)^{k}(2k-1)!}{(k-1)!(k+1)!2^{2k}}, \quad k=2,3,\ldots.
\end{equation}

Из \eqref{5.13} и \eqref{5.15}  для $k\ge2$ находим
$$
T_{1,k+1}(x)=\sum_{j=0}^1(-1)^j
{kT_{k+1-2j}(x)\over2(k+1-j)^{[2]}}-\frac{k!^2}{(2k)!}
\frac{2^{2k+1}}{(k-1)}\frac{(-1)^{k}(2k-1)!}{(k-1)!(k+1)!2^{2k}}
$$
$$
=\sum_{j=0}^1(-1)^j
{kT_{k+1-2j}(x)\over2(k+1-j)^{[2]}}-\frac{(-1)^k}{k^2-1}.
$$
Отсюда и из \eqref{5.2} и \eqref{5.3} мы водим
\begin{corollary} Имеют место равенства
\begin{equation}\label{5.16}
T_{1,k+1}(x)={T_{k+1}(x)\over2(k+1)}- {T_{k-1}(x)\over2(k-1)} -\frac{(-1)^k}{k^2-1}\quad (k\ge 2),
\end{equation}
\begin{equation}\label{5.17}
T_{1,0}(x)=1, \quad T_{1,1}(x)=\frac{1+x}{\sqrt{2}}, \quad T_{1,2}(x)=\frac12(x^2-1).
\end{equation}
\end{corollary}


\noindent\textbf{2)} Для $r=2$ и $k\ge3$ из \eqref{5.14} и \eqref{5.17} имеем
$$
A_{0,k,2}=\frac{6(-1)^{k}(2k-1)!}{(k-1)! (k+2)!2^{2(k+1)}},\quad A_{1,k,2}=\frac{(-1)^{k}(2k-1)!(k-2)}{(k-1)! (k+1)!2^{2k+1}},
$$
$$
T_{2,k+2}(x)=\sum_{j=0}^2{r\choose j}
{(-1)^jkT_{k+2-2j}(x)\over2^2(k+2-j)^{[3]}}
-\frac{k!^2}{(2k)!}
\frac{2^{2+2k}}{(k-1)^{[2]}}\sum^1_{\nu=0}\frac{A_{\nu,k,2}}{\nu!}(1+x)^{\nu},
$$
поэтому при $k\ge3$
$$
T_{2,k+2}(x)=\sum_{j=0}^2{2\choose j}
{(-1)^jkT_{k+2-2j}(x)\over4(k+2-j)^{[3]}}
-(-1)^k\left[\frac{1+x}{k^2-1}+\frac{3}{(k^2-1)(k^2-4)}\right].
$$
Отсюда и из \eqref{5.2} и \eqref{5.3} мы выводим
\begin{corollary} Имеют место равенства
$$
T_{2,k+2}(x)={T_{k+2}(x)\over4(k+2)(k+1)}-{T_{k}(x)\over2(k^2-1)}+
{T_{k-2}(x)\over4(k-1)(k-2)}-
$$
\begin{equation}\label{5.18}
(-1)^k\left[\frac{1+x}{k^2-1}+\frac{3}{(k^2-1)(k^2-4)}\right]\quad(k\ge3),
\end{equation}
\begin{equation}\label{5.19}
T_{2,0}(x)=1, \quad T_{2,1}(x)=1+x, \quad T_{2,2}(x)=\frac{(1+x)^2}{2\sqrt{2}},
\end{equation}
\begin{equation}\label{5.20}
T_{2,3}(x)=\frac16(x-2)(x+1)^2, \quad T_{2,4}(x)=\frac16x(x-2)(x+1)^2.
\end{equation}
\end{corollary}

   В заключение этого параграфа мы выведем явный вид для полиномов  $T_{r,k}(x)$, представляющий из себя частный случай равенства \eqref{4.4} (теорема 4). С этой целью установим их связь с полиномами $p_{r,k}^{-\frac12,-\frac12}(x)$, определенными равенством \eqref{1.10} с $a=-1$, $\varphi_k(x)=p_{k}^{-\frac12,-\frac12}(x)$. В силу \eqref{3.12},  \eqref{3.3} и \eqref{4.1} имеем
 $$
 T_{r,r+n}(x) =  \frac{2^{2n}{n!}^2}{(2n)!}\frac{1}{(r-1)!}\int\limits_{-1}^x(x-t)^{r-1}
   P_n^{-\frac{1}{2},-\frac{1}{2}}(t)dt=
$$
\begin{equation}\label{5.21}
\frac{2^{2n}{n!}^2}{(2n)!}\frac{\sqrt{h_n^{-\frac{1}{2},-\frac{1}{2}}}}{(r-1)!}
\int\limits_{-1}^x(x-t)^{r-1}
   p_n^{-\frac{1}{2},-\frac{1}{2}}(t)dt= \frac{2^{2n}{n!}^2}{(2n)!}\sqrt{h_n^{-\frac{1}{2},-\frac{1}{2}}}
   p_{r,r+n}^{-\frac12,-\frac12}(x).
\end{equation}
Обратимся теперь к теореме 4, из которой следует равенство
\begin{equation}\label{5.22}
p_{r,n+r}^{-\frac{1}{2},-\frac{1}{2}}(x)=\frac{(-1)^n2^{r}}{\sqrt{ h_n^{-\frac{1}{2},-\frac{1}{2}}}}
{n-\frac{1}{2}\choose n}
\sum_{k=0}^n{(-n)_k(n)_k\over (\frac12)_k(k+r)!}
\left({1+x\over 2}\right)^{k+r}.
\end{equation}
 Сопоставля \eqref{5.21} и \eqref{5.22} и учитывая, что ${n-\frac{1}{2}\choose n}=\frac{(2n)!}{n!^22^{2n}}$, мы приходим к следующему результату.
\begin{corollary} При $n\ge1$ имеют место равенство
\begin{equation}\label{5.23}
T_{r,n+r}(x)=(-1)^n2^r\sum\nolimits_{k=0}^n{(-n)_k(n)_k\over (\frac12)_k(k+r)!}
\left({1+x\over 2}\right)^{k+r}.
\end{equation}
\end{corollary}

\section{О предсталении решения задачи Коши для ОДУ рядом Фурье по полиномам $T_{r,k}(x)$}
Мы ограничимся рассмотрением задачи Коши вида
\begin{equation}\label{6.1}
y'(x)=f(x,y), \quad y(a)=y_0,
\end{equation}
 в которой функцию   $f(x,y)$  будем считать достаточно гладкой в некоторой области $G$ переменных $(x,y)$, для которой точка $(a,y_0)$ является внутренней. Требуется аппроксимировать с заданной точностью  функцию $y=y(x)$, определенную на $[a,b]$, которая является решением задачи Коши \eqref{6.1}. Для этого разобьем отрезок $[a,b]$ на $m$ частей с шагом $2h$ и пусть $[a_i,a_{i+1}]$ -- $i$-тый отрезок деления, в частности, $[a_0,a_1]=[a,a+2h]$.
 Будем приближать искомое решение на отрезках $[a_i,a_{i+1}]$, $0\le i\le m-1$, остановившись для определенности на $i=0$.  Полагая $x=a+h(t+1)$, отобразим линейно отрезок $[-1,1]$ на $[a,a+2h]$. Относительно новой переменной $t\in [-1,1]$ уравнение \eqref{6.1} принимает следующий вид
\begin{equation}\label{6.2}
\varphi'(t)=hf(a+h(t+1),\varphi(t)), \quad \varphi(-1)=y_0, -1\le t\le 1,
\end{equation}
где $\varphi(t)=y(a+h(t+1))$. Представим функцию $\varphi(t)$ в виде ряда Фурье \eqref{5.2} с $r=1$:
\begin{equation}\label{6.3}
\varphi(t)= \varphi(-1)+ \sum\nolimits_{k=1}^\infty \hat \varphi_{1,k}T_{1,k}(t),
\end{equation}
где
  \begin{equation}\label{6.4}
\hat \varphi_{1,1+j}=\int_{-1}^1 \varphi'(t)T_{j}(t)\mu(t)dt\quad(j\ge0).
\end{equation}
Применим  дифференциальное свойство \eqref{1.16} к смешанному ряду \eqref{6.3}. Это дает
\begin{equation}\label{6.5}
\varphi'(t)=  \sum\nolimits_{k=0}^\infty \hat \varphi_{1,k+1}T_{k}(t).
\end{equation}
Положим $\phi(t)=f(a+h(t+1),\varphi(t))$ и заметим, что в силу \eqref{6.2} и \eqref{6.4} коэффициенты Фурье-Чебышева функции $\phi=\phi(t)$ имеют вид
\begin{equation}\label{6.6}
 c_k(\phi)=\int\limits_{-1}^1\phi(t)T_k(t)\mu(t)dt=\frac1h\hat \varphi_{1,k+1} \quad (k\ge0).
\end{equation}
С учетом этих равенств мы можем переписать \eqref{6.3} в следующем виде
\begin{equation}\label{6.7}
\varphi(t)= \varphi(-1)+ h\sum\nolimits_{k=0}^\infty c_k(\phi)T_{1,k+1}(t).
\end{equation}
Из  \eqref{6.6}, \eqref{6.7}, в свою очедь, выводим следующие соотношения
\begin{equation}\label{6.8}
c_k(\phi)=\int\limits_{-1}^1f\left[a+h(t+1),\varphi(-1)+ h\sum\nolimits_{j=0}^\infty c_j(\phi)T_{1,j+1}(t)\right]T_k(t) \mu(t)dt,
\end{equation}
$$
k=0,1,2,\ldots.
$$
Введем в рассмотрение гильбертово пространство $l_2$, состоящее из последовательностей $C=(c_0,c_1,\ldots)$, для которых введена норма
$\|C\|=\left(\sum_{j=0}^\infty c_j^2\right)^\frac12$.  В пространстве $l_2$ рассмотрим оператор $A$, сопоставляющий точке $C\in l_2$ точку $C'\in l_2$ по правилу
\begin{equation}\label{6.9}
c_k'=\int\limits_{-1}^1f\left[a+h(t+1),\varphi(-1)+ h\sum\nolimits_{j=0}^\infty c_jT_{1,j+1}(t)\right]T_k(t) \mu(t)dt,
\end{equation}
$$
k=0,1,2,\ldots.
$$
Из  \eqref{6.8} следует, что точка $C(\phi)=(c_0(\phi),c_1(\phi),\ldots)$ является неподвижной точкой оператора $A:l_2\to l_2$. Для того, чтобы найти точку $C(\phi)$ методом простых итераций достаточно показать, что оператор $A:l_2\to l_2$ является сжимающим в метрике пространства $l_2$. С этой целью рассмотрим две точки $P,Q\in l_2$, где $P=(p_0,p_1,\ldots)$, $Q=(q_0,q_1,\ldots)$ и положим $P'=A(P)$, $Q'=A(Q)$. Имеем
\begin{equation}\label{6.10}
p'_k-q'_k=\int_{-1}^1F_{P,Q}(t)T_k(t) \mu(t)dt,\quad k=0,1,\ldots
\end{equation}
где
\begin{multline}\label{6.11}
 F_{P,Q}(t)=f\left[a+h(t+1),\varphi(-1)+ h\sum\nolimits_{j=0}^\infty p_jT_{1,j+1}(t)\right] \\
  -f\left[a+h(t+1),\varphi(-1)+ h\sum\nolimits_{j=0}^\infty q_jT_{1,j+1}(t)\right].
\end{multline}
Из \eqref{6.10}, пользуясь неравенством Бесселя, находим
 \begin{equation}\label{6.12}
\sum\nolimits_{k=0}^\infty (p'_k-q'_k)^2\le\int_{-1}^1(F_{P,Q}(t))^2 \mu(t)dt.
\end{equation}
 Предположим, что по переменной $y$ функция $f(x,y)$ удовлетворяет условию Липшица
 \begin{equation}\label{6.13}
|f(x,y')-f(x,y'')|\le \lambda|y'-y''|, \quad a\le x \le a+2h.
\end{equation}
Из \eqref{6.11} и \eqref{6.13}  имеем
 \begin{equation}\label{6.14}
(F_{P,Q}(t))^2\le (h\lambda)^2   \left(\sum\nolimits_{j=0}^\infty( p_j-q_j)T_{1,j+1}(t)\right)^2,
\end{equation}
откуда,  воспользовавшись неравенством Коши-Буняковского, выводим
$$
(F_{P,Q}(t))^2\le(h\lambda)^2   \sum\nolimits_{j=0}^\infty( p_j-q_j)^2\sum\nolimits_{j=0}^\infty(T_{1,j+1}(t))^2
$$
Сопоставляя \eqref{6.14} с \eqref{6.12}, находим
\begin{equation}\label{6.15}
\sum\nolimits_{k=0}^\infty (p'_k-q'_k)^2\le(h\lambda)^2 \sum\nolimits_{k=0}^\infty( p_k-q_k)^2\int_{-1}^1 \sum\nolimits_{j=0}^\infty(T_{1,j+1}(t))^2   \mu(t)dt.
\end{equation}
Нам остается оценить последний интеграл. С этой целью обратимся к следствию 5, из которого замечаем, что ряд, фигурирующий под интегралом в \eqref{6.15}, равномерно сходится при $t\in[-1,1]$. Положим
\begin{equation}\label{6.16}
\kappa=\left(\int_{-1}^1\sum\nolimits_{j=0}^\infty(T_{1,j+1}(t))^2\mu(t)dt\right)^\frac12.
\end{equation}
Тогда из \eqref{6.15} и \eqref{6.16} окончательно находим
\begin{equation}\label{6.17}
\left(\sum\nolimits_{k=0}^\infty (p'_k-q'_k)^2\right)^\frac12\le h\kappa\lambda \left(\sum\nolimits_{k=0}^\infty (p_k-q_k)^2\right)^\frac12. \end{equation}
Неравенство \eqref{6.17} показывает, что если $h\kappa\lambda<1$, то оператор  $A:l_2\to l_2$ является сжимающим и, как следствие, итерационный процесс $C^{\nu+1}=A(C^{\nu})$  сходится к точке $C(\phi)$ при $\nu\to\infty$. Однако, с точки зрения приложений важно рассмотреть конечномерный аналог оператора $A$. Мы рассмотрим оператор $A_N:\mathbb{R}^N\to \mathbb{R}^N$, cопоставляющий точке
$C_N=(c_0,\ldots,c_{N-1})\in \mathbb{R}^N $ точку  $C'_N=(c_0',\ldots,c_{N-1}')\in \mathbb{R}^N $ по правилу
\begin{equation}\label{6.18}
c_k'=\int\limits_{-1}^1f\left[a+h(t+1),\varphi(-1)+ h\sum\nolimits_{j=0}^{N-1} c_jT_{1,j+1}(t)\right]T_k(t) \mu(t)dt,
\end{equation}
$$
k=0,1,2,\ldots, N-1.
$$
 Рассмотрим две точки $P_N,Q_N\in \mathbb{R}^N$, где $P_N=(p_0,p_1,\ldots,p_{N-1})$,\\   $Q_N=(q_0,q_1,\ldots,p_{N-1})$ и положим $P'_N=A_N(P_N)$, $Q'_N=A_N(Q_N)$. Дословно повторяя рассуждения, которые привели нас к неравенству \eqref{6.17}, мы получим
\begin{equation}\label{6.19}
\left(\sum\nolimits_{k=0}^{N-1} (p'_k-q'_k)^2\right)^\frac12\le h\kappa\lambda \left(\sum\nolimits_{k=0}^{N-1} (p_k-q_k)^2\right)^\frac12.
\end{equation}
Неравенство \eqref{6.19} показывает, что если $h\kappa\lambda<1$, то оператор  $A_N:\mathbb{R}^N\to \mathbb{R}^N$ является сжимающим и, как следствие, итерационный процесс $C_N^{\nu+1}=A_N(C_N^{\nu})$  при $\nu\to\infty$ сходится к его неподвижной точке, которую мы обозначим через  $\bar C_N(\phi)=(\bar c_0(\phi),\bar c_1(\phi),\ldots,\bar c_{N-1}(\phi))$ . С другой стороны, рассмотрим точку $C_N(\phi)=(c_0(\phi),c_1(\phi),\ldots,c_{N-1}(\phi))$, составленную из искомых коэффициентов Фурье-Чебышева функции $\phi$. Нам остается оценить погрешность, проистекающую в результате замены точки $C_N(\phi)$ точкой $\bar C_N(\phi)$. Другими словами, требуется оценить величину
$\|C_N(\phi)-\bar C_N(\phi)\|_N= \left(\sum_{j=0}^{N-1}(c_j(\phi)-\bar c_j(\phi))^2\right)^\frac12$. С этой целью рассмотрим точку $C'_N(\phi)=A_N(C_N(\phi))=(c_0'(\phi),c_1'(\phi),\ldots,c_{N-1}'(\phi))$ и запишем
\begin{equation}\label{6.20}
\|C_N(\phi)-\bar C_N(\phi)\|_N\le \|C_N(\phi)- C_N'(\phi)\|_N+\|C_N'(\phi)-\bar C_N(\phi)\|_N.
\end{equation}
Далее, пользуясь неравенством \eqref{6.19}, имеем
$$
\|C_N'(\phi)-\bar C_N(\phi)\|_N=\|A_N(C_N(\phi))-A_N(\bar C_N(\phi))\|\le
$$
\begin{equation}\label{6.21}
h\kappa\lambda\|C_N(\phi)-\bar C_N(\phi)\|_N.
\end{equation}
Из \eqref{6.20} и \eqref{6.21} выводим
\begin{equation}\label{6.22}
\|C_N(\phi)-\bar C_N(\phi)\|_N\le \frac1{1-h\kappa\lambda}\|C_N(\phi)- C_N'(\phi)\|_N.
\end{equation}
Чтобы оценить норму в правой чвасти неравенства \eqref{6.22} заметим, что в силу неравенства Бесселя
\begin{equation}\label{6.23}
\|C_N(\phi)- C_N'(\phi)\|_N^2\le \int_{-1}^1(F_{C(\phi),C_N(\phi)}(t))^2 \mu(t)dt,
\end{equation}
где
\begin{multline}\label{6.24}
 F_{C(\phi),C_N(\phi)}(t)=f\left[a+h(t+1),\varphi(-1)+ h\sum\nolimits_{j=0}^\infty c_j(\phi)T_{1,j+1}(t)\right] \\
  -f\left[a+h(t+1),\varphi(-1)+ h\sum\nolimits_{j=0}^{N-1}c_j(\phi)T_{1,j+1}(t)\right].
\end{multline}
Из \eqref{6.24} и \eqref{6.14} следует, что
$$
(F_{C(\phi),C_N(\phi)}(t))^2\le \lambda^2   \left(\sum\nolimits_{j=N}^\infty hc_j(\phi)T_{1,j+1}(t)\right)^2,
$$
отсюда с учетом \eqref{6.6} имеем
\begin{equation}\label{6.25}
(F_{C(\phi),C_N(\phi)}(t))^2\le \lambda^2   \left(\sum\nolimits_{j=N}^\infty \hat \varphi_{1,j+1}T_{1,j+1}(t)\right)^2.
\end{equation}
Сопоставляя \eqref{6.25} с\eqref{6.23}, получаем
\begin{equation}\label{6.26}
\|C_N(\phi)- C_N'(\phi)\|_N^2\le \lambda^2\int_{-1}^1\left(\sum\nolimits_{j=N+1}^\infty \hat \varphi_{1,j}T_{1,j}(t)\right)^2 \mu(t)dt,
\end{equation}
где согласно \eqref{6.6}
\begin{equation}\label{6.27}
 \hat \varphi_{1,j}=\int\limits_{-1}^1\varphi'(t)T_{j-1}(t)\mu(t)dt \quad(j\ge1)
\end{equation}
-- $j-1$-вый коэффициент Фурье-Чебышева функции $\varphi'(t)=hf(a+h(t+1),\varphi(t))$.

Подводя итоги, из \eqref{6.22} и \eqref{6.26}  мы можем выводим следующий результат.
\begin{theorem} Пусть область $G$ такова, что $[a,a+2h]\times\mathbb{R}\subset G$, функция $f(x,y)$ непрерывна в области $G$ и удовлетворяет условию Липшица \eqref{6.13}, а для $h$ и $\lambda$ справедливо неравенство $h\lambda\kappa<1$, где $\kappa$ определено равенством \eqref{6.16}. Далее, пусть $l_2$ гильбертово пространство, состоящее из последовательностей $C=(c_0,c_1,\ldots)$, для которых введена норма $\|C\|=\left(\sum_{j=0}^\infty c_j^2\right)^\frac12$,   оператор $A: l_2\to l_2$ сопоставлят точке $C\in l_2$ точку $C'\in l_2$ по правилу \eqref{6.9}. Кроме того, пусть $A_N:\mathbb{R}^N\to \mathbb{R}^N$ -- конечномерный аналог оператора $A$, cопоставляющий точке $C_N=(c_0,\ldots,c_{N-1})\in \mathbb{R}^N $ точку  $C'_N=(c_0',\ldots,c_{N-1}')\in \mathbb{R}^N $ по правилу \eqref{6.18}.
Тогда операторы $A: l_2\to l_2$ и $A_N:\mathbb{R}^N\to \mathbb{R}^N$ являются сжимающими и, следовательно, существуют их неподвижные точки $C(\phi)=(c_0(\phi),c_1(\phi),\ldots,c_{N-1}(\phi),\ldots)\in l_2$ и $\bar C_N(\phi)=(\bar c_0(\phi),\bar c_1(\phi),\ldots,\bar c_{N-1}(\phi))\in \mathbb{R}^N$, соответственно, для которых имеет место следующее неравенство
\begin{equation}\label{6.28}
\|C_N(\phi)-\bar C_N(\phi)\|_N\le \frac\lambda{1-h\kappa\lambda}\left(\int_{-1}^1\left(\sum\nolimits_{j=N+1}^\infty \hat \varphi_{1,j}T_{1,j}(t)\right)^2 \mu(t)dt\right)^\frac12,
\end{equation}
где $C_N(\phi)=(c_0(\phi),c_1(\phi),\ldots,c_{N-1}(\phi))$ -- конечная последовательность, составленная из первых $N$ компонент точки  $C(\phi)$ .
\end{theorem}



Заметим, что величина $V_N(t)=\sum\nolimits_{j=N+1}^\infty \hat \varphi_{1,j}T_{1,j}(t)$,
фигурирующая в правой части неравенства \eqref{6.28}, представляет собой остаточный член ряда Фурье функции $\varphi(t)$ по полиномам $T_{1,k}(t)$ $(k=0,1,\ldots)$, образующим ортонормированную систему по Соболеву относительно скалярного произведения \eqref{1.3}, ассоциированным с полиномами Чебышева $T_k(t)$ посредством равенств \eqref{1.4} и \eqref{1.5} с $r=1$. Скорость стремления к нулю $V_N(t)$ при $N\to\infty$, а следовательно (см.\eqref{6.28}) и погрешности $\|C_N(\phi)- \bar C_N(\phi)\|_N$, существенно зависит от свойств гладкости на $[-1,1]$ искомого решения $\varphi(t)$, а это, в свою очередь, непосредственно зависит от свойств функции $f(x,y)$, фигурирующей в исходном уравнении \eqref{6.1}.

В заключение этого параграфа мы займемся преобраованиями сумм вида
\begin{equation}\label{6.29}
 S_N(x)=h\sum\nolimits_{k=0}^{N-1}p_kT_{1,k+1}(x),
\end{equation}
направленными на то, чтобы показать, что при их вычислении на сетках $x_i=\cos\frac{i\pi}{M}$ $(0\le i\le M)$ и $t_i=\cos\frac{(2i+1)\pi}{2M}$ $(0\le i\le M-1)$может быть использовано быстрое дискретное косинусное преобразование Фурье. С этой целью обратимся к равенствам \eqref{5.16} и \eqref{5.17} (следствие 5) и запишем  \eqref{6.29} в следующем виде
$$
S_N(x)=\frac{hp_0}{\sqrt{2}}(1+x)+\frac{hp_1}2(x^2-1)+ h\sum\nolimits_{k=2}^{N-1}p_k[{T_{k+1}(x)\over2(k+1)}- {T_{k-1}(x)\over2(k-1)} -\frac{(-1)^k}{k^2-1}]
$$
\begin{multline}\label{6.30}
  =-h\sum\nolimits_{k=2}^{N-1} \frac{(-1)^kp_k}{k^2-1}+\frac{hp_0}{\sqrt{2}}(1+x)+\frac{hp_1}2(x^2-1)+
 \\
  h\sum\nolimits_{k=2}^{N-1}p_k[{T_{k+1}(x)\over2(k+1)}- {T_{k-1}(x)\over2(k-1)} ].
\end{multline}
Поскольку
$$
\sum\nolimits_{k=2}^{N-1}p_k[{T_{k+1}(x)\over2(k+1)}- {T_{k-1}(x)\over2(k-1)} ]=
\sum\nolimits_{k=2}^{N-1}p_k{T_{k+1}(x)\over2(k+1)}-\sum\nolimits_{k=2}^{N-1}p_k{T_{k-1}(x)\over2(k-1)}=
$$
$$
\sum\nolimits_{k=2}^{N-1}p_k{T_{k+1}(x)\over2(k+1)}-\sum\nolimits_{k=0}^{N-3}p_{k+2}{T_{k+1}(x)\over2(k+1)}=
$$
$$
-{p_{2}\over2}T_{1}(x)-{p_{3}\over4}T_{2}(x)+\sum\nolimits_{k=3}^{N-2}{p_{k-1}-p_{k+1}\over2k}T_k(x)
+{p_{N-2}\over2(N-1))}T_{N-1}(x)+{p_{N-1}\over2N}T_{N}(x),
$$
то   мы можем переписать \eqref{6.30}   следующем виде
\begin{multline}\label{6.31}
 S_N(x)=h\bar p(N)+\frac{hp_0}{\sqrt{2}}(1+x)+\frac{hp_1}2(x^2-1)-{hp_{2}\over2}T_{1}(x)-{hp_{3}\over4}T_{2}(x)+
 \\
h\sum\nolimits_{k=3}^{N-2}{p_{k-1}-p_{k+1}\over2k}T_k(x)
+{hp_{N-2}\over2(N-1))}T_{N-1}(x)+{hp_{N-1}\over2N}T_{N}(x),
  \end{multline}
где
\begin{equation}\label{6.32}
 \bar p(N)= -\sum\nolimits_{k=2}^{N-1} \frac{(-1)^kp_k}{k^2-1}.
\end{equation}
 Далее имеем
$$
\frac{hp_0}{\sqrt{2}}(1+x)=hp_0T_0+\frac{hp_0}{\sqrt{2}}T_1(x), \frac{hp_1}2(x^2-1)= \frac{hp_1}4T_2(x) -\frac{hp_1}{2\sqrt{2}}T_0,
$$
поэтому из \eqref{6.31} и  \eqref{6.32} окончательно получаем
\begin{multline}\label{6.33}
 S_N(x)=h\bar p(N)+h(p_0-\frac{p_1}{2\sqrt{2}})T_0+h(\frac{p_0}{\sqrt{2}}-{p_{2}\over2})T_1(x)+
 \\
h\sum\nolimits_{k=2}^{N-2}{p_{k-1}-p_{k+1}\over2k}T_k(x)
+{hp_{N-2}\over2(N-1))}T_{N-1}(x)+{hp_{N-1}\over2N}T_{N}(x).
  \end{multline}


















%%%%%%%%%%%%%%%%%%%%%%%%%%%%%
%%%%%%%%%%%%%%%%%%%%%%%%%%%%%
%%%%%%%%%%%%%%%%%%%%%%%%%%%%%



%
%\begin{thebibliography}{20}
%
%
%
%\RBibitem{KwonLittl1}
%\by K.\,H. Kwon and L.\,L. Littlejohn
%\paper The orthogonality of the Laguerre polynomials $\{L_n^{(-k)}(x)\}$ for positive integers $k$
%\inbook Ann. Numer. Anal.
%\vol
%\issue 2
%\yr 1995
%\pages 289 –- 303.
%
%\RBibitem{ KwonLittl2}
%\by K.\,H. Kwon and L.\,L. Littlejohn
%\paper Sobolev orthogonal polynomials and second-order differential equations
%\inbook Rocky Mountain J. Math.
%\vol 28
%\issue
%\yr 1998
%\pages 547 –- 594.
%
%\RBibitem{MarcelAlfaroRezola }
%\by F. Marcellan, M. Alfaro and M.L. Rezola
%\paper Orthogonal polynomials on Sobolev spaces: old and new directions
%\inbook Journal of Computational and Applied Mathematics
%\vol 48
%\issue
%\yr 1993
%\pages 113 -- 131.
%\publaddr North-Holland
%
%\RBibitem{IserKoch }
%\by A. Iserles, P.E. Koch, S.P. Norsett and J.M. Sanz-Serna
%\paper On polynomials  orthogonal  with respect  to certain Sobolev inner products
%\inbook ,  J. Approx. Theory
%\vol 65
%\issue
%\yr 1991
%\pages 151-175.
%
%\RBibitem{Meijer}
%\by H.\,G. Meijer, Laguerre polynimials generalized to a certain
%\paper  Laguerre polynimials generalized to a certain discrete Sobolev inner product space
%\inbook ,  J. Approx. Theory
%\vol 73
%\issue
%\yr 1993
%\pages 1-16.
%
%\RBibitem{Lopez1995}
%\by Lopez G. Marcellan F. Vanassche W.
%\paper Relative Asymptotics for Polynomials Orthogonal with Respect to a Discrete Sobolev Inner-Product
%\inbook Constr. Approx.
%\vol 11:1
%\yr 1995
%\pages 107–137
%
%
%
%
%\RBibitem{MarcelXu}
%\by F. Marcellan and Yuan Xu
%\paper On Sobolev orthogonal polynomials
%\inbook  Expositiones Mathematicae
%\vol 33
%\issue 3
%\yr 2015
%\pages 308--352
%
%\RBibitem{Shar2016}
%\by И.И. Шарапудинов
%\paper Системы функций, ортогональные по Соболеву, порожденные ортогональными функциями
%\inbook Материалы 18-й международной Саратовской зимней школы «Современные проблемы теории функций и их приложения»
%\publ ООО «Издательство «Научная книга»
%\yr 2016
%\pages 329-332
%\publaddr Саратов
%
%\RBibitem{Gonchar1975}
%\by А. А. Гончар
%\paper О сходимости аппроксимаций Паде для некоторых классов мероморфных функций
%\inbook Матем. сб.
%\vol 97(139):4(8)
%\yr 1975
%\pages 607–629
%
%
%\RBibitem{Tref1}
%\by   L.N. Trefethen
%\book Spectral methods in Matlab
%\yr 2000
%\serial
%\publ SIAM
%\publaddr Fhiladelphia
%\vol
%
%\RBibitem{Tref2}
%\by   L.N. Trefethen
%\book Finite difference and spectral methods for ordinary and partial differential equation
%\yr 1996
%\serial
%\publ Cornell University
%\publaddr
%\vol
%
%
%\RBibitem{SolDmEg}
%\by  В.В. Солодовников, А.Н. Дмитриев, Н.Д. Егупов
%\book Спектральные методы расчета и проектирования систем управления
%\yr 1986
%\serial
%\publ Машиностроение
%\publaddr Москва
%\vol
%
%\RBibitem{Pash}
%\by С. Пашковский
%\paper
%\inbook Вычислительные применения многочленов и рядов Чебышева
%\publ Наука
%\yr 1983
%\pages
%\publaddr Москва
%
%\RBibitem{Arush2010}
%\by О. Б. Арушанян, Н. И. Волченскова, С. Ф. Залеткин
%\paper Приближенное решение обыкновенных дифференциальных уравнений с использованием рядов Чебышева
%\inbook Сиб. электрон. матем. изв.
%\issue 7
%\yr 1983
%\pages 122–131
%
%\RBibitem{Arush2013}
%\by О. Б. Арушанян, Н. И. Волченскова, С. Ф. Залеткин
%\paper Метод решения задачи Коши для обыкновенных дифференциальных уравнений с использованием рядов Чебышeва
%\inbook Выч. мет. программирование
%\issue 14:2
%\yr 2013
%\pages 203-214
%
%\RBibitem{Arush2014}
%\by О. Б. Арушанян, Н. И. Волченскова, С. Ф. Залеткин
%\paper Применение рядов Чебышева для интегрирования обыкновенных дифференциальных уравнений
%\inbook Сиб. электрон. матем. изв.
%\issue 11
%\yr 2014
%\pages 517-531
%
%\RBibitem{Lukom2016}
%\by Д.С. Лукомский, П.А. Терехин
%\paper Применение системы Хаара к численному решению задачи Коши для линейного дифференциального уравнения первого порядка
%\inbook Материалы 18-й международной Саратовской зимней школы «Современные проблемы теории функций и их приложения»
%\publ ООО «Издательство «Научная книга»
%\yr 2016
%\pages 171-173
%\publaddr Саратов
%
%
%\RBibitem{MMG2016}
%\by М.Г. Магомед-Касумов
%\paper Приближенное решение обыкновенных дифференциальных уравнений с использованием смешанных рядов по системе Хаара
%\inbook Материалы 18-й международной Саратовской зимней школы «Современные проблемы теории функций и их приложения»
%\publ ООО «Издательство «Научная книга»
%\yr 2016
%\pages 176-178
%\publaddr Саратов
%
%
%
%
%
%
%
%
%
%
%
%
%
%
%\RBibitem{Shar11}
%\by И.\,И. Шарапудинов
%\paper Приближение функций с переменной гладкостью суммами Фурье Лежандра
%\inbook Математический сборник
%\vol 191
%\issue 5
%\yr 2000
%\pages 143 -- 160.
%
%
%\RBibitem{Shar12}
%\by И.\,И. Шарапудинов
%\paper Аппроксимативные свойства операторов $\mathcal{ Y}_{n+2r}(f)$ и их дискретных аналогов
%\inbook Математические заметки
%\vol 72
%\issue 5
%\yr 2002
%\pages 765–-795.
%
%\RBibitem{Shar13}
%\by И.\,И. Шарапудинов
%\paper Смешанные ряды по ортогональным полиномам
%\inbook
%\publ Издательство Дагестанского научного центра
%\yr 2004
%\pages 1 --176
%\publaddr Махачкала
%
%
%
%
%\RBibitem{Shar14}
%\by И.\,И. Шарапудинов
%\paper Смешанные ряды по полиномам Чебышева, ортогональным на равномерной сетке
%\inbook Математические заметки
%\vol 78
%\issue 3
%\yr 2005
%\pages 442–-465.
%
%\RBibitem{Shar15}
%\by И.\,И. Шарапудинов
%\paper Аппроксимативные свойства смешанных рядов по полиномам Лежандра на классах $W^r$
%\inbook Математический сборник
%\vol 197
%\issue 3
%\yr 2006
%\pages 135–154.
%
%
%\RBibitem{Shar16}
%\by И.\,И. Шарапудинов
%\paper Аппроксимативные свойства средних типа Валле-Пуссена частичных сумм смешанных рядов по полиномам Лежандра
%\inbook Математические заметки
%\vol 84
%\issue 3
%\yr 2008
%\pages 452 -- 471.
%
%\RBibitem{Shar17}
%\by И.\,И. Шарапудинов
%\paper Смешанные ряды по ультрасферическим полиномам и их аппроксимативные свойства
%\inbook Математический сборник
%\vol 194
%\issue 3
%\yr 2003
%\pages 115--148
%
%
%
%
%\RBibitem{Shar18}
%\by И.\,И. Шарапудинов, Т.\, И. Шарапудинов
%\paper Смешанные ряды по полиномам Якоби и Чебышева и их дискретизация
%\inbook Математические заметки
%\vol 88
%\issue 1
%\yr 2010
%\pages 116 -- 147
%
%\RBibitem{Shar19}
%\by И.\,И. Шарапудинов,  Г.\, Н. Муратова
%\paper  Некоторые свойства r-кратно интегрированных рядов по системе Хаара
%\inbook Изв. Сарат. ун-та. Нов. сер. Сер. Математика. Механика. Информатика
%\vol 9
%\issue 1
%\yr 2009
%\pages 68 -- 76
%
%
%\RBibitem{Lopez1995}
%\by Lopez G. Marcellan F. Vanassche W.
%\paper Relative Asymptotics for Polynomials Orthogonal with Respect to a Discrete Sobolev Inner-Product
%\inbook Constr. Approx.
%\vol 11:1
%\yr 1995
%\pages 107–137
%
%\RBibitem{Gonchar1975}
%\by А. А. Гончар
%\paper О сходимости аппроксимаций Паде для некоторых классов мероморфных функций
%\inbook Матем. сб.
%\vol 97(139):4(8)
%\yr 1975
%\pages 607–629
%
%
%
%
%
%\RBibitem{Sege}
%\by Г. Сеге
%\paper
%\inbook Ортогональные многочлены
%\publ Физматгиз
%\yr 1962
%\pages
%\publaddr Москва
%
%\RBibitem{fiht2}
%\by Г.М. Фихтенгольц
%\inbook Курс дифференциального и интегрального исчисления
%\publ Физматлит
%\vol 2
%\yr 2001
%\pages 810
%\publaddr Москва
%
%
%
%
%\RBibitem{SHII}
%\by И.\,И. Шарапудинов
%\paper Некоторые специальные ряды по общим полиномам Лагерра и ряды Фурье по полиномам Лагерра, ортогональным по Соболеву
%\inbook Дагестанские электронные математические известия
%\vol 4
%\issue
%\yr 2015
%\pages
%
%
%
%
%
%
%
%\RBibitem{Gasper}
%\by G. Gasper
%\paper Positiviti and special function
%\inbook  Theory and appl.Spec.Funct. Edited by Richard A.Askey.
%\vol
%\issue
%\yr 1975
%\pages 375-433
%
%
%
%\end{thebibliography}
%
