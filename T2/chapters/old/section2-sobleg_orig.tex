\chapter{Ортогональные по Соболеву полиномы, порожденные полиномами Якоби и Лежандра, и специальные ряды со свойством прилипания их частичных сумм }


%Для  произвольного натурального $r$ рассмотрены полиномы $p^{\alpha,\beta}_{r,k}(x)$ $(k=0,1,\ldots)$, ортонормированные относительно скалярного произведения типа Соболева следующего вида
%$$
%<f,g>=\sum_{\nu=0}^{r-1}f^{(\nu)}(-1)g^{(\nu)}(-1)+
%\int_{-1}^{1}f^{(r)}(t)g^{(r)}(t)(1-t)^\alpha(1+t)^\beta dt
%$$
%и изучены  их свойства. Введены в рассмотрение ряды Фурье по полиномам $p_{r,k}(x)=p^{0,0}_{r,k}(x)$ и некоторые их обобщения, частичные суммы которых  сохраняют некоторые важные  свойства частичных сумм ряда Фурье по полиномам $p_{r,k}(x)$, в том числе и свойство $r$-кратного совпадения (<<прилипания>>) частичных сумм ряда Фурье по полиномам $p_{r,k}(x)$  в  точках $-1$ и $1$ между собой и с исходной функцией $f(x)$.  Основное внимание уделено  исследованию вопросов приближения гладких и аналитических функций  частичными суммами упомянутых обобщений, представляющих собой   специальные ряды  по ультрасферическим полиномам Якоби со свойством <<прилипания>> их частичных сумм  точках $-1$ и $1$ .




\section{Введение}
В настоящей статье мы продолжаем исследования по теории смешанных рядов, ассоциированных с  полиномами Якоби и Лежандра, и  некоторых специальных рядов по ультрасферическим полиномам Якоби, которые были начаты в \cite{Shar11} и продолжены в работах
\cite{Shar12} -- \cite{sharap3}.  Следует при этом отметить, что  в \cite{Shar11} -- \cite{sharap3} смешанные ряды не трактовались как ряды Фурье по полиномам, ортогональным по Соболеву. Вместо этого при исследовании аппроксимативных свойств смешанных рядов в \cite{Shar11} -- \cite{sharap3} использовался тот факт, что  частичная сумма смешанного ряда представляет собой проектор на соответствующее подпространство алгебраических полиномов, не обращая внимание на то, что она является суммой Фурье по тем самым полиномам, которые фигурируют в смешанном ряде и образуют ортонормированную систему относительно скалярного произведения типа Соболева. В настоящей работе, напротив,  мы будем рассматривать смешанные ряды именно как ряды Фурье по полиномам, ортонормированным относительно скалярных произведений типа Соболева с дискретными массами, что позволяет применять к исследованию некоторых важных свойств этих рядов  терминологию и аппарат теории гильбертовых пространств. С этой целью для  натурального $r$ и $\alpha,\beta>-1$ рассмотрены полиномы $p^{\alpha,\beta}_{r,k}(x)$ $(k=0,1,\ldots)$, ассоциированные с  классическими полиномами Якоби, ортонормированные относительно скалярного произведения типа Соболева
\begin{equation}\label{1.1}
<f,g>=\sum_{k=0}^{r-1}f^{(k)}(-1)g^{(k)}(-1)+\int_{-1}^1 f^{(r)}(t)g^{(r)}(t)\kappa(t)dt,
\end{equation}
где $\kappa(t)=(1-t)^\alpha(1+t)^\beta$. Для этих полиномов установлены явный вид (теорема 1)  и связь с полиномами Якоби $P_{k+r}^{\alpha-r,\beta-r}(x)$ (теорема 2).  Особое внимание уделено случаю $\alpha=\beta=0$. Соответствующие полиномы $p_{r,k}(x)=p^{0,0}_{r,k}(x)$, ортогональные по Соболеву, определяются равенством \eqref{3.27} при $\alpha=\beta=0$ (порождены полиномами Лежандра). Они связаны с полиномами Якоби особенно просто (см.  \eqref{3.42} и \eqref{3.44}). Именно этот факт позволяет   (см. \S 5) ввести в рассмотрение некоторые специальные ряды по ультрасферическим полиномам Якоби $p_k^{\alpha,\alpha}(x)$, частичные суммы $\sigma_{r,n}^\alpha(f)$ которых обладают важным для приложений (см. начало \S 3) свойством <<прилипания>> в точках $\pm1$. Эти ряды возникают как естественное обобщение смешанного ряда по полиномам Лежандра $p_k(x)=p_k^{0,0}(x)$, представляющего собой ряд Фурье  по полиномам $p_{r,k}(x)$  $(k=0,1,\ldots)$. При этом отметим, что в случае $\alpha=r$ введенный в \S 5 специальный ряд совпадает с рядом Фурье по полиномам $p_{r,k}(x)$  $(k=0,1,\ldots)$ (или, что то же, со смешанным рядом по полиномам Лежандра).
Упомянутое выше свойство прилипания частичных сумм $\sigma_{r,n}^\alpha(f)=\sigma_{r,n}^\alpha(f,x)$  с исходной функцией $f(x)$ в точках $\pm1$  (и, следовательно, $\sigma_{r,n}^\alpha(f,x)$ и $\sigma_{r,m}^\alpha(f,x)$ друг с другом) состоит в том, что если функция $f(x)$ $r-1$-раз дифференцируема в точках $\pm1$ , то  $f^{(\nu)}(\pm1)=(\sigma_{r,N}^\alpha(f,\pm1))^{(\nu)}$, $\nu=0,1,\ldots, r-1$. Это свойство позволяет использовать $\sigma_{r,n}^\alpha(f)$ как аппарат приближения в задаче о взвешенной с весом $(1-x^2)^{-r+\frac\alpha2+\frac14}$ аппроксимации алгебраическими полиномами дифференцируемых и аналитических функций, заданных на $[-1,1]$. Основные результаты настоящей работы, сформулированные в теоремах 3 и 4, касаются оценок  взвешенного приближения на $[-1,1]$ с весом $(1-x^2)^{-r+\frac\alpha2+\frac14}$  функции $f(x)$ частичными суммами $\sigma_{r,n}^\alpha(f,x)$. Оценка
$ |f(x)-\sigma_{r,N}^\alpha(f,x)|/(1-x^2)^{r-\frac{\alpha}{2}-\frac14}\le c(r,\alpha)\sum_{k=0}^N\frac{E_{N+k}^r(f)}{k+1}$, установленная в теореме 3, в которой величина $E_{n}^r(f)$ представляет собой взвешенное с весом $(1-x^2)^{-r/2}$ наилучшее приближение функции $f$ на $[-1,1]$ алгебраическими полиномами $p_n(x)$ степени $n$, которые подчиняются условиям $p_n^{(\nu)}(\pm1)=f^{(\nu)}(\pm1)$ $(\nu=0,\ldots, r-1)$, показывает, что  частичные суммы $\sigma_{r,n}^\alpha(f,x)$ обладают весьма привлекательными с точки зрения различных приложений аппроксимативными свойствами. Кроме того, оценка $|f(x)-\sigma_{r,N}^\alpha(f,x)|/(1-x^2)^{r-\frac{\alpha}{2}-\frac14}\le\frac{c(r,\alpha, B)}{(1-q)^2}q^{N-2r+1}$, полученная в теореме 4 для аналитической в эллипсе $\mathcal{ E}_q$  с фокусами в точках $\pm1$, сумма полуосей которого равна $R=1/q>1$, функции $f$ показывает, что   по своим аппроксимативным свойствам на классах аналитических функций $A_q(B)$ (см. \S5) $\sigma_{r,n}^\alpha(f,x)$ не уступают суммам Фурье $S_n(f,x)$ по полиномам Чебышева первого рода. При этом, как уже отмечалось, $\sigma_{r,n}^\alpha(f,x)$ обладает свойством прилипания:  $f^{(\nu)}(\pm1)=(\sigma_{r,N}^\alpha(f,\pm1))^{(\nu)}$, $\nu=0,1,\ldots, r-1$, тогда как суммы Фурье-Чебышева $S_n(f,x)$ этим важным свойством, вообще говоря, не обладают.

Результаты, сформулированные в виде теорем 1 и 2 (\S 4), направлены на исследование свойств самих полиномов $p^{\alpha,\beta}_{r,n+r}(x)$.   В теореме 1 получен явный вид полинома $p^{\alpha,\beta}_{r,n+r}(x)$, представляющий из себя разложение $p^{\alpha,\beta}_{r,n+r}(x)$ по степеням $(x+1)^{k+r}$ с $0\le k\le n$. Этот результат нацелен в первую очередь на изучение асимптотического поведения полиномов  $p^{\alpha,\beta}_{r,n+r}(x)$  в окрестности точки $x=-1$ при $n\to\infty$. Кроме того, он может быть использован для нахождения значений полиномов $p^{\alpha,\beta}_{r,n+r}(x)$ с небольшими степенями $n+r$ в заданной точке $x$. Это касается также наиболее детально изученного нами ранее случая $\alpha=\beta=0$.  Дело заключается в том, что результат, полученный в теореме 2, в которой установлена связь полиномов $p^{\alpha,\beta}_{r,n+r}(x)$ с классическими полиномами Якоби $P_{n+r}^{\alpha-r,\beta-r}(x)$, не справедлив для нескольких начальных степеней $n+r$ и, стало быть, не может быть использован для описания свойств полиномов $p^{\alpha,\beta}_{r,n+r}(x)$ с такими степенями. Например, для отмеченного уже частного случая $\alpha=\beta=0$ равенства \eqref{3.42} и \eqref{3.44} теряют смысл для полиномов $p_{r,k}(x)=p^{0,0}_{r,k}(x)$, у которых $r\le k\le 2r-1$. Теорема 1 позволяет, в частности, восполнить этот прообел (см. \eqref{3.46}). Что касается самой теоремы 2, то она может быть использована при исследовании асимптотических свойств полиномов $p^{\alpha,\beta}_{r,n+r}(z)$ при $|1+z|\ge\varepsilon>0$ и $n\to\infty$. Мы не будем останавливаться в настоящей работе на подробном обсуждении этого вопроса, поскольку он является объектом рассмотрения другой работы.

 На протяжении всей работы существенно будет использован аппарат теории общих полиномов Якоби (не только ортогональных). Поэтому мы соберем для удобства ссылок в следующем параграфе необходимые сведения об этих полиномах.

 \section{Некоторые сведения о полиномах Якоби}

 Для произвольных действительных $\alpha$ и $\beta$ полиномы Якоби  $P_n^{\alpha,\beta}(x)$ можно определить \cite{Sege}  с помощью формулы Родрига
 \begin{equation}\label{2.1}
P_n^{\alpha,\beta}(x) = {(-1)^n\over2^nn!}{1\over\kappa(x)}{d^n\over
dx^n} \left\{\kappa(x)\sigma^n(x)\right\},
\end{equation}
где $\alpha,\beta$ -- произвольные действительные числа, $\kappa(x)=
(1-x)^\alpha(1+x)^\beta,\,\,\sigma(x)=1-x^2$. Если
$\alpha,\beta>-1$, то полиномы Якоби образуют ортогональную
систему с весом $\kappa(x)$, т.е.
\begin{equation}\label{2.2}
\int_{-1}^1P_n^{\alpha,\beta}(x)P_m^{\alpha,\beta}(x)\kappa(x)dx =
h_n^{\alpha,\beta}\delta_{nm},
\end{equation}
где
\begin{equation}\label{2.3}
h_n^{\alpha,\beta} =
{\Gamma(n+\alpha+1)\Gamma(n+\beta+1)2^{\alpha+\beta+1} \over
n!\Gamma(n+\alpha+\beta+1)(2n+\alpha+\beta+1)}.
\end{equation}
В частности, при $\alpha,\beta=0$ мы получим классические полиномы Лежандра $P_n(x)=P_n^{0,0}(x)$, для которых
\begin{equation*}
\int_{-1}^1P_n(x)P_m(x)\kappa(x)dx = \frac{2}{2n+1}\delta_{nm}.
\end{equation*}



Нам понадобятся еще следующие свойства полиномов Якоби~\cite{Sege}:


\begin{equation}\label{2.4}
{d\over dx}P_n^{\alpha,\beta}(x) =
{1\over2}(n+\alpha+\beta+1)P_{n-1}^{\alpha+1,\beta+1}(x),
\end{equation}
\begin{equation}\label{2.5}
{d^\nu\over dx^\nu}P_n^{\alpha,\beta}(x) =
{(n+\alpha+\beta+1)_\nu\over2^\nu} P_{n-\nu}^{\alpha+\nu,\beta+\nu}(x),
\end{equation}
где $(a)_0=1$, $(a)_\nu=a(a+1)\dots(a+\nu-1)$,

 \begin{equation}\label{2.6}
 {n\choose l}P_n^{\alpha,-l}(x)= {n+\alpha\choose
l}\left({x+1\over2}\right)^lP_{n-l}^{\alpha,l}(x),
     \quad 1\le l \le n,
\end{equation}
\begin{equation}\label{2.7}
P_n^{\alpha,\beta}(t) ={n+\alpha\choose n}
\sum_{k=0}^n{(-n)_k(n+\alpha+\beta+1)_k\over k!(\alpha+1)_k}
\left({1-t\over 2}\right)^k,
\end{equation}
\begin{equation}\label{2.8}
(1-x)P_n^{\alpha+1,\beta}(x)={2\over2n+\alpha+\beta+2}
\left[(n+ \alpha+1)P_n^{\alpha,\beta}(x)-
(n+1)P_{n+1}^{\alpha,\beta}(x)\right],
\end{equation}

\begin{equation}\label{2.9}
P_n^{\alpha,\beta}(-1)=(-1)^n {n+\beta\choose n},\quad P_n^{\alpha,\beta}(1)= {n+\alpha\choose n},
\end{equation}


Пусть $-1<\alpha, \beta$ -- произвольные вещественные числа,
$$
s(\theta)=s^{\alpha,\beta}(\theta)=\pi^{-\frac12}
\left(\sin\frac{\theta}{2}\right)^{-\alpha-\frac12}
\left(\cos\frac{\theta}{2}\right)^{-\beta-\frac12},
$$
$$
\lambda_n=n+\frac{\alpha+\beta+1}{2}, \quad\gamma=-
\left(\alpha+\frac{1}{2}\right)\frac{\pi}{2}.
$$
Тогда для $0<\theta<\pi$ имеет место асимптотическая формула
\begin{equation}\label{2.10}
P_n^{\alpha,\beta}(\cos\theta)=n^{-\frac12}s(\theta)
\left(\cos(\lambda_n\theta+\gamma)+
\frac{v_n(\theta)}{n\sin\theta}\right),
\end{equation}
в которой для функции $v_n(\theta)=v_n(\theta;\alpha,\beta)$
справедлива оценка
\begin{equation}\label{2.11}
|v_n(\theta)|\le c(\alpha,\beta,\delta)\quad
 \left(0<\frac{\delta}{n}\le\theta\le\pi-\frac{\delta}{n}\right).
\end{equation}
Имеет место также следующая оценка
\begin{equation}\label{2.12}
|P_n^{\alpha,\beta}(\cos\theta)|\le c(\alpha,\beta,\gamma)n^\alpha, \quad |\theta|\le \frac{\gamma}{n}.
\end{equation}

Из \eqref{2.3}, \eqref{2.10} -- \eqref{2.12} мы выводим для ортонормированных   полиномов Якоби
\begin{equation}\label{2.13}
p_n^{\alpha,\beta}(x)=P_n^{\alpha,\beta}(x)/\sqrt{h_n^{\alpha,\beta}}\quad(n=0,1,\ldots)
\end{equation}
 следующую весовую оценку
\begin{equation}\label{2.14}
\left(\sin\frac{\theta}{2}\right)^{\alpha+\frac12}
\left(\cos\frac{\theta}{2}\right)^{\beta+\frac12}
|p_n^{\alpha,\beta}(\cos\theta)|\le c(\alpha,\beta),
\end{equation}
где $\alpha,\beta\ge-1/2$, $0\le\theta\le\pi$. Указанные оценки вместе с \\
\textit{ формулой Кристоффеля-Дарбу }
$$
 K_n^{\alpha,\beta}(x,y)=
\sum_{k=0}^n{P_k^{\alpha,\beta}(x)P_k^{\alpha,\beta}(y)\over
h_k^{\alpha,\beta}}=
 $$
\begin{equation}\label{2.15}
 {2^{-\alpha-\beta}\over
2n+\alpha+\beta+2} {\Gamma(n+2)\Gamma(n+\alpha+\beta+2)\over
\Gamma(n+\alpha+1)\Gamma(n+\beta+1)}
 {P_{n+1}^{\alpha,\beta}(x)P_n^{\alpha,\beta}(y)-
P_n^{\alpha,\beta}(x)P_{n+1}^{\alpha,\beta}(y)\over x-y}
\end{equation}
играют основополагающую роль при изучении аппроксимативных свойств частичных сумм специальных рядов со свойством прилипания по ультрасферическим полиномам Якоби.

Отметим еще следующую формулу, доказательство которой можно найти в работе \cite{Gasper}:


$$
{P_n^{a,a}(x)\over P_n^{a,a}(1)}=\sum_{j=0}^{[n/2]}
   { n!(\alpha+1)_{n-2j}(n+2a+1)_{n-2j}(1/2)_j(a-\alpha)_j
    \over (n-2j)!(2j)!(a+1)_{n-2j}(n-2j+2\alpha+1)_{n-2j}}
     $$
\begin{equation}\label{2.15}
    \times{1\over (n-2j+a+1)_j(n-2j+\alpha+3/2)_j}
     {P_{n-2j}^{\alpha,\alpha}(x)\over
P_{n-2j}^{\alpha,\alpha}(1)},
\end{equation}
где $[b]$ -- целая часть числа $b$.





\section{Полиномы, ортогональные по Соболеву, порожденные полиномами Якоби и Лежандра}

В течении последних двух десятилетий в теории ортогональных полиномов появилось и стало
интенсивно развиваться (см. \cite{KwonLittl1} -- \cite{Shar2016} и цитированную там литературу) новое направление, которое принято обозначать словами: <<полиномы, ортогональные по Соболеву>>. Растущий интерес  к этому направлению теории ортогональных полиномов можно объяснить в том числе и тем обстоятельством, что ряды Фурье по полиномам, ортогональным по Соболеву, оказались естественным и весьма удобным инструментом для представления решений  дифференциальных (разностных) уравнений. Это можно показать, в частности, на примере  задачи Коши для линейного дифференциального уравнения
\begin{equation}\label{3.1}
 a_r(x)y^{(r)}(x)+a_{r-1}(x)y^{(r-1)}(x)+\cdots+a_0(x)y(x)=h(x)
 \end{equation}
с начальными условиями $y^{(k)}(-1)=y_k$, $k=0,1,\ldots,r-1$.  Наряду с различными сеточными методами для решения этой задачи часто применяют так называемые спектральные методы \cite{Tref1} -- \cite{MMG2016}. Напомним, что суть спектрального метода решения задачи Коши  для ОДУ \eqref{3.1} заключается в том, что в первую очередь искомое решение $y(x)$ представляется в виде ряда Фурье
\begin{equation}\label{3.2}
 y(x)=\sum_{k=0}^\infty \hat y_k\psi_k(x)
 \end{equation}
по подходящей ортонормированной системе $\{\psi_k(x)\}_{k=0}^\infty$ (чаще всего в качестве $\{\psi_k(x)\}_{k=0}^\infty$ используют    тригонометрическую систему, ортогональные полиномы, вэйвлеты, корневые функции того или иного дифференциального оператора  и некоторые другие). На втором этапе осуществляется подстановка вместо $y(x)$ ряда \eqref{3.2} в уравнение \eqref{3.1}. Это приводит к системе уравнений относительно неизвестных коэффициентов $\hat y_k$ ($k=0,1,\ldots$). На третьем этапе требуется решить эту систему с учетом начальных условий  $y^{(
k)}(-1)=y_k$, $k=0,1,\ldots,r-1$ исходной задачи Коши.
Одна из основных трудностей, которая возникает на этом этапе, состоит в том, чтобы
выбрать такой ортонормированный базис $\{\psi_k(x)\}_{k=0}^\infty$, для которого искомое решение $y(x)$ уравнения \eqref{3.1}, представленное в виде ряда  \eqref{3.2}, удовлетворяло начальным условиям $y^{(k)}(-1)=y_k$, $k=0,1,\ldots,r-1$. Более того, поскольку в результате решения системы уранений относительно неизвестных коэффициентов $\hat y_k$  будет найдено только конечное их число с $k=0,1,\ldots, n$, то весьма важно, чтобы частичная сумма ряда \eqref{3.2} вида
$ y_n(x)=\sum_{k=0}^n\hat y_k\psi_k(x)$, будучи приближенным решением рассматриваемой задачи Коши, также удовлетворяла  начальным условиям $y_n^{(k)}(-1)=y_k$, $k=0,1,\ldots,r-1$. Покажем, что  базис $\{\psi_k(x)=p_{r,k}^{\alpha,\beta}(x)\}_{k=0}^\infty$, состоящий из полиномов
$p_{r,k}^{\alpha,\beta}x)$, ортонормированных по Соболеву относительно скалярного произведения \eqref{1.1} и порожденных ортонормированными полиномами Якоби $p_k^{\alpha,\beta}(x)$  посредством равенства \eqref{3.27}, требуемыми свойствами обладает.
   С этой целью заметим, что ряд Фурье \eqref{3.2} в случае, когда $\{\psi_k(x)=p_{r,k}^{\alpha,\beta}(x)\}_{k=0}^\infty$, приобретает, как показано ниже, следующий смешанный вид
   \begin{equation}\label{3.3}
y(x)= \sum_{k=0}^{r-1} y^{(k)}(-1)\frac{(x+1)^k}{k!}+ \sum_{k=r}^\infty \hat y_{r,k}p_{r,k}^{\alpha,\beta}(x),
\end{equation}
где  $ \hat y_{r,k}=\int_{-1}^1 y^{(r)}(t)p_{k-r}^{\alpha,\beta}(t)\kappa(t)dt$.
С другой стороны, из определения \eqref{3.27} вытекает (см.\eqref{3.7}), что $(p_{r,k}^{\alpha,\beta}(-1))^{(\nu)}=0$ для всех $0\le\nu\le r-1$, поэтому как функция $y(x)$, представленная в виде ряда \eqref{3.3}, так и частичная  сумма этого ряда вида
$
y_n(x)= \sum_{k=0}^{r-1} y^{(k)}(-1)\frac{(x+1)^k}{k!}+ \sum_{k=r}^n \hat y_{r,k}p_{r,k}^{\alpha,\beta}(x)
$
удовлетворяют начальным условиям задачи Коши для уравнения \eqref{3.1}.
Таким образом, полиномы, ортогональные по Соболеву относительно скалярного произведения \eqref{1.1}, тесно связаны с задачей Коши для уравнения \eqref{3.1}. Можно также показать,что полиномы $p_{r,k}^{0,0}(x)$ могут служить удобным и эффективным методом приближенного решения двухточечной краевой задачи для уравнений типа \eqref{3.1}.





\subsection{Некоторые результаты общего характера}
Полиномы $p_{r,k}^{\alpha,\beta}x)$, определямые равенством   \eqref{3.27},  существенно использовались ранее в работах \cite{Shar11} -- \cite{SHII} в качестве вспомогательного аппарата при исследовании вопросов сходимости смешанных рядов вида \eqref{3.3}. Следует при этом отметить, что термин <<полиномы, ортогональные по Соболеву>> в работах \cite{Shar11} -- \cite{SHII}  не применялся и, соответственно, при исследовании свойств смешанных рядов аппарат теории ортогональных систем использовался неявно.  В настоящей работе мы будем, как уже отмечалось, рассматривать смешанные ряды по полиномам Якоби и Лежандра именно как ряды Фурье по полиномам $p_{r,k}^{\alpha,\beta}x)$, образующим  ортонормированную систему относительно скалярного произведения \eqref{1.1}. Общая идея, которая лежит в основе построения смешанных рядов с точки зрения геометрии гильбертовых пространств заключается в следующем. Предположим, что система полиномов  $\left\{q_k(x)\right\}$ ортонормирована  на $(a,b)$  c весом   $\rho(x)$, т.е.
 \begin{equation}\label{3.4}
\int\limits_a^bq_k(x)q_l(x)\rho(x)dx=\delta_{kl},
\end{equation}
где $\delta_{kl}$ -- символ Кронекера.  Мы можем определить следующие порожденные системой $\{q_k(x)\}$ полиномы
 \begin{equation}\label{3.5}
q_{r,r+k}(x) =\frac{1}{(r-1)!}\int\limits_a^x(x-t)^{r-1}q_{k}(t)dt, \quad k=0,1,\ldots.
\end{equation}
 Кроме того, определим конечный набор полиномов
  \begin{equation}\label{3.6}
q_{r,k}(x) =\frac{(x-a)^k}{k!}, \quad k=0,1,\ldots, r-1.
\end{equation}

 Из \eqref{3.5} и \eqref{3.6} следует, что для $x\in (a,b)$
 \begin{equation}\label{3.7}
(q_{r,k}(x))^{(\nu)} =\begin{cases}q_{r-\nu,k-\nu}(x),&\text{если $0\le\nu\le r-1$, $r\le k$,}\\
q_{k-r}(x),&\text{если  $\nu=r\le k$,}\\
q_{r-\nu,k-\nu}(x),&\text{если $\nu\le k< r$,}\\
0,&\text{если $k< \nu\le r-1$}.
  \end{cases}
\end{equation}
Через $L^p_\rho(a,b)$ обозначим пространство  функций $f(x)$, измеримых  на  $(a,b)$, для которых $\int_a^b|f(x)|^p\rho(x)dx<\infty$.
Если $\rho(x)\equiv1$, то будем писать $L^p_\rho(a,b)=L^p(a,b)$ и $L(a,b)=L^1(a,b)$.
Пусть $W^r_{L^p_\rho(a,b)}$ -- пространство Соболева $W^r_{L^p_\rho(a,b)}$, состоящее из функций $f(x)$, непрерывно дифференцируемых на $[a,b]$ $r-1$ раз, причем $f^{(r-1)}(x)$ абсолютно непрерывна на $[a,b]$  и $f^{(r)}(x)\in L^p_\rho(a,b)$.
Скалярное произведение в пространстве $W^r_{L^2_\rho(a,b)}$ определим с помощью равенства
\begin{equation}\label{3.8}
<f,g>=\sum_{\nu=0}^{r-1}f^{(\nu)}(a)g^{(\nu)}(a)+\int_{a}^{b} f^{(r)}(t)g^{(r)}(t)\rho(t) dt.
\end{equation}
Тогда, пользуясь определением полиномов $q_{r,k}(x)$ (см. \eqref{3.5} и \eqref{3.6}) и равенством  \eqref{3.7}, нетрудно увидеть (см. предложение 1),  что система $\{q_{r,k}(x)\}_{k=0}^\infty$ является ортонормированной в пространстве $W^r_{L^2_\rho(a,b)}$.  Мы будем называть систему $\{q_{r,k}(x)\}_{k=0}^\infty$ \textit{ ортонормированной по Соболеву } относительно скалярного произведения \eqref{3.8} и  \textit{ порожденной} ортонормированной системой $\{q_{k}(x)\}_{k=0}^\infty$.
Нетрудно увидеть,  что ряд Фурье функции $f(x)\in W^r_{L^2_\rho(a,b)}$ по системе  $\{q_{r,k}(x)\}_{k=0}^\infty$ имеет смешанный характер, а, более точно, имеет (см.доказательство предложения 1) следующий вид
  \begin{equation}\label{3.9}
f(x)\sim \sum_{k=0}^{r-1} f^{(k)}(a)\frac{(x-a)^k}{k!}+ \sum_{k=r}^\infty \hat f_{r,k}q_{r,k}(x),
\end{equation}
где
  \begin{equation}\label{3.10}
 \hat f_{r,k}=\int\limits_a^b f^{(r)}(t) q^{(r)}_{r,k}(t)\rho(t)dt=\int\limits_a^b f^{(r)}(t) q_{k-r}(t)\rho(t)dt.
\end{equation}
Ряд  вида \eqref{3.9} будем (следуя  \cite{Shar11} -- \cite{SHII})   называть \textit{ смешанным рядом} по  системе $\{q_{k}(x)\}_{k=0}^\infty$, считая это название условным и сокращенным обозначением полного названия: <<\textit{ ряд Фурье по системе  $\{q_{r,k}(x)\}_{k=0}^\infty$, ортонормированной по Соболеву, порожденной ортонормированной системой $\{q_{k}(x)\}_{k=0}^\infty$}>>.
Важное значение имеет свойство  смешанного ряда \eqref{3.9}, которое заключается в том, что его частичная сумма вида
\begin{equation}\label{3.11}
Y_{r,N}(f,x)=\sum_{k=0}^{r-1} f^{(k)}(a)\frac{(x-a)^k}{k!}+ \sum_{k=r}^{N} \hat f_{r,k}q_{r,k}(x)
\end{equation}
 при   $r\le N$  совпадает с исходной функцией $f(x)$   в точке $x=a$ $r$-кратно , т.е.
\begin{equation}\label{3.12}
(Y_{r,N}(f,x))^{(\nu)}_{x=a}=f^{(\nu)}(a)\quad (0\le\nu\le r-1).
\end{equation}
Кроме того, из \eqref{3.7} и \eqref{3.11} следует, что $(0\le\nu\le r-1)$
\begin{equation}\label{3.13}
 Y_{r,N}^{(\nu)}(f,x)=Y_{r-\nu,N-\nu}(f^{(\nu)},x),
 \end{equation}
отсюда, в свою очередь, выводим $(0\le\nu\le r-2)$
 $$
f^{(\nu)}(x)-Y_{r,N}^{(\nu)}(f,x)=
$$
  \begin{equation}\label{3.14}
\frac{1}{(r-\nu-2)!}\int_a^x (x-t)^{r-\nu-2}(f^{(r-1)}(t)-Y_{1,N-r+1}(f^{(r-1)},t))dt.
 \end{equation}

В  \cite{Shar11} -- \cite{SHII}, воспользовавшись равенствами типа \eqref{3.13} и \eqref{3.14}, было показано, что частичные суммы смешанных рядов по классическим ортогональным полиномам, в отличие от сумм Фурье по этим же полиномам, успешно могут быть использованы в задачах, в которых требуется одновременно приближать дифференцируемую функцию и ее несколько производных. Как было показано выше, такие задачи непосредственно возникают, например, в связи с решением краевых задач для дифференциальных уравнений спектральными методами.
В связи с отмеченными выше и рядом других важных задач, в которых полиномы, ортогональные по Соболеву, предстают перед нами как естественный и эффективный инструмент для их решения, возникают важные вопросы об изучении различных  свойств самих этих полиномов. Наиболее трудными из них, как это и бывает в теории ортогональных полиномов, являются вопросы, связанные с их асимптотическим поведением.   В связи с этой проблемой  отметим  работу
\cite{Lopez1995}, в которой, используя  идеи и технику А.\,А. Гончара \cite{Gonchar1975}, исследована задача о сравнительной асимптотике полиномов, ортогональных относительно скалярного произведения типа  Соболева с дискретными массами.
В настоящей  статье получены (теоремы 1 и 2) различные выражения для полиномов $p_{r,k}^{\alpha,\beta}(x)$, порожденных ортонормированными полиномами Якоби $p_{k}^{\alpha,\beta}(x)$ (см.\eqref{2.13}) посредством  равенства \eqref{3.27}. Эти результаты, в сочетании с  хорошо известными асимптотическими формулами для полиномов Якоби, могут быть использованы  при изучении асимптотических свойств полиномов $p_{r,n}^{\alpha,\beta}(x)$ при $x\in[-1,1]$ и $n\to\infty$ и аппроксимативных свойств рядов Фурье по ним. Например,  благодаря формуле \eqref{2.6}  и теореме 2, задача об асимптотических свойствах полиномов $p_{r,k}^{\alpha,\beta}(x)$ с целыми $\alpha,\beta>-1$ и их нулей непосредственно сводится к аналогичным задачам соответствующих полиномов Якоби $P_n^{|r-\alpha|,|r-\beta|}(x)$.  Однако в настоящей работе мы не будем  останавливаться на задаче о получении с помощью теорем 1 и 2  асимптотических свойств полиномов $p_{r,n}^{\alpha,\beta}(x)$ с произвольными $\alpha,\beta>-1$, поскольку, как уже отмечалось, она является объектом исследования другой нашей работы. Следует также отметить, что в упомянутых выше работах \cite{Shar11} -- \cite{SHII} задача об асимптотических свойствах полиномов $p_{r,k}^{\alpha,\beta}(x)$ с произвольными $\alpha,\beta>-1$ также не рассматривалась. В то же время отметим, что
равенство \eqref{3.44}, которое является следствием теоремы 2, соответствующим случаю $\alpha=\beta=0$, существенно использовалось в \cite{Shar15} -- \cite{Shar17} в связи с исследованием задачи об аппроксимативных свойствах смешанных рядов по полиномам Лежандра и
оно же, как уже отмечалось,  является отправной точкой при конструировании в \S 4 настоящей статьи специального ряда  по ультрасферическим полиномам Якоби с  прилипающими в точках $\pm1$ частичными суммами $\sigma_{r,n}^\alpha(f,x)$.

Прежде чем  перейти к исследованию свойств полиномов $p_{r,k}^{\alpha,\beta}(x)$, мы рассмотрим некоторые общие свойства  полиномов $q_{r,k}(x)$, $k=0,1,\ldots$, определенных равенствами \eqref{3.5} и \eqref{3.6},  ортонормированных по Соболеву относительно скалярного произведения \eqref{3.8}. Рассмотрим сначала задачу о полноте в $W^r_{L^2_\rho(a,b)}$ системы $\{q_{r,k}(x)\}_{k=0}^\infty$.

\begin{state} 
Предположим, что полиномы $q_k(x)$ $(k=0,1,\ldots)$ образуют полную в $L^2_\rho(a,b)$ ортонормированную   c весом   $\rho(x)$ систему на  $(a,b)$. Тогда система $\{q_{r,k}(x)\}_{k=0}^\infty$, порожденная системой $\{q_{k}(x)\}_{k=0}^\infty$ посредством равенств \eqref{3.5} и \eqref{3.6}, полна  в $W^r_{L^2_\rho(a,b)}$ и ортонормирована относительно скалярного произведения \eqref{3.8}.
\end{state}

%\begin{proof}
%Из равенства \eqref{3.7} следует, что если $r\le k$ и $0\le\nu\le r-1$, то  $(q_{r,k}(x))^{(\nu)}_{x=a}=0$, поэтому
%в силу \eqref{3.8} и второго из равенств \eqref{3.7}  имеем
%$$
%<q_{r,k},q_{r,l}>= \int_{a}^b(q_{r,k}(x))^{(r)}(q_{r,l}(x))^{(r)}\rho(x) dx=
%$$
%\begin{equation}\label{3.15}
%    \int_{a}^bq_{k-r}(x)q_{l-r}(x)\rho(x) dx=\delta_{kl},\quad k,l\ge r ,
%  \end{equation}
%  а из \eqref{3.6}  имеем
%\begin{equation}\label{3.16}
%  <q_{r,k},q_{r,l}>=\sum_{\nu=0}^{r-1}(q_{r,k}(x))^{(\nu)}|_{x=a}(q_{r,l}(x))^{(\nu)}|_{x=a}=\delta_{kl},\quad k,l< r.
%  \end{equation}
%  Очевидно также, что
%  \begin{equation}\label{3.17}
%  <q_{r,k},q_{r,l}>=0,\quad \text{если}\quad k< r\le l\quad \text{или} \quad l< r\le k.
%  \end{equation}
% Это означает, что полиномы $q_{r,k}(t)\, (k=0,1,\ldots) $ образуют   в $W^r_{L^2_\rho(a,b)}$ ортонормированную  систему относительно скалярного произведения \eqref{3.8}.  Остается убедиться в ее полноте в $W^r_{L^2_\rho(a,b)}$. С этой целью покажем, что если для некоторой функции $f=f(x)\in W^r_{L^2_\rho(a,b)}$ и для  всех $k=0,1,\ldots$ справедливы равенства $<f,q_{r,k}>=0$, то $f(x)\equiv0$. В самом деле, если $k\le r-1$, то  $<f,q_{r,k}>=f^{(k)}(a)$, поэтому с учетом того, что $<f,q_{r,k}>=0$,  для нашей функции  $f(x)$ формула Тейлора имеет вид
%\begin{equation}\label{3.18}
%f(x)={1\over (r-1)!}\int\limits_{a}^x(x-t)^{r-1} f^{(r)}(t)dt.
%     \end{equation}
%С другой стороны, для всех $k\ge r$ имеем
%$$
% 0= <f,q_{r,k}>=\int_{a}^bf^{(r)}(x) (q_{r,k}(x))^{(r)}\rho(x) dx=
%  \int_{a}^b f^{(r)}(x)q_{k-r}(x) \rho(x) dx .
%$$
%Отсюда и из того, что $q_m(x)$ ($m=0,1,\ldots$)  образуют в $L^2_{\rho}(a,b)$ полную ортонормированную систему, имеем $f^{(r)}(x)=0$ почти всюду на $(a,b)$. Поэтому в силу \eqref{3.18}  $f(x)\equiv0$.  Предложение 1 доказано.
%\end{proof}


\begin{state} 
Предположим, что  $ \frac{1}{\rho(x)}\in L(a,b) $, а  полиномы $q_k(x)$ $(k=0,1,\ldots)$  образуют полную в $L^2_\rho(a,b)$ ортонормированную   c весом   $\rho(x)$ систему на $(a,b)$, $\{q_{r,k}(x)\}_{k=0}^\infty$ -- система, ортонормированная в $W^r_{L^2_\rho(a,b)}$ относительно скалярного произведения \eqref{3.8}, порожденная системой $\{q_{k}(x)\}_{k=0}^\infty$ посредством равенств \eqref{3.5} и \eqref{3.6}.
Тогда, если $f(x)\in W^r_{L^2_\rho(a,b)}$, то ряд Фурье (смешанный ряд) \eqref{3.9} сходится к функции $f(x)$ равномерно относительно $x\in[a,b]$.
\end{state} 

%\begin{proof}
%Обозначим через $S_n(f^{(r)})=S_n(f^{(r)},x)$ частичную сумму ряда Фурье функции $f^{(r)}(x)\in L^2_\rho(a,b) $ по системе $\{q_k(x)\}_{k=0}^n$, т.е.
%\begin{equation}\label{3.19}
%S_n(f^{(r)},x)=\sum_{k=0}^n \hat f_{r,k+r}q_k(x),
%     \end{equation}
%где коэффициенты $\hat f_{r,k+r}$ $(k=0,1,\ldots)$ определены равенством \eqref{3.10}.  Из условий теоремы 2 следует, что если $n\to\infty$, то
% \begin{equation}\label{3.20}
%\|f^{(r)}-S_n(f^{(r)})\|_{L^2_\rho(a,b)}\to0.
% \end{equation}
%Запишем формулу Тейлора
%\begin{equation}\label{3.21}
% f(x)= \sum_{k=0}^{r-1} f^{(k)}(a)\frac{(x-a)^k}{k!}+ {1\over (r-1)!}\int\limits_{a}^x(x-t)^{r-1} f^{(r)}(t)dt
%\end{equation}
%и заметим, что из  \eqref{3.11} и \eqref{3.21}  имеем
%\begin{equation}\label{3.22}
% f(x)-Y_{r,n+r}(f,x)=  {1\over (r-1)!}\int\limits_{a}^x(x-t)^{r-1} f^{(r)}(t)dt-\sum_{k=r}^{n+r} \hat f_{r,k}q_{r,k}(x).
%\end{equation}
%Обратимся к равенству \eqref{3.5}, тогда \eqref{3.22} можно переписать так
%$$
%f(x)-Y_{r,n+r}(f,x)=
%$$
%$$
%  {1\over (r-1)!}\int\limits_{a}^x(x-t)^{r-1} f^{(r)}(t)dt-{1\over (r-1)!}\int\limits_{a}^x(x-t)^{r-1}\sum_{k=r}^{n+r} \hat f_{r,k}q_{k-r}(t)dt=
%$$
%\begin{equation}\label{3.23}
% {1\over (r-1)!}\int\limits_{a}^x(x-t)^{r-1}[f^{(r)}(t)-S_n(f^{(r)},t)]dt .
%\end{equation}
%Из \eqref{3.23} и неравенства Гельдера имеем
%$$
%|f(x)-Y_{r,n+r}(f,x)|\le
%$$
%\begin{equation}\label{3.24}
% {1\over (r-1)!} \left(\int\limits_{a}^b\frac{|x-t|^{2(r-1)}}{\rho(t)}dt\right)^\frac12\left(\int\limits_{a}^b [f^{(r)}(t)-S_n(f^{(r)},t)]^2\rho(t)dt\right)^\frac12.
% \end{equation}
%Сопоставляя \eqref{3.24} с \eqref{3.20}, убеждаемся в справедливости предложения 2.
%\end{proof}


\subsection{Полиномы, порожденные  ортонормированными полиномами Якоби $p_{k}^{\alpha,\beta}(x)$}


Из \eqref{2.2} и \eqref{2.13} следует, что если $\alpha,\beta>-1$, то полиномы $p_n^{\alpha,\beta}(x)=P_n^{\alpha,\beta}(x)/\sqrt{ h_n^{\alpha,\beta}}\quad(n=0,1,\ldots)$
образуют ортонормированную  в $L_\kappa^2(-1,1)$ с весом $\kappa(x)=(1-x)^\alpha(1+x)^\beta$ систему. В частности, если $\alpha=\beta=\frac12$, $x=\cos\theta$, то
$$
 p_n^{\frac12,\frac12}(x)=\sqrt{2/\pi}\frac{\sin(n+1)\theta}{\sin\theta}\quad (n=0,1,\ldots)
 $$
 -- ортонормированные полиномы Чебышева второго рода, а если $\alpha=\beta=-\frac12$, то
\begin{equation}\label{3.25}
p_0^{-\frac12,-\frac12}(x)=\sqrt{1/\pi},\quad  p_n^{-\frac12,-\frac12}(x)=\sqrt{2/\pi}\cos n\theta\quad (n=1,\ldots)
\end{equation}
 -- ортонормированные полиномы Чебышева первого рода.
 Как хорошо известно \cite{Sege}, система полиномов Якоби \eqref{2.13} полна в $L_\kappa^2(-1,1)$.   Она порождает на $[-1,1]$ систему полиномов  $p_{r,k}^{\alpha,\beta}(x)$ $(k=0,1,\ldots)$, определенных равенствами
   \begin{equation}\label{3.26}
p_{r,k}^{\alpha,\beta}(x) =\frac{(x+1)^k}{k!}, \quad k=0,1,\ldots, r-1,
\end{equation}
  \begin{equation}\label{3.27}
p_{r,r+n}^{\alpha,\beta}(x) =\frac{1}{(r-1)!}\int\limits_{-1}^x(x-t)^{r-1}p_n^{\alpha,\beta}(t)dt, \quad n=0,1,\ldots.
\end{equation}
Если мы обратимся к скалярному произведению \eqref{1.1}, то из предложения 1 непосредственно вытекает
\begin{corollary}
Пусть $\alpha,\beta>-1$. Тогда система полиномов $\{p_{r,k}^{\alpha,\beta}(x)\}_{k=0}^\infty$, порожденная системой ортонормированных полиномов Якоби \eqref{2.13} посредством равенств \eqref{3.5} и \eqref{3.6}, полна  в $W^r_{L^2_\kappa(-1,1)}$ и ортонормирована относительно скалярного произведения \eqref{1.1}.
\end{corollary}
Ряд Фурье \eqref{3.9} для системы   $\{p_{r,k}^{\alpha,\beta}(x)\}_{k=0}^\infty$ приобретает вид
\begin{equation}\label{3.28}
f(x)\sim \sum_{k=0}^{r-1} f^{(k)}(-1)\frac{(x+1)^k}{k!}+ \sum_{k=r}^\infty \hat f_{r,k}p_{r,k}^{\alpha,\beta}(x),
\end{equation}
где
  \begin{equation}\label{3.29}
 \hat f_{r,k}=\int_{-1}^1 f^{(r)}(t)p_{k-r}^{\alpha,\beta}(t)\kappa(t)dt.
\end{equation}

\begin{corollary}
Пусть $-1<\alpha,\beta<1$. Тогда если $f(x)\in W^r_{L^2_\kappa(-1,1)}$, то ряд Фурье (смешанный ряд) \eqref{3.28} сходится к функции $f(x)$ равномерно относительно $x\in[-1,1]$.
\end{corollary}
\begin{proof}
Заметим, что если $-1<\alpha,\beta<1$, то $\frac{1}{\kappa(x)}\in L(-1,1)$, где $\kappa(x)=(1-x)^\alpha(1+x)^\beta$. Поэтому утверждение следствия 2 вытекает из предложения 2 и следствия 1.
\end{proof}

При исследовании дальнейших (более глубоких) аппроксимативных свойств частичных сумм смешанного ряда \eqref{3.28} возникает задача об асимптотических свойствах полиномов $p_{r,k}^{\alpha,\beta}(x)$, которая, в свою очередь, приводит к вопросу о получении различных представлений для этих полиномов, отличных от  \eqref{3.27} и не содержащих знаков интеграла с переменным пределом. Прежде всего мы найдем явный вид полиномов $p_{r,k}^{\alpha,\beta}(x)$.

\begin{theorem} Для произвольных $\alpha, \beta>-1$ и $n\ge0$
имеет место следующее равенство
\begin{equation}\label{3.30}
p_{r,n+r}^{\alpha,\beta}(x)=\frac{(-1)^n2^{r}}{\sqrt{ h_n^{\alpha,\beta}}}
{n+\beta\choose n}
\sum_{k=0}^n{(-n)_k(n+\alpha+\beta+1)_k\over (\beta+1)_k(k+r)!}
\left({1+x\over 2}\right)^{k+r}.
\end{equation}
\end{theorem}
%\begin{proof}
%Воспользуемся равенством \eqref{2.7} и запишем
%$$
%P_{n}^{\alpha,\beta}(x)=(-1)^nP_{n}^{\beta,\alpha}(-x)=(-1)^n{n+\beta\choose n}
%\sum_{k=0}^n{(-n)_k(n+\alpha+\beta+1)_k\over k!(\beta+1)_k}
%\left({1+x\over 2}\right)^k,
%$$
%поэтому, в силу \eqref{2.13} имеем
%\begin{equation}\label{3.31}
%p_{n}^{\alpha,\beta}(x)=\frac{(-1)^n}{\sqrt{ h_n^{\alpha,\beta}}}{n+\beta\choose n}
%\sum_{k=0}^n{(-n)_k(n+\alpha+\beta+1)_k\over k!(\beta+1)_k}
%\left({1+x\over 2}\right)^k.
%\end{equation}
%С другой стороны в силу формулы Тейлора
%\begin{equation}\label{3.32}
%\left({1+x\over 2}\right)^{k+r}={(k+r)^{[r]}\over 2^r(r-1)!}\int\limits_{-1}^x(x-t)^{r-1} \left({1+t\over 2}\right)^{k}dt,
%\end{equation}
%где $a^{[0]}=1$, $a^{[r]}=a(a-1)\dots(a-r+1)$.
%Сопоставляя \eqref{3.31} и \eqref{3.32} с \eqref{3.27}, убеждаемся в справедливости утверждения теоремы 1.
%\end{proof}

Равенство \eqref{3.30}, установленное в теореме 1, может быть использовано при исследовании асимптотических свойств полиномов $p_{r,n+r}^{\alpha,\beta}(x)$ в окрестности точки $x=-1$. Если же $-1+\varepsilon\le x$, то формула  \eqref{3.30} становится непригодной для изучения асимптотического  поведения полиномов  $p_{r,n+r}^{\alpha,\beta}(x)$ при $n\to\infty$, поэтому возникает задача найти иные представления для этих полиномов, которые могли бы быть использованы для исследования их поведения при $n\to\infty$ в том случае, когда точка $x$ не находится в непосредственной близости от $-1$. Мы перейдем теперь к рассмотрению этого вопроса. Пусть $\lambda=\alpha+\beta$. Тогда если $(k+\lambda)^{[r]}\neq0$,
то мы можем воспользоваться равенством \eqref{2.5} и записать
\begin{equation}\label{3.33}
P^{\alpha,\beta}_k(t)={2^r \over (k+\lambda )^{[r]}}\frac{d^r}{dt^r}P_{k+r}^{\alpha-r,\beta-r}(t).
\end{equation}
Заметим, что если $-1<\alpha,\beta<1$ и $\lambda\notin\{-1,0,1\}$, то
 равенство \eqref{3.33} справедливо при всех
$k=0,1,\ldots$. Если  $k\ge r-\lambda$, то, очевидно,
$(k+\lambda)^{[r]}\neq0$ и для таких $k$ мы
можем снова воспользоваться равенством \eqref{3.33}. Наконец, если одно из чисел $\alpha$ или $\beta$ целое, а другое дробно, то
$(k+\lambda)^{[r]}\neq0$ для всех $k=0,1,\ldots$ и опять верна формула \eqref{3.33}. Итак, пусть $(k+\lambda)^{[r]}\neq0$, тогда в силу  \eqref{3.33}
$$
\frac{1}{(r-1)!}\int\limits^x_{-1}(x-t)^{r-1}P_k^{\alpha,\beta}(t)\,dt=
$$
$$
\frac{2^r}{(k+\lambda)^{[r]}}\frac{1}{(r-1)!}\int\limits^x_{-1}(x-t)^{r-1}
\frac{d^r}{dt^r}P_{k+r}^{\alpha-r,\beta-r}(t)\,dt=
$$
\begin{equation}\label{3.34}
\frac{2^r}{(k+\lambda)^{[r]}}\left[P_{k+r}^{\alpha-r,\beta-r}(x)-\sum^{r-1}_{\nu=0}
\frac{(1+x)^\nu}{\nu!}\left\{P_{k+r}^{\alpha-r,\beta-r}(t)
\right\}_{t=-1}^{(\nu)}\right].
\end{equation}
 Далее, в силу \eqref{2.5}
 \begin{equation}\label{3.35}
\left\{P_{k+r}^{\alpha-r,\beta-r}(t)\right\}^{(\nu)}=
\frac{(k+\lambda-r+1)_\nu}{2^\nu}P_{k+r-\nu}^{\alpha+\nu-r,\beta+\nu-r}(t),
\end{equation}
а из \eqref{2.9} имеем
$$P_{k+r-\nu}^{\alpha+\nu-r,\beta+\nu-r}(-1)=(-1)^{k+r-\nu}{k+\beta\choose k+r-\nu}=$$
\begin{equation}\label{3.36}
\frac{(-1)^{k+r-\nu}\Gamma(k+\beta+1)}{\Gamma(\nu-r+\beta+1)(k+r-\nu)!}.
\end{equation}
Из \eqref{3.35}  и \eqref{3.36} находим
\begin{equation}\label{3.37}
\left\{P_{k+r}^{\alpha-r,\beta-r}(t)\right\}_{t=-1}^{(\nu)}=
\frac{(-1)^{k+r-\nu}\Gamma(k+\beta+1)(k+\lambda-r+1)_{\nu}}
{\Gamma(\nu-r+\beta+1)(k+r-\nu)!2^\nu}
=A_{\nu,k,r}^{\alpha,\beta}.
\end{equation}
Сопоставляя \eqref{3.34} и \eqref{3.37} мы можем записать
$$\frac{1}{(r-1)!}\int\limits^x_{-1}(x-t)^{r-1}P_k^{\alpha,\beta}(t)\,dt=$$
\begin{equation}\label{3.38}
\frac{2^r}{(k+\lambda)^{[r]}}\left[P_{k+r}^{\alpha-r,\beta-r}(x)
-\sum^{r-1}_{\nu=0}\frac{A_{\nu,k,r}^{\alpha,\beta}}{\nu!}(1+x)^{\nu}\right].
\end{equation}
Из \eqref{3.27}, \eqref{2.13} и \eqref{3.34} мы выводим следующий результат.

\begin{theorem} Пусть $\alpha, \beta>-1$, $\lambda=\alpha+\beta$. Тогда  при условии $(k+\lambda)^{[r]}\neq0$ имеет место следующее равенство
\begin{equation}\label{3.39}
p_{r,r+k}^{\alpha,\beta}(x) ={1\over\sqrt{ h_k^{\alpha,\beta}}}
\frac{2^r}{(k+\lambda)^{[r]}}\left[P_{k+r}^{\alpha-r,\beta-r}(x)
-\sum^{r-1}_{\nu=0}\frac{A_{\nu,k,r}^{\alpha,\beta}}{\nu!}(1+x)^{\nu}\right] ,
\end{equation}
в котором числа $A_{\nu,k,r}^{\alpha,\beta}$ определены равенством \eqref{3.37}.
\end{theorem}

\begin{remark} 
  Выражения, аналогичные тем, которые фигурируют в правой части равенства  \eqref{3.39}, впервые появились в   работах \cite{Shar13}, \cite{Shar17}, \cite{Shar18} в связи исследованием задачи об аппроксимативных свойствах смешанных рядов \eqref{3.28} по общим полиномам Якоби $P_{k}^{\alpha,\beta}(x)$.
\end{remark} 
  
  
  Рассмотрим некоторые частные случаи.

\subsection{Полиномы, порожденные полиномами Якоби $p_{k}^{\alpha,0}(x)$}


 Пусть $\beta=0$, $\alpha$ -- дробное. Тогда, во-первых $(k+\lambda)^{[r]}\neq0$ для всех $k\ge0$, во-вторых,  из \eqref{3.37} следует, что $A_{\nu,k,r}^{\alpha,0}=0$ при всех  $\nu=0,1,\dots, r-1$, поэтому равенство \eqref{3.39} можно переписать так
 \begin{equation}\label{3.40}
p_{r,r+k}^{\alpha,0}(x) ={1\over\sqrt{ h_k^{\alpha,0}}}
\frac{2^r}{(k+\alpha)^{[r]}}P_{k+r}^{\alpha-r,-r}(x) \quad (k=0,1,\ldots).
\end{equation}
С учетом свойства \eqref{2.6} этому равенству можно придать также следующий вид
\begin{equation}\label{3.41}
p_{r,r+k}^{\alpha,0}(x) =
\frac{(1+x)^rP_{k}^{\alpha-r,r}(x)}{(k+r)^{[r]}\sqrt{ h_k^{\alpha,0}}},
 \quad (k=0,1,\ldots).
\end{equation}
\subsection{ Полиномы, порожденные полиномами Лежандра $p_{n}^{0,0}(x)$}

 Рассмотрим ортогональные по Соболеву полиномы $p_{r,n}(x)=p_{r,n}^{0,0}(x)$, порожденные полиномами \textit{ Лежандра}. С этой целью положим $\beta=\alpha=0$. Тогда  $(k+\lambda)^{[r]}\neq0$ для всех $k\ge r$, а из \eqref{3.37} следует, что $A_{\nu,k,r}^{0,0}=0$ при всех  $\nu=0,1,\dots, r-1$. Поэтому из теоремы 2 имеем
\begin{equation}\label{3.42}
p_{r,r+k}(x) =\sqrt{k+1/2}
\frac{2^r}{k^{[r]}}P_{k+r}^{-r,-r}(x) \quad (k=r,r+1,\ldots).
\end{equation}
Если мы обратимся к равенству \eqref{2.6}, то можем записать
\begin{equation}\label{3.43}
P_{k+r}^{-r,-r}(x)= \frac{(-1)^r(1-x^2)^r}{2^{2r}}P_{k-r}^{r,r}(x) \quad (k=r,r+1,\ldots).
\end{equation}
Из \eqref{2.13}, \eqref{3.42} и \eqref{3.43} имеем
\begin{equation}\label{3.44}
p_{r,r+k}(x) =
\frac{(-1)^r}{2^rk^{[r]}}\sqrt{(k+1/2)h_{k-r}^{r,r}}(1-x^2)^rp_{k-r}^{r,r}(x) \quad (k=r,r+1,\ldots).
\end{equation}
Соответствующий этому случаю ряд Фурье \eqref{3.28} по полиномам $p_{r,k}(x)=p_{r,k}^{0,0}(x)$, ортогональным по Соболеву (или, что то же, \textit{ смешанный ряд по полиномам  Лежандра}), приобретает вид
\begin{equation}\label{3.45}
f(x)\sim \mathcal{ Y}_{r,2r-1}(f,x)+\left(\frac{x^2-1}2\right)^r\sum_{k=0}^\infty\hat f_{r,k+2r} \frac{\sqrt{(k+r+\frac12)h_k^{r,r}}}{ (k+r)^{[r]}}p_{k}^{r,r}(x),
\end{equation}
где
\begin{equation*}
 \mathcal{ Y}_{r,2r-1}(f,x)=\sum_{k=0}^{r-1} f^{(k)}(-1)\frac{(x+1)^k}{k!}+\sum_{k=r}^{2r-1} \hat f_{r,k}p_{r,k}(x),
\end{equation*}
а для полиномов $p_{r,k}(x)=p_{r,k}^{0,0}(x)$, фигурирующих в в правой части последнего равенства, в силу теоремы 1 имеет место представление
\begin{equation}\label{3.46}
p_{r,k}(x)=(-1)^{k-r}2^{r}\sqrt{k-r+\frac12}
\sum_{l=0}^{k-r}{(r-k)_l(k-r+1)_l\over l!(l+r)!}
\left({1+x\over 2}\right)^{l+r}.
\end{equation}
Аппроксимативные свойства частичных сумм ряда \eqref{3.45} вида
$$
\mathcal{ Y}_{r,N}(f,x)=\sum_{k=0}^{r-1} f^{(k)}(-1)\frac{(x+1)^k}{k!}+\sum_{k=r}^N \hat f_{r,k}p_{r,k}(x)=
$$
\begin{equation}\label{3.47}
\mathcal{ Y}_{r,2r-1}(f,x)+\left(\frac{x^2-1}2\right)^r\sum_{k=0}^{N-2r}\hat f_{r,k+2r} \frac{\sqrt{(k+r+\frac12)h_k^{r,r}}}{ (k+r)^{[r]}}p_{k}^{r,r}(x)
\end{equation}
 были весьма подробно исследованы в работах \cite{Shar11} -- \cite{sharap3}. Мы напомним здесь некоторые из них. Прежде всего отметим, что оператор  $f\to \mathcal{ Y}_{r,n}(f)$ представляет собой проектор на подпространство алгебраических полиномов $p_n$ степени не выше $n$, т.е. $\mathcal{ Y}_{r,n}(p_n)=p_n$. С другой стороны, если $f\in W^r_{L^2(-1,1)}$, то в силу предложения 2 (см. также следствие 2) имеет место равенство
$$
f(x)=\sum_{k=0}^{r-1} f^{(k)}(-1)\frac{(x+1)^k}{k!}+\sum_{k=r}^\infty \hat f_{r,k}p_{r,k}(x)
$$
\begin{equation}\label{3.48}
=\mathcal{ Y}_{r,2r-1}(f,x)+\left(\frac{x^2-1}2\right)^r\sum_{k=0}^\infty\hat f_{r,k+2r} \frac{\sqrt{(k+r+\frac12)h_k^{r,r}}}{ (k+r)^{[r]}}p_{k}^{r,r}(x),
\end{equation}
причем ряд, фигурирующий в правой части равенства \eqref{3.48}, сходится равномерно на $[-1,1]$. Отсюда, в свою очередь, следует, что   $\mathcal{ Y}_{r,n}(f,x)$ при $n\ge2r-1 $ совпадает с функцией $f(x)\in W^r_{L^2_\kappa(-1,1)}$ $r$-кратно в точках $-1$ и $1$, т.е. $f^{(\nu)}(\pm1)=\mathcal{ Y}_{r,n}^{(\nu)}(f,\pm1),\quad \nu=0,1,\ldots, r-1$. Стало быть,
 $D_{2r-1}(f,x)=\mathcal{ Y}_{r,2r-1}(f,x)$
представляет собой \cite{Shar17} интерполяционный полином Эрмита степени $2r-1$.
В работе \cite{Shar15}  была доказана следующая неулучшаемая по порядку (при $N\to\infty$) оценка
\begin{equation}\label{3.49}
\sup_{f\in W^r}\max_{-1\le x\le 1}{\left|f^{(\nu)}(x)-\left(\mathcal{ Y}_{r,N}(f,x)\right)^{(\nu)}\right|\over(1-x^2)^{(r-\nu)/2-1/4}}
\le c(r)\frac{\ln N}{N^{r-\nu}},\,\, 0\le \nu\le r-1,
\end{equation}
 где  $W^r$ -- класс функций, непрерывно дифференцируемых $r$-раз, для которых $\max_{-1\le x\le 1}|f^{(r)}(x)|\le1$.
Доказательство оценки \eqref{3.49}  основано \cite{Shar15} на неравенстве типа Лебега для $(\mathcal{ Y}_{r,n+2r}(f,x))^{(\nu)}$, которое имеют следующий вид ($0\le\nu\le r-1$)
$$
{\left|f^{(\nu)}(x)-\left(\mathcal{ Y}_{r,n+2r}(f,x)\right)^{(\nu)}\right|
\over(1-x^2)^{\frac{r-\nu}{2}-\frac14}}\le
$$
  \begin{equation}\label{3.50}
    ((1-x^2)^\frac14+(1-x^2)^{\frac{r-\nu}{2}+\frac14}I^{r-\nu}_{n+\nu}(x))
E^{r-\nu}_{n+2r-\nu}(f^{(\nu)}),
\end{equation}
где
\begin{equation}\label{3.51}
I^{d}_{l}(x)= \int_{-1}^{1}|K^{d,d}_{l}(x,t)|(1-t^2)^{\frac{d}{2}}dt,
\end{equation}
а величина
\begin{equation}\label{3.52}
E_s^d(f)=\inf_{p_s}\sup_{-1<x<1}{|f(x)-p_s(x)|\over (1-x^2)^\frac{d}{2}}
\end{equation}
представляет собой наилучшее (весовое) приближение функции $f\in W^d$ алгебраическими полиномами $p_s(x)$ степени
$s$, обладающими свойством $p_s^{(\nu)}(\pm1)=f^{(\nu)}(\pm1)$ $(\nu=0,\ldots, d-1)$. Для величины, определенной равенством  \eqref{3.51}, в \cite{Shar15} получена следующая оценка
\begin{equation}\label{3.53}
I^d_n(x)\le c(d)(1-x^2)^{-\frac{d}{2}}\left[\ln(n\sqrt{1-x^2}+1)+(1-x^2)^{-\frac14}\right]\quad(-1<x<1),
\end{equation}
а из известной теоремы Теляковского -- Гопенгауза \cite{TEL},\cite{GOP} следует, что
\begin{equation}\label{3.54}
E_N^d(f)\le c(d)N^{-d}\omega(f^{(d)},\frac{1}{N})\quad (f\in W^d),
\end{equation}
где $\omega(g,\delta)$ -- модуль непрерывности функции $g\in C[-1,1]$. Отдельно отметим частный случай неравенства \eqref{3.49}, соответствующий выбору $\nu=0$. В этом случае,  учитывая оценку \eqref{3.53} и полагая $N=n+2r$, мы получаем следующее неравенство типа Лебега
\begin{equation}\label{3.55}
{\left|f(x)-\mathcal{ Y}_{r,N}(f,x)\right|
\over(1-x^2)^{\frac{r}{2}-\frac14}}\le c(r)\left[(1-x^2)^{\frac14}\ln(N\sqrt{1-x^2}+1)+1\right]E^{r}_{N}(f).
  \end{equation}
\section{Специальные <<прилипающие>> ряды по ультрасферическим полиномам Якоби}
Пусть $\alpha>-1$, $\kappa=\kappa(x)=(1-x^2)^\alpha$, $p\ge1$, $f\in W^r_{L^p_\kappa(-1,1)}$. Тогда $f^{(r)}\in W^r_{L(-1,1)}$ (т.к. $f^{(r-1)}(x)$ абсолютно непрерывна на $[-1,1]$) и можем ввести в рассмотрение интерполяционный полином Эрмита $D_{2r-1}(x)=D_{2r-1}(f,x)$ степени $2r-1$,  совпадающий с функцией $f(x)$ $r$-кратно в точках $-1$ и $1$, т.е. $f^{(\nu)}(\pm1)=D_{2r-1}^{(\nu)}(f,\pm1),\quad \nu=0,1,\ldots, r-1$. Как уже отмечалось выше  $D_{2r-1}(x)$ допускает представление $D_{2r-1}(x)=\mathcal{ Y}_{r,2r-1}(f,x)$, поэтому равенство \eqref{3.48} мы можем переписать так
\begin{equation*}
f(x)=D_{2r-1}(x)+
(1-x^2)^r\sum_{k=0}^\infty\frac{(-1)^r\hat f_{r,k+2r}}{2^r} \frac{\sqrt{(k+r+\frac12)h_k^{r,r}}}{ (k+r)^{[r]}}p_{k}^{r,r}(x)
 \end{equation*}
 и отсюда имеем $(-1<x<1)$
\begin{equation}\label{4.1}
F_r(x)={f(x)-D_{2r-1}(x)\over(1-x^2)^r}=
 \sum_{k=0}^\infty\frac{(-1)^r\hat f_{r,k+2r}}{2^r} \frac{\sqrt{(k+r+\frac12)h_k^{r,r}}}{ (k+r)^{[r]}}p_{k}^{r,r}(x).
\end{equation}
Правая часть этого равенства представляет собой ряд Фуре-Якоби функции $F_r(x)$ по ортонормированной системе полиномов Якоби $p_k^{r,r}(x)$ (см. \eqref{3.28}). Вместо \eqref{4.1} мы можем рассмотреть более общий ряд
\begin{equation}\label{4.2}
F_r(x)={f(x)-D_{2r-1}(x)\over(1-x^2)^r}=
 \sum_{k=0}^\infty \hat F_{r,k}^\alpha p_{k}^{\alpha,\alpha}(x),
\end{equation}
по ортонормированной системе полиномов Якоби $p_{k}^{\alpha,\alpha}(x)$, где
\begin{equation}\label{4.3}
\hat F^\alpha_{r,k}=\int_{-1}^1F_r(t)\kappa(t) p_{k}^{\alpha,\alpha}(t)=\int_{-1}^1(f(t)-D_{2r-1}(t))(1-t^2)^{\alpha-r} p_{k}^{\alpha,\alpha}(t)
\end{equation}
-- $k$-тый коэффициент Фурье-Якоби функции $F_r(x)$. Равенство \eqref{4.2} перепишем следующим образом
\begin{equation}\label{4.4}
f(x)=D_{2r-1}(f,x)+(1-x^2)^r \sum_{k=0}^\infty \hat F^\alpha_{r,k}p_{k}^{\alpha,\alpha}(x).
\end{equation}
Частичная сумма полученного разложения вида
\begin{equation}\label{4.5}
 \sigma_{r,N}^\alpha(f,x)=
 \begin{cases}
  D_{2r-1}(f,x),&\text{$N=2r-1$,}\\
 D_{2r-1}(f,x)+(1-x^2)^r \sum_{k=0}^{N-2r} \hat F^\alpha_{r,k}p_{k}^{\alpha,\alpha}(x),&\text{$N\ge 2r$.}
 \end{cases}
\end{equation}
обладает свойством $f^{(\nu)}(\pm1)=\sigma_{r,N}^\alpha(f,\pm1))^{(\nu)}$, $\nu=0,1,\ldots, r-1$, другими словами, $\sigma_{r,N}^\alpha(f,x)$ <<\textit{прилипает}>> к $f(x)$ в точках $-1$ и $1$. Поэтому правую часть  равенства \eqref{4.4} мы будем называть специальным рядом по полиномам Якоби $p_{k}^{\alpha,\alpha}(x)$, обладающим свойством <<\textit{прилипания}>> частичных сумм или просто специальным прилипающим рядом по этим  полиномам.  Отметим, что при $\alpha=r$ ряд \eqref{4.4} совпадает в силу \eqref{4.1} с рядом \eqref{3.48}, т.е. рядом Фурье функции $f$ по ортогональным по Соболеву полиномам $p_{r,k}(x)$, порожденным полиномами Лежандра и, соответственно, $\sigma_{r,N}^r(f,x)=\mathcal{ Y}_{r,N}(f,x)$. Нетрудно увидеть, что для произвольного алгебраического полинома $q_N(x)$ степени не выше $N$ имеет место равенство
\begin{equation}\label{4.6}
\sigma_{r,N}^\alpha(q_N,x)\equiv q_N(x),
\end{equation}
другими словами, $\sigma_{r,N}^\alpha$ является проектором на подпространство алгебраических полиномов $q_N$ степени не выше $N$.

Отметим, что специальные ряды \eqref{4.4} для $r=1$ впервые были введены и исследованы в работе \cite{sharap3}, в которой для операторов $\sigma_{1,N}^\alpha$  при $\frac12\le \alpha\le\frac32$ было получено неравенство типа Лебега и доказаны неулучшаемые по порядку при $N\to\infty$ оценки для констант Лебега этих операторов.

Заметим также, что для определения коэффициентов $\hat F^\alpha_{r,k}$  с помощью равенства \eqref{4.3} и ряда \eqref{4.4} нет необходимости, чтобы функция $f$ принадлежала пространству $W^r_{L(-1,1)}$, а достаточно, чтобы для интегрируемой с весом $(1-x^2)^{\alpha-r}$ функции $f$ существовали производные $f^{(\nu)}(\pm1)$ при $\nu=0,\ldots r-1$. Тогда  интерполяционный полином Эрмита $D_{2r-1}(f,x)$ можно определить равенством
\begin{equation}\label{4.7}
D_{2r-1}(f,x)=
{(1-x^2)^r\over2^r}\sum_{\nu=0}^{r-1}{1\over\nu!}
\sum_{s=0}^{r-1-\nu}{(r)_s\over2^ss!}\left[{f^{(\nu)}(-1)\over(1+x)^{r-\nu-s}}+
{(-1)^\nu f^{(\nu)}(1)\over(1-x)^{r-\nu-s}}\right].
\end{equation}
В частности, для $f\in C[-1,1]$ при $r=1$ ряд \eqref{4.4} и, следовательно,  оператор $\sigma_{1,N}^\alpha(f)$ существуют. Отметим еще, что если функция $F_r(x)$, определенная первым из равенств \eqref{4.1}, интерируема на $(-1,1)$ с весом $\kappa(x)$, то ряд \eqref{4.4} и оператор $\sigma_{r,N}^\alpha(f)$ также существуют. В частности, это имеет место для  произвольной функции $f(x)$, аналитической на $[-1,1]$ при любом $\alpha>-1$.





\section{Аппроксимативные свойства операторов $\sigma_{r,N}^\alpha$}

   Частичную сумму $\sigma_{r,N}^\alpha=\sigma_{r,N}^\alpha(f)=\sigma_{r,N}^\alpha(f,x)$ можно рассмотреть как линейный оператор, действующий в пространстве  $C_r[-1,1]$, состоящем из непрерывных на $[-1,1]$ функций $f$, для которых существуют производные $f^{(\nu)}(\pm1)$ при $\nu=0,\ldots r-1$ и   \begin{equation}\label{5.1}
 \mathcal{ E}_r(f)=\sup_{-1<x<1}{|f(x)-D_{2r-1}(f,x)|\over (1-x^2)^\frac{r}{2}}<\infty,
\end{equation}
где $D_{2r-1}(f,x)$ -- нтерполяционный полином Эрмита, определенный равенством \eqref{4.7}.
Нетрудно проверить, например, что $W^r\subset C_r[-1,1]$. Если $f\in C_r[-1,1]$, то  величина $E_N^r(f)$, определенная равенством \eqref{3.52}, принимает конечные значения для при всех $N\ge 2r-1$, причем $\mathcal{ E}_r(f)\ge E_{2r-1}^r(f)\ge E_{2r}^r(f)\cdots$. Будем рассматривать $\sigma_{r,N}^\alpha(f)$ как аппарат приближения функций $f\in C_r[-1,1]$.
Если $f\in C_r[-1,1]$, то мы можем записать
$$
f(x)-\sigma_{r,N}^\alpha(f,x)=
 f(x)-q_N(f,x)-\sigma_{r,N}^\alpha(f-q_N(f),x)=f(x)-q_N(f,x)
$$
$$
-(1-x^2)^r \sum_{k=0}^{N-2r} \int_{-1}^1(f(t)-q_N(t))(1-t^2)^{\alpha-r} p_{k}^{\alpha,\alpha}(t)p_{k}^{\alpha,\alpha}(x)dt=f(x)-q_N(f,x)
$$
\begin{equation}\label{5.2}
-(1-x^2)^r\int_{-1}^1(f(t)-q_N(f,t))(1-t^2)^{\alpha-r}\sum_{k=0}^{N-2r}  p_{k}^{\alpha,\alpha}(t)p_{k}^{\alpha,\alpha}(x)dt,
\end{equation}
где $q_N(x)=q_N(f,x)$ -- алгебраический полином  степени $N$, который обладает тем свойством, что $q^{(\nu)}_N(f,\pm1)=f^{(\nu)}(\pm1)$ при $\nu=0,\ldots,r-1$  и
\begin{equation}\label{5.3}
E_N^r(f)=\sup_{-1<x<1}{|f(x)-q_N(x)|\over (1-x^2)^\frac{r}{2}}.
\end{equation}
 Из равенства \eqref{5.2} мы выводим следующее неравенство типа Лебега для сумм $\sigma_{r,N}^\alpha(f,x)$:
$$
\frac{|f(x)-\sigma_{r,N}^\alpha(f,x)|}{(1-x^2)^{r-\frac{\alpha}{2}-\frac14}}\le \frac{|f(x)-q_N(f,x)|}{(1-x^2)^\frac{r}{2}}(1-x^2)^\frac{\alpha-r+1/2}{2}+
$$
$$
(1-x^2)^{\frac{\alpha}{2}+\frac14}  \int_{-1}^1\frac{|f(t)-q_N(t)|}{(1-t^2)^{r/2}}(1-t^2)^{\alpha-r/2} \left|\sum_{k=0}^{N-2r}p_{k}^{\alpha,\alpha}(t)p_{k}^{\alpha,\alpha}(x)\right|dt\le
$$
\begin{equation}\label{5.4}
E_N^r(f)\left((1-x^2)^\frac{\alpha-r+1/2}{2}+\Lambda^{r,\alpha}_N(x)\right),
\end{equation}
где
\begin{equation}\label{5.5}
\Lambda^{r,\alpha}_N(x)=(1-x^2)^{\frac{\alpha}{2}+\frac14}  \int_{-1}^1(1-t^2)^{\alpha-r/2} \left|\sum_{k=0}^{N-2r}p_{k}^{\alpha,\alpha}(t)p_{k}^{\alpha,\alpha}(x)\right|dt.
\end{equation}
В связи с неравенством \eqref{5.4} возникает задача об оценке величины $\Lambda^r_N(x)$, определенной равенством \eqref{5.5}, которое мы можем переписать еще так
\begin{equation}\label{5.6}
\Lambda^{r,\alpha}_N(\cos\theta)=(\sin\theta)^{\alpha+\frac12}  \int_{0}^\pi(\sin\tau)^{2\alpha-r+1} \left|\sum_{k=0}^{N-2r}
p_{k}^{\alpha,\alpha}(\cos\tau)p_{k}^{\alpha,\alpha}(\cos\theta)\right|d\tau.
\end{equation}
Мы рассмотрим   более общую задачу, которая будет нужна в дальнейшем. Положим $\Lambda^{r,\alpha}_{N,N}(\cos\theta)=0$, а  при $M>N$
\begin{equation}\label{5.7}
\Lambda^{r,\alpha}_{N,M}(\cos\theta)=(\sin\theta)^{\alpha+\frac12}  \int\limits_{0}^\pi(\sin\tau)^{2\alpha-r+1} \left|\sum_{k=N-2r+1}^{M-2r}
p_{k}^{\alpha,\alpha}(\cos\tau)p_{k}^{\alpha,\alpha}(\cos\theta)\right|d\tau.
\end{equation}

\begin{lemma} Пусть $2r-1\le N< M$, $r-\frac12\le\alpha$. Тогда имеет место следующая оценка
\begin{equation}\label{5.8}
 |\Lambda^{r,\alpha}_{N,M}(\cos\theta)|\le  c(r,\alpha)\ln(M-N+1).
 \end{equation}
\end{lemma}
%\begin{proof} В первую очередь заметим, что в силу симметрии\\  $p_{k}^{\alpha,\alpha}(-x)=(-1)^kp_{k}^{\alpha,\alpha}(x)$ имеем $\Lambda^{r,\alpha}_{N,M}(-x)=\Lambda^{r,\alpha}_{N,M}(x)$, поэтому мы можем ограничиться случаем $0\le\theta\le\pi/2$. Положим $h=\pi/(M-N)$ и рассмотрим   два случая: 1) $0\le \theta\le h$; 2) $h<\theta\le \pi/2$. В первом из этих случаев мы представим $\Lambda^{r,\alpha}_{N,M}(\cos\theta)$ так
%$$
%\Lambda^{r,\alpha}_{N,M}(\cos\theta)=(\sin\theta)^{\alpha+\frac12}  \int\limits_{0}^{\theta+h}(\sin\tau)^{2\alpha-r+1} \left|K_{N,M}^\alpha(\theta,\tau)\right|d\tau
%$$
%\begin{equation}\label{5.9}
%+(\sin\theta)^{\alpha+\frac12}  \int\limits_{\theta+h}^\pi(\sin\tau)^{2\alpha-r+1} \left|K_{N,M}^\alpha(\theta,\tau)\right|d\tau=I_1+I_2,
%\end{equation}
%где
%\begin{equation}\label{5.10}
%K_{N,M}^\alpha(\theta,\tau)=\sum_{k=N-2r+1}^{M-2r}
%p_{k}^{\alpha,\alpha}(\cos\tau)p_{k}^{\alpha,\alpha}(\cos\theta).
% \end{equation}
%Чтобы оценить величину $I_1$, мы воспользуемся оценкой \eqref{2.14}. Поскольку по условию леммы 5.1 $2\alpha-r+1\ge \alpha+1/2$, то с учетом \eqref{5.9} и \eqref{5.10} имеем
%\begin{equation}\label{5.11}
%I_1\le c(\alpha,r)(M -N)  \int\limits_{0}^{\theta+h}d\tau\le c(\alpha).
%\end{equation}
%Перейдем к оценке величины $I_2$.  С этой целью мы обратимся к формуле Кристоффеля-Дарбу
%\eqref{2.14}, которая с учетом  \eqref{2.13}  принимает следующий вид
%$$
% K_n^{\alpha,\beta}(x,y)=
%\sum_{k=0}^np_k^{\alpha,\beta}(x)p_k^{\alpha,\beta}(y)={2^{-\alpha-\beta}\sqrt{h_n^{\alpha,\beta}
%h_{n+1}^{\alpha,\beta}}\over
%2n+\alpha+\beta+2}\times
%$$
%\begin{equation}\label{5.12}
%{\Gamma(n+2)\Gamma(n+\alpha+\beta+2)\over
%\Gamma(n+\alpha+1)\Gamma(n+\beta+1)}
%  {p_{n+1}^{\alpha,\beta}(x)p_n^{\alpha,\beta}(y)-
%p_n^{\alpha,\beta}(x)p_{n+1}^{\alpha,\beta}(y)\over x-y},
%\end{equation}
%а отсюда для $\alpha=\beta$ находим
%\begin{equation}\label{5.13}
% K_n^{\alpha,\alpha}(x,y)=
%\sum_{k=0}^np_k^{\alpha,\alpha}(x)p_k^{\alpha,\alpha}(y)=
%\lambda_n^\alpha  {p_{n+1}^{\alpha,\alpha}(x)p_n^{\alpha,\alpha}(y)-
%p_n^{\alpha,\alpha}(x)p_{n+1}^{\alpha,\alpha}(y)\over x-y},
%\end{equation}
%где $\lambda_n^\alpha=O(1)$ при $n\to\infty$. В дальнейшем для удобства мы будем считать, что $K_{-1}^{\alpha,\alpha}(x,y)=0$, $p_{-1}^{\alpha,\alpha}(x)=0$. Из \eqref{5.10} и \eqref{5.13} имеем $(x=\cos\theta, y=\cos\tau)$
%$$
%K_{N,M}^\alpha(\theta,\tau)=K_{M-2r}^{\alpha,\alpha}(x,y)
%-K_{N-2r}^{\alpha,\alpha}(x,y)=
%$$
%$$
%\lambda_{M-2r}^\alpha  {p_{M-2r+1}^{\alpha,\alpha}(x)p_{M-2r}^{\alpha,\alpha}(y)-
%p_{M-2r}^{\alpha,\alpha}(x)p_{M-2r+1}^{\alpha,\alpha}(y)\over \cos\theta-\cos\tau}-
%$$
%\begin{equation}\label{5.14}
%\lambda_{N-2r}^\alpha  {p_{N-2r+1}^{\alpha,\alpha}(x)p_{N-2r}^{\alpha,\alpha}(y)-
%p_{N-2r}^{\alpha,\alpha}(x)p_{N-2r+1}^{\alpha,\alpha}(y)\over \cos\theta-\cos\tau}.
% \end{equation}
%Теперь обратимся к формуле \eqref{2.8}, которую для $\alpha=\beta$ принимает следующий вид
%\begin{equation}\label{5.15}
%(1-x)P_n^{\alpha+1,\alpha}(x)=P_n^{\alpha,\alpha}(x)-
%\frac{n+1}{n+\alpha+1}P_{n+1}^{\alpha,\alpha}(x).
%\end{equation}
%Из \eqref{5.15} с учетом \eqref{2.13} имеем
%$$
%p_{n+1}^{\alpha,\alpha}(x) =\frac{n+\alpha+1}{n+1}\sqrt{\frac{h_{n+1}^{\alpha,\alpha}}{h_n^{\alpha,\alpha}} }p_n^{\alpha,\alpha}(x)- \frac{n+\alpha+1}{n+1}(1-x)\sqrt{\frac{h_{n+1}^{\alpha,\alpha}}{h_{n}^{\alpha+1,\alpha}}}
%p_n^{\alpha+1,\alpha}(x),
%$$
%а отсюда, в свою очередь,
%$$
%p_n^{\alpha,\alpha}(x)p_{n+1}^{\alpha,\alpha}(y)-p_n^{\alpha,\alpha}(y)p_{n+1}^{\alpha,\alpha}(x)=
%$$
%\begin{equation}\label{5.16}
%\frac{n+\alpha+1}{n+1}\sqrt{\frac{h_{n+1}^{\alpha,\alpha}}{h_{n}^{\alpha+1,\alpha}}}
%[(1-x)p_n^{\alpha+1,\alpha}(x)p_{n}^{\alpha,\alpha}(y)
%-(1-y)p_n^{\alpha+1,\alpha}(y)p_{n}^{\alpha,\alpha}(x)].
%\end{equation}
%Равенство \eqref{5.16} в сочетании с \eqref{2.14} дает
%$$
%(\sin\theta\sin\tau)^{\alpha+\frac12}
%|p_n^{\alpha,\alpha}(\cos\theta)p_{n+1}^{\alpha,\alpha}(\cos\tau)
%-p_n^{\alpha,\alpha}(\cos\tau)p_{n+1}^{\alpha,\alpha}(\cos\theta)|\le
%$$
%\begin{equation}\label{5.17}
%c(\alpha)(\sin(\theta/2)+\sin(\tau/2)),\quad 0\le \theta,\tau\le\pi.
%\end{equation}
%Сопоставляя оценку  \eqref{5.17} с  \eqref{5.14},   при $0\le\theta\le\pi/2$ имеем
%$$
%(\sin\theta\sin\tau)^{\alpha+\frac12}
%|K_{N,M}^\alpha(\theta,\tau)|\le c(r,\alpha){\sin\frac\theta2+\sin\frac\tau2\over|\cos\theta-\cos\tau|}=
%$$
%\begin{equation}\label{5.18}
%c(r,\alpha)
%{\sin\frac{\theta+\tau}{4}\cos\frac{\theta-\tau}{4}\over
%\left|\sin\frac{\theta-\tau}{2}\sin\frac{\theta+\tau}{2}\right|}=
%c(r,\alpha)
%{\cos\frac{\theta-\tau}{4}\over2
%\left|\sin\frac{\theta-\tau}{2}\cos\frac{\theta+\tau}{4}\right|}\le
%\frac{c(r,\alpha)}{|\theta-\tau|}.
%\end{equation}
%Для величины $I_2$ из \eqref{5.9} с учетом  \eqref{5.18} получаем
%$$
%I_2\le\int\limits_{\theta+h}^\pi(\sin\theta\sin\tau)^{\alpha+\frac12}
%|K_{N,M}^\alpha(\theta,\tau)|d\tau\le c(r,\alpha)\int\limits_{\theta+h}^\pi\frac{d\tau}{\tau-\theta}=
%$$
%\begin{equation}\label{5.19}
%c(r,\alpha)\ln\frac{\pi-\theta}{h}\le c(r,\alpha)\ln(M-N+1).
%\end{equation}
%Оценка \eqref{5.8} при $0\le\theta\le h$ вытекает из \eqref{5.9}, \eqref{5.11} и \eqref{5.19}. Перейдем к случаю $h<\theta\le \pi/2$ и запишем
%$$
%\Lambda^{r,\alpha}_{N,M}(\cos\theta)=
%(\sin\theta)^{\alpha+\frac12}  \int\limits_{\theta-h}^{\theta+h}(\sin\tau)^{2\alpha-r+1} \left|K_{N,M}^\alpha(\theta,\tau)\right|d\tau+
%$$
%$$
%(\sin\theta)^{\alpha+\frac12}  \int\limits_{0}^{\theta-h}(\sin\tau)^{2\alpha-r+1} \left|K_{N,M}^\alpha(\theta,\tau)\right|d\tau+
%$$
%\begin{equation}\label{5.20}
%(\sin\theta)^{\alpha+\frac12}  \int\limits_{\theta+h}^\pi(\sin\tau)^{2\alpha-r+1} \left|K_{N,M}^\alpha(\theta,\tau)\right|d\tau=J_1+J_2+J_3.
%\end{equation}
%Величину $J_1$ можно оценить так же, как была оценена $I_1$, а что касается  величин $J_2$ и $J_3$, то они могут быть оценены совершенно аналогично тому, как это было сделано для $I_2$. Другими словами, мы можем записать следующие оценки
%\begin{equation}\label{5.21}
%J_1\le c(\alpha,r), \quad J_2\le c(\alpha,r)\ln(M-N+1),\quad J_3\le c(\alpha,r)\ln(M-N+1).
%\end{equation}
%Из \eqref{5.20} и \eqref{5.21} вытекает справедливость оценки \eqref{5.8} при $h<\theta\le \pi/2$. Лемма 5.1 доказана полностью.
%\end{proof}

\begin{lemma} Пусть $2r-1\le N\le M$, $r-\frac12\le\alpha$, $f\in C_r[-1,1]$. Тогда имеет место следующая оценка
\begin{equation}\label{5.22}
\frac{|\sigma_{r,M}^\alpha(f,x)-\sigma_{r,N}^\alpha(f,x)|}
{(1-x^2)^{r-\frac{\alpha}{2}-\frac14}}\le
c(\alpha,r)E_N^r(f)\ln(M-N+1).
 \end{equation}
\end{lemma}
%\begin{proof} Достаточно рассмотреть случай $M>N$. Мы имеем
%$$
%\frac{|\sigma_{r,M}^\alpha(f,x)-\sigma_{r,N}^\alpha(f,x)|}{(1-x^2)^{r-\frac{\alpha}{2}-\frac14}}\le
%$$
%$$
%(1-x^2)^{\frac{\alpha}{2}+\frac14}  \int_{-1}^1\frac{|f(t)-q_N(t)|}{(1-t^2)^{r/2}}(1-t^2)^{\alpha-r/2} \left|\sum_{k=N-2r+1}^{M-2r}p_{k}^{\alpha,\alpha}(t)p_{k}^{\alpha,\alpha}(x)\right|dt\le
%$$
%\begin{equation}\label{5.23}
%E_N^r(f)\Lambda^{r,\alpha}_{M,N}(x),
%\end{equation}
%где величина $\Lambda^{r,\alpha}_{M,N}(\cos\theta)$ определена равенством \eqref{5.7}, а полином $q_N(t)$ имеет тот же смысл, что выше. Утверждение леммы 5.2 вытекает из \eqref{5.4} и леммы 5.1.
%\end{proof}


\begin{lemma} Пусть $r\ge1$, $f\in C_r[-1,1]$, $r-\frac12\le\alpha$. Тогда имеет место следующая оценка
\begin{equation}\label{5.24}
\frac{|f(x)-\sigma_{r,N}^\alpha(f,x)|}
{(1-x^2)^{r-\frac{\alpha}{2}-\frac14}}\le
c(\alpha,r)E_N^r(f)\ln(N+1).
 \end{equation}
\end{lemma}
%\begin{proof} Из \eqref{5.7} и \eqref{5.5}
%  следует, что $\Lambda^{r,\alpha}_{2r-1,N}(x)=\Lambda^{r,\alpha}_{N}(x)$, поэтому оценка \eqref{5.24} вытекает из леммы 5.1 и неравенства \eqref{5.4}.
%\end{proof}

В работе \cite{Shar15} показано, что оценка \eqref{3.55}  является неулучшаемой по порядку при $N\to\infty$ для  $f\in W^r$, для которых $E^{r}_{N}(f)\asymp N^{-r}$. Подобным же способом можно показать, что оценка  \eqref{5.24} является неулучшаемой по порядку, если $E^{r}_{N}(f)\asymp N^{-r}$.   Если же  наилучшие приближения  $E^{r}_{N}(f)$ убывают при $N\to\infty$ <<быстро>>, то оценка \eqref{5.24}   становится грубой. Возникает задача о получении вместо \eqref{5.24} другой оценки, более точно учитывающей поведение последовательности  $\{E^{r}_{k}(f)\}$. Впервые подобная задача с для тригонометричесих сумм Фурье была решена в работе \cite{OSK}. А в работах \cite{sharap1}, \cite{sharap2} аналогичные задачи были решены для интерполяционных полиномов и сумм Фурье-Якоби. Мы здесь рассмотрим  эту задачу  для операторов $\sigma_{r,N}^\alpha$.

\begin{theorem} Пусть $2r\le N$, $r-1/2\le \alpha\le 2r-1/2$, $f\in C_r[-1,1]$, $-1<x<1$. Тогда имеет место следующая оценка
\begin{equation}\label{5.25}
 \frac{|f(x)-\sigma_{r,N}^\alpha(f,x)|}
{(1-x^2)^{r-\frac{\alpha}{2}-\frac14}}\le c(r,\alpha)\sum_{k=0}^N\frac{E_{N+k}^r(f)}{k+1}.
 \end{equation}
\end{theorem}

%\begin{proof} Имеем
%\begin{equation}\label{5.26}
% \frac{|f(x)-\sigma_{r,N}^\alpha(f,x)|}
%{(1-x^2)^{r-\frac{\alpha}{2}-\frac14}}\le \frac{|f(x)-\sigma_{r,2N}^{\alpha}(f,x)|}{(1-x^2)^{r-\frac{\alpha}{2}-\frac14}}+
%\frac{|\sigma_{r,2N}^{\alpha}(f,x)-\sigma_{r,N}^{\alpha}(f,x)|}
%{(1-x^2)^{r-\frac{\alpha}{2}-\frac14}}.
% \end{equation}
%
%Далее, в силу леммы 5.3,
%$$
%  \frac{|f(x)-\sigma_{r,2N}^{\alpha}(f,x)|}{(1-x^2)^{r-\frac{\alpha}{2}-\frac14}}\le
%$$
%\begin{equation}\label{5.27}
%   c(r,\alpha)E_{2N}^1(f)\ln (2N)\le
%  c(r,\alpha)E_{2N}^r(f)\sum_{k=0}^{N} \frac{1}{k+1}\le c(r,\alpha)\sum_{k=0}^N \frac{E_{N+k}^r(f)}{k+1}.
%  \end{equation}
%С другой стороны, в силу леммы 5.2
%$$
% \frac{|\sigma_{r,2N}^{\alpha}(f,x)-\sigma_{r,N}^{\alpha}(f,x)|}
% {(1-x^2)^{r-\frac{\alpha}{2}-\frac14}}\le \sum_{\nu=0}^{l-1}
% \frac{|\sigma_{r,N_{\nu+1}}^
% {\alpha}(f,x)-\sigma_{r,N_\nu}^{\alpha}(f,x)|}{(1-x^2)^{r-\frac{\alpha}{2}-\frac14}}\le
%$$
%\begin{equation}\label{5.28}
%c(r,\alpha)\sum_{\nu=0}^{l-1}E_{N_\nu}^r(f)\sum_{k=0}^{N_{\nu+1}-N_\nu-1} \frac{1}{k+1},
% \end{equation}
%где $N=N_0<N_1<\ldots<N_l=2N$. Будем считать, что последовательность $\{N_\nu\}_{\nu=1}^{l-1}$ выбрана следующим способом:
%\begin{equation}\label{5.29}
%N_{\nu+1}=\min\{n:E_n^r(f)<\frac12 E_{N_{\nu}}^r(f)\}.
% \end{equation}
% Имеем
%$$
%\sum_{\nu=0}^{l-1}E_{N_\nu}^r(f)\sum_{k=0}^{N_{\nu+1}-N_\nu-1} \frac{1}{k+1}\le\sum_{\nu=0}^{l-1}E_{N_\nu}^r(f)\sum_{k=0}^{N_{\nu+1}-N-1} \frac{1}{k+1}=
%$$
% \begin{equation}\label{5.30}
%\sum_{k=0}^N \frac{1}{k+1}\sum_{\nu:\atop N_{\nu+1}-1\ge N+k}E_{N_\nu}^r(f).
% \end{equation}
%Далее, если мы обозначим через $\nu(k)$ индекс, для которого  $N_{\nu(k)}\le N+k\le N_{\nu(k)+1}-1$, то из \eqref{5.29} следует, что $E_{N_{\nu(k)}}^r(f)\le2E^r_{N+k}(f)$ и
%$$
%\sum_{k=0}^N \frac{1}{k+1}\sum_{\nu:\atop N_{\nu+1}-1\ge N+k}E_{N_\nu}^r(f)=
%\sum_{k=0}^{N-1} \frac{1}{k+1}\sum_{\nu:\atop N_{\nu+1}-1\ge N+ k}E_{N_\nu}^r(f)+
%\frac{E_{N_{l-1}}^r(f)}{N}\le
%$$
%$$
%c\sum_{k=0}^{N-1} \frac{E^r_{N_{\nu(k)}}(f)}{k+1}+\frac{E_{N_{l-1}}^r(f)}{N}\le c\sum_{k=0}^{N-2} \frac{E^r_{N+k}(f)}{k+1}+c\frac{E^r_{N_{l-1}}(f)}{N-1}+\frac{E_{N_{l-1}}^r(f)}{N}\le
%$$
%$$
%c\sum_{k=0}^{N-1} \frac{E^r_{N+k}(f)}{k+1}+
%c\frac{E^r_{2N-1}(f)}{N}\le c\sum_{k=0}^{N-1} \frac{E^r_{N+k}(f)}{k+1}\le c\sum_{k=0}^{N} \frac{E^r_{N+k}(f)}{k+1}.
%$$
%Поэтому утверждение теоремы 3 вытекает из \eqref{5.25} --  \eqref{5.28} и \eqref{5.30}.
%\end{proof}

\begin{corollary} Пусть $2r\le N$, $r-1/2\le \alpha\le 2r-1/2$, $f\in W^r$, $-1<x<1$. Тогда имеет место следующая оценка
\begin{equation}\label{5.31}
 \frac{|f(x)-\sigma_{r,N}^\alpha(f,x)|}
{(1-x^2)^{r-\frac{\alpha}{2}-\frac14}}\le c(r,\alpha)\frac{\ln(N+1)}{N^r}.
 \end{equation}
\end{corollary}

\begin{proof}
Оценка \eqref{5.31} непосредственно вытекает из \eqref{5.25} и \eqref{3.54}.
\end{proof}

В связи с результатом, установленным в теореме 5, возникает задача об изучении поведения величины $E_{n}^r(f)$ при $n\to\infty$. Как уже отмечалось выше, для $f\in W^r$ имеет место неравенство \eqref{3.54}. Но если функция $f$ является аналитической в области, содержащей отрезок $[-1,1]$, то задача о скорости стремления $E_{n}^r(f)$ к нулю при $n\to\infty$ оставалась нерешенной. Нижеследующая лемма дает один из возможных вариантов ответа на этот вопрос. Для того чтобы сформулировать этот результат нам понадобятся некоторые бозначения.

Пусть $0<q<1$, $\mathcal{ E}_q$ -- эллипс с фокусами в точках $\pm1$, сумма полуосей которого равна $R=1/q$, $A_q(B)$ -- класс функций $f(z)$, принимающих вещественные значения при $z\in\mathbb{R}$,  аналитических в эллипсе $\mathcal{ E}_q$ и ограниченных там по модулю числом $B$. Хорошо известно  (см. п.3.7.3 из \cite{Timan}), что если $f\in A_q(B)$, то для коэффициентов Фурье-Чебышева этой функции
\begin{equation}\label{5.32}
 a_k(f)=\int_{-1}^1\frac{f(t)p_k^{-1/2,-1/2}(t)dt}{\sqrt{1-t^2}}
 \end{equation}
имеет место оценка
\begin{equation}\label{5.33}
 |a_k(f)|\le\sqrt{2\pi}Bq^k.
 \end{equation}
Пусть $f\in A_q(B)$, тогда, очевидно, функция $F_r(x)$, определенная первым из равенств \eqref{4.1},  принадлежит классу $A_q(B_r)$ c некоторой константой $B_r$, зависящей лишь от $B$  и $r$. Поэтому из \eqref{5.33}
имеем
\begin{equation}\label{5.34}
 |a_k(F_r)|\le\sqrt{2\pi}B_rq^k.
 \end{equation}
Теперь обратимся к равенству \eqref{4.2}, в котором подставим  $\alpha=-1/2$, тогда получим равенство
 \begin{equation}\label{5.35}
{f(x)-D_{2r-1}(x)\over(1-x^2)^r}=
 \sum_{k=0}^\infty \hat F_{r,k}^{-\frac12} p_{k}^{-\frac12,-\frac12}(x)=
 \sum_{k=0}^\infty a_k(F_r) p_{k}^{-\frac12,-\frac12}(x)
\end{equation}
которое мы можем переписать так ($x=\cos\theta$)
$$
{f(x)-\sigma_{r,N}^{-\frac12}(f,x)\over(1-x^2)^r}=
$$
\begin{equation}\label{5.36}
  \sum_{k=N-2r+1}^\infty a_k(F_r) p_{k}^{-\frac12,-\frac12}(x)=\sqrt{\frac2\pi}
  \sum_{k=N-2r+1}^\infty a_k(F_r) \cos k\theta.
\end{equation}
Из \eqref{5.34} и \eqref{5.36} мы находим
\begin{equation}\label{5.37}
 {|f(x)-\sigma_{r,N}^{-\frac12}(f,x)|\over(1-x^2)^r}\le 2B_r\sum_{k=N-2r+1}^\infty q^k=
 \frac{2B_r}{1-q}q^{N-2r+1}.
  \end{equation}
Из \eqref{5.37} получаем
\begin{equation}\label{5.38}
 E_{n}^r(f)\le \frac{2B_r}{1-q}q^{N-2r+1},\quad f\in A_q(B).
  \end{equation}
Сопоставляя оценку \eqref{5.38} с неравенством \eqref{5.25}, приходим к следующему утверждению.
\begin{corollary} Пусть $2r\le N$, $r-1/2\le \alpha\le 2r-1/2$, $f\in A_q(B)$, $-1<x<1$. Тогда имеет место следующая оценка
\begin{equation}\label{5.39}
 \frac{|f(x)-\sigma_{r,N}^\alpha(f,x)|}
{(1-x^2)^{r-\frac{\alpha}{2}-\frac14}}\le\frac{c(r,\alpha, B)}{(1-q)}q^{N-2r+1}.
 \end{equation}
\end{corollary}



С другой стороны, если  $f\in A_q(B)$, то сопоставляя оценки \eqref{5.37} и \eqref{5.39}, мы замечаем, что ограничение $r-1/2\le \alpha\le 2r-1/2$, содержащееся в условиях следствия 4 является избыточным, так как из \eqref{5.37} следует, что оценка \eqref{5.39} верна и для $\alpha=-\frac12$. Покажем, что ограничение $r-1/2\le \alpha\le 2r-1/2$, содержащееся  в следствии 4, может быть ослаблено до $-1/2\le \alpha\le 2r-1/2$. Для этого нам понадобятся некоторые вспомогательные утверждения.

\begin{lemma} Пусть $\alpha>-1$, $k=n+2j$, $j=0,1,\ldots$. Тогда имеет место следующее равенство
$$
v_{n,j}^\alpha=\int_{-1}^1p_k^{-1/2,-1/2}(t)p_n^{\alpha,\alpha}(t)(1-t^2)^\alpha dt=
$$
\begin{equation}\label{5.40}
\left(\frac{h_n^{\alpha,\alpha}}{h_k^{-\frac12,-\frac12}}\right)^\frac12
 {\Gamma(k+\frac12)(k)_n(1/2)_j(-1/2-\alpha)_j\over\Gamma(n+\frac12)(n+2\alpha+1)_n
(n+\frac12)_j(n+\alpha+\frac32)_j(2j)!}.
  \end{equation}
\end{lemma}
%\begin{proof}
%Воспользуемся равенством \eqref{2.15}, в котором положим $a=-\frac12$. Тогда для $k=n+2j$ имеем
%
%$$
%\int_{-1}^1p_k^{-1/2,-1/2}(t)p_n^{\alpha,\alpha}(t)(1-t^2)^\alpha dt=
%h_n^{\alpha,\alpha}(h_k^{-1/2,-1/2}h_{n}^{\alpha,\alpha})^{-\frac12}
%\times
%$$
%\begin{equation}\label{5.41}
%\frac{P_k^{-1/2,-1/2}(1)}{P_n^{\alpha,\alpha}(1)}
%{k!(\alpha+1)_n(k)_n(1/2)_j(-1/2-\alpha)_j\over n!(n+2\alpha+1)_n
%(1/2)_n(n+1/2)_j(n+\alpha+3/2)_j(2j)!}
%\end{equation}
%и отсюда, имея ввиду \eqref{2.9}, приходим к равенству \eqref{5.40}.
%Лемма доказана.
%\end{proof}

\begin{lemma} Пусть $\alpha>-1$, $j$ -- натурально. Тогда имеет место следующая оценка
$$
|v_{n,j}^\alpha|\le c(\alpha)\left({n\over (j+1)(n+j)}\right)^{\alpha+1}.$$
\end{lemma}

%\begin{proof}
%Пользуясь формулой Стирлинга, имеем ($k=n+2j$)
%$$
% {\Gamma(k+\frac12)(k)_n(1/2)_j|(-1/2-\alpha)_j|\over\Gamma(n+\frac12)(n+2\alpha+1)_n
%(n+\frac12)_j(n+\alpha+\frac32)_j(2j)!}=
%$$
%$$
%{\Gamma(n+2\alpha+1)\Gamma(n+\alpha+\frac32)\over\Gamma(2n+2\alpha+1)}
%{|(-\frac12-\alpha)_j|\Gamma(j+\frac12)\over
%(2j)!\Gamma(\frac12)}\times
%$$
%$$
%\frac{\Gamma(n+2j+\frac12)}{\Gamma(n+2j)}{\Gamma(2(n+j))\over
%\Gamma(n+j+\frac12)\Gamma(n+j+\alpha+\frac32)}
%\le
% $$
% \begin{equation}\label{5.42}
%c(\alpha)n^{\alpha+1}(j+1)^{-\alpha-1}(n+j)^{-\alpha-1}=
%c(\alpha)\left({n\over (j+1)(n+j)}\right)^{\alpha+1}.
% \end{equation}
% Отметим также, что
% \begin{equation}\label{5.43}
% \left(\frac{h_n^{\alpha,\alpha}}{h_k^{-\frac12,-\frac12}}\right)^\frac12\le c(\alpha).
% \end{equation}
% Сопоставляя  оценки \eqref{5.42} и \eqref{5.43}  с леммой 5.4, убеждаемся в справедливости утверждения леммы 5.5.
% \end{proof}

\begin{lemma} Пусть $\alpha>-1$, $f\in A_q(B)$. Тогда имеет место следующая оценка
 \begin{equation}\label{5.44}
|\hat F^\alpha_{r,n}|\le \frac{c(\alpha)\sqrt{2\pi}B_r}{1-q^2}q^n.
 \end{equation}
\end{lemma}

%\begin{proof} Из \eqref{4.3} и \eqref{5.35} имеем
%$$
%\hat F^\alpha_{r,n}=\int_{-1}^1F_r(t)\kappa(t) p_{k}^{\alpha,\alpha}(t)dt=\sum_{k=0}^\infty a_k(F_r)\int_{-1}^1 p_{k}^{-\frac12,-\frac12}(t)p_{n}^{\alpha,\alpha}(t)(1-t^2)^\alpha dt
%$$
%\begin{equation}\label{5.45}
%=\sum_{j=0}^\infty a_{n+2j}(F_r)v_{n,j}^\alpha,
%\end{equation}
%где величина $v_{n,j}^\alpha$ определена равенством \eqref{5.40}. Обратимся теперь к лемме 5.5, тогда из \eqref{5.44} и неравенства \eqref{5.34} выводим
%$$
%|\hat F^\alpha_{r,n}|\le \sum_{j=0}^\infty |a_{n+2j}(F_r)v_{n,j}^\alpha|\le
%$$
%$$
%c(\alpha)\sqrt{2\pi}B_r\sum_{j=0}^\infty q^{n+2j}\left({n\over (j+1)(n+j)}\right)^{\alpha+1}\le \frac{c(\alpha)\sqrt{2\pi}B_r}{1-q^2}q^n.
%$$
%Что и требовалось доказать.
%\end{proof}

Теперь мы можем сформулировать следующий результат.
\begin{theorem} Пусть $2r\le N$, $-1/2\le \alpha\le 2r-1/2$, $f\in A_q(B)$, $-1<x<1$. Тогда имеет место следующая оценка
\begin{equation}\label{5.46}
 \frac{|f(x)-\sigma_{r,N}^\alpha(f,x)|}
{(1-x^2)^{r-\frac{\alpha}{2}-\frac14}}\le\frac{c(r,\alpha, B)}{(1-q)^2}q^{N-2r+1}.
 \end{equation}
\end{theorem}
%\begin{proof} Обратимся к равенству \eqref{4.2}  и приведем его к виду (см.\eqref{4.5})
%$$
%\frac{f(x)-\sigma_{r,N}^\alpha(f,x)}{(1-x^2)^{r}}=
% \sum_{k=N-2r+1}^\infty \hat F_{r,k}^\alpha p_{k}^{\alpha,\alpha}(x),
%$$
%которое, в свою очередь, перепишем так ($x=\cos\theta$)
% \begin{equation}\label{5.47}
%\frac{f(x)-\sigma_{r,N}^\alpha(f,x)}{(1-x^2)^{r-\frac{\alpha}{2}-\frac14}}=
% \sum_{k=N-2r+1}^\infty \hat F_{r,k}^\alpha  p_{k}^{\alpha,\alpha}(\cos\theta)(\sin\theta)^{\alpha+\frac12}.
%\end{equation}
%Воспользовавшись оценкой \eqref{2.14}, из \eqref{5.47} имеем
%\begin{equation}\label{5.48}
%\frac{|f(x)-\sigma_{r,N}^\alpha(f,x)|}{(1-x^2)^{r-\frac{\alpha}{2}-\frac14}}\le c(\alpha)
% \sum_{k=N-2r+1}^\infty |\hat F_{r,k}^\alpha|.
%\end{equation}
%Утверждение теоремы 4  вытекает из оценки \eqref{5.48} и леммы 5.6.
%\end{proof}













%%%%%%%%%%%%%%%%%%%%%%%%%%%%%%%%%
%%%%%%%%%%%%%%%%%%%%%%%%%%%%%%%%%
%%%%%%%%%%%%%%%%%%%%%%%%%%%%%%%%%





%
%
%\begin{thebibliography}{20}
%
%\RBibitem{Shar11}
%\by И.\,И. Шарапудинов
%\paper Приближение функций с переменной гладкостью суммами Фурье Лежандра
%\inbook Математический сборник
%\vol 191
%\issue 5
%\yr 2000
%\pages 143 -- 160.
%
%
%
%
%
%\RBibitem{Shar12}
%\by И.\,И. Шарапудинов
%\paper Аппроксимативные свойства операторов $\mathcal{ Y}_{n+2r}(f)$ и их дискретных аналогов
%\inbook Математические заметки
%\vol 72
%\issue 5
%\yr 2002
%\pages 765–-795.
%
%\RBibitem{Shar13}
%\by И.\,И. Шарапудинов
%\paper Смешанные ряды по ортогональным полиномам
%\inbook
%\publ Издательство Дагестанского научного центра
%\yr 2004
%\pages 1 --176
%\publaddr Махачкала
%
%
%
%\RBibitem{Shar15}
%\by И.\,И. Шарапудинов
%\paper Аппроксимативные свойства смешанных рядов по полиномам Лежандра на классах $W^r$
%\inbook Математический сборник
%\vol 197
%\issue 3
%\yr 2006
%\pages 135–154.
%
%
%\RBibitem{Shar16}
%\by И.\,И. Шарапудинов
%\paper Аппроксимативные свойства средних типа Валле-Пуссена частичных сумм смешанных рядов по полиномам Лежандра
%\inbook Математические заметки
%\vol 84
%\issue 3
%\yr 2008
%\pages 452 -- 471.
%
%\RBibitem{Shar17}
%\by И.\,И. Шарапудинов
%\paper Смешанные ряды по ультрасферическим полиномам и их аппроксимативные свойства
%\inbook Математический сборник
%\vol 194
%\issue 3
%\yr 2003
%\pages 115--148
%
%
%\RBibitem{Shar18}
%\by И.\,И. Шарапудинов, Т.\, И. Шарапудинов
%\paper Смешанные ряды по полиномам Якоби и Чебышева и их дискретизация
%\inbook Математические заметки
%\vol 88
%\issue 1
%\yr 2010
%\pages 116 -- 147
%
%\RBibitem{sharap3}
%\by И.\,И. Шарапудинов
%\paper Некоторые специальные ряды по ультрасферическим полиномам и их аппроксимативные свойства
%\inbook Изв. РАН. Сер. матем.
%\vol 78
%\issue 5
%\yr 2014
%\pages 201 -- 224
%
%
%\RBibitem{SHII}
%\by И.\,И. Шарапудинов
%\paper Некоторые специальные ряды по общим полиномам Лагерра и ряды Фурье по полиномам Лагерра, ортогональным по Соболеву
%\inbook Дагестанские электронные математические известия
%\vol 4
%\issue
%\yr 2015
%\pages
%
%
%
%
%\RBibitem{Sege}
%\by Г. Сеге
%\paper
%\inbook Ортогональные многочлены
%\publ Физматгиз
%\yr 1962
%\pages
%\publaddr Москва
%
%
%\RBibitem{Gasper}
%\by G. Gasper
%\paper Positiviti and special function
%\inbook  Theory and appl.Spec.Funct. Edited by Richard A.Askey.
%\vol
%\issue
%\yr 1975
%\pages 375 -- 433
%
%
%\RBibitem{KwonLittl1}
%\by K.\,H. Kwon and L.\,L. Littlejohn
%\paper The orthogonality of the Laguerre polynomials $\{L_n^{(-k)}(x)\}$ for positive integers $k$
%\inbook Ann. Numer. Anal.
%\vol
%\issue 2
%\yr 1995
%\pages 289 –- 303.
%
%\RBibitem{ KwonLittl2}
%\by K.\,H. Kwon and L.\,L. Littlejohn
%\paper Sobolev orthogonal polynomials and second-order differential equations
%\inbook Rocky Mountain J. Math.
%\vol 28
%\issue
%\yr 1998
%\pages 547 –- 594.
%
%\RBibitem{MarcelAlfaroRezola }
%\by F. Marcellan, M. Alfaro and M.L. Rezola
%\paper Orthogonal polynomials on Sobolev spaces: old and new directions
%\inbook Journal of Computational and Applied Mathematics
%\vol 48
%\issue
%\yr 1993
%\pages 113 -- 131.
%\publaddr North-Holland
%
%\RBibitem{IserKoch }
%\by A. Iserles, P.E. Koch, S.P. Norsett and J.M. Sanz-Serna
%\paper On polynomials  orthogonal  with respect  to certain Sobolev inner products
%\inbook ,  J. Approx. Theory
%\vol 65
%\issue
%\yr 1991
%\pages 151-175.
%
%\RBibitem{Meijer}
%\by H.\,G. Meijer, Laguerre polynimials generalized to a certain
%\paper  Laguerre polynimials generalized to a certain discrete Sobolev inner product space
%\inbook ,  J. Approx. Theory
%\vol 73
%\issue
%\yr 1993
%\pages 1-16.
%
%\RBibitem{Lopez1995}
%\by Lopez G. Marcellan F. Vanassche W.
%\paper Relative Asymptotics for Polynomials Orthogonal with Respect to a Discrete Sobolev Inner-Product
%\inbook Constr. Approx.
%\vol 11:1
%\yr 1995
%\pages 107–137
%
%
%
%
%\RBibitem{MarcelXu}
%\by F. Marcellan and Yuan Xu
%\paper On Sobolev orthogonal polynomials
%\inbook  Expositiones Mathematicae
%\vol 33
%\issue 3
%\yr 2015
%\pages 308--352
%
%\RBibitem{Shar2016}
%\by И.И. Шарапудинов
%\paper Системы функций, ортогональные по Соболеву, порожденные ортогональными функциями
%\inbook Материалы 18-й международной Саратовской зимней школы «Современные проблемы теории функций и их приложения»
%\publ ООО «Издательство «Научная книга»
%\yr 2016
%\pages 329-332
%\publaddr Саратов
%
%
%
%
%
%\RBibitem{Tref1}
%\by   L.N. Trefethen
%\book Spectral methods in Matlab
%\yr 2000
%\serial
%\publ SIAM
%\publaddr Fhiladelphia
%\vol
%
%\RBibitem{Tref2}
%\by   L.N. Trefethen
%\book Finite difference and spectral methods for ordinary and partial differential equation
%\yr 1996
%\serial
%\publ Cornell University
%\publaddr
%\vol
%
%
%\RBibitem{SolDmEg}
%\by  В.В. Солодовников, А.Н. Дмитриев, Н.Д. Егупов
%\book Спектральные методы расчета и проектирования систем управления
%\yr 1986
%\serial
%\publ Машиностроение
%\publaddr Москва
%\vol
%
%\RBibitem{Pash}
%\by С. Пашковский
%\paper
%\inbook Вычислительные применения многочленов и рядов Чебышева
%\publ Наука
%\yr 1983
%\pages
%\publaddr Москва
%\pages 143 -- 160.
%
%\RBibitem{MMG2016}
%\by М.Г. Магомед-Касумов
%\paper Приближенное решение обыкновенных дифференциальных уравнений с использованием смешанных рядов по системе Хаара
%\inbook Материалы 18-й международной Саратовской зимней школы «Современные проблемы теории функций и их приложения»
%\publ ООО «Издательство «Научная книга»
%\yr 2016
%\pages 176-178
%\publaddr Саратов
%
%
%
%
%
%
%\RBibitem{Gonchar1975}
%\by А. А. Гончар
%\paper О сходимости аппроксимаций Паде для некоторых классов мероморфных функций
%\inbook Матем. сб.
%\vol 97(139):4(8)
%\yr 1975
%\pages 607–629
%
%
%
%
%
%
%
%
%
%
%\RBibitem{TEL}
%\by С.\, А. Теляковский
%\paper Две теоремы о приближении функций алгебраическими многочленами
%\inbook Математический сборник
%\vol 70
%\issue 2
%\yr 1966
%\pages 252 -- 265
%
%
%\RBibitem{GOP}
%\by И.\, З. Гопенгауз
%\paper К теореме А. Ф. Тимана о приближении функций многочленами на
%конечном отрезке
%\inbook Математические  заметки
%\vol 1
%\issue 2
%\yr 1967
%\pages 163 -- 172
%
%
%
%
%
%\RBibitem{OSK}
%\by К.\, И. Осколков
%\paper К неравенству Лебега в равномерной метрике и на множестве полной меры
%\inbook Математические  заметки
%\vol 18
%\issue 4
%\yr 1975
%\pages 515 -- 526
%
%\RBibitem{sharap1}
%\by I.\,I. Sharapudinov
%\paper On the best approximation and polinomial of the least quadratic deviation
%\inbook Analysis Mathematica
%\vol 9
%\issue 3
%\yr 1983
%\pages 223 -- 234
%
%
%\RBibitem{sharap2}
%\by И.\,И. Шарапудинов
%\paper О наилучшем приближении и суммах Фурье-Якоби
%\inbook Математические заметки
%\vol 34
%\issue 5
%\yr 1983
%\pages 651 -- 661
%
%
%
%\RBibitem{Timan}
%\by А.Ф. Тиман
%\paper
%\inbook  Теория приближения функций действительного переменного
%\publ Физматгиз
%\yr 1960
%\pages
%\publaddr Москва
%
%
%
%
%
%
%
%\end{thebibliography}
%
%
