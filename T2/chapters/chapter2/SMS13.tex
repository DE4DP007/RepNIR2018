
\chapter{Аппроксимативные свойства средних Валле -- Пуссена частичных сумм
конечного предельного ряда по полиномам Чебышeва, ортогональным на равномерной сетке}

%\section{Введение}

%Рассмотрена задача о приближении дискретной действительнозначной функции $f$, заданной на равномерной сетке $\Omega_N = \left\{ 0, 1, \ldots, N-1 \right\}$, средними типа Валле -- Пуссена частичных сумм так называемого \textit{конечного предельного ряда} по классическим ортогональным полиномам Чебышева, ортогональным на $\Omega_N$. Основной результат настоящей работы касается оценки функции Лебега для рассматриваемых средних типа Валле -- Пуссена.

%Для всех $\alpha,\beta>-1$ через $T_{n}^{\alpha,\beta}(x,N), (0\le n\le N-1)$ обозначим полиномы Чебышева, ортогональные на равномерной сетке $\Omega_N$ с весом
%\begin{equation}
%\label{sob-tcheb-difference-3.5}
%\mu(x) = \mu(x; \alpha,\beta, N)={\Gamma(N)2^{\alpha+\beta+1}\over \Gamma(N+\alpha+\beta+1)}
%{\Gamma(x+\beta+1)\Gamma(N-x+\alpha)\over \Gamma(x+1)\Gamma(N-x)}
%\end{equation}
%и нормированные условием $T_{n}^{\alpha,\beta}(N-1,N)={n+\alpha\choose n}$. Более точно, имеет место равенство
%\begin{equation*}
%\sum_{x=0}^{N-1}T_{n}^{\alpha,\beta}(x) T_{m}^{\alpha,\beta}(x)
%\mu(x)= \delta_{n,m}h_{n,N}^{\alpha,\beta},\quad 0\le n,m\le N-1,
%\end{equation*}
%где
%\begin{equation}
%\label{sms2hn}
%h_{n,N}^{\alpha,\beta}={(N+n+\alpha+\beta)^{[n]}\over
%(N-1)^{[n]}}{\Gamma(n+\alpha+1)\Gamma(n+\beta+1)
%2^{\alpha+\beta+1}\over
%n!\Gamma(n+\alpha+\beta+1)(2n+\alpha+\beta+1)}.
%\end{equation}
%Построим тогда на основе полиномов $T_{n}^{\alpha,\beta}(x)$ ортонормированную на
%$\Omega_N$ с весом $\mu(x)$ систему полиномов:
%\begin{equation}
%\label{sms2eq5}
%\tau_{n}^{\alpha,\beta}(x) = \tau_{n}^{\alpha,\beta}(x,N)=
%\left[h_{n,N}^{\alpha,\beta}\right]^{-1/2}
%T_{n}^{\alpha,\beta}(x,N),
%\end{equation}
%\begin{equation}
%\label{sms2eq6}
%\sum_{x=0}^{N-1}\tau_{n}^{\alpha,\beta}(x)
%\tau_{m}^{\alpha,\beta}(x)\mu(x)=\delta_{n,m}, \quad (0\le n\le N-1).
%\end{equation}
%
%В таком случае для заданной на сетке $\Omega_N$ дискретной функции $f(x)$ мы можем определить дискретный ряд Фурье-Чебышева
%\begin{equation}
%\label{sms2eq1}
%f(x)=\sum\limits_{k=0}^{N-1}\hat{f}^{\alpha,\beta}_k\tau_{k}^{\alpha,\beta}(x),\quad (x \in \Omega_N),
%\end{equation}
%где коэффициенты Фурье-Чебышева задаются формулой
%\begin{equation}
%\label{sms2eq2}
%\hat{f}_k = \sum\limits_{x=0}^{N-1}f(x)\tau_{k}^{\alpha,\beta}(x)\mu(x),\quad (0\le n\le N-1).
%\end{equation}
%Соответственно частичная сумма $n$-го порядка дискретной суммы Фурье -- Чебышева может быть записана в виде
%\begin{equation}
%\label{sms2eq5}
%S_{n,N}^{\alpha,\beta}(f,x) = \sum\limits_{k=0}^{n}\hat{f}^{\alpha,\beta}_k \tau_{k}^{\alpha,\beta}(x),\quad (0\le x\le N-1).
%\end{equation}
%
%%Далее для краткости мы примем следующие обозначения: $\tau_{n,N}^{\alpha}(x) = \tau_{n,N}^{\alpha,\alpha}(x)$, $S_{n,N}^{\alpha}(f,x) = S_{n,N}^{\alpha,\alpha}(f,x)$.
%Далее в случаях, когда $\alpha = \beta$, для краткости мы будем записывать в качестве индекса лишь один параметр, например: $\tau_{n}^{\alpha}(x) = \tau_{n}^{\alpha,\alpha}(x)$, $S_{n,N}^{\alpha}(f,x) = S_{n,N}^{\alpha,\alpha}(f,x)$ и т.д.
%
%
%\section{Некоторые свойства дискретных полиномов Чебышева}
%
%В дальнейшем изложении мы будем опираться на ряд свойств дискретных полиномов Чебышева, которые мы соберем в данном пункте (см., например, [1, \S\,3.2, ШИИ, Спецвыпуск в ДЭМИ]).
%
%Имеет место следующее явное представление дискретных полиномов Чебышева:
%\begin{equation}
%\label{sms2eq3}
%T_{n}^{\alpha,\beta}(x) = T_{n}^{\alpha,\beta}(x,N)=(-1)^n{\Gamma(n+\beta+1)\over n!}\sum_{k=0}^n (-1)^k
%{n^{[k]}(n+\alpha+\beta+1)_kx^{[k]}\over \Gamma(k+\beta+1)
%k!(N-1)^{[k]}}.
%\end{equation}
%Здесь $a^{[0]}=1, a^{[n]}=a(a-1)\ldots (a-n+1)$; $(a)_0=1, (a)_n=a(a+1)\ldots(a+n-1)$.
%
%
%Для них справедливы следующие рекуррентные формулы:
%\begin{equation}\label{recur}
%  A^{\alpha,\beta}_{n,N} T^{\alpha,\beta}_{n+1}(x,N) = (x-B^{\alpha,\beta}_{n,N}) T^{\alpha,\beta}_{n}(x,N) - C^{\alpha,\beta}_{n,N} T^{\alpha,\beta}_{n-1}(x,N),
%\end{equation}
%где $T^{\alpha,\beta}_{0}(x,N)=1$, $T^{\alpha,\beta}_{1}(x,N)=-1-\beta+x(2+\alpha+\beta)/(N-1)$,
%\begin{equation*}
%  A^{\alpha,\beta}_{n,N} = \frac{(n+1)(n+\alpha+\beta+1)(N-n-1)}{(2n+\alpha+\beta+1)(2n+\alpha+\beta+2)},
%\end{equation*}
%\begin{equation*}
%  C^{\alpha,\beta}_{n,N} = \frac{(n+\alpha)(n+\beta)(N+n+\alpha+\beta)}{(2n+\alpha+\beta)(2n+\alpha+\beta+1)},
%\end{equation*}
%\begin{equation*}
%  B^{\alpha,\beta}_{n,N} = A^{\alpha,\beta}_{n,N}\,\frac{n+\beta+1}{n+1}+C^{\alpha,\beta}_{n,N}\,\frac{n}{n+\beta},
%\end{equation*}
%а также свойство симметрии:
%\begin{equation}\label{simmetry}
%  T^{\alpha,\beta}_{n}(x) = (-1)^{n} T^{\beta, \alpha}_{n}(N-1-x).
%\end{equation}
%Кроме того, из явного вида и свойств конечной разности вытекает следующее равенство
%\begin{equation}\label{recur1}
%  (n+1) T^{\alpha,\beta}_{n+1}(x,N) + (n+\beta+1)T^{\alpha,\beta}_{n}(x,N) = \frac{2n+\alpha+\beta+2}{N-1}\,x\,T^{\alpha,\beta+1}_{n}(x-1,N-1),
%\end{equation}
%из которого, в свою очередь, используя свойство симметрии \eqref{simmetry} получаем
%\begin{equation}\label{recur2}
%  (n+\alpha+1) T^{\alpha,\beta}_{n}(x,N) - (n+1)T^{\alpha,\beta}_{n+1}(x,N) = \frac{2n+\alpha+\beta+2}{N-1}\,(N-1-x)\,T^{\alpha+1,\beta}_{n}(x,N-1).
%\end{equation}
%
%
%
%
%
%Для этих полиномов справедлива формула Кристоффеля -- Дарбу:
%\begin{equation*}
%  D^{\alpha,\beta}_{n} (x,y) = D^{\alpha,\beta}_{n,N} (x,y) = \sum_{k=0}^{n} \frac{T^{\alpha,\beta}_{k}(x) T^{\alpha,\beta}_{k}(y)}{h^{\alpha,\beta}_{n,N}} =
%\end{equation*}
%\begin{equation*}
%  \frac{(N-1)^{[n+1]}}{(N+n+\alpha+\beta)^{[n]}} \frac{2^{-\alpha-\beta-1}}{2n+\alpha+\beta+2}
%  \frac{\Gamma(n+2)\Gamma(n+\alpha+\beta+2)}{\Gamma(n+\alpha+1)\Gamma(n+\beta+1)} \cdot
%\end{equation*}
%\begin{equation}\label{kristT}
%  \frac{T_{n+1}^{\alpha,\beta}(x) T_{n}^{\alpha,\beta}(y) - T_{n+1}^{\alpha,\beta}(y) T_{n}^{\alpha,\beta}(x)}{x-y}.
%\end{equation}
%
%В работе [2, ШИИ, Асимпт.св-ва и вес.оценки Чеб.-Хана] показано, что для любых целых $\alpha,\beta \geq 0$ имеет место асимптотическая формула
%\begin{equation}\label{asymptf}
% T_{n}^{\alpha,\beta}\left( \frac{N-1}{2}(1+t) \right) = P_{n}^{\alpha,\beta}(t)+r_{n,N}^{\alpha,\beta}(t),
%\end{equation}
%в которой $P_{n}^{\alpha,\beta}(t)$ -- классический полином Якоби, а для остаточного члена при $1 \leq n \leq a\sqrt{N}, (a>0)$, справедливы оценки
%
%\begin{equation}\label{asymptRest1}
%  \max\limits_{0\leq t \leq 1} |r_{n,N}^{\alpha,\beta}(t)| \leq c(\alpha,\beta,a)\,n^{\alpha+1/2},
%\end{equation}
%\begin{equation}\label{asymptRest1}
%  |r_{n,N}^{\alpha,\beta}(\cos{\theta})| \leq c(\alpha,\beta,a,c)\,\theta^{-\alpha-1/2}, \quad \left( cn^{-1} \leq \theta \leq \pi/2\right).
%\end{equation}
%В качестве следствия этой формулы, при тех же ограничениях на $n$ и $N$, получена весовая оценка
%\begin{equation}\label{weightEst}
%  \left| T_{n}^{\alpha,\beta}\left(\frac{N-1}{2}(1+t)\right)\right| \leq
%    c(\alpha,\beta,a)\,n^{-1/2}\,
%    \left[  \sqrt{1-t} + \frac1n \right]^{-\alpha-1/2}
%    \left[  \sqrt{1+t} + \frac1n \right]^{-\beta-1/2},
%\end{equation}
%
%или иначе
%
%\begin{equation}\label{weightEst1}
%  \left| T_{n}^{\alpha,\beta}\left(\frac{N-1}{2}(1+t)\right)\right| \leq
%  c(\alpha,\beta,a,c)\,
%  \left\{
%    \begin{aligned}
%    n^{\beta}, \quad (-1 \leq \theta \leq -1+cn^{-2}),\\
%    \left(n(1+t)^{\beta+1/2}\right)^{-1/2}, \quad (-1+cn^{-2} \leq \theta \leq 0), \\
%    \left(n(1-t)^{\alpha+1/2}\right)^{-1/2}, \quad (0 \leq \theta \leq 1-cn^{-2}), \\
%    n^{\alpha}, \quad (1-cn^{-2} \leq \theta \leq 1).\\
%    \end{aligned}
%  \right.
%\end{equation}
%
%В работе [3, ШИИ, Об ограниченности...] показано, что оценки \eqref{weightEst} -- \eqref{weightEst1} сохраняются также для $T_{n}^{\alpha,\beta}\left(\frac{N-1}{2}(1+t)-j_1, N-j_2\right)$, где $j_1,j_2$ --- фиксированные целые числа.






\section{Конечный предельный ряд по дискретным полиномам Чебышева}


Для заданной на сетке $\Omega_N$ дискретной функции $f(x)$ мы можем определить дискретный ряд Фурье-Чебышева
\begin{equation}
\label{sms2eq1}
f(x)=\sum\limits_{k=0}^{N-1}\hat{f}^{\alpha,\beta}_k\tau_{k}^{\alpha,\beta}(x),\quad (x \in \Omega_N),
\end{equation}
где коэффициенты Фурье-Чебышева задаются формулой
\begin{equation}
\label{sms2eq2}
\hat{f}_k = \sum\limits_{x=0}^{N-1}f(x)\tau_{k}^{\alpha,\beta}(x)\mu(x),\quad (0\le n\le N-1).
\end{equation}
Соответственно частичная сумма $n$-го порядка дискретной суммы Фурье -- Чебышева может быть записана в виде
\begin{equation}
\label{sms2eq5}
S_{n,N}^{\alpha,\beta}(f,x) = \sum\limits_{k=0}^{n}\hat{f}^{\alpha,\beta}_k \tau_{k}^{\alpha,\beta}(x),\quad (0\le x\le N-1).
\end{equation}

%Далее для краткости мы примем следующие обозначения: $\tau_{n,N}^{\alpha}(x) = \tau_{n,N}^{\alpha,\alpha}(x)$, $S_{n,N}^{\alpha}(f,x) = S_{n,N}^{\alpha,\alpha}(f,x)$.
Далее в случаях, когда $\alpha = \beta$, для краткости мы будем записывать в качестве индекса лишь один параметр, например: $\tau_{n}^{\alpha}(x) = \tau_{n}^{\alpha,\alpha}(x)$, $S_{n,N}^{\alpha}(f,x) = S_{n,N}^{\alpha,\alpha}(f,x)$ и т.д.



Как было отмечено выше, все перечисленные свойства справедливы при $\alpha, \beta > -1$. Если же взять $\alpha = \beta = -1$, то вес $\eqref{sob-tcheb-difference-3.5}$ обращается в бесконечность в точках $x=0$ и $x=N-1$, и суммы $\sum\limits_{x=0}^{N-1} f(x)\tau_{k}^{-1}(x)\mu^{-1}(x)$ теряют смысл.
%, а получаемая система полиномов $\left\{ \tau_n^{-1}(x) \right\}_{n=0}^{N-1}$ перестает быть ортогональной.
Однако, как показано в работе \cite{smsshti1}, можно рассмотреть дискретный ряд, получаемый в результате почленного предельного перехода в дискретном ряде \eqref{sms2eq1} при $\alpha,\beta \rightarrow -1$, т.е. $f(x) = \sum\limits_{k=0}^{N-1}\hat{f}^{-1}_k\tau_{k}^{-1}(x),\quad (x \in \Omega_N)$. %, где $\hat{f}^{-1}_k\tau_k^{-1}(x) = \lim_{\alpha \rightarrow -1} \hat{f}^{\alpha}_k\tau_k^{\alpha}(x)$.
%Не смотря на то, что суммы $\sum\limits_{x=0}^{N-1} f(x)\tau_k^{-1}(x)\mu^{-1}(x,N)$ невозможно вычислить (ввиду того что $\mu^{-1}(0)=\mu^{-1}(N-1)=\infty$), рассматривая выражения $\hat{f}^{-1}_k\tau_k^{-1}(x) = \lim_{\alpha \rightarrow -1} \hat{f}^{\alpha}_k\tau_k^{\alpha}(x)$ в целом, удается показать, что
Не смотря на то, что сами коэффициенты $f_k^{-1}$ не удается вычислить (ввиду того что $\mu^{-1}(0)=\mu^{-1}(N-1)=\infty$), рассматривая выражения $\hat{f}^{-1}_k\tau_{k}^{-1}(x) = \lim_{\alpha \rightarrow -1} \hat{f}^{\alpha}_k\tau_k^{\alpha}(x)$ в целом, удается показать, что
\begin{equation}
\label{limitR}
 f(x) = \sum\limits_{k=0}^{N-1}\hat{f}^{-1}_k  \tau_{k}^{-1}(x) =
a_N(f,x) + \frac{8x(N-x-1)}{N(N-1)} \sum\limits_{k=0}^{N-3} \hat{g}_k \tau_{k}^{1}(x-1,N-2),
\end{equation}
где
\begin{equation}
\label{af}
a_N(f,x) = \frac{f(N-1)+f(0)}{2} + \frac{f(N-1)-f(0)}{2}\left( \frac{2x}{N-1}-1\right),
\end{equation}
%\begin{equation}
%\label{bnN}
%b_{n,N} = \frac{8x(N-x-1)}{N(N-1)},
%\end{equation}
а коэффициенты
\begin{equation}
\label{gk}
\hat{g}_k = \hat{g}_k(N) = \frac{1}{N-2} \sum\limits_{j=1}^{N-2} g(j) \tau_{k}^{1}(j-1,N-2) ,
\end{equation}
получены для функции $g(x) = f(x) - a_N(f,x)$.

Ряд \eqref{limitR} называется \textit{предельным рядом } по полиномам Чебышева, ортогональным на равномерной сетке. Соответствующий предельный случай частичных сумм Фурье -- Чебышева тогда может быть записано
\begin{equation}
\label{limitS}
 S_{n,N}^{-1}(f,x) = \lim_{\alpha \rightarrow -1} S_{n,N}^{\alpha}(f,x) =
 a_N(f,x) + \frac{8x(N-x-1)}{N(N-1)} \sum\limits_{k=0}^{n-2} \hat{g}_k \tau_{k}^{1}(x-1,N-2).
\end{equation}

В работе \cite{smsshti1} были исследованы аппроксимативные свойства частичных сумм предельного ряда и показано, что они обладают следующими тремя свойствами:
\begin{enumerate}[1)]
  \item $S_{n,N}^{-1}(f,x)$ совпадает с $f(x)$ в точках $x=0$ и $x=N-1$:%, т.е. $S_{n,N}^{-1}(f,0) = f(0)$, $S_{n,N}^{-1}(f,N-1)=f(N-1)$;
    \begin{equation}\label{limprop1}
        S_{n,N}^{-1}(f,0) = f(0), \quad S_{n,N}^{-1}(f,N-1)=f(N-1);
    \end{equation}
  \item $S_{n,N}^{-1}(f,x)$ представляет собой проектор на пространство $H_n$ всех алгебраических полиномов $P_n(x)$ степени, не выше $n$:%, т.е.   $S_{n,N}^{-1}(P_n,x) = P_n(x)$;
    \begin{equation}\label{limprop2}
        S_{n,N}^{-1}(P_n,x) = P_n(x);
    \end{equation}
  \item Подставляя в \eqref{limitS} выражение для \eqref{gk}, получим следующее представление для частичных сумм предельного ряда:
    %\begin{equation*}
%     S_{n,N}^{-1}(f,x) =
%     a_N(f,x) + \frac{8x(N-x-1)}{N(N-1)(N-2)} \sum\limits_{j=1}^{N-2} g(j) \sum\limits_{k=0}^{n-2} \tau_{k}^{1}(j-1,N-2) \tau_{k}^{1}(x-1,N-2).
%    \end{equation*}
    \begin{equation}
    \label{Sn_jay}
     S_{n,N}^{-1}(f,x) =
     a_N(f,x) + \frac{8x(N-x-1)}{N(N-1)(N-2)} \sum\limits_{j=1}^{N-2} g(j) J_{n-2,N-2}(x-1,j-1) ,
    \end{equation}
    где
    \begin{equation}\label{Jay}
      J_{l,N}(x,j) = \sum_{r=0}^{l} \tau^{1}_{r}(x) \tau^{1}_{r}(j).
    \end{equation}

  \item Если $0\le x\le N-3$, $n\le a\sqrt{N}$, то имеет место следующая оценка
    \begin{equation}\label{applimitR}
        |f(x)-S_{n,N}^{-1}(f,x)|\le c(a)E_{n,N}^{*}(f)\left[1+\ln\left(2+\frac{n}{N-1}\sqrt{x(N-3-x)}\right)\right],
    \end{equation}
    где
    $$E_{n,N}^{*}(f) = \inf_{p_n \in \hat{H}^n} \sup_{0\le x\le N-3} |f(x)-p_n(x)|$$
    -- наилучшее приближение функции $f(x)$ алгебраическими полиномами, которые в точках $0$ и $N-1$ совпадают со значениями самой функции $f(x)$.

    Здесь и далее через $c, c(a), c(a,b)$ и т.д. -- мы обозначаем положительные числа, зависящее только от указанных в скобках параметров и, вообще говоря, различные в разных местах.
\end{enumerate}

%\section{Вспомогательные утверждения}
%
%Далее нам понадобится модифицированный вариант полиномов Чебышева:
%\begin{equation}\label{Qnn}
% Q_{n}^{\alpha,\beta}(t) = Q_{n}^{\alpha,\beta}(t,N) = T_{n}^{\alpha,\beta}\left( \frac{N-1}{2}(1+t) - 1, N-1 \right).
%\end{equation}
%для которого в работе [3, ШИИ, Об ограниченности...] показана справедливость оценок \eqref{weightEst} -- \eqref{weightEst1}, а также следующая рекуррентная формула
%\begin{equation}\label{recurQ}
%  \hat{A}^{\alpha,\beta}_{n,N} Q^{\alpha,\beta}_{n+1}(t) = \frac{N-1}{2}\,(t+\lambda^{\alpha,\beta}_{n,N}) Q^{\alpha,\beta}_{n}(t) - \hat{C}^{\alpha,\beta}_{n,N} Q^{\alpha,\beta}_{n-1}(t),
%\end{equation}
%где $\hat{A}^{\alpha,\beta}_{n,N}= A^{\alpha,\beta}_{n,N-1}$, $\hat{C}^{\alpha,\beta}_{n,N} = C^{\alpha,\beta}_{n,N-1}$,
%\begin{equation*}
%  \hat{A}^{\alpha,\beta}_{n,N} = \frac{(n+1)(n+\alpha+\beta+1)(N-n-2)}{(2n+\alpha+\beta+1)(2n+\alpha+\beta+2)},
%\end{equation*}
%\begin{equation*}
%  \hat{C}^{\alpha,\beta}_{n,N} = \frac{(n+\alpha)(n+\beta)(N+n+\alpha+\beta-1)}{(2n+\alpha+\beta)(2n+\alpha+\beta+1)},
%\end{equation*}
%\begin{equation*}
%  \lambda^{\alpha,\beta}_{n,N} = \frac{\alpha^2-\beta^2}{(2n+\alpha+\beta)(2n+\alpha+\beta+2)}+\frac{q(n,\alpha,\beta)}{N-1}, \quad \text{где } |q(n,\alpha,\beta)| \leq c(\alpha,\beta).
%\end{equation*}
%Рассмотрим отдельно два частных случая:
%
%1. $\alpha=1, \beta=2$:
%
%\begin{equation*}
%    \frac{(n+1)(n+4)(N-n-2)}{(n+2)(2n+5)(N-1)} \, Q^{1,2}_{n+1}(t) =
%    \left(t-
%    \frac{3}{(2n+3)(2n+5)}+\frac{q(n,1,2)}{N-1}
%    \right)\,
%    Q^{1,2}_{n}(t) -
%\end{equation*}
%\begin{equation}\label{34}
%    \frac{(n+1)(N+n+2)}{(2n+3)(N-1)}\,
%    Q^{1,2}_{n-1}(t).
%\end{equation}
%
%
%2. $\alpha=2, \beta=1$:
%
%
%\begin{equation*}
%  \frac{(n+1)(n+4)(N-n-2)}{(n+2)(2n+5)(N-1)} Q^{2,1}_{n+1}(t) =
%  \left(t+\frac{3}{(2n+3)(2n+5)}+\frac{q(n,2,1)}{N-1}\right) Q^{2,1}_{n}(t)
%\end{equation*}
%\begin{equation}\label{35}
%  - \frac{(n+1)(N+n+2)}{(2n+3)(N-1)} \, Q^{2,1}_{n-1}(t).
%\end{equation}
%
%\begin{lemma}\label{lemma0}
%Для всех $t,v \in [-1,1]$ имеет место равенство
%\begin{equation*}
%  T_{l+1}^1(x) T_{l}^1(j) - T_{l+1}^1(j) T_{l}^1(x) =
%    \frac{(l+2)}{2(l+1)} \,
%  \left[
%    (1+t)(1-v)\,Q^{1,2}_{l}(t)\,Q^{2,1}_{l}\left(v+\frac{2}{N-1}\right)
%  \right.
%\end{equation*}
%\begin{equation}\label{lem0eq}
%  \left.
%  - (1+v)(1-t)\,Q^{1,2}_{l}(v)\,Q^{2,1}_{l}\left(t+\frac{2}{N-1}\right)\right],
%  \quad (l=0, \ldots, N-2).
%\end{equation}
%
%\end{lemma}
%
%\begin{lemma}\label{lemma1}
%Пусть $k=0, \ldots, N-2$, тогда для любых $t,v \in [-1,1]$, тогда имеет место
%\begin{equation*}
%    \left(
%    t-v
%    \right)\,
%   \sum_{l=0}^{k}
%    \frac{(N-2)^{[l]}}{(N+l+2)^{[l]}} \, \frac{(l+2)(l+3)}{(l+1)}\,
%    Q^{1,2}_{l}(t) \,Q^{2,1}_{l}\left(v +\frac{2}{N-1} \right)
%    =
%\end{equation*}
%\begin{equation*}
%   \frac{(N-2)^{[k+1]}}{(N+k+2)^{[k]}}  \,
%   \frac{(k+3)(k+4)}{(2k+5)(N-1)}
%   \left[
%        Q^{1,2}_{k+1}(t) \,Q^{2,1}_{k}\left(v +\frac{2}{N-1} \right) -  Q^{1,2}_{k}(t) \, Q^{2,1}_{k+1} \left(v +\frac{2}{N-1} \right)
%   \right]
%\end{equation*}
%\begin{equation}\label{lem1eq}
%   +
%   \sum_{l=0}^{k} \lambda_{l,N} \,
%    \frac{(N-2)^{[l]}}{(N+l+2)^{[l]}} \, \frac{(l+2)(l+3)}{(l+1)}\,
%    Q^{1,2}_{l}(t) \,Q^{2,1}_{l}\left(v +\frac{2}{N-1} \right),
%\end{equation}
%где $\lambda_{l,N} = \frac{6}{(2l+3)(2l+5)}+\frac{q(l)}{N-1}$, $|q(l)|<c$.
%\end{lemma}
%
%\begin{lemma}\label{lemma2}
%Для любых $t \in [0, 1]$ и $v \in [-1,1]$ при $k=0, \ldots, N-2$ справедлива оценка
%\begin{equation*}
%\frac{(N-2)^{[k+1]}}{(N+k+2)^{[k]}}  \,
%   \frac{(k+3)(k+4)}{(2k+5)(N-1)}
%   \left|
%        Q^{1,2}_{k+1}(t) \,Q^{2,1}_{k}\left(v +\frac{2}{N-1} \right) -
%        \right.
%\end{equation*}
%\begin{equation*}
%\left.
%        Q^{1,2}_{k}(t) \, Q^{2,1}_{k+1} \left(v +\frac{2}{N-1} \right)
%   \right|
%   \leq
%\end{equation*}
%\begin{equation}\label{lem2eq}
% c(a) \, (1-t)^{-1/4} (1-v)^{-3/4}(1+v)^{-3/4} \left[
%    (1-t)^{-1/2} +
%    (1-v)^{-1/2}
%    \right].
%\end{equation}
%\end{lemma}



\section{Средние Валле -- Пуссена для предельного ряда}

Аналогично классическому случаю введем в рассмотрение усреднение частичных сумм предельного ряда \eqref{limitS} вида
\begin{equation}\label{limitVP}
  \mathcal{V}^{-1}_{m,n}(f,x) = \frac{S_{m,N}^{-1}(f,x) + S_{m+1,N}^{-1}(f,x) + \ldots + S_{m+n,N}^{-1}(f,x)}{n+1},
\end{equation}
которое назовем \textit{средними Валле -- Пуссена для частичных сумм предельного ряда}.


Эти новые операторы могут быть записаны с помощью следующего выражения
$$
\mathcal{V}_{m,n,N}^{-1}(f,x)=
a_f(x) + { 8x(N-1-x)\over N(N-1)}
\left[
\sum_{k=0}^{m-2}\hat g_k\tau_{k,N-2}^{1,1}(x-1) + \right.
$$
\begin{equation*}
\left.
\sum_{k=m-1}^{m+n-2} \frac{m+n-k+1}{n+1} \hat g_k \tau_{k,N-2}^{1,1}(x-1)
\right],
\end{equation*}
откуда
$$
\mathcal{V}_{m,n,N}^{-1}(f,x)=
S_{m+n,N}^{-1}(f,x) -
{ 8x(N-1-x)\over N(N-1)}
\sum_{k=m-1}^{m+n-2} \frac{k-m}{n+1} \hat g_k \tau_{k,N-2}^{1,1}(x-1).
$$


Используя представление \eqref{Sn_jay}, можем переписать
\begin{equation*}
  \mathcal{V}^{-1}_{m,n}(f,x) = a_N(f,x) + %\tilde{b}_{n,N}
  \frac{8x(N-x-1)}{N(N-1)(N-2)(n+1)}
   \sum_{j=1}^{N-2} g(j) \left[ J_{m-2,N-2}(x-1,j-1) + \right.
\end{equation*}
\begin{equation*}
  \left.\ldots + J_{m+n-2,N-2}(x-1,j-1) \right] = a_N(f,x) +
\end{equation*}
\begin{equation*}
   \frac{8x(N-x-1)}{N(N-1)(N-2)(n+1)}
   \sum_{j=1}^{N-2} g(j) \left[ I_{m+n-2,N-2}(x-1,j-1) -  \right.
\end{equation*}
\begin{equation} \label{limVPf}
 \left. I_{m+n-2,N-2}(x-1,j-1) - I_{m-2,N-2}(x-1,j-1) \right],
\end{equation}
где
\begin{equation}\label{I}
  I_{k,N}(x,j) = \sum_{l=0}^{k} J_{l,N}(x,j).
\end{equation}

Используя формулу Кристоффеля -- Дарбу \eqref{sob-tcheb-difference-3.9}, выводим:

\begin{equation*}
  J_{l,N} (x,j) = D^{1}_{l} (x,j) =
  \sum_{r=0}^{l} \frac{T^{1}_{r}(x) T^{1}_{r}(j)}{h^{1}_{r,N}} =
\end{equation*}
\begin{equation}\label{krist}
  \frac{(N-1)^{[l+1]}}{(N+l+2)^{[l]}} \, \frac{l+3}{16}\,
  \frac{T_{l+1}^1(x) T_{l}^1(j) - T_{l+1}^1(j) T_{l}^1(x)}{x-j}.
\end{equation}

Отсюда и из определения \eqref{I} записываем%, обозначив для краткости $B_{l,N} = \frac{(N-1)^{[l+1]}}{(N+l+2)^{[l]}}\,\frac{l+3}{16}$,
\begin{equation*}
  I_{k,N}(x,j) = %\sum_{l=0}^{k} J_{l,N}(x,j) =
  \sum_{l=0}^{k} \frac{(N-1)^{[l+1]}}{(N+l+2)^{[l]}}\,\frac{l+3}{16}\,
  \frac{T_{l+1}^1(x) T_{l}^1(j) - T_{l+1}^1(j) T_{l}^1(x)}{x-j}.
\end{equation*}



Нетрудно показать, что операторы $\mathcal{V}_{m,n,N}^{-1}(f,x)$ обладают свойством совпадения с исходной функцией в концевых узлах сетки
$$
\mathcal{V}_{m,n,N}^{-1}(f,0)= f(0), \quad \mathcal{V}_{m,n,N}^{-1}(f,N-1)= f(N-1).
$$
а также свойством проективности над пространством $H_{m}$ алгебраических полиномов степени не выше $m$: $\mathcal{V}_{m,n,N}^{-1}(P_m,x)\equiv P_m(x)$.

Остается открытой задача исследования аппроксимативных свойств операторов \linebreak $\mathcal{V}_{m,n,N}^{-1}(f,x)$, которая в свою очередь сводится к задаче исследования нормы оператора в пространстве $C[0, N-1]$:
$\| \mathcal{V}^{-1}_{m,n} \| = \sup_{\| f \| \leq 1} \| \mathcal{V}^{-1}_{m,n}(f,x) \|$.

В отчетном году на пути к решению этой задачи было доказано несколько вспомогательных утверждений и разработан специализированный пакет прикладных программ, с помощью которого проведена серия численных экспериментов по исследованию поведения указанного оператора на дискретных функциях разных типов. В следующем году планируется получение окончательных теоретических результатов по этой теме.



%Рассмотрим норму оператора $\mathcal{V}^{-1}_{m,n}$, действующего  в пространстве $C[0, N-1]$:
%\begin{equation}\label{norm}
%  \| \mathcal{V}^{-1}_{m,n} \| = \sup_{\| f \| \leq 1} \| \mathcal{V}^{-1}_{m,n}(f,x) \|.
%\end{equation}
%Заметим, что $\max_{x \in [0,N-1]} |g(x)| \leq 1$ при $\| f(x) \| \leq 1$, а также что $\| a_N(f,x) \| = 1$. Тогда
%\begin{equation}\label{norm2}
%  \| \mathcal{V}^{-1}_{m,n} \| = 1 + \| {V}^{-1}_{m,n}(x) \|,
%\end{equation}
%где
%\begin{equation*}
%   V^{-1}_{m,n}  = \frac{8x(N-x-1)}{N(N-1)(N-2)(n+1)}
%   \sum_{j=1}^{N-2} \left| I_{m+n-2,N-2}(x-1,j-1) - \right.
%\end{equation*}
%\begin{equation}\label{norm2}
%   \left. I_{m-2,N-2}(x-1,j-1) \right|.
%\end{equation}

%Для дальнейшего нам понадобится преобразовать некоторые выражения для $J_{l,N}(x,j)$ и $I_{k,N}(x,j)$.
%Используя формулу Кристоффеля -- Дарбу \eqref{sob-tcheb-difference-3.9}, выводим:
%
%\begin{equation*}
%  J_{l,N} (x,j) = D^{1}_{l} (x,j) =
%  \sum_{r=0}^{l} \frac{T^{1}_{r}(x) T^{1}_{r}(j)}{h^{1}_{r,N}} =
%\end{equation*}
%\begin{equation}\label{krist}
%  \frac{(N-1)^{[l+1]}}{(N+l+2)^{[l]}} \, \frac{l+3}{16}\,
%  \frac{T_{l+1}^1(x) T_{l}^1(j) - T_{l+1}^1(j) T_{l}^1(x)}{x-j}.
%\end{equation}
%
%Отсюда и из определения \eqref{I} записываем%, обозначив для краткости $B_{l,N} = \frac{(N-1)^{[l+1]}}{(N+l+2)^{[l]}}\,\frac{l+3}{16}$,
%\begin{equation*}
%  I_{k,N}(x,j) = %\sum_{l=0}^{k} J_{l,N}(x,j) =
%  \sum_{l=0}^{k} \frac{(N-1)^{[l+1]}}{(N+l+2)^{[l]}}\,\frac{l+3}{16}\,
%  \frac{T_{l+1}^1(x) T_{l}^1(j) - T_{l+1}^1(j) T_{l}^1(x)}{x-j}.
%\end{equation*}

%\begin{equation*}
%  \hat{I}_{k,N}(t,v) = I_{k,N}\left( \frac{N-1}{2}(1+t) , \frac{N-1}{2}(1+v) \right)=
%\end{equation*}
%\begin{equation*}
%  \sum_{l=0}^{k} B_{l,N} \,
%  \left[
%    \frac{(1+t)(1-v)}{t-v}\,Q^{1,2}_{l}(t)\,Q^{2,1}_{l}\left(v+\frac{2}{N-1}\right)
%  \right.
%\end{equation*}
%\begin{equation}\label{Usum}
%\left.
%  - \frac{(1+v)(1-t)}{t-v}\,Q^{1,2}_{l}(v)\,Q^{2,1}_{l}\left(t+\frac{2}{N-1}\right)
%\right] = U_{k,N}(t,v) + U_{k,N}(v,t),
%\end{equation}
%
%где
%\begin{equation}\label{UkN}
%  U_{k,N}(t,v)=
%  \frac{(1+t)(1-v)}{t-v}\,
%  \sum_{l=0}^{k} B_{l,N} \,
%    Q^{1,2}_{l}(t)\,Q^{2,1}_{l}\left(v+\frac{2}{N-1}\right).
%\end{equation}
%С помощью Леммы \eqref{lemma1} мы можем преобразовать последнее выражение к виду
%
%\begin{equation*}
%U_{k,N}(t,v)=
%\frac{(1+t)(1-v)}{16 (t-v)^2}\,
%\left\{
%   \frac{(N-2)^{[k+1]}}{(N+k+2)^{[k]}}  \,
%   \frac{(k+3)(k+4)}{(2k+5)(N-1)}
%\right.
%\end{equation*}
%\begin{equation*}
%   \left[
%        Q^{1,2}_{k+1}(t) \,Q^{2,1}_{k}\left(v +\frac{2}{N-1} \right) -  Q^{1,2}_{k}(t) \, Q^{2,1}_{k+1} \left(v +\frac{2}{N-1} \right)
%   \right]
%\end{equation*}
%\begin{equation*}
%\left.
%   +
%   \sum_{l=0}^{k} \lambda_{l,N}\,
%    \frac{(N-2)^{[l]}}{(N+l+2)^{[l]}} \, \frac{(l+2)(l+3)}{(l+1)}\,
%    Q^{1,2}_{l}(t) \,Q^{2,1}_{l}\left(v +\frac{2}{N-1} \right)
%\right\}=
%\end{equation*}
%\begin{equation*}
%\frac{(1+t)(1-v)}{16 (t-v)^2}\,\left\{
%    X^{1}_{k,N}(t,v) + X^{2}_{k,N}(t,v)
%\right\}.
%\end{equation*}
%Таким образом, приходим к следующей схеме оценивания выражения \eqref{limVPf}:
%
%Заметим теперь, что для $X^{1}_{k,N}(t,v)$ справедлива оценка \eqref{lem2eq} из Леммы \eqref{lemma2}, из которой следует
%\begin{equation*}
% \left|
% X^{1}_{m+n-2,N}(t,v) - X^{1}_{m-2,N}(t,v)
% \right| \leq
%\end{equation*}
%\begin{equation}\label{x1mn}
% c(a) \, (1-t)^{-1/4} (1-v)^{-3/4}(1+v)^{-3/4} \left[
%    (1-t)^{-1/2} +
%    (1-v)^{-1/2}
%    \right].
%\end{equation}
%Оценим кроме того разность
%\begin{equation*}
%  \left| X^{2}_{m+n-2,N}(t,v) - X^{2}_{m-2,N}(t,v) \right| \leq
%\end{equation*}
%\begin{equation*}
%   \sum_{l=m-2}^{m+n-2}  \lambda_{l,N}\,
%    \frac{(N-2)^{[l]}}{(N+l+2)^{[l]}} \, \frac{(l+2)(l+3)}{(l+1)}\,
%    \left| Q^{1,2}_{l}(t) \right|\,\left| Q^{2,1}_{l}\left(v +\frac{2}{N-1} \right)\right|
%\leq
%\end{equation*}
%\begin{equation*}
%   c(a)\sum_{l=m-2}^{m+n-2}  \lambda_{l,N}\,
%    \frac{(N-2)^{[l]}}{(N+l+2)^{[l]}} \, \frac{(l+2)(l+3)}{l(l+1)}\,
%    \left(  \sqrt{1-t} + \frac1l \right)^{-3/2}
%    %\left(  \sqrt{1+t} + \frac1l \right)^{-5/2}
%    \left(  \sqrt{1-v} + \frac1l \right)^{-5/2}
%    \left(  \sqrt{1+v} + \frac1l \right)^{-3/2}
%\leq
%\end{equation*}
%\begin{equation*}
%\left(  \sqrt{1-t} + \frac{1}{n+m} \right)^{-3/2}
%%\left( 1-t\right)^{-3/4}
%    \left(1-v\right)^{-5/4}
%    \left(1+v \right)^{-3/4}
%\sum_{l=m-2}^{m+n-2}
%    \left[ \frac{ c(a)}{(2l+3)(2l+5)}+\frac{q(l)}{N-1}\right]
%\leq
%\end{equation*}
%\begin{equation*}
%    c(a)
%    \left(  \sqrt{1-t} + \frac{1}{n+m} \right)^{-3/2}
%    %\left( 1-t\right)^{-3/4}
%    \left(1-v\right)^{-5/4}
%    \left(1+v \right)^{-3/4}
%\sum_{l=m-2}^{m+n-2}
%    \left[ \frac{ 1}{(2l+3)(2l+5)}+\frac{1}{N-1}\right]
%   =
%\end{equation*}
%\begin{equation*}
%    c(a)
%    \left(  \sqrt{1-t} + \frac{1}{n+m} \right)^{-3/2}
%    %\left( 1-t\right)^{-3/4}
%    \left(1-v\right)^{-5/4}
%    \left(1+v \right)^{-3/4}
%    \left[ \frac{ n+1}{4m^2+4mn-2n-1}+\frac{n+1}{N-1}\right].
%\end{equation*}
%Пользуясь теперь тем что $dn \leq m \leq bn$ и $n \leq a\sqrt{N}$, выводим
%\begin{equation*}
%  \left| X^{2}_{m+n-2,N}(t,v) - X^{2}_{m-2,N}(t,v) \right| \leq
%    \frac{c(a)}{n}
%    \left(  \sqrt{1-t} + \frac{1}{n} \right)^{-1-1/2}
%    %\left( 1-t\right)^{-3/4}
%    \left(1-v\right)^{-5/4}
%    \left(1+v \right)^{-3/4}
%    \leq
%\end{equation*}
%\begin{equation}\label{x2mn}
%    c(a)
%    \left(  \sqrt{1-t} + \frac{1}{n} \right)^{-1/2}
%    \left(1-v\right)^{-5/4}
%    \left(1+v \right)^{-3/4}
%    \leq
%    c(a)
%    \left( 1-t\right)^{-1/4}
%    \left(1-v\right)^{-5/4}
%    \left(1+v \right)^{-3/4}.
%\end{equation}
%Объединяя оценки \eqref{x1mn} -- \eqref{x2mn}, получим
%\begin{equation*}
%\left| U_{m+n-2,N}(t,v) - U_{m-2,N}(t,v) \right| \leq
%\frac{(1+t)(1-v)}{16 (t-v)^2}\,
%\left(
%   \left| X^{1}_{m+n-2,N}(t,v) - X^{1}_{m-2,N}(t,v) \right| +
%\right.
%\end{equation*}
%\begin{equation*}
%\left.
%   \left| X^{2}_{m+n-2,N}(t,v) - X^{2}_{m-2,N}(t,v) \right|
%\right) \leq
%\end{equation*}
%\begin{equation*}
%c(a)\frac{(1+t)(1-v)}{(t-v)^2}\,
%\left\{
%(1-t)^{-1/4} (1-v)^{-3/4}(1+v)^{-3/4} \left[
%    (1-t)^{-1/2} +
%    (1-v)^{-1/2}
%    \right]
%+ \right.
%\end{equation*}
%\begin{equation*}
%\left.
%    \left( 1-t\right)^{-1/4}
%    \left(1-v\right)^{-5/4}
%    \left(1+v \right)^{-3/4}
%\right\} \leq
%\end{equation*}
%\begin{equation*}
%\frac{c(a)(1+t)}{(t-v)^2}\,
%\left\{
%    (1-t)^{-1/4} (1-v)^{1/4}(1+v)^{-3/4} \left[
%    (1-t)^{-1/2} +
%    (1-v)^{-1/2}
%    \right]
%+ \right.
%\end{equation*}
%\begin{equation*}
%\left.
%    \left(1-t\right)^{-1/4}
%    \left(1-v\right)^{-1/4}
%    \left(1+v \right)^{-3/4}
%\right\} \leq
%\end{equation*}
%\begin{equation*}
%\frac{c(a)(1-t)^{-1/4}(1-v)^{1/4}(1+v)^{-3/4} }{(t-v)^2}\,
%\left\{
%    (1-t)^{-1/2} +    (1-v)^{-1/2}
%\right\}.
%\end{equation*}
%
%Рассмотрим
%\begin{equation*}
%\frac{8x(N-x-1)}{N(N-1)(N-2)(n+1)} \left| U_{m+n-2,N}(t,v) - U_{m-2,N}(t,v) \right| \leq
%\end{equation*}
%\begin{equation*}
%\frac{8x(N-x-1)}{N(N-1)(N-2)(n+1)}
%\frac{c(a)(1-t)^{-1/4}(1-v)^{1/4}(1+v)^{-3/4} }{(t-v)^2}\,
%\left\{
%    (1-t)^{-1/2} +    (1-v)^{-1/2}
%\right\}
%\leq
%\end{equation*}
%\begin{equation*}
%(1+t)(1-t)
%\frac{c(a)(1-t)^{-1/4}(1-v)^{1/4}(1+v)^{-3/4} }{nN(t-v)^2}\,
%\left\{
%    (1-t)^{-1/2} +    (1-v)^{-1/2}
%\right\}
%\leq
%\end{equation*}
%\begin{equation*}
%\frac{c(a)(1+t)(1-t)^{3/4}(1-v)^{1/4}(1+v)^{-3/4} }{nN(t-v)^2}\,
%\left\{
%    (1-t)^{-1/2} +    (1-v)^{-1/2}
%\right\}
%\leq
%\end{equation*}
%\begin{equation*}
%\frac{c(a)(1+t)(1+v)^{-3/4} }{nN(t-v)^2}\,
%\left\{
%    (1-t)^{1/4}(1-v)^{1/4}+   (1-t)^{3/4}(1-v)^{-1/4}
%\right\}
%\leq
%\end{equation*}
%
