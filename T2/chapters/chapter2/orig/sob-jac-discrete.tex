\chapter{Аппроксимативные свойства рядов Фурье по полиномам, ортогональным по Соболеву и порожденным полиномами Якоби}
%\begin{abstract}
%Рассмотрены ряды Фурье по полиномам Якоби $P_k^{\alpha-r,-r}(x)$ $(k=r,r+1,\ldots)$, ортогональным относительно скалярного произведения типа Соболева следующего вида
%\begin{equation*}
%<f,g>=\sum_{\nu=0}^{r-1}f^{(\nu)}(-1)g^{(\nu)}(-1)+\int_{-1}^{1} f^{(r)}(t)g^{(r)}(t)(1-t)^\alpha dt.
%\end{equation*}
%Показано, что такие ряды представляют собой частный случай смешанных рядов по полиномам Якоби $P_k^{\alpha,\beta}(x)$ $(k=0,1,\ldots)$, рассмотренным автором ранее.  Исследованы вопросы сходимости смешанных  рядов по общим полиномам Якоби и их аппроксимативные свойства.  Полученные результаты применяются к исследованию аппроксимативных свойств рядов Фурье по полиномам Якоби $P_k^{\alpha-r,-r}(x)$, ортогональным по Соболеву.
%\end{abstract}


%\section{Введение}
Пусть $-1<\alpha$ -- нецелое, $\rho=\rho(x)=\rho_\alpha(x)=(1-x)^\alpha$,  $L_2^{\alpha,0}$ --  весовое пространство Лебега, состоящее
из измеримых на $[-1,1]$ функций $f=f(x)$, для которых $\int_{-1}^1f^2(x)\rho_\alpha(x)dx<\infty$. Для целого  $r\ge1$ через $W^r_{L_2^{\alpha,0}}$ обозначим пространство функций $f=f(x)$, непрерывно дифференцируемых $r-1$-раз, причем $f^{(r-1)}(x)$ абсолютно непрерывна на $[-1,1]$, а $f^{(r)}(x)\in L_2^{\alpha,0} $. В пространстве  $W^r_{L_2^{\alpha,0}}$ можно ввести скалярное произведение следующего вида
\begin{equation}\label{sob-jac-discrete-eq3.1}
<f,g>=\sum_{\nu=0}^{r-1}f^{(\nu)}(-1)g^{(\nu)}(-1)+\int_{-1}^{1} f^{(r)}(t)g^{(r)}(t)(1-t)^\alpha dt.
\end{equation}
Следствие \ref{completeness-jacobi} показывает, что произвольная  функция $f\in W^r_{L_2^{\alpha,0}}$ может быть представлена в виде ряда Фурье \eqref{sob-leg-3.50}, который сходится в метрике пространства $W^r_{L_2^{\alpha,0}}$, определяемой скалярным произведением \eqref{sob-jac-discrete-eq3.1}.
 Другими словами, если для $f\in W^r_{L_2^{\alpha,0}}$ мы положим $\|f\|=\sqrt{<f,f>}$, то $\|S_n(f)-f\|\to 0$ при $n\to\infty$, где
\begin{equation}\label{sob-jac-discrete-3.10}
S_m^\alpha(f)=S_m^\alpha(f,x)=
 \sum_{k=0}^{r-1}\hat f_k\frac{(1+x)^k}{k!}+ \sum_{k=r}^{m} \frac{2^r\hat f_kP_{k}^{\alpha-r,-r}(x)}{\sqrt{h_{k-r}^{\alpha,0}}(k+\alpha-r)^{[r]}},
     \end{equation}
 частичная сумма ряда Фурье \eqref{sob-leg-3.50}.  Однако отсюда не следует, что $|S_m^\alpha(f,x)-f(x)|\to 0$ при $n\to\infty$ в заданной точке $x\in[-1,1]$ и, тем более,  мы не можем утверждать, что $|S_m^\alpha(f,x)-f(x)|\to 0$ при $m\to\infty$ равномерно относительно  $x\in[-1,1]$. С другой стороны, ряд Фурье  \eqref{sob-leg-3.50} является, как  это будет показано в следующем параграфе, частным случаем смешанного ряда по общим полиномам Якоби, рассмотренным в работах \cite{sob-jac-discrete-Shar13},   \cite{sob-jac-discrete-Shar18},  в которых, в частности, исследованы  вопросы поточечной и равномерной сходимости. В качестве следствия результатов, установленных, например,  в \cite{sob-jac-discrete-Shar18}, мы получим достаточные условия поточечной и равномерной сходимости рядов Фурье \eqref{sob-leg-3.50}. Следует отметить, что в работах \cite{sob-jac-discrete-Shar11} -- \cite{sob-jac-discrete-Shar17} детально исследованы также аппроксимативные свойства  частичных сумм смешанных рядов по ультрасферическим полиномам Якоби $P_n^{\alpha,\alpha}(x)$. Некоторые  методы и подходы, разработанные в работах  \cite{sob-jac-discrete-Shar11} -- \cite{sob-jac-discrete-Shar17} в ультрасферическом случае, в настоящей работе будут использованы при исследовании аппроксимативных свойств частичных сумм смешанных рядов по общим полиномам Якоби  $P_n^{\alpha,\beta}(x)$ с $\alpha,\beta>-1$  для функций $f\in W^r_{L_2^{\alpha,\beta}}$  и, как следствие, будут получены соответствующие результаты для рядов Фурье  \eqref{sob-leg-3.50}.
 \section{Смешанные ряды по общим полиномам Якоби}

 Следуя обозначениям из работ \cite{sob-jac-discrete-Shar13},   \cite{sob-jac-discrete-Shar18}, напомним сначала определение смешанных рядов по общим полиномам Якоби
 $P_n^{\alpha,\beta}(x)$ с параметрами $\alpha,\beta$, удовлетворяющими условию
$-1<\alpha,\beta<1$.  Пусть целое $r\ge1$, функция $f=f(x)$ непрерывно-дифференцируема
$r-1$ раз на $[-1,1]$, $f^{(r-1)}(x)$ абсолютно непрерывна и

\begin{equation}\label{sob-jac-discrete-4.1}
\int\limits^1_{-1}(1-x)^\alpha(1+x)^\beta\left|f^{(r)}(x)\right|\,dx<\infty.
\end{equation}

Тогда мы можем определить коэффициенты
\begin{equation}\label{sob-jac-discrete-4.2}
f^{\alpha,\beta}_{r,k}=\frac{1}{h^{\alpha,\beta}_k}\int\limits^1_{-1}(1-t)^\alpha(1+t)^\beta f^{(r)}(t)P_k^{\alpha,\beta}(t)\,dt
\end{equation}
и ряд Фурье-Якоби
\begin{equation}\label{sob-jac-discrete-4.3}
f^{(r)}(x)\sim\sum^\infty_{k=0}f^{\alpha,\beta}_{r,k}P_k^{\alpha,\beta}(x).
\end{equation}
для функции $g(x)=f^{(r)}(x)$. Далее, применяя для функции $f(x)$ формулу Тейлора с остатком в интегральной форме и
выполняя в ней формальную подстановку вместо
$f^{(r)}(t)$ ряда Фурье-Якоби \eqref{sob-jac-discrete-4.3}, придем к следующему равенству
\begin{equation}\label{sob-jac-discrete-4.4}
f(x)=Q_{r-1}(f,x)+\frac{1}{(r-1)!}
\sum^\infty_{k=0}f^{\alpha,\beta}_{r,k}
\int\limits^x_{-1}(x-t)^{r-1}P_k^{\alpha,\beta}(t)dt,
\end{equation}

Пусть $\lambda=\alpha+\beta$. Тогда если $(k+\lambda)^{[r]}\neq0$,
то мы можем воспользоваться равенством \eqref{Haar-Tcheb-eq5.5} и записать
\begin{equation}\label{sob-jac-discrete-4.5}
P^{\alpha,\beta}_k(t)={2^r \over (k+\lambda )^{[r]}}\frac{d^r}{dt^r}P_{k+r}^{\alpha-r,\beta-r}(t).
\end{equation}


Заметим, что если $\lambda\notin\{-1,0,1\}$, то при
$-1<\alpha,\beta<1$ равенство \eqref{sob-jac-discrete-4.5} справедливо при всех
$k=0,1,\ldots$. Если же $\lambda\in\{-1,0,1\}$, то
$(k+\lambda)^{[r]}\neq0$ при $k\ge r-\lambda$ и для таких $k$ мы
можем снова воспользоваться равенством \eqref{sob-jac-discrete-4.5}.

\subsection{Случай $\lambda \notin \{-1,0,1\}$.}

При $\lambda \notin \{-1,0,1\}$, пользуясь равенством \eqref{sob-jac-discrete-4.5} имеем
для любого $k=0,1,\ldots$
$$
\frac{1}{(r-1)!}\int\limits^x_{-1}(x-t)^{r-1}P_k^{\alpha,\beta}(t)\,dt=
\frac{2^r}{(k+\lambda)^{[r]}}\frac{1}{(r-1)!}\int\limits^x_{-1}(x-t)^{r-1}
\frac{d^r}{dt^r}P_{k+r}^{\alpha-r,\beta-r}(t)\,dt=
$$
\begin{equation}\label{sob-jac-discrete-4.6}
\frac{2^r}{(k+\lambda)^{[r]}}\left[P_{k+r}^{\alpha-r,\beta-r}(x)-\sum^{r-1}_{\nu=0}
\frac{(1+x)^\nu}{\nu!}\left\{P_{k+r}^{\alpha-r,\beta-r}(t)
\right\}_{t=-1}^{(\nu)}\right].
\end{equation}
 Далее в силу \eqref{Haar-Tcheb-eq5.5}
 \begin{equation}\label{sob-jac-discrete-4.7}
\left\{P_{k+r}^{\alpha-r,\beta-r}(t)\right\}^{(\nu)}=\frac{(k+\lambda-r+1)_\nu}{2^\nu}P_{k+r-\nu}^{\alpha+\nu-r,\beta+\nu-r}(t),
\end{equation}
а из \eqref{Haar-Tcheb-eq5.9} имеем
$$P_{k+r-\nu}^{\alpha+\nu-r,\beta+\nu-r}(-1)=(-1)^{k+r-\nu}{k+\beta\choose k+r-\nu}=$$
\begin{equation}\label{sob-jac-discrete-4.8}
\frac{(-1)^{k+r-\nu}\Gamma(k+\beta+1)}{\Gamma(\nu-r+\beta+1)(k+r-\nu)!}.
\end{equation}
Из \eqref{sob-jac-discrete-4.7}  и \eqref{sob-jac-discrete-4.8} находим
\begin{equation}\label{sob-jac-discrete-4.9}
\left\{P_{k+r}^{\alpha-r,\beta-r}(t)\right\}_{t=-1}^{(\nu)}=
\frac{(-1)^{k+r-\nu}\Gamma(k+\beta+1)(k+\lambda-r+1)_{\nu}}{\Gamma(\nu-r+\beta+1)(k+r-\nu)!2^\nu}=A_{\nu,k}.
\end{equation}
Сопоставляя \eqref{sob-jac-discrete-4.6} и \eqref{sob-jac-discrete-4.9} мы можем записать
$$\frac{1}{(r-1)!}\int\limits^x_{-1}(x-t)^{r-1}P_k^{\alpha,\beta}(t)\,dt=$$
\begin{equation}\label{sob-jac-discrete-4.10}
\left[P_{k+r}^{\alpha-r,\beta-r}(x)-\sum^{r-1}_{\nu=0}\frac{A_{\nu,k}}{\nu!}(1+x)^{\nu}\right]\frac{2^r}{(k+\lambda)^{[r]}}.
\end{equation}
Из \eqref{sob-jac-discrete-4.4} и \eqref{sob-jac-discrete-4.10}  имеем
\begin{equation}\label{sob-jac-discrete-4.11}
f(x)=Q_{r-1}(f,x)+
2^r\sum_{k=0}^\infty\frac{f^{\alpha,\beta}_{r,k}}{(k+\lambda)^{[r]}}\left[P_{k+r}^{\alpha-r,\beta-r}(x)-
\sum^{r-1}_{\nu=0}\frac{A_{\nu,k}}{\nu!}(1+x)^{\nu}\right].
\end{equation}
 Правую часть \eqref{sob-jac-discrete-4.11} можно переписать так
\begin{equation}\label{sob-jac-discrete-4.12}
f(x)=D^{\alpha,\beta}_{r-1}(f,x)+2^r\sum_{k=0}^\infty\frac{f^{\alpha,\beta}_{r,k}}{(k+\lambda)^{[r]}}P_{k+r}^{\alpha-r,\beta-r}(x),
\end{equation}
где
\begin{equation}\label{sob-jac-discrete-4.13}
D^{\alpha,\beta}_{r-1}(f,x)=\sum^{r-1}_{\nu=0}\frac{f^{(\nu)}(-1)-A_\nu}{\nu!}(1+x)^\nu,
\end{equation}
\begin{equation}\label{sob-jac-discrete-4.14}
A_\nu=\sum_{k=0}^\infty {2^rA_{\nu,k}\over(k+\lambda)^{[r]}}f_{r,k}^{\alpha,\beta}.
\end{equation}

\subsection{Случай $\lambda\in\{-1,0,1\}, \ -1<\alpha, \beta<1 $}

В случае, когда $\lambda\in\{-1,0,1\}$ выражение $(k+\lambda)^{[r]}$
обращается в нуль при $k+\lambda\le r-1$ и для таких значений
формула \eqref{sob-jac-discrete-4.5}  не пригодна. Но если $k\ge r-\lambda$, то
$(k+\lambda)^{[r]}>0$ и мы снова можем воспользоваться равенством
\eqref{sob-jac-discrete-4.5}. Имея в виду это свойство полиномов Якоби, перепишем равенство
\eqref{sob-jac-discrete-4.4} следующим образом
\begin{equation}\label{sob-jac-discrete-4.15}
f(x)=Q_{r-1}(x)+B^{\alpha,\beta}_{2r-1-\lambda}
(f,x)+G^{\alpha,\beta}_r(f,x),
\end{equation}
где
\begin{equation}\label{sob-jac-discrete-4.16}
B^{\alpha,\beta}_{2r-1-\lambda}(f,x)=
\frac{1}{(r-1)!}\sum_{k=0}^{r-1-\lambda}f^{\alpha,\beta}_{r,k}
\int\limits^x_{-1}(x-t)^{r-1}P_k^{\alpha,\beta}(t)\,dt,
\end{equation}
\begin{equation}\label{sob-jac-discrete-4.17}
G^{\alpha,\beta}_r(f,x)=\frac{1}{(r-1)!}
\sum_{k=r-\lambda}^\infty
f^{\alpha,\beta}_{r,k}\int\limits^x_{-1}(x-t)^{r-1}
P_k^{\alpha,\beta}(t)\,dt.
\end{equation}
 Как уже было отмечено, для
$k\ge r-\lambda$ мы можем воспользоваться равенством \eqref{sob-jac-discrete-4.5} и, как
следствие, равенством \eqref{sob-jac-discrete-4.10}. Это дает
\begin{equation}\label{sob-jac-discrete-4.18}
G^{\alpha,\beta}_r(x)=
2^r\sum_{k=r-\lambda}^\infty\frac{f^{\alpha,\beta}_{r,k}}
{(k+\lambda)^{[r]}}\left[P_{k+r}^{\alpha-r,\beta-r}(x)
-\sum^{r-1}_{\nu=0}\frac{A_{\nu,k}}{\nu!}(1+x)^{\nu}\right].
\end{equation}
Пусть
\begin{equation}\label{sob-jac-discrete-4.19}
E^{\alpha,\beta}_{r-1}(f,x)=-\sum^{r-1}_{\nu=0}\frac{g_\nu}{\nu!}(1+x)^\nu,
\end{equation}
где
\begin{equation}\label{sob-jac-discrete-4.20}
g_\nu=\sum^{\infty}_{k=r-\lambda}\frac{2^rA_{\nu,k}}
{(k+\lambda)^{[r]}}f^{\alpha,\beta}_{r,k}.
\end{equation}
Из \eqref{sob-jac-discrete-4.16}, \eqref{sob-jac-discrete-4.19} следует, что
$$\overline{D}^{\alpha,\beta}_{2r-1-\lambda}(f,x)=Q_{r-1}(f,x)+
E^{\alpha,\beta}_{r-1}(f,x)+B^{\alpha,\beta}_{2r-1-\lambda}(f,x)=$$
\begin{equation}\label{sob-jac-discrete-4.21}
\sum^{r-1}_{\nu=0}\frac{f^{(\nu)}(-1)-g_\nu}{\nu!}(1+x)^\nu+\frac{1}{(r-1)!}
\sum_{k=0}^{r-1-\lambda}f^{\alpha,\beta}_{r,k}\int\limits^x_{-1}(x-t)^{r-1}P_k^{\alpha,\beta}(t)\,dt.
\end{equation}
-- алгебраический полином степени $2r-1-\lambda$. Сопоставляя
\eqref{sob-jac-discrete-4.15}, \eqref{sob-jac-discrete-4.18} и \eqref{sob-jac-discrete-4.21} мы приходим для $\lambda\in\{-1,0,1\}$ к
следующему равенству

\begin{equation}\label{sob-jac-discrete-4.22}
f(x)=\overline{D}^{\alpha,\beta}_{2r-1-\lambda}(f,x)+2^r\sum_{k=r-\lambda}^\infty
\frac{f^{\alpha,\beta}_{r,k}}{(k+\lambda)^{[r]}}P_{k+r}^{\alpha-r,\beta-r}(x).
\end{equation}
Правые части равенств \eqref{sob-jac-discrete-4.12} и \eqref{sob-jac-discrete-4.22} мы будем называть смешанными
рядами по полиномам Якоби $P_n^{\alpha,\beta}(x)$ или смешанными
рядами, ассоциированными с полиномами Якоби $P_n^{\alpha,\beta}(x)$.

Рассмотрим один интересный частный случай смешанного ряда \eqref{sob-jac-discrete-4.12}. А именно, пусть $\beta=0$, $-1<\alpha$ -- нецелое. Тогда из \eqref{sob-jac-discrete-4.9} следует, что $A_{\nu,k}=0$ $(0\le \nu \le r-1, k\ge0)$, поэтому в силу \eqref{sob-jac-discrete-4.14} $A_\nu=0$ $(0\le \nu \le r-1)$. Поэтому, обратившись  к выражению \eqref{sob-jac-discrete-4.13} заметим, что
\begin{equation}\label{sob-jac-discrete-4.23}
D^{\alpha,\beta}_{r-1}(f,x)=\sum^{r-1}_{\nu=0}f^{(\nu)}(-1)\frac{(1+x)^\nu}{\nu!}=Q_{r-1}(x).
\end{equation}
Из \eqref{sob-jac-discrete-4.23}  мы замечаем, что смешанный ряд  \eqref{sob-jac-discrete-4.12} принимает следующий вид
\begin{equation}\label{sob-jac-discrete-4.24}
f(x)=\sum^{r-1}_{\nu=0}f^{(\nu)}(-1)\frac{(1+x)^\nu}{\nu!}
+\sum_{k=r}^\infty\frac{2^rf^{\alpha,0}_{r,k-r}}{(k-r+\alpha)^{[r]}}P_{k}^{\alpha-r,-r}(x).
\end{equation}

Далее заметим, что в силу \eqref{sob-jac-discrete-4.2}
\begin{equation}\label{sob-jac-discrete-4.25}
f^{\alpha,0}_{r,k-r}=\frac{1}{h^{\alpha,0}_{k-r}}\int\limits^1_{-1}(1-t)^\alpha f^{(r)}(t)P_{k-r}^{\alpha,0}(t)dt=
\frac{\hat f_k}{\sqrt{h^{\alpha,0}_{k-r}}}.
\end{equation}
С учетом  \eqref{sob-jac-discrete-4.25} мы перепишем ряд \eqref{sob-jac-discrete-4.24} в следующем виде
\begin{equation}\label{sob-jac-discrete-4.26}
f(x)=\sum^{r-1}_{\nu=0}f^{(\nu)}(-1)\frac{(1+x)^\nu}{\nu!}
+\sum_{k=r}^\infty\frac{2^r\hat f_k}{(k-r+\alpha)^{[r]}}{P_{k}^{\alpha-r,-r}(x)\over\sqrt{h^{\alpha,0}_{k-r}}}.
\end{equation}
Сопоставляя это равенство с  \eqref{sob-leg-3.50}, мы видим, что правая часть равенства \eqref{sob-jac-discrete-4.26}, а следовательно, и правая часть равенства \eqref{sob-jac-discrete-4.24},   представляет собой  ряд Фурье по полиномам Якоби $P_{k}^{\alpha-r,-r}(x)$, ортогональным по Соболеву. Это означает, что ряд
Фурье  \eqref{sob-leg-3.50} по полиномам Якоби $P_{k}^{\alpha-r,-r}(x)$, ортогональным по Соболеву,
является частным случаем  смешанного ряда  \eqref{sob-jac-discrete-4.12}  по полиномам Якоби $P_{k}^{\alpha,\beta}(x)$, соответствующим выбору:  $\alpha$ -- дробное, $\beta=0$.

\section{Условия равномерной сходимости смешанного ряда по общим полиномам Якоби}

В работе  \cite{sob-jac-discrete-Shar17} был рассмотрен вопрос  о сходимости смешанных рядов \eqref{sob-jac-discrete-4.12}  и \eqref{sob-jac-discrete-4.22}. Следуя \cite{sob-jac-discrete-Shar17},
 введем необходимые обозначения. Пусть $a,b\in\mathbb{R}$, $\mu(x)=\mu_{a,b}(x)=(1-x)^a(1+x)^b$,
$p>1$. Через $L^{a,b}_p$ мы обозначим пространство измеримых
функций, заданных на $(-1,1)$, для которых существует норма
$$ \|f\|_{p,\mu}=\left(\int\limits^1_{-1}|f(x)|^p\mu(x)\,dx\right)^{1/p}. $$
Для целого $r\ge1$ через $W^r_{L^{a,b}_p}$ мы обозначим класс
$r$-раз дифференцируемых функций, для которых $f^{(r-1)}(x)$
абсолютно непрерывна на $(-1,1)$, а $f^{(r)}\in L^{a,b}_p$. В работе \cite{sob-jac-discrete-Shar18} установлена следующая
\begin{theorem}\label{mixed-series-conv-mack}
 Пусть
$-1<\alpha,\beta\le\frac{1}{2},\,a,b\in\mathbb{R},\,p>1$ таковы, что
$$\left|\frac{a+1}{p}-\frac{\alpha+1}{2}\right|<
\min\left\{\frac{1}{4},\frac{\alpha+1}{2}\right\},$$$$
\left|\frac{b+1}{p}-\frac{\beta+1}{2}\right|<\min\left\{\frac{1}{4},\frac{\beta+1}{2}\right\},
$$  $\mu(x)=(1-x)^a(1+x)^b$. Тогда, если $r\ge1$,
$f\in W^r_{L^{a,b}_p}$, то смешанные ряды \eqref{sob-jac-discrete-4.12}  и \eqref{sob-jac-discrete-4.22} существуют
и равномерно на $[-1,1]$ сходятся к $f(x)$.
\end{theorem}
Как было показано выше, ряд Фурье  по полиномам Якоби $P_{k}^{\alpha-r,-r}(x)$, ортогональным по Соболеву (см.\eqref{sob-leg-3.50})  представляет собой частный случай смешанного ряда \eqref{sob-jac-discrete-4.12} с $\alpha$ -- дробное, $\beta=0$, поэтому из теоремы \ref{mixed-series-conv-mack} непосредственно вытекает
\begin{corollary}
  Пусть
$-1<\alpha\le\frac{1}{2},\,a,b\in\mathbb{R},\,p>1$ таковы, что
$$\left|\frac{a+1}{p}-\frac{\alpha+1}{2}\right|<
\min\left\{\frac{1}{4},\frac{\alpha+1}{2}\right\},\quad
\left|\frac{b+1}{p}-\frac{1}{2}\right|<\frac{1}{4},
$$
 $\mu(x)=(1-x)^a(1+x)^b$. Тогда, если $r\ge1$, $f\in W^r_{L^{a,b}_p}$, то  ряд Фурье \eqref{sob-leg-3.50} по полиномам Якоби $P_{k}^{\alpha-r,-r}(x)$, ортогональным по Соболеву,  равномерно на $[-1,1]$ сходятся к $f(x)$.
\end{corollary}

\section{ Операторы $\mathcal{Y}_{n+2r}^{\alpha, \beta}(f)$ и их аппроксимативные свойства для $f\in W^m_{L^{a,b}_2}$}

Пусть $-1<\alpha, \beta\le1/2$, $a , b \in \mathbb{R} $ удовлетворяют
условиям теоремы \ref{mixed-series-conv-mack}, $\mu(x)=(1-x)^a(1+x)^b $, $f\in W^r_{L^{a,b}_p}$. Тогда мы можем записать
\begin{equation}\label{sob-jac-discrete-6.1}
f(x)=p^{\alpha, \beta}(f,x)+\mathcal{F}_r^{\alpha, \beta} (f,x) =
 p ^{\alpha, \beta} (f,x) +\sum _{k=\overline r} ^ \infty {2^r f_{r,k}^{\alpha, \beta}
\over (k+\lambda)^{[r]} } P_{k+r}^{\alpha -r, \beta -r} (x),
\end{equation}
где $\lambda = \alpha +\beta $,
$$\overline r = \begin{cases} r-\lambda,&\text{ $\lambda \in \{-1,0\}$,}\\
0,&\text{ $\lambda \notin \{-1,0\}$,}\end{cases} $$ $p^{\alpha,
\beta}(x)$
--- некоторый алгебраический полином степени не выше, чем $2r$,
более точно
$$p^{\alpha, \beta}(f, x) = \begin{cases} \overline D _{2r-1-\lambda}^{\alpha, \beta}(x),&
\text{ если $\lambda \in \{-1,0\}$}, \\D _{r-1}^{\alpha, \beta}(x),&
\text{ если  $\lambda \notin \{-1,0\}$,}\end{cases}$$
(см.\eqref{sob-jac-discrete-4.12} и \eqref{sob-jac-discrete-4.22}). Причем ряд $\mathcal{F}_r^{\alpha, \beta} (f,x) $,
фигурирующий в \eqref{sob-jac-discrete-6.1} сходится равномерно относительно $x\in [-1,1] $
(теорема \ref{mixed-series-conv-mack}).

Через $\mathcal{Y}_{n+2r}^{\alpha, \beta}(f)=\mathcal{Y}_{n+2r}^{\alpha,
\beta}(f,x) $ мы обозначим частичные суммы ряда \eqref{sob-jac-discrete-6.1} следующего
вида
\begin{equation}\label{sob-jac-discrete-6.2}
\mathcal{Y}_{n+2r}^{\alpha, \beta}(f,x)=p^{\alpha, \beta}(x)+
\sum _{k=\overline r}^ {n+r}{2^r f_{r,k}^{\alpha, \beta } \over (k+\lambda)^{[r]}}
 P^{\alpha-r,\beta-r}_{k+r}(x).
\end{equation}
Это алгебраический полином степени $n+2r$. Будем его рассматривать
как аппарат приближения дифференцируемых и аналитических функций. Мы
заметим, что если $f\in W_{L^{a,b}_p}^r $, где $p$ и $\mu(x)$
удовлетворяют условиям теоремы \ref{mixed-series-conv-mack}, то разность
$R_{r,n}^{\alpha,\beta}(f,x)=f(x)-\mathcal{Y}_{n+2r}^{\alpha,
\beta}(f,x),$ как это вытекает из \eqref{sob-jac-discrete-6.1} и \eqref{sob-jac-discrete-6.2}, можно представить
следующим образом
\begin{equation}\label{sob-jac-discrete-6.3}
R_{r,n}^{\alpha, \beta } (f,x) = f(x) - \mathcal{Y}_{n+2r}^{\alpha, \beta}(f,x)=
\sum _{k=n+r+1}^ \infty {2^r f_{r,k} ^{\alpha, \beta} \over
(k+\lambda)^{[r]}}P_{k+r}^{\alpha-r,\ \beta - r}(x).
\end{equation}

Отметим также, что если $f(x)=S_{n+2r}(x)$ --  алгебраический полиномом степени $n+2r$, то $R_{r,n}^{\alpha, \beta } (f,x)\equiv0$. В самом деле, из равенства \eqref{sob-jac-discrete-4.2} следует, что если $k>n+r$, то  $f_{r,k} ^{\alpha, \beta}=0$, то наше утверждение непосредственно вытекает из \eqref{sob-jac-discrete-6.3}. Таким образом, оператор $\mathcal{Y}_{n+2r}^{\alpha, \beta}=\mathcal{Y}_{n+2r}^{\alpha, \beta}(f,x)$ представляет собой проектор на подпространство алгебраических полиномов степени $n+2r$. Далее, если $0\le\nu<r$, то $f^{(r)}(x)=\left(f^{(\nu )}(x)\right)^{(r-\nu)}$ и, стало быть, $ f _{r,k} ^ {\alpha, \beta}= (f^{(\nu)}) _{r-\nu,k} ^ {\alpha, \beta}$, поэтому в силу \eqref{sob-jac-discrete-6.3} и \eqref{Haar-Tcheb-eq5.5} имеем
$$
f^{(\nu)} (x) -{d^\nu \over dx^\nu}\mathcal{Y}_{n+2r}^{\alpha,
\beta}(f,x)=
$$
$$
\sum_{k=n+r+1}^\infty{2^{r-\nu} f _{r,k} ^ {\alpha, \beta} \over
(k+\lambda)^{[r-\nu]} }P_{k+r-\nu}^{\alpha -r+\nu, \ \beta -r+\nu}
(x)=
$$
$$
\sum_{k=n+\nu+(r-\nu)+1}^\infty{2^{r-\nu} (f^{(\nu)}) _{r-\nu,k} ^ {\alpha, \beta} \over
(k+\lambda)^{[r-\nu]} }P_{k+r-\nu}^{\alpha -r+\nu, \ \beta -r+\nu}
(x)=R_{r-\nu, n+\nu}^{\alpha, \beta}(f^{(\nu)},x)=
$$
\begin{equation}\label{sob-jac-discrete-6.4}
f^{(\nu)}(x)-\mathcal{Y}_{n+\nu +2(r-\nu)}^{\alpha,\beta}(f^{(\nu)},x).
\end{equation}
Это свойство в сочетании с весовой оценкой \eqref{Haar-Tcheb-eq5.11} и асимптотической формулой \eqref{Haar-Tcheb-eq5.12} для полиномов Якоби
$P_n^{\alpha,\beta}(x)$ с произвольными $\alpha,\beta \in \mathbb{R}$ (в том числе для
$\alpha,\beta<-1$) показывает, что операторы $\mathcal{Y}_{n+2r}^{\alpha,
\beta}(f)=\mathcal{Y}_{n+2r}^{\alpha, \beta}(f,x)$ успешно могут быть
использованы для решения задачи об одновременном приближении
дифференцируемой функции $f(x) $ и нескольких ее производных
$f^{(\nu)} $ в то время, как суммы Фурье-Якоби для этой цели не
подходят. Подробное доказательство этого утверждение для различных
классов гладких и аналитических функций в случае $\alpha= \beta $
(случай ультрасферических полиномов) было дано в работе \cite{sob-jac-discrete-Shar17}.
В настоящем параграфе мы рассмотрим  задачу о приближении
полиномами $\mathcal{Y}_{n+2r}^{\alpha,\beta}(f,x)$ функций
$f\in W_{L_2^{\alpha+m-r,\beta+m-r}}^m$, где $m\ge r$, в случае произвольных $\alpha$ и $\beta$, удовлетворяющих условию
$-1<\alpha,\beta\le \frac12$. При этом нам понадобятся
некоторые обозначения: ${L}_p={L}_p^{0,0}$,
$W^rH_{{L}_p}^\mu(B)$ -- подкласс функций   $f=f(x)$ из $W^r_{{L}_p}$, для которых
$\omega(f^{(r)},\delta)_{{L}_p}\le B\delta^\mu$, $0<\mu\le1$, где
     $$
 \omega(g,\delta)_{{L}_p}=\sup_{0<h\le\delta}
\left(\int\limits_{-1}^{1-h}|g(x+h)-g(x)|^pdx\right)^{1/p}
     $$
 -- модуль непрерывности функции $g=g(x)\in {L}_p$, $E_n(f)_{{L}_p^{s,q}}$   -- наилучшее приближение функции
$f\in {L}_p^{s,q}$ алгебраическими полиномами степени $n$, $E_n(f)_{C[-1,1]}$ --  наилучшее приближение $f\in C[-1,1]$
 алгебраическими полиномами  степени $n$. Имеет место следующая

\begin{theorem}     Пусть $-1<\alpha,\beta\le
1/2$, $m\ge r\ge1$, $0\le \nu\le r-1$, $n\ge m-2r$, $f\in W_{{L}_2^{\alpha+m-r,\beta+m-r}}^m$. Тогда имеет  место оценка $$
\left|f^{(\nu)}(x)-{d^\nu\over
dx^\nu} \mathcal{Y}_{n+2r}^{\alpha,\beta}(f,x)\right|\le {E_{n+2r-m}(f^{(m)})_{{L}_2^{\alpha+m-r,\beta+m-r}}\over n^{m-\nu-1/2}}\times
     $$
     $$
c(\alpha,\beta,r,m)
\left(\sqrt{1-x}+{1\over n}\right)^{r-\nu-\alpha-{1\over2}}
\left(\sqrt{1+x}+{1\over n}\right)^{r-\nu-\beta-{1\over2}}.
$$
\end{theorem}

\begin{corollary}
       Пусть $r\ge1$, $0<\mu\le1$. Тогда
                $$
       \sup_{f\in W^rH^\mu_{{L}_2}(1)}
      \|f-\mathcal{Y}_{n+2r}^{0,0}(f)\|_{C[-1,1]}\asymp
      \sup_{f\in W^rH^\mu_{{L}_2}(1)}
      E_{n+2r}(f)_{C[-1,1]}\asymp {1\over (n+1)^{r+\mu-1/2}},
                $$
  где $a_n\asymp b_n$ $(n=0,1\ldots)$ означает, что
 найдутся положительные постоянные
      $c_1$ и $c_2$, для которых $c_1a_n\le b_n\le c_2a_n$
 $(n=0,1\ldots)$.
\end{corollary}

Заметим, что если $\beta=0$, $-1<\alpha$ -- дробное, то, как было показано выше (см.\eqref{sob-jac-discrete-4.24}  и \eqref{sob-jac-discrete-4.26}), смешанный ряд по полиномам Якоби $P_k^{\alpha,0}(x)$ совпадает с рядом Фурье по полиномам Якоби  $P_k^{\alpha-r,-r}(x)$, ортогональным по Соболеву, в частности, имеет место равенство
\begin{equation}\label{sob-jac-discrete-6.12}
S_{n+2r}^\alpha(f,x)= \mathcal{Y}_{n+2r}^{\alpha,0}(f,x),
\end{equation}
 где $S_{n+2r}^\alpha(f,x)$ -- частичная сумма ряда Фурье \eqref{sob-leg-3.50} (см. также  \eqref{sob-jac-discrete-4.24} и   \eqref{sob-jac-discrete-4.26}), определенная равенством
\eqref{sob-jac-discrete-3.10}. Из теоремы 3 и равенства \eqref{sob-jac-discrete-6.12} непосредственно вытекает справедливость следующего утверждения.


\begin{corollary}
   Пусть $-1<\alpha\le1/2$,  $m\ge r\ge1$, $0\le \nu\le r-1$, $n\ge m-2r$,  $f\in W_{{
L}_2^{\alpha+m-r,m-r}}^m$. Тогда имеет  место оценка
$$
\left|f^{(\nu)}(x)-{d^\nu\over dx^\nu} S_{n+2r}^\alpha(f,x)\right|\le {E_{n+2r-m}(f^{(m)})_{{L}_2^{\alpha+m-r,m-r}}\over n^{m-\nu-1/2}}\times
     $$
     $$
c(\alpha,r,m)
\left(\sqrt{1-x}+{1\over n}\right)^{r-\nu-\alpha-{1\over2}}
\left(\sqrt{1+x}+{1\over n}\right)^{r-\nu-{1\over2}}.
$$
\end{corollary}

\section{Приближение функций из $W^r$ частичными суммами ряда  Фурье по полиномам Якоби, ортогональным по Соболеву}
 Через $W^r$ обозначим класс $r$-раз непрерывно дифференцируемых функций $f(x)$, заданных на $[-1,1]$, для которых $r$--я производная $f^{(r)}(x)$ удовлетворяет неравенству $\|f^{(r)}\|\le1$.  Рассмотрим вопрос об оценке отклонения  функции  $f\in W^r$ от частичных сумм $S_{n+2r}^\alpha(f,x)$ ряда Фурье по полиномам Якоби $P_k^{\alpha-r,-r}(x)$, ортогональным по Соболеву (см. \eqref{sob-jac-discrete-3.10}).  Один из подходов к решению этой задачи    связан с использованием  известного в теории приближений
   результата, полученного в работах \cite{sob-jac-discrete-Tel}, \cite{sob-jac-discrete-Gop}, а именно, если $f\in
W^r$, то найдется последовательность алгебраических полиномов
$\{p_n(x)\}_{n=4r+5}^\infty$ таких, что $\deg p_n(x)=n$ и
положительная постоянная $c=c(r)$, для которой справедливы
следующие неравенства ($ 0\le\nu\le r$)
\begin{equation}\label{sob-jac-discrete-7.1}
|f^{(\nu)}(x)-p_n^{(\nu)}(x)|\le
c(r)\left(\frac{\sqrt{1-x^2}}{n}\right)^{r-\nu}\omega\left(f^{(r)},{{\sqrt{1-x^2}\over
n}}\right), \quad x\in [-1,1],
\end{equation}
где
 $$
 \omega(g,\delta)
=\sup_{x,t\in[-1,1],|x-t|\le\delta}|g(x)-g(t)|
$$
--модуль непрерывности функции $g\in C[-1,1]$. Мы можем переписать это неравенство
в следующем виде ($0\le\nu\le r$)
\begin{equation}\label{sob-jac-discrete-7.2}
\left(\frac{n}{\sqrt{1-x^2}}\right)^{r-\nu}|f^{(\nu)}(x)-p_n^{(\nu)}(x)|\le
c(r)\omega\left(f^{(r)},{{\sqrt{1-x^2}\over n}}\right),
x\in [-1,1].
\end{equation}
 Отсюда следует, что если $f\in W^r$,
то  мы можем определить следующую величину
\begin{equation}\label{sob-jac-discrete-7.3}
U_m^r(f)= \inf_{p_m}\max_{0\le\nu\le r}\max_{-1\le x\le1}\left(\frac{n}{\sqrt{1-x^2}}\right)^{r-\nu}|f^{(\nu)}(x)-p^{(\nu)}_m(x)|,
\end{equation}
 где нижняя грань берется по всем алгебраическим
полиномам $p_m(x)$ степени $m$, для которых
$f^{(\nu)}(\pm1)=p_m^{(\nu)}(\pm1)$ при $\nu=0,1,\ldots,r $ и выполняется неравенства \eqref{sob-jac-discrete-7.2}. Из
\eqref{sob-jac-discrete-7.2} следует, что
\begin{equation}\label{sob-jac-discrete-7.4}
U_m^r(f)\le c(r)\omega(f^{(r)},{1\over m}),
\end{equation}
если только $f\in W^r$. Среди полиномов
$p_m(x)$ степени $m\ge 2r-1$, удовлетворяющих условиям
$f^{(\nu)}(\pm1)=p_m^{(\nu)}(\pm1)$ при $\nu=0,1,\ldots,r$ и неравенству  \eqref{sob-jac-discrete-7.2},
через $p_m^r(f)=p_m^r(f,x)$--мы обозначим тот, для которого
\begin{equation}\label{sob-jac-discrete-7.5}
U_m^r(f)=  \max_{0\le\nu\le r}\max_{-1\le x\le1}\left(\frac{n}{\sqrt{1-x^2}}\right)^{r-\nu}|f^{(\nu)}(x)-(p_m^r(f,x))^{(\nu)}|.
\end{equation}


Для полинома $p_{n+2r}^r(f)=p_{n+2r}^r(f,x)$, удовлетворяющего условию \eqref{sob-jac-discrete-7.5} при $m=n+2r$,  имеет место равенство
\begin{equation}\label{sob-jac-discrete-7.6}
 S_{n+2r}^\alpha(p_{n+2r}^r(f), x)=p_{n+2r}^r(f,x),
\end{equation}
поэтому, если $f\in W^r$, то мы можем записать
\begin{equation}\label{sob-jac-discrete-7.7}
 f(x)-S_{n+2r}^\alpha(f,x)=f(x)-
p_{n+2r}^r(f,x)+S_{n+2r}^\alpha(p_{n+2r}^r(f)-f,x).
\end{equation}
В силу  \eqref{sob-jac-discrete-3.10} и того, что $f^{(\nu)}(-1)=(p_{n+2r}^r(f,x))^{(\nu)}|_{x=-1}$ при $\nu=0,1,\ldots,r-1 $
мы имеем

\begin{equation}\label{sob-jac-discrete-7.8}
 Z(x)=S_{n+2r}^\alpha(p_{n+2r}^r(f)-f,x)=\sum_{k=r}^{n+2r} \frac{2^r\hat g_k P_{k}^{\alpha-r,-r}(x)}{\sqrt{h_{k-r}^{\alpha,0}}(k+\alpha-r)^{[r]}},
     \end{equation}
где
$$
\hat g_k= \int\limits_{-1}^1(p_{n+2r}^r(f,t)-f(t))^{(r)}\frac{P_{k-r}^{\alpha,0}(t)(1-t)^\alpha}{\sqrt{h_{k-r}^{\alpha,0}}} dt.
$$
Отсюда, с учетом равенств $f^{(\nu)}(\pm1)=(p_{n+2r}^r(f,x))^{(\nu)}|_{x=\pm1}$ при $\nu=0,1,\ldots,r-1 $  путем интегрирования по частям имеем
\begin{equation}\label{sob-jac-discrete-7.9}
\hat g_k= (-1)^r\int\limits_{-1}^1(p_{n+2r}^r(f,t)-f(t))\frac{(P_{k-r}^{\alpha,0}(t)(1-t)^\alpha)^{(r)}}{\sqrt{h_{k-r}^{\alpha,0}}} dt.
\end{equation}
Обратимся теперь к равенству \eqref{Haar-Tcheb-eq5.8}, в котором заменим $m$ на $r$, $\alpha$ на $\alpha-r$, $\beta$ на $-r$  и запишем
\begin{equation*}
(P_{k-r}^{\alpha,0}(t)(1-t)^\alpha)^{(r)} = {d^r\over dt^r}((1-t)^{r+\alpha-r}(1+t)^{r-r}P_{k-r}^{r+\alpha-r,r-r}(t))=
\end{equation*}
$$
(-2)^rk^{[r]}(1-t)^{\alpha-r}(1+t)^{-r}P_{k}^{\alpha-r,-r}(t).
$$
Подставляя это выражение в \eqref{sob-jac-discrete-7.9}, имеем
\begin{equation}\label{sob-jac-discrete-7.10}
\hat g_k= 2^rk^{[r]}\int\limits_{-1}^1(p_{n+2r}^r(f,t)-f(t))\frac{P_{k}^{\alpha-r,-r}(t)}{\sqrt{h_{k-r}^{\alpha,0}}} dt.
\end{equation}
Из \eqref{sob-jac-discrete-7.8} и \eqref{sob-jac-discrete-7.10} находим

\begin{equation}\label{sob-jac-discrete-7.11}
 Z(x)=2^{2r}\int\limits_{-1}^1\frac{p_{n+2r}^r(f,t)-f(t)}{(1-t)^{r-\alpha}(1+t)^{r}}\sum_{k=r}^{n+2r} \frac{k^{[r]} P_{k}^{\alpha-r,-r}(x)P_{k}^{\alpha-r,-r}(t)}{(k+\alpha-r)^{[r]}h_{k-r}^{\alpha,0}}dt.
 \end{equation}


Далее, из \eqref{Haar-Tcheb-eq5.6} имеем
$$
P_{k}^{\alpha-r,-r}(x)=\frac{(k+\alpha-r)^{[r]}}{k^{[r]}2^r}(1+x)^rP_{k-r}^{\alpha-r,r}(t),
$$
поэтому равенство \eqref{sob-jac-discrete-7.11} можно переписать так
\begin{equation}\label{sob-jac-discrete-7.12}
 Z(x)=(1+x)^r\int\limits_{-1}^1\frac{p_{n+2r}^r(f,t)-f(t)}{(1-t)^{r-\alpha}}\sum_{k=r}^{n+2r} \frac{(k+\alpha-r)^{[r]} P_{k-r}^{\alpha-r,r}(x)P_{k-r}^{\alpha-r,r}(t)}{k^{[r]}h_{k-r}^{\alpha,0}}dt.
     \end{equation}
С другой стороны, в силу  \eqref{Haar-Tcheb-eq5.3}
\begin{equation}\label{sob-jac-discrete-7.13}
  h_{k-r}^{\alpha,0}\frac{k^{[r]}}{(k+\alpha-r)^{[r]}}=\frac{2^{\alpha+1}\Gamma(k-2r+\alpha+1)k!}{(k-r)!\Gamma(k-r+\alpha+1)(2(k-r)+\alpha+1)}=
  h_{k-r}^{\alpha-r,r}.
     \end{equation}
Из \eqref{sob-jac-discrete-7.12}, \eqref{sob-jac-discrete-7.13} и \eqref{sob-jac-discrete-eq2.13} имеем
$$
Z(x)=(1+x)^r\int\limits_{-1}^1\frac{p_{n+2r}^r(f,t)-f(t)}{(1-t)^{r-\alpha}}\sum_{k=r}^{n+2r} \frac{ P_{k-r}^{\alpha-r,r}(x)P_{k-r}^{\alpha-r,r}(t)}{h_{k-r}^{\alpha-r,r}}dt=.
 $$
$$
 (1+x)^r\int\limits_{-1}^1\frac{p_{n+2r}^r(f,t)-f(t)}{(1-t)^{r-\alpha}}\sum_{k=0}^{n+r} \frac{ P_{k}^{\alpha-r,r}(x)P_{k}^{\alpha-r,r}(t)}{h_{k}^{\alpha-r,r}}dt=
 $$
 \begin{equation}\label{sob-jac-discrete-7.14}
 (1+x)^r\int\limits_{-1}^1\frac{p_{n+2r}^r(f,t)-f(t)}{(1-t)^{r-\alpha}}K_{n+r}^{\alpha-r,r}(x,t)dt.
     \end{equation}

Сопоставляя равенства \eqref{sob-jac-discrete-7.7}, \eqref{sob-jac-discrete-7.8} и \eqref{sob-jac-discrete-7.14}, мы можем записать следующее неравенство
\begin{equation}\label{sob-jac-discrete-7.15}
|f(x)-S_{n+2r}^\alpha(f,x)|\le  |f(x)-p_{n+2r}^r(f,x)|+ R_{n+2r}^\alpha(f)(x),
\end{equation}
где
\begin{equation}\label{sob-jac-discrete-7.16}
R_{n+2r}^\alpha(f)(x)=(1+x)^r\int\limits_{-1}^1\frac{|p_{n+2r}^r(f,t)-f(t)|}{(1-t)^{r-\alpha}}|K_{n+r}^{\alpha-r,r}(x,t)|dt.
\end{equation}
В связи с равенствами \eqref{sob-jac-discrete-7.15} и \eqref{sob-jac-discrete-7.16} возникает задача об оценке величины $R_{n+2r}^\alpha(f)(x)$ при $x\in[-1,1]$. Рассмотрим один из возможных подходов к решению этой задачи, опирающийся на использовании неравенства \eqref{sob-jac-discrete-7.2} и \eqref{sob-jac-discrete-7.4} . С этой целью представим величину $R_{n+2r}^\alpha(f)(x)$ в следующем виде
\begin{equation}\label{sob-jac-discrete-7.17}
R_{n+2r}^\alpha(f)(x)=R_{n+2r,1}^\alpha(f)(x)+ R_{n+2r,2}^\alpha(f)(x)
\end{equation}
где
\begin{equation}\label{sob-jac-discrete-7.18}
R_{n+2r,1}^\alpha(f)(x)=(1+x)^r\int\limits_{-1}^{1-\frac{1}{n^2}}\frac{|p_{n+2r}^r(f,t)-f(t)|}{(1-t)^{r-\alpha}}
|K_{n+r}^{\alpha-r,r}(x,t)|dt.
\end{equation}
\begin{equation}\label{sob-jac-discrete-7.19}
R_{n+2r,2}^\alpha(f)(x)=(1+x)^r\int\limits_{1-\frac{1}{n^2}}^1\frac{|p_{n+2r}^r(f,t)-f(t)|}{(1-t)^{r-\alpha}}
|K_{n+r}^{\alpha-r,r}(x,t)|dt.
\end{equation}
Чтобы оценить величину $R_{n+2r,1}^\alpha(f)(x)$ обратимся к равенству \eqref{sob-jac-discrete-7.5}. Тогда в силу \eqref{sob-jac-discrete-7.18} получим
\begin{equation}\label{sob-jac-discrete-7.20}
R_{n+2r,1}^\alpha(f)(x)\le \frac{2U_{n+2r}^r(f)}{(n+2r)^{r}}(1+x)^r\int\limits_{-1}^{1-\frac{1}{n^2}}\frac{|K_{n+r}^{\alpha-r,r}(x,t)|}
{(1-t)^{\frac r2-\alpha}}dt.
\end{equation}
Далее, поскольку $f^{(\nu)}(1)=(p_{n+2r}^r(f,x))^{(\nu)}|_{x=1}$ при $\nu=0,1,\ldots,r-1 $, то мы можем записать
$$
 \frac{|p_{n+2r}^r(f,t)-f(t)|}{(1-t)^r} =\frac{1}{(r-1)!(1-t)^r}\left|\int\limits_t^1(\tau-t)^{r-1}(p_{n+2r}^r(f,\tau)-f(\tau))^{(r)}d\tau\right|,
$$
а в силу \eqref{sob-jac-discrete-7.2}
$$
\frac{1}{(r-1)!(1-t)^r}\left|\int\limits_t^1(\tau-t)^{r-1}(p_{n+2r}^r(f,\tau)-f(\tau))^{(r)}d\tau\right|\le
$$
$$
c(r)\omega\left(f^{(r)},{\sqrt{1-t^2}\over n+2r} \right)\frac{1}{1-t}\int\limits_t^1\left(\frac{\tau-t}{1-t}\right)^{r-1}d\tau\le c(r)\omega\left(f^{(r)},{{\sqrt{1-t^2}\over n+2r} }\right),
$$
поэтому
\begin{equation}\label{sob-jac-discrete-7.21}
\frac{|p_{n+2r}^r(f,t)-f(t)|}{(1-t)^r}\le c(r)\omega\left(f^{(r)},{{\sqrt{1-t^2}\over n+2r} }\right).
\end{equation}
Из \eqref{sob-jac-discrete-7.19} и \eqref{sob-jac-discrete-7.21} находим
\begin{equation}\label{sob-jac-discrete-7.22}
R_{n+2r,2}^\alpha(f)(x)\le c(r)\omega\left(f^{(r)},{1\over (n+2r)^2} \right)(1+x)^r\int\limits_{1-\frac{1}{n^2}}^1(1-t)^{\alpha}|K_{n+r}^{\alpha-r,r}(x,t)|dt.
\end{equation}
Равенство \eqref{sob-jac-discrete-7.17} вместе с неравенствами  \eqref{sob-jac-discrete-7.4}, \eqref{sob-jac-discrete-7.15}, \eqref{sob-jac-discrete-7.20} и \eqref{sob-jac-discrete-7.22} позволяет сформулировать следующий результат.
\begin{theorem}%теоремой 4
Пусть $-1<\alpha$   --  нецелое, $1\le r$ -- натуральное, $f\in W^r$. Тогда
$$
|f(x)-S_{n+2r}^\alpha(f,x)|\le  c(r)\left(\frac{\sqrt{1-x^2}}{n+2r}\right)^{r}\omega\left(f^{(r)},{{\sqrt{1-x^2}\over
n+2r}}\right)+
$$

\begin{equation}\label{sob-jac-discrete-7.23}
c(r)(1+x)^r\left[\omega\left(f^{(r)},{1\over n+2r}\right)\frac{I^\alpha_{r,n}(x)}{(n+2r)^{r}}+\omega\left(f^{(r)},{1\over (n+2r)^2} \right)J^\alpha_{r,n}(x)\right],
\end{equation}
где
\begin{equation}\label{sob-jac-discrete-7.24}
I^\alpha_{r,n}(x)=\int\limits_{-1}^{1-\frac{1}{n^2}}\frac{|K_{n+r}^{\alpha-r,r}(x,t)|}
{(1-t)^{\frac r2-\alpha}}dt,
\end{equation}
\begin{equation}\label{sob-jac-discrete-7.25}
J^\alpha_{r,n}(x)=\int\limits_{1-\frac{1}{n^2}}^1(1-t)^{\alpha}|K_{n+r}^{\alpha-r,r}(x,t)|dt.
\end{equation}
\end{theorem}

\textit{Замечание 2.} В связи с данной теоремой возникает задача об исследовании поведения при $n\to\infty$ величин $I^\alpha_{r,n}(x)$ и $J^\alpha_{r,n}(x)$, определенных равенствами  \eqref{sob-jac-discrete-7.24} и \eqref{sob-jac-discrete-7.25}, соответственно. Но эта задача является объектом исследования другой нашей работы, поэтом
у мы здесь на этом не остановимся.
Отметим только, что если в равенстве \eqref{sob-jac-discrete-7.16} мы перейдем к пределу при $\alpha\to0$, то можно показать, что
$$
R_{n+2r}^0(f)(x)=(1-x^2)^r\int\limits_{-1}^1|p_{n+2r}^r(f,t)-f(t)|
|K_{n}^{r,r}(x,t)|dt.
$$
Отсюда, в свою очередь, выводим
$$
R_{n+2r}^0(f)(x)\le E_{n+2r}^r(f)\Lambda_{n}^r(x),
$$
где
$$
\Lambda_{n}^r(x)=(1-x^2)^r\int\limits_{-1}^1(1-t^2)^{\frac{r}{2}}|K_{n}^{r,r}(x,t)|dt,
$$
$$
E_{n+2r}^r(f)=\max_{-1\le x\le1}\frac{|f(x)-p_{n+2r}^r(f,x)|}{(1-x^2)^\frac{r}{2}}.
$$
Так что, в случае $\alpha=0$ вместо двух величин $I^\alpha_{r,n}(x)$ и $J^\alpha_{r,n}(x)$ возникает одна величина  $\Lambda_{n}^r(x)$, для которой в работе \cite{sob-jac-discrete-Shar15} получена следующая оценка
$
\Lambda_{n}^r(x)\le c(r)(1-x^2)^{\frac{r}{2}}[\ln(1+n\sqrt{1-x^2})+(1-x^2)^{-\frac{1}{4}}].
$
С другой стороны, из \eqref{sob-jac-discrete-7.4} следует, что если $f\in W^r$, то
$
E_{n+2r}^r(f)\le c(r)\omega(f^{(r)},1/(n+2r))/(n+2r)^r,
$
и поэтому
$$
\frac{R_{n+2r}^0(f)(x)}{(1-x^2)^{\frac{r}{2}-\frac14}}\le c(r)\frac{\omega(f^{(r)},{1\over n+2r})}{(n+2r)^r}[(1-x^2)^{\frac{1}{4}}\ln(1+n\sqrt{1-x^2})+1].
$$
