\chapter{Разностные уравнения и полиномы, ортогональные по Соболеву, порожденные классическими многочленами Чебышева дискретной переменной}

%\section{Введение}\label{sob-tcheb-difference-s1}
\section{Решение задачи Коши для разностного уравнения с помощью полиномов, ортогональных по Соболеву}
\label{sob-tcheb-difference-s1}
В данном параграфе рассмотрен вопрос о представлении решения задачи Коши разностного уравнения
\begin{equation}\label{sob-tcheb-difference-1.1}
  \sum_{l=0}^{r}a_l(j)\Delta^{l}y(j)=f(j), \, j\in \Omega_{N}=\{0,1,\ldots,N-1\},
\end{equation}
с начальными условиями $\Delta^{l}y(0)=y_l, \, l=0,1,\ldots,r-1$,  путем разложения $y(j)$  на сетке $\Omega_{N+r}=\{0,1,\ldots,N-1+r\}$ в конечный ряд Фурье по полиномам, ортогональным по Соболеву на $\Omega_{N+r}$,  где функции $a_l$, $l=0,1,\ldots,r-1$, заданы на множестве $\Omega_{N}$, а $\Delta^{l}y$ -- оператор конечной разности порядка $l$.
 Такая задача представляет интерес не только сама по себе, но и в связи с тем, что к ней может быть сведена проблема  о приближенном решении задачи  Коши для обыкновенного дифференциального уравнения вида
\begin{equation}\label{sob-tcheb-difference-1.2}
  \sum_{l=0}^{r}a_l(t)y^{(l)}(t)=f(t)
\end{equation}
с начальными условиями $y^{(l)}(0)=y_l, \, l=0,1,\ldots,r-1$.
Заметим, что уравнение \eqref{sob-tcheb-difference-1.1} можно переписать в следующем рекуррентном виде
\begin{equation}\label{sob-tcheb-difference-1.3}
a_r(j)y(j+r)=\sum_{l=0}^{r-1}b_l(j)y(j+l)+f(j), \, j\in \Omega_{N},
\end{equation}
в котором $b_l(j)$, $l=0,1,\ldots,r-1$, --  заданные функции, определенные на сетке~$\Omega_{N}$. Если функция $|a_r(j)|\ge c>0$, $j\in \Omega_N$, то точное решение уравнения \eqref{sob-tcheb-difference-1.1} можно найти с помощью рекуррентной формулы \eqref{sob-tcheb-difference-1.3}. Если же для некоторых  $j\in \Omega_N$ будет $a_r(j)=0$, то однозначно найти соответствующие значения  $y(j+r)$  с помощью равенства  \eqref{sob-tcheb-difference-1.3} невозможно. Кроме того, отметим еще, что если значения  $f(j)$ функции $f$, фигурирующей в правой части уравнения \eqref{sob-tcheb-difference-1.1}, содержат погрешности измерений, а их число $N$ велико, то использование рекуррентной формулы \eqref{sob-tcheb-difference-1.3} для нахождения решения задачи $y=y(j)$ может дать неудовлетворительные результаты даже в том случае, когда $|a_r(j)|\ge c>0$, $j\in \Omega_N$. Таким образом, возникает задача о поиске альтернативных методов решения уравнения \eqref{sob-tcheb-difference-1.1}.


В данном параграфе предлагается новый метод приближенного решения задачи Коши \eqref{sob-tcheb-difference-1.1}, основанный на применении полиномов $\tau _{r,n}^{\alpha,\beta}(x,N)$, $\alpha,\beta>-1$, $n\in\Omega_{N+r}$, ортогональных по Соболеву относительно скалярного произведения вида
\begin{equation}\label{sob-tcheb-difference-1.4}
\langle f,g \rangle =
\sum_{\nu=0}^{r-1}\Delta^\nu f(0)\Delta^\nu g(0)+
\sum_{j=0}^{N-1}\Delta^rf(j)\Delta^rg(j)\mu(j),
\end{equation}
где $\mu(x)$ -- весовая функция, определенная равенством \eqref{sob-tcheb-difference-3.5}.
Полиномы $\tau _{r,k+r}^{\alpha,\beta}(x,N)$ порождаются классическими полиномами Чебышева $\tau_n^{\alpha,\beta}(x,N)$, $n\in\Omega_N$, ортонормированными на сетке $\Omega_N$  с весом $\mu(x)$, и определяются посредством равенств \eqref{sob-tcheb-difference-4.1} для $r<k \le N+r-1$ и \eqref{sob-tcheb-difference-4.2} для $0 \le k \le r-1$.

Следует отметить, что теория   полиномов, ортогональных относительно скалярных произведений типа Соболева, получила  в последние три десятилетия интенсивное развитие  и нашла ряд важных приложений (см. \cite{Haar-Tcheb-IserKoch,Haar-Tcheb-MarcelAlfaroRezola,Haar-Tcheb-Meijer,Haar-Tcheb-KwonLittl1,Haar-Tcheb-KwonLittl2,Haar-Tcheb-MarcelXu} 
и цитированную там литературу). Характерной особенностью скалярных  произведений типа Соболева является, в частности, то, что они, как правило,  содержат слагаемые, которые <<контролируют>> поведение соответствующих ортогональных полиномов в одной или нескольких точках числовой оси.

С другой стороны, в работах \cite{meixner-18} -- \cite{sob-tcheb-difference-Shar9} были введены так называемые смешанные ряды по классическим ортогональным полиномам   как альтернативный рядам Фурье по тем же полиномам аппарат решения задач, в которых требуется одновременно приближать дифференцируемую функцию и несколько ее производных.
В частности, такая задача часто возникает при решении  дифференциальных уравнений численно-аналитическими (спектральными) методами  \cite{Haar-Tcheb-SolDmEg,Haar-Tcheb-Tref1,Haar-Tcheb-Tref2}. Смешанные ряды по ортонормированным системам функций оказались естественным и весьма эффективным средством   решения краевых задач для дифференциальных уравнений спектральными методами. По этому поводу мы можем отослать, например, к работе \cite{Haar-Tcheb-MMG2016}. В работе \cite{meixner-18} показано, что смешанный ряд функции $f$ по ортонормированной системе $\{\varphi_k(x)\}$ представляет собой  ряд Фурье этой функции по новой системе $\{\varphi_{r,k}(x)\}$, ортонормированной по Соболеву относительно скалярного произведения вида
\begin{equation}\label{sob-tcheb-difference-1.5}
\langle f,g \rangle=
\sum_{\nu=0}^{r-1}f^{(\nu)}(a)g^{(\nu)}(a)+\int_{a}^{b} f^{(r)}(t)g^{(r)}(t)\rho(t) dt
\end{equation}
и порожденной самой системой $\{\varphi_k(x)\}$.

Дискретный аналог скалярного произведения \eqref{sob-tcheb-difference-1.5}  имеет вид \eqref{sob-tcheb-difference-1.4},
 где функции $f$ и $g$ заданы на  множестве $\Omega_{N+r}$, $\mu=\mu(j)$ -- дискретная весовая функция, заданная на множестве $\Omega_{N}$. В случае, когда $r=0$, мы будем считать, что $\sum_{\nu=0}^{r-1}\Delta^\nu f(0)\Delta^\nu g(0)=0$. При $r\ge1$ особой точкой в скалярном произведении \eqref{sob-tcheb-difference-1.4} является  $x=0$, в которой контролируется поведение соответствующих ортогональных по Соболеву полиномов дискретной переменной благодаря присутствию в  \eqref{sob-tcheb-difference-1.4} выражения  $\sum_{k=0}^{r-1}\Delta^kf(0)\Delta^kg(0)$.
В настоящей работе, наряду с конструированием полиномов $\tau _{r,n}^{\alpha,\beta}(x,N)$, $n=0,1,\ldots,N-1+r$, ортонормированных по Соболеву относительно скалярного произведения вида \eqref{sob-tcheb-difference-1.4}, и изучением некоторых важных свойств этих полиномов, рассмотрена также задача об одновременном приближении   на сетке $\Omega_{N+r}$ решения $y=y(j)$ задачи Коши~\eqref{sob-tcheb-difference-1.4} и его конечных разностей $\Delta^{\nu}y(j)$  частичными суммами Фурье функции $y$ по системе
$\{\tau _{r,n}^{\alpha,\beta}(x,N)\}_{n=0}^{N-1+r}$ и их соответствующими конечными разностями.
Общая идея построения систем функций, ортонормированных по Соболеву и порожденных заданной ортонормированной системой функций дискретной переменной, была рассмотрена в пункте \ref{sobolev-common}. В данном параграфе речь пойдет о конструировании системы полиномов $\tau _{r,n}^{\alpha,\beta}(x,N)$, $n=0,1,\ldots,N-1+r$,  ортонормированных  по Соболеву относительно скалярного произведения вида \eqref{sob-tcheb-difference-1.4} и порожденных классическими полиномами Чебышева дискретной переменной,


%\section{Системы дискретных функций, ортонормированных по Соболеву, порожденные  ортонормированными  функциями}
%Перейдем к конструированию дискретных функций, ортонормированных по Соболеву относительно скалярного произведения \eqref{sob-tcheb-difference-1.4}, порожденных заданной системой $\{\psi_k(x)\}_{k=0}^{N-1}$, ортонормированной на дискретном множестве $\Omega_{N}=\{0,1,\ldots,N-1\}$ с весом $\mu(x)$. Для этого нам понадобятся некоторые обозначения и понятия. Если целое $k\ge0$, то положим $a^{[k]}=a(a-1)\cdots (a-k+1)$, $a^{[0]}=1$ и рассмотрим следующие функции
%\begin{equation}\label{sob-tcheb-difference-2.1}
%\psi_{r,k}(x)={x^{[k]}\over k!},\, k=0,1,\ldots,r-1,
%\end{equation}
%\begin{equation}\label{sob-tcheb-difference-2.2}
%\psi_{r,k+r}(x)=
%\left.
%\begin{cases}
%\frac{1}{(r-1)!}
%\sum\limits_{t=0}^{x-r}(x-1-t)^{[r-1]}\psi_{k}(t),
%&x \in \Omega_{N+r}\setminus\Omega_r,\\
%0,&x \in \Omega_r,
%\end{cases}
%\right\},
%k \in \Omega_N,
%\end{equation}
%которые определены на сетке $\Omega_{N+r}$. Рассмотрим некоторые важные разностные  свойства системы функций $\psi_{r,k}(x)$, определенных равенствами \eqref{sob-tcheb-difference-2.1} и \eqref{sob-tcheb-difference-2.2}. Введем оператор  конечной разности $\Delta f$: $\Delta f(x)=f(x+1)-f(x)$ и положим $\Delta^{\nu+1} f(x)=\Delta\Delta^\nu f(x)$. Имеет место следующая
%\begin{lemma}Имеют место равенства $(r \ge 1,  \nu \ge 1, x \in \Omega_{N+r-\nu})$:
%\begin{equation}\label{sob-tcheb-difference-2.3}
%\Delta^\nu \psi_{r,k}(x)=
%\begin{cases}
%\psi_{r-\nu,k-\nu}(x), &\nu \le \min\{k,r\},\\
%\Delta^{\nu-r}\psi_{k-r}(x), &\nu > r, k \ge r,\\
%0, &\nu, r > k,
%\end{cases}
%\end{equation}
%где полагается, что $\psi_{0,k}(x)=\psi_k(x)$.
%\end{lemma}
%
%Пусть $0\le r$ -- целое. Обозначим через $l_\mu^{N+r}$ пространство дискретных функций $f=f(x)$, заданных на сетке $\Omega_{N+r}$, в которых скалярное произведение $\langle f,g \rangle$ определено с помощью равенства \eqref{sob-tcheb-difference-1.4}.
%
% Рассмотрим  задачу об ортогональности, нормированности и полноте в $l_\mu^{N+r}$ системы $\{\psi_{r,k}(x)\}_{k=0}^{N+r-1}$, состоящей из функций, определенных равенствами   \eqref{sob-tcheb-difference-2.1} и \eqref{sob-tcheb-difference-2.2}.
%
% \begin{theorem} Предположим, что функции $\psi_k(x)$, $k=0,1,\ldots,N-1,$ образуют полную в $l^{N}_\mu$ $=l^{N+0}_\mu$ ортонормированную систему  c весом   $\mu(x)$. Тогда система $\{\psi_{r,k}(x)\}_{k=0}^{N+r-1}$, порожденная системой $\{\psi_{k}(x)\}_{k=0}^{N-1}$ посредством равенств \eqref{sob-tcheb-difference-2.1} и \eqref{sob-tcheb-difference-2.2}, полна  в $l_\mu^{N+r}$ и ортонормирована относительно скалярного произведения \eqref{sob-tcheb-difference-1.4}.
% \end{theorem}
%
%Систему функций $\psi_{r,k}(t)$, $k=0,1,\ldots,N+r-1$, мы будем называть системой, ортонормированной по Соболеву относительно скалярного произведения \eqref{sob-tcheb-difference-1.4} и порожденной системой $\{\psi_k(x)\}_{k=0}^{N-1}$.
%
%Из теоремы 1 следует, что система дискретных функций $\psi_{r,k}(t)$, $k=0,1,\ldots,N+r-1$,
%является ортонормированным базисом (ОНБ) в пространстве $l_\mu^{N+r}$, поэтому для произвольной функции $f(x)\in l_\mu^{N+r}$ мы можем записать равенство
%\begin{equation}\label{sob-tcheb-difference-2.8}
%f(x)= \sum_{k=0}^{N+r-1}f_{r,k} \psi_{r,k}(x),
%\quad f_{r,k}=\langle f,\psi_{r,k}\rangle,
%\end{equation}
%которое представляет собой конечный ряд Фурье функции $f(x)\in l_\mu^{N+r}$ по системе
%$\{\psi_{r,k}(t)\}_{k=0}^{N+r-1}$, ортонормированной по Соболеву. Поскольку коэффициенты Фурье $f_{r,k}$ имеют  вид
%\begin{gather}
%\label{sob-tcheb-difference-2.9}
%f_{r,k} =\sum_{\nu=0}^{r-1}\Delta^\nu f(0)\Delta^\nu\psi_{r,k}(0)=\Delta^kf(0),\, k=0,\ldots, r-1,\\
%\notag
%f_{r,k}= \sum_{j=0}^{N-1}\Delta^rf(j)\Delta^r\psi_{r,k}(j)\mu(j)=\\
%\label{sob-tcheb-difference-2.10}
%\sum_{j=0}^{N-1}\Delta^rf(j)\psi_{k-r}(j)\mu(j),\, k=r,\ldots, N+r-1,
%\end{gather}
%то равенство \eqref{sob-tcheb-difference-2.8} можно переписать в следующем  смешанном виде
%\begin{equation}\label{sob-tcheb-difference-2.11}
%f(x)= \sum_{k=0}^{r-1}\Delta^kf(0){x^{[k]}\over k!} +\sum_{k=r}^{N+r-1}f_{r,k} \psi_{r,k}(x),\, x\in \Omega_{N+r},
%\end{equation}
%где $f_{r,k}=\sum_{j=0}^{N-1}\Delta^rf(j)\psi_{k-r}(j)\mu(j)$.
%В связи с этим ряд Фурье по системе $\{\psi_{r,k}(t)\}_{k=0}^{N+r-1}$ мы будем, следуя \cite{sob-tcheb-difference-Shar9} --  \cite{meixner-18}, называть смешанным рядом по исходной ортонормированной системе $\{\psi_{k}(t)\}_{k=0}^{N-1}$.
%
%Отметим некоторые важные свойства смешанных рядов  \eqref{sob-tcheb-difference-2.11} и их частичных сумм вида
%\begin{equation}\label{sob-tcheb-difference-2.12}
% \mathcal{Y}_{r,n}(f,x)= \sum_{k=0}^{r-1}\Delta^kf(0){x^{[k]}\over k!} +\sum_{k=r}^{n}f_{r,k} \psi_{r,k}(x).
%  \end{equation}
%Из \eqref{sob-tcheb-difference-2.11} и \eqref{sob-tcheb-difference-2.12} с учетом равенств \eqref{sob-tcheb-difference-2.3} мы можем записать ($0\le\nu\le r-1$, $x\in \Omega_{N+r-\nu}$)
%\begin{equation}\label{sob-tcheb-difference-2.13}
%\Delta^\nu f(x)= \sum_{k=0}^{r-\nu-1}\Delta^{k+\nu}f(0){x^{[k]}\over k!} +\sum_{k=r-\nu}^{N+r-\nu-1} f_{r,k+\nu} \psi_{r-\nu,k}(x),
%\end{equation}
%\begin{equation}\label{sob-tcheb-difference-2.14}
%\Delta^\nu\mathcal{Y}_{r,n}(f,x)= \sum_{k=0}^{r-\nu-1}\Delta^{k+\nu}f(0){x^{[k]}\over k!} +\sum_{k=r-\nu}^{n-\nu} f_{r,k+\nu} \psi_{r-\nu,k}(x),
%\end{equation}
%\begin{equation}\label{sob-tcheb-difference-2.15}
%\Delta^\nu\mathcal{Y}_{r,n}(f,x) = \mathcal{Y}_{r-\nu,n-\nu}(\Delta^\nu f,x),
%\end{equation}
%\begin{equation}\label{sob-tcheb-difference-2.16}
%\Delta^\nu\mathcal{Y}_{r,n}(f,0) = \Delta^\nu f(0), \quad 0\le \nu\le r-1.
%\end{equation}
%
%
%
%
%Кроме того, из \eqref{sob-tcheb-difference-2.3} и \eqref{sob-tcheb-difference-2.11} имеем
% \begin{equation}\label{sob-tcheb-difference-2.17}
% \Delta^r f(x)= \sum_{k=0}^{N-1} f_{r,r+k} \psi_{k}(x), \quad x\in \Omega_{N}.
%  \end{equation}
%
%



\section{Некоторые сведения о полиномах Чебышева,\\ ортогональных на равномерной  сетке} \label{sob-tcheb-difference-s3}
При конструировании полиномов, ортогональных по Соболеву и порожденных классическими многочленами Чебышева, ортогональными на равномерной сетке, нам понадобится ряд свойств этих многочленов, которые мы приведем в настоящем параграфе.
 Пусть $N$ -- натуральное, $\alpha$, $\beta$ -- произвольные  числа. Положим
\begin{equation}
\rho(x)=\rho(x;\alpha,\beta,N)={\Gamma(x+\beta+1)
\Gamma(N-x+\alpha)\over \Gamma(x+1)\Gamma(N-x)}, \label{sob-tcheb-difference-3.1}
\end{equation}
\begin{equation}\label{sob-tcheb-difference-3.2}
T_n^{\alpha,\beta}(x,N)={(-1)^n\over n!(N-1)^{[n]}\rho(x)}\Delta^n
\left\{\rho(x)(x-N-\alpha)^{[n]}x^{[n]}\right\},
\end{equation}
 где $\Delta^nf(x)$ -- конечная разность $n$-го порядка функции
     $f(x)$ в точке $x$, т.е. $\Delta^0f(x)=f(x)$,
$\Delta^1f(x)=\Delta f(x)=f(x+1)-f(x)$, $\Delta^nf(x)=\Delta
\Delta^{n-1}f(x)$ $(n\ge1)$, $a^{[0]}=1$,
$a^{[k]}=a(a-1)\cdots(a-k+1)$ при $k\ge1$. Для каждого $0\le n\le
N-1$ равенство \eqref{sob-tcheb-difference-3.2} определяет \cite{sob-tcheb-difference-Cheb5}, алгебраический полином степени $n$,   для которого
$$
T_n^{\alpha,\beta}(N-1,N)={n+\alpha\choose n},\qquad
T_n^{\alpha,\beta}(0,N)=(-1)^n{n+\beta\choose n}.
$$
Полные доказательства приведенных ниже свойств полиномов Чебышева $T_n^{\alpha,\beta}(x,N)$
можно найти, например, в  \cite{meixner-22}. Прежде всего, отметим, что полиномы  $T_n^{\alpha,\beta}(x,N)$ допускают  следующее явное представление
\begin{equation}
T_n^{\alpha,\beta}(x,N)=(-1)^n{\Gamma(n+\beta+1)\over
n!}\sum_{k=0}^n(-1)^k {n^{[k]}(n+\alpha+\beta+1)_kx^{[k]}\over
\Gamma(k+\beta+1) k!(N-1)^{[k]}}. \label{sob-tcheb-difference-3.3}
\end{equation}
Если $\alpha$,
$\beta>-1$, то полиномы $T_n^{\alpha,\beta}(x,N)\quad (0\le n\le
N-1)$ образуют ортогональную  с весом $\rho(x)$ (см. \eqref{sob-tcheb-difference-3.1}) систему  на множестве
$\Omega_N=\{0,1,\ldots,N-1\}$, точнее
\begin{equation}
\sum_{x\in\Omega_N}\mu(x)T_n^{\alpha,\beta}(x,N)T_m^{\alpha,\beta}(x,N)
=h_{n,N}^{\alpha,\beta}\delta_{nm},\label{sob-tcheb-difference-3.4}
\end{equation}
где  $\delta_{nm}$ -- символ Кронекера,
\begin{equation}
 \mu(x)=\mu(x;\alpha,\beta,N)={\Gamma(N)2^{\alpha+\beta+1} \over
\Gamma(N+\alpha+\beta+1)}\rho(x),\label{sob-tcheb-difference-3.5}
\end{equation}
 \begin{equation}
h_{n,N}^{\alpha,\beta}={(N+n+\alpha+\beta)^{[n]}\over
(N-1)^{[n]}}{\Gamma(n+\alpha+1)\Gamma(n+\beta+1)
2^{\alpha+\beta+1}\over
n!\Gamma(n+\alpha+\beta+1)(2n+\alpha+\beta+1)}. \label{sob-tcheb-difference-3.6}
\end{equation}
 При $n=0$  произведение $(\alpha+\beta+1)\Gamma(\alpha+ \beta+1)$
следует заменить на $\Gamma(\alpha+\beta+2)$. Для $0\le n\le N-1$
положим
 \begin{equation}
  \tau_n^{\alpha,\beta}(x)=\tau_n^{\alpha,\beta}(x,N)=
\left\{h_{n,N}^{\alpha,\beta}\right\}^{-1/2}
T_n^{\alpha,\beta}(x,N).\label{sob-tcheb-difference-3.7}
\end{equation}
 Очевидно, если $0\le n,m\le N-1$, то
\begin{equation} \sum_{x=0}^{N-1}\mu(x)\tau_n^{\alpha,\beta}(x,N)
\tau_m^{\alpha,\beta}(x,N)=\delta_{nm}.\label{sob-tcheb-difference-3.8}
\end{equation}
Другими словами, многочлены $\tau_n^{\alpha,\beta}(x,N)$ $(0\le n\le N-1)$
образуют ортонормированную с весом $\mu(x)$ систему на
$\Omega_N$.

Формула Кристоффеля -- Дарбу  для многочленов Чебышева  имеет следующий вид:
\begin{multline}
\mathcal{K}_{n,N}^{\alpha,\beta}(x,y)=\sum_{k=0}^n\tau_k^{\alpha,\beta}(x)\tau_k^{\alpha,\beta}(y)=
\sum_{k=0}^n{T_k^{\alpha,\beta}(x)T_k^{\alpha,\beta}(y)\over
h_{k,N}^{\alpha,\beta}}=\\
{(N-1)^{[n+1]}\over
(N+n+\alpha+\beta)^{[n]}} {2^{-\alpha-\beta-1}\over
2n+\alpha+\beta+2} {\Gamma(n+2)\Gamma(n+\alpha+\beta+2)\over
\Gamma(n+\alpha+1)\Gamma(n+\beta+1)}\times\\
{T_{n+1}^{\alpha,\beta}(x)T_n^{\alpha,\beta}(y)-
T_n^{\alpha,\beta}(x)T_{n+1}^{\alpha,\beta}(y)\over x-y}.\label{sob-tcheb-difference-3.9}
\end{multline}
Поскольку $\Delta a^{[k]}=ka^{[k-1]}$, то из \eqref{sob-tcheb-difference-3.3} находим
\begin{multline}
(n+1)T_{n+1}^{\alpha,\beta}(x,N)+(n+\beta+1)T_n^{\alpha, \beta}(x,N)=\\
{2n+\alpha+\beta+2\over N-1}xT_n^{\alpha,\beta+1}(x-1,N-1). \label{sob-tcheb-difference-3.10}
\end{multline}
 Из равенства $\mu(N-1-x;\beta,\alpha,N)=\mu(x;\alpha,\beta,N)$, непосредственно
вытекающего из  соотношения ортогональности \eqref{sob-tcheb-difference-3.4}, следует,
что при $\alpha,\beta>-1$
 \begin{equation}
T_n^{\alpha,\beta}(x,N)=(-1)^nT_n^{\beta,\alpha}(N-1-x,N). \label{sob-tcheb-difference-3.11}
 \end{equation}
  Поскольку обе части этого равенства аналитичны
относительно $\alpha$ и $\beta$, то оно справедливо для произвольных
$\alpha$ и $\beta$. Из \eqref{sob-tcheb-difference-3.10} и \eqref{sob-tcheb-difference-3.11} имеем также следующее
     равенство
\begin{multline}
(n+\alpha+1)T_n^{\alpha,\beta}(x,N)-(n+1)T_{n+1}^{\alpha,\beta}(x,N)=\\
{2n+\alpha+\beta+2\over   N-1}(N-1-x)T_n^{\alpha+1,
\beta}(x,N-1). \label{sob-tcheb-difference-3.12}
\end{multline}
  Непосредственно из явной формулы \eqref{sob-tcheb-difference-3.3} мы можем вывести следующее
     полезное равенство
\begin{equation}
\Delta^m T_n^{\alpha,\beta}(x,N)={(n+\alpha+\beta+1)_m\over
(N-1)^{[m]}} T_{n-m}^{\alpha+m,\beta+m}(x,N-m),\label{sob-tcheb-difference-3.13}
\end{equation}
где $(a)_0=1$, $(a)_k=a(a+1)\cdots(a+k-1)$ при $k\ge1$. Если $\beta$
такое целое число, что $-n\le\beta\le-1$, то из \eqref{sob-tcheb-difference-3.13} выводим также
\begin{equation}
T_n^{\alpha,\beta}(x,N)={(n+\beta)!\over n!}
{(n+\alpha)^{[-\beta]}x^{[-\beta]}\over (N-1)^{[-\beta]}}
T_{n+\beta}^{\alpha,-\beta}(x+\beta,N+\beta), \label{sob-tcheb-difference-3.14}
 \end{equation}
 а если $\alpha$ и $\beta$ -- целые, $-n\le\beta\le-1$, $-(n+\beta)\le\alpha\le-1$,
$N\ge2$, то
\begin{equation} T_n^{\alpha,\beta}(x,N)={(-1)^\alpha
x^{[-\beta]}(N-x-1)^{[-\alpha]}\over
(N-1)^{[-\beta]}(N-1+\beta)^{[-\alpha]}}
T_{n+\alpha+\beta}^{-\alpha,-\beta}(x+\beta,N+\alpha+\beta).
\label{sob-tcheb-difference-3.15}
\end{equation}
Разностная формула Родрига \eqref{sob-tcheb-difference-3.1} допускает следующее
обобщение
\begin{multline}
\rho(x+m;\alpha,\beta,N+m)T_n^{\alpha,\beta}(x+m,N+m)=\\
{(-1)^m\over n^{[m]}(N)_m}\Delta^m
\left\{\rho(x;\alpha+m,\beta+m,N)T_{n-m}^{\alpha+m,\beta+m}(x,N)\right\},\label{sob-tcheb-difference-3.16}
\end{multline}
 которое, впрочем, непосредственно вытекает из \eqref{sob-tcheb-difference-3.1}.
Заменяя здесь  $m$ на $\nu$,  $\alpha$ и $\beta$  на $m-\nu$, $n$ на
$k+\nu-m$, мы можем также записать
\begin{multline}
\Delta^\nu\{(x+1)_m(N-x)_mT_{k-m}^{m,m}(x,N)\}=\\
(-1)^\nu(k+\nu-m)^{[\nu]}(N+\nu-1)^{[\nu]}(x+1+\nu)_{m-\nu}(N-x)_{m-\nu}T_{k+\nu-m}^{m-\nu,m-\nu}(x+\nu,N+\nu).
\label{sob-tcheb-difference-3.17}
\end{multline}
 Если в равенстве \eqref{sob-tcheb-difference-3.13}  мы заменим $\alpha$, $\beta$ и $n$,
     соответственно, на   $\alpha-m$, $\beta-m$ и $k+m$, то придем
     к формуле
\begin{equation}
\Delta^m T_{k+m}^{\alpha-m,\beta-m}(x,N)={(k+\alpha+\beta)^{[m]}
\over (N-1)^{[m]}}T_k^{\alpha,\beta}(x,N-m).\label{sob-tcheb-difference-3.18}
\end{equation}

При $\alpha, \beta > -1$ полиномы $\tau_n^{\alpha,\beta}(x,N)$, $n=2,3,\ldots,N-1$, связаны трехчленной рекуррентной формулой, которая в случае $\alpha=\beta$ имеет следующий вид:
\begin{equation}
\label{sob-tcheb-difference-3.19}
\tau_n^{\alpha,\alpha}(x,N)=
\hat\kappa^{\alpha,\alpha}_n(2x-N+1)\tau_{n-1}^{\alpha,\alpha}(x,N)-\hat\mu^{\alpha,\alpha}_n\tau_{n-2}^{\alpha,\alpha}(x,N),
\end{equation}
где
\begin{gather*}
\hat\kappa^{\alpha,\alpha}_n= \left[\frac{(2n+2\alpha-1)(2n+2\alpha+1)}{(N+n+2\alpha)(N-n)n(n+2\alpha)}\right]^\frac12,\\
\hat\mu^{\alpha,\alpha}_n=
\left[\frac{N+n+2\alpha-1}{N+n+2\alpha}\,\frac{N-n+1}{N-n}\,\frac{n-1}n\,
\frac{n+2\alpha-1}{n+2\alpha}\,\frac{2n+2\alpha+1}{2n+2\alpha-3}\right]^\frac12,\\
\tau_0^{\alpha,\alpha}(x,N)=\left[\frac{\Gamma(2\alpha+2)}{2^{2\alpha+1}\Gamma(\alpha+1)^2}\right]^\frac12,\\
\tau_1^{\alpha,\alpha}(x,N)=\left[\frac{(2\alpha+3)\Gamma(2\alpha+2)}{(N-1)(N+2\alpha+1)2^{2\alpha+1}\Gamma(\alpha+1)^2
}\right]^\frac12
(2\,x\,-N+1).
\end{gather*}


В работе \cite{sob-tcheb-difference-Shar16} %[2, ШИИ, Асимпт.св-ва и вес.оценки Чеб.-Хана]
 показано, что для любых целых $\alpha,\beta \geq 0$ имеет место асимптотическая формула
\begin{equation}\label{asymptf}
 T_{n}^{\alpha,\beta}\left( \frac{N-1}{2}(1+t) \right) = P_{n}^{\alpha,\beta}(t)+r_{n,N}^{\alpha,\beta}(t),
\end{equation}
в которой $P_{n}^{\alpha,\beta}(t)$ -- классический полином Якоби, а для остаточного члена при $1 \leq n \leq a\sqrt{N}, (a>0)$, справедливы оценки

\begin{equation}\label{asymptRest1}
  \max\limits_{0\leq t \leq 1} |r_{n,N}^{\alpha,\beta}(t)| \leq c(\alpha,\beta,a)\,n^{\alpha+1/2},
\end{equation}
\begin{equation}\label{asymptRest1}
  |r_{n,N}^{\alpha,\beta}(\cos{\theta})| \leq c(\alpha,\beta,a,c)\,\theta^{-\alpha-1/2}, \quad \left( cn^{-1} \leq \theta \leq \pi/2\right).
\end{equation}
В качестве следствия этой формулы, при тех же ограничениях на $n$ и $N$, получена весовая оценка
\begin{equation}\label{weightEst}
  \left| T_{n}^{\alpha,\beta}\left(\frac{N-1}{2}(1+t)\right)\right| \leq
    c(\alpha,\beta,a)\,n^{-1/2}\,
    \left[  \sqrt{1-t} + \frac1n \right]^{-\alpha-1/2}
    \left[  \sqrt{1+t} + \frac1n \right]^{-\beta-1/2}.
\end{equation}











\section{Ортогональные по Соболеву полиномы, порожденные полиномами Чебышева, ортогональными на равномерной сетке }
Из равенства \eqref{sob-tcheb-difference-3.8} следует, что если $\alpha,\beta>-1$, то полиномы $\tau_n^{\alpha,\beta}(x,N)$ $(n=0,1,\ldots, N-1)$
образуют ортонормированную на $\Omega_N$ с весом $\mu(x)$ (см. \eqref{sob-tcheb-difference-3.5}) систему.  Эта система порождает для $x \in \Omega_{N+r}\setminus\Omega_r$ систему полиномов $\tau_{r,k+r}^{\alpha,\beta}(x,N)$ $(k=0, 1,\ldots, N-1)$, определенных равенством
 \begin{equation}\label{sob-tcheb-difference-4.1}
\tau_{r,k+r}^{\alpha,\beta}(x,N)=\frac{1}{(r-1)!}
\sum\limits_{t=0}^{x-r}(x-1-t)^{[r-1]}\tau_{k}^{\alpha,\beta}(t,N).
\end{equation}
Кроме того, определим полиномы
 \begin{equation}\label{sob-tcheb-difference-4.2}
\tau_{r,k}^{\alpha,\beta}(x,N)={x^{[k]}\over k!},\, k=0,1,\ldots,r-1.
\end{equation}
Покажем, что полином $\tau_{r,k+r}^{\alpha,\beta}(x,N)$ обращается в нуль в узлах $x \in \Omega_r$.  С этой целью мы воспользуемся дискретным аналогом формулы Тейлора:
\begin{equation}\label{sob-tcheb-difference-4.3}
F(x)=Q_{r-1}(F,x) + {1\over(r-1)!}\sum_{t=0}^{x-r} (x-1-t)^{[r-1]}\Delta^rF(t),
\,\,
x \in \Omega_{N+r}\setminus\Omega_r,
\end{equation}
где
\begin{equation}\label{sob-tcheb-difference-4.4}
Q_{r-1}(F,x)= F(0)+{\Delta F(0)\over 1!}x+{\Delta^2 F(0)\over 2!}
x^{[2]}+\ldots+{\Delta^{r-1} F(0)\over (r-1)!}x^{[r-1]}.
\end{equation}
Так как для функции $F(x)=x^{[l+r]}$, где целое $l\ge0$, имеем $\Delta^{r}F(x)=(l+r)^{[r]}x^{[l]}$ и $Q_{r-1}(F,x)\equiv0$, то из \eqref{sob-tcheb-difference-4.3} следует, что
\begin{equation}\label{sob-tcheb-difference-4.5}
\frac{x^{[l+r]}}{(l+r)^{[r]}}=
\frac{1}{(r-1)!}\sum_{t=0}^{x-r} (x-1-t)^{[r-1]}t^{[l]}.
\end{equation}
С другой стороны, в силу \eqref{sob-tcheb-difference-3.3} и \eqref{sob-tcheb-difference-3.7} полиномы  $\tau_{k}^{\alpha,\beta}(t,N)$ можно представить в виде линейной комбинации функций вида $t^{[l]}$. Подставляя это представление в \eqref{sob-tcheb-difference-4.1} и учитывая \eqref{sob-tcheb-difference-4.5}, получим:
%С другой стороны, функция $x^{[l+r]}$ обращается в нуль в узлах $x \in \Omega_r$, поэтому наше утверждение вытекает из определения \eqref{sob-tcheb-difference-4.1} и того, что полином  $\tau_{k}^{\alpha,\beta}(t,N)$ в силу \eqref{sob-tcheb-difference-3.3} и \eqref{sob-tcheb-difference-3.7} можно представить в виде линейной комбинации функций вида $t^{[l]}$:
\begin{multline}\label{sob-tcheb-difference-tauRepresent}
\tau_{r,k+r}^{\alpha, \beta}(x,N)=
\frac{1}{(r-1)!}
\sum\limits_{t=0}^{x-r}(x-1-t)^{[r-1]}
\Bigl(
c_1(k)\sum\limits_{l=0}^{k}c_2(k,l)t^{[l]}
\Bigr)=\\
c_1(k)\sum\limits_{l=0}^{k}
c_2(k,l)\frac{x^{[l+r]}}{(l+r)!},
\end{multline}
где $c_1(k)=c_1(k,N; \alpha, \beta) = (-1)^k\Bigl(h_{k,N}^{\alpha, \beta}\Bigr)^{-\frac{1}{2}} \frac{\Gamma(k+\beta+1)}{k!}$,
$c_2(k,l)=c_2(k,l,N; \alpha, \beta) = (-1)^l \frac{k^{[l]}(k+\alpha+\beta+1)_l}{\Gamma(l+\beta+1)(N-1)^{[l]}}$.
Для доказательства нашего утверждения остается только заметить, что функции $x^{[l+r]}$, $l \ge 0$, обращаются в нуль в узлах $x \in \Omega_r$.

Используя это свойство полиномов $\tau_{r,k+r}^{\alpha, \beta}(x,N)$, из теоремы \ref{meitheo1} и соотношения \eqref{sob-tcheb-difference-3.8} непосредственно выводим  следующий результат.

\begin{theorem} Если $\alpha,\beta>-1$, то система полиномов $\tau_{r,k}^{\alpha,\beta}(x,N)$ $(k=0, 1,\ldots, N-1+r)$, порожденная многочленами Чебышева $\tau_n^{\alpha,\beta}(x,N)\quad(n=0,1,\ldots, N-1)$ посредством равенств \eqref{sob-tcheb-difference-4.1} и \eqref{sob-tcheb-difference-4.2}, полна  в $l_\mu^{N+r}$ и ортонормирована относительно скалярного произведения \eqref{sob-tcheb-difference-1.4}.
 \end{theorem}

\section{Дальнейшие свойства полиномов $\tau_{r,k}^{\alpha,\beta}(x,N)$ }

Перейдем к исследованию дальнейших свойств полиномов $\tau_{r,k}^{\alpha,\beta}(x,N)$ $(k=0, 1,\ldots, N-1+r)$. Речь, в первую очередь,  идет о том, чтобы получить представление полиномов   $\tau_{r,k}^{\alpha,\beta}(x,N)$, которое не содержит знаков суммирования с переменным верхним пределом типа \eqref{sob-tcheb-difference-4.1}. С этой целью  применим  формулу \eqref{sob-tcheb-difference-4.3}    к полиному $F(x)=T_{k+r}^{\alpha-r,\beta-r}(x,N+r)$ и запишем
 \begin{equation}\label{sob-tcheb-difference-5.1}
 F(x)=Q_{r-1}(F,x)+ {1\over(r-1)!}\sum_{t=0}^{x-r} (x-1-t)^{[r-1]}\Delta^rT_{k+r}^{\alpha-r,\beta-r}(t,N+r).
\end{equation}
Вместо $\Delta^rT_{k+r}^{\alpha-r,\beta-r}(x,N+r)$ подставим его значение, которое согласно формуле \eqref{sob-tcheb-difference-3.18} равно ${(k+\alpha+\beta)^{[r]}
\over (N-1+r)^{[r]}}T_k^{\alpha,\beta}(x,N)$. Тогда из \eqref{sob-tcheb-difference-5.1} получим
 \begin{equation}\label{sob-tcheb-difference-5.2}
F(x)-Q_{r-1}(F,x)={(k+\alpha+\beta)^{[r]}
\over (N-1+r)^{[r]}}{1\over(r-1)!}\sum_{t=0}^{x-r} (x-1-t)^{[r-1]}T_k^{\alpha,\beta}(t,N)
\end{equation}
Сопоставляя \eqref{sob-tcheb-difference-4.1} и \eqref{sob-tcheb-difference-5.1} с \eqref{sob-tcheb-difference-3.7}, находим
\begin{equation}\label{sob-tcheb-difference-5.3}
{(k+\alpha+\beta)^{[r]}\over (N-1+r)^{[r]}}\left\{h_{k,N}^{\alpha,\beta}\right\}^{1/2}\tau_{r,k+r}^{\alpha,\beta}(x,N)=
F(x)-Q_{r-1}(F,x).
\end{equation}
Если  $(k+\alpha+\beta)^{[r]}\ne0$, то из \eqref{sob-tcheb-difference-5.3} имеем $\left(F(x)=T_{k+r}^{\alpha-r,\beta-r}(x,N+r)\right)$
\begin{equation}\label{sob-tcheb-difference-5.4}
\tau_{r,k+r}^{\alpha,\beta}(x,N)= { (N-1+r)^{[r]}\over(k+\alpha+\beta)^{[r]}}\left\{h_{k,N}^{\alpha,\beta}\right\}^{-1/2}
[F(x)-Q_{r-1}(F,x)].
\end{equation}
Далее, в силу \eqref{sob-tcheb-difference-3.13}
\begin{equation}\label{sob-tcheb-difference-5.5}
\Delta^m T_{k+r}^{\alpha-r,\beta-r}(0,N+r)={(k+\alpha+\beta-r+1)_m\over
(N-1+r)^{[m]}} T_{k+r-m}^{\alpha-r+m,\beta-r+m}(0,N+r-m),
\end{equation}
а из \eqref{sob-tcheb-difference-3.3} имеем
\begin{equation}\label{sob-tcheb-difference-5.6}
 T_{k+r-m}^{\alpha-r+m,\beta-r+m}(0,N+r-m)=
 \frac{(-1)^{k+r-m}\Gamma(k+\beta+1)}{\Gamma(m-r+\beta+1)(k+r-m)!}.
\end{equation}
Из \eqref{sob-tcheb-difference-5.5} и \eqref{sob-tcheb-difference-5.6} следует, что $(0\le m\le r-1)$
$$
A_{r,k,m}=A_{r,k,m}(\alpha,\beta)=\Delta^m T_{k+r}^{\alpha-r,\beta-r}(0,N+r)=
$$
\begin{equation}\label{sob-tcheb-difference-5.7}
{(k+\alpha+\beta-r+1)_m\over
(N-1+r)^{[m]}}\frac{(-1)^{k+r-m}\Gamma(k+\beta+1)}{\Gamma(m-r+\beta+1)(k+r-m)!}.
\end{equation}
 Равенства \eqref{sob-tcheb-difference-4.4}  и  \eqref{sob-tcheb-difference-5.7}, взятые вместе, дают
\begin{equation}\label{sob-tcheb-difference-5.8}
F(x)-Q_{r-1}(F,x)=T_{k+r}^{\alpha-r,\beta-r}(x,N+r)-\sum_{m=0}^{r-1}\frac{A_{r,k,m}x^{[m]}}{m!}.
 \end{equation}
 Подставим это выражение в \eqref{sob-tcheb-difference-5.4}, тогда при $(k+\alpha+\beta)^{[r]}\ne0$ находим
\begin{equation}\label{sob-tcheb-difference-5.9}
\tau_{r,k+r}^{\alpha,\beta}(x,N)=
{(N-1+r)^{[r]}\over(k+\alpha+\beta)^{[r]}\left\{h_{k,N}^{\alpha,\beta}\right\}^{\frac12}}
\left[T_{k+r}^{\alpha-r,\beta-r}(x,N+r)-\sum_{m=0}^{r-1}\frac{A_{r,k,m}x^{[m]}}{m!}\right].
\end{equation}

Другое важное представление для полиномов $\tau_{r,k+r}^{\alpha,\beta}(x,N)$ было получено в предыдущем пункте (см. \eqref{sob-tcheb-difference-tauRepresent}).
%можно получить, если мы обратимся к равенствам \eqref{sob-tcheb-difference-3.3} и \eqref{sob-tcheb-difference-3.7} и запишем
%\begin{equation}\label{sob-tcheb-difference-5.10}
%\tau_{k}^{\alpha,\beta}(x,N)=(-1)^k{\Gamma(k+\beta+1)\over
%k!\left\{h_{k,N}^{\alpha,\beta}\right\}^{1/2}}\sum_{l=0}^k(-1)^l {k^{[l]}(k+\alpha+\beta+1)_lx^{[l]}\over
%\Gamma(l+\beta+1) l!(N-1)^{[l]}}.
%\end{equation}
%Подставим это выражение в \eqref{sob-tcheb-difference-4.1} и воспользуемся равенством \eqref{sob-tcheb-difference-4.5}. Это приводит к следующему явному виду для полиномов  $\tau_{r,k+r}^{\alpha,\beta}(x,N)$:
%\begin{equation}\label{sob-tcheb-difference-5.10}
%\tau_{r,k+r}^{\alpha,\beta}(x,N)=(-1)^k{\Gamma(k+\beta+1)\over
%k!\left\{h_{k,N}^{\alpha,\beta}\right\}^{1/2}}\sum_{l=0}^k(-1)^l {k^{[l]}(k+\alpha+\beta+1)_lx^{[l+r]}\over
%\Gamma(l+\beta+1) (N-1)^{[l]}(l+r)!   }.
%\end{equation}

\section{Некоторые важные частные случаи систем $\{\tau_{r,k}^{\alpha,\beta}(x,N)\}_{k=0}^{N-1+r} $}

Рассмотрим некоторые частные случаи систем  $\{\tau_{r,k}^{\alpha,\beta}(x,N)\}_{k=0}^{N-1+r}$, соответствующие значениям  параметров $\alpha$ и $\beta$, выбранных   следующими двумя способами: $1)\, \alpha=\beta=0$;
$2)\, \alpha$ -- дробное, $\beta=0$.

\subsubsection{Система $\{\tau_{r,k}^{0,0}(x,N)\}_{k=0}^{N-1+r} $}
 Заметим, что если $\beta=\alpha=0$, то  $(k+\alpha+\beta)^{[r]}=k^{[r]}\neq0$ для всех $k\ge r$, а из \eqref{sob-tcheb-difference-5.7} следует, что $A_{r,k,m}(0,0)=0$ при всех  $m=0,1,\dots, r-1$. Поэтому для $r\le k\le N-1$ из \eqref{sob-tcheb-difference-5.9}  имеем
\begin{equation}\label{sob-tcheb-difference-6.1}
\tau_{r,k+r}^{0,0}(x,N)=
{(N-1+r)^{[r]}\over k^{[r]}\left\{h_{k,N}^{0,0}\right\}^{\frac12}}
T_{k+r}^{-r,-r}(x,N+r).
\end{equation}
Далее, если мы обратимся к равенству \eqref{sob-tcheb-difference-3.14}, то можем записать
\begin{equation}\label{sob-tcheb-difference-6.2}
T_{k+r}^{-r,-r}(x,N+r)= \frac{(-1)^rx^{[r]}(N-1+r-x)^{[r]}}{(N-1+r)^{[r]}(N-1)^{[r]}}T_{k-r}^{r,r}(x-r,N-r) .
\end{equation}
Из   \eqref{sob-tcheb-difference-6.1}, \eqref{sob-tcheb-difference-6.2} и \eqref{sob-tcheb-difference-3.7} находим для $r\le k\le N-1$
\begin{multline}\label{sob-tcheb-difference-6.3}
\tau_{r,r+k}^{0,0}(x) =\frac{(-1)^rx^{[r]}(N-1+r-x)^{[r]}}{(N-1)^{[r]}k^{[r]}
\left\{h_{k,N}^{0,0}\right\}^{\frac12}}T_{k-r}^{r,r}(x-r,N-r)=\\
\frac{(-1)^rx^{[r]}(N-1+r-x)^{[r]}}{(N-1)^{[r]}k^{[r]}} \left\{\frac{h_{k-r,N-r}^{r,r}}{h_{k,N}^{0,0}}\right\}^{\frac12}
\tau_{k-r}^{r,r}(x-r,N-r).
\end{multline}
Что касается полиномов $\tau_{r,k+r}^{0,0}(x,N)$  с $0\le k \le r-1$, то в силу \eqref{sob-tcheb-difference-tauRepresent}
\begin{equation}\label{sob-tcheb-difference-6.4}
\tau_{r,k+r}^{0,0}(x,N)={(-1)^k\over
\left\{h_{k,N}^{0,0}\right\}^{1/2}}\sum_{l=0}^k {(-1)^lk^{[l]}(k+1)_lx^{[l+r]}\over
 l!^2(N-1)^{[l]}(l+r)^{[r]}   }.
\end{equation}
Наконец, если $0\le k\le r-1$, то в силу определения \eqref{sob-tcheb-difference-4.2}
\begin{equation}\label{sob-tcheb-difference-6.5}
\tau_{r,k}^{0,0}(x,N)={x^{[k]}\over k!}.
\end{equation}


Из теоремы \ref{meitheo1} следует, что система полиномов $\tau_{r,k}^{0,0}(x,N)\, (k=0,1,\ldots,N+r-1)$
является ортонормированным базисом (ОНБ) в пространстве $l_\mu^{N+r}$, поэтому для произвольной функции $f(x)\in l_\mu^{N+r}$ мы можем записать равенство
\begin{equation}\label{sob-tcheb-difference-6.6}
f(x)= \sum_{k=0}^{N+r-1} f_{r,k} \tau_{r,k}^{0,0}(x,N), \quad
f_{r,k}=\langle f,\tau_{r,k}^{0,0} \rangle,
\end{equation}
которое представляет собой конечный ряд Фурье функции $f(x)\in l_\mu^{N+r}$ по системе
$\{\tau_{r,k}^{0,0}(x,N)\}_{k=0}^{N+r-1}$, ортонормированной по Соболеву относительно скалярного произведения \eqref{sob-tcheb-difference-1.4}. Поскольку коэффициенты Фурье $f_{r,k}$ имеют  вид
\begin{gather}\label{sob-tcheb-difference-6.7}
f_{r,k}=
\sum_{\nu=0}^{r-1}\Delta^\nu f(0)\Delta^\nu \tau_{r,k}^{0,0}(0,N)=\Delta^kf(0),\, k=0,\ldots, r-1,\\
\label{sob-tcheb-difference-6.8}
f_{r,k}=
\sum_{j=0}^{N-1}\Delta^rf(j)\tau_{k-r}^{0,0}(j,N)\mu(j),\, k=r,\ldots, N+r-1,
\end{gather}
то равенство \eqref{sob-tcheb-difference-6.6} можно переписать в следующем  смешанном виде
\begin{equation}\label{sob-tcheb-difference-6.9}
 f(x)= \sum_{k=0}^{r-1}\Delta^kf(0){x^{[k]}\over k!} +\sum_{k=r}^{N+r-1}f_{r,k} \tau_{r,k}^{0,0}(x,N),\, x\in \Omega_{N+r}.
\end{equation}






\subsubsection{Система $\{\tau_{r,k}^{\alpha,0}(x,N)\}_{k=0}^{N-1+r}$}
 Если $\alpha$ -- дробное, а $\beta=0$,  то  $(k+\alpha+\beta)^{[r]}=(k+\alpha)^{[r]}\neq0$ для всех $k\ge 0$, а из \eqref{sob-tcheb-difference-5.7} следует, что $A_{r,k,m}(\alpha,0)=0$ при всех  $m=0,1,\dots, r-1$. Поэтому для $0\le k\le N-1$ из \eqref{sob-tcheb-difference-5.9}  имеем
\begin{equation}\label{sob-tcheb-difference-6.10}
\tau_{r,k+r}^{\alpha,0}(x,N)=
{(N-1+r)^{[r]}\over(k+\alpha)^{[r]}\left\{h_{k,N}^{\alpha,0}\right\}^{\frac12}}
T_{k+r}^{\alpha-r,-r}(x,N+r).
\end{equation}
Кроме того, если $0\le k\le r-1$, то в силу определения \eqref{sob-tcheb-difference-4.2}
\begin{equation}\label{sob-tcheb-difference-6.11}
\tau_{r,k}^{\alpha,0}(x,N)={x^{[k]}\over k!}.
\end{equation}
Конечный ряд Фурье по системе \{$ \tau_{r,k}^{\alpha,0}(x,N)\}_{k=0}^{N-1+r} $  имеет вид
  \begin{equation}\label{sob-tcheb-difference-6.12}
 f(x)= \sum_{k=0}^{r-1}\Delta^kf(0){x^{[k]}\over k!} +\sum_{k=r}^{N+r-1}f_{r,k} \tau_{r,k}^{\alpha,0}(x,N),\, x\in \Omega_{N+r}.
  \end{equation}
где
\begin{equation}\label{sob-tcheb-difference-6.13}
f_{r,k}=
\langle f,\tau_{r,k}^{\alpha,0} \rangle=
\sum_{j=0}^{N-1}\Delta^rf(j)\tau_{k-r}^{\alpha,0}(j,N)\mu(j),\, k=r,\ldots, N+r-1.
 \end{equation}
\section{О разностных и аппроксимативных свойствах частичных сумм Фурье по системе
\{$ \tau_{r,k}^{\alpha,\beta}(x,N)\}_{k=0}^{N-1+r} $}
Основные разностные свойства  сумм Фурье
\begin{equation}\label{sob-tcheb-difference-7.1}
 \mathcal{Y}_{r,n}^{\alpha,\beta}(f,x)= \sum_{k=0}^{r-1}\Delta^kf(0){x^{[k]}\over k!} +\sum_{k=r}^{n}f_{r,k}\tau_{r,k}^{\alpha,\beta}(x,N),
  \end{equation}
по полиномам $\tau_{r,k}^{\alpha,\beta}(x,N)$, где
$f_{r,k}=
\langle f,\tau_{r,k}^{\alpha,\beta} \rangle=
\sum_{j=0}^{N-1}\Delta^rf(j)\tau_{k-r}^{\alpha,\beta}(j,N)\mu(j),\, k=r,\ldots, N+r-1$, выражены равенствами \eqref{meixner-2.13} -- \eqref{meixner-2.15}.
Из \eqref{meixner-2.13} и \eqref{meixner-2.14} мы также можем вывести для $n\ge r>\nu\ge0$ равенство
\begin{equation}\label{sob-tcheb-difference-7.5}
 \Delta^\nu f(x)-\Delta^\nu\mathcal{Y}_{r,n}^{\alpha,\beta}(f,x)= \sum_{k=n-\nu+1}^{N+r-\nu-1} f_{r,k+\nu} \tau_{r-\nu,k}^{\alpha,\beta}(x,N).
  \end{equation}
Равенство  \eqref{sob-tcheb-difference-7.5} дает выражение для погрешности, проистекающей в результате замены конечной разности  $\Delta^\nu f(x)$ ее приближенным значением $\Delta^\nu\mathcal{Y}_{r,n}^{\alpha,\beta}(f,x)$. При решении задачи об оценке этой погрешности возникает вопрос об асимптотических свойствах полиномов $\tau_{r-\nu,k}^{\alpha,\beta}(x,N)$. Этот вопрос, в свою очередь, сводится, как это было показано выше, к задаче об асимптотических свойствах полиномов Чебышева  $\tau_k^{\alpha,\beta}(x,N)$, которые весьма подробно исследованы в работах \cite{sob-tcheb-difference-Shar17,meixner-22}.
Пользуясь результатами, установленными в \cite{sob-tcheb-difference-Shar17,meixner-22}, и формулой Кристоффеля-Дарбу \eqref{sob-tcheb-difference-3.9}, в работе \cite{Haar-Tcheb-Shar15} была исследована задача об оценке отклонения
 $ |\Delta^\nu f(x)-\Delta^\nu\mathcal{Y}_{r,n}^{\alpha,\beta}(f,x)|$   в случае, когда $\alpha=\beta=\nu=0$. 
 Отметим также, что смешанные ряды по полиномам Чебышева $\tau_k^{\alpha,\beta}(x,N)$, исследованные в работах \cite{sob-tcheb-difference-Shar9}, \cite{sob-tcheb-difference-Shar2} и \cite{meixner-12}, в случае $\alpha=\beta=0$ по существу совпадают с рядами Фурье \eqref{sob-tcheb-difference-6.9} по системе $\{ \tau_{r,k}^{0,0}(x,N)\}_{k=0}^{N-1+r} $ (с точностью до некоторых обозначений). Поэтому результаты, полученные в цитируемых работах \cite{sob-tcheb-difference-Shar9,sob-tcheb-difference-Shar2,Haar-Tcheb-Shar11,sob-jac-discrete-Shar17,Haar-Tcheb-Shar13,meixner-12,Haar-Tcheb-Shar15,Haar-Tcheb-Shar16,Haar-Tcheb-Shar19,Haar-Tcheb-Shar18,meixner-18}, касающиеся  приближения дискретных функций частичными суммами смешанных рядов по полиномам Чебышева $\tau_k^{\alpha,\beta}(x,N)$,  непосредственно могут быть использованы для оценки отклонения
 $ |\Delta^\nu f(x)-\Delta^\nu\mathcal{Y}_{r,n}^{0,0}(f,x)|$. Это же замечание справедливо и в случае, когда $\alpha$ -- дробно, а $\beta=0$. Однако в общем случае, когда $\alpha,\beta>-1$, задача об аппроксимативных свойствах сумм Фурье
 $\mathcal{Y}_{r,n}^{\alpha,\beta>-1}(f,x)$ остается мало исследованной.

\section{О представлении решения задачи Коши для разностного уравнения рядами Фурье по функциям, ортогональным по Соболеву}

Как уже отмечалось выше, одним из эффективных подходов решения уравнений различных типов
(дифференциальных, интегральных, разностных и т.д.) является \cite{Haar-Tcheb-SolDmEg, Haar-Tcheb-Tref1, Haar-Tcheb-Tref2} так называемый спектральный метод, основанный на представлении искомого решения рассматриваемого уравнения в виде ряда по подходящей ортонормированной системе функций  и последующего  преобразования его к двойственному виду, в котором вместо искомого решения уравнения фигурируют  неизвестные коэффициенты его разложения по выбранной ортонормированной системе. Мы вернемся к задаче Коши \eqref{sob-tcheb-difference-1.1} и представим ее искомое решение $y$ в виде конечного ряда Фурье
\begin{equation}\label{sob-tcheb-difference-8.1}
 y(x)= \sum_{k=0}^{r-1}\Delta^ky(0){x^{[k]}\over k!} +\sum_{k=0}^{N-1}y_{r,k+r} \psi_{r,k+r}(x),\, x\in \Omega_{N+r},
  \end{equation}
по системе  $\{\psi_{r,k}(x)\}_{k=0}^{N+r-1}$, состоящей из функций, определенных равенствами   \eqref{meixner-2.1} и \eqref{meixner-2.2}, ортонормированных на сетке $\Omega_{N+r}$ по Соболеву относительно скалярного произведения \eqref{sob-tcheb-difference-1.4}. Кроме того, в силу \eqref{meixner-2.13} для $x\in \Omega_{N+r-l} $ и  $0\le l\le r-1$ имеем
\begin{equation}\label{sob-tcheb-difference-8.2}
 \Delta^l y(x)= \sum_{k=0}^{r-l-1}\Delta^{k+l}y(0){x^{[k]}\over k!} +\sum_{k=0}^{N-1} y_{r,r+k} \psi_{r-l,r-l+k}(x),
  \end{equation}
а из \eqref{sob-tcheb-difference-2.17} следует, что
 \begin{equation}\label{sob-tcheb-difference-8.3}
 \Delta^r y(x)= \sum_{k=0}^{N-1} y_{r,r+k} \psi_{k}(x), \quad x\in \Omega_{N},
  \end{equation}
где для коэффициентов $y_{r,k+r}$ имеет место равенство
\begin{equation}\label{sob-tcheb-difference-8.4}
y_{r,k+r}= \langle y,\psi_{r,k+r} \rangle =
\sum_{j=0}^{N-1}\Delta^ry(j)\psi_{k}(j)\mu(j),\, k=0,\ldots, N-1.
\end{equation}
Отметим, что если будут найдены коэффициенты $y_{r,k}$ из  \eqref{sob-tcheb-difference-8.1}  так, чтобы функция $y(x)$ оказалась решением разностного уравнения \eqref{sob-tcheb-difference-1.1}, то мы, очевидно, получим именно то решение этого уравнения, которое удовлетворяет начальным условиям $\Delta^{l}y(0)=y_l, \, l=0,1,\ldots,r-1$. Другими словами, мы получим решение поставленной задачи Коши. При этом важно заметить, что частичная сумма конечного ряда   Фурье \eqref{sob-tcheb-difference-8.1} вида
\begin{equation}\label{sob-tcheb-difference-8.5}
 \mathcal{Y}_{r,n}(y,x)= \sum_{k=0}^{r-1}\Delta^ky(0){x^{[k]}\over k!} +\sum_{k=0}^ny_{r,k+r} \psi_{r,k+r}(x)
  \end{equation}
c $n\ge r$ также удовлетворяет (см.\eqref{sob-tcheb-difference-2.16}) начальным условиям задачи Коши \eqref{sob-tcheb-difference-1.1} и поэтому может быть рассмотрена в качестве приближенного решения этой задачи. В связи с этим попутно возникает вопрос об оценке отклонения приближенного решения $\mathcal{Y}_{r,n}(y,x)$ задачи Коши \eqref{sob-tcheb-difference-1.1} от ее точного решения $y(x)$, представленного в виде \eqref{sob-tcheb-difference-8.1}, другими словами, возникает задача об оценке остатка $|y(x)-\mathcal{Y}_{r,n}(y,x)|$, $x \in \Omega_{N+r}$. При решении этой задачи, в частности, может быть использовано установленное выше равенство \eqref{sob-tcheb-difference-7.5}. На подробном анализе этой проблемы мы здесь не будем останавливаться.

Чтобы завершить переход от уравнения  \eqref{sob-tcheb-difference-1.1} к его двойственному (спектральному) виду,
подставим в левой части равенства \eqref{sob-tcheb-difference-1.1} вместо конечных разностей $\Delta^l y(j)$
их представления из \eqref{sob-tcheb-difference-8.1}-- \eqref{sob-tcheb-difference-8.3} и в результате получим равенство

\begin{equation*}
a_r(x) \sum_{i=0}^{N-1} y_{r,r+i} \psi_{i}(x)+
\sum_{l=0}^{r-1}a_l(x)\sum_{i=0}^{N-1} y_{r,r+i} \psi_{r-l,r-l+i}(x)=
f(x)-\sum_{l=0}^{r-1}a_l(x)\sum_{k=0}^{r-l-1}\Delta^{k+l}y(0){x^{[k]}\over k!},
\end{equation*}
которое можно переписать еще так:
\begin{equation}\label{sob-tcheb-difference-8.6}
\sum_{i=0}^{N-1} y_{r,r+i}\left[a_r(x) \psi_{i}(x)+ \sum_{l=0}^{r-1}a_l(x) \psi_{r-l,r-l+i}(x)\right]=
f(x)-\sum_{l=0}^{r-1}a_l(x)\sum_{k=0}^{r-l-1}\Delta^{k+l}y(0){x^{[k]}\over k!}.
\end{equation}
Вводя обозначения
\begin{equation}\label{sob-tcheb-difference-8.7}
g(x)=f(x)-\sum_{l=0}^{r-1}a_l(x)\sum_{k=0}^{r-l-1}\Delta^{k+l}y(0){x^{[k]}\over k!},
\end{equation}
\begin{equation}\label{sob-tcheb-difference-8.8}
F_i(x)=a_r(x) \psi_{i}(x)+ \sum_{l=0}^{r-1}a_l(x) \psi_{r-l,r-l+i}(x),
\end{equation}
равенство \eqref{sob-tcheb-difference-8.6} можно записать в более компактной форме:
 \begin{equation}\label{sob-tcheb-difference-8.9}
 \sum_{i=0}^{N-1} y_{r,r+i}F_i(x)=g(x), \quad x\in \Omega_N.
  \end{equation}
Тем самым мы пришли  системе $N$ линейных уравнений \eqref{sob-tcheb-difference-8.9} относительно неизвестных коэффициентов $y_{r,r+i}$ ($i=0,1,\ldots, N-1$). Эта система имеет единственное решение, если функции $\{F_i(x)\}_{i=0}^{N-1}$ линейно независимы на множестве $\Omega_N$ или, что то же, если определитель матрицы $[F_i(x)]_{0\le i,x\le N-1}$ отличен от нуля. В этом случае мы можем найти неизвестные коэффициенты $y_{r,r+i}$ ($i=0,1,\ldots, N-1$), решая систему уравнений \eqref{sob-tcheb-difference-8.9}. Эту же задачу можно решить и несколько иначе. А именно, пусть $\{\varphi_k(x)\}_{k=0}^{N-1}$ некоторая ортонормированная с весом $\rho(x)$ система, состоящая из функций $\varphi_k(x)$, заданных на $\Omega_N$. Тогда мы можем каждую из функций $F_i(x)$ ($i=0,1,\ldots, N-1$) разложить в конечный ряд Фурье по системе $\{\varphi_k(x)\}_{k=0}^{N-1}$ и получить представление
 \begin{equation}\label{sob-tcheb-difference-8.10}
F_i(x)= \sum_{k=0}^{N-1}\hat F_{i,k}\varphi_k(x), \quad i, x\in \Omega_N,
  \end{equation}
где $\hat F_{i,k}= \sum_{j=0}^{N-1} F_{i}(j)\varphi_k(j)\rho(j)$.
Аналогично,
\begin{equation}\label{sob-tcheb-difference-8.12}
g(x)= \sum_{k=0}^{N-1}\hat g_{k}\varphi_k(x), \quad  x\in \Omega_N,
  \end{equation}
где $\hat g_{k}= \sum_{j=0}^{N-1} g(j)\varphi_k(j)\rho(j)$.

Подставим в \eqref{sob-tcheb-difference-8.9} вместо $F_{i}(x)$ и $g(x)$ правые части равенств \eqref{sob-tcheb-difference-8.10} и \eqref{sob-tcheb-difference-8.12}. Тогда мы получим
\begin{equation}\label{sob-tcheb-difference-8.14}
 \sum_{k=0}^{N-1}\left(\sum_{i=0}^{N-1} y_{r,r+i}\hat F_{i,k}\right)\varphi_k(x)=\sum_{k=0}^{N-1}\hat g_{k}\varphi_k(x), \quad x\in \Omega_N.
  \end{equation}
Из \eqref{sob-tcheb-difference-8.14}, в свою очередь, получаем систему уравнений
\begin{equation}\label{sob-tcheb-difference-8.15}
 \sum_{i=0}^{N-1} y_{r,r+i}\hat F_{i,k}=\hat g_{k}, \quad k\in \Omega_N,
  \end{equation}
эквивалентную системе \eqref{sob-tcheb-difference-8.9}.

 Можно поставить также задачу о приближенном решении системы \eqref{sob-tcheb-difference-8.15}, <<отбрасывая>>
 из нее коэффициенты $y_{r,r+i}$, для которых $i=n+1,\ldots, N-1$ и коэффициенты
 $\hat g_{k}$ с $k=m+1,\ldots, N-1$, где $n\le m\le N-1$. Другими словами, рассматривается  система
 \begin{equation}\label{sob-tcheb-difference-8.16}
 \sum_{i=0}^n y_{r,r+i}\hat F_{i,k}=\hat g_{k}, \quad k=0,1,\ldots, m,
   \end{equation}
 где $n\le m\le N-1$, которую можно решить, например, методом наименьших квадратов.
 Какой из систем \eqref{sob-tcheb-difference-8.9} или \eqref{sob-tcheb-difference-8.16} при нахождении коэффициентов $y_{r,r+i}$ следует отдать предпочтение зависит, в частности, от того, каков характер погрешностей  значений функции $g(x)$, фигурирующей в правой части \eqref{sob-tcheb-difference-8.9}. Если они являются случайными величинами с законом распределения, близким к нормальному, то можно рекомендовать воспользоваться системой \eqref{sob-tcheb-difference-8.16}, выбрав $n$ и $m$ существенно меньшими, чем $N-1$. С другой стороны, если $N$ не очень велико и значения функции $g(x)$ заданы с удовлетворительной точностью, то можно воспользоваться системой \eqref{sob-tcheb-difference-8.9}.
\section{О  решении задачи Коши для разностного уравнения \eqref{sob-tcheb-difference-1.1} посредством системы $\{ \tau_{r,k}^{0,0}(x,N)\}_{k=0}^{N-1+r} $}
В этом параграфе мы вкратце остановимся на задаче о решении системы линейных уравнений \eqref{sob-tcheb-difference-8.9} в частном случае, когда в качестве исходной  ортонормированной системы  $\{\psi_k(x)\}_{k=0}^{N-1}$ берется система полиномов Чебышева
$\{\tau_k^{0,0}(x,N)\}_{k=0}^{N-1} $, ортонормированная на сетке $\Omega_{N}$ с весом $\mu(x)=\mu(x;0,0,N)=2/N$. Соответствующая  система полиномов
$\{\tau_{r,k}^{0,0}(x,N)\}_{k=0}^{N-1+r}$, ортонормированная по Соболеву на сетке $\Omega_{N+r}$, задается равенствами  \eqref{sob-tcheb-difference-6.3} -- \eqref{sob-tcheb-difference-6.5}, из которых при $0\le l\le r-1$ мы выводим
\begin{equation}\label{sob-tcheb-difference-9.1}
\psi_{r-l,k}(x)=\tau_{r-l,k}^{0,0}(x,N)=\frac{x^{[k]}}{k!},\quad 0\le k\le r-l-1,
   \end{equation}
\begin{multline}\label{sob-tcheb-difference-9.2}
\psi_{r-l,r-l+k}(x)=\tau_{r-l,r-l+k}^{0,0}(x,N)=\\
{(-1)^k\over
\left\{h_{k,N}^{0,0}\right\}^{1/2}}
\sum_{j=0}^k
{(-1)^jk^{[j]}(k+1)_jx^{[j+r-l]}\over
 j!^2(N-1)^{[j]}(j+r-l)^{[r-l]}},\quad 0\le k\le r-l-1,
\end{multline}
\begin{multline}\label{sob-tcheb-difference-9.3}
\psi_{r-l,r-l+k}(x)=\tau_{r-l,r-l+k}^{0,0}(x,N)=
(-1)^{r-l}\left\{\frac{h_{k-r+l,N-r+l}^{r-l,r-l}}{h_{k,N}^{0,0}}\right\}^{\frac12}\times\\
\frac{x^{[r-l]}(N-1+r-l-x)^{[r-l]}}{(N-1)^{[r-l]}k^{[r-l]}}
\tau_{k-r+l}^{r-l,r-l}(x-r+l,N-r+l),\,r-l\le k\le N-1.
\end{multline}
Используя равенства \eqref{sob-tcheb-difference-9.1} -- \eqref{sob-tcheb-difference-9.3} и \eqref{sob-tcheb-difference-8.8}, можно найти численные значения элементов матрицы $[F_i(x)]_{0\le i,x\le N-1}$, если только мы найдем  способ для вычисления значений полиномов Чебышева $\tau_{k-r+l}^{r-l,r-l}(x-r+l,N-r+l)$ ($r-l\le k\le N-1$), фигурирующих в \eqref{sob-tcheb-difference-9.3}. С этой целью мы можем обратиться к рекуррентной формуле \eqref{sob-tcheb-difference-3.19}, в которой $\alpha$ заменим на $\nu=r-l$, а $N$ -- на $N-\nu$. Полученную в результате формулу можно записать в следующем виде, удобном для проведения численных расчетов:
\begin{multline*}
\tau_{k-\nu}^{\nu,\nu}(x-\nu, N-\nu)=
\hat{\kappa}_{k-\nu}^{\nu,\nu}
\Bigl(
\frac{2x}{N+\nu-1}-1
\Bigr)
\tau_{k-\nu-1}^{\nu,\nu}(x-\nu, N-\nu)-\\
\hat{\mu}_{k-\nu}^{\nu,\nu}
\tau_{k-\nu-2}^{\nu,\nu}(x-\nu, N-\nu),
\quad 2 \le k-\nu \le N-\nu-1,
\end{multline*}
\begin{equation*}
\hat{\kappa}_{k-\nu}^{\nu,\nu}=
\Bigl[
\frac{N+\nu-1}{N+k}
\frac{N+\nu-1}{N-k}
\frac{2k-1}{k-\nu}
\frac{2k+1}{k+\nu}
\Bigr]^\frac{1}{2},
\end{equation*}
\begin{equation*}
\hat{\mu}_{k-\nu}^{\nu,\nu}=
\Bigl[
\frac{N+k-1}{N+k}
\frac{N-k+1}{N-k}
\frac{k-\nu-1}{k-\nu}
\frac{k+\nu-1}{k+\nu}
\frac{2k+1}{2k-3}
\Bigr]^\frac{1}{2},
\end{equation*}
\begin{equation*}
\tau_{0}^{\nu,\nu}(x-\nu, N-\nu)=
\Bigl[
\frac{(2\nu+1)!}{2^{2\nu+1}\nu!^2}
\Bigr]^\frac{1}{2}=
\Bigl[
\frac{2\nu+1}{2}
\prod_{j=1}^{\nu}\frac{\nu+j}{4j}
\Bigr]^\frac{1}{2}
\equiv \mathcal{A}_\nu,
\end{equation*}
\begin{equation*}
\tau_{1}^{\nu,\nu}(x-\nu, N-\nu)=
\mathcal{A}_\nu
\Bigl[
(2\nu+3)
\frac{N+\nu-1}{N-\nu-1}
\frac{N+\nu-1}{N+\nu+1}
\Bigr]^\frac{1}{2}
\Bigl(
\frac{2x}{N+\nu-1}-1
\Bigr).
\end{equation*} 