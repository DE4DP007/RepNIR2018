\Introduction

Согласно плану научно-исследовательской работы за 2014--2016 годы исследования, проводимые в Отделе математики и информатики Дагестанского научного центра РАН, включают в себя 6 тем. Рассмотрим каждую из этих тем отдельно.

\begin{enumerate}[1.]
\item
\textbf{Некоторые вопросы теории приближений в функциональных пространствах с переменным показателем суммируемости и их приложения.}

\input chapters/intros/intro1.tex



\item
\textbf{Исследования по теории аппроксимации и интерполяции в вещественной и комплексной области.}

\input chapters/intros/intro2.tex


\item
\textbf{Развитие общей теории дифференциальных уравнений в частных производных и ее приложений к задачам математической физики, в частности, исследование системы уравнений Навье-Стокса.}

\input chapters/intros/intro3.tex


\item
\textbf{Развитие теории устойчивости и исследование качественных свойств решений дифференциально-разностных и интегро-дифференциальных уравнений и их приложение.}

\input chapters/intros/intro4.tex


\item
\textbf{Разработка вычислительных алгоритмов по массопереносу в пористых и трещинных средах.}

\input chapters/intros/intro5.tex


\item
\textbf{Разработка методов, алгоритмов и создание наукоемкого программного обеспечения для моделирования сложных систем, возникающих в задачах обработки сигналов и изображений. Исследования в области теории графов и теории расписаний.
Разработка математических методов и алгоритмов распознавания образов и восстановление зависимостей по некомплектным и зашумленным данным.}

\input chapters/intros/intro6.tex



\end{enumerate}




