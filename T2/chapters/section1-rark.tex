\chapter{Рациональные сплайн-функции с автономными полюсами}
\section{Некоторые сведения}

Для сетки узлов $\Delta: a=x_0<x_1<\dots<x_N=b$ $(N\geqslant 2)$ положим $h_i=x_i-x_{i-1}$ $(i=1,2,\dots,N)$,
$\Delta_i=\max\{h_{i-1},h_i, h_{i+1}\}$ $(i=2,3,\dots,N-1)$, $\Delta_1=\max\{h_1,h_2\}$, $\Delta_N=\max\{h_{N-1}, h_N\}$,
$\|\Delta\|=\max\{h_i|i=1,2,\dots,N\}$, $\rho_\Delta=\max\{h_i h_j^{-1}|\,|i-j|=1,\,1 \leqslant i,\, j\leqslant N\}$.
По произвольному $\lambda>0$ образуем набор чисел $g=\{g_1,g_2,\dots,g_{N-1}\}$ таких, что при $i=1,2,\dots,N-1$ имеем
\begin{equation}\label{rark1}
g_i= \left\{
\begin{array}{l}
x_{i+1}+\lambda h_{i+1}\text{ при } h_{i+1}\leqslant h_i,\\[1ex]
x_{i-1}-\lambda h_i\text{ при } h_{i+1}> h_i.
\end{array}
\right.
\end{equation}
Для функции $f\in C_{[a,b]}$ при $i=1,2,\dots,N-1$ рассмотрим рациональные функции вида
\begin{equation}\label{rark2}
R_i(x)=\alpha_i+\beta_i (x-x_i)+\gamma_i \frac 1{x-g_i},
\end{equation}
коэффициенты которых $\alpha_i, \beta_i,\gamma_i$ определяются условиями $R_i(x_j)=f(x_j)$ $(j=i-1,i,i+1)$;
при этом
\begin{equation*}
\begin{array}{l}
\alpha_i=f(x_i)-f(x_{i-1}, x_i, x_{i+1})(x_{i-1}-g_i)(x_{i+1}-g_i),\\[1ex]
\beta_i=f(x_{i-1}, x_{i+1})+f(x_{i-1}, x_i, x_{i+1})(x_i-g_i),\\[1ex]
\gamma_i=f(x_{i-1}, x_i, x_{i+1})(x_{i-1}-g_i)(x_i-g_i)(x_{i+1}-g_i).
\end{array}
\end{equation*}
Считаем также, что $R_0(x)\equiv R_1(x)$, $R_N(x)\equiv R_{N-1}(x)$.
\section{Основные результаты}

Для данной сетки узлов $\Delta$ и набора полюсов $g=\{g_1,g_2,\dots,g_{N-1}\}$ по соответствующим рациональным интерполянтам
$R_i(x)$ $(i=0,1,\dots,N)$ определим сплайн $R_{N,k}(x)=R_{N,k} (x, f, \Delta,g)$ $(k=1,2,\dots)$, удовлетворяющий при
$x\in [x_{i-1}, x_i]$ $(i=1,2,\dots,N)$ равенству
\begin{equation}\label{rark3}
R_{N,k}(x)=\frac{R_i(x)(x-x_{i-1})^k+R_{i-1}(x)(x_i-x)^k}{(x-x_{i-1})^k+(x_i-x)^k}.
\end{equation}
Тогда $R_{N,k} (x)$ представляет собой гладкий сплайн из класса $C_{[a,b]}^{(k)}$ $(k=1,2,\dots)$.
Можно показать, что если функция $f(x)$  выпукла или вогнута на отрезке $[x_{i-1}, x_{i+1}]$ при некотором $i=1,2,\dots,N-1$,
то рациональный интерполянт $R_i(x)$ также является соответственно выпуклой или вогнутой функцией на этом отрезке.
Сплайны $R_{N,k}(x)$  $(k\geqslant 2)$ сохраняют выпуклость (вогнутость) функции $f(x)$ в некоторых окрестностях узлов сетки $\Delta$.

Ниже для функций $\varphi\in C_{[a,b]}$ используются также обозначения
$$
\|\varphi\|_{[a,b]}=\max\{|\varphi (x)|: x\in [a,b]\},
$$
$$
\omega (\delta, \varphi)=\sup \{|\varphi(x+h)-\varphi(x)|: 0\leqslant h\leqslant \delta; x,x+h\in [a,b]\}.
$$

Основные результаты сформулируем в виде следующих двух утверждений.

\begin{theorem}\label{rarkteor1}
Пусть для произвольной сетки узлов $\Delta: a=x_0<x_1<\dots<x_N=b$ $(N\geqslant 2)$ и произвольного $\lambda>0$
выбраны числа  $g=\{g_1,g_2,\dots,g_{N-1}\}$ согласно \eqref{rark1}.
Тогда для любой $f\in C_{[a,b]}$ и любой $f\in C_{[a,b]}^{(1)}$ и соответствующего рационального сплайна
$R_{N,k}(x)=R_{N,k}(x,f,\Delta, g)$ при любом натуральном $k$ и $x\in[x_{i-1}, x_i]$ $(i=1,2,\dots,N)$
выполняются соответственно неравенства
\begin{equation}\label{rark4}
|f(x)-R_{N,k}(x)|\leqslant (2+\max\{1, \lambda\})\omega(\Delta_i, f),
\end{equation}
\begin{equation}\label{rark5}
|f^\prime(x)-R^\prime_{N,k}(x)|\leqslant \left(4+\frac 2\lambda+8k\right)\omega(\Delta_i, f^\prime).
\end{equation}
 \end{theorem}

\begin{theorem}\label{rarkteor2}
Пусть для произвольной сетки узлов $\Delta: a=x_0<x_1<\dots<x_N=b$ $(N\geqslant 2)$ и произвольного $\lambda\geqslant 1$
выбраны числа  $g=\{g_1,g_2,\dots,g_{N-1}\}$ согласно \eqref{rark1}.

Тогда для любой функции $f\in C^{(2)}_{[a,b]}$ и соответствующего рационального сплайна
$R_{N,2}(x)=R_{N,2}(x,f,\Delta, g)$ при $x\in[a,b]$ выполняется неравенство
\begin{equation}\label{rark6}
|f^{\prime\prime}(x)-R_{N,2}^{\prime\prime}(x)|\leqslant 26 \omega(\|\Delta\|, f^{\prime\prime})+\frac {49}{\lambda}
\rho_\Delta\|f^{\prime\prime}\|_{[a,b]}.
\end{equation}

При этом для всех $x\in [a,b]$ выполняется также неравенство
\begin{equation}\label{rark7}
|f(x)-R_{N,2}(x)|\leqslant 2\|\Delta\|^2 \|f^{\prime\prime}\|_{[a,b]}.
\end{equation}
 \end{theorem}


\chapter{Скорость сходимости рациональных сплайнов для непрерывных и
 непрерывно дифференцируемых функций}

\section{Вспомогательные утверждения}
Трехточечные рациональные интерполянты --- функции вида

\begin{equation}\label{rark1.1}
R_i(x)=\alpha_i+\beta_i (x-x_i)+\frac{\gamma_i}{x-g_i}
\end{equation}
строятся \cite{rark9} для непрерывных на данном отрезке $[a,b]$ функций $f(x)$
по произвольной сетке попарно различных узлов $\Delta: a=x_0<x_1<\dots<x_N=b$
$(N\geqslant 2)$ так, что коэффициенты $\alpha_i, \beta_i,\gamma_i$ $(i=1,2,\dots,N-1)$
удовлетворяют условиям $R_i(x_j)=f(x_j)$ при $j=i-1,i,i+1$, а в качестве полюса
$g_i$ можно взять любое действительное число вне отрезка $[x_{i-1}, x_{i+1}]$.
Тогда при $i=1,2,\dots,N-1$ имеем
\begin{equation}
\begin{array}{l}
\alpha_i=f(x_i)-f(x_{i-1}, x_i, x_{i+1})(x_{i-1}-g_i)(x_{i+1}-g_i),\\
\beta_i=f(x_{i-1}, x_{i+1})+f(x_{i-1}, x_i, x_{i+1})(x_i-g_i),\\
\gamma_i=f(x_{i-1}, x_i, x_{i+1})(x_{i-1}-g_i)(x_i-g_i)(x_{i+1}-g_i).
\end{array}\label{rark1.2}
\end{equation}

Как легко увидеть из выражения $R^{\prime\prime}_i(x)$, если при некотором $i=1,2,\dots,N-1$
функция $f(x)$ является выпуклой или вогнутой на отрезке $[x_{i-1}, x_{i+1}]$, то
на этом отрезке $R_i(x)$ является соответственно выпуклой или вогнутой функцией.

Для данных $f\in C_{[a,b]}$, натуральных чисел $N\geqslant 2$ и $k$, разбиения
$\Delta: a=x_0<x_1<\dots<x_N=b$ и произвольного набора чисел
$g=\{g_1,g_2,\dots,g_{N-1}\}$ с $g_i\not \in[x_{i-1}, x_{i+1}]$ $(i=1,2,\dots,N-1)$
построим кусочно--рациональную функцию
$R_{N,k} (x)=R_{N,k}(x,f,\Delta, g)$ такую, чо при $x\in [x_{i-1}, x_i]$
$(i=1,2,\dots,N)$ выполняется равенство
$$
R_{N,k}(x)=\frac{R_i(x)(x-x_{i-1})^k+R_{i-1}(x)(x_i-x)^k}{(x-x_{i-1})^k+(x_i-x)^k};
$$
считаем, что $R_0(x)\equiv R_1(x)$, $R_N(x)\equiv R_{N-1} (x)$.

Как известно \cite{rark9}, при каждом $k=1,2,\dots$ функция $R_{N,k}\in C^{(k)}_{[a,b]}$
и в случае
$$
g_i=\begin{cases} 2x_{i+1}-x_i \text{ при } x_{i+1}-x_i\leqslant x_i-x_{i-1},\\
2x_{i-1}-x_i \text{ при } x_{i+1}-x_i>x_i-x_{i-1}
\end{cases}
$$
$(i=1,2,\dots, N-1)$ для всех $f\in C_{[a,b]}$ и $f\in C^{(1)}_{[a,b]}$
соответственно имеем

\begin{equation}\label{rark1.3}
\|f-R_{N,k}\|_{[a,b]}\leqslant 15\,\omega(\|\Delta\|, f),
\end{equation}
$$
\|f^\prime-R^\prime_{N,k}\|_{[a,b]}\leqslant (8k+6)\omega (\|\Delta\|, f^\prime);
$$
здесь и далее приняты обозначения
$$
\|\varphi\|_{[a,b]}=\sup\{|\varphi (x)|: x\in[a,b]\},
$$
$$
\|\Delta\|=\max\{x_i-x_{i-1}: i=1,2,\dots, N\},
$$
$$
\omega(\delta, \varphi)=\sup\{|\varphi(x+h)-\varphi(x)|:
0\leqslant h\leqslant \delta; x, x+h\in[a,b]\}.
$$
Значит, по терминологии Ю.Н. Субботина \cite{rark11}, гладкие сплайны $R_{N,k} (x)$
и их производные $R^\prime_{N, k}(x)$ обладают свойством безусловной
сходимости для всех функций классов $C_{[a,b]}$ и $C^{(1)}_{[a,b]}$ соответственно
(в отличие от гладких полиномиальных сплайнов).

Отметим также, что при $k\geqslant 2$ сплайны $R_{N,k}(x)=R_{N,k}(x,f,\Delta, g)$
сохраняют выпуклость (вниз или вверх) функции $f(x)$ в некоторых окрестностях узлов сетки
$\Delta$.

Заметим, что $R_{N,k} (x)$ имеют наименьшую степень как рациональные функции
на отрезках $[x_{i-1},x_i]$ при $k=1$, а именно, при
$x\in [x_{i-1}, x_i]$ $(i=1,2,\dots,N)$ имеем
\begin{equation}\label{rark1.4}
R_{N,1}(x)=R_i(x)\frac{x-x_{i-1}}{x_i-x_{i-1}}+R_{i-1} (x)\frac{x_i-x}{x_i-x_{i-1}}.
\end{equation}
Поэтому приводимые ниже оценки сплайн-приближений даны для $R_{N,1} (x)$ через
различные структурные характеристики.

Модуль непрерывности (гладкости) третьего порядка функции $f\in C_{[a,b]}$ определяем,
как обычно, через соответствующую конечную разность $\Delta_h^3 f(x)$ равенством
$$
\omega_3(\delta, f)=\sup\{|\Delta_h^3 f(x)|: 0\leqslant h\leqslant \delta;
x, x+3h\in[a,b]\}\quad (\delta\geqslant 0).
$$

Вариация функции $\varphi\in C_{[a,b]}$ на отрезке $[a,b]$ определяется равенством
$$
V(\varphi, [a,b])=\sup\sum_{i=1}^n |\varphi (t_i)-\varphi(t_{i-1})|,
$$
где супремум берется по всем разбиениям $a=t_0<t_1<\dots<t_n=b$ и при всех $n=1,2,\dots$

Следующее утверждение используется при построении сетки узлов сплайна в случае функций,
 имеющих на данном отрезке непрерывную вторую производную конечной вариации.
\begin{lemma}\label{rarkram-lem1}
Если $\varphi\in C_{[a,b]}$ и $V=V(\varphi, [a,b])<\infty$, то при любом натуральном
$n$ существует разбиение   $a=t_0<t_1<\dots<t_m=b$ с $m\leqslant n$ такое, что
$$
V(\varphi, [t_{i-1}, t_i])(t_i-t_{i-1})^2\leqslant \frac 1{n^3} V(b-a)^2\quad (i=1,2,\dots,m),
$$
а при $i=1,2,\dots,m-1$ это неравенство обращается в равенство.
\end{lemma}
Для оценки скорости сплайн--приближений в случае произвольных дважды
 непрерывно дифференцируемых функций (без ограничений на вариацию
второй производной) ниже используется модуль изменения функции. Это
позволяет при необходимости распространить полученную оценку также на обобщенные вариации.

Модуль изменения порядка $n$ $(n=1,2,\dots)$ функции $\varphi \in C[a,b]$ определяется
равенством \cite{rark14}
$$
V_n(\varphi, [a,b])=\sup\left\{\sum_{i=1}^n |\varphi(\beta_i)-\varphi(\alpha_i)|\right\},
$$
где супремум берется при фиксированном $n$ по всем точкам $\alpha_1<\beta_1\leqslant
 \alpha_2<\beta_2\leqslant \dots\leqslant \alpha_n<\beta_n$ из отрезка $[a,b]$.
Близкие определения модуля изменения даны в \cite{rark15}, \cite{rark16}.
Ниже использовано также обычное обозначение
$$
\Omega(\varphi, [a,b])=\sup\{|\varphi(x)-\varphi(y)|: x,y\in [a,b]\}
$$
полного колебания функции $\varphi(x)$ на данном отрезке $[a,b]$.

\section{Оценки сплайн-приближений}

\begin{theorem}\label{rarkteor2.1}
Пусть $f\in C_{[a,b]}$, натуральное $N\geqslant 2$, $\Delta: x_i=a+i\frac{b-a}N$
$(i=0,1,\dots,N)$ --- сетка узлов.

Тогда при любом выборе чисел $g=\{g_1, g_2, \dots,g_{N-1}\}$
с $g_i\not \in [x_{i-1}, x_{i+1}]$ $(i=1,2,\dots,N-1)$ сплайн
 $R_{N,1} (x)=R_{N,1} (x, f, \Delta, g)$ удовлетворяет неравенству
\begin{equation}\label{rark2.1}
\|f-R_{N,1}\|_{[a,b]}\leqslant W_3 \omega_3 \left(\frac{2(b-a)}{3N},f\right)+
\frac 2{3\sqrt 3}\frac M \mu \left(\frac{b-a}N\right)^3,
\end{equation}
где $W_3$ --- константа Уитни,
$M=\max\{|f(x_{i-1}, x_i, x_{i+1})|: i=1,2,\dots,N-1\},$
\newline $\mu=\min\{|x_{i-1}-g_i|, |x_{i+1}-g_i|: i=1,2,\dots,N-1\}.$
\end{theorem}



Заметим, что правую часть неравенства \eqref{rark2.1} можно сделать сколь угодно близкой к первому ее слагаемому за счет выбора сколь угодно больших по модулю значений полюсов
$g_1,g_2,\dots, g_{N-1}$.

\begin{theorem}\label{rarkteor2.2}
Пусть $f\in C^{(2)}_{[a,b]}$, вариация $V=V(f^{\prime\prime}, [a,b])<\infty$,
$n$ --- любое натуральное число.
Тогда существует сетка узлов $\Delta: a=x_0<x_1<\dots<x_N=b$ с $N\leqslant 2n$, для
которой при любом выборе чисел $g=\{g_1, g_2,\dots,g_{N-1}\}$ с $g_i\not \in [x_{i-1}, x_{i+1}]$
$(i=1,2,\dots,N-1)$ сплайн $R_{N,1}(x)=R_{N,1}(x, f, \Delta, g)$ удовлетворяет
неравенству
\begin{equation}\label{rark2.7}
\|f-R_{N,1}\|_{[a,b]} \leqslant \frac{(b-a)^2}{N^3} V+
\frac{\|\Delta\|^3}{4\mu} \|f^{\prime\prime}\|_{[a,b]},
\end{equation}
где $\mu=\min\{|x_{i-1}-g_i|, |x_{i+1}-g_i|: i=1,2,\dots, N-1\}.$
\end{theorem}

\begin{theorem}\label{rarkteor2.3}
Пусть $f\in C^{(2)}_{[a,b]}$, $n$ --- любое натуральное число. Тогда существует сетка узлов
$\Delta: a=x_0<x_1<\dots<x_N=b$ с $N\leqslant 2n-1$, для которой при любом выборе чисел
$g=\{g_1,g_2,\dots,g_{N-1}\}$ с $g_i\not \in [x_{i-1}, x_i]$ $(i=1,2,\dots,N-1)$ сплайн
$R_{N,1}(x)=R_{N,1} (x, f, \Delta, g)$ удовлетворяет неравенству
\begin{equation}\label{rark2.13}
\|f-R_{N,1}\|_{[a,b]}\leqslant 4\frac{(b-a)^2}{N^3} V_n(f^{\prime\prime},[a,b])+
\frac{2(b-a)^3}{\mu N^3}\|f^{\prime\prime}\|_{[a,b]},
\end{equation}
где $\mu=\min\{|x_{i-1}-g_i|, |x_{i+1}-g_i|: i=1,2,\dots,N-1\}.$
\end{theorem}


Приведем также оценку скорости сходимости сплайнов по трехточечным рациональным
 интерполянтам для непрерывно дифференцируемой на отрезке функции через модуль изменения ее производной.

\begin{theorem}\label{rarkteor2.4}
Пусть $f\in C^{(1)}_{[a,b]}$, $n$ --- любое натуральное число. Тогда существует сетка узлов
$\Delta: a=x_0<x_1<\dots<x_N=b$ с $N\leqslant 2n-1$, для которой при любом выборе чисел
$g=\{g_1,g_2,\dots,g_{N-1}\}$ с $g_i\not \in [x_{i-1}, x_{i+1}]$ $(i=1,2,\dots,N-1)$ сплайн
$R_{N,1}(x)=R_{N,1} (x, f, \Delta, g)$ удовлетворяет неравенству
\begin{equation}\label{rark2.16}
\|f-R_{N,1}\|_{[a,b]}\leqslant 6\frac{b-a}{N^2} \left(1+\frac{b-a}{N\mu}\right)
V_n(f^\prime,[a,b]),
\end{equation}
где $\mu=\min\{|x_{i-1}-g_i|, |x_{i+1}-g_i|: i=1,2,\dots,N-1\}.$
\end{theorem}