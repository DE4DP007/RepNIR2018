\section{Дискретные преобразования со свойством прилипания на основе системы $\{\sin x\sin kx\}_{k=1}^{\infty}$ и системы полиномов Чебышева II рода} \label{sect-6.2}
%\textbf{Введены специальные дискретные преобразования со свойством «прилипания» для периодических (на основе системы $\{\sin x\sin kx\}$) и непериодических (на основе полиномов Чебышева II рода) сигналов, получены оценки функций Лебега для них.}


%\subsection{Введение}
Представление функций в виде рядов по тем или иным ортонормированным системам с целью последующего их приближения
частичными суммами выбранного ортогонального ряда является, пожалуй, одним из самых часто применяемых подходов в теории приближений и ее приложениях. Наряду с задачами математической физики, для решения которых указанный подход является традиционным, появились и продолжают появляться все новые важные задачи, для решения которых также все чаще применяются методы, основанные на представлении функций (сигналов) в виде рядов по подходящим ортонормированным системам (см., например, \cite{shii1, shii2, dedus3, pash4, arush5, tref6, tref7, muku8}). При этом часто возникает такая ситуация, когда функция (сигнал, временной ряд, изображение и т.д) $f=f(t)$ задана на достаточно длинном промежутке $[0,T]$ и нам требуется разбить этот промежуток на части $[a_j,a_{j+1}]$ $(j=0,1,\ldots,m)$, рассмотреть отдельные фрагменты функции определенные на этих частичных отрезках, представить их в виде рядов по выбранной ортонормированной системе и аппроксимировать каждый такой фрагмент частичными суммами соответствующего ряда. Такая ситуация является типичной для задач, связанных с решением нелинейных дифференциальных уравнений численно-аналитическими методами \cite{pash4, tref6}, обработкой временных рядов и изображений и других \cite{arush5, tref6, tref7}, в которых
возникает необходимость разбить заданный ряд данных на части,
аппроксимировать каждую часть и заменить приближенно исходный
временный ряд (изображение) функцией, полученной в результате
<<пристыковки>> функций, аппроксимирующих отдельные части. Но тогда в
местах <<стыка>> возникают нежелательные разрывы (артефакты) (см.\cite{muku8}), которые искажают общий вид временного ряда (изображения). Такая
картина непременно возникает при использовании для приближения
<<кусков>> исходной функции сумм Фурье по классическим
ортонормированным системам. В работах \cite{shii1, shii2} введены некоторые специальные ряды по ультрасферическим полиномам Якоби, частичные суммы $\sigma_n^\alpha(f,x)$ которых на на концах отрезка $[-1,1]$ совпадает с исходной функцией $f(x)$, т.е. $\sigma_n^\alpha(f,\pm1)=f(\pm1)$.
В качестве одного из частных случаев таких рядов возникает ряд вида $\Phi(\theta)=a_\Phi(\theta)+\sin\theta \sum_{k=1}^\infty\varphi_k\sin k \theta$,
где $a_\Phi(\theta)={\Phi(0)+\Phi(\pi)\over2}+{\Phi(0)-\Phi(\pi)\over2}\cos\theta$,
$\varphi(\theta)=\Phi(\theta)-a_\Phi(\theta),\quad \varphi_k={\frac2\pi}
\int\limits_{0}^\pi \varphi(\tau){\sin k\tau\over\sin\tau}d\tau.$
В работе \cite{shii2} исследованы, в частности, аппроксимативные свойства этого ряда в пространстве $C^e_{2\pi}$, состоящем из четных непрерывных $2\pi$-периодических функций. В настоящей работе рассматриваются дискретные аналоги таких рядов и исследованы их аппроксимативные свойства.

Для натурального N рассмотрим сетку узлов
$\Lambda^I_N = { \{ t_j = \frac{(2j+1)\pi}{2N} \} }_{j=-N}^{N-1}$.
Через $l_2 (\Lambda^I_N)$ обозначим евклидово пространство дискретных функций $f=f(x)$ вида $f : \Lambda^I_N \rightarrow \mathbb{R} $,
в котором определено скалярное произведение

\begin{equation}
  <f,g> = \frac{1}{N}\sum\limits_{j=-N}^{N-1} f(t_j)g(t_j). \label{iish_gga_1}
\end{equation}

Нетрудно проверить, что функции
\begin{equation}
\frac{1}{\sqrt{2}}, \cos x, \sin x, ..., \cos (N-1)x, \sin (N-1), \frac{1}{\sqrt{2}}\sin Nx \label{iish_gga_2}
\end{equation}
образуют в $l_2(\Lambda^I_N)$ полную ортонормированную систему и, как следствие, для произвольного $f \in l_2(\Lambda^I_N)$ имеем представление
\begin{equation}
f(x) = \frac{a_{0,N}}{2} + \sum\limits_{k=1}^{N-1} a_{k,N}\cos kx + b_{k,N}\sin kx + \frac{b_{N,N}}{2}\sin Nx, \label{iish_gga_3}
\end{equation}
где
\begin{equation}
a_{k,N}=\frac{1}{N}\sum\limits_{j=-N+1}^{N-1} f(t_j)\cos kt_j, \quad b_{k,N}=\frac{1}{N}\sum\limits_{j=-N+1}^{N-1} f(t_j)\sin kt_j. \label{iish_gga_4}
\end{equation}
%2
Если функция $f(x) \in l_2(\Lambda^I_N)$ является четной, то из \eqref{iish_gga_4} следует, что $b_{k,N} = 0 \quad (k = 1,...,N)$, поэтому равенство \eqref{iish_gga_3} принимает вид
\begin{equation}
f(x) = \frac{\hat{f}_0}{2} + \sum\limits_{k=1}^{N-1} \hat{f}_k \cos kx, \label{iish_gga_5}
\end{equation}
где
\begin{equation}
\hat{f}_k = \frac{2}{N}\sum\limits_{j=0}^{N-1} f(t_j)\cos kt_j, \quad (0 \le k \le N-1). \label{iish_gga_6}
\end{equation}
Отображение $\mathfrak{F}_N^I : l_2(\Lambda^I_N) \rightarrow \mathbb{R}^N $, сопоставляющее функции $f \in l_2(\Lambda^I_N)$ вектор
$\mathfrak{F}_N = (\hat{f}_0,...,\hat{f}_{N-1})$ и называемое дискретным косинус-преобразованием Фурье, находит многочисленные применения
в различных областях приложений, в таких, например, как обработка изображений и временных рядов и др. В приложениях, как правило, вместо полного разложения
\eqref{iish_gga_5} используют приближенное равенство
\begin{equation}
f(x) \approx \frac{\hat{f}_0}{2} + \sum\limits_{\nu=1}^n \hat{f}_{k_\nu} \cos k_\nu x, \quad (n \leq N - 1) \label{iish_gga_7}
\end{equation}
по следующей схеме. Заданный <<длинный>> ряд данных разбивается на части, аппроксимируется каждая часть, используя равенство \eqref{iish_gga_7}. После этого заменяется
приближенно исходный ряд функцией, полученной в результате <<пристыковки>> функций (правых частей равенств вида \eqref{iish_gga_7}), аппроксимирующих отдельные части.
Но тогда в местах <<стыка>>, как правило возникают нежелательные разрывы, которые искажают общий вид временного ряда (изображения). Указанные разрывы в точках стыка возникают из-за того, что суммы, фигурирующие в правых частях в приближенных равенствах вида \eqref{iish_gga_7} заметно отклоняются от исходной функции $f(x)$ в
окрестностях точек $x = 0$ и $x = \pi$.

Рассмотрена задача о конструировании аппроксимирующих операторов $\sigma_{n,N}(f)=\sigma_{n,N}(f,x)$, обладающих тем важным свойством, что в окрестностях точек $x = 0$ и $x = \pi$ $\sigma_{n,N}(f,x)$ приближает $f(x)$ значительно лучше, чем на всем отрезке $[0,\pi]$. Кроме того, требуется, чтобы $\sigma_{n,N}(f,x)$ приближал функцию $f(x)$ на всем $[0,\pi]$ не хуже, чем частичные суммы конечного ряда \eqref{iish_gga_5} вида
\begin{equation}
S_{n,N}(f,x) = \frac{\hat{f}_0}{2} + \sum\limits_{k=1}^{n} \hat{f}_k \cos kx. \label{iish_gga_8}
\end{equation}
Наконец, $\sigma_{n,N}(f,x), (0 \leq n \leq N)$ должен допускать численную реализацию со скоростью, сопоставимой со скоростью численной реализации $S_{n,N}(f,x)$, использующей быстрое преобразование Фурье.

Как будет показано ниже, операторы $\sigma_{n,N}(f,x)$, сконструированные в настоящей работе (см. \S2) обладают указанными выше свойствами. Среди отмеченных свойств операторов $\sigma_{n,N}(f,x)$ мы особо выделяем первое, согласно которому $\sigma_{n,N}(f,x)$ приближает $f(x)$ в окрестности точек $x=0$ и $x=\pi$ значительно лучше, чем по всей сетке $\Lambda^I_N$. Это свойство операторов $\sigma_{n,N}(f,x)$ мы будем называть свойством <<прилипания>> ($\sigma_{n,N}(f,x)$ <<прилипает>> к $f(x)$ в окрестностях точек $x=0$ и $x=\pi$).

Именно свойство прилипания операторов $\sigma_{n,N}(f,x)$ в точках $x=0$ и $x=\pi$ существенно и выгодно отличает их от операторов Фурье \eqref{iish_gga_8}.
Наряду с $\Lambda^I_N = { \{ t_j = \frac{(2j+1)\pi}{2N} \} }_{j=-N}^{N-1}$ мы будем рассматривать также сетку $\Lambda^{II}_N = { \{\frac{j\pi}{N} \} }_{j=-N}^{N-1}$ и соответствующее пространство $l_2(\Lambda^{II}_{N})$. Для $f \in l_2(\Lambda^{II}_{N})$ мы рассмотрим операторы, сконструированные по той же схеме, что и $\sigma_{n,N}(f,x)$ и исследуем их аппроксимативные свойства.

\subsection{Дискретные специальные ряды по системе ${\{\sin x \sin kx\}}_{k=1}^{N}$ на сетках $\Omega^I_N = { \{ \frac{(2j+1)\pi}{2N} \} }_{j=1}^{N-1}$ и $\Omega^I_N = { \{ \frac{j\pi}{N} \} }_{j=1}^{N-1}$}

Пусть N --- натуральное число, $t_j = \frac{(2j+1)\pi}{2N}, \Omega^I_N = { \{ t_j \} }_{j=0}^{N-1}$. Через $l_2(\Omega^{I}_{N})$ обозначим евклидово пространство дискретных функций $g = g(x)$ вида $g : \Omega^I_N \rightarrow \mathbb{R}$, в котором для $f, g \in l_2(\Omega^I_N)$ определено скалярное произведение
\begin{equation}
<f, g>_I = \frac{2}{N} \sum\limits_{j=0}^{N-1} f(t_j)g(t_j) \label{iish_gga_9}
\end{equation}
Заметим, что функцию $f(x) \in l_2(\Omega^I_N)$ можно продолжить на сетку $\Lambda^I_N$, полагая $f(-t_{j-1}) = f(t_j) $ для всех $t_j \in \Omega^I_N$.
Нетрудно проверить, что функции
$$
\sin{x}, \sin{2x}, ..., \sin{Nx}
$$
образуют в $l_2(\Omega^I_N)$ ортогональную систему относительно скалярного произведения \eqref{iish_gga_9}, также при $1 \leq k, n \leq N$ имеют место равенства
\begin{equation}
\frac{2}{N} \sum\limits_{j=0}^{N-1} \sin{k t_j} \sin{n t_j} = \begin{cases} 0, & k \neq n, \\ 1, & k = n < N, \\ 2, & k = n = N. \end{cases} \label{iish_gga_10}
\end{equation}
Из свойства \eqref{iish_gga_10} следует, что система ${\{ \sin kx \}}_{k=1}^N$ является ортогональным базисом в $l_2(\Omega_N^I)$ и, следовательно, произвольная дискретная функция $\varphi \in l_2(\Omega_N^I)$
допустит представление
\begin{equation}
  \varphi(x) = \sum\limits_{k=1}^N \varphi_k \sin kx, x \in \Omega_N^I, \label{iish_gga_11}
\end{equation}
где
\begin{equation}
  \varphi_k = \sum\limits_{j=0}^{N-1} \varphi(t_j)\sin kt_j \cdot \begin{cases}\frac2N, & k < N, \\ \frac1N, & k = N.\end{cases} \label{iish_gga_12}
\end{equation}
Рассмотрим сетку $\Omega_N^I$, добавив в нее две точки: $t_{-1}=0$ и $t_N=\pi$ и обозначим через $\overline{\Omega}_N^I = \Omega_N^I \cup \{0, \pi\}$.
Пусть дискретная функция $f$ определена на новой сетке $\overline{\Omega}_N^I$, то есть $f: \overline{\Omega}_N^I \rightarrow \mathbb{R}$.
Положим
\begin{gather}
a_f(x) = \frac{f(0) + f(\pi)}2 + \frac{f(0) - f(\pi)}2\cos x, \label{iish_gga_13}\\
g(x)=f(x) - a_f(x), \quad
\varphi(x) = \frac{g(x)}{\sin x}, \quad x \in \Omega_N^I. \label{iish_gga_14}
\end{gather}
Тогда ряд \eqref{iish_gga_11} принимает следующий вид

\begin{equation}
  \frac{g(x)}{\sin x} = \sum\limits_{k=1}^{N} g_{k,N} \sin kx, \quad x \in \Omega_N^I, \label{iish_gga_15}
\end{equation}
где
\begin{equation}
  g_{k,N} = \sum\limits_{j=0}^{N-1} g(t_j) \frac{\sin kt_j}{\sin t_j} \cdot \begin{cases} \frac2N, & k < N \\ \frac1N, & k = N \end{cases} \label{iish_gga_16}
\end{equation}
Ввиду \eqref{iish_gga_13} ряд \eqref{iish_gga_15} \eqref{iish_gga_15} можно переписать еще так
\begin{equation}
  f(x) = a_f(x) + \sin x \sum\limits_{k=1}^{N} g_{k,N} \sin kx, \quad x \in \overline{\Omega}_N^I \label{iish_gga_17}
\end{equation}
Обозначим через $\sigma_{n,N}(f,x)$ частичную сумму ряда \eqref{iish_gga_17} следующего вида
\begin{equation}
  \sigma_{n,N}(f) = \sigma_{n,N}(f,x) = a_f(x) + \sin x \sum\limits_{k=1}^{n-1} g_{k,N} \sin kx, \quad k, n \leq N, \label{iish_gga_18}
\end{equation}
%Покажем, что для произвольного четного тригонометрического полинома $T_n(x)$ порядка $n \leq N$ справедливо тождество
Нетрудно показать, что для произвольного четного тригонометрического полинома $T_n(x)$ порядка $n \leq N$ справедливо тождество
\begin{equation}
  \sigma_{n,N}(T_n, x) \equiv T_n(x), \label{iish_gga_19}
\end{equation}
другими словами $\sigma_{n,N}(f)$  является проектором на подпространство четных тригонометрический полиномов порядка $n$.

Мы можем трактовать $\sigma_{n,N}(f)$ как линейный оператор, действующий в различных функциональных пространствах. С этой целью введем некоторые обозначения. Пусть
\begin{equation*}
  \omega_N^I = \left\{ \frac{(2j+1)\pi}{2N} \right\}_{j \in \mathbb{Z}}, \mathbb{Z}_\pi = \{j\pi\}_{j \in \mathbb{Z}},
  \quad \overline{\omega}_N^I = \omega_N^I \cup \mathbb{Z}_\pi.
\end{equation*}

Тогда функцию $f = f(x)$, заданную на конечной сетке $\overline{\Omega}_N^I$, можно продолжить на бесконечную сетку $\overline{\omega}_N^I$ так, чтобы продолженная функция
на $\overline{\omega}_N^I$ была четной и $2\pi$-периодической, то есть при $x \in \overline{\omega}_N^I$ $f(-x) = f(x)$ и $f(x + 2\pi) = f(x)$.
Множество $C^e(\overline{\omega}_N^I)$ всех таких дискретных функций $f(x)$ является нормированным в пространстве с нормой
$\|f\|_N = \max_{x \in \overline{\omega}_N^I} |f(x)| = \max_{x \in \overline{\Omega}_N^I} |f(x)|$.
Через $E_{n,N}^\pi (f)$ обозначим наилучшее приближение функции $f \in C^e(\overline{\omega}_N^I)$ тригонометрическими полиномами $T_n(x)$ порядка $n$,
удовлетворяющими условиям $f(0)=T_n(0)$, $f(\pi)=T_n(\pi)$. Пусть $T_n(f)=T_n(f,x)$ --- тригонометрический полином порядка $n$, удовлетворяющий
условиям $f(0)=T_n(f,0)$, $f(\pi)=T_n(f,\pi)$, для которого
\begin{equation}
  E_{n,N}^\pi(f) = ||f - T_n(f)||_N. \label{iish_gga_20}
\end{equation}
Далее, с учетом \eqref{iish_gga_19} имеем
\begin{multline*}
  f(x)-\sigma_{n,N}(f,x) = f(x) - T_n(f,x) + T_n(f,x) - \sigma_{n,N}(f,x) = \\
  f(x) - T_n(f,x) + \sigma_{n,N}(T_n(f) - f, x).
\end{multline*}
Отсюда и из \eqref{iish_gga_20} находим
\begin{equation}
  |f(x) - \sigma_{n,N}(f,x)| \leq E_{n,N}^\pi (f) + |\sigma_{n,N}(T_n(f) - f, x)|. \label{iish_gga_21}
\end{equation}
Теперь обратимся к равенству \eqref{iish_gga_18}, из которого, в силу того, что $f(0) = T_n(0)$, $f(\pi) = T_n(\pi)$,
\begin{equation}
  \sigma_{n,N} (T_n(f) - f, x) = \sin x \sum\limits_{k=1}^{n-1} (T_n(f) - f)_{k,N} \sin kx, \label{iish_gga_22}
\end{equation}
где
\begin{equation}
  (T_n(f) - f)_{k,N} = \frac2N \sum\limits_{j=0}^{N-1} \Bigl[T_n(f,t_j) - f(t_j)\Bigr]\frac{\sin kt_j}{\sin t_j}. \label{iish_gga_23}
\end{equation}
Из \eqref{iish_gga_22} и \eqref{iish_gga_23} получим
\begin{equation}
  \sigma_{n,N}(T_n - f, x) = \sin x \frac2N \sum\limits_{j=0}^{N-1} \Bigl[T_n(f, t_j) - f(t_j)\Bigr] \sum\limits_{k=1}^{n-1} \frac{\sin kx \sin kt_j}{\sin t_j}. \label{iish_gga_24}
\end{equation}
Поэтому в силу \eqref{iish_gga_20}
\begin{equation*}
  |\sigma_{n,N}(T_n(f) - f, x)| \leq E_{n,N}^\pi(f) \frac2\pi \sum\limits_{j=0}^{N-1} \left| \sin x \sum\limits_{k=1}^{n-1} \frac{\sin kx \sin kt_j}{\sin t_j} \right|.
\end{equation*}

Из \eqref{iish_gga_21} и \eqref{iish_gga_24} имеем
\begin{equation}
  | f(x) - \sigma_{n,N}(f,x)| \leq E_{n,N}^\pi(f) (1 + L_{n,N}^\pi(x)), \label{iish_gga_25}
\end{equation}
где
\begin{equation}
  L_{n,N}^\pi(x) = \frac2N \sum\limits_{j=0}^{N-1} \left| \sin x \sum\limits_{k=1}^{n-1} \frac{\sin kx \sin kt_j}{\sin t_j} \right|. \label{iish_gga_26}
\end{equation}

В связи с неравенством \eqref{iish_gga_25} возникает задача об оценке величины $L_{n,N}^\pi(x)$ при
$x \in \overline{\Omega}_N$
Аналогичная задача возникает и в случае, если рассмотреть $\sigma_{n,N}(f)$ как оператор, дайствующий в пространстве $C_{2\pi}^e$,
состоящем из четных $2\pi$-периодических непрерывных функций $f(x)$, для которых норма определяется
обычным образом, а именно $||f|| = \max_{x \in \mathbb{R}} |f(x)|$. В этом случае аналог неравенства \eqref{iish_gga_25} имеет вид
\begin{equation}
  |f(x) - \sigma_{n,N}(f,x)| \leq E_n(f)(1 + L_{n,N}^\pi(x)), \label{iish_gga_27}
\end{equation}
где $E_n(f)$ -- наилучшее приближение функции $f \in C_{2\pi}^e$ тригонометрическими полиномами $T_n(x)$ порядка n. Требуется
оценить величину $L_{n,N}^\pi(x)$ определенную равенством \eqref{iish_gga_26} для произвольного $x \in \mathbb{R}$.

Перейдем к сетке $\Omega_{N-1}^{II} = {\left\{ \frac{j\pi}{N} \right\}}_{j=1}^{N-1}$. Через $l_2(\Omega_{N-1}^{II})$ мы обозначим
евклидово пространство дискретных функций $f(x)$ вида $f : \Omega_{N-1}^{II} \rightarrow \mathbb{R}$, в котором скалярное произведение
определено с помощью равенства
\begin{equation}
  <f,g>_{II} = \frac2N \sum\limits_{j=1}^{N-1} f(\frac{j\pi}{N}) g(\frac{j\pi}{N}). \label{iish_gga_28}
\end{equation}
Полную ортонормированную систему в $l_2(\Omega_{N-1}^{II})$ образуют функции
\begin{equation*}
  \sin x, \sin 2x, ..., \sin (N-1)x,
\end{equation*}
то есть
\begin{equation}
  \frac2N \sum\limits_{j=1}^{N-1} \sin k\frac{j\pi}{N} \sin n \frac{j\pi}{N} = \delta_{kn} =
  \begin{cases} 0, & k \neq n, \\ 1, & k = n.\end{cases}  \label{iish_gga_29}
\end{equation}

Из \eqref{iish_gga_29} следует, что произвольная функция $\psi \in l_2(\Omega_{N-1}^{II})$ допускает представление
\begin{equation}
  \psi(x) = \sum\limits_{k=1}^{N-1} \psi_k \sin kx, x \in \Omega_{N-1}^{II}, \label{iish_gga_30}
\end{equation}
где
\begin{equation}
  \psi_k = \frac2N \sum\limits_{j=1}^{N-1} \psi(\frac{j\pi}{N}) \sin \frac{kj\pi}{N}. \label{iish_gga_31}
\end{equation}
Пусть теперь $f(x)$ произвольная функция из $l_2(\Omega_{N-1}^{II})$, которую доопределим в точках $x = 0$ и  $x = \pi$.
Положим
\begin{equation}
  a_f(x) = \frac{f(0) + f(\pi)}{2} + \frac{f(0) - f(\pi)}{2}\cos x, h(x) = f(x) - a_f(x), \label{iish_gga_32}
\end{equation}
и заметим, что $h(0) = h(\pi) = 0$. Для функции $\psi(x) = \frac{h(x)}{\sin x}$ коэффициенты из \eqref{iish_gga_31} принимают
следующий вид:
\begin{equation}
  \psi_k=\frac2N\sum\limits_{j=1}^{N-1}\frac{h(\frac{j\pi}{N})-a_f(\frac{j\pi}{N})}{\sin\frac{j\pi}{N}}\sin\frac{kj\pi}{N} = \hat{h}_k. \label{iish_gga_33}
\end{equation}
Из \eqref{iish_gga_30} - \eqref{iish_gga_33} выводим
\begin{equation}
  f(x) = a_f(x) + \sin x \sum\limits_{k=1}^{N-1} \hat{h}_k \sin kx, \quad x \in \left\{ \frac{j\pi}{N} \right\}_{j=0}^{N}. \label{iish_gga_34}
\end{equation}
Обозначим через $\tau_{n,N}(f,x)$ частичную сумму ряда \eqref{iish_gga_34} следующего вида
\begin{equation}
  \tau_{n,N}(f,x) = a_f(x) + \sin x \sum\limits_{k=1}^{n-1} \hat{h}_k \sin kx, \quad 1 \leq n \leq N-1. \label{iish_gga_35}
\end{equation}

Можно показать, что для произвольного четного тригонометрического полинома $T_n(x)$ порядка $n \leq N-1$ имеет место тождество
\begin{equation}
  \tau_{n,N}(T_n, x) \equiv T_n(x). \label{iish_gga_36}
\end{equation}

Дискретную функцию $f(x)$, заданную на конечной сетке $\left\{ \frac{j\pi}{N} \right\}_{j=0}^N$ мы продолжим на бесконечную сетку
$\mathbb{Z}_{\pi/N} = \left\{ \frac{j\pi}{N} \right\}_{j=-\infty}^{\infty}$ так, чтобы продолженная функция была четной и $2\pi$ -
периодической при $x \in \mathbb{Z}_{\pi/N}$, то есть $x \in \mathbb{Z}_{\pi/N}, f(-x) = f(x)$ и $f(x + 2\pi) = f(x)$.
Множество всех таких функций мы обозначим $C^e(\mathbb{Z}_{\pi/N})$, которое мы превратим в нормированное пространство с нормой
\begin{equation*}
  \|f\|_{\pi/N} = \max\limits_{x \in \mathbb{Z}_{\pi/N}} |f(x)|
\end{equation*}
Через $E_{n,N/\pi}$ обозначим наилучшее приближение функции $f \in C^e(\mathbb{Z}_{n, N/\pi})$ тригонометрическими полиномами $T_n(x)$
порядка $n$, удовлетворяющих условиям $f(0) = T_n(0), f(\pi) = T_n(\pi)$. Пусть $T_n(f) = T_n(f,x)$ --- тригонометрическуий
полином порядка $n$, удовлетворяющий условиям $f(0) = T_n(f,0), f(\pi) = T_n(f,\pi)$, для которого
\begin{equation}
  E_{n,\pi/N}(f) = \|f - T_n(f)\|_{\pi/N}. \label{iish_gga_37}
\end{equation}
Тогда в силу \eqref{iish_gga_36} $f(x) - \tau_{n,N}(f,x) = f(x) - T_n(f,x) + \tau_{n,N}(T_n(f) - f, x)$,
а отсюда и из \eqref{iish_gga_37} находим
\begin{equation}
  |f(x) - \tau_{n,N}(f,x)| \leq E_{n,\pi/N}(f) + |\tau_{n,N}(T_{n}(f)-f, x)|. \label{iish_gga_38}
\end{equation}

Далее из \eqref{iish_gga_33} и \eqref{iish_gga_34} имеем
\begin{equation}
  \tau_{n,N}(T_n(f) - f, x) = \sin x \sum\limits_{j=1}^{n-1} (T_n(f)-f)_k \sin kx, \label{iish_gga_39}
\end{equation}
где
\begin{equation}
  (T_n(f)-f)_k = \frac2N \sum\limits_{j=1}^{N-1} \Bigl[T_n(f, \frac{j\pi}{N}) - f(\frac{j\pi}{N})\Bigr]\frac{\sin \frac{jk\pi}{N}}{\sin\frac{j\pi}{N}} \label{iish_gga_40}
\end{equation}
Из \eqref{iish_gga_39} и \eqref{iish_gga_40} находим
\begin{equation*}
  \tau_{n,N}(T_n(f)-f, x) = \frac{2 \sin x}{N} \sum\limits_{j=1}^{N-1}\Bigl[T_n(f, \frac{j\pi}{N}) - f(\frac{j\pi}{N})\Bigr]
  \sum\limits_{k=1}^{n-1} \frac{\sin kx \sin \frac{jk\pi}{N}}{\sin \frac{j\pi}{N}}
\end{equation*}
Отсюда и из \eqref{iish_gga_39} имеем
\begin{equation}
  |\tau_{n,N}(T_n(f) - f, x)| \leq E_{n,\pi/N}(f) \frac2N \sum\limits_{j=1}^{N-1}
  \left| \sin x \sum\limits_{k=1}^{n-1} \frac{\sin kx \sin \frac{jk\pi}{N}}{\sin \frac{j\pi}{N}} \right| \label{iish_gga_41}
\end{equation}
Сопостовляя \eqref{iish_gga_39} с \eqref{iish_gga_41} получаем
\begin{equation}
  |f(x) - \tau_{n,N}(f,x)| \leq E_{n,\pi/N}(f)(1+L_{n,\pi/N}(x)) \label{iish_gga_42}
\end{equation}
где
\begin{equation}
  L_{n,\pi/N}(x) = \frac2N \sum\limits_{j=1}^{N-1} \sin x
  \left| \sum\limits_{k=1}^{n-1} \frac{\sin kx \sin \frac{jk\pi}{N}}{\sin \frac{j\pi}{N}} \right| \label{iish_gga_43}
\end{equation}

Отсюда возникает задача об оценке величины $L_{n,\pi/N}(x)$ при $n, N \rightarrow \infty$

\subsection{Оценки для функций Лебега операторов $\sigma_{n,N}(f)$ и $\tau_{n,N}(f)$}
В настоящем мы приводим оценки для величин $L_{n,N}^\pi(x)$ и $L_{n,\pi/N}$, определенных равенствами \eqref{iish_gga_26} и \eqref{iish_gga_43}
и представляющих собой функции Лебега для частичных сумм $\sigma_{n,N}(f,x)$ и $\tau_{n,N}(f,x)$, соответственно.

\begin{theorem}
\label{idshasin.th1}
Имеют место следующие оценки:

\begin{equation}
  L_{n,N}^\pi(x) \leq C(1 + \ln(n| \sin x | + 1)) \label{iish_gga_44}, \quad (1 \leq n \leq N),
\end{equation}
\begin{equation}
  L_{n,\pi/N}(x) \leq C(1 + \ln(n| \sin x | + 1)) \label{iish_gga_45}, \quad (1 \leq n \leq N-1).
\end{equation}
\end{theorem}


    \subsection{Дискретное преобразование на основе полиномов Чебышева II рода}
Мы рассмотрим на отрезке $[-1,1]$ сетку узлов
\begin{equation*}
  x_j = \cos\frac{j\pi}{N}, \quad (j=1,2,...,N-1).
\end{equation*}
Положим $A_N = \{x_j\}_{j=1}^{N-1}$ и рассмотрим евклидово $(N-1)$-мерное пространство $l_2(A_N)$, состоящее из
дискретных функций вида $f: A_N \rightarrow \mathbb{R}$, в котором введено скалярное произведение
\begin{equation}
  <f,g>_N = \frac{2}{N}\sum\limits_{j=1}^{N-1}(1-x^2_j)f(x_j)g(x_j). \label{iish_gga_46}
\end{equation}
Полную ортонормированную систему в $l_2(A_N)$ образуют функции
\begin{equation}
  U_k(x) = \frac{\sin(k+1)\arccos(x)}{\sqrt{1-x^2}}, \quad (k=0,1,...,N-2), \label{iish_gga_47}
\end{equation}
представляющие собой классические полиномы Чебышева второго рода. В самом деле, из \eqref{iish_gga_29} и \eqref{iish_gga_47}
имеем
\begin{multline*}
  <U_k, U_n> = \frac{2}{N} \sum\limits_{j=1}^{N-1} (1 - x_j^2) U_k(x_j)U_n(x_j) = \\
  \frac2N\sum\limits_{j=1}^{N-1}(1-x_j^2)\frac{\sin k\arccos x_j}{\sqrt{1-x_j}}\frac{\sin n\arccos x_j}{\sqrt{1-x_j^2}}=
  \frac2N\sum\limits_{j=1}^{N-1}\sin k \frac{j\pi}{N}\sin n\frac{j\pi}{N} = \delta_{kn}.
\end{multline*}
Отсюда следует, что произвольная функция $\varphi(x) \in l_2(A_N)$ допускает представление
\begin{equation}
  \varphi(x) = \sum\limits_{k=0}^{N-2} \varphi_k U_k(x), \quad x \in A_N, \label{iish_gga_48}
\end{equation}
где
\begin{equation}
  \varphi_k = \frac2N \sum\limits_{j=1}^{N-1} \varphi(x_j)U_k(x_j)(1-x_j^2). \label{iish_gga_49}
\end{equation}
Пусть $f(x)$ --- произвольная функция из $l_2(A_N)$, которую доопределим в точках $x=-1$ и $x=1$. Положим
\begin{equation*}
  b_f(x)=\frac{f(1)+f(-1)}{2}+\frac{f(1)-f(-1)}{2}x, \quad g(x)=f(x)-b_f(x),
\end{equation*}
и заметим, что $g(1)=g(-1)=0$. Для функции $\varphi(x)=\frac{g(x)}{(1-x^2)}$ коэффициенты \eqref{iish_gga_49} имеют вид
\begin{equation}
  \varphi_k = \frac2N\sum\limits_{j=1}^{N-1}g(x_j)U_k(x_j)=\hat{g}_k \quad (k=0,1,...,N-2). \label{iish_gga_50}
\end{equation}
Из \eqref{iish_gga_48}-\eqref{iish_gga_50} для $x\in A_N \cup \{-1, 1\}$ находим
\begin{equation}
  f(x)=b_f(x)+(1-x^2)\sum\limits_{k=0}^{N-2}\hat{g}_k U_k(x). \label{iish_gga_51}
\end{equation}
Обозначим через $S_{n,N}(f)=S_{n,N}(f,x)$ частичную сумму конечного ряда \eqref{iish_gga_51} следующего вида
\begin{equation}
  S_{n,N}(f,x) = b_f(x) + (1-x^2)\sum\limits_{k=0}^{n-2}\hat{g}_k U_k(x). \label{iish_gga_52}
\end{equation}
Тогда нетрудно увидеть, что для произвольного алгебраического полинома $P_n(x)$ степени не выше $n$ имеет место
тождество
\begin{equation}
  S_{n,N}(P_n, x) \equiv P_n(x). \label{iish_gga_53}
\end{equation}
В самом деле $q_{n-2}(x) = \varphi(x) = \frac{g(x)}{1-x^2} = \frac{P_n(x)-b_{P_n}(x)}{1-x^2}$ представляет собой
алгебраический полином степени не выше $n-2$, поэтому в силу свойства \eqref{iish_gga_47} и равенства \eqref{iish_gga_49}
имеем $\hat{g}_k = \varphi_k = 0$ при $k=n-1, n, ...,N-2$. Поэтому, в силу \eqref{iish_gga_50} и \eqref{iish_gga_51} мы
можем записать
\begin{equation*}
  P_n(x) = b_{P_n}(x)+(1-x^2)\sum\limits_{k=0}^{n-2}\hat{g}_k U_k(x)=S_{n,N}(P_n,x).
\end{equation*}

Расширим сетку $A_N$, добавив к ней две точки $x_0=1$ и $x_N=-1$. Положим $\overline{A}_N=A_N \cup \{-1,1\}$.
Будем рассматривать дискретные функции вида $f: \overline{A}_N \rightarrow \mathbb{R}$. Множество всех таких функций,
для которых определена норма $\|f\|_{A_N} = \max_{x\in\overline{A}_N} |f(x)|$ обозначим через $C(\overline{A}_N)$.
Пусть $E_{n,N}^A(f)$ --- наилучшее приближение функции $f\in C(\overline{A}_N)$ алгебраическими полиномами $P_n(x)$
степени $n$, удовлетворяющие условию $P_n(-1)=f(-1)$, $P_n(1)=f(1)$. Пусть $P_n(f)=P_n(f,x)$ --- алгебраический
полином степени $n$, удовлетворяющий условиям $f(-1)=P_n(f,-1)$ и $f(1)=P_n(f,1)$, для которого
\begin{equation}
  E_{n,A_N}(f)=||f-P_n(f)||_{A_N} \label{iish_gga_54}
\end{equation}
Тогда в силу \eqref{iish_gga_53}
\begin{equation*}
  f(x)-S_{n,N}(f,x) = f(x)-P_n(f,x)+S_{n,N}(P_n(f)-f,x),
\end{equation*}
а отсюда и из \eqref{iish_gga_54} находим
\begin{equation}
  |f(x)-S_{n,N}(f,x)| \leq E_{n,A_N}(f)+|S_{n,N}(P_n(f)-f,x)| \label{iish_gga_55}
\end{equation}
Далее, из \eqref{iish_gga_52} имеем ($b_{P_n(f)-f}(x) \equiv 0$)
\begin{equation}
  S_{n,N}(P_n(f)-f) = (1-x^2)\sum\limits_{k=0}^{n-2} \hat{(P_n(f)-f)}_k U_k(x), \label{iish_gga_56}
\end{equation}
где в силу \eqref{iish_gga_50}
\begin{equation}
  \widehat{(P_n(f)-f)}_k=\frac2N\sum\limits_{j=1}^{N-1}(P_n(f,x_j)-f(x_j))U_k(x_j). \label{iish_gga_57}
\end{equation}
Из \eqref{iish_gga_56} и \eqref{iish_gga_57} находим
\begin{equation*}
  S_{n,N}(P_n(f)-f,x) = \frac{2(1-x^2)}{N}\sum\limits_{j=1}^{N-1}(P_n(f,x_j)-f(x_j))
  \sum\limits_{k=0}^{n-2}U_k(x)U_k(x_j),
\end{equation*}
откуда, в свою очередь, в силу \eqref{iish_gga_54} выводим
\begin{equation}
  |S_{n,N}(P_n(f)-f, x)| \leq E_{n,A_N}(f) \frac2N\sum\limits_{j=1}^{N-1} (1-x^2)
  \left|
    \sum\limits_{k=0}^{n-2} U_k(x)U_k(x_j)
  \right|. \label{iish_gga_58}
\end{equation}
Сопоставляя \eqref{iish_gga_58} с \eqref{iish_gga_55}, мы можем записать
\begin{equation}
  |f(x)-S_{n,N}(f,x)| \leq E_{n,A_N}(f)(1+\Lambda_{n,N}(x)), \label{iish_gga_59}
\end{equation}
где
\begin{equation}
  \Lambda_{n,N}(x)=\frac2N\sum\limits_{j=1}^{N-1}(1-x^2)
  \left| \sum\limits_{k=0}^{n-2}U_k(x)U_k(x_j) \right|. \label{iish_gga_60}
\end{equation}
В связи с неравенством \eqref{iish_gga_59} возникает задача об оценке величины $\Lambda_{n,N}(x)$. Положим $x=\cos\theta$.
Тогда из \eqref{iish_gga_60}, \eqref{iish_gga_47} и \eqref{iish_gga_43} имеем
\begin{multline}
  \Lambda_{n,N}(\cos\theta)=\frac2N\sum\limits_{j=1}^{N-1}\sin\theta
  \left|
  \sum\limits_{k=0}^{n-2}\frac{\sin(k+1)\theta\sin\frac{j(k+1)\pi}{N}}{\sin\frac{j\pi}{N}}
  \right| = \\
  \frac{2\sin\theta}{N}\sum\limits_{j=1}^{N-1}
  \left|
  \sum\limits_{k=1}^{n-1}\frac{\sin k\theta\sin k\frac{j\pi}{N}}{\sin\frac{j\pi}{N}}
  \right| = L_{n,\frac{\pi}{N}}(\theta). \label{iish_gga_61}
\end{multline}
Поэтому в силу  Теоремы \ref{idshasin.th2} выводим следующий результат.

\begin{theorem}
\label{idshasin.th2}
 Имеет место оценка:
\begin{equation*}
  \Lambda_{n,N}(x) \leq C(1+\ln(n\sqrt{1-x^2}+1)), \quad -1 \leq x \leq 1, \quad 1 \leq n \leq N-1.
\end{equation*}
\end{theorem}
