Исследована проблема изменения граничного условия в автомодельной задаче извлечения сухого тепла пород, в которой возникают сложности, связанные с численной реализации обращения преобразования Лапласа.
В текущем году показано, что нарастание температуры на границе от начального значения по экспоненциальному закону к некоторому постоянному значению приводит
к появлению полюса на отрицательной действительной оси параметра преобразования.
Интегралы по отрицательной действительной оси при этом становятся несобственными, а погрешности вычислений растут.

%Предложено решение новой задачи о росте температуры термальной воды, фонтанирующей во времени с постоянной скоростью. Задача контактная, включающая как температуру горной породы с геотермическим градиентом, так и температуру жидкости в трещине. Она смоделирована по схеме Ловерье, решена в нестационарной постановке с применением преобразования Лапласа и приведена к универсальному виду в безразмерных переменных. При этом отмечаются особенности, вносимые геотермальным градиентом и температурным фронтом воды в процедуру обращения изображения по Лапласу.


Предложена модель для задачи определения роста температуры добываемой термальной воды в строго нестационарной постановке с учётом обмена тепла с горными породами. Для случая подъёма вод по трещине она сводится к универсальному виду подбором безразмерных переменных и имеет точное аналитическое решение, поддающееся устойчивому расчёту достаточно простыми численными средствами. 