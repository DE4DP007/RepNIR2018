Представление функций в виде рядов по тем или иным ортонормированным системам с целью последующего их приближения
частичными суммами выбранного ортогонального ряда является, пожалуй, одним из самых часто применяемых подходов в теории приближений и ее приложениях. Наряду с задачами математической физики, для решения которых указанный подход является традиционным, появились и продолжают появляться все новые важные задачи, для решения которых также все чаще применяются методы, основанные на представлении функций (сигналов) в виде рядов по подходящим ортонормированным системам (см., например, \cite{shii1, shii2, dedus3, pash4, arush5, tref6, tref7, muku8}). При этом часто возникает такая ситуация, когда функция (сигнал, временной ряд, изображение и т.д) $f=f(t)$ задана на достаточно длинном промежутке $[0,T]$ и нам требуется разбить этот промежуток на части $[a_j,a_{j+1}]$ $(j=0,1,\ldots,m)$, рассмотреть отдельные фрагменты функции определенные на этих частичных отрезках, представить их в виде рядов по выбранной ортонормированной системе и аппроксимировать каждый такой фрагмент частичными суммами соответствующего ряда. Такая ситуация является типичной для задач, связанных с решением нелинейных дифференциальных уравнений численно-аналитическими методами \cite{pash4, tref6}, обработкой временных рядов и изображений и других \cite{arush5, tref6, tref7}, в которых
возникает необходимость разбить заданный ряд данных на части,
аппроксимировать каждую часть и заменить приближенно исходный
временный ряд (изображение) функцией, полученной в результате
<<пристыковки>> функций, аппроксимирующих отдельные части. Но тогда в
местах <<стыка>> возникают нежелательные разрывы (артефакты) (см.\cite{muku8}), которые искажают общий вид временного ряда (изображения). Такая
картина непременно возникает при использовании для приближения
<<кусков>> исходной функции сумм Фурье по классическим
ортонормированным системам. В работах \cite{shii1, shii2} введены некоторые специальные ряды по ультрасферическим полиномам Якоби, частичные суммы $\sigma_n^\alpha(f,x)$ которых на на концах отрезка $[-1,1]$ совпадает с исходной функцией $f(x)$, т.е. $\sigma_n^\alpha(f,\pm1)=f(\pm1)$.
В качестве одного из частных случаев таких рядов возникает ряд вида $\Phi(\theta)=a_\Phi(\theta)+\sin\theta \sum_{k=1}^\infty\varphi_k\sin k \theta$,
где $a_\Phi(\theta)={\Phi(0)+\Phi(\pi)\over2}+{\Phi(0)-\Phi(\pi)\over2}\cos\theta$,
$\varphi(\theta)=\Phi(\theta)-a_\Phi(\theta),\quad \varphi_k={\frac2\pi}
\int\limits_{0}^\pi \varphi(\tau){\sin k\tau\over\sin\tau}d\tau.$
В работе \cite{shii2} исследованы, в частности, аппроксимативные свойства этого ряда в пространстве $C^e_{2\pi}$, состоящем из четных непрерывных $2\pi$-периодических функций.
%%%%%%%%%%%%%%%%%%%%%%%%%%%%%%%%%%%%%%%%
В настоящей работе рассмотрена задача о конструировании аппроксимирующих операторов, обладающих тем важным свойством, что в окрестностях граничных точек отрезка аппроксимации приближает $f(x)$ значительно лучше, чем на всем отрезке $[0,\pi]$. Кроме того, требуется, чтобы $\sigma_{n,N}(f,x)$ приближал функцию $f(x)$ на всем $[0,\pi]$ не хуже, чем частичные суммы конечного ряда \eqref{iish_gga_5} вида. Также рассматриваются дискретные аналоги таких рядов и исследованы их аппроксимативные свойства.
