В нефтепромысловой механике для решения ряда линейных краевых задач тепло- и массопереноса широко применяется преобразование Лапласа и его обращение \cite{alish1,alish2}. Задача отыскания точного выражения для изображения по Лапласу является более простой и менее трудоёмкой, и для большого класса функций удаётся найти точное представление изображения \cite{alish1}. Есть готовые таблицы обращений для изображений \cite{alish3}, есть и формальное представление в квадратурах по контуру  комплексной плоскости для любого изображения, т.е. для оригинала, однако численное получение результата представляет иногда значительные трудности. Трудность состоит в том, что интегрирование по контуру в комплексной плоскости оказывается недостаточно привычным делом при современном математическом образовании инженеров. Численное обращение преобразования Лапласа остаётся одной из актуальных проблем вычислительной математики, несмотря на то, что многие математические программные пакеты (Mathcad, Mathlab и др.) содержат операторы обращения изображения по Лапласу. Успех компьютерного расчета обращения некоторого изображения при этом зачастую зависит от выбора контура интегрирования.

Для начала приведём простой пример извлечения тепла горных пород, который демонстрирует метод Лапласа, и появление вычислительных трудностей использования формулы обращения Меллина при изменении граничных условий. Затем дадим постановку и решение эталонной нестационарной задачи определения температуры жидкости при её добыче по вертикальной трещине. 