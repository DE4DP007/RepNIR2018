\chapter{Некоторые сведения из теории ортогональных функций}
На протяжении всей работы существенно будет использован аппарат теории общих ортогональных полиномов. Поэтому мы соберем в этом параграфе необходимые сведения о некоторых классических системах ортогональных полиномов.

\section{Полиномы Якоби}
Для произвольных действительных $\alpha$ и $\beta$ полиномы Якоби  $P_n^{\alpha,\beta}(x)$ можно определить \cite{sobleg-Sege} с помощью формулы Родрига
 \begin{equation}\label{sobleg-2.1}
P_n^{\alpha,\beta}(x) = {(-1)^n\over2^nn!}{1\over\kappa(x)}{d^n\over
dx^n} \left\{\kappa(x)\sigma^n(x)\right\},
\end{equation}
где $\alpha,\beta$ -- произвольные действительные числа, $\kappa(x)=
(1-x)^\alpha(1+x)^\beta,\,\,\sigma(x)=1-x^2$. Если
$\alpha,\beta>-1$, то полиномы Якоби образуют ортогональную
систему с весом $\kappa(x)$, т.е.
\begin{equation}\label{sobleg-2.2}
\int_{-1}^1P_n^{\alpha,\beta}(x)P_m^{\alpha,\beta}(x)\kappa(x)dx =
h_n^{\alpha,\beta}\delta_{nm},
\end{equation}
где
\begin{equation}\label{sobleg-2.3}
h_n^{\alpha,\beta} =
{\Gamma(n+\alpha+1)\Gamma(n+\beta+1)2^{\alpha+\beta+1} \over
n!\Gamma(n+\alpha+\beta+1)(2n+\alpha+\beta+1)}.
\end{equation}
В частности, при $\alpha,\beta=0$ мы получим классические полиномы Лежандра $P_n(x)=P_n^{0,0}(x)$, для которых
\begin{equation*}
\int_{-1}^1P_n(x)P_m(x)\kappa(x)dx = \frac{2}{2n+1}\delta_{nm}.
\end{equation*}



Нам понадобятся еще следующие свойства полиномов Якоби~\cite{sobleg-Sege}:


\begin{equation}\label{sobleg-2.4}
{d\over dx}P_n^{\alpha,\beta}(x) =
{1\over2}(n+\alpha+\beta+1)P_{n-1}^{\alpha+1,\beta+1}(x),
\end{equation}
\begin{equation}\label{sobleg-2.5}
{d^\nu\over dx^\nu}P_n^{\alpha,\beta}(x) =
{(n+\alpha+\beta+1)_\nu\over2^\nu} P_{n-\nu}^{\alpha+\nu,\beta+\nu}(x),
\end{equation}
где $(a)_0=1$, $(a)_\nu=a(a+1)\dots(a+\nu-1)$,

 \begin{equation}\label{sobleg-2.6}
 {n\choose l}P_n^{\alpha,-l}(x)= {n+\alpha\choose
l}\left({x+1\over2}\right)^lP_{n-l}^{\alpha,l}(x),
     \quad 1\le l \le n,
\end{equation}
\begin{equation}\label{sobleg-2.7}
P_n^{\alpha,\beta}(t) ={n+\alpha\choose n}
\sum_{k=0}^n{(-n)_k(n+\alpha+\beta+1)_k\over k!(\alpha+1)_k}
\left({1-t\over 2}\right)^k,
\end{equation}
\begin{equation}\label{sobleg-2.8}
(1-x)P_n^{\alpha+1,\beta}(x)={2\over2n+\alpha+\beta+2}
\left[(n+ \alpha+1)P_n^{\alpha,\beta}(x)-
(n+1)P_{n+1}^{\alpha,\beta}(x)\right],
\end{equation}

\begin{equation}\label{sobleg-2.9}
P_n^{\alpha,\beta}(-1)=(-1)^n {n+\beta\choose n},\quad P_n^{\alpha,\beta}(1)= {n+\alpha\choose n},
\end{equation}


Пусть $-1<\alpha, \beta$ -- произвольные вещественные числа,
$$
s(\theta)=s^{\alpha,\beta}(\theta)=\pi^{-\frac12}
\left(\sin\frac{\theta}{2}\right)^{-\alpha-\frac12}
\left(\cos\frac{\theta}{2}\right)^{-\beta-\frac12},
$$
$$
\lambda_n=n+\frac{\alpha+\beta+1}{2}, \quad\gamma=-
\left(\alpha+\frac{1}{2}\right)\frac{\pi}{2}.
$$
Тогда для $0<\theta<\pi$ имеет место асимптотическая формула
\begin{equation}\label{sobleg-2.10}
P_n^{\alpha,\beta}(\cos\theta)=n^{-\frac12}s(\theta)
\left(\cos(\lambda_n\theta+\gamma)+
\frac{v_n(\theta)}{n\sin\theta}\right),
\end{equation}
в которой для функции $v_n(\theta)=v_n(\theta;\alpha,\beta)$
справедлива оценка
\begin{equation}\label{sobleg-2.11}
|v_n(\theta)|\le c(\alpha,\beta,\delta)\quad
 \left(0<\frac{\delta}{n}\le\theta\le\pi-\frac{\delta}{n}\right).
\end{equation}
Имеет место также следующая оценка
\begin{equation}\label{sobleg-2.12}
|P_n^{\alpha,\beta}(\cos\theta)|\le c(\alpha,\beta,\gamma)n^\alpha, \quad |\theta|\le \frac{\gamma}{n}.
\end{equation}

Из \eqref{sobleg-2.3}, \eqref{sobleg-2.10} -- \eqref{sobleg-2.12} мы выводим для ортонормированных   полиномов Якоби
\begin{equation}\label{sobleg-2.13}
p_n^{\alpha,\beta}(x)=P_n^{\alpha,\beta}(x)/\sqrt{h_n^{\alpha,\beta}}\quad(n=0,1,\ldots)
\end{equation}
 следующую весовую оценку
\begin{equation}\label{sobleg-2.14}
\left(\sin\frac{\theta}{2}\right)^{\alpha+\frac12}
\left(\cos\frac{\theta}{2}\right)^{\beta+\frac12}
|p_n^{\alpha,\beta}(\cos\theta)|\le c(\alpha,\beta),
\end{equation}
где $\alpha,\beta\ge-1/2$, $0\le\theta\le\pi$. Указанные оценки вместе с \\
\textit{ формулой Кристоффеля-Дарбу }
$$
 K_n^{\alpha,\beta}(x,y)=
\sum_{k=0}^n{P_k^{\alpha,\beta}(x)P_k^{\alpha,\beta}(y)\over
h_k^{\alpha,\beta}}=
 $$
\begin{equation}\label{sobleg-2.15}
 {2^{-\alpha-\beta}\over
2n+\alpha+\beta+2} {\Gamma(n+2)\Gamma(n+\alpha+\beta+2)\over
\Gamma(n+\alpha+1)\Gamma(n+\beta+1)}
 {P_{n+1}^{\alpha,\beta}(x)P_n^{\alpha,\beta}(y)-
P_n^{\alpha,\beta}(x)P_{n+1}^{\alpha,\beta}(y)\over x-y}
\end{equation}
играют основополагающую роль при изучении аппроксимативных свойств частичных сумм специальных рядов со свойством прилипания по ультрасферическим полиномам Якоби.

Отметим еще следующую формулу, доказательство которой можно найти в работе \cite{sobleg-Gasper}:


$$
{P_n^{a,a}(x)\over P_n^{a,a}(1)}=\sum_{j=0}^{[n/2]}
   { n!(\alpha+1)_{n-2j}(n+2a+1)_{n-2j}(1/2)_j(a-\alpha)_j
    \over (n-2j)!(2j)!(a+1)_{n-2j}(n-2j+2\alpha+1)_{n-2j}}
     $$
\begin{equation}\label{sobleg-2.15}
    \times{1\over (n-2j+a+1)_j(n-2j+\alpha+3/2)_j}
     {P_{n-2j}^{\alpha,\alpha}(x)\over
P_{n-2j}^{\alpha,\alpha}(1)},
\end{equation}
где $[b]$ -- целая часть числа $b$.

\section{Полиномы Лагерра}
Приведем тут ряд свойств классических полиномов Лагерра $L_n^{\alpha}(t)$. Пусть $\alpha$ -- произвольное действительное число. Тогда  имеют место \cite{laplas-Sege}:

\textit{Формула Родрига}
\begin{equation}\label{laplas-2.1}
L_n^{\alpha}(t) = \frac{1}{n!}t^{-\alpha}e^{t} \left\{ t^{n+\alpha} e^{-t} \right\}^{(n)};
\end{equation}

\textit{Явный вид}
\begin{equation}\label{laplas-2.2}
L_n^\alpha(t) =
\sum\limits_{\nu=0}^{n}
\binom{n+\alpha}{n-\nu}
\frac{(-x)^\nu}{\nu!};
\end{equation}

\textit{Соотношение ортогональности}

\begin{equation}
\label{laplas-2.3}
\int_0^{\infty} t^{\alpha} e^{-t} L^{\alpha}_{n}(t) L^{\alpha}_{m}(t) dt = \delta_{nm} h^{\alpha}_n \quad (\alpha > -1),
\end{equation}
где $\delta_{nm}$ --- символ Кронекера,
\begin{equation}\label{laplas-2.4}
h^{\alpha}_n = \left( n+\alpha \atop n \right) \Gamma(\alpha +1);
\end{equation}
\textit{Формула Кристоффеля -- Дарбу}
\begin{equation}\label{laplas-2.5}
\mathcal{K}_n^\alpha(t,\tau)=
\sum\limits_{k=0}^{n}\frac{L_\nu^\alpha(t)L_\nu^\alpha(\tau)}{h_\nu^\alpha}=
\frac{n+1}{h_n^\alpha}
\frac{L_n^\alpha(t)L_{n+1}^\alpha(\tau) - L_n^\alpha(\tau)L_{n+1}^\alpha(t)}{t-\tau};
\end{equation}

\textit{Свертка}
\begin{equation}
\label{laplas-2.6}
\int_0^{t} L_{n}(t-\tau) L_{m}(\tau) d\tau = L_{n+m}(t) - L_{n+m+1}(t).
\end{equation}

Далее отметим следующие равенства
\begin{equation}\label{laplas-2.7}
\frac{d}{dt} L_n^{\alpha}(t) = -L_{n-1}^{\alpha+1}(t),
\end{equation}

\begin{equation} \label{laplas-2.8}
\frac{d^r}{dt^r} L_{k+r}^{\alpha-r}(t) = (-1)^{r} L_{k}^{\alpha}(t),
\end{equation}
\begin{equation}\label{laplas-2.9}
L_{k}^{-r}(t) = \frac{(-t)^{r}}{k^{[r]}} L_{k-r}^{r}(t),
\end{equation}
где $k^{[r]} = k(k-1)\ldots(k-r+1)$,
\begin{equation}\label{laplas-2.10}
L_n^{\alpha+1}(t)-L_{n-1}^{\alpha+1}(t)=L_n^\alpha(t),
     \end{equation}
 \begin{equation}\label{laplas-2.11}
(n+\alpha)L_n^{\alpha-1}(t)=\alpha L_n^\alpha(t)-
xL_{n-1}^{\alpha+1}(t),
\end{equation}

\textit{весовая оценка} \cite{laplas-AskeyWaiger}
\begin{equation}\label{laplas-2.12}
e^{-\frac{t}{2}}|L_n^\alpha(t)| \le c(\alpha) B_n^\alpha(t), \quad \alpha>-1,
\end{equation}
где здесь и далее $c,c(\alpha),c(\alpha,\ldots,\beta)$ -- положительные числа, зависящие лишь от указанных параметров,
\begin{equation*}
B_n^\alpha(t)=
\begin{cases}
\theta^\alpha, &0 \le t \le \frac{1}{\theta},\\
\theta^{\frac{\alpha}{2} - \frac{1}{4}}\,t^{-\frac{\alpha}{2} - \frac{1}{4}}, & \frac{1}{\theta} < t \le \frac{\theta}{2},\\
\Bigl[
\theta(\theta^{\frac{1}{3}}+|t-\theta|)
\Bigr]^{-\frac{1}{4}}, & \frac{\theta}{2} < t \le \frac{3\theta}{2},\\
e^{-\frac{t}{4}}, &\frac{3\theta}{2}< t,
\end{cases}
\end{equation*}
где $\theta=\theta_n=\theta_n(\alpha)=4n+2\alpha+2$.

Для нормированных полиномов Лагерра
\begin{equation}\label{laplas-2.13}
l_n^\alpha(t)=
\Bigl\{h_n^\alpha \Bigr\}^{-\frac{1}{2}} L_n^\alpha(t)
\end{equation}
имеет место оценка \cite{laplas-AskeyWaiger}
\begin{equation}\label{laplas-2.14}
e^{-\frac{t}{2}}
\Bigl|
l_{n+1}^\alpha(t)-
l_{n-1}^\alpha(t)
\Bigr|\le
\begin{cases}
\theta^{\frac{\alpha}{2}-1}, &0 \le t \le \frac{1}{\theta},\\
\theta^{-\frac{3}{4}}\,t^{-\frac{\alpha}{2} + \frac{1}{4}}, & \frac{1}{\theta} < t \le \frac{\theta}{2},\\
t^{-\frac{\alpha}{2}}\,
\theta^{-\frac{3}{4}}
\Bigl[
\theta^{\frac{1}{3}}+|t-\theta|
\Bigr]^{\frac{1}{4}}, & \frac{\theta}{2} < t \le \frac{3\theta}{2},\\
e^{-\frac{t}{4}}, &\frac{3\theta}{2}< t.
\end{cases}
\end{equation}
Поскольку $h_n^\alpha=\frac{\Gamma(n+\alpha+1)}{n!} \asymp n^\alpha$, то из \eqref{laplas-2.12} и \eqref{laplas-2.13} следует, что
\begin{equation}\label{laplas-2.15}
e^{-\frac{t}{2}}
|l_n^\alpha(t)|\le
c(\alpha)\theta_n^{-\frac{\alpha}{2}}B_n^\alpha(t), \quad t \ge 0.
\end{equation}


\section{Полиномы Шарлье}

При конструировании полиномов, ортогональных по Соболеву и порожденных классическими полиномами Шарлье нам понадобится ряд свойств этих полиномов, которые мы приведем в настоящем параграфе.
Для произвольного $\alpha$ положим
\begin{equation}\label{charlier-Shar_eq12}
\rho(x)=\rho(x;\alpha)=\frac{\alpha^x e^{-\alpha}}{\Gamma(x+1)},
\end{equation}
\begin{equation}\label{charlier-Shar_eq13}
S_n^{\alpha}(x)=\frac{1}{\alpha^n \rho(x)} \Delta^n \{\rho(x)x^{[n]}\},
\end{equation}
%где $\Delta^nf(x)$ --- конечная разность $n$-го порядка функции
%$f(x)$ в точке $x$, т.е. $\Delta^0f(x)=f(x)$,
%$\Delta^1f(x)=\Delta f(x)=f(x+1)-f(x)$, $\Delta^nf(x)=\Delta
%\Delta^{n-1}f(x)$ $(n\ge1)$, $a^{[0]}=1$,
%$a^{[k]}=a(a-1)\cdots(a-k+1)$ при $k\ge1$.
Для каждого $0\le n$ равенство \eqref{charlier-Shar_eq13} определяет \cite{ramis-Ram1, charlier-Shar10} алгебраический полином степени $n$.
Полные доказательства приведенных ниже свойств полиномов Шарлье $S_n^{\alpha}(x)$
можно найти, например, в \cite{ramis-Ram1}.

Если $\alpha>0$, то полиномы $S_n^{\alpha}(x)$ ($n=0,1,\ldots$) образуют полную \cite{ramis-Ram1}, \cite{charlier-Shar11} в $l_\rho$ ортогональную с весом $\rho(x)$ (см. \eqref{charlier-Shar_eq12}) систему на множестве $\Omega=\{0,1,\ldots\}$:
\begin{equation}\label{charlier-Shar_eq14}
\sum_{x\in\Omega}S_k^\alpha(x)S_n^\alpha(x)\rho(x)=\delta_{nk} h_n(\alpha),
\end{equation}
где
\begin{equation}\label{charlier-Shar_eq15}
h_n(\alpha) = \sum_{x=0}^{\infty} \rho(x) \{S_n^\alpha(x)\}^2 = \alpha^{-n} n! .
\end{equation}

Из \eqref{charlier-Shar_eq14} и \eqref{charlier-Shar_eq15} следует, что полиномы
\begin{equation}\label{charlier-Shar_eq16}
s_n^\alpha(x)=(h_n(\alpha))^{-\frac12} S_n^\alpha(x) \quad (n=0,1,\ldots)
\end{equation}
образуют ортонормированную систему на множестве $\Omega$ с весом $\rho(x)=\rho(x,\alpha)$, т.е.
\begin{equation*}
\sum_{x\in\Omega} s_k^\alpha(x) s_n^\alpha(x)\rho(x)=\delta_{nk}.
\end{equation*}

Полиномы Шарлье допускают следующее явное представление
\begin{equation} \label{charlier-Shar_eq17}
S_n^\alpha(x) = \sum_{l=0}^n \frac{(-n)_l(-x)_l}{l!} (-\alpha)^{-l} = \sum_{l=0}^n \frac{n^{[l]}x^{[l]}}{l!} (-\alpha)^{-l},
\end{equation}
где $(a)_l=a(a+1)\ldots(a+l-1)$ --- символ Похгаммера. Из \eqref{charlier-Shar_eq17} непосредственно следует, что
\begin{equation}\label{charlier-Shar_eq18}
\Delta S_n^{\alpha}(x) = -\frac{n}{\alpha} S_{n-1}^{\alpha}(x).
\end{equation}